\documentclass[12pt]{article}
\usepackage[utf8]{inputenc}
\usepackage{amsmath}
\usepackage{amsthm} 
\usepackage{fullpage}
\usepackage{amsfonts}
\usepackage{amssymb}
\usepackage{bm}

\def\A{{\frak A}}
\def\C{{\cal C}}
\def\F{{\cal F}}
\def\L{{\cal L}}
\def\N{\mathbb{N}}
\def\P{\vec{P}}
\def\Q{\vec{Q}}
\def\S{\vec{S}}
\def\X{\vec{X}}
\def\Y{\vec{Y}}
\def\Z{\vec{Z}}
\def\d{\vec{d}}
\def\e{\vec{e}}
\def\s{\vec{s}}
\def\u{\vec{u}}
\def\v{\vec{v}}
\def\w{\vec{w}}
\def\x{\vec{x}}
\def\y{\vec{y}}
\def\z{\vec{z}}

\newtheorem{theo}{Theorem}
\newtheorem{lemma}[theo]{Lemma}
\newtheorem{claim}[theo]{Claim}
\newtheorem{coro}[theo]{Corollary}

\begin{document}


\begin{center}
{ \LARGE \bf
  Some properties of $\#\Sigma_1^{+}$
}
\end{center}

We define the vocabulary $\L = \{ S_1,...,S_t, \leq \}$, where $S_1,...,S_t$ have arity $b_1,...,b_t$. Let
\begin{eqnarray*}
\textsc{Struct}[\L] &=& \{\A \mid \A \text{ is an } \L \text{-structure with a finite domain } A \text{ such that} \\
&& \leq \text{ is interpreted as a total order for } A \}.
\end{eqnarray*}
We also define a set of second order variables ${\cal X} = \{ X_i \mid i\in\N \}$ where $X_i$ has arity $a_i$. A quantifier-free $\L$-formula is defined by the following grammar:
\begin{eqnarray*}
\varphi &::=& x = y \ \mid \ S_i(x_1,...,x_{b_i}), i \in \{1,...,t\} \ \mid \ x \leq y \ \mid \\
&& X_i(x_1,...,x_{a_i}), i\in\N \ \mid \\ 
&& (\neg \varphi) \ \mid \ (\varphi \wedge \varphi) \ \mid \ (\varphi \vee \varphi),
\end{eqnarray*}
where $y$ and $x_i$ are first order variables for every $i$. We now define an extended quantifier-free $\L$-formula as such:
\begin{eqnarray*}
\varphi &::=& \alpha, \alpha \text{ is an FO-formula over } \L^\prime  \ \mid \\
&& X_i(x_1,...,x_{a_i}), i\in\N \ \mid \ \\
&& (\neg \varphi) \ \mid \ (\varphi \wedge \varphi) \ \mid \ (\varphi \vee \varphi).
\end{eqnarray*}
Let $\X = (X_1,...,X_r)$ and $\P = (P_1,...,P_r)$, where $P_i$ is a predicate of arity $a_i$, for every $i$. Let $\z$ be a tuple of variables and $\e$ be a tuple of constants with the same dimension as $\z$. A function $f:\textsc{Struct}[\L] \to \mathbb{N}$ is in $\#\Sigma_1^+$ if there exists an extended quantifier-free $\L$-formula $\varphi(\x,\X,\z)$ such that
\begin{eqnarray*}
f(\A) &=& \mid \{ \langle \P, \e \rangle \mid \A \models \exists \x \: \varphi(\x,\P,\e) \} \mid,
\end{eqnarray*}
for every $\A \in \textsc{Struct}[\L]$.

\begin{theo}
The decision version of a function in $\#\Sigma_1^{+}$ is in \text{P}.
\end{theo}
\begin{proof}
Let $f$ be a function in $\#\Sigma_1^{+}$. Then there is a formula $\varphi(\x,\X,\z)$ such that
\begin{eqnarray*}
f(\A) &=& \mid \{ \langle\P,\z\rangle \mid \A \models \exists \x \, \varphi(\x,\P,\z) \} \mid,
\end{eqnarray*}
where $\A \in \textsc{Struct}[\L]$ and $\z$ is an $m$-tuple. For each $\z \in A^m$, every arbitrarily quantified first order sub-formula can be evaluated in polynomial time. Let $\varphi^\prime(\x,\P,\z)$ be the formula that results of changing every satisfied sub-formula for a tautology and every non-satisfied sub-formula by a contradiction. Note that $\varphi^\prime(\x,\P,\z)$ is quantifier-free. Then, we compute a similar function,
\begin{eqnarray*}
f'(\A) &=& \mid \{ \langle\P,(\z,\x)\rangle \mid \A \models \varphi(\x,\P,\z) \} \mid
\end{eqnarray*}
which is in $\#\Sigma_0$\footnote[1]{(paper)}. Note that for every $\A$, $f(\A) > 0$ iff $f'(\A) > 0$, so computing $f'$ is enough to solve the decision version of $f$. However, it is showed in $^{[1]}$ that any counting problem in $\#\Sigma_0$ is computable in polynomial time, therefore, the decision version of $f$ is in P.
\end{proof}

\begin{theo}
$\#\Sigma_1$ is closed under substraction $\Rightarrow$ P = NP.
\end{theo}
\begin{proof}
\#3DNF $\in \#\Sigma_1$. Let $F_{2^n}$ be a $\#\Sigma_1$ function that counts every possible truth assignment in a \#3DNF instance. Suppose that $F_{2^n}-F_{\#3DNF} \in \#\Sigma_1$. This function equals 0 only if the instanced formula is a tautology, so the decision version of it is co-NP-complete. However, as we showed previously (Theorem 1), it is also in P. Then, co-NP $\subseteq$ P.
\end{proof}

\begin{theo}
$\#\Sigma_1 \subsetneq \#\Sigma_1^{+}$
\end{theo}
\begin{proof}
We will show that the $\#\Sigma_1^{+}$ function defined by $\varphi(x_1) = (x_1 = x_1) \wedge \forall x_2 S(x_2)$ is not in $\#\Sigma_1$.
\end{proof}

%% F-1 F-1 F-1 F-1 F-1 F-1 F-1 F-1 F-1 F-1 F-1 F-1 F-1 F-1 F-1 F-1 F-1 F-1 F-1 F-1 F-1 F-1 F-1 F-1 F-1 F-1 F-1 F-1 F-1 F-1 
%% F-1 F-1 F-1 F-1 F-1 F-1 F-1 F-1 F-1 F-1 F-1 F-1 F-1 F-1 F-1 F-1 F-1 F-1 F-1 F-1 F-1 F-1 F-1 F-1 F-1 F-1 F-1 F-1 F-1 F-1 

A function class $\F$ is {\em closed under substraction by one} if for every function $f \in \F$ there exists $f' \in \F$ such that 
\begin{eqnarray*}
f'(\A) =
\begin{cases}
f(\A)-1, & \text{if }f(\A) > 0 \\
0, & \text{if }f(\A) = 0.
\end{cases}
\end{eqnarray*}
\begin{theo}
$\#\Sigma_1^+$ is closed under substraction by one.
\end{theo}
\begin{proof}
A function in $\#\Sigma_1^+$ has three possible ways of counting. Counting only variables, only predicates, and predicates with variables. This separates the proof in three cases:
\begin{enumerate}

% CASE 1

\item Let $f \in \#\Sigma_1^+$, which is defined by an $\L$-formula $\varphi(\x,\z)$, where $\z = (z_1,...,z_d)$. That is,
\begin{eqnarray*}
f(\A) &=& \mid \{ \langle\e\rangle \mid \A \models \exists \x \ \varphi(\x,\e) \} \mid,
\end{eqnarray*}
for every $A \in \textsc{Struct}[\L]$. Our goal here is to eliminate the lexicographically smallest sequence of variables, which we can do easily. First, let $\y = (y_1,...,y_k)$, $\y\,^\prime = (y_1^\prime,...,y_k^\prime)$ and
\begin{eqnarray*}
\varphi_{k,<}(\y\,^\prime,\y) &=& \bigvee_{i = 1}^k \left( \bigwedge_{j=1}^{i-1} y_j = y_j^\prime \wedge y_i < y_i^\prime \right).
\end{eqnarray*}
This formula is true if $\y\,^\prime$ is lexicographically smaller than $\y$. Now, let $f'$ be defined by
\begin{eqnarray*}
\varphi^\prime(\x,\z) &=& \varphi(\x,\z) \wedge \exists \z\,^\prime (\varphi(\x,\z\,^\prime) \wedge \varphi_{d,<}(\z\,^\prime,\z ) ).
\end{eqnarray*}
Therefore, if $f(\A)>0$, $f'(\A)$ will count exactly one element less than $f(\A)$; and if $f(\A)=0$, $\A \not\models\exists x\,\varphi(\x,\e)$ for every tuple $\e$ of elements in $A$, so $\A \not\models\exists x\,\varphi^\prime(\x,\e)$ for every $\e$. Thus, $f'(\A)=0$.

%CASE 2

\item Let $f \in \#\Sigma_1^+$, which is defined by an $\L$-formula $\varphi(\x,\X)$ where $\x = (x_1,...,x_d)$ and $\X = (X_1,...,X_r)$. Then
\begin{eqnarray*}
f &=& \mid \{ \langle\P\rangle \mid \A \models \exists \x \ \varphi(\x,\P) \} \mid,
\end{eqnarray*}
for every $\A \in \textsc{Struct}[\L]$. For the time being, suppose that
\begin{eqnarray*}
\varphi(\x,\X) &=& \left( \bigwedge_{i=1}^n Y_i(\x_i) \right) \wedge \varphi_{\X}^{-}(\y) \wedge \theta(\x) \wedge \beta(\x)
\end{eqnarray*}
where $Y_i$ is in $\X$, $Y_i$ is of arity $c_i$ and $\x_i$ is an $c_i$-tuple of variables in $\x$ for all $i \in \{1,...,n\}$, $\y$ is a $d$-tuple of variables in $\x$, $\varphi_{\X}^{-}(\y)$ is a conjunction of negated predicates in $\P$, $\theta(\x)$ defines a total order on a partition of $\x$, and $\beta(\x)$ is an FO-formula over $\L$. We also assume that $(\x_1,...,\x_n,\y) = \x$. As an example, we will show that the following formula is of this form:
\begin{eqnarray*}
\varphi^{*}(\x,\X) &=& \left( X_1(x_1,x_2) \wedge \neg X_1(x_3,x_4) \wedge (x_1 < x_2 \wedge x_1 = x_3 \wedge x_1 = x_4 ) \wedge \forall z\big( S_1(x_1,z) \big) \right),
\end{eqnarray*}
where $\x = (x_1,x_2,x_3,x_4)$ and $\X = (X_1)$. Here, $n = 1$, $Y_1 = X_1$, $\x_1 = (x_1,x_2)$ and $\y = (x_3,x_4)$. $\varphi_{\X}^{-}(\y) = \neg X_1(x_3,x_4)$, $\theta^*(\x) = (x_1 < x_2 \wedge x_1 = x_3 \wedge x_1 = x_4)$ which defines a total order on the partition $\{\{x_1,x_3,x_4\},\{x_2\}\}$, and $\beta^*(\x) = \forall z\big( S_1(x_1,z) \big)$.

Similarly to the previous proof, we would like to eliminate the lexicographically smallest tuple of predicates that satisfies the formula. Let $\u_i$ be a $c_i$-tuple of variables for every $i \in \{1,...,n\}$, and let $m = \sum_{i = i}^n c_i$ be the number of variables of $(\x_1,...,\x_n)$. We now define
\begin{multline*}
\alpha_{\min}(\u_1,...,\u_n) = \\ \exists\y\left( \alpha(\u_1,...,\u_n,\y)\wedge \forall\v_1...\v_n\w(\alpha(\v_1,...,\v_n,\w)\rightarrow \varphi_{m,<}((\u_1,...,\u_n),(\v_1,...,\v_n))\right),
\end{multline*}
where $\alpha(\x) = \theta(\x) \wedge \beta(\x)$. Note that $\alpha_{\min}$ is satisfied only by the lexicographically \linebreak smallest assignment $\d$ to $(x_1,...,x_n)$ such that $\exists\y\,(\theta(d_1,...,d_n,\y)\wedge\beta(d_1,...,d_n,\y))$ is true. Our new formula is
\begin{multline*}
\varphi^\prime(\x,\X) = \left( \bigwedge_{i=1}^n Y_i(\x_i) \right) \wedge \varphi_{\X}^{-}(\y) \wedge \theta(\x) \wedge \beta(\x)\wedge\exists\u_1,...,\u_n\bigg[\alpha_{\min}(\u_1,...,\u_n) \wedge \\ \bigg(\bigvee_{i = 1}^{n}\neg Y_i(\u_i) \vee \bigvee_{Y\in\X} \exists \v\Big( Y(\v) \wedge \bigwedge_{i\in[1,n]: Y_i = Y} \v \neq \u_i\Big) \bigg) \bigg].
\end{multline*}
We will show that if there is an assignment to satisfy $\exists\x\,\varphi(\x,\X)$, this formula will be satisfied by exactly one predicate less than $\exists\x\,\varphi(\x,\X)$. Suppose $\d$ is the lexicographically smallest assignment which satisfies $\alpha(\x)$. Let $\d_i$ be the respective assignment to $x_i$, for every $i\in\{1,...,n\}$. Consider now the tuple $\P = (P_i,...,P_r)$ where $P_i = \{d_j \mid j\in\{1,...,n\} \text{ and } X_i = Y_j \}$. We will now show that this tuple of predicates is such that $\exists\x\,\varphi(\x,\P)$ is true. Suppose that it's not, that is, there is no assignation to $\x$ such that $\varphi(\x,\P)$ is true. Since $\d$ satisfies $\left( \bigwedge_{i=1}^n Y_i(\x_i) \right)$ and $\theta(\x) \wedge \beta(\x)$, then $\varphi_{\X}^{-}(\d)$ must be false. Therefore, there is a negated predicate $Y$ in $\varphi_{\X}^{-}(\x)$ such that its respective variables in $\y$ are equal to an $x_i$ where $Y_i = Y$. Then, since $\theta(\x)$ defines a partition of $\x$, these variables are equal, and thus, $\exists\x\,\varphi(\x,\X)$ is unsatisfiable ($\rightarrow\leftarrow$). $\P$ is excluded by the new formula since every $d_i$ is in its respective predicate, which does not satisfy the first part of the new formula, and does not have any tuples that are not in $\d$. $\P$ is also the only tuple excluded by the new formula since any other tuple that satisfies it would have at least the same tuples as $\P$, and would not have any more, and thus would be equal to $\P$. Therefore, if $\exists\x\,\varphi(\x,\X)$ is satisfiable, $\exists\x\,\varphi^\prime(\x,\X)$ is satisfied by exactly one predicate less. Naturally, if $\varphi(\x,\X)$ is unsatisfiable for every assignment to $\x$ and $\X$, $\varphi^\prime(\x,\X)$ also is.

%% Let $\P$ and $\P^\prime$ be tuples of predicates such that $\exists\x\,\varphi(\x,\P)$ and $\exists\x\,\varphi(\x,\P^\prime)$ are true but $\exists\x\,\varphi^\prime(\x,\P)$ and $\exists\x\,\varphi^\prime(\x,\P^\prime)$ are not. Let $Q_i$ and $Q_i^\prime$ be the assignment of $\P$ and $\P^\prime$ to $Y_i$, respectively, for every $i\in\{i,...n\}$. Then, every tuple $\d_i$ is in its respective relation $Q_i$ in both $\P$ and $\P^\prime$ and $(d_1,...,d_n)$ includes every tuple in $P^\prime$. If that is the case, $\P^\prime$ contains exactly the same tuples as $\P$, and are thus the same.




To use this result for the general case, let
\begin{eqnarray*}
f(\A) &=& \mid \{ \langle\P\rangle \mid \A \models 
\exists \x \varphi_1(\x,\P) \vee
\exists \x \varphi_2(\x,\P) \vee
... \vee
\exists \x \varphi_k(\x,\P)
 \} \mid
\end{eqnarray*}
where $\varphi(\x,\P) \equiv \varphi_1(\x,\P) \vee \varphi_2(\x,\P) \vee ...  \vee \varphi_k(\x,\P) $. This is possible if we consider all first order formulas as literals, and find an equivalent DNF formula. Now we need to exclude that assignment to the next disjuncts, which we do by adding the same sub-formula. We also need to include the case where there are no possible assignments to $\x$ for the first disjunct, so we add this sub-formula to $\varphi^2(\x,\X)$:
\begin{multline*}
\exists\x\left(\varphi^1(\x,\X)\right) \vee \exists\vec{x_1}',...,\vec{x_n}'\Bigg(\alpha^2_{\min}(\vec{x_1}',...,\vec{x_n}') \wedge \\
\bigg(\bigvee_{i = 1}^{n}\neg Y^2_i(\vec{x_i}') \vee \bigvee_{P\in\P} \exists \x''\Big( P(\x'') \bigwedge_{\substack{i\in\{1,...,n\} \\ P_i = P}} \x'' \neq \vec{x_i}'\Big) \bigg) \Bigg).
\end{multline*}
The full disjunct is as follows,
\begin{multline*}
\varphi_j^\prime(\x,\X) = \bigwedge_{i=1}^n Y_i(\x_i) \wedge \varphi_{\overline{P}}(\vec{y}) \wedge \theta(\x) \wedge \varphi_{\S}(\x) \wedge \Bigg(\Big(\forall\x\left(\neg\alpha^1_{\min}(\x)\right)\wedge ... \wedge \forall\x\left(\neg\alpha^{j-1}_{\min}(\x)\right)\Big) \to \\ 
\exists\u_1,...,\u_n\Bigg(\alpha^j_{\min}(\u_1,...,\u_n) \wedge
\bigg(\bigvee_{i = 1}^n\neg P(\u_i) \vee \bigvee_{P\in\P} \exists \v\Big( P(\v) \bigwedge_{\substack{i\in\{1,...,n\} \\ P_i = P}} \v \neq \u_i\Big) \bigg) \Bigg) \Bigg)
\end{multline*}
which, if all of the previous disjuncts have no possible assignments, eliminates the least assignment to $\P$ in that disjunct. Finally,
\begin{eqnarray*}
f^\prime(\A) &=& \mid \{ \langle\P\rangle \mid \A \models 
\exists \x \varphi^\prime_1(\x,\P) \vee
\exists \x \varphi^\prime_2(\x,\P) \vee
... \vee
\exists \x \varphi^\prime_k(\x,\P)
 \} \mid
\end{eqnarray*}
counts exactly one assignment to $\P$ less than $F(\A)$.

Therefore,

% CASE 3

\item The counting set is:
\begin{eqnarray*}
F(\A) &=& \mid \{ \langle\P,\z\rangle \mid \A \models \exists \x \ \varphi(\x, \P,\z) \} \mid
\end{eqnarray*}
Then, we going to isolate the minimal predicate $\P$ that holds the formula true and eliminate the lexicographically smallest $\z$ that satisfies it. We mix both previous strategies.
\end{enumerate}
\end{proof}

\end{document}