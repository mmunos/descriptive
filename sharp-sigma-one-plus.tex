\documentclass[12pt]{article}
\usepackage[utf8]{inputenc}
\usepackage{amsmath}
\usepackage{amsthm} 
\usepackage{fullpage}
\usepackage{amsfonts}
\usepackage{amssymb}
\usepackage{bm}

\def\A{{\frak A}}
\def\C{{\cal C}}
\def\L{{\cal L}}
\def\F{{\cal F}}
\def\P{\vec{P}}
\def\Q{\vec{Q}}
\def\S{\vec{S}}
\def\s{\vec{s}}
\def\u{\vec{u}}
\def\v{\vec{v}}
\def\w{\vec{w}}
\def\x{\vec{x}}
\def\y{\vec{y}}
\def\z{\vec{z}}

\newtheorem{theo}{Theorem}
\newtheorem{lemma}[theo]{Lemma}
\newtheorem{claim}[theo]{Claim}
\newtheorem{coro}[theo]{Corollary}

\begin{document}


\begin{center}
{ \LARGE \bf
  Some properties of $\#\Sigma_1^{+}$
}
\end{center}

We define the vocabulary $\L = \{ P_1, ..., P_r, S_1,...,S_t, \leq \}$, where $P_1, ..., P_r$ have arity $a_1,...,a_r$, and $S_1,...,S_t$ have arity $b_1,...,b_t$. Let $\L\prime = \{ S_1,...,S_t, \leq \}$, and let
\begin{eqnarray*}
\textsc{Struct}[\L] &=& \{\A \mid \A \text{ is an } \L\prime \text{-structure with a finite domain } A \text{ such that} \\
&& \leq \text{ is interpreted as a total order for } A \}.
\end{eqnarray*}
A quantifier-free $\L$-formula is defined by the following grammar:
\begin{eqnarray*}
\varphi &::=& x = y \mid S_i(x_1,...,x_{b_i}), i \in \{1,...,t\} \mid x \leq y \mid \\
&& P_i(x_1,...,x_{a_i}), i \in \{1,...,r\} \mid \\ 
&& (\neg \varphi) \mid (\varphi \wedge \varphi) \mid (\varphi \vee \varphi),
\end{eqnarray*}
where $x, y$ are variables. We now define a quantifier-free$^{+}$ $\L$-formula as such:
\begin{eqnarray*}
\varphi &::=& \alpha, \alpha \text{ is a FO-formula over } \L\prime  \mid \\
&& P_i(x_1,...,x_{a_i}), i \in \{1,...,r\} \mid \\
&& (\neg \varphi) \mid (\varphi \wedge \varphi) \mid (\varphi \vee \varphi).
\end{eqnarray*}
Let $\P = (P_1, ..., P_r)$ and $\S = (S_1,...,S_t)$. A function $\F:\textsc{Struct[\L]} \to \mathbb{N}$ is in $\#\Sigma_1^+$ if there exists a quantifier-free$^{+}$ $\L$-formula $\varphi(\x,\P,\z)$ such that
\begin{eqnarray*}
\F &=& \mid \{ \langle \P, \z \rangle \mid \A \models \exists \x \: \varphi(\x,\P,\z) \} \mid
\end{eqnarray*}
for every $\A \in \textsc{Struct[\L]}$. We refer to predicates in $\P$ and variables in $\z$ as ``open", and predicates in $\S$ and variables in $\x$ as ``closed".

\begin{theo}
The decision version of a function in $\#\Sigma_1^{+}$ is in \text{P}.
\end{theo}
\begin{proof}
Let $F$ be a function in $\#\Sigma_1^{+}$. Then there is a formula $\varphi$ such that
\begin{eqnarray*}
F(\A) &=& \mid \{ \langle\P,\z\rangle \mid \A \models \exists \x \varphi(\S,\x,\z) \} \mid,
\end{eqnarray*}
where $\A$ is a finite ordered structure and $\z$ is an $m$-tuple. For each $\z \in A^m$, every arbitrarily quantified sub-formula which does not include $\P$ can be evaluated in polynomial time. Let $\varphi'$ be the formula that results of changing every satisfied sub-formula for a tautology and every non-satisfied sub-formula by a contradiction. Note that $\varphi'$ is quantifier-free.

Then, we compute a similar function,
\begin{eqnarray*}
F'(\A) &=& \mid \{ \langle\P,(\z,\x)\rangle \mid \A \models \varphi(\S,\x,\z) \} \mid
\end{eqnarray*}
which is in $\#\Sigma_0$. Note that for every $\A$, $F(\A) > 0$ iff $F'(\A) > 0$, so computing $F'$ is enough to solve the decision version of $F$. However, Saluja and Subrahmanyam showed that any counting problem in $\#\Sigma_0$ is computable in polynomial time \footnote[1]{(paper)}, therefore, the decision version of $F$ is in P.
\end{proof}

\begin{theo}
$\#\Sigma_1$ is closed under substraction $\Rightarrow$ P = NP.
\end{theo}
\begin{proof}
\#3DNF $\in \#\Sigma_1$. Let $F_{2^n}$ be a $\#\Sigma_1$ function that counts every possible truth assignment in a \#3DNF instance. Suppose that $F_{2^n}-F_{\#3DNF} \in \#\Sigma_1$. This function equals 0 only if the instanced formula is a tautology, so the decision version of it is co-NP-complete. However, as we showed previously (Theorem 1), it's also in P. Then, co-NP $\subseteq$ P.
\end{proof}

\begin{theo}
$\#\Sigma_1 \subsetneq \#\Sigma_1^{+}$
\end{theo}
\begin{proof}
We will show that the $\#\Sigma_1^{+}$ function defined by $\varphi(x_1) = (x_1 = x_1) \wedge \forall x_2 S(x_2)$ is not in $\#\Sigma_1$.
\end{proof}

%% F-1 F-1 F-1 F-1 F-1 F-1 F-1 F-1 F-1 F-1 F-1 F-1 F-1 F-1 F-1 F-1 F-1 F-1 F-1 F-1 F-1 F-1 F-1 F-1 F-1 F-1 F-1 F-1 F-1 F-1 
%% F-1 F-1 F-1 F-1 F-1 F-1 F-1 F-1 F-1 F-1 F-1 F-1 F-1 F-1 F-1 F-1 F-1 F-1 F-1 F-1 F-1 F-1 F-1 F-1 F-1 F-1 F-1 F-1 F-1 F-1 
A counting class $\C$ is {\em closed under substraction by one} if for every function $\F \in \C$ there exists $\F' \in \C$ where 
\begin{eqnarray*}
\F'(\A) =
\begin{cases}
\F(\A)-1, & \text{if }\F(\A) > 0 \\
0, & \text{if }\F(\A) = 0.
\end{cases}
\end{eqnarray*}
\begin{theo}
$\#\Sigma_1^+$ is closed under substraction by one
\end{theo}
\begin{proof}
A function in $\#\Sigma_1^+$ has three possible ways of counting. Counting only variables, only predicates, and predicates with variables. This separates the proof in three cases:
\begin{enumerate}

% CASE 1

\item Let $\F \in \#\Sigma_1^+$, which is defined by an $\L$-formula $\varphi(\x,\z)$, where $\z = (z_1,...,z_d)$.:
\begin{eqnarray*}
\F(\A) &=& \mid \{ \langle\z\rangle \mid \A \models \exists \x \ \varphi(\x,\z) \} \mid.
\end{eqnarray*}
Our goal here is to eliminate the lexicographically smallest sequence of variables, which we can do easily. First, let $\y = (y_1,...,y_k)$, $\y\prime = (y_1\prime,...,y_k\prime)$ and
\begin{eqnarray*}
\varphi_{k,<}(\y\prime,\y) &=& \bigvee_{i = 1}^k \left( \bigwedge_{j=1}^{i-1} y_j = y_j\prime \wedge y_i < y_i\prime \right).
\end{eqnarray*}
This formula is true if $\y\prime$ is lexicographically smaller than $\y$. Now, let $\F'$ be
\begin{eqnarray*}
\F'(\A) &=& \mid \{ \langle\z\rangle \mid \A \models \exists \x\left( \varphi(\x,\z) \wedge \exists \z\prime (\varphi(\x,\z\prime) \wedge \varphi_{d,<}(\z\prime,\z ) ) \right) \} \mid.
\end{eqnarray*}
Thus, if $\F>0$, $\F'(\A)$ will count exactly one element less than $\F(\A)$, and if $\F(\A)=0$, $\A \not\models\varphi(\x,\z)$ so $\F'(\A)=0$.

%CASE 2

\item Let $\F \in \#\Sigma_1^+$, which is defined by an $\L$-formula $\varphi(\x,\P)$ where $\x = (x_1,...,x_d)$. Then
\begin{eqnarray*}
\F &=& \mid \{ \langle\P\rangle \mid \A \models \exists \x \ \varphi(\x,\P) \} \mid.
\end{eqnarray*}
For the time being, suppose that
\begin{eqnarray*}
\varphi(\x,\P) &=& \bigwedge_{i=1}^n Q_j(\x_j) \wedge \varphi_{\P}^{-}(\y) \wedge \theta(\x) \wedge \varphi_{\S}(\x)
\end{eqnarray*}
where $Q_i$ is in $\P$, $Q_i$ is of arity $c_i$ and $\x_i$ is an $c_i$-tuple of variables in $\x$ for all $i \in \{1,...,n\}$, $\y$ is a $k$-tuple of variables in $\x$, $\varphi_{\P}^{-}(\y)$ is a conjunction of negated predicates in $\P$, $\theta(\x)$ defines a total order on a partition of $\x$, and $\varphi_{\S}(\x)$ involves only closed predicates. We also assume that $(\x_1,...,\x_n,\y) = \x$. For example,
\begin{eqnarray*}
\exists x_1\exists x_2\exists x_3\exists x_4 \left( P(x_1,x_2) \wedge \neg P(x_3,x_4) \wedge (x_1 < x_2 \wedge x_1 = x_3 \wedge x_1 = x_4 ) \wedge \forall z\big( S(x_1,z) \big) \right).
\end{eqnarray*}
Similarly, we would like to eliminate the lexicographically smallest predicate $P$ that satisfies the formula. We now define
\begin{multline*}
\alpha_{\min}(\u_1,...,\u_n) = \\ \exists\y\left( \alpha(\u_1,...,\u_n,\y)\wedge \forall\v_1...\v_n\y\prime(\alpha(\v_1,...,\v_n,\y\prime)\rightarrow \varphi_{n,<}((\u_1,...,\u_n),(\v_1,...,\v_n))\right),
\end{multline*}
where $\alpha(\x) = \theta(\x) \wedge \varphi_{\S}(\x)$. Note that $\alpha_{\min}$ is satisfied only by the lexicographically smallest assignment $\sigma$ to $(\x_1,...,\x_n)$ such that $\theta(\sigma(\x_1),...,\sigma(\x_n),\y)$ and $\varphi_{\S}(\sigma(\x_1),...,\sigma(\x_n),\y)$ are true for some $\y$. Our new formula is

\begin{multline*}
\varphi\prime(\x,\P) = \bigwedge_{i=1}^n Q_i(\x_i) \wedge \varphi^{-}_{\overline{P}}(\y) \wedge \theta(\x) \wedge \varphi_{\S}(\x)\wedge\exists\u_1,...,\u_n\Bigg(\alpha_{\min}(\u_1,...,\u_n) \wedge \\ \bigg(\bigvee_{i = 1}^{n}\neg P_i(\u_i) \vee \bigvee_{P\in\P} \exists \v\Big( P(\v) \wedge \bigwedge_{\substack{i\in\{1,...,n\} \\ P_i = P}} \v \neq \u_i\Big) \bigg) \Bigg)
\end{multline*}

where $\u_i$ is a an $a_i$-tuple of variables in $\x$. We will show that if there is an assignation to satisfy $\alpha(\x)$, this formula will be satisfied by exactly one predicate less than $\exists \x \: \varphi(\x,\P)$. Suppose $\sigma$ is such an assignation. Let $\pi$ be the excluded assignation to $\P$. Suppose there is another assignation $\pi\prime$ to $\P$ that is also excluded by the new sub-formula. Then, $\sigma(\x)$ is such that every tuple $\s_i = \sigma(x_i)$ is in $\pi\prime(\P)$ and $\sigma(x)$ includes every tuple in $\pi\prime(\P)$. If that is the case, $\pi\prime(\P)$ contains exactly the same tuples as $\pi(\P)$, and are thus the same. Naturally, if $\varphi(\x,\P)$ is unsatisfiable for every assignation to $\x$ and $\P$, $\varphi\prime(\x,\P)$ also is.

To use this result for the general case, let
\begin{eqnarray*}
\F(\A) &=& \mid \{ \langle\P\rangle \mid \A \models 
\exists \x \varphi_1(\x,\P) \vee
\exists \x \varphi_2(\x,\P) \vee
... \vee
\exists \x \varphi_k(\x,\P)
 \} \mid
\end{eqnarray*}
where $\varphi(\x,\P) \equiv \varphi_1(\x,\P) \vee \varphi_2(\x,\P) \vee ...  \vee \varphi_k(\x,\P) $. Now we need to exclude that assignation to the next disjuncts, which we do by adding the same sub-formula. We also need to include the case where there are no possible assignations to $\x$ for the first disjunct, so we add the sub-formula
\begin{multline*}
\forall\x\left(\neg\alpha^1_{\min}(\x)\right) \to \exists\vec{x_1}',...,\vec{x_n}'\Bigg(\alpha^2_{\min}(\vec{x_1}',...,\vec{x_n}') \wedge \\
\bigg(\bigvee_{i = 1}^{n}\neg P(\vec{x_i}') \vee \bigvee_{P\in\P} \exists \x''\Big( P(\x'') \bigwedge_{\substack{i\in\{1,...,n\} \\ P_i = P}} \x'' \neq \vec{x_i}'\Big) \bigg) \Bigg)
\end{multline*}
The full disjunct is as follows,
\begin{multline*}
\varphi_j\prime(\x,\P) = \bigwedge_{i=1}^n P_i(\x_i) \wedge \varphi_{\overline{P}}(\vec{y}) \wedge \theta(\x) \wedge \varphi_{\S}(\x) \wedge \Bigg(\Big(\forall\x\left(\neg\alpha^1_{\min}(\x)\right)\wedge ... \wedge \forall\x\left(\neg\alpha^{j-1}_{\min}(\x)\right)\Big) \to \\ 
\exists\u_1,...,\u_n\Bigg(\alpha^j_{\min}(\u_1,...,\u_n) \wedge
\bigg(\bigvee_{i = 1}^n\neg P(\u_i) \vee \bigvee_{P\in\P} \exists \v\Big( P(\v) \bigwedge_{\substack{i\in\{1,...,n\} \\ P_i = P}} \v \neq \u_i\Big) \bigg) \Bigg) \Bigg)
\end{multline*}
which, if all of the previous disjuncts have no possible assignations, eliminates the least assignation to $\P$ in that disjunct. Finally,
\begin{eqnarray*}
\F\prime(\A) &=& \mid \{ \langle\P\rangle \mid \A \models 
\exists \x \varphi\prime_1(\x,\P) \vee
\exists \x \varphi\prime_2(\x,\P) \vee
... \vee
\exists \x \varphi\prime_k(\x,\P)
 \} \mid
\end{eqnarray*}
counts exactly one assignation to $\P$ less than $F(\A)$.

Therefore,

% CASE 3

\item The counting set is:
\begin{eqnarray*}
F(\A) &=& \mid \{ \langle\P,\z\rangle \mid \A \models \exists \x \ \varphi(\x, \P,\z) \} \mid
\end{eqnarray*}
Then, we going to isolate the minimal predicate $\P$ that holds the formula true and eliminate the lexicographically smallest $\z$ that satisfies it. We mix both previous strategies.
\end{enumerate}
\end{proof}

\end{document}