\documentclass[12pt]{article}
\usepackage[utf8]{inputenc}
\usepackage{amsmath}
\usepackage{amsthm} 
\usepackage{fullpage}
\usepackage{amsfonts}
\usepackage{amssymb}
\usepackage{bm}

\def\dotminus{\mathbin{\ooalign{\hss\raise1ex\hbox{.}\hss\cr
  \mathsurround=0pt$-$}}}

\def\E1{\#\Sigma_1^{+}}

\def\Truc{\textsc{Struct}[\L]}

\def\A{{\frak A}}
\def\C{{\cal C}}
\def\F{{\cal F}}
\def\L{{\cal L}}
\def\N{\mathbb{N}}
\def\P{\vec{P}}
\def\R{\vec{R}}
\def\S{\vec{S}}
\def\X{\vec{X}}
\def\Y{\vec{Y}}
\def\Z{\vec{Z}}
%% a - arity of \X / arity of assignments \P to \X
%% b - arity of predicates in \S
%% c - arity of auxiliar predicates/variables
\def\d{\vec{d}} %% counted elements
\def\e{\vec{e}} %% counted elements
%% f - counting function
%% g - other functions
%% h - other functions
%% i - index
%% j - index
%% k - emergency index / size of tuple
%% l - emergency index / size of tuple
\def\l{\vec{\ell}}
%% m - size of variable tuple
%% n - size of predicate tuple
%% o - not used
%% p - 
%% q - 
%% r - size of \X / \P
\def\s{\vec{s}}
%% t - size of \S
\def\u{\vec{u}} %% auxiliary variables
\def\v{\vec{v}} %% auxiliary variables
\def\w{\vec{w}} %% auxiliary variables
\def\x{\vec{x}} %% quantified variables
\def\y{\vec{y}} %% auxiliary variables
\def\z{\vec{z}} %% open variables
\def\ep{\vec{o}}
\def\ga{\vec{p}}

\newtheorem{theo}{Theorem}
\newtheorem{lemma}[theo]{Lemma}
\newtheorem{claim}[theo]{Claim}
\newtheorem{coro}[theo]{Corollary}

\begin{document}


\begin{center}
{ \LARGE \bf
  Some properties of $\E1$
}
\end{center}

We define the vocabulary $\L = \{ S_1,\dots,S_t, \leq \}$, where $S_1,\dots,S_t$ have arity $b_1,\dots,b_t$. Let
\begin{eqnarray*}
\Truc &=& \{\A \mid \A \text{ is an } \L \text{-structure with a finite domain } A \text{ such that} \\
&& \leq \text{ is interpreted as a total order for } A \}.
\end{eqnarray*}
We also define a set of second order variables ${\cal X} = \{ X_i \mid i\in\N \}$ where $X_i$ has arity $a_i$, and for every $n \in \N$ there are infinite variables in ${\cal X}$ of arity $n$. A quantifier-free $\L$-formula is defined by the following grammar:
\begin{eqnarray*}
\varphi &::=& x = y \ \mid \ S_i(x_1,\dots,x_{b_i}), i \in \{1,\dots,t\} \ \mid \ x \leq y \ \mid \\
&& X_i(x_1,\dots,x_{a_i}), i\in\N \ \mid \\ 
&& (\neg \varphi) \ \mid \ (\varphi \wedge \varphi) \ \mid \ (\varphi \vee \varphi),
\end{eqnarray*}
where $x,y$ and $x_i$ are first order variables for every $i$. We now define an extended quantifier-free $\L$-formula as follows:
\begin{eqnarray*}
\varphi &::=& \alpha, \alpha \text{ is an FO-formula over } \L  \ \mid \\
&& X_i(x_1,\dots,x_{a_i}), i\in\N \ \mid \ \\
&& (\neg \varphi) \ \mid \ (\varphi \wedge \varphi) \ \mid \ (\varphi \vee \varphi).
\end{eqnarray*}
Let $\Y = (Y_1,\dots,Y_q)$ be a tuple of second-order variables of arity $c_1,\ldots,c_q$, and let $\y$ be a tuple of first order variables. For every $\L$-formula $\psi(\Y,\y)$, we define the function $f_{\psi(\Y,\y)}:\Truc \to \N$ as follows:
\begin{eqnarray*}
f_{\psi(\Y,\y)}(\A) &=& \mid \{ \langle \P, \e \rangle \mid \A \models \psi(\P,\e) \} \mid,
\end{eqnarray*}
for every $\A = \langle A, \S^{\A}, \leq^{\A} \rangle \in \Truc$, where $\P = (P_1,\ldots,P_q)$ is a tuple of predicates of arity $a_1,\ldots,a_q$, and $P_i \subseteq A^{a_i}$ for every $i \in \{1,\ldots,q\}$, and $\e$ is a tuple of elements in $\A$.

Let $\X = (X_1,\dots,X_r)$, and let $\z$ be a tuple of variables. A function $f:\Truc \to \N$ is in $\E1$ if there exists an extended quantifier-free $\L$-formula $\varphi(\x,\X,\z)$ such that $f = f_{\exists \x \: \varphi(\x,\X,\z)}.$\\

The {\em decision version} of a function $f$ is defined by the language $L_f = \{\A \mid f(\A) > 0\}$.
\begin{theo}
The decision version of a function in $\E1$ is in \textsc{P}.
\end{theo}
\begin{proof}
Let $f$ be a function in $\E1$. Then there is a formula $\varphi(\x,\X,\z)$ such that
\begin{eqnarray*}
f(\A) &=& \mid \{ \langle\P,\e\rangle \mid \A \models \exists \x \, \varphi(\x,\P,\e) \} \mid,
\end{eqnarray*}
where $\A = \langle A, \S^{\A}, \leq^{\A} \rangle \in \Truc$, $\z$ is an $m$-tuple of variables, $\e$ is an $m$-tuple of elements and $\x$ is a $k$-tuple of variables. Let $n = \vert A \vert^m$ and $\e_1,\dots,\e_n \in A^m$ be all possible evaluations for $\z$. Let $\ell = \vert A \vert^k$ and $\d_1,\dots,d_\ell\in A^k$ be all possible evaluations for $\x$. Let $\varphi_\A(\X)$ be defined as follows:
\begin{eqnarray*}
\varphi_\A(\X) = \varphi(\d_1,\X,\e_1) \vee \cdots \vee \varphi(\d_1,\X,\e_n) \vee \varphi(\d_2,\X,\e_1) \vee \cdots \vee \varphi(\d_n,\X,\e_n).
\end{eqnarray*}
Note that $\varphi_\A(\X)$ will have at least one assignment for $\X$ iff $f(\A)>0$. Let $\psi_\A(\X)$ be the formula that results of changing every satisfied sub-formula for a tautology and every non-satisfied sub-formula for a contradiction. Note that $\psi_\A(\X)$ is quantifier-free and its size is polynomial to the size of $\varphi(\x,\X,\z)$. Then, let $f'$ be defined as follows:
\begin{eqnarray*}
f'(\A) &=& \mid \{ \langle\P\rangle \mid \A \models \psi_\A(\P) \} \mid.
\end{eqnarray*}
Note that $f' \in\#\Sigma_0$ \footnote[1]{(paper)}. Also, note that for every $\A$, $f(\A) > 0$ iff $f'(\A) > 0$, so computing $f'$ is enough to solve the decision version of $f$. However, it is showed in $^{[1]}$ that any counting problem in $\#\Sigma_0$ is computable in polynomial time, therefore, the decision version of $f$ is in P.
\end{proof}

\begin{theo}
$\#\Sigma_1$ is closed under substraction $\Rightarrow$ P = NP.
\end{theo}
\begin{proof}
\#3DNF $\in \#\Sigma_1$. Let $F_{2^n}$ be a $\#\Sigma_1$ function that counts every possible truth assignment in a \#3DNF instance. Suppose that $F_{2^n}-F_{\#3DNF} \in \#\Sigma_1$. This function equals 0 only if the instanced formula is a tautology, so the decision version of it is co-NP-complete. However, as we showed previously (Theorem 1), it is also in P. Then, co-NP $\subseteq$ P.
\end{proof}

\begin{theo}
$\#\Sigma_1 \subsetneq \E1$
\end{theo}
\begin{proof}
We will show that the $\E1$ function defined by $\varphi(x_1) = (x_1 = x_1) \wedge \forall y \, S(y)$ is not in $\#\Sigma_1$.
\end{proof}

For a given function $f$, we define $f \dotminus 1$ as follows:
\begin{eqnarray*}
f \dotminus 1(\A) =
\begin{cases}
f(\A)-1, & \text{if }f(\A) > 0 \\
0, & \text{if }f(\A) = 0.
\end{cases}
\end{eqnarray*}
for every structure $\A$. A function class $\F$ is {\em closed under substraction by one} if for every function $f \in \F$, it holds that $f \dotminus 1 \in \F$.

%% F-1 F-1 F-1 F-1 F-1 F-1 F-1 F-1 F-1 F-1 F-1 F-1 F-1 F-1 F-1 F-1 F-1 F-1 F-1 F-1 F-1 F-1 F-1 F-1 F-1 F-1 F-1 F-1 F-1 F-1 
%% F-1 F-1 F-1 F-1 F-1 F-1 F-1 F-1 F-1 F-1 F-1 F-1 F-1 F-1 F-1 F-1 F-1 F-1 F-1 F-1 F-1 F-1 F-1 F-1 F-1 F-1 F-1 F-1 F-1 F-1 

\begin{theo}
$\E1$ is closed under substraction by one.
\end{theo}
\begin{proof}
For every $\L$-formula of the form $\exists\x\:\varphi(\x,\X,\z)$, let $\z = (z_1,\dots,z_d)$ and let $\X = (X_1,\dots,X_r)$. There are three possibilities regarding the size of the tuples of free variables $\z$ and $\X$: (1) $d>0$ and $r=0$ (2) $d=0$ and $r>0$ (3) $d,r>0$. This separates the proof in three cases:
\begin{enumerate}

% CASE 1

\item Let $f \in \E1$ be defined by an extended quantifier free $\L$-formula $\varphi(\x,\z)$, where $\z = (z_1,\dots,z_d)$. That is,
\begin{eqnarray*}
f(\A) &=& \mid \{ \langle\e\rangle \mid \A \models \exists \x \ \varphi(\x,\e) \} \mid,
\end{eqnarray*}
for every $\A = \langle A, \S^{\A}, \leq^{\A} \rangle \in \Truc$, where $\e \in A^d$. Our goal here is to eliminate the lexicographically smallest sequence of variables, which can be done easily. First, let $\y = (y_1,\dots,y_k)$, $\y\,^\prime = (y_1^\prime,\dots,y_k^\prime)$ and
\begin{eqnarray*}
\varphi_{k,<}(\y\,^\prime,\y) &=& \bigvee_{i = 1}^k \left( \bigwedge_{j=1}^{i-1} y_j^\prime = y_j \wedge y_i^\prime < y_i \right).
\end{eqnarray*}
This formula is true if $\y\,^\prime$ is lexicographically smaller than $\y$. Now, let $f'$ be defined by
\begin{eqnarray*}
\varphi^\prime(\x,\z) &=& \varphi(\x,\z) \wedge \exists \z\,^\prime (\varphi(\x,\z\,^\prime) \wedge \varphi_{d,<}(\z\,^\prime,\z ) ).
\end{eqnarray*}
If $f(\A)>0$, then $f'(\A)$ will count exactly one element less than $f(\A)$. Otherwise, if $f(\A)=0$, then $\A \not\models\exists \x\,\varphi(\x,\e)$ for every tuple $\e$ of elements in $A$, so $\A \not\models\exists \x\,\varphi^\prime(\x,\e)$ for every $\e$ and, therefore, $f'(\A)=0$. Hence, $f' = f\dotminus 1,$ from which we conclude that $f\dotminus 1\in\E1.$

%CASE 2

\item Let $f \in \E1$ be defined by an extended quantifier free $\L$-formula $\varphi(\x,\X)$ where $\x = (x_1,\dots,x_d)$ and $\X = (X_1,\dots,X_r)$. That is,
\begin{eqnarray*}
f(\A) &=& \mid \{ \langle\P\rangle \mid \A \models \exists \x \ \varphi(\x,\P) \} \mid \label{f1},
\end{eqnarray*}
for every $\A = \langle A, \S^{\A}, \leq^{\A} \rangle \in \Truc$, where $\P = (P_1,\ldots,P_r)$ and $P_i \subseteq A^{a_i}$ for every $i \in \{1,\ldots,r\}$. For the time being, suppose that
\begin{eqnarray}
\varphi(\x,\X) &=& \left( \bigwedge_{i=1}^n X_{\lambda(i)}(\x_i) \right) \wedge \varphi^{-}(\X,\y) \wedge \theta(\x) \wedge \beta(\x)
\end{eqnarray}
where $n$ is the number of times a non-negated variable in $\X$ is referred to, according to the function $\lambda:\{1,\ldots,n\}\to\{1,\ldots,r\}$, $\y$ is a $p$-tuple of variables in $\x$, $\varphi^{-}(\X,\y)$ is a conjunction of negated predicates in $\X$, $\theta(\x)$ defines a total order on a partition of $\x$, and $\beta(\x)$ is an FO-formula over $\L$. We also assume that $(\x_1,\dots,\x_n,\y) = \x$. As an example, the following formula is of this form:
\begin{align*}
\varphi(\x,\X) =  X_1(x_1,x_2) \wedge \neg X_1(x_3,x_4) \wedge (x_1 < x_2 \wedge x_1 = x_3 \wedge x_1 = x_4 ) \wedge \forall z\big( S_1(x_1,z) \big),
\end{align*}
where $\x = (x_1,x_2,x_3,x_4)$ and $\X = (X_1)$. Here, $n = 1$, $\lambda(1) = \lambda(2) = 1$, $\x_1 = (x_1,x_2)$ and $\y = (x_3,x_4)$. Moreover, $\varphi^{-}(\X,\y) = \neg X_1(x_3,x_4)$, $\theta(\x) = (x_1 < x_2 \wedge x_1 = x_3 \wedge x_1 = x_4)$, which defines a total order on the partition of $\x$ $\{\{x_1,x_3,x_4\},\{x_2\}\}$, and $\beta(\x) = \forall z\big( S_1(x_1,z) \big)$.

Similarly to the previous proof, we would like to eliminate the lexicographically smallest tuple of predicates that satisfies the formula \eqref{f1}. Let $\u_i$ be a $a_{\lambda(i)}$-tuple of variables for every $i \in \{1,\dots,n\}$, and let $m = \sum_{i = 1}^n a_{\lambda(i)}$ be the number of variables of $(\x_1,\dots,\x_n)$. We now define
\begin{multline*}
\alpha_{\min}(\u_1,\dots,\u_n) = \exists\y\Big[ \alpha(\u_1,\dots,\u_n,\y)\wedge \\ \forall\v_1\cdots\forall\v_n\forall\w\Big(\big(\alpha(\v_1,\dots,\v_n,\w)\wedge\bigvee_{i=1}^n(\u_i\neq\v_i)\big)\to \varphi_{m,<}((\u_1,\dots,\u_n),(\v_1,\dots,\v_n))\Big)\Big],
\end{multline*}
where $\alpha(\x) = \theta(\x) \wedge \beta(\x)$. Note that $\alpha_{\min}$ is satisfied only by the lexicographically \linebreak smallest assignment $(\d_1,\dots,\d_n)$ to $(\x_1,\dots,\x_n)$ such that $\A\models\theta(\d_1,\dots,\d_n,\l)$ and $\A\models\beta(\d_1,\dots,\d_n,\l)$ for some $\l \in A^p$. Our new formula is
\begin{multline}
\varphi^\prime(\x,\X) = \left( \bigwedge_{i=1}^n X_{\lambda(i)}(\x_i) \right) \wedge \varphi^{-}(\X,\y) \wedge \theta(\x) \wedge \beta(\x)\wedge\exists\u_1\cdots\exists\u_n\bigg[\alpha_{\min}(\u_1,\dots,\u_n) \wedge \\ \bigg(\bigg(\bigvee_{i = 1}^{n}\neg X_{\lambda(i)}(\u_i) \bigg) \vee \bigvee_{i=1}^r \exists \v\Big( X_i(\v) \wedge \bigwedge_{j\in[1,n]:\: \lambda(j) = i} \v \neq \u_j\Big) \bigg) \bigg] \label{f2}.
\end{multline}
We now show a result by which the main proof will follow.
\begin{lemma}  \label{lemmaone}
$f_{\exists \x \: \varphi^\prime(\x,\X)} = f_{\exists \x \: \varphi(\x,\X)} \dotminus 1$.
\end{lemma}
\begin{proof}
Let $\A \in \Truc$. Consider two cases: assume first that $\A\models\exists\x\,\varphi(\x,\R)$ for some assignment $\R$ to $\X$. Let $\d = (\d_1,\dots,\d_n,\ep)$ be the lexicographically smallest assignment to $\x$ for which $\A\models\alpha(\d)$, where $\d_i$ is the respective assignment to $\x_i$, for every $i\in\{1,\dots,n\}$, and $\ep$ is an assignment for $\y$. Consider now the tuple $\P = (P_1,\dots,P_r)$ where $P_i = \bigcup_{\lambda(j)=i}\{\d_j\}$. We will show that this assignment to $\X$ is such that (a) $\A\models\exists\x\,\varphi(\x,\P),$ (b) $\A\not\models\exists\x\,\varphi^\prime(\x,\P)$ and (c) $\P$ is the only assignment that satisfies (a) and (b).
\begin{enumerate}
\item[(a)] By contradiction, suppose that $\A\not\models\exists\x\,\varphi(\x,\P)$. That is, there is no assignment $\s$ to $\x$ such that $\A\models\varphi(\s,\P).$ Since $\d$ is such that $\A\models \bigwedge_{i=1}^n P_{\lambda(i)}(\d_i)$ and $\A\models \alpha(\d)$, it follows that $\A\not\models\varphi^{-}(\P,\ep)$ (since $\alpha(\d)=\theta(\d)\wedge\beta(\d)$). Therefore, there is an $i\in\{1,\ldots,n\}$ such that $\neg P_{\lambda(i)}(\d_i)$ appears in $\varphi^{-}(\P,\ep)$. Then, there is an $a_{\lambda(i)}$-tuple $\z$ in $\y$ such that $\neg X_{\lambda(i)}(\z)$ appears in $\varphi^{-}(\X,\x)$. We know that either (1) $\theta(\x)\models \z = \x_i$, (2) $\theta(\x)\models \z < \x_i$ or (3) $\theta(\x)\models \z > \x_i$. Considering that (2) and (3) are not possible since $\A\models\theta(\d)$, and both $\z$ and $\x_i$ are assigned the value $\d_i$, then $\theta(\x)\models \z = \x_i$. But if this is the case, then $X_{\lambda(i)}(\x_i), \neg X_{\lambda(i)}(\z)$ and $\z = \x_i$ are all logical consequences of $\varphi(\x,\X)$ from which we conclude that $\A\not\models\exists\x\,\varphi(\x,\R^\prime)$ for every possible assignment $\R^\prime$ to $\X$, which leads to a contradiction.
\item[(b)] By the construction of $\P$, we see that 
$$\A\not\models\bigvee_{i = 1}^{n}\neg P_{\lambda(i)}(\d_i) \text{ and that } \A\not\models\bigvee_{i=1}^r \exists \v\Big( P_i(\v) \wedge \bigwedge_{j\in[1,n]:\: \lambda(j) = i} \v \neq \u_j\Big).$$ However, the only possible assignment to $\alpha_{\min}(\x_1,\dots,\x_n)$ is $(\d_1,\dots,\d_n)$. Then, $\A\not\models\exists\x\,\varphi^\prime(\x,\P)$.
\item[(c)] By contradiction, let $\P^\prime \neq \P$ be such that $\A\models\exists\x\,\varphi(\x,\P^\prime)$ and $\A\not\models\exists\x\,\varphi^\prime(\x,\P^\prime)$. We consider two cases: first, suppose that $\P^\prime$ is missing a tuple of $\P$. Let $i\in\{1,\ldots,n\}$ such that in $\d_i$ is not in $\P^\prime$, then $\A\models \neg P^\prime_i(\d_i)$. And so,$$\A\models \bigg(\bigvee_{i = 1}^{n}\neg X_{\lambda(i)}(\d_i) \bigg), $$ from which we conclude that $\A\models\exists\x\,\varphi^\prime(\x,\P^\prime)$. Second, suppose there is some predicate $\P^\prime_i$ in $\P^\prime$ which has a tuple that $\P_i$ does not have. If this is the case, then $$\A\models\bigvee_{i=1}^r \exists \v\Big( P_i(\v) \wedge \bigwedge_{j\in[1,n]:\: \lambda(j) = i} \v \neq \u_j\Big),$$ so $\A\models\exists\x\,\varphi^\prime(\x,\P^\prime)$. On both cases, we have a contradiction.
\end{enumerate}
With this, we conclude that for every $\A\in\Truc$ such that $f_{\exists\x\,\varphi(\x,\X)}(\A)>0,$ then $f_{\exists\x\,\varphi^\prime(\x,\X)}(\A) = f_{\exists\x\,\varphi(\x,\X)}(\A)-1.$

Second, assume that there is no assignment $\R$ to $\X$ such that $\A\models\exists\x\,\varphi(\x,\R)$. Let $\P$ be an arbitrary assignment to $\X$. Since $\A\not\models\exists\x\,\varphi(\x,\P)$, we see that $\A\not\models\exists\x\,(\varphi(\x,\P)\wedge\psi(\x,\P))$ for any formula $\psi(\x,\P)$. It follows that there is no assignment $\R$ to $\X$ such that $\A\models\exists\x\,\varphi^\prime(\x,\R)$. And so, for every $\A\in\Truc$ such that $f_{\exists\x\,\varphi(\x,\X)}(\A)=0,$ then $f_{\exists\x\,\varphi^\prime(\x,\X)}(\A) = 0.$ We conclude that $f_{\exists\x\,\varphi^\prime(\x,\X)} = f_{\exists\x\,\varphi(\x,\X)}\dotminus 1.$
\end{proof}

We now continue with the general case, in which $\varphi(\x,\X)$ is an arbirtary extended quantifier-free $\L$-formula. By using a standard DNF transformation algorithm and considering FO-formulas over $\L$ as literals, we can find formulas $\hat{\varphi}_i(\x,\X)$ with $i\in\{1,\ldots,\hat{k}\}$ such that
\begin{eqnarray*}
\varphi(\x,\X) &\equiv& \hat{\varphi}_1(\x,\X) \vee \hat{\varphi}_2(\x,\X) \vee \dots  \vee \hat{\varphi}_{\hat{k}}(\x,\X),
\end{eqnarray*}
where, for every $i\in\{1,\ldots,\hat{k}\}$, 
\begin{eqnarray*}
\hat{\varphi}_i(\x,\X) &=& \left( \bigwedge_{j=1}^{n_i} X_{\lambda_i(j)}(\x_{i,j}) \right) \wedge \hat{\varphi}^{-}_i(\X,\y_i)  \wedge \beta_i(\x).
\end{eqnarray*}
The function $\lambda_i$ is defined analogously to the first part of the proof, the tuple $\x_{i,j}$ has $a_{\lambda_i(j)}$ variables for $j\in\{1,\ldots,n_i\}$, $\y_i$ has $p_i$ variables and are such that $(\x_{i,1},\ldots,\x_{i,n_i},\y) = \x.$ The formulas $\hat{\varphi}^{-}_i(\X,\y_i)$, $\beta_i(\x)$, are also defined analogously. Let $g$ be a function that counts the number of possible orders over partitions on a $d$-tuple of variables. Let the formulas $\theta^i(\x)$ for $i\in\{1,\ldots,g(d)\}$ represent each of these orders over $\x$. Note that for every formula $\psi(\x)$,
\begin{eqnarray*}
\exists\x\:\psi(\x) &\equiv& \exists\x(\psi(\x)\wedge\theta^1(\x)) \vee \cdots \vee \exists\x(\psi(\x)\wedge\theta^{g(d)}(\x)).
\end{eqnarray*}
We define the following formulas $\check{\varphi}_i(\x,\X)$, for $i\in\{1,\ldots,\check{k}\},$ as follows: 
\begin{eqnarray*}
\check{\varphi}_1(\x,\X) &=& \hat{\varphi}_1(\x,\X) \wedge \theta^1(\x), \\
& \vdots & \\
\check{\varphi}_{g(d)}(\x,\X) &=& \hat{\varphi}_1(\x,\X) \wedge \theta^{g(d)}(\x), \\
\check{\varphi}_{g(d)+1}(\x,\X) &=& \hat{\varphi}_2(\x,\X) \wedge \theta^1(\x), \\
& \vdots & \\
\check{\varphi}_{2\cdot g(d)}(\x,\X) &=& \hat{\varphi}_2(\x,\X) \wedge \theta^{g(d)}(\x), \\
& \vdots & \\
\check{\varphi}_{(\hat{k}-1) g(d)+1}(\x,\X) &=& \hat{\varphi}_{\hat{k}}(\x,\X) \wedge \theta^1(\x), \\
& \vdots & \\
\check{\varphi}_{\hat{k} \cdot g(d)}(\x,\X) &=& \hat{\varphi}_{\hat{k}}(\x,\X) \wedge \theta^{g(d)}(\x).
\end{eqnarray*}
Note that $\check{k} = \hat{k} \cdot g(d).$ Having every disjunct with a total order allows us to eliminate the ones that are unsatisfiable for every $\L-$structure $\A$, that is, every $i\in\{1,\ldots,\check{k}\}$ such that for every assignment $\s$ to $\x,$ $f_{\check{\varphi}_i(\s,\X)} = 0.$ Let $k$ be the number of disjuncts that are left after eliminating the unsatisfiable disjuncts. We order the remaining disjuncts and define the disjuncts $\varphi_i(\x,\X)$ such that for every $j\in\{1,\ldots,\check{k}\}$ where $\check{\varphi}_j(\x,\X)$ is satisfiable, there is an $i\in\{1,\ldots,k\}$ such that $\varphi_i(\x,\X) = \check{\varphi}_j(\x,\X).$

As seen in lemma \ref{lemmaone}, for every $\L-$structure $\A$ and $i\in\{1,\ldots,k\}$, if there is an assignment $\s$ to $\x$ such that $\A\models\alpha_i(\s)$ but $f_{\varphi_i(\s,\X)}(\A) = 0,$ then $f_{\exists\x\:\varphi_i(\x,\X)}(\A) = 0$, so it follows that for every assignment $\s$ to $\x$ such that $\A\models\alpha_i(\s)$, then $f_{\exists\x\:\varphi_i(\x,\X)}(\A) > 0.$

Now we can define the formula like we did on the first part of the proof. For every $i\in\{1,\ldots,k\}$,
\begin{eqnarray*}
\varphi_i(\x,\X) &=& \left( \bigwedge_{j=1}^{n_i} X_{\lambda_i(j)}(\x_{i,j}) \right) \wedge \varphi^{-}_i(\X,\y_i) \wedge \theta_i(\x) \wedge \beta_i(\x),
\end{eqnarray*}
 Also, note that
\begin{eqnarray*}
\varphi(\x,\X) &\equiv& \varphi_1(\x,\X) \vee \varphi_2(\x,\X) \vee \dots  \vee \varphi_k(\x,\X).
\end{eqnarray*}

Our plan now is to exclude the lexicographically smallest assignment $\P$ such that $\A\models\varphi_1(\x,\P)$, and if there is no such $\P$, exclude the lexicographically smallest $\P$ such that $\A\models\varphi_2(\x,\P)$, and so on. As we already know how to exclude that assignment in the first disjunct, we now move to the next disjuncts. Similarly to the first part of the proof, let $m_i = \sum_{j = 1}^{n_i} a_{\lambda_i(j)}$ and let
\begin{multline*}
\alpha^i_{\min}(\u_1,\dots,\u_n) = \exists\y\Big[ \alpha_i(\u_1,\dots,\u_n,\y)\wedge \\ \forall\v_1\cdots\forall\v_n\forall\w\Big(\big(\alpha_i(\v_1,\dots,\v_n,\w)\wedge\bigvee_{j=1}^n(\u_j\neq\v_j)\big)\to \varphi_{m_i,<}((\u_1,\dots,\u_n),(\v_1,\dots,\v_n))\Big)\Big],
\end{multline*}
where $\u_j,\v_j,$ for $\{1,\ldots,n_i\}$ and $\w$ have the adequate number of variables. Also, let
\begin{multline*}
\psi_i(\X) = \forall\x\:\neg\alpha_i(\x) \vee \exists\u_1\cdots\exists\u_n\bigg[\alpha^i_{\min}(\u_1,\dots,\u_n) \\ \wedge \bigg(\bigg(\bigvee_{j = 1}^{n}\neg X_{\lambda_i(j)}(\u_i) \bigg) \vee \bigvee_{j=1}^r \exists \v\Big( X_j(\v) \wedge \bigwedge_{\ell\in[1,n]: \lambda_i(\ell) = j} \v \neq \u_\ell\Big) \bigg) \bigg].
\end{multline*}
Our new formula $\varphi_i^\prime(\x,\X)$ is defined as follows:
\begin{multline}
\varphi_i^\prime(\x,\X) = \varphi_i(\x,\X) \wedge \psi_1(\X) \wedge (\exists\v\:\alpha_1(\v)\vee\psi_2(\X)) \wedge \cdots \wedge \\ (\exists\v\:\alpha_1(\v)\vee\cdots\vee\exists\v\:\alpha_{i-1}(\v)\vee\psi_i(\X)).
\end{multline}
Let $\varphi^\prime(\x,\X) = \varphi_1^\prime(\x,\X)\vee\cdots\vee\varphi_k^\prime(\x,\X).$ We will now show that $f_{\exists\x\varphi^\prime(\x,\X)} = f_{\exists\x\varphi(\x,\X)}\dotminus 1.$ Let $\A\in\Truc$. Suppose first that there is at least an assignment $\R$ to $\X$ such that $\A\models\exists\x\:\varphi(\x,\R).$ Let $q$ be the least $i\in\{1,\ldots,k\}$ such that there exists at least an assignment $\R^\prime$ to $\X$ for which $\A\models\exists\x\:\varphi_i(\x,\R^\prime).$ Let $\P$ be an assignment to $\X$ defined as in Lemma \ref{lemmaone} for this sub-formula. As we did in Lemma \ref{lemmaone} we will show that $\P$ is such that (a) $\A\models\exists\x\:\varphi(\x,\P),$ (b) $\A\not\models\exists\x\:\varphi^\prime(\x,\P),$ and (c) $\P$ is the only assignment to $\X$ that satisfies (a) and (b)
\begin{enumerate}
\item[(a)] We know that $\A\models\exists\x\:\varphi_q(\x,\P),$ so it immediately follows that $\A\models\exists\x\:\varphi(\x,\P).$
\item[(b)] We will show that $\A\not\models\exists\x\:\varphi^\prime_i(\x,\P)$ for (1) $i\in\{1,\ldots,q-1\}$, and (2) $i\in\{q,\ldots,k\}$. (1) By the choice of $q$, it holds that $\A\not\models\exists\x\:\varphi^\prime_i(\x,\P)$ for every $i\in\{1,\ldots,q-1\}$ since there is no possible assignment to $\X$ for any of their sub-formulas $\varphi_i(\x,\X)$. (2) We can use the proof in Lemma \ref{lemmaone} to see that $\A\not\models\exists\x\:\psi_q(\P).$ For each $i\in\{q,\ldots,k\}$, the sub-formula $$ \hat{\psi}_q(\X) =  (\exists\v\:\varphi_1(\v,\X)\vee\cdots\vee\exists\v\:\varphi_{q-1}(\v,\X)\vee\psi_q(\X)) $$ appears in $\varphi^\prime_i(\x,\X)$. However, $\A\not\models\exists\x\:\hat{\psi}_q(\P)$ so $\A\not\models\exists\x\:\varphi^\prime_i(\P)$. And so, we conclude that $\A\not\models\exists\x\:\varphi^\prime(\x,\P).$
\item[(c)] Suppose there is an assignment $\P^\prime \neq \P$ to $\X$ that satisfies both (a) and (b). As we deduce from the part (c) of Lemma \ref{lemmaone}, $\P$ is the only assignment to $\X$ such that $\A\not\models\:\psi_q(\P)$, so necessarily $\A\models\:\psi_q(\P^\prime)$. Since $\P^\prime$ assigned to $\X$ satisfies (a), then $\A\models\exists\x\:\varphi(\x,\P^\prime).$ Let $q^\prime$ be such that $\A\models\exists\x\:\varphi_{q^\prime}(\x,\P^\prime).$ By the choice of $q,$ then $q^\prime \geq q,$ but since $\P \neq \P^\prime,$ then $q^\prime > q.$ By the choice of $q,$ every $i\in\{1,\ldots,q-1\}$ is such that $\A\models\forall\x\:\neg\alpha_i(\x),$ so $\A\models\psi_i(\P^\prime)$ for each $i$. But also, $\A\models\exists\x\:\alpha_q(\x),$ which means that $\A\models\exists\x\:\varphi^\prime_q(\x,\P^\prime),$ and so, $\A\models\exists\x\:\varphi^\prime(\x,\P^\prime),$ which is a contradiction.
\end{enumerate}

With this, we conclude that for every $\A\in\Truc$ such that $f_{\exists\x\,\varphi(\x,\X)}(\A)>0,$ then $f_{\exists\x\,\varphi^\prime(\x,\X)}(\A) = f_{\exists\x\,\varphi(\x,\X)}(\A)-1.$

Second, assume that there is no assignment $\R$ to $\X$ such that $\A\models\exists\x\,\varphi_i(\x,\R)$ for any $i\in\{1,\ldots,k\}$. Let $\P$ be an arbitrary assignment to $\X$. Since $\A\not\models\exists\x\,\varphi_i(\x,\P)$, for any $i,$ we see that $\A\not\models\exists\x\,(\varphi(\x,\P)\wedge\psi(\x,\P))$ for any formula $\psi(\x,\P)$. It follows that there is no assignment $\R$ to $\X$ such that $\A\models\exists\x\,\varphi^\prime_i(\x,\R),$ for any $i\in\{1,\ldots,k\}$. And so, for every $\A\in\Truc$ such that $f_{\exists\x\,\varphi(\x,\X)}(\A)=0,$ then $f_{\exists\x\,\varphi^\prime(\x,\X)}(\A) = 0.$ We conclude that $f_{\exists\x\,\varphi^\prime(\x,\X)} = f_{\exists\x\,\varphi(\x,\X)}\dotminus 1.$

% CASE 3

\item The counting set is:
\begin{eqnarray*}
F(\A) &=& \mid \{ \langle\P,\z\rangle \mid \A \models \exists \x \ \varphi(\x, \P,\z) \} \mid
\end{eqnarray*}
Then, we going to isolate the minimal predicate $\P$ that holds the formula true and eliminate the lexicographically smallest $\z$ that satisfies it. We mix both previous strategies.
\end{enumerate}
\end{proof}

\end{document}