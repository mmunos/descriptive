\documentclass[12pt]{article}
\usepackage[utf8]{inputenc}
\usepackage{amsmath}
\usepackage{amsthm} 
\usepackage{fullpage}
\usepackage{amsfonts}
\usepackage{amssymb}
\usepackage{bm}

\def\A{{\frak A}}
\def\C{{\cal C}}
\def\F{{\cal F}}
\def\L{{\cal L}}
\def\N{\mathbb{N}}
\def\P{\vec{P}}
\def\Q{\vec{Q}}
\def\R{\vec{R}}
\def\S{\vec{S}}
\def\X{\vec{X}}
\def\Y{\vec{Y}}
\def\Z{\vec{Z}}
%% a - arity of \X / arity of assignations \P to \X
%% b - arity of predicates in \S
%% c - arity of auxiliar predicates/variables
\def\d{\vec{d}} %% counted elements
\def\e{\vec{e}} %% counted elements
%% f - counting function
%% g - other functions
%% h - other functions
%% i - index
%% j - index
%% k - emergency index / size of tuple
%% l - emergency index / size of tuple
%% m - size of variable tuple
%% n - size of predicate tuple
%% o - not used
%% p - 
%% q - 
%% r - size of \X / \P
\def\s{\vec{s}}
%% t - size of \S
\def\u{\vec{u}} %% auxiliary variables
\def\v{\vec{v}} %% auxiliary variables
\def\w{\vec{w}} %% auxiliary variables
\def\x{\vec{x}} %% quantified variables
\def\y{\vec{y}} %% auxiliary variables
\def\z{\vec{z}} %% open variables
\def\ep{\vec{\varepsilon}}
\def\ga{\vec{\gamma}}

\newtheorem{theo}{Theorem}
\newtheorem{lemma}[theo]{Lemma}
\newtheorem{claim}[theo]{Claim}
\newtheorem{coro}[theo]{Corollary}

\begin{document}


\begin{center}
{ \LARGE \bf
  Some properties of $\#\Sigma_1^{+}$
}
\end{center}

We define the vocabulary $\L = \{ S_1,\dots,S_t, \leq \}$, where $S_1,\dots,S_t$ have arity $b_1,\dots,b_t$. Let
\begin{eqnarray*}
\textsc{Struct}[\L] &=& \{\A \mid \A \text{ is an } \L \text{-structure with a finite domain } A \text{ such that} \\
&& \leq \text{ is interpreted as a total order for } A \}.
\end{eqnarray*}
We also define a set of second order variables ${\cal X} = \{ X_i \mid i\in\N \}$ where $X_i$ has arity $a_i$, and for every $n \in N$ there are infinite variables in ${\cal X}$ of arity $n$. A quantifier-free $\L$-formula is defined by the following grammar:
\begin{eqnarray*}
\varphi &::=& x = y \ \mid \ S_i(x_1,\dots,x_{b_i}), i \in \{1,\dots,t\} \ \mid \ x \leq y \ \mid \\
&& X_i(x_1,\dots,x_{a_i}), i\in\N \ \mid \\ 
&& (\neg \varphi) \ \mid \ (\varphi \wedge \varphi) \ \mid \ (\varphi \vee \varphi),
\end{eqnarray*}
where $x,y$ and $x_i$ are first order variables for every $i$. We now define an extended quantifier-free $\L$-formula as follows:
\begin{eqnarray*}
\varphi &::=& \alpha, \alpha \text{ is an FO-formula over } \L  \ \mid \\
&& X_i(x_1,\dots,x_{a_i}), i\in\N \ \mid \ \\
&& (\neg \varphi) \ \mid \ (\varphi \wedge \varphi) \ \mid \ (\varphi \vee \varphi).
\end{eqnarray*}
Let $\X = (X_1,\dots,X_r)$ and $\P = (P_1,\dots,P_r)$, where $P_i$ is a predicate of arity $a_i$, for every $i$. Let $\z$ be a tuple of variables and $\e$ be a tuple of constants with the same number of elements as $\z$. A function $f:\textsc{Struct}[\L] \to \mathbb{N}$ is in $\#\Sigma_1^+$ if there exists an extended quantifier-free $\L$-formula $\varphi(\x,\X,\z)$ such that
\begin{eqnarray*}
f(\A) &=& \mid \{ \langle \P, \e \rangle \mid \A \models \exists \x \: \varphi(\x,\P,\e) \} \mid,
\end{eqnarray*}
for every $\A \in \textsc{Struct}[\L]$.\\

The {\em decision version} of a function $f$ is the language $L_f = \{\A \mid f(\A) > 0\}$.
\begin{theo}
The decision version of a function in $\#\Sigma_1^{+}$ is in \textsc{P}.
\end{theo}
\begin{proof}
Let $f$ be a function in $\#\Sigma_1^{+}$. Then there is a formula $\varphi(\x,\X,\z)$ such that
\begin{eqnarray*}
f(\A) &=& \mid \{ \langle\P,\e\rangle \mid \A \models \exists \x \, \varphi(\x,\P,\e) \} \mid,
\end{eqnarray*}
where $\A = \langle A, \S^{\A}, \leq^{\A} \rangle \in \textsc{Struct}[\L]$, $\z$ is an $m$-tuple of variables, $\e$ is an $m$-tuple of elements and $\x$ is a $k$-tuple of variables. Let $n = \vert A \vert^m$ and $\e_1,\dots,\e_n \in A^m$ be all possible evaluations for $\z$. Let $\ell = \vert A \vert^k$ and $\d_1,\dots,d_\ell\in A^k$ be all possible evaluations for $\x$. Let $\varphi_\A(\X)$ be defined as follows:
\begin{eqnarray*}
\varphi_\A(\X) = \varphi(\d_1,\X,\e_1) \vee \cdots \vee \varphi(\d_1,\X,\e_n) \vee \varphi(\d_2,\X,\e_1) \vee \cdots \vee \varphi(\d_n,\X,\e_n).
\end{eqnarray*}
Note that $\varphi_\A(\X)$ will have at least one assignation for $\X$ iff $f(\A)>0$. Let $\psi_\A(\X)$ be the formula that results of changing every satisfied sub-formula for a tautology and every non-satisfied sub-formula for a contradiction. Note that $\psi_\A(\X)$ is quantifier-free and its size is polynomial to the size of $\varphi(\x,\X,\z)$. Then, let $f'$ be defined as follows:
\begin{eqnarray*}
f'(\A) &=& \mid \{ \langle\P\rangle \mid \A \models \psi_\A(\P) \} \mid.
\end{eqnarray*}
Note that $f' \in\#\Sigma_0$ \footnote[1]{(paper)}. Also, note that for every $\A$, $f(\A) > 0$ iff $f'(\A) > 0$, so computing $f'$ is enough to solve the decision version of $f$. However, it is showed in $^{[1]}$ that any counting problem in $\#\Sigma_0$ is computable in polynomial time, therefore, the decision version of $f$ is in P.
\end{proof}

\begin{theo}
$\#\Sigma_1$ is closed under substraction $\Rightarrow$ P = NP.
\end{theo}
\begin{proof}
\#3DNF $\in \#\Sigma_1$. Let $F_{2^n}$ be a $\#\Sigma_1$ function that counts every possible truth assignment in a \#3DNF instance. Suppose that $F_{2^n}-F_{\#3DNF} \in \#\Sigma_1$. This function equals 0 only if the instanced formula is a tautology, so the decision version of it is co-NP-complete. However, as we showed previously (Theorem 1), it is also in P. Then, co-NP $\subseteq$ P.
\end{proof}

\begin{theo}
$\#\Sigma_1 \subsetneq \#\Sigma_1^{+}$
\end{theo}
\begin{proof}
We will show that the $\#\Sigma_1^{+}$ function defined by $\varphi(x_1) = (x_1 = x_1) \wedge \forall y \, S(y)$ is not in $\#\Sigma_1$.
\end{proof}

%% F-1 F-1 F-1 F-1 F-1 F-1 F-1 F-1 F-1 F-1 F-1 F-1 F-1 F-1 F-1 F-1 F-1 F-1 F-1 F-1 F-1 F-1 F-1 F-1 F-1 F-1 F-1 F-1 F-1 F-1 
%% F-1 F-1 F-1 F-1 F-1 F-1 F-1 F-1 F-1 F-1 F-1 F-1 F-1 F-1 F-1 F-1 F-1 F-1 F-1 F-1 F-1 F-1 F-1 F-1 F-1 F-1 F-1 F-1 F-1 F-1 

A function class $\F$ is {\em closed under substraction by one} if for every function $f \in \F$, there exists $f' \in \F$ such that 
\begin{eqnarray*}
f'(\A) =
\begin{cases}
f(\A)-1, & \text{if }f(\A) > 0 \\
0, & \text{if }f(\A) = 0.
\end{cases}
\end{eqnarray*}
\begin{theo}
$\#\Sigma_1^+$ is closed under substraction by one.
\end{theo}
\begin{proof}
A function in $\#\Sigma_1^+$ has three possible ways of counting: counting only variables, only predicates, and predicates with variables. Let us separate each of this cases on the classes $\#\Sigma_1^{+(a)}$, $\#\Sigma_1^{+(b)}$ and $\#\Sigma_1^{+(c)}$ respectively:
\begin{enumerate}

% CASE 1

\item Let $f \in \#\Sigma_1^{+(a)}$, which is defined by an $\L$-formula $\varphi(\x,\z)$, where $\z = (z_1,\dots,z_d)$. That is,
\begin{eqnarray*}
f(\A) &=& \mid \{ \langle\e\rangle \mid \A \models \exists \x \ \varphi(\x,\e) \} \mid,
\end{eqnarray*}
for every $\A \in \textsc{Struct}[\L]$. Our goal here is to eliminate the lexicographically smallest sequence of variables, which we can do easily. First, let $\y = (y_1,\dots,y_k)$, $\y\,^\prime = (y_1^\prime,\dots,y_k^\prime)$ and
\begin{eqnarray*}
\varphi_{k,<}(\y\,^\prime,\y) &=& \bigvee_{i = 1}^k \left( \bigwedge_{j=1}^{i-1} y_j = y_j^\prime \wedge y_i < y_i^\prime \right).
\end{eqnarray*}
This formula is true if $\y\,^\prime$ is lexicographically smaller than $\y$. Now, let $f'$ be defined by
\begin{eqnarray*}
\varphi^\prime(\x,\z) &=& \varphi(\x,\z) \wedge \exists \z\,^\prime (\varphi(\x,\z\,^\prime) \wedge \varphi_{d,<}(\z\,^\prime,\z ) ).
\end{eqnarray*}
If $f(\A)>0$, then $f'(\A)$ will count exactly one element less than $f(\A)$. Otherwise, if $f(\A)=0$, then $\A \not\models\exists \x\,\varphi(\x,\e)$ for every tuple $\e$ of elements in $A$, so $\A \not\models\exists \x\,\varphi^\prime(\x,\e)$ for every $\e$, therefore $f'(\A)=0$. We conclude that $\#\Sigma_1^{+(a)}$ is closed under substraction by one.

%CASE 2

\item Let $f \in \#\Sigma_1^{+(b)}$, which is defined by an $\L$-formula $\varphi(\x,\X)$ where $\x = (x_1,\dots,x_d)$ and $\X = (X_1,\dots,X_r)$. Then
\begin{eqnarray*}
f &=& \mid \{ \langle\P\rangle \mid \A \models \exists \x \ \varphi(\x,\P) \} \mid \label{f1},
\end{eqnarray*}
for every $\A \in \textsc{Struct}[\L]$. For the time being, suppose that
\begin{eqnarray}
\varphi(\x,\X) &=& \left( \bigwedge_{i=1}^n Y_i(\x_i) \right) \wedge \varphi^{-}(\X,\y) \wedge \theta(\x) \wedge \beta(\x)
\end{eqnarray}
where $Y_i$ is in $\X$, $Y_i$ is of arity $c_i$ and $\x_i$ is a $c_i$-tuple of variables in $\x$ for all $i \in \{1,\dots,n\}$, $\y$ is a $p$-tuple of variables in $\x$, $\varphi^{-}(\X,\y)$ is a conjunction of negated predicates in $\P$, $\theta(\x)$ defines a total order on a partition of $\x$, and $\beta(\x)$ is an FO-formula over $\L$. We also assume that $(\x_1,\dots,\x_n,\y) = \x$. As an example, the following formula is of this form:
\begin{align*}
\varphi(\x,\X) =  X_1(x_1,x_2) \wedge \neg X_1(x_3,x_4) \wedge (x_1 < x_2 \wedge x_1 = x_3 \wedge x_1 = x_4 ) \wedge \forall z\big( S_1(x_1,z) \big),
\end{align*}
where $\x = (x_1,x_2,x_3,x_4)$ and $\X = (X_1)$. Here, $n = 1$, $Y_1 = X_1$, $\x_1 = (x_1,x_2)$ and $\y = (x_3,x_4)$. Moreover $\varphi^{-}(\X,\y) = \neg X_1(x_3,x_4)$, $\theta(\x) = (x_1 < x_2 \wedge x_1 = x_3 \wedge x_1 = x_4)$, which defines a total order on the partition $\{\{x_1,x_3,x_4\},\{x_2\}\}$, and $\beta(\x) = \forall z\big( S_1(x_1,z) \big)$.

Similarly to the previous proof, we would like to eliminate the lexicographically smallest tuple of predicates that satisfies the formula \eqref{f1}. Let $\u_i$ be a $c_i$-tuple of variables for every $i \in \{1,\dots,n\}$, and let $m = \sum_{i = i}^n c_i$ be the number of variables of $(\x_1,\dots,\x_n)$. We now define
\begin{multline*}
\alpha_{\min}(\u_1,\dots,\u_n) = \\ \exists\y\left( \alpha(\u_1,\dots,\u_n,\y)\wedge \forall\v_1\cdots\forall\v_n\w(\alpha(\v_1,\dots,\v_n,\w)\rightarrow \varphi_{m,<}((\u_1,\dots,\u_n),(\v_1,\dots,\v_n))\right),
\end{multline*}
where $\alpha(\x) = \theta(\x) \wedge \beta(\x)$. Note that $\alpha_{\min}$ is satisfied only by the lexicographically \linebreak smallest assignment $(\d_1,\dots,\d_n)$ to $(\x_1,\dots,\x_n)$ such that $\A\models\theta(\d_1,\dots,\d_n,\ell)$ and $\A\models\beta(\d_1,\dots,\d_n,\ell)$ for some $\ell \in A^p$. Let $\d = (\d_1,\dots,\d_n)$. Our new formula is
\begin{multline}
\varphi^\prime(\x,\X) = \left( \bigwedge_{i=1}^n Y_i(\x_i) \right) \wedge \varphi^{-}(\X,\y) \wedge \theta(\x) \wedge \beta(\x)\wedge\exists\u_1\cdots\exists\u_n\bigg[\alpha_{\min}(\u_1,\dots,\u_n) \wedge \\ \bigg(\bigg(\bigvee_{i = 1}^{n}\neg Y_i(\u_i) \bigg) \vee \bigvee_{Y\in\X} \exists \v\Big( Y(\v) \wedge \bigwedge_{i\in[1,n]: Y_i = Y} \v \neq \u_i\Big) \bigg) \bigg] \label{f2}.
\end{multline}
We will now show a result by which the main proof will follow. Consider two cases: assume first that $\A\models\exists\x\,\varphi(\x,\R)$ for some assignation $\R$ to $\X$. Let $\d = (\d_1,\dots,\d_n,\ep)$ be the lexicographically smallest assignment which satisfies $\alpha(\x)$, where $\d_i$ is the respective assignment to $\x_i$, for every $i\in\{1,\dots,n\}$, and $\ep$ is an assignment for $\y$. Consider now the tuple $\P = (P_1,\dots,P_r)$ where $P_i = \{\d_j \mid j\in\{1,\dots,n\} \text{ and } X_i = Y_j \}$. We will show that this assignation to $\X$ is such that (a) $\A\models\exists\x\,\varphi(\x,\P)$ (b) $\A\not\models\exists\x\,\varphi^\prime(\x,\P)$ and (c) $\P$ is the only assignment that does this.
\begin{enumerate}
\item[(a)] By contradiction, suppose that $\A\not\models\exists\x\,\varphi(\x,\P)$. That is, there is no assignment $\s$ to $\x$ such that $\varphi(\s,\P)$ is true. Let $Q_i$ be the respective assignment of $\P$ to $Y_i$, for $i\in\{1,\dots,n\}$. Since $\d$ is such that $\A\models \bigwedge_{i=1}^n Q_i(\d_i)$, $\A\models\theta(\d)$ and  $\A\models\beta(\d)$, then $\A\not\models\varphi^{-}(\P,\ep)$. Therefore, there is a predicate $Q_i$ which contains a $c_i$-tuple $\ga$ in $\ep$ such that $\neg \Q_i(\ga)$ appears in $\varphi^{-}(\P,\ep)$. Then, there exists an $i\in\{1,\cdots,n\}$ such that $\neg Y_i(\z)$ appears in $\varphi^{-}(\X,\x)$, where $\z$ is in $\y$. We know that either (1) $\theta(\x)\models \z = \x_i$, (2) $\theta(\x)\models \z < \x_i$ or (3) $\theta(\x)\models \z > \x_i$. Considering that (2) and (3) are not possible since $\A\models\theta(\d)$, then $\theta(\x)\models \z = \x_i$. If this is the case, then $\A\not\models\exists\x\,\varphi(\x,\R)$ for every possible assignation $\R$ to $\X$, which leads to a contradiction.
\item[(b)] By the construction of $\P$, we see that 
$$\A\not\models\bigvee_{i = 1}^{n}\neg Q_i(\d_i) \text{ and that } \A\not\models\bigvee_{Y\in\P} \exists \v\Big( Y(\v) \wedge \bigwedge_{i\in[1,n]: Y_i = Y} \v \neq \d_i\Big).$$ However, the only possible assignment to $\alpha_{\min}(\x_1,\dots,\x_n)$ is $(\d_1,\dots,\d_n)$. Then, $\A\not\models\exists\x\,\varphi^\prime(\x,\P)$.
\item[(c)] By contradiction, let $\P^\prime \neq \P$ be such that $\A\models\exists\x\,\varphi(\x,\P^\prime)$ and $\A\not\models\exists\x\,\varphi^\prime(\x,\P^\prime)$. If $\P^\prime$ is missing any tuple of $\P$ in some predicate $\P_i$, then $\A\models \neg Q_i(\d_i)$, so $\A\models\exists\x\,\varphi^\prime(\x,\P^\prime)$. Otherwise, if some predicate $\P^\prime_i$ in $\P^\prime$ has any tuple that $\P^_i$ does not have, then $$\A\models\bigvee_{Y\in\P} \exists \v\Big( Y(\v) \wedge \bigwedge_{i\in[1,n]: Y_i = Y} \v \neq \d_i\Big),$$ so $\A\models\exists\x\,\varphi^\prime(\x,\P^\prime)$. On both cases, we see follow to a contradiction.
\end{enumerate}
Second, assume that there is no assignation $\R$ to $\X$ such that $\A\models\exists\x\,\varphi(\x,\R)$. Let $\P$ be an arbitrary assignation to $\X$. Since $\A\not\models\exists\x\,\varphi(\x,\P)$, we see that $\A\not\models\exists\x\,(\varphi(\x,\P)\wedge\psi(\x,\P))$ for any formula $\psi(\x,\P)$. It follows that there is no assignation $\R$ to $\X$ such that $\A\models\exists\x\,\varphi^\prime(\x,\R)$.

We now follow to the general case, in which $\varphi(\x,\X,\z)$ is an arbirtary extended quantifier-free $\L$-formula. We introduce the formulas $\varphi_i(\x,\X),$ for $i\in\{1,\dots,k\}$ such that:
\begin{eqnarray*}
\varphi(\x,\X) &\equiv& \varphi_1(\x,\X) \vee \varphi_2(\x,\X) \vee \dots  \vee \varphi_k(\x,\X).
\end{eqnarray*}
We do this by finding an equivalent DNF formula to $\varphi(\x,\X)$ which considers FO-formulas over $\L$ as literals. Now we need to exclude that assignment to the next disjuncts, which we do by adding the same sub-formula. We also need to include the case where there are no possible assignments to $\x$ for the first disjunct, so we add this sub-formula to $\varphi^2(\x,\X)$:
\begin{multline*}
\exists\x\left(\varphi^1(\x,\X)\right) \vee \exists\vec{x_1}',\dots,\vec{x_n}'\Bigg(\alpha^2_{\min}(\vec{x_1}',\dots,\vec{x_n}') \wedge \\
\bigg(\bigvee_{i = 1}^{n}\neg Y^2_i(\vec{x_i}') \vee \bigvee_{P\in\P} \exists \x''\Big( P(\x'') \bigwedge_{\substack{i\in\{1,\dots,n\} \\ P_i = P}} \x'' \neq \vec{x_i}'\Big) \bigg) \Bigg).
\end{multline*}
The full disjunct is as follows,
\begin{multline*}
\varphi_j^\prime(\x,\X) = \bigwedge_{i=1}^n Y_i(\x_i) \wedge \varphi_{\overline{P}}(\vec{y}) \wedge \theta(\x) \wedge \varphi_{\S}(\x) \wedge \Bigg(\Big(\forall\x\left(\neg\alpha^1_{\min}(\x)\right)\wedge \dots \wedge \forall\x\left(\neg\alpha^{j-1}_{\min}(\x)\right)\Big) \to \\ 
\exists\u_1,\dots,\u_n\Bigg(\alpha^j_{\min}(\u_1,\dots,\u_n) \wedge
\bigg(\bigvee_{i = 1}^n\neg P(\u_i) \vee \bigvee_{P\in\P} \exists \v\Big( P(\v) \bigwedge_{\substack{i\in\{1,\dots,n\} \\ P_i = P}} \v \neq \u_i\Big) \bigg) \Bigg) \Bigg)
\end{multline*}
which, if all of the previous disjuncts have no possible assignments, eliminates the least assignment to $\P$ in that disjunct. Finally,
\begin{eqnarray*}
f^\prime(\A) &=& \mid \{ \langle\P\rangle \mid \A \models 
\exists \x \varphi^\prime_1(\x,\P) \vee
\exists \x \varphi^\prime_2(\x,\P) \vee
\dots \vee
\exists \x \varphi^\prime_k(\x,\P)
 \} \mid
\end{eqnarray*}
counts exactly one assignment to $\P$ less than $F(\A)$.

Therefore,

% CASE 3

\item The counting set is:
\begin{eqnarray*}
F(\A) &=& \mid \{ \langle\P,\z\rangle \mid \A \models \exists \x \ \varphi(\x, \P,\z) \} \mid
\end{eqnarray*}
Then, we going to isolate the minimal predicate $\P$ that holds the formula true and eliminate the lexicographically smallest $\z$ that satisfies it. We mix both previous strategies.
\end{enumerate}
\end{proof}

\end{document}