\documentclass[12pt]{article}
\usepackage[utf8]{inputenc}
\usepackage{amsmath}
\usepackage{amsthm} 
\usepackage{fullpage}
\usepackage{amsfonts}
\usepackage{amssymb}
\usepackage{bm}

\def\A{{\frak A}}

\newtheorem{theo}{Theorem}
\newtheorem{lemma}[theo]{Lemma}
\newtheorem{claim}[theo]{Claim}
\newtheorem{coro}[theo]{Corollary}

\begin{document}

\begin{flushleft}
{\footnotesize Pontificia Universidad Cat\'olica de Chile - Facultad de Ingenier\'ia\\
Departamento de Ciencias de la Computaci\'on\\
}
\end{flushleft}
\begin{center}
{ \LARGE \bf
  Some properties of $\#\Sigma_1^{+}$
}
\end{center}

The $\#\Sigma_1^{+}$ class is defined as an extension of $\#\Sigma_1$ which includes the possibility of adding arbitrarily quantified sub-formulas that don't include counted predicates.

Formally, let $\vec{P}$ be a sequence of predicates and $\vec{x},\vec{z}$ sequences of variables. Let $\varphi(\vec{S},\vec{x},\vec{z})$ be a formula where the variables in $\vec{P}$ are not in a quantified sub-formula. Note that sub-formulas that include any other predicate are not restricted at all. A counting problem $F \in \#\Sigma_1^{+}$ if given a structure $\A$
$$
F(\A) = \vert \{ <\!\!\vec{P},\vec{z}\!\!> :\A \models \exists \vec{x} \varphi(\vec{S},\vec{x},\vec{z}) \} \vert
$$ 
Some of its properties are:

\begin{theo}
The decision version of a counting problem in $\#\Sigma_1^{+}$ is in \text{P}.
\end{theo}
\begin{proof}
Let $F$ be a function in $\#\Sigma_1^{+}$. Then there is a formula $\varphi$ such that:
$$
F(\A) = \vert \{ <\!\!\vec{P},\vec{z}\!\!> :\A \models \exists \vec{x} \varphi(\vec{S},\vec{x},\vec{z}) \} \vert
$$ 
where $\A$ is a finite ordered structure and $\vec{z}$ is an $m$-tuple. For each $\vec{z} \in A^m$, every arbitrarily quantified sub-formula which does not include $\vec{P}$ can be evaluated in polynomial time. Let $\varphi'$ be the formula that results of changing every satisfied sub-formula for a tautology and every non-satisfied sub-formula by a contradiction. Note that $\varphi'$ is quantifier-free.

Then, we compute a similar counting function:
$$
F'(\A) = \vert \{ <\!\!\vec{P},(\vec{z},\vec{x})\!\!> :\A \models \varphi(\vec{S},\vec{x},\vec{z}) \} \vert
$$ 
which is in $\#\Sigma_0$. Note that for every $\A$, $F(\A) > 0$ iff $F'(\A) > 0$, so computing $F'$ is enough to solve the decision version of $F$. However, Saluja and Subrahmanyam showed that any counting problem in $\#\Sigma_0$ is computable in polynomial time \footnote[1]{(paper)}, therefore, the decision version of $F$ is in P.
\end{proof}

\begin{theo}
$\#\Sigma_1$ is closed under substraction $\Rightarrow$ P = NP.
\end{theo}
\begin{proof}
\#3DNF $\in \#\Sigma_1$. Let $F_{2^n}$ be a $\#\Sigma_1$ function that counts every possible truth assignment in a \#3DNF instance. Suppose that $F_{2^n}-F_{\#3DNF} \in \#\Sigma_1$. This function equals 0 only if the instanced formula is a tautology, so the decision version of it is co-NP-complete. However, as we showed previously (Theorem 1), it's also in P. Then, co-NP $\subseteq$ P.
\end{proof}

\begin{theo}
$\#\Sigma_1 \subsetneq \#\Sigma_1^{+}$
\end{theo}
\begin{proof}
We will show that the $\#\Sigma_1^{+}$ function defined by $\varphi(x_1) = (x_1 = x_1) \wedge \forall x_2 S(x_2)$ is not in $\#\Sigma_1$.
\end{proof}

\begin{theo}
$\#\Sigma_1$ is closed under substraction of 1 element.
\end{theo}
\begin{proof}
The counting set has three possible ways of counting. Counting only variables, only predicates, and predicates with variables. This separates the proof in three cases:
\begin{enumerate}
\item The counting set is:
$$
F(\A) = \vert \{ <\!\!\vec{z}\!\!> :\A \models \exists \vec{x} \varphi(\vec{x},\vec{z}) \} \vert
$$
our goal here is to eliminate the lexicographically smallest sequence of variables, which we can do easily. Let $F'$ be:
$$
F'(\A) = \vert \{ <\!\!\vec{z}\!\!> :\A \models \exists \vec{x}\left( \varphi(\vec{x},\vec{z}) \wedge \exists \vec{z'} (\varphi(\vec{x},\vec{z'}) \wedge \varphi_{\lneq}(\vec{z'},\vec{z} ) ) \right) \} \vert
$$
where $\varphi_{\lneq}(\vec{z'},\vec{z})$ is a formula that indicates that $\vec{z'}$ is lexicographically smaller than $\vec{z}$. Thus, $F'(\A)$ will count exactly one element less than $F(\A)$ if $F(\A)>0$.
\item The counting set is:
$$
F(\A) = \vert \{ <\!\!\vec{P}\!\!> :\A \models \exists \vec{x} \varphi(\vec{x},\vec{P}) \} \vert
$$
similarly, we would like to eliminate the lexicographically smallest predicate $P$ that holds the formula true with the exact number of tuples. First we treat sub-formulas without $\vec{P}$ as literals, and take a DNF formula that's equivalent to $\varphi$. Then
$$
F(\A) = \vert \{ <\!\!\vec{P}\!\!> :\A \models 
\exists \vec{x} \varphi_1(\vec{x},\vec{P}) \vee
\exists \vec{x} \varphi_2(\vec{x},\vec{P}) \vee
... \vee
\exists \vec{x} \varphi_k(\vec{x},\vec{P})
 \} \vert
$$
Note that
$$
\varphi_i(\vec{x},\vec{P}) = \vec{P_1}(\vec{x}) \wedge \neg \vec{P_0}(\vec{x}) \wedge \varphi'_i(\vec{x})
$$
where $\vec{P_0},\vec{P_1}$ are disjunct subsequences of $\vec{P}$ and $\varphi'_i$ is arbitrarily quantified. We can assume every predicate in $\vec{P}$ uses all variables in $\vec{x}$ and they are not permutated \footnote[1]{revisar esto}. Hence, if there is at least one assignation to $\vec{x}$ that satisfies $\varphi'_i(\vec{x})$, there is a predicate that has exactly one tuple that is lexicographically minimal. Thus, if
$$
\psi_i(\vec{x}) = \exists x' (x' \neq x \wedge P_1(x')) \vee \exists x' (\varphi'_i(x') \wedge \varphi_{\lneq}(x',x))
$$
then 
$$
\exists \vec{x} \left( \varphi_i(\vec{x},\vec{P}) \wedge \psi_i(\vec{x}) \right)
$$
will be satisfied by exactly one predicate less than $\exists \vec{x} \varphi_1(\vec{x},\vec{P})$. If $\varphi'_i(\vec{x})$ is unsatisfiable, we eliminate one predicate on the next $\varphi_i$. If this sub-formula is also unsatisfiable, we move to the next, and so on. Therefore:
\begin{multline*}
F(\A) = \vert \{ <\!\!\vec{P}\!\!> :\A \models 
\exists \vec{x} \left( \varphi_1(\vec{x},\vec{P}) \wedge \psi_1(\vec{x}) \right) \vee
\exists \vec{x} \left( \varphi_2(\vec{x},\vec{P}) \wedge \left( (\forall \vec{x'} \neg \varphi'_1(\vec{x'})) \rightarrow \psi_2(\vec{x}) \right)\right) \vee \\
... \vee
\exists \vec{x} \left( \varphi_k(\vec{x},\vec{P}) \wedge \left( (\forall \vec{x'} (\neg \varphi'_1(\vec{x'}) \wedge \neg \varphi'_2(\vec{x'}) \wedge ... \wedge \neg \varphi'_{k-1}(\vec{x'})) ) \rightarrow \psi_k(\vec{x}) \right)\right)
 \} \vert
\end{multline}
\end{enumerate}
\end{proof}

\end{document}
