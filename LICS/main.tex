\documentclass[conference]{IEEEtran}

\usepackage{cite}

\usepackage[utf8]{inputenc}	
\usepackage{amsmath}
\usepackage{amsfonts}
\usepackage{wrapfig}
%\usepackage{amssymb}
\usepackage{stmaryrd}
\usepackage{thmtools}
\usepackage{bbold}
\usepackage{multicol}
%\usepackage{MnSymbol}
\usepackage{tikz}
\usetikzlibrary{arrows,automata}
\usepackage{calc}
\usepackage[textwidth=2cm,textsize=small]{todonotes}

\usepackage{array}
\usepackage[caption=false,font=normalsize,labelfont=sf,textfont=sf]{subfig}
\usepackage{fixltx2e}
\usepackage{stfloats}
\usepackage{url}

\hyphenation{op-tical net-works semi-conduc-tor}

% commands
\newcommand{\marcelo}[1]{\todo[size=\scriptsize, color=blue!40]{#1}}
\newcommand{\cristian}[1]{\todo[size=\scriptsize,color=green!40]{#1}}

%logic
\newcommand{\fo}{{\rm FO}}
\newcommand{\so}{{\rm SO}}
\newcommand{\lfp}{{\rm LFP}}
\newcommand{\alfp}{{\bf alfp}}
\newcommand{\pth}{{\bf path}}
\newcommand{\tc}{{\rm TC}}
\newcommand{\dtc}{{\rm DTC}}
\newcommand{\pfp}{{\rm PFP}}
\newcommand{\eso}{\exists\so}
\newcommand{\first}{\operatorname{first}}
\newcommand{\last}{\operatorname{last}}
\newcommand{\succesor}{\operatorname{succ}}
\newcommand{\partition}{\operatorname{partition}}

\newcommand{\R}{\mathbf{R}}
\newcommand{\A}{\mathfrak{A}}
\newcommand{\all}{\text{\sc All}}
\newcommand{\allo}{\text{\sc AllOrd}}
\newcommand{\qso}{{\rm QSO}}
\newcommand{\qfo}{{\rm QFO}}
\newcommand{\eqso}{\Sigma\qso}
\newcommand{\fv}{\mathbf{FV}}
\newcommand{\sv}{\mathbf{SV}}
\newcommand{\arity}{{\rm arity}}
\newcommand{\shp}{\text{\sc \#P}}
\newcommand{\shl}{\text{\sc \#L}}
\newcommand{\spp}{\text{\sc span-P}}
\newcommand{\gp}{\text{\sc gap-P}}
\newcommand{\fp}{\text{\sc FP}}
\newcommand{\totp}{\text{\sc TotP}}
\newcommand{\fpspace}{\text{\sc FPSPACE}}
\newcommand{\nfpspace}{\natural\text{\sc PSPACE}}
\newcommand{\CC}{\mathscr{C}}
\newcommand{\KK}{\mathscr{K}}
\newcommand{\FF}{\mathscr{F}}
\newcommand{\GG}{\mathscr{G}}
\newcommand{\LL}{\mathscr{L}}
\newcommand{\QQ}{\mathscr{Q}}
\newcommand{\enc}{{\rm enc}}
\newcommand{\str}{\text{\sc Struct}}
\newcommand{\ostr}{\text{\sc OrdStruct}}
\newcommand{\res}[2]{#1|_{#2}}

%semiring
\newcommand{\nat}{\mathbb{N}}
\newcommand{\natinf}{\mathbb{N}_\infty}
\newcommand{\trop}{\mathbb{N}_{\min,+}}
\newcommand{\integ}{\mathbb{Z}}
\newcommand{\bln}{\mathbb{B}}
\newcommand{\pwset}[1]{2^{#1}}
\newcommand{\true}{\operatorname{true}}
\newcommand{\false}{\operatorname{false}}

\newcommand{\SR}{\bbS}
\newcommand{\add}{+}
\newcommand{\bigadd}{\sum}
\newcommand{\mult}{\cdot}
\newcommand{\bigmult}{\prod}
\newcommand{\adds}{\oplus}
\newcommand{\bigadds}{\bigoplus}
\newcommand{\mults}{\odot}
\newcommand{\bigmults}{\bigodot}
\newcommand{\zero}{\mathbb{0}}
\newcommand{\one}{\mathbb{1}}

%quantitative logic
\newcommand{\QL}{\operatorname{QL}}
\newcommand{\QMSO}{\operatorname{QMSO}}
\newcommand{\Op}{\operatorname{O}}
\newcommand{\sem}[1]{{\lsem{}{#1}\rsem}}
\newcommand{\pa}[1]{\Pi{#1}.\,}
\newcommand{\pas}{\Pi}
\newcommand{\paq}[1]{\Pi{#1}}
\newcommand{\sa}[1]{\Sigma{#1}.\,}
\newcommand{\sas}{\Sigma}
\newcommand{\saq}[1]{\Sigma{#1}}
\newcommand{\fpa}[1]{\overrightarrow{\prod}{#1}.\:}
\newcommand{\lmid}{\;\mid\;}

% equations and quotes skip
\abovedisplayskip=6pt 
\belowdisplayskip=6pt
\newenvironment{myquote}{\begin{quote}\vspace{-0.75mm}}{\end{quote}\vspace{-0.75mm}}

%tikz definition
\tikzset{
defaultstyle/.style={>=stealth,semithick, auto,font=\small,
initial text= {},
initial distance= {3.5mm},
accepting distance= {3.5mm}},
accepting/.style=accepting by arrow,
nstate/.style={circle, semithick,inner sep=1pt, minimum size=4mm}}





\begin{document}

\title{Descriptive complexity \\
	for counting complexity classes}


% author names and affiliations
% use a multiple column layout for up to three different
% affiliations
\author{\IEEEauthorblockN{Marcelo Arenas}
\IEEEauthorblockA{Department of Computer Science\\
Pontificia Universidad Cat\'olica de Chile \\
marenas@ing.puc.cl}
\and
\IEEEauthorblockN{Martin Mu\~noz}
\IEEEauthorblockA{Department of Computer Science\\
	Pontificia Universidad Cat\'olica de Chile \\
	mmunos@uc.cl}
\and
\IEEEauthorblockN{Cristian Riveros}
\IEEEauthorblockA{Department of Computer Science\\
	Pontificia Universidad Cat\'olica de Chile \\
	cristian.riveros@uc.cl}}

\maketitle


\begin{abstract}
The abstract goes here.
\end{abstract}

\IEEEpeerreviewmaketitle

\section{Introduction}
%!TEX root = main.tex

%\marcelo{Enfatizar el rol fundamental de la logica para obtener los resultados, por ejemplo para obtener cerrado bajo menos uno}
%
%\marcelo{Poner un comentario sobre funciones totales versus parciales, dado que estamos considerando clases con funciones totales}
%
%\marcelo{Enfatizar el rol fundamental de la logica para obtener los resultados, por ejemplo para obtener cerrado bajo menos uno}
%
%Strategy:
%\begin{enumerate}
%	\item Descriptive complexity and application.
%	
%	\item Counting complexity classes. 
%	
%	\item Our contribution in terms of logic.
%	
%	\item sharP and its structure. 
%	
%	\item Syntactic classes with good properties. 
%\end{enumerate}
%
%\cristian{Aqui empieza la intro.}

The goal of descriptive complexity is to measure the complexity of a problem in terms of the logical constructors needed to express it~\cite{immerman1999descriptive}. 
The starting point of this branch of complexity theory is Fagin's theorem, which states that $\np$ is equal to existential second-order logic. Since then, many more complexity classes have been characterized in terms of logics (see \cite{G07} for a survey) and descriptive complexity has found a variety of applications in different areas~\cite{immerman1999descriptive, L04}.
For instance, Fagin's theorem was the key ingredient to define the class {\sc MaxSNP}~\cite{PY91}, which was later shown to be a fundamental class in the study of hardness of approximation \cite{ALMSS98}. 
It is important to mention here that the definition of {\sc MaxSNP} would not have been possible without the machine-independent point of view of descriptive complexity, as pointed out in~\cite{PY91}.

Counting problems differ from decision problems in that the value of a function has to be computed.
More generally, a counting problem corresponds to compute a function $f$ from a set of instances (e.g. graphs, formulae, etc) to natural numbers.\footnote{This value is usually associated to counting the number of solution 
	%for a given instance 
	in a search problem, but here we consider a more general definition.} 
The study of counting problems has given rise to a rich theory of counting complexity classes \cite{HV95,F97,arora2009computational}. Some of these classes are natural counterparts of some classes of decision problems; for example, $\fp$ 
%(resp., $\fpspace$) 
is the class of all functions that can be computed in polynomial time, 
%(resp., polynomial space), 
the natural counterpart of $\ptime$.
% ($\pspace$ resp.). 
However, other function complexity classes have emerged from the need to understand the complexity of some computation problems for which little can be said if their decision counterparts are considered. This is the case of the class $\shp$, a counting complexity class introduced in \cite{Valiant79} to prove that natural problems like counting the number of satisfying assignments of a propositional formula or the number of perfect matchings of a bipartite graph~\cite{Valiant79} are difficult, namely, $\shp$-complete.
Starting from $\shp$,
many more natural 
%the zoo of 
counting complexity classes have been defined, such as 
%was open with other natural counting classes like 
$\shl$, $\spp$ and $\gp$~\cite{HV95,F97}.
%among others~\cite{HV95,F97}.

Although counting problems play a prominent role in computational complexity, descriptive complexity for this type of problems has not been systematically studied and it is not as developed as for the case of decision problems. Insightful characterizations of $\shp$ and some of its extensions have been provided \cite{SalujaST95,ComptonG96}. However, these characterizations do not define function problems in terms of a logic, but instead in terms of some counting problems associated to a logic like $\fo$. Thus, it is not clear how these characterizations can be used to provide a general descriptive complexity framework for counting complexity classes like $\fp$ and $\fpspace$ (the class of functions computable in polynomial space). 
%It should be mentioned that logical definability has also been studied for the case of optimization problems \cite{KT94} and computation over the real numbers \cite{GM95}. As for the previous cases, it is not clear how these approaches can be extended to provide logical characterizations of a variety of function complexity classes. 

In this paper, we propose to study the descriptive complexity of counting complexity classes in terms of Weighted Logics (WL)~\cite{DrosteG07}, a general logical framework that combines Boolean formulae (e.g. in $\fo$ or $\so$) with operations over a fix semirings (e.g. $\bbN$). 
Specifically, we propose to restrict WL over natural numbers, called Quantitative Second Order Logics (QSO), and study its expressive power for defining counting complexity classes over general structures. 
As a proof of concept, we show that natural syntactical fragments of $\qso$ captures counting complexity classes like $\shp$, $\spp$, $\fp$ and $\fpspace$.
Furthermore, by slightly extending the framework we can prove that $\qso$ can also capture classes like $\gp$ and $\optp$, showing the robustness of our approach.

The next step is to use the machine-independent point of view of $\qso$ to search for subclasses of $\shp$ with some fundamental properties.
%inside $\shp$. 
The question here is, what properties are desirable for a subclass of $\shp$?
First, it is desirable to have a class of counting problems whose associated decision versions are tractable, in the sense that one can decide in $\ptime$ whether the value of the function is greater than $0$. 
In fact, this requirement is crucial in order to have any chance of finding efficient approximation algorithms for a given function (see Section~\ref{sec:syntactic}).
Second, we expect that our class is closed under basic arithmetical operations like sum, multiplication and subtraction by one. 
This is a common topic for counting complexity classes; for example, it is known that $\shp$ is not closed under subtraction by one (under some complexity-theoretical assumption). 
Finally, we want a class with natural complete problems, which characterize all problems in it.
%from the complexity point of view.

In this paper, we give the first results towards defining subclasses of $\shp$ that are robust in terms of approximation, closure properties, and natural complete problems. 
Specifically, we introduce a syntactic hierarchy inside $\shp$, called $\eqso(\fo)$-hierarchy, and we show that it is closely related to the $\fo$-hierarchy introduced in~\cite{SalujaST95}. 
Looking inside the $\eqso(\fo)$-hierarchy, we propose the class $\eqso(\logex{1})$ and show that every function in it has a tractable associated decision version, and it is closed under sum, multiplication, and subtraction by one.
%minus one. 
Unfortunately, it is not clear whether this class admits 
%we cannot show a
a natural complete problem.
% for $\eqso(\logex{1})$.
Thus, 
%Despite of this 
we also introduce a Horn-style syntactic class, inspired by the approach in~\cite{G92}, that has tractable associated decision versions and a natural complete problem.

After studying the 
%internal 
structure of $\shp$, we move beyond $\qso$ by introducing new quantifiers. 
By adding functional variables in top of $\qso$, we introduce a quantitative least fixed point operator to the logic. 
Adding finite recursion to a numerical setting is subtle since functions over natural numbers can easily diverge without finding any fixed point. 
By using the support of the functions, we give a natural halting condition that generalizes the least fixed point operator of Boolean logics. 
Then, with a quantitative recursion at hand we show how to capture $\fp$ from a different perspective and, moreover, how to restrict recursion to capture lower complexity classes 
%below $\fp$ 
such as~$\shl$, the counting version of $\nlog$.
%(Nondeterministic Logarithmic-space).

\smallskip

\noindent{\bf Organisation.} The main terminology used in the paper is given in Section~\ref{sec:preliminaries}. Then the logical framework is introduced in Section~\ref{sec:logic}, and it is used to capture standard counting complexity classes in Section~\ref{sec:complexity}. The structure of $\shp$ is studied in Section~\ref{sec:syntactic}. Section~\ref{sec:beyond} is devoted to define recursion in $\qso$, and to show how to capture classes below $\fp$. 
Finally, we give some concluding remarks in Section~\ref{sec:conclusions}. 

\section{Preliminaries}
%!TEX root = main.tex

A $\mathbb{N}$ relational signature $\R$ is a finite set $\{R_1, \ldots, R_k\}$, where each $R_i$ ($1 \leq i \leq k$) is a relation name with an associate arity greater than 0, which is denoted by $\arity(R_i)$. A finite structure over $\R$ (or just finite $\R$-structure) is a tuple $\A = \langle A, R_1^\A, \ldots, R_k^\A \rangle$ such that $A$ is a finite set and $R_i^\A \subseteq A^{\arity(R_i)}$ for every $i \in \{1, \ldots, k\}$. A finite $\R$-structure $\A$ is said to be ordered if $<$ is a binary predicate name in $\R$ and $<^\A$ is a linear order on $A$. Let $\str[\R]$ be the class of all finite $\R$-structures and $\ostr[\R]$ be the class of all ordered finite $\R$-structures. 

From now on, assume given disjoint infinite sets $\fv$ and $\sv$ of first-order variables and second-order variables, respectively. Notice that every variable in $\sv$ has an associated arity, which is denoted by $\arity(X)$. Then given a relational signature $\R$, the set of second-order logic formulas ($\so$-formulas) over $\R$ is given by the following grammar:
\begin{eqnarray*}\ 
	\varphi &:=& R(\bar u) \ \mid\  
	X(\bar v)  \ \mid\ 
	\neg \varphi \ \mid\ 
	(\varphi \vee \varphi) \ \mid\ 
	\exists x \, \varphi \ \mid\ 
	\exists X \, \varphi
\end{eqnarray*}
where $R \in \R$, $\bar u$ is a tuple of (non-necessarily distinct) variables from $\fv$ whose length is $\arity(R)$, $X \in \sv$, $\bar v$ is a tuple of (non-necessarily distinct) variables from $\fv$ whose length is $\arity(X)$, and $x \in \fv$. 

To define the semantics of $\so$, we need to introduce some terminology. Given a relational signature $\R$ and a finite $\R$-structure $\A$ with domain $A$, a first-order assignment $v$ for $\A$ is a total function from $\fv$ to $A$, while a second-order assignment $V$ for $\A$ is a total function with domain $\sv$ that maps each $X \in \sv$ to a subset of $A^{\arity(X)}$. Moreover, given a first-order assignment $v$ for $\A$, $x \in \fv$ and $a \in A$, we denote by $v[a/x]$ a first-order assignment such that $v[a/x](x) = a$ and $v[a/x](y) = v(y)$ for every $y \in \fv$ distinct from $x$. Similarly, given a second-order assignment $V$ for $\A$, $X \in \sv$ and $B  \subseteq A^{\arity(X)}$, we denote by $V[B/X]$ a second-order assignment such that $V[B/X](X) = B$ and $V[B/X](Y) = V(Y)$ for every $Y \in \sv$ distinct from $X$. 

Assume that $\varphi$ is an $\so$-formula over a signature $\R$. Then given a finite $\R$-structure $\A$ with domain $A$, a first-order assignment $v$ for $\A$ and a second-order assignment $V$ for $\A$, we say that $(\A, v, V)$ satisfies $\varphi$, denoted by $(\A, v, V) \models \varphi$, if: (1) $\varphi$ is the formula $R(x_1, \ldots, x_\ell)$ and $(v(x_1), \ldots, v(x_\ell)) \in R^\A$; (2) $\varphi$ is the formula $X(x_1, \ldots, x_m)$ and $(v(x_1), \ldots, v(x_m)) \in V(X)$; (3) $\varphi$ is the formula $\neg \psi$ and it not the case that $(\A, v, V) \models \psi$; (4) $\varphi$ is the formula $(\varphi_1 \vee \varphi_2)$, and $(\A, v, V) \models \varphi_1$ or $(\A, v, V) \models \varphi_2$; (5) $\varphi$ is the formula $\exists x \, \psi$ and there exists $a \in A$ such that $(\A, v[a/x], V) \models \psi$; or (6) $\varphi$ is the formula $\exists X \, \psi$ and there exists $B \subseteq A^{\arity(X)}$ such that $(\A, v, V[B/X]) \models \psi$.  As usual, we consider the propositional operators $\wedge$, $\rightarrow$, and $\leftrightarrow$ that can be obtained from $\vee$ and $\neg$. 
%Moreover, we use the abbreviations $x \not \leq y$ and $x \notin X$ for the negation of the atoms $\leq$ and $\in$. 
%Finally, we consider standard abbreviations of formulas that can be defined in $\so$-logic (actually, $\fo$-logic) like $\first(x) := \fa{x} y \leq x$ and $\last(x) := \fa{x} x \leq y$ to denote the first and last element of the linear order $\leq$, respectively, and $\succesor(x,y) := x \leq y \wedge y \not\leq x \wedge \fa{z} ( z \leq x \vee y \leq z)$ to denote the successor relation.

\marcelo{Agregar las definiciones de LFP y PFP}
\subsection{Function complexity classes}
Assume that $\R = \{R_1, \ldots, R_k\}$ is a relational signature and $\A$ is a finite $\R$-structure with a domain $A$ containing $n$ elements, and assume that  $<$ is a linear order on $A$, say $a_1 < a_2 < \ldots < a_n$. For every $i \in \{1, \ldots, k\}$, define the encoding of $R_i^\A$, denoted by $\enc(R_i^\A)$, as the following binary string. Assume that $\ell = \arity(R_i)$ and consider an enumeration of the $\ell$-tuples over $A$ in the lexicographic order induced by $<$ (that is, $(a_1, \ldots, a_1, a_1)$, $(a_1, \ldots, a_1, a_2)$, $\ldots$, $(a_n, \ldots, a_n, a_{n-1})$, $(a_n, \ldots, a_n, a_n)$). Then let $\enc(R_i^\A)$ be a binary string of length $n^\ell$ such that the $i$-th bit of $\enc(R_i^\A)$ is 1 if the $i$-th tuple in the previous enumeration belongs to $R_i^\A$, and 0 otherwise. Moreover, define the encoding of $\A$, denoted by $\enc(\A)$, as the following binary string~\cite{L04}:
\begin{eqnarray*}
	\enc(\A) & = & 0^n \, 1 \, \enc(R_1^\A) \, \cdots \, \enc(R_k^\A).
\end{eqnarray*}
Given a set $\bbD$ and a relational signature $\R$, a function $f$ is said to be from $\R$ to $\bbD$ if $f$ is a total function from $\{\enc(\A) \mid \A \in \str[\R]\}$ to $\bbD$. Moreover, a set $\CC$ of functions over $\bbD$ is said to be a {\em function complexity class over $\bbD$} if every $f \in \CC$ is a function from some relational signature $\R$ to $\bbD$.

\marcelo{Ya no consideramos clases de estructuras, hay que revisar que esta seccion tenga una notacion consistente}

%If $\KK$ is a class of finite structures and $f$ is a function from a relational signature $\R$ to $\bbD$, we denote by $\res{f}{\KK}$ the restriction of $f$ to $\KK$, that is, a function such that: the domain of $\res{f}{\KK}$ is $\{ \enc(\A) \mid \A \in \str[\R] \cap \KK\}$, and 
%$\res{f}{\KK}(\enc(\A)) = f(\enc(\A))$ for every $\A \in \str[\R] \cap \KK$.

\section{A logic for counting functions}
%!TEX root = main.tex

We introduce here the logical framework that we use for studying counting complexity classes. 
This framework is based on the framework of Weighted Logics (WL)~\cite{DrosteG07}  that has been used in the context of weighted automata for studying functions from words (or trees) to semirings. 
We propose here to use the framework of WL over any relational structure and to restrict the semiring to natural numbers. 
The extension to any relational structure will allow us to study general counting complexity classes and the restriction to the natural numbers will simplify the notation in this context (see Section~\ref{sec:previous} for a more detailed discussion).

Given a relational signature $\R$, the set of Quantitative Second-Order logic formulae (or just $\qso$-formulae) over $\R$ is given by the following grammar:
%\[
%\begin{array}{rcl}
%\alpha & := & \varphi \ \mid \ s \ \mid \ (\alpha \add \alpha) \ \mid\ (\alpha \mult \alpha) \ \mid \ \\
%& &  \sa{x} \alpha \ \mid \pa{x} \alpha \ \mid \ \sa{X} \alpha \ \mid \ \pa{X} \alpha 
%\end{array}
%\]
\begin{align}
\alpha \ &:= \ \varphi \ \mid \ s \ \mid \ (\alpha \add \alpha) \ \mid\ (\alpha \mult \alpha) \ \mid \sa{x} \alpha \ \mid \ \pa{x} \alpha \ \mid \ \sa{X} \alpha \ \mid \ \pa{X} \alpha \label{syntax} 
\end{align}
where $\varphi$ is an $\so$-formula over $\R$, $s \in \bbN$, $x \in \fv$ and $X \in \sv$. Moreover, if $\R$ is not mentioned, then $\qso$ refers to the set of $\qso$ formulae over all possible relational signatures.
%\marcelo{En la gramatica de la logica deberiamos incluir la formula $\top$ que discutimos, la cual representa true y para la cual se tiene que $\sem{\top}(\A, v, V) = 1$. Les parece?}

The syntax of QSO formulae is divided in two levels. 
The first level is composed by $\so$-formulae over $\R$ (called Boolean formulae) and the second level is made by counting operators of addition and multiplication. 
For this reason, the quantifiers in $\so$ (e.g. $\exists x$ or $\exists X$) are called Boolean quantifiers and the quantifiers that make use of addition and multiplication (e.g. $\Sigma x$ or $\Pi X$) are called {\em quantitative quantifiers}.
Furthermore, $\Sigma x$ and $\Sigma X$ are called first- and second-order sum, whereas $\Pi x$ and $\Pi X$ are called first- and second-order product, respectively.
%This division between Boolean and quantitative level is essential for understanding the difference between the logic and the quantitative part. 
This separation between the Boolean and quantitative levels is essential for understanding the difference between the logic and the quantitative parts of the framework.
%\martin{creo que eso quer\'ia decir esta oraci\'on}
%\marcelo{OK}
Furthermore, this will later allow us to parametrize both levels of the logic in order to capture different counting complexity classes.

\begin{table}
	\addtolength{\jot}{0.5em}
	\begin{align*}
	\sem{\varphi}(\A, v, V) & = 
	\begin{cases}
	1 & \mbox{if } (\A, v, V) \models \varphi \\
	0 & \mbox{otherwise}
	\end{cases}\\
	\sem{s}(\A, v, V) & = s \\
	\sem{\alpha_1 \add \alpha_2}(\A, v, V) & = \sem{\alpha_1}(\A, v, V) + \sem{\alpha_2}(\A, v, V)\\
	\sem{\alpha_1 \mult \alpha_2}(\A, v, V) & = \sem{\alpha_1}(\A, v, V) \cdot \sem{\alpha_2}(\A, v, V)\\ 
	\sem{\sa{x} \alpha}(\A, v, V) & = \displaystyle \sum_{a \in A} \sem{\alpha}(\A,v[a/x],V)\\
	\sem{\pa{x} \alpha}(\A, v, V) & = \displaystyle \prod_{a \in A} \sem{\alpha}(\A,v[a/x],V)\\
	\sem{\sa{X} \alpha}(\A, v, V) & = \displaystyle \sum_{B \subseteq A^{\arity(X)}} \sem{\alpha}(\A, v, V[B/X])\\
	\sem{\pa{X} \alpha}(\A, v, V) & = \displaystyle \prod_{B \subseteq A^{\arity(X)}} \sem{\alpha}(\A, v, V[B/X])
	\end{align*}
	\caption{The semantics of QSO formulae.}
	\label{tab-semantics}
\end{table}
Let $\R$ be a signature, $\A$ an $\R$-structure with domain $A$, $v$ a first-order assignment for $\A$ and $V$ a second-order assignment for $\A$. Then the \emph{evaluation} of a $\qso$-formula $\alpha$ over $(\A, v, V)$ is defined as a function $\sem{\alpha}$ that on input $(\A, v, V)$ returns a number in $\bbN$. Formally, the function $\sem{\alpha}$ is recursively defined in Table~\ref{tab-semantics}.
A $\qso$-formula $\alpha$ is said to be a \emph{sentence} if it does not have any free variable, that is, every variable in $\alpha$ is under the scope of a usual quantifier or a quantitative quantifier. It is important to notice that if $\alpha$ is a $\qso$-sentence over a signature $\R$, then for every $\R$-structure $\A$, first-order assignments $v_1$, $v_2$ for $\A$ and second-order assignments $V_1$, $V_2$ for $\A$, it holds that $\sem{\alpha}(\A, v_1, V_1) = \sem{\alpha}(\A, v_2, V_2)$.
Thus, in such a case we use the term $\sem{\alpha}(\A)$ to denote $\sem{\alpha}(\A, v, V)$, for some arbitrary first-order assignment $v$ for $\A$ and some arbitrary second-order assignment $V$ for $\A$. 
\begin{exa}\label{ex:cliques}
Let $\bG = \{E(\cdot,\cdot),<\}$ be the vocabulary for graphs and $\fG$ be an ordered $\bG$-structure encoding a non-directed graph. 
Suppose that we want to count the number of triangles in $\fG$. Then this can be defined as follows:
\begin{align*}
\alpha_1 \ &:= \ \sa{x} \sa{y} \sa{z} ( E(x,y) \, \wedge \, E(y,z) \, \wedge \, E(z,x) \, \wedge x < y \, \wedge \, y < z )
\end{align*}
We encode a triangle in $\alpha_1$ as an increasing sequence of nodes $\{x, y, z\}$, in order to count each triangle once. Then the Boolean subformula  $E(x,y) \wedge E(y,z) \wedge E(z,x) \wedge
x < y \wedge y < z$ is checking the triangle property, by returning $1$ if $\{x, y, z\}$ forms a triangle in $\fG$ and $0$ otherwise.
Finally, the sum quantifiers in $\alpha_1$ aggregate all the values, counting the number of triangles in $\fG$.

Suppose now that we want to count the number of cliques in~$\fG$. We can define this function with the following formula:
\[
\alpha_2 \ := \ \sa{X} \clique(X),
\] 
where $\clique(X) := \fa{x} \fa{y} ((X(x) \wedge X(y) \wedge x \neq y)  \rightarrow E(x,y))$.
In the Boolean sub-formula of $\alpha_2$ we check whether $X$ is a clique, and with the sum quantifier we add one for each clique in $\fG$. 
But in contrast to $\alpha_1$, 
in $\alpha_2$ we need a second-order quantifier in the quantitative level.
This is according to the
complexity of evaluating each formula:
$\alpha_1$ defines an $\fp$-function while $\alpha_2$ defines a $\shp$-complete function. \qed
\end{exa}
\begin{exa}\label{exa-perm}
For an example that includes multiplication, let $\bM = \{M(\cdot,\cdot),<\}$ be a vocabulary for storing 0-1 matrices; in particular, a structure $\fM$ over $\bM$ encodes a 0-1 matrix $A$ as follows: if $A[i,j] = 1$, then $M(i,j)$ is true, otherwise $M(i,j)$ is false.
Suppose now that we want to compute the permanent of an $n$-by-$n$ 0-1 matrix $A$, that is:
\begin{align*}
\op{perm}(A) &= \sum_{\sigma \in S_n} \prod_{i=1}^n A[i, \sigma(i)],  
\end{align*}
where $S_n$ is the set of all permutations over $\{1, \ldots, n\}$.
The permanent is a fundamental function on matrices that has found many applications;
% in different areas,
%~\cite{permanent-applications},
in fact, showing that this function is hard to compute was one of the main motivations behind the definition of the class $\shp$~\cite{Valiant79}.
%and it was one of the first function that was shown to be difficult for counting~\cite{Valiant79} (i.e. $\shp$-complete). 

To define the permanent of a 0-1 matrix in $\qso$, assume that for a binary relation symbol $S$, $\op{permut}(S)$ is an $\fo$-formula that is true if, and only if, $S$ is a permutation, namely, a total bijective function (the definition of $\op{permut}(S)$ is straightforward).
%\martin{abreviar $\op{permutation}$ a $\op{permut}$ ahorra un par de lineas}
Then the following is a $\qso$-formula defining the permanent of a matrix:
\[
\alpha_3 := \sa{S} \op{permut}(S) \cdot \pa{x} (\ex{y} S(x,y) \wedge M(x,y)).
\]
Intuitively, the subformula $\beta(S) := \pa{x} (\ex{y} S(x,y) \wedge M(x,y))$ calculates the value \linebreak $\prod_{i=1}^n A[i, \sigma(i)]$ whenever $S$ encodes a permutation $\sigma$.
Moreover, the subformula $\op{permut}(S) \cdot \beta(S)$ returns $\beta(S)$ when $S$ is a permutation, and returns $0$ otherwise (i.e. $\op{permut}(S)$ behaves like a filter). 
Finally, the second order sum aggregates these values iterating over all binary relations and calculating the permanent of the matrix.
We would like to finish with this example by highlighting the similarity of $\alpha_3$ to the permanent formula. 
Indeed, an advantage of $\qso$-formulae is that the first- and second-order quantifiers in the quantitative level naturally reflect the operations used to define mathematical formulae. \qed
\end{exa}

We consider several fragments of $\qso$, which are obtained by restricting the syntax of the Boolean formulae or the use of the quantitative quantifiers, and we consider some extensions that are obtained by adding recursive operators to $\qso$.
In this regard, we denote by $\qfo$ the fragment of $\qso$ where second-order sum and product are not allowed. 
For instance, for the $\qso$-formulae defined in Example \ref{ex:cliques}, we have that $\alpha_1$ is in $\qfo$ and $\alpha_2$ is not.
Another interesting fragment of $\qso$ consists of the $\qso$-formulae where only sum operators and sum quantifiers are allowed. 
Formally, we denote by $\eqso$ the fragment of $\qso$ where first- and second-order products (i.e. $\pa{x}$ and $\pa{X}$) are not allowed.
For example, $\alpha_1$ and $\alpha_2$ in Example \ref{ex:cliques} are formulae of $\eqso$, while $\alpha_3$ in Example \ref{exa-perm} is not. 
We also consider fragments of $\qso$ by further restricting the Boolean part of the logic.
If $\LL$ is a fragment of $\so$, then we define the quantitative logic $\qso(\LL)$ to be the fragment of $\qso$ obtained by restricting $\varphi$ in \eqref{syntax} to be a formula in $\LL$. Moreover, we also restrict other fragments of $\qso$ by using the same idea. 
For example, we define $\qfo(\fo)$ to be the fragment of $\qfo$ obtained by restricting $\varphi$ in \eqref{syntax} to be an $\fo$-formula, and likewise for $\eqso(\fo)$.

In the following section, we use different fragments or extensions of $\qso$ to capture counting complexity classes. But before doing this, we show the connection of $\qso$ to previous frameworks for defining functions over relational structures.

\subsection{Previous frameworks for quantitative functions} \label{sec:previous}

In this section, we discuss some previous frameworks proposed in the literature and how they differ from our approach.
We start by discussing the connection between $\qso$ and weighted logics (WL)~\cite{DrosteG07}. 
As it was previously discussed, $\qso$ is a fragment of WL.
The main difference is that we restrict the semiring used in WL to natural numbers in order to study counting complexity classes.
Another difference between WL and our approach is that, to the best of our knowledge, this is the first paper to study weighted logics over general relational signatures, in order  to do descriptive complexity for counting complexity classes. 
Previous works on WL usually restrict the signature of the logic to strings, trees, and other specific structures (see \cite{droste2009handbook} for more examples), and they did not study the logic over general structures. 
Furthermore, in this paper we propose further extensions for $\qso$ (see Section~\ref{sec:beyond}) which differ from previous approaches in WL.

Another approach that resembles $\qso$ are logics with counting~\cite{IL90,E97,GG98,L04}, which include operators that extend $\fo$ with quantifiers that allow to count in how many ways a formula  is satisfied (the result of this counting is a value of a second sort, in this case the  natural numbers). 
In contrast to our approach, counting operators are usually used for checking Boolean properties over structures and not for producing values (i.e. they do not define a function).
In particular, we are not aware of any paper that uses this approach for capturing counting complexity classes.

Finally, the work in~\cite{SalujaST95,ComptonG96,0001HKV16} is of particular interest for our research. 
In~\cite{SalujaST95}, it was proposed to define a function over a structure by using free variables in an SO-formula; in particular, the function is defined by the number of instantiations of the free variables that are satisfied by the structure.
Formally, Saluja et. al \cite{SalujaST95} define a family of counting classes $\#\LL$ for a fragment $\LL$ of $\fo$. For a formula $\varphi(\bar{x},\bar{X})$ over $\R$, the function $f_{\varphi(\bar x, \bar X)}$ is defined as
$
f_{\varphi(\bar x, \bar X)}(\A) = \vert \{(\bar{a},\bar{A}) \mid \A\models\varphi(\bar{a},\bar{A})\}\vert
$
for every $\A\in\ostr[\R]$. Then a function $g\colon \ostr[\R]\to\nat$ is in $\#\LL$ if there exists a formula $\varphi(\bar{x},\bar{X})$ in $\LL$ such that $g = f_{\varphi(\bar x, \bar X)}$.
In~\cite{SalujaST95}, they proved several results about capturing counting complexity classes which are relevant for our work. We discuss and use these results in Sections~\ref{sec:complexity} and~\ref{sec:syntactic}.
Notice that for every formula $\varphi(\bar{x},\bar{X})$, it holds that $f_{\varphi(\bar{x},\bar{X})}$ is the same function as $\sem{\sa{\bar{X}} \sa{\bar{x}} \varphi(\bar{x},\bar{X})}$, that is, the approach in \cite{SalujaST95} can be seen as a syntactical restriction of our approach based on $\qso$. 
Thus, the advantage of our approach relies on the flexibility to define functions by alternating sum with product operators and, moreover, by introducing new quantitative operators (see Section~\ref{sec:beyond}).
Furthermore, we show in the next section how to capture some classes that cannot be captured by following the approach in~\cite{SalujaST95}.


\section{Counting Under QSO}
%!TEX root = main.tex

In this section, we show that by syntactically restricting $\qso$ one can capture different counting complexity classes. 
In other words, by using $\qso$ we can extend the theory of descriptive complexity~\cite{immerman1999descriptive} from decision problems to counting problems. 
For this, we first formalize the notion of \emph{capturing} a complexity class of functions.
%, and then show how to capture classes like $\shp$, $\fp$, and $\fpspace$.

Fix a signature $\R = \{R_1, \ldots, R_k\}$ and assume that $\A$ is an ordered $\R$-structure with a domain $A = \{a_1, \ldots, a_n\}$, $R_k =\, <$, and $a_1 <^{\A} a_2 <^{\A} \ldots <^{\A} a_n$. For every $i \in \{1, \ldots, k-1\}$, define the encoding of $R_i^\A$, denoted by $\enc(R_i^\A)$, as the following binary string. Assume that $\ell = \arity(R_i)$ and consider an enumeration of the $\ell$-tuples over $A$ in the lexicographic order induced by $<$. 
%(that is, $(a_1, \ldots, a_1, a_1)$, $(a_1, \ldots, a_1, a_2)$, $\ldots$, $(a_n, \ldots, a_n, a_{n-1})$, $(a_n, \ldots, a_n, a_n)$). 
Then let $\enc(R_i^\A)$ be a binary string of length $n^\ell$ such that the $i$-th bit of $\enc(R_i^\A)$ is 1 if the $i$-th tuple in the previous enumeration belongs to $R_i^\A$, and 0 otherwise. Moreover, define the encoding of $\A$, denoted by $\enc(\A)$, as the string~\cite{L04}:
%following binary string~\cite{L04}:
%\begin{eqnarray*}
	%\enc(\A) & = & 0^n \, 1 \, \enc(R_1^\A) \, \cdots \, \enc(R_k^\A).
%\end{eqnarray*}
$$
0^n \, 1 \, \enc(R_1^\A) \, \cdots \, \enc(R_{k-1}^\A).
$$
\martin{modifique la definicion para que no sea necesario codificar el $<$.}
\marcelo{OK}
%We define the class of all $\R$-functions, denoted by $\Func(\R)$, as the class of all functions $f: \ostr \rightarrow \bbN$.
%Given a function complexity class $\CC$ (i.e. $f: \Sigma^* \rightarrow \bbN$ for every $f \in \CC$), we say that a function $f \in \Func(\R)$ can be computed in $\CC$ if there exists $g \in \CC$ such that $f(\A) = g(\enc(\A))$ for every $\A \in \ostr$. 
%Note that the function $g$ outputs $f$ for encodings of structures and can behave arbitrarily otherwise.
We can now formalize the notion of capturing a counting complexity class.
\begin{defi} \label{def:cap}
	Let $\FF$ be a fragment of $\qso$ and $\CC$ a counting complexity class. Then {\em  $\FF$ captures $\CC$ over ordered $\R$-structures} if the  following conditions hold:
	\begin{enumerate}
		\item for every $\alpha \in \FF$, there exists $f \in \CC$ such that $\sem{\alpha}(\A) = f(\enc(\A))$ for every $\A \in \ostr[\R]$. 
		
		\item for every $f \in \CC$, there exists $\alpha \in \FF$ such that   $f(\enc(\A)) = \sem{\alpha}(\A)$ for every $\A \in \ostr[\R]$.
	\end{enumerate} 
	Moreover, {\em $\FF$ captures $\CC$ over ordered structures} if $\FF$ captures~$\CC$ over ordered $\R$-structures for every signature~$\R$. \qed
\end{defi}
%For the sake of simplification, we denote the first condition by $\FF \subseteq \CC$ and the second condition by $\CC \subseteq \FF$.
In Definition~\ref{def:cap}, function $f \in \CC$ and formula $\alpha \in \FF$ must coincide in all the strings that encode ordered $\R$-structures. Notice that this restriction is natural as we want to capture %Since we want to capture 
$\CC$ over a fixed set of structures (e.g. graphs, matrices).
%, it is natural to just consider strings that encodes $\R$-structures. 
Moreover, this restriction is fairly standard in descriptive complexity \cite{immerman1999descriptive,L04}, and it has also been used in the previous work on capturing complexity classes of functions \cite{SalujaST95,ComptonG96}.
%all notions for capturing complexity classes restrict $f \in \CC$ similarly. 

What counting complexity classes can be captured with fragments of $\qso$?
For answering this question, it is reasonable to start with $\shp$, a well-known and widely-studied counting complexity class~\cite{arora2009computational}. 
Since $\shp$ has a strong similarity with $\np$, one could expect a ``Fagin-like'' Theorem~\cite{F75} for this class. 
Actually, in~\cite{SalujaST95} it was shown that the class $\sfo$ captures $\shp$.
In our setting, the class $\sfo$ is contained in $\eqso(\fo)$, which also captures $\shp$ as expected.
 
\begin{prop} \label{prop:capture-shP}
	$\eqso(\fo)$ captures $\shp$ over ordered structures.
\end{prop}
\proof
We briefly explain how the two conditions of Definition~\ref{def:cap} are satisfied. First, for condition (2) Saluja et al. proved that $\shp = \sfo$\cite{SalujaST95}. Hence, given that every function in $\sfo$ can be trivially defined as a formula in $\eqso(\fo)$ (see Section~\ref{sec:previous}), condition~(2) holds.
For condition (1), let $\alpha\in\eqso(\fo)$ over some signature $\R$. Given an $\fo$ formula $\varphi$, checking whether $\A\models\varphi$ can be done in deterministic polynomial time on the size of $\A$ and any constant function $s$ can be trivially simulated in $\shp$. These facts, together with the closures under exponential sum and polynomial product of $\shp$\cite{F97}, suffice to show that the function represented by $\alpha$ is in $\shp$.
%We construct recursively a $\shp$-machine $M_{\alpha}$ for each $\eqso(\fo)$ formula $\alpha$ over a signature $\R$. This machine, on input $(\A,v,V)$ accepts in $\sem{\alpha}(\A,v,V)$ of its non-deterministic paths for each $(\A,v,V) \in \ostr[\R]^*$. Suppose $\A$ has domain $A$. If $\alpha$ is a $\fo$-formula $\varphi$, then the machine checks if $(\A,v,V)\models\varphi$ deterministically in polynomial time, and accepts if and only if it holds true. If $\alpha$ is a constant $s$, it produces $s$ branches and accepts in all of them. If $\alpha = (\beta \add \gamma)$, then it chooses between 0 or 1, if it is 0 (1), it simulates $M_{\beta}$ ($M_{\gamma}$) on input $(\A,v,V)$. 
%If $\alpha = \sa{x}\beta$, it chooses $a\in A$ non-deterministically and simulates $M_{\beta}$ on input $(\A,v[a/x],V)$.
%If $\alpha = \sa{X}\beta$, it chooses $B\in A^{arity(X)}$ and simulates $M_{\beta}$ on input $(\A,v,V[B/X])$. This covers all possible cases for $\alpha$. Let $\alpha$ be a formula in $\eqso(\fo)$ over a signature $\R$ and let $f$ be a function over $\R$ such that $f(\enc(\A))$ is equal to the accepting paths of $M_{\alpha}$ on input $(\A,v,V)$ for some $(\A,v,V) \in \ostr[\R]^*$. We have that $f$ is a $\shp$-function over $\R$ and $f(\enc(\A)) = \sem{\alpha}(\A)$ for every $\A\in\ostr[\R]$.
 
\qed

By following the same approach as~\cite{SalujaST95}, Compton and Gr\"adel~\cite{ComptonG96} show that $\seso$ captures $\spp$, where $\eso$ is the existential fragment of $\so$. As one could expect, if we parametrize $\eqso$ with $\eso$, we can also capture~$\spp$.
\begin{prop} \label{prop:capture-spanP}
	$\eqso(\eso)$ captures $\spp$ over ordered structures.
\end{prop}
\proof
Similar than the previous proof, we construct recursively a $\spp$ machine $M_{\alpha}$ for each $\eqso(\eso)$ formula $\alpha$ over a signature $\R$. This machine, on input $(\A,v,V)$, non-deterministically produces $\sem{\alpha}(\A,v,V)$ distinct accepting outputs for each $(\A,v,V) \in \ostr[\R]^*$. Suppose $\A$ has domain $A$. 
If $\alpha$ is a $\eso$-formula $\varphi$ it checks if $(\A,v,V)\models\varphi$ non-deterministically in polynomial time \cite{F75}, and accepts if and only if the condition holds true. 
If $\alpha$ is a constant $s$, then the machine produces $s$ branches and accepts in all of them. 
If $\alpha = (\beta \add \gamma)$, then it chooses between 0 or 1, if it is 0 (1), it simulates $M_{\beta}$ ($M_{\gamma}$) on input $(\A,v,V)$.  
If $\alpha = \sa{x}\beta$, it chooses $a\in A$ non-deterministically and simulates $M_{\beta}$ on input $(\A,v[a/x],V)$. 
If $\alpha = \sa{X} \beta$, it chooses $B\in A^{\arity(X)}$ and simulates $M_{\beta}$ on input $(\A,v,V[B/X])$. 
This covers all possible cases for $\alpha$.
Additionally, the machine produces a different output on each path. This can be done by printing the trace of all the non-deterministic choices.
However, when the machine starts checking whether $(\A,v,V)\models\varphi$ for some $\eso$ formula $\varphi$, it stops printing in the output tape. This way the machine produces exactly one output from that point onwards.
Let $\alpha$ be a formula in $\eqso(\eso)$ over a signature $\R$ and let $f$ be a function over $\R$ such that $f(\enc(\A))$ is equal to the number of accepting outputs of $M_{\alpha}$ on input $(\A,v,V)$ for some $(\A,v,V) \in \ostr[\R]^*$. 
We have that $f$ is a $\spp$ function over $\R$ and that $f(\enc(\A)) = \sem{\alpha}(\A)$ for every $\A\in\ostr[\R]$.

For the other direction, Compton et al.~\cite{ComptonG96} proved that $\spp = \#\eso$. Since a function in $\#\eso$ can also be defined in $\eqso(\eso)$, then $\eqso(\eso)$ captures $\spp$ over ordered structures.
\qed
Can we capture $\fp$ by using $\# \LL$ for some fragment $\LL$ of $\so$? A first attempt could be based on the use of a fragment $\LL$ of $\so$ that captures either $\ptime$ or $\nlog$~\cite{G92}. Such an approach fails as $\# \LL$ can encode $\shp$-complete problems in both cases; in the first case, one can encode the problem of counting the number of satisfying assignments of a Horn  propositional formula, while in the second case one can encode the problem of counting the number of satisfying assignments of a 2-CNF propositional formula. A second attempt could then be based on considering a fragment $\LL$ of $\fo$. 
But even if we consider the existential fragment $\Sigma_1$ of $\fo$ the approach fails, as $\# \Sigma_1$ can encode $\shp$-complete problems like counting the number of satisfying assignments of a 3-DNF propositional formula\cite{SalujaST95}. One last attempt could be based on disallowing the use of second-order free variables in $\sfo$. But in this case one 
cannot capture exponential functions definable in $\fp$ such as~$2^n$.
Thus, it is not clear how to capture $\fp$ 
by following the approach proposed in~\cite{SalujaST95}. 
On the other hand, if we consider our framework and move out from $\eqso$, we have other alternatives for counting like first- and second-order products. In fact, the combination of $\qfo$ with $\lfp$ is exactly what we need to capture $\fp$.
\begin{thm} \label{theo:capture-fp}
	$\qfo(\lfp)$ captures $\fp$ over ordered structures.
\end{thm}
\proof
For the first condition, let $\alpha\in\qfo(\lfp)$ over some signature $\R$, defined by the grammar in \ref{syntax}. Notice that for each $\lfp$ formula $\varphi$,  checking whether $\A\models\varphi$ can be done in deterministic polynomial time on the size of $\A$ [cite here]. Also, the constant function $s$ can be trivially simulated in $\fp$. These facts, together with closure properties of $\fp$ of polynomial sum and product [cite here?] suffice to show that the function represented by $\alpha$ is in $\fp$.
	
For the second condition, let $f\in \fp$ and consider some signature $\R$.
Let $\ell\in\nat$ be such that for each $\A\in\ostr[\R]$, $\lceil\log_2 f(\enc(\A)) \rceil \leq \size{\A}^\ell$ (i.e. $\size{\A}^\ell$ is an upper bound for the output size).
Let $\bar{x} = (x_1,\ldots,x_{\ell})$.
Define a language
\[
L = \{(\A,a_1,\ldots,a_{\ell})\mid a_1,\ldots,a_{\ell}\in A \text{ and the } (a_1,\ldots,a_{\ell})\text{-th bit of }f(\enc(\A))\text{ is 1}\}.
\]

%Consider a procedure that receives $\enc(\A)$ and an assignment $\bar{a}$ to $\bar{x}$. Let $m$ be the position of $\bar{a}$ in the lexicographic order of the tuples in $A^{\ell}$. The procedure then computes the $m$-th bit of $f(\enc(\A))$, from least to most significant. 
Since this language is in $\ptime$, by \cite{I86} there exists a formula $\Phi(\bar{x})$ in $\lfp$ such that $\A\models\Phi(\bar{a})$ if and only if $(\A,\bar{a})\in L$. 
Then we use
$$
\alpha = \sa{\bar{x}} \Phi(\bar{x})\cdot\varphi(\bar{x}),
$$
where $\varphi(\bar{x}) := \pa{\bar{y}}(\bar{y} < \bar{x} \mapsto 2)$. This formula takes the value $2^m$ if there exists $m$ tuples in $A^{\ell}$ that are smaller than $\bar{x}$. Adding these values for each $\bar{a}\in A^{\ell}$ gives exactly $f(\enc(\A))$. 
In other words, $\Phi(\bar{x})$ simulates the behavior of the $\fp$-machine and the formula $\alpha$ reconstruct the binary output.
Then, $\alpha$ is in $\qfo(\lfp)$ over $\R$ and $\sem{\alpha}(\A) = f(\enc(\A))$.
\qed
%To prove this theorem, 
%capture $\fp$ 
%one first shows that every formula in $\qfo(\lfp)$ can be evaluated in polynomial time. 
%Indeed, $\lfp$ is a polynomial-time logic~\cite{I86,vardi1982complexity}, and the sum and product quantifiers can also be computed in polynomial time. 
%For the other direction, one has to use $\qfo(\lfp)$ to simulate the run of a polynomial time TM $M$ computing a function, in particular using the quantitative quantifiers to reconstruct the natural number returned by $M$ in the output tape. 
%It is important to notice that the alternation between sum and product quantifiers is crucial for this reconstructions and, thus, crucial for capturing $\fp$.

At this point it is natural to ask whether one can extend the previous idea to capture $\fpspace$~\cite{Ladner89}, the class of functions computable in polynomial space. 
Of course, for capturing this class one needs a logical core powerful enough, like $\pfp$, for simulating the run of a polynomial-space TM.
Moreover, 
one also needs more powerful quantitative quantifiers as functions like $2^{2^n}$ can be computed in polynomial space,
so $\eqso$ is not enough for the quantitative layer of a logic for $\fpspace$.
In fact, by considering second-order product we obtain the fragment $\qso(\pfp)$ that captures $\fpspace$. 
\begin{thm} \label{theo:capture-fpspace}
	$\qso(\pfp)$ captures $\fpspace$ over ordered structures.
\end{thm}
\proof
%!TEX root = main.tex

For the first condition of Definition~\ref{def:cap}, notice that each $\pfp$ formula can be evaluated in deterministic polynomial space, the constant function $s$ can be trivially simulated in $\fpspace$, and $\fpspace$ is closed under exponential sum and multiplication. This suffices to show that the condition holds.
For the second condition, the proof is similar to the proof of Theorem~\ref{theo:capture-fp}. Let $f\in \fpspace$ defined over some $\R$ and $\ell\in\nat$ such that $\log_2\left( f(\enc(\A)) \right) \leq 2^{{|\A|}^\ell}$ for every $\A\in\ostr[\R]$  (i.e. $2^{{|\A|}^\ell}$ is an upper bound for the size of the output). Let $X$ be a second-order variable of arity $\ell$. Consider the linear order induced by $<$ over predicates of arity $\ell$ which can be defined by the following formula:
$$
\varphi_{<}(X,Y) = \ex{\bar{u}}\big[\neg X(\bar{u})\wedge Y(\bar{u})\wedge \fa{\bar{v}}\big(
\bar{u}<\bar{v}\to(X(\bar{u})\iff Y(\bar{v}))\big)\big].
$$
Namely, we use predicates to encode a number that will have most $2^{{|\A|}^\ell}$ bits. We define this encoding through the function $\tau\colon 2^{A^\ell}\to\nat$, such that $\tau(B)$ is equal to the number of predicates in $2^{A^\ell}$ that are smaller than $B$ with respect to the induced order. For example, we have that $\tau(\emptyset) = 0$ and $\tau(A^{\ell}) = 2^{{|\A|}^\ell}-1$. Furthermore, we can use a relation~$X$ to index a position in the binary output of $f(\enc(\A))$ as follows.
%Consider a polynomial space machine over the $\R$ that receives as input an $\R$-structure $\A$ and a number $p$ encoded by a relation $X$. Then the machine accepts if, and only if, the $p$-th bit of $f(\enc(\A))$ is $1$. 
Define the language:
\[
L = \{(\A,B)\mid B \subseteq A^{\ell}\text{ and the $\tau(B)$-th bit of $f(\enc(\A))$ is 1}\}.
\]
Since $L$ is in $\pspace$, it can be specified in $\pfp$ \cite{AbiteboulV89} by a formula $\Phi(X)$ such that $\A\models\Phi(B)$ if and only if $(\A,B)\in L$. Then, similarly as for the previous proof we define:
$$
\alpha := \sa{X} \Phi(X)\mult  \pa{Y}(\varphi_{<}(Y,X)\mapsto 2).
$$ 
where $\pa{Y}(\varphi_{<}(Y,X)\mapsto 2)$ takes the value $2^{\tau(X)}$ and $\alpha$ reconstructs the output of $f(\enc(\A))$. Using an argument analogous to the previous proof, we conclude that $\alpha\in\qso(\pfp)$ and $\sem{\alpha}(\A) = f(\enc(\A))$.
%\martin{Reescrib\'i varias l\'ineas de esta demostraci\'on}

\qed
%The proof of the previous theorem follows the same line as for the logical characterization of $\fp$: one shows that each function in $\qso(\pfp)$ can be computed in $\fpspace$ and, conversely, the output of a polynomial-space TM can be reconstructed by using $\pfp$ and quantitative quantifiers.

From the proof of the previous theorem a natural question follows: what happens if we use first-order quantitative quantifiers and $\pfp$?
In~\cite{Ladner89}, Ladner also introduced the class $\nfpspace$ of all functions computed by polynomial-space TMs 
with output length bounded by a polynomial.
Interestingly, if we restrict to FO-quantitative quantifiers we can also capture this class.
\begin{cor} \label{cor:capture-fpspace-poly}
	$\qfo(\pfp)$ captures $\nfpspace$ over ordered structures.
\end{cor}
\proof
In this proof, both conditions are analogous to Theorem~\ref{theo:capture-fp} and~\ref{theo:capture-fpspace}. For the first condition, each $\pfp$ formula $\varphi$ can be evaluated in $\pspace$ and the class is closed under first sum and product. For the second condition, we use the same language $L$ defined in the proof of Theorem~\ref{theo:capture-fp}, which in this case is in $\pspace$. The same construction of $\alpha$, which in turn is in $\qfo(\pfp)$, is used to show that the condition holds.
\qed

The results of this section validate $\qso$ as an appropriate logical framework for extending the theory of descriptive complexity to counting complexity classes. In the following sections, we provide more arguments for this claim, by considering some fragments of $\eqso$ and, moreover, by showing how to go beyond $\eqso$ to capture other classes.


\section{Syntactic classes of QSO}
%!TEX root = main.tex

The class $\shp$ was introduced in \cite{Valiant79} to prove that computing the permanent of a matrix, as defined in Example \ref{exa-perm}, is a difficult problem. More specifically, it was shown in  \cite{Valiant79}  that this problem is $\shp$-complete. As a consequence of this result the problem of computing the number of perfect matchings in a bipartite graph was also shown to be $\shp$-complete. Since then  many counting problems have been proved to be $\shp$-complete \cite{V79b,PB83,P86,L86,BW91,HMRS98,BW05,DS12, PS13,PS14}. Among them, problems having easy decision counterparts play a fundamental role, as a counting problem with a hard decision version is expected to be hard. A first prominent example of such problems is counting the number of perfect matching in a bipartite graph, as it is well-known that the problem of verifying whether there exists a perfect matching in a bipartite graph can be solved in polynomial time. Other prominent examples of such problems include counting the number of: satisfying assignments of a 2-CNF propositional formula \cite{V79b}, satisfying assignments of a DNF propositional formula \cite{DHK05}, simple paths from a source node to a target node in a directed graph \cite{V79b}, extensions of a partial order to a linear order \cite{BW91} and Eulerian cycles in an undirected graph \cite{BW05}. 

Counting problems with easy decision versions play a fundamental role in the search of efficient approximations algorithms for functions in $\shp$. A fully-polynomial randomized approximation scheme (FPRAS) for a function $f : \Sigma^* \to \bbN$ is a randomized algorithm ${\cal A} : \Sigma^* \times (0,1) \to \bbN$ such that: (1) for every string $x \in \Sigma^*$ and real value $\varepsilon \in (0,1)$, the probability that $|f(x) - {\cal A}(x,\varepsilon)| \leq \varepsilon \cdot f(x)$ is at least $\frac{3}{4}$, and (2) the running time of ${\cal A}$ is polynomial in the size of $x$ and $1/\varepsilon$ \cite{KL83}. Notably, there exist $\shp$-complete functions that can be efficiently approximated as they admit FPRAS; for instance, there exist FPRAS for the problems of counting the number of satisfying assignments of a DNF propositional formula \cite{KL83} and the number of perfect matchings of a bipartite graph \cite{JSV04}. A key observation here is that if a $\shp$-complete function admits an FPRAS, then its associated decision problem is in the complexity class $\bpp$ (Bounded-Error Probabilistic Polynomial-Time). Hence, under standard complexity-theoretical assumptions we cannot hope for an FPRAS for a function in $\shp$ whose decision counterpart is $\np$-complete, and we have to concentrate on the class of counting problems with easy decision versions (in $\bpp$ or in a lower complexity class such as $\ptime$). 

The importance of the class counting problems with easy decision counterparts has motivated the search of robust definitions of classes of functions in $\shp$ with easy decision versions \cite{PagourtzisZ06}. In this section, we use the framework developed in this paper to address this problem. More specifically, we introduce in Section \ref{sec-hier-shp} a hierarchy of 


consider several fragments of $\eqso(\LL)$ where $\LL$ is a boolean logic contained in $\fo$, and we study 

% It should be noted that such class can be directly defined as the set functions f ? #P such that Lf ? P, which is denoted as #Pe in [50, 51]. However, such a definition does not lead to a well-behaved and robust function complexity class. In particular, for every function f ? #P, we have that f + 1 is trivially in #Pe, which is an undesirable property. This has led to the introduction of the more robust class TotP, which is defined as the class of functions f for which there exists a non-deterministic Turing machine M running in polynomial time such that, f(x) is the result of subtracting 1 to the number of (non-necessarily accepting) runs of M with input x [40]. In [51], it is proved that TotP ? #Pe and that TotP has a complete function problem under parsimonious reductions. However, no natural problem is known to be TotP-complete under this type of reductions [51].





In this section we study the fragment of $\eqso(\LL)$ when $\LL$ is a boolean logic contained in $\fo$. We show that by restricting $\LL$ we can find different subclasses below $\shp$ with interesting computational and closure properties. 

\cite{OH93,FH08}

From this point on, for each fragment $\FF$ of $\qso$, we will also use $\FF$ to refer to the class of functions defined by the formulas in $\FF$.

\subsection{A counting hierarchy below $\shp$}
\label{sec-hier-shp}
Saluja et. al \cite{DBLP:journals/jcss/SalujaST95} define a family of counting classes $\#\cL$ for each fragment $\cL$ of $\fo$. For a formula $\varphi(x,X)$, the function $f_{\varphi(x,X)}$ is defined as
\[
f_{\varphi(x,X)}(\A) = \vert \{\langle e,P\rangle\mid \A\models\varphi(e,P)\}\vert.
\]
for each $\A\in\str$. A function $f:\Sigma^*\to\nat$ is in $\#\cL$ if there exists an $\cL$ formula $\varphi(x,X)$ such that $f = f_{\varphi(x,X)}$.

\begin{theorem} \label{saluja-eq}
	$\eqso(\cL) = \#\cL$ for each fragment $\L$ of $\fo$.
\end{theorem}

For every logic $\cL$, we define an $\cL$-extended quantifier-free (QF) formula as follows:
\begin{eqnarray*}
	\varphi &::=& \alpha, \alpha \text{ is an $\cL$-formula} \ \mid \\
	&& X_i(x_1,\dots,x_{a_i}), i\in\N \ \mid \ \\
	&& (\neg \varphi) \ \mid \ (\varphi \wedge \varphi) \ \mid \ (\varphi \vee \varphi).
\end{eqnarray*}

We define syntactically the fragments $\logex{i}$ and $\logux{i}$ according to the following grammar:
\begin{align*}
\logex{0} = \logux{0} &::= \varphi , \varphi \mbox{ is an $\fo$-extended QF formula,} \\
\logex{i+1} &::= \logux{i} \ \mid \ \exists x\, \logex{i+1}, \\
\logux{i+1} &::= \logex{i} \ \mid \ \forall x\, \logux{i+1}.
\end{align*}

\begin{theorem} \label{fp1}
	$\qfo(\logex{0}) \subseteq$ {\sc FP}.
\end{theorem}

The {\em decision problem} associated to a function $f$ is defined by the language $L_f = \{\A \in \str \mid f(\A) > 0\}$.

\begin{theorem} \label{decisionptime}
	The decision problem associated to a function in $\eqso(\logex{1})$ is in \textsc{P}.
\end{theorem}

For a given pair of functions $f,g$, we define $f \dotminus g$ as follows:
\begin{eqnarray*}
	(f \dotminus g)(\A) =
	\begin{cases}
		f(\A)-g(\A), & \text{if }f(\A)>g(\A) \\
		0, & \text{if }f(\A) \leq g(\A).
	\end{cases}
\end{eqnarray*}
for every $\L$-structure $\A \in \str$. A function class $\F$ is {\em closed under substraction} if for every pair of functions $f,g \in \F$, it holds that $f \dotminus g \in \F$.

\begin{theorem} \label{sub-pnp}
	If $\eqso(\loge{1})$ is closed under substraction, then {\sc P} = {\sc NP}.
\end{theorem}

\begin{theorem} \label{sigma1strict}
	$\eqso(\loge{1}) \subsetneq \eqso(\logex{1})$
\end{theorem}

For a given function $f$, we define $f \dotminus 1$ as follows:
\begin{eqnarray*}
	f \dotminus 1(\A) =
	\begin{cases}
		f(\A)-1, & \text{if }f(\A) > 0 \\
		0, & \text{if }f(\A) = 0.
	\end{cases}
\end{eqnarray*}
for every $\L$-structure $\A \in \str$. A function class $\F$ is {\em closed under substraction by one} if for every function $f \in \F$, it holds that $f \dotminus 1 \in \F$.

\begin{theorem} \label{sigmafo-minusone}
	$\eqso(\logex{1})$ is closed under substraction by one.
\end{theorem}

\begin{theorem} \label{dnf-pars}
	{\sc \#DNF} is hard for $\eqso(\loge{1})$ under parsimonious reductions. 
\end{theorem}

\begin{theorem} \label{nplusone-strict}
	$\U{1}$ with $n$ open first-order variables is properly contained in $\U{1}$ with $n+1$ open first-order variables for $n\in\N$.  
\end{theorem}

\subsection{Counting hierarchy below $\shp$}


\subsection{Horn Counting Classes}
%!TEX root = main.tex
\newcommand{\pP}{\textit{P}}
\newcommand{\pN}{\textit{N}}
\newcommand{\pV}{\textit{V}}
\newcommand{\pT}{\textit{T}}
\newcommand{\pA}{\textit{A}}
\newcommand{\pNC}{\textit{NC}}
\newcommand{\pD}{\textit{D}}


A positive literal is a formula of the form $X(\x)$, where $X$ is a second-order variable and $\x$ is a tuple of first-order variables, and a negative literal is a formula of the form $\exists \v \, \neg X(\u,\v)$, where $\u$ and $\v$ are tuples of first-order variables. Given a relational signature $\R$, a clause over $\R$ is a formula of the form:
$$
\forall \x \, (\varphi_1 \vee \cdots \vee \varphi_n),
$$
where each $\varphi_i$ ($1 \leq i \leq n$) is either a positive literal, a negative literal or an \fo-formula over $\R$.  A clause is said to be Horn if it contains at most one positive literal, and a formula is said to be Horn if it is a conjunction of Horn clauses over a relational signature $\R$. With this terminology, we define $\uhorn$ as the set of formulas $\psi$ such that $\psi$ is a conjunction of Horn clauses over a relational signature $\R$. 

\begin{proposition}\label{prop:uhorn-pe}
$\eqso(\uhorn) \subseteq \totp$
\end{proposition}

\begin{example} \label{ex-hornsat-esop1}
Let $\R = \{\pP(\cdot,\cdot), \pN(\cdot,\cdot), \pV(\cdot), \pNC(\cdot)\}$. This vocabulary is used as follows to encode a Horn formula. A fact $\pP(c,x)$ indicates that propositional variable $x$ is a disjunct in a clause $c$, while $\pN(c,x)$ indicates that $\neg x$ is a disjunct in $c$. Furthermore, $\pV(x)$ holds if  $x$ is a propositional variable, and $\pNC(c)$ holds if $c$ is a clause containing only negative literals, that is, $c$ is of the form $(\neg x_1 \vee \cdots \vee \neg x_n)$.

To encode $\chsat$, we define an \so-formula $\varphi(\pT)$ over $\R$, where $\pT$ is a unary predicate, such that for every Horn formula $\theta$ encoded by an $\R$-structure $\A$, the number of satisfying assignments of $\theta$ is equal to $\sem{\sa{\pT} \varphi(\pT)}(\A)$. In particular, $\pT(x)$ holds if and only if $x$ is a propositional variable that is assigned value 1.  More specifically, $\varphi(\pT)$ is defined as follows:
\begin{align*}
&\forall x \, (\pT(x) \to \pV(x)) \ \wedge\\
&\forall c \, (\pNC(c) \to \exists x \, (\pN(c,x) \wedge \neg \pT(x))) \ \wedge\\
&\forall c \forall x \, ([\pP(c,x) \wedge \forall y \, (\pN(c,y) \to \pT(y))] \to \pT(x)).
\end{align*}
Given that $\uhorn$ is designed with the goal in mind of capturing $\chsat$, we expect $\varphi(\pT)$ to be a formula in $\uhorn$. However, if we rewrite it as a conjunction of clauses we obtain the following:
\begin{align*}
&\forall x \, (\neg \pT(x) \vee \pV(x)) \ \wedge\\
&\forall c \, (\neg \pNC(c) \vee \exists x \, (\pN(c,x) \wedge \neg \pT(x)))\ \wedge\\
&\forall c \forall x \, (\neg \pP(c,x) \vee \exists y \, (\pN(c,y) \wedge \neg \pT(y)) \vee \pT(x)).
\end{align*}
The resulting formula $\varphi(\pT)$ is not in $\uhorn$, but it can be easily transformed into a formula in this class  by introducing an auxiliary binary predicate $\pA$ defined as follows:
\begin{align*}
\forall c \forall x \, (\neg \pA(c,x) \leftrightarrow [\pN(c,x) \wedge \neg \pT(x)]).
\end{align*}
In this way, we obtain the following formula $\psi(\pT,\pA)$ in $\uhorn$:
\begin{align*}
&\forall x \, (\neg \pT(x) \vee \pV(x)) \ \wedge\\
&\forall c \, (\neg \textit{NC}(c) \vee \exists x \, \neg \textit{A}(c,x)) \ \wedge\\
&\forall c \forall x \, (\neg \textit{P}(c,x) \vee \exists y \, \neg \textit{A}(c,y) \vee \textit{T}(x)) \ \wedge\\
&\forall c \forall x \, (\neg \textit{N}(c,x) \vee \textit{T}(x) \vee \neg \textit{A}(c,x)) \ \wedge\\
&\forall c \forall x \, (\textit{A}(c,x) \vee \textit{N}(c,x)) \ \wedge\\
&\forall c \forall x \, (\textit{A}(c,x) \vee \neg\textit{T}(x)).
\end{align*}
This formula effectively defines $\chsat$
as for every Horn formula $\theta$ encoded by an $\R$-structure $\A$, the number of satisfying assignments of $\theta$ is equal to $\sem{\sa{\pT} \sa{\pA} \psi(\pT,\pA)}(\A)$.  Therefore, we conclude that $\chsat \in \eqso(\uhorn)$. 
\end{example}
We extend the definition of $\uhorn$ to allow existential quantification. More precisely, a formula $\varphi$ is in $\ehorn$ if $\varphi$ is of the form $\exists \bar x \, \psi$ with $\psi$ a Horn formula. Interestingly, it hold that $\cdnf \in \eqso(\ehorn)$ and

\begin{proposition}\label{prop:ehorn-pe}
$\eqso(\ehorn) \subseteq \totp$.
\end{proposition}
A natural question at this point is whether in the definitions of $\uhorn$ and $\ehorn$, it is necessary to allow negative literals of the form $\exists \v \, \neg X(\u,\v)$. The following result shows that it is indeed the case:

\begin{proposition}\label{prop:hsat-not-sigma2}	
$\chsat \not\in \eqso(\loge{2})$.
\end{proposition}
We conclude this section by showing that a natural extension of $\chsat$ is $\eqso(\ehorn)$-complete under parsimonious reductions. We define the decision problem:
\begin{multline*}
\dhsat = \{\Phi \mid \Phi \text{ is a disjunction of}\\  \text{Horn formulas and $\Phi$ is satisfiable}\},
\end{multline*}
and the counting problem $\shdhsat$ as a function that counts all satisfying assignments of a formula $\Phi$ that is a disjunction of Horn formulas.

\begin{theorem} \label{sigma2hard}
	$\shdhsat$ is $\eqso(\ehorn)$-complete under parsimonious reductions. 
\end{theorem}


\section{Beyond QSO}
%!TEX root = main.tex

\subsection{Transitive Logics}

It was shown in \cite{I86,I88} that first-order logic extended with a transitive closure operator captures $\nlog$. Inspired by this work, we extend the definition of $\qfo$ with an operator for counting the number of paths in a directed graph, and we show that it can be used to capture $\shl$. Besides, we show that the same idea can be used to extend $\qso$ allowing to capture harder complexity classes. 

Given a relation signature $\R$, the set of transitive $\qso$ formulas ($\tqso$-formulas) is defined by the following grammar:
\begin{multline}
	\label{eq-def-tqso}
	\alpha := \varphi \, \mid \, s \, \mid \, (\alpha \add \alpha) \, \mid\, (\alpha \mult \alpha) \, \mid \, 
	\sa{x} \alpha \, \mid\, \
	\pa{x} \alpha \, \mid \\ 
	\sa{X} \alpha \, \mid \, \pa{X} \alpha \, \mid \, [\pth \psi(\bar{x}, \bar{X},\bar{y}, \bar{Y})],
\end{multline}
where $\varphi$ is an $\so$-formula over $\R$, $\psi(\bar{x},\bar{X},\bar{y},\bar{Y})$ is an $\fo$-formula over $\R$, $\bar{x} = (x_1, \ldots, x_k)$, $\bar{y} = (y_1, \ldots, y_k)$ are tuples of pairwise distinct first-order variables, and $\bar{X} = (X_1, \ldots, X_\ell)$, $\bar{Y} = (Y_1, \ldots, Y_\ell)$ are tuples of pairwise distinct second-order variables such that $\arity(X_i) = \arity(Y_i) = m_i$ for every $i \in \{1, \ldots, \ell\}$. The semantics of $[\pth \psi(\bar{x},\bar{X},\bar{y}.\bar{Y})]$ is defined as follows. Given an $\R$-structure $\A$, define a (directed) graph $\cG_{\psi}(\A) = (N,E)$ such that $N = A^k \times \prod_{i=1}^\ell 2^{A^{m_i}}$, where $A$ is the domain of $\A$, and for every pair $(\bar b, \bar B)$, $(\bar c, \bar C)$ of elements of $N$, it holds that $((\bar b, \bar B), (\bar c, \bar C)) \in E$ if and only if $\A \models \psi(\bar b, \bar{B}, \bar c, \bar{C})$. Then given first-order and second-order assignments $v$, $V$ for $\A$, we have that $\sem{[\pth \psi(\bar{x},\bar{X}, \bar{y}, \bar{Y})]}(\A,v,V)$ is the number of paths in $\cG_\psi(\A)$ from $(v(\bar x), V(\bar X))$ to $(v(\bar y), V(\bar Y))$ whose length is at most $n = |N|$.

As for the case of $\qso$, the logic $\tqso(\LL)$ is obtained by restricting $\varphi$ in \eqref{eq-def-tqso} to be a formula in $\LL$. Moreover, the logic $\tqfo$ is obtained by disallowing in \eqref{eq-def-tqso} formulas $\sa{X} \alpha$ and $\pa{X} \alpha$, and by only allowing  first-order free-variables in the formula $\psi$ used in $[\pth \psi]$ in \eqref{eq-def-tqso}. With this notation, we have the following results:


\begin{theorem} \label{tqfo-shl}
	$\tqfo(\fo)$ captures $\shl$ over the class of ordered structures.
\end{theorem}

\begin{theorem} \label{tqso-fo-fpsace}
	$\tqso$ and $\tqso(\fo)$ captures $\fpspace$ over the class of ordered structures.
\end{theorem}

\marcelo{Puede que este equivocado, pero me parece que teniamos una forma de capturar $\shp$ usando el operator ${\bf path}$. Pero no logro recordar como se hacia esto, y me parece que lo que habiamos escrito antes en esta seccion estaba equivocado: ``$\tqso(\fo)$ captures $\shp$ over the class of ordered structures". Claro que yo puedo estar usando una definicion distinta de $\tqso(\fo)$.}


%We also define the set of transitive $\qso$ formulas ($\tqso$-formulas) using the following grammar:
%\begin{multline*}
%%	\label{eq-def-tqso}
%	\alpha := \varphi \ \mid \ s \ \mid \ (\alpha \add \alpha) \ \mid\ (\alpha \mult \alpha) \ \mid \\ \sa{x} \alpha \ \mid \ \pa{x} \alpha \ \mid \ \sa{X} \alpha \ \mid \ \pa{X} \alpha \ \mid \ [\pth \varphi]
%\end{multline*}
%
%
% 
%We define the operator {\bf path} as follows. Let $\A$ be an ordered structure. Given a formula $\psi(\bar{x},\bar{y})$, where $\vert \bar{x} \vert = \vert \bar{y} \vert = k$ let ${\cal G} = ({\cal V},\cal{E})$ be induced graph over the set of vertices ${\cal V} = A^k$, and for every $\bar{a},\bar{b}\in A^k$ it holds that ${\cal E}(\bar{a},\bar{b})$ if and only if $\A \models \psi(\bar{a},\bar{b})$. To formalize the semantics for this operator, let $n = \vert A^k \vert$.
%For a given first order assignment $v$ and a second order asssignment $V$, let $\bar{a} = v(\bar{x})$ and $\bar{b} = v(\bar{y})$, and $\sem{[\pth\, \psi(\bar{x},\bar{y})]}(\A,v,V)$ will take the value of the number of paths of size less or equal to $n$ from $\bar{a}$ to $\bar{b}$ in the graph ${\cal G}$. This operator lets us define the set of transitive $\qfo$ formulas over $\R$ ($\tqfo$-formulas) using the following grammar:
%\begin{multline*} 
%%	\label{eq-def-tqfo}
%	\alpha := \varphi \ \mid \ s \ \mid \ (\alpha \add \alpha) \ \mid\ (\alpha \mult \alpha) \\ \mid \ \sa{x} \alpha \ \mid \ \pa{x} \alpha \ \mid \ [\pth \varphi]
%\end{multline*}
%where $\varphi$ is an $\fo$-formula over $\R$, $s \in \bbN$ and $x \in \fv$.
%
%We also define the set of transitive $\qso$ formulas ($\tqso$-formulas) using the following grammar:
%\begin{multline*}
%%	\label{eq-def-tqso}
%	\alpha := \varphi \ \mid \ s \ \mid \ (\alpha \add \alpha) \ \mid\ (\alpha \mult \alpha) \ \mid \\ \sa{x} \alpha \ \mid \ \pa{x} \alpha \ \mid \ \sa{X} \alpha \ \mid \ \pa{X} \alpha \ \mid \ [\pth \varphi]
%\end{multline*}
%where $\varphi$ is an $\so$-formula over $\R$, $s \in \bbN$, $x \in \fv$ and $X \in \sv$.
%\begin{theorem} \label{so-rec}
%	Given a positive $\fo$ formula $\varphi(\bar{x},R)$ and a $\qfo$ formula $\alpha(\bar{x})$, there exists a $\qso$ formula $\beta(\bar{x})$ such that $\sem{[\alfp\varphi(\bar{x},R)\mid \alpha(\bar{x},R)](\bar{x})} = \sem{\beta(\bar{x})}$.
%\end{theorem}
%
%\begin{theorem} \label{tqfo-fo-cap}
%	$\tqfo(\fo)$ captures $\shl$ over the class of ordered structures.
%\end{theorem}
%
%\begin{theorem} \label{tqso-fo-cap}
%	$\tqso(\fo)$ captures $\shp$ over the class of ordered structures.
%\end{theorem}


\subsection{Recursive Logics}

We define an operator which extends least fixed point logic \cite{I86,vardi1982complexity} to allow counting. 
Fix a relational signature $\R$. Then the set of $\rqfo$ formulas over $\R$ is defined by the following grammar:
\begin{multline*}
	\alpha := \varphi \, \mid \, s \, \mid \, (\alpha \add \alpha) \, \mid\, (\alpha \mult \alpha) \, \mid \, 
	\sa{x} \alpha \, \mid\\ 
	\pa{x} \alpha \, \mid\,
	\clfp{\psi(x_1,\ldots,x_k,R)}{\beta(y_1, \ldots, y_\ell, R, \pi)},
\end{multline*}
where $R$ is second-order variable of arity $k$, $\psi(x_1, \ldots, x_k, R)$ is an $\fo$-formula over $(\R \cup \{R\})$ that is positive on $R$ and $\beta(y_1, \ldots, y_\ell, R, \pi)$ is a $\qfo$-formula over $\R$ including a fresh symbol $\pi$ that represents a function of arity $\ell$, that is, every atomic formula in $\beta(y_1, \ldots, y_\ell, R, \pi)$ is either an $\fo$-formula over $\R$, a constant $s \in \N$ or a term of the form $\pi(u_1, \ldots, u_\ell)$ with $u_1, \ldots, u_\ell$ non-necessarily distinct first-order variables. 
The free variables of the formula $\clfp{\psi(x_1,\ldots,x_k,R)}{\beta(y_1, \ldots, y_\ell, R, \pi)}$
are $y_1, \ldots, y_\ell$, in particular it does not have any second-order free variable.

We need to introduce some terminology to define the semantics of the formula $\clfp{\psi(x_1,\ldots,x_k,R)}{\beta(y_1, \ldots, y_\ell, R, \pi)}$
Given a $\qfo$-formula $\gamma$ with $\ell$ first-order free variables and no second-order free variable, define $\beta[\gamma/\pi]$ to be the $\qfo$-formula obtained from $\beta$ by replacing every occurrence of a term $\pi(u_1, \ldots, u_\ell)$ by $\gamma(u_1, \ldots, u_\ell)$.
Let $\A$ be an $\R$-structure with domain $A$. Then for every $a \in A$, let $\varphi_a(x)$ be an $\fo$-formula such that $\A \models \varphi_a(b)$ if and only if $b = a$. Formally, assuming that $a$ is the $p$-th element of $A$ according to the linear order $<^\A$, we have that:
%then $\varphi_a(x) = \forall y(x < y \vee x = y)$
%for every natural number $i \geq 1$, let $\varphi_i(x)$ be an $\fo$-formula such that $\varphi_i(a)$ holds in an $\R$-structure $\A$ if and only if $a$ is the $i$-th element in the linear order  $<^{\A}$,
%that is, $\varphi_1(x) = \forall y(x < y \vee x = y)$ and for every $p > 1$:
\begin{multline*}
\varphi_a(x) \ = \ \exists x_1 \cdots \exists x_{p-1}\bigg[\bigwedge_{i =1}^{p-2}x_i < x_{i+1} \wedge\,\\  
x_{p-1} < x  \wedge \forall y(y < x \to \bigvee_{i = 1}^{p-1} y = x_i)\bigg].
\end{multline*}
Finally, given a function $f : A^\ell \to \N$, define $\qfo$-formula $\delta_{f,\A}(z_1, \ldots, z_\ell)$ as follows:
\begin{multline*}
 \mathop{+}_{(a_1,\ldots,a_{\ell})\in A^{\ell}} \ \varphi_{a_1}(z_1) \cdot \ldots \cdot \varphi_{a_{\ell}}(z_{\ell})\cdot f(a_1,\ldots,a_{\ell}), 
\end{multline*}
where $z_1, \ldots, z_\ell$ is a sequence of pairwise distinct first-order variables.
We have that $\delta_{f,\A}(z_1, \ldots, z_\ell)$ encodes $f$ in the structure $\A$ since:
\begin{eqnarray*}
\sem{\delta_{f,\A}(z_1, \ldots, z_\ell)}(\A,v)  & = & f(v(z_1), \ldots, v(z_\ell)).
\end{eqnarray*}
%Recall that the least fixed point operator is defined by a formula $\psi(x_1,\ldots,x_k,R)$ that is positive on $R$, where $R$  is a predicate of arity $k$. For a structure $\A$ with domain $A$, the operator $T_{\varphi}:2^{A^k} \to 2^{A^k}$ is defined as $T_{\varphi}(X) = \{(a_1,\ldots,a_k)\mid (\A,X)\models \psi(a_1,\ldots,a_k,R) \}$, for each $X\subseteq A^k$. Let $T_0 = \emptyset$ and $T_{i+1} = T_{\varphi}(T_i)$ for each $i \in \nat$. Note that there exists $n\in \nat$ such that $T_{n+1} = T_n$ because $R$ is positive in $\psi(x_1,\ldots,x_k,R)$. Then the evaluation of $[\lfpop\psi(x_1,\ldots,x_k,R)]$ is defined by the fixed point $T_n$, that is, for every $(a_1,\ldots,a_k)\in A^k$, it holds that $\A\models[\lfpop\psi(x_1,\ldots,x_k,R)](a_1,\ldots,a_k)$ if and only if $(a_1,\ldots,a_k) \in T_n$.
We are now ready to define the semantics of the formula $\clfp{\psi(x_1,\ldots,x_k,R)}{\beta(y_1,\ldots,y_{\ell},R,\pi)}$,  whose free variables are $y_1$, $\ldots$, $y_\ell$. Assume that $\A$ is an $\R$-structure with domain $A$. Following the definition of the semantics of least fixed point logic \cite{I86,vardi1982complexity}, define an operator $T_{\varphi}:2^{A^k} \to 2^{A^k}$ such that:
$$
T_{\varphi}(X)  =  \{(a_1,\ldots,a_k) \in A^k \mid \A \models \psi(a_1,\ldots,a_k,X) \},
$$
for every $X \subseteq A^k$. This operator is used to defined 
a sequence $T_0 \subsetneq T_1 \subsetneq \cdots \subsetneq T_n \subseteq A^k$  such that $n \geq 0$, $T_0 = \emptyset$, $T_{i+1} = T_{\varphi}(T_i)$ for every $i \in \{0, \ldots, n-1\}$, and $T_n = T_\varphi(T_n)$. We know that such a sequence exists as formula $\psi(x_1,\ldots,x_k,R)$ is positive on $R$. Now, the sequence $T_0 \subsetneq T_1 \subsetneq \cdots \subsetneq T_n$ is used to defined  a sequence of functions $f_0,f_1,\ldots,f_n:A^{\ell}\to\nat$. Formally, we have that $f_0(a_1, \ldots, a_\ell) = 0$ for every $(a_1, \ldots, a_\ell) \in A^\ell$. Moreover, assuming that $(a_1, \ldots, a_\ell) \in A^\ell$, $v$ is a first-order assignment for $\A$ such that $v(y_i) = a_i$ for every $i \in \{1, \ldots, \ell\}$, and that $V$ is a second-order assignment for $\A$ such that $V(R) = T_{i+1}$, we have that:
\begin{eqnarray*}
f_{i+1}(a_1, \ldots, a_\ell) & = & \sem{\beta[\delta_{f_i,\A}/\pi]}(\A,v,V).
\end{eqnarray*}
%, for each $T_i \in \T$, let $V$ be a second-order assignment for $\A$ that assigns $T_i$ to $R$, let $\zeta_i: A^{\ell}\to\nat$ be such that for each $(a_1,\ldots,a_{\ell})\in A^{\ell}$ it holds $\zeta_i(a_1,\ldots,a_{\ell}) = \sem{\alpha\mid_{\pi(u_1,\ldots,u_{\ell})\to \beta_i(u_1,\ldots,u_{\ell})}}(\A,v,V)$, where $v$ is a first-order assignment for $\A$ that satisfies $a_i = v(x_i)$ for each free $x_i$ in $\alpha$.
Finally, the semantics of the formula $\clfp{\psi(x_1,\ldots,x_k,R)}{\beta(y_1,\ldots,y_{\ell},R,\pi)}$ is defined by means of function $f_n$:
%For a given first order assignment $v$ and a second order assignment $V$, let $a_i = v(x_i)$, the operator is then evaluated as:
\begin{align*}
&\llbracket [{\bf lfp} \, \psi(x_1,\ldots,x_k,R) \,|\\
&\hspace{10mm}\beta(y_1,\ldots,y_{\ell},R,\pi)]\rrbracket(\A,v) \ = \
f_n(v(y_1),\ldots,v(y_{\ell})).
\end{align*}

\begin{example}
%As an example, 
We would like to define a formula that, given a graph $G$ with $n$ nodes and a pair of nodes $b$, $c$ in $G$, counts the number of paths of length at most $n$ from $b$ to $c$ in $G$.
To this end, assume that $\R = \{ E(\cdot,\cdot) \}$, and define $\psi(x,R)$ as follows:
\begin{eqnarray*}
%\psi(x,R) = 
\forall y(x < y \vee x = y) \vee \exists z(R(z) \wedge \varphi_{succ}(z,x)),
\end{eqnarray*}
where $\varphi_{succ}(x,y)$ is a formula that is satisfied by pairs $(x,y)$ that are consecutive in the order $<$. That is, $\varphi_{succ}(x,y) = x < y \wedge \neg \exists z \, (x < z \wedge z < y)$. Moreover, define formula $\beta(x,y,R,\pi)$ as follows:
$$
%\alpha(x,y,R,\pi) = 
%(\neg \exists zR(z))\cdot(x = y) 
E(x,y) + \sa{z} [\pi(x,z)\cdot E(z,y)].
$$
Then we have that $\clfp{\psi(x,R)}{\beta(x,y,R,\pi)}$ defines our counting function. In fact, assume that $\A$ is an $\R$-structure with $n$ elements in its domain, $b,c$ are elements of $\A$ and $v$ is a first-order assignment over $\A$ such that $v(x) = b$ and $v(y) = c$. Then we have that $\sem{\clfp{\psi(x,R)}{\beta(x,y,R,\pi)}}(\A,v)$ is equal to the  number of paths in $\A$ from $b$ to $c$ of length at most $n$.
\end{example}

\marcelo{Deberiamos mostrar como el ejemplo anterior se generaliza para definir $[\pth \psi(\bar x, \bar y)]$ en terminos del operador ${\bf lfp}$.}

It is well known that least fixed point logic is contained in second-order logic \cite{L04}. In the following theorem we show that the same holds in our case.
\begin{theorem} \label{so-rec}
$\rqfo \subseteq \qso$
%	Given a positive $\fo$ formula $\varphi(\bar{x},R)$ and a $\qfo$ formula $\alpha(\bar{x})$, there exists a $\qso$ formula $\beta(\bar{x})$ such that $\sem{[\alfp\varphi(\bar{x},R)\mid \alpha(\bar{x},R)](\bar{x})} = \sem{\beta(\bar{x})}$.
\end{theorem}

Finally, the following is the main result of this section:
\begin{theorem} \label{rqfo-fo-cap}
	$\rqfo(\fo)$ captures $\fp$ over the class of ordered structures.
\end{theorem}

\marcelo{Notese que estoy permitiendo en la formula $\beta(y_1, \ldots, y_\ell,R,\pi)$ tener subformulas $\pa{x} \alpha$. Esta bien el resultado con esto? O tenemos que eliminar estas formulas?}

%Given a relation signature $\R$, the set of recursive $\qfo$ formulas ($\rqfo$-formulas) is defined by the following grammar:
%%This operator lets us define the set of recursive $\qfo$ formulas over $\R$ ($\rqfo$-formulas) using the following grammar:
%\begin{multline*}
%%	\label{eq-def-rqfo}
%	\alpha := \varphi \ \mid \ s \ \mid \ (\alpha \add \alpha) \ \mid \\ (\alpha \mult \alpha) \ \mid \ \sa{x} \alpha \ \mid \ \pa{x} \alpha \ \mid \ [\alfp \varphi \mid \alpha]
%\end{multline*}
%where $\varphi$ is an $\fo$-formula over $\R$, $s \in \bbN$ and $x \in \fv$.
%
%\marcelo{Vamos a permitir anidacion del operador $\alfp$? Esta gramatica lo permite.}
%
%\begin{theorem} \label{rqfo-fo-cap}
%	$\rqfo(\fo)$ captures $\fp$ over the class of ordered structures.
%\end{theorem}
%


\section{Conclusion}
%!TEX root = main.tex

Here comes the conclusions.


\section*{Acknowledgments}
The authors would like to thank ~\cite{DrosteG07}



\bibliographystyle{IEEEtran}
\bibliography{biblio}

\newpage

\onecolumn
\appendix
%!TEX root = main.tex

Here starts the appendix.


\end{document}


