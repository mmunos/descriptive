\documentclass[conference]{IEEEtran}

\usepackage{cite}

\usepackage[utf8]{inputenc}	
\usepackage{amsmath}
\usepackage{amsfonts}
\usepackage{wrapfig}
%\usepackage{amssymb}
\usepackage{stmaryrd}
\usepackage{thmtools}
\usepackage{bbold}
\usepackage{multicol}
%\usepackage{MnSymbol}
\usepackage{tikz}
\usetikzlibrary{arrows,automata}
\usepackage{calc}
\usepackage{mathabx}
\usepackage[textwidth=2cm,textsize=small]{todonotes}

\usepackage{array}
\usepackage[caption=false,font=normalsize,labelfont=sf,textfont=sf]{subfig}
\usepackage{fixltx2e}
\usepackage{stfloats}
\usepackage{url}

\usetikzlibrary{chains,fit,shapes}
\usetikzlibrary{arrows,positioning} 

\tikzset{
    rect/.style={
           rectangle,
           rounded corners,
           draw=black, 
           thick,
           text centered},
    rectw/.style={
           rectangle,
           rounded corners,
           draw=white, 
           thick,
           text centered},
    sq/.style={
           rectangle,
           draw=black, 
           thick,
           text centered},
    sqw/.style={
           rectangle,
           draw=white, 
           thick,
           text centered},
    arrout/.style={
           ->,
           -latex,
           thick,
           },
    arrin/.style={
           <-,
           latex-,
           thick,
           },
    arrd/.style={
           <->,
           >=latex,
           thick,
           },
    arrw/.style={
           thick,
           }
}


\hyphenation{op-tical net-works semi-conduc-tor}

% packages
\usepackage[english]{babel}
\usepackage{amsmath,amsfonts,amssymb,mathrsfs,latexsym,stmaryrd}
\usepackage{amsthm}
\usepackage{algorithm,algpseudocode}
\usepackage{array}
\usepackage{url}
\usepackage{comment}
\usepackage[newitem,newenum,flushleft,neverdecrease]{paralist}
\usepackage{epsfig}
\usepackage{ifpdf}
%\usepackage[usenames,dvipsnames]{color}
%\usepackage[colorlinks=true,linkcolor=Blue,citecolor=Blue,urlcolor=Magenta,pdfpagelabels,plainpages=false]{hyperref}
%\usepackage[all]{hypcap}

% fix for metapost
\ifpdf
\DeclareGraphicsRule{*}{mps}{*}{}
\else
\DeclareGraphicsRule{*}{eps}{*}{}
\fi

\newtheorem{theorem}{Theorem}[section]
\newtheorem{example}[theorem]{Example}
\newtheorem{conjecture}[theorem]{Conjecture}
\newtheorem{definition}[theorem]{Definition}
\newtheorem{lemma}[theorem]{Lemma}
\newtheorem{claim}[theorem]{Claim}
\newtheorem{corollary}[theorem]{Corollary}
\newtheorem{property}[theorem]{Property}
\newtheorem{proposition}[theorem]{Proposition}
\newtheorem{invariant}[theorem]{Invariant}

\newenvironment{reptheorem}[1]{\begin{trivlist} \item[ {\textbf{Theorem #1}.}] \it}{\end{trivlist}}
\newenvironment{replemma}[1]{\begin{trivlist} \item[ {\textbf{Lemma #1}.}] \it}{\end{trivlist}}
\newenvironment{repproposition}[1]{\begin{trivlist} \item[{\textbf{Proposition #1}.}] \it}{\end{trivlist}}

\newenvironment{runexample}{\begin{example}}{\end{example}}
\newenvironment{runexamplec}[1]{\begin{trivlist} \item[\hskip 15pt {{\sc Example #1 (continued)}.}] \it}{\end{trivlist}}

% environments
\def\labelitemi{\textbullet}
\setlength{\pltopsep}{\smallskipamount}
\setlength{\plitemsep}{\smallskipamount}
\newenvironment{dotlist}{\begin{compactitem}}{\end{compactitem}}
\newenvironment{numlist}{\begin{compactenum}}{\end{compactenum}}
\newenvironment{romlist}{\begin{compactenum}[i)]}{\end{compactenum}}
\newenvironment{condlist}{\begin{compactenum}[{(C}1{)}]}{\end{compactenum}}
\newenvironment{proplist}{\begin{compactenum}[{(P}1{)}]}{\end{compactenum}}
\newenvironment{axiomlist}{\begin{compactenum}[{(A}1{)}]}{\end{compactenum}}
\newcommand{\refeq}[1]{\hyperref[#1]{(\ref*{#1})}}
\newcommand{\refcond}[1]{\hyperref[#1]{C\ref*{#1}}}
\newcommand{\refprop}[1]{\hyperref[#1]{P\ref*{#1}}}
\newcommand{\refaxiom}[1]{\hyperref[#1]{A\ref*{#1}}}
\newcommand{\margincomment}[1]{\marginpar{\small\textit{#1}}}
\newcommand{\op}[1]{\operatorname{#1}}

\providecommand{\qed}{\hfill $\Box$}
\renewcommand{\qed}{\hfill $\Box$}
\newcommand{\eproof}{\qed}

% domains
\newcommand{\bbB}{\mathbb{B}}
\newcommand{\bbD}{\mathbb{D}}
\newcommand{\bbI}{\mathbb{I}}
\newcommand{\bbJ}{\mathbb{J}}
\newcommand{\bbN}{\mathbb{N}}
\newcommand{\bbNp}{\mathbb{N}_{>0}}
\newcommand{\bbNinf}{\mathbb{N}\cup\{\infty\}}
\newcommand{\bbNpinf}{\mathbb{N}_{>0}\cup\{\infty\}}
\newcommand{\bbZ}{\mathbb{Z}}
\newcommand{\bbP}{\mathbb{P}}
\newcommand{\bbQ}{\mathbb{Q}}
\newcommand{\bbQp}{\mathbb{Q}_{>0}}
\newcommand{\bbR}{\mathbb{R}}
\newcommand{\bbRp}{\mathbb{R}_{>0}}
\newcommand{\bbC}{\mathbb{C}}
\newcommand{\bbS}{\mathbb{S}}

% classes
\newcommand{\cA}{\mathcal{A}}
\newcommand{\cB}{\mathcal{B}}
\newcommand{\cC}{\mathcal{C}}
\newcommand{\cD}{\mathcal{D}}
\newcommand{\cE}{\mathcal{E}}
\newcommand{\cF}{\mathcal{F}}
\newcommand{\cG}{\mathcal{G}}
\newcommand{\cH}{\mathcal{H}}
\newcommand{\cI}{\mathcal{I}}
\newcommand{\cJ}{\mathcal{J}}
\newcommand{\cK}{\mathcal{K}}
\newcommand{\cL}{\mathcal{L}}
\newcommand{\cM}{\mathcal{M}}
\newcommand{\cN}{\mathcal{N}}
\newcommand{\cO}{\mathcal{O}}
\newcommand{\cP}{\mathcal{P}}
\newcommand{\cQ}{\mathcal{Q}}
\newcommand{\cR}{\mathcal{R}}
\newcommand{\cS}{\mathcal{S}}
\newcommand{\cT}{\mathcal{T}}
\newcommand{\cU}{\mathcal{U}}
\newcommand{\cV}{\mathcal{V}}
\newcommand{\cX}{\mathcal{X}}
\newcommand{\cY}{\mathcal{Y}}
\newcommand{\cZ}{\mathcal{Z}}
\newcommand{\cW}{\mathcal{W}}

% languages
\newcommand{\sA}{\mathscr{A}}
\newcommand{\sB}{\mathscr{B}}
\newcommand{\sC}{\mathscr{C}}
\newcommand{\sD}{\mathscr{D}}
\newcommand{\sE}{\mathscr{E}}
\newcommand{\sF}{\mathscr{F}}
\newcommand{\sG}{\mathscr{G}}
\newcommand{\sH}{\mathscr{H}}
\newcommand{\sI}{\mathscr{I}}
\newcommand{\sJ}{\mathscr{J}}
\newcommand{\sK}{\mathscr{K}}
\newcommand{\sL}{\mathscr{L}}
\newcommand{\sM}{\mathscr{M}}
\newcommand{\sN}{\mathscr{N}}
\newcommand{\sO}{\mathscr{O}}
\newcommand{\sP}{\mathscr{P}}
\newcommand{\sQ}{\mathscr{Q}}
\newcommand{\sR}{\mathscr{R}}
\newcommand{\sS}{\mathscr{S}}
\newcommand{\sT}{\mathscr{T}}
\newcommand{\sU}{\mathscr{U}}
\newcommand{\sV}{\mathscr{V}}
\newcommand{\sX}{\mathscr{X}}
\newcommand{\sY}{\mathscr{Y}}
\newcommand{\sZ}{\mathscr{Z}}
\newcommand{\sW}{\mathscr{W}}

% structures
\newcommand{\fA}{\mathfrak{A}}
\newcommand{\fB}{\mathfrak{B}}
\newcommand{\fC}{\mathfrak{C}}
\newcommand{\fD}{\mathfrak{D}}
\newcommand{\fE}{\mathfrak{E}}
\newcommand{\fF}{\mathfrak{F}}
\newcommand{\fG}{\mathfrak{G}}
\newcommand{\fH}{\mathfrak{H}}
\newcommand{\fI}{\mathfrak{I}}
\newcommand{\fJ}{\mathfrak{J}}
\newcommand{\fK}{\mathfrak{K}}
\newcommand{\fL}{\mathfrak{L}}
\newcommand{\fM}{\mathfrak{M}}
\newcommand{\fN}{\mathfrak{N}}
\newcommand{\fO}{\mathfrak{O}}
\newcommand{\fP}{\mathfrak{P}}
\newcommand{\fQ}{\mathfrak{Q}}
\newcommand{\fR}{\mathfrak{R}}
\newcommand{\fS}{\mathfrak{S}}
\newcommand{\fT}{\mathfrak{T}}
\newcommand{\fU}{\mathfrak{U}}
\newcommand{\fV}{\mathfrak{V}}
\newcommand{\fX}{\mathfrak{X}}
\newcommand{\fY}{\mathfrak{Y}}
\newcommand{\fZ}{\mathfrak{Z}}
\newcommand{\fW}{\mathfrak{W}}

% objects
\newcommand{\bA}{\mathbf{A}}
\newcommand{\bB}{\mathbf{B}}
\newcommand{\bC}{\mathbf{C}}
\newcommand{\bD}{\mathbf{D}}
\newcommand{\bE}{\mathbf{E}}
\newcommand{\bF}{\mathbf{F}}
\newcommand{\bG}{\mathbf{G}}
\newcommand{\bH}{\mathbf{H}}
\newcommand{\bI}{\mathbf{I}}
\newcommand{\bJ}{\mathbf{J}}
\newcommand{\bK}{\mathbf{K}}
\newcommand{\bL}{\mathbf{L}}
\newcommand{\bM}{\mathbf{M}}
\newcommand{\bN}{\mathbf{N}}
\newcommand{\bO}{\mathbf{O}}
\newcommand{\bP}{\mathbf{P}}
\newcommand{\bQ}{\mathbf{Q}}
\newcommand{\bR}{\mathbf{R}}
\newcommand{\bS}{\mathbf{S}}
\newcommand{\bT}{\mathbf{T}}
\newcommand{\bU}{\mathbf{U}}
\newcommand{\bV}{\mathbf{V}}
\newcommand{\bX}{\mathbf{X}}
\newcommand{\bY}{\mathbf{Y}}
\newcommand{\bZ}{\mathbf{Z}}
\newcommand{\bW}{\mathbf{W}}

% other objects
\newcommand{\tA}{\mathtt{A}}
\newcommand{\tB}{\mathtt{B}}
\newcommand{\tC}{\mathtt{C}}
\newcommand{\tD}{\mathtt{D}}
\newcommand{\tE}{\mathtt{E}}
\newcommand{\tF}{\mathtt{F}}
\newcommand{\tG}{\mathtt{G}}
\newcommand{\tH}{\mathtt{H}}
\newcommand{\tI}{\mathtt{I}}
\newcommand{\tJ}{\mathtt{J}}
\newcommand{\tK}{\mathtt{K}}
\newcommand{\tL}{\mathtt{L}}
\newcommand{\tM}{\mathtt{M}}
\newcommand{\tN}{\mathtt{N}}
\newcommand{\tO}{\mathtt{O}}
\newcommand{\tP}{\mathtt{P}}
\newcommand{\tQ}{\mathtt{Q}}
\newcommand{\tR}{\mathtt{R}}
\newcommand{\tS}{\mathtt{S}}
\newcommand{\tT}{\mathtt{T}}
\newcommand{\tU}{\mathtt{U}}
\newcommand{\tV}{\mathtt{V}}
\newcommand{\tX}{\mathtt{X}}
\newcommand{\tY}{\mathtt{Y}}
\newcommand{\tZ}{\mathtt{Z}}
\newcommand{\tW}{\mathtt{W}}

% shorthands
\newcommand{\s}[1]{\vspace{#1mm}}
\newcommand{\proofskip}{\smallskip\noindent}

\providecommand{\mit}{\mathit}
\renewcommand{\mit}{\mathit}
\newcommand{\mt}{\operatorname}
\newcommand{\mtt}{\mathtt}
\newcommand{\msl}{\mathsl}
\newcommand{\mbf}{\mathbf}
\newcommand{\und}{\underline}

% other objects
\newcommand{\emptystr}{\varepsilon}
\newcommand{\fil}{\blacksquare}
\newcommand{\gap}{\square}
\newcommand{\sep}{\raisebox{0.9pt}{\ensuremath{\mspace{1mu}\wr\mspace{1mu}}}}
\newcommand{\filsep}{\raisebox{1pt}{\text{$\blacktriangleleft$}}}

% vectors
\makeatletter
\def\shortrightarrowfill@{\arrowfill@\relbar\relbar\shortrightarrow}
\newcommand{\ort}{\mathpalette{\overarrow@\shortrightarrowfill@}}
\makeatother
\makeatletter
\def\shortleftarrowfill@{\arrowfill@\relbar\relbar\shortleftarrow}
\newcommand{\olft}{\mathpalette{\overarrow@\shortleftarrowfill@}}
\makeatother
\makeatletter
\def\shortleftrightarrowfill@{\arrowfill@\relbar\relbar\leftrightarrow}
\newcommand{\olftrt}{\mathpalette{\overarrow@\shortleftrightarrowfill@}}
\makeatother
\newcommand{\til}[1]{\widetilde{#1}}

% formulas
\newcommand{\sat}{\vDash}
\newcommand{\nsat}{\nvDash}
\newcommand{\et}{\;\wedge\;}
\newcommand{\vel}{\;\vee\;}
\newcommand{\then}{\;\rightarrow\;}
\newcommand{\Then}{\;\Rightarrow\;}
\renewcommand{\iff}{\;\leftrightarrow\;}
\newcommand{\Iff}{\;\Leftrightarrow\;}
\newcommand{\fa}[1]{\forall{#1}.\:}
\newcommand{\ex}[1]{\exists{#1}.\:}
\newcommand{\exinfinite}[1]{\exists^{\omega}{\:#1}.\:}
\newcommand{\exfinite}[1]{\exists^{<\omega}{\:#1}.\:}
\newcommand{\nex}[1]{\nexists{\:#1}.\:}

% sets, tuples, ...
\newcommand{\set}[1]{{\{ #1 \}}}
\newcommand{\bigset}[1]{{\bigl\{ #1 \bigr\}}}
\newcommand{\biggset}[1]{{\left\{ #1 \right\}}}
\newcommand{\settc}[2]{{\{ #1 \,:\, #2 \}}}
\newcommand{\bigsettc}[2]{{\bigl\{ #1 \,:\,#2 \bigr\}}}
\newcommand{\biggsettc}[2]{{\left\{ #1 \,:\,#2 \right\}}}
\newcommand{\ang}[1]{{\langle #1 \rangle}}
\newcommand{\bigang}[1]{{\bigl\langle #1 \bigr\rangle}}
\newcommand{\biggang}[1]{{\left\langle #1 \right\rangle}}
\newcommand{\intr}[1]{{\llbracket #1 \rrbracket}}
\newcommand{\bigintr}[1]{{\bigl\llbracket #1 \bigr\rrbracket}}
\newcommand{\biggintr}[1]{{\left\llbracket #1 \right\rrbracket}}
\newcommand{\len}[1]{{\lvert #1 \rvert}}
\newcommand{\biglen}[1]{{\bigl\lvert #1 \bigr\rvert}}
\newcommand{\bigglen}[1]{{\left\lvert #1 \right\rvert}}
\newcommand{\dlen}[1]{{\lVert #1 \rVert}}
\newcommand{\bigdlen}[1]{{\bigl\lVert #1 \bigr\rVert}}
\newcommand{\biggdlen}[1]{{\left\lVert #1 \right\rVert}}
\newcommand{\occ}[2]{\len{#2}_{#1}}
\newcommand{\bigocc}[2]{\biglen{#2}_{#1}}
\newcommand{\biggocc}[2]{\bigglen{#2}_{#1}}
\newcommand{\prj}[2]{\mspace{2mu}\downarrow_{#1}\mspace{-5mu}{#2}}

% relations
\newcommand{\trans}[2][]{\raisebox{-1pt}[10pt][0pt]{$\overset{#2}{\underset{^{#1}}{\raisebox{0pt}[3pt][0pt]{$\relbar\mspace{-8mu}\rightarrow$}}}$}}
\newcommand{\ftrans}[2][]{\raisebox{-1pt}[10pt][0pt]{$\overset{#2}{\underset{^{#1}}{\raisebox{0pt}[3pt][0pt]{$\relbar\mspace{-8mu}\circledcirc\mspace{-7.5mu}\rightarrow$}}}$}}
\newcommand{\noftrans}[2][]{\raisebox{-1pt}[10pt][0pt]{$\overset{#2}{\underset{^{#1}}{\raisebox{0pt}[3pt][0pt]{$\relbar\mspace{-8mu}\otimes\mspace{-7.5mu}\rightarrow$}}}$}}
\newcommand{\leads}[1]{\hookrightarrow^{#1}}
\newcommand{\leadsconst}[2]{\hookrightarrow^{#1}_{^{(#2)}}}
\newcommand{\groupsinto}{\trianglelefteq}
\newcommand{\finerthan}{\preceq}
\newcommand{\subgran}{\sqsubseteq}
\newcommand{\border}{\dashv}
\newcommand{\maxborder}{\dashv\!\!\dashv}
\newcommand{\notborder}{\not\dashv}
\newcommand{\notmaxborder}{\not\maxborder}
\newcommand{\dsim}{\approx}
\newcommand{\quotient}[1]{/\mspace{-4mu}#1}
\newcommand{\entails}{\rightvdash}

% operators
\DeclareMathOperator{\lcm}{lcm}
\newcommand{\dom}{{\cD\mit{om}}}
\newcommand{\img}{{\cI\mspace{-2mu}\mit{mg}}}
\newcommand{\fr}{{\cF\mspace{-2mu}\mit{r}}}
\newcommand{\bch}{{\cB\mspace{-2mu}\mit{ch}}}
\newcommand{\infocc}{{\cI\mspace{-2mu}\mit{nf}}}
\newcommand{\unf}{\cU\mspace{-2mu}\mit{nf}}
\newcommand{\nop}{\mathtt{nop}}
\renewcommand{\b}[2]{\bigl(\begin{smallmatrix} #1 \\ #2 \end{smallmatrix}\bigr)}
\newcommand{\bb}[2]{\text{\small $\bigl(\begin{smallmatrix} #1 \\ #2 \end{smallmatrix}\bigr)$}}
\newcommand{\bbb}[2]{\text{\scriptsize $\bigl(\begin{smallmatrix} #1 \\ #2 \end{smallmatrix}\bigr)$}}
\renewcommand{\t}[3]{\biggl(\begin{smallmatrix} #1 \\ #2 \\ #3 \end{smallmatrix}\biggr)}
\renewcommand{\tt}[3]{\text{\small $\biggl(\begin{smallmatrix} #1 \\ #2 \\ #3 \end{smallmatrix}\biggr)$}}
\newcommand{\ttt}[3]{\text{\scriptsize $\biggl(\begin{smallmatrix} #1 \\ #2 \\ #3 \end{smallmatrix}\biggr)$}}

% complexity
\newcommand{\lowerbound}[1]{\Omega(#1)}
\newcommand{\biglowerbound}[1]{\Omega\bigl(#1\bigr)}
\newcommand{\bigglowerbound}[1]{\Omega\left(#1\right)}
\newcommand{\upperbound}[1]{\cO(#1)}
\newcommand{\bigupperbound}[1]{\cO\bigl(#1\bigr)}
\newcommand{\biggupperbound}[1]{\cO\left(#1\right)}
\newcommand{\comparable}[1]{\Theta(#1)}
\newcommand{\bigcomparable}[1]{\Theta\bigl(#1\bigr)}
\newcommand{\biggcomparable}[1]{\Theta\left(#1\right)}
%%% Local Variables: 
%%% mode: latex
%%% TeX-master: "main"
%%% End: 


\def\dotminus{\mathbin{\ooalign{\hss\raise1ex\hbox{.}\hss\cr
			\mathsurround=0pt$-$}}}

\newcommand\loge[1]{\Sigma_{#1}} %Existential logic
\newcommand\logu[1]{\Pi_{#1}} %Universal logic
\newcommand\logex[1]{\Sigma_{#1}\textsc{[FO]}} %Existential extended logic
\newcommand\logux[1]{\Pi_{#1}\textsc{[FO]}} %Universal extended logic
\newcommand\logeh[1]{\Sigma_{#1}\textsc{[FO]-Horn}} %Existential extended Horn logic
\newcommand\loguh[1]{\Pi_{#1}\textsc{[FO]-Horn}} %Universal extended Horn logic
\newcommand\ehorn{\Sigma_2\textsc{-Horn}} 
\newcommand\uhorn{\Pi_1\textsc{-Horn}} 
\newcommand\E[1]{\#\Sigma_{#1}} %Existential
\newcommand\U[1]{\#\Pi_{#1}} %Universal
\newcommand\sfo{\#\fo} %#FO
\newcommand\seso{\text{\sc \#($\eso$)}} %#FO
%\newcommand\sh[1]{\text{\sc \#} #1} %Universal
\newcommand\sh[1]{\##1} %Universal
\newcommand\QE[1]{\eqso(\Sigma_{#1})} %Existential
\newcommand\QU[1]{\eqso(\Pi_{#1})} %Universal
\newcommand\XE[1]{\#\Sigma_{#1}\textsc{[FO]}} %Extended Existential
\newcommand\XU[1]{\#\Pi_{#1}\textsc{[FO]}} %Extended Universal
\newcommand\HE[1]{\#\Sigma_{#1}\textsc{[FO]-Horn}} %Horn Existential
\newcommand\HU[1]{\#\Pi_{#1}\textsc{[FO]-Horn}} %Horn Universal

\def\dhsat{\textsc{DisjHornSAT}}
\def\shdhsat{\textsc{\#DisjHornSAT}}
\def\cpm{\textsc{\#PerfectMatching}}
\def\chsat{\textsc{\#HornSAT}}
\def\cdnf{\textsc{\#DNF}}
\def\ctdnf{\textsc{\#3-DNF}}
\def\ccnf{\textsc{\#CNF}}
\def\ctcnf{\textsc{\#3-CNF}}
\def\ctwcnf{\textsc{\#2-CNF}}
\def\csp{\textsc{\#SimplePath}}
\def\csat{\textsc{\#SAT}}

%\def\A{{\frak A}}
\def\B{{\frak B}}
\def\C{{\cal C}}
\def\F{{\cal F}}
\def\L{{\cal L}}
\def\cG{{\cal G}}
\def\N{\mathbb{N}}
\def\P{\bar{P}}
\def\Q{\bar{Q}}
%\def\R{\bar{R}}
\def\S{\bar{S}}
\def\X{\bar{X}}
\def\Y{\bar{Y}}
\def\Z{\bar{Z}}
%% a - arity of \X / arity of assignments \P to \X
\def\a{\bar{a}}
%% b - arity of predicates in \S
\def\b{\bar{b}}
%% c - arity of auxiliar predicates/variables
\def\c{\bar{c}} %% super auxiliar elements
\def\d{\bar{d}} %% counted elements
\def\e{\bar{e}} %% counted elements
%% f - counting function
%% g - other functions
%% h - other functions
%% i - index
%% j - index
%% k - emergency index / size of tuple
%% l - emergency index / size of tuple
\def\l{\bar{\ell}}
%% m - size of variable tuple
%% n - size of predicate tuple
%% o - not used
\def\p{\bar{p}}
%% q - 
%% r - size of \X / \P
\def\s{\bar{s}}
%% t - size of \S
\def\t{\bar{t}}
\def\u{\bar{u}} %% auxiliary variables
\def\v{\bar{v}} %% auxiliary variables
\def\w{\bar{w}} %% auxiliary variables
\def\x{\bar{x}} %% quantified variables
\def\y{\bar{y}} %% auxiliary variables
\def\z{\bar{z}} %% open variables
\def\ep{\bar{o}}
\def\ga{\bar{p}}





% commands

\newcommand{\cristian}[1]{\todo[inline, color=blue!10]{{\bf Cristian:} #1}}
\newcommand{\marcelo}[1]{\todo[inline, color=red!20]{{\bf Marcelo:} #1}}
\newcommand{\martin}[1]{\todo[inline, color=green!20]{{\bf Martin:} #1}}


%logic
\newcommand{\fo}{{\rm FO}}
\newcommand{\so}{{\rm SO}}
\newcommand{\lfp}{{\rm LFP}}
\newcommand{\lfpop}{{\bf lfp} \,\, }
\newcommand{\alfp}{{\bf alfp} \,\, }
\newcommand{\clfp}[1]{[{\bf lsfp} \, #1]}
\newcommand{\fqfo}{{\rm FQFO}}
\newcommand{\fqso}{{\rm FQSO}}
\newcommand{\pth}{{\bf path} \,\, }
\newcommand{\tc}{{\rm TC}}
\newcommand{\dtc}{{\rm DTC}}
\newcommand{\pfp}{{\rm PFP}}
\newcommand{\eso}{\exists\so}
\newcommand{\first}{\operatorname{first}}
\newcommand{\last}{\operatorname{last}}
\newcommand{\succesor}{\operatorname{succ}}
\newcommand{\partition}{\operatorname{partition}}

\newcommand{\R}{\mathbf{R}}
\newcommand{\T}{\mathcal{T}}
\newcommand{\A}{\mathfrak{A}}
\newcommand{\G}{\mathbf{G}}
\newcommand{\all}{\text{\sc All}}
\newcommand{\allo}{\text{\sc AllOrd}}
\newcommand{\qso}{{\rm QSO}}
\newcommand{\qsoz}{\qso_{\bbZ}}
\newcommand{\rqfo}{{\rm RQFO}}
\newcommand{\tqfo}{{\rm TQFO}}
\newcommand{\tqso}{{\rm TQSO}}
\newcommand{\tqsos}{{\rm TQSO}_{\rm succ}}
\newcommand{\qfo}{{\rm QFO}}
\newcommand{\qfoz}{\qfo_{\bbZ}}
\newcommand{\eqfo}{\Sigma\qfo}
\newcommand{\eqso}{\Sigma\qso}
\newcommand{\eqsoz}{\eqso_{\bbZ}}
\newcommand{\sqso}{\text{\rm Saluja}\qso}
\newcommand{\fv}{\mathbf{FV}}
\newcommand{\sv}{\mathbf{SV}}
\newcommand{\fs}{\mathbf{FS}}
\newcommand{\arity}{{\rm arity}}
\newcommand{\length}{\ell}
\newcommand{\shp}{\text{\sc \#P}}
\newcommand{\ptime}{\text{\sc P}}
\newcommand{\np}{\text{\sc NP}}
\newcommand{\bpp}{\text{\sc BPP}}
\newcommand{\cspp}{\text{\sc SPP}}
\newcommand{\pp}{\text{\sc PP}}
\newcommand{\rp}{\text{\sc RP}}
\newcommand{\pspace}{\text{\sc PSPACE}}
\newcommand{\nlog}{\text{\sc NL}}
\newcommand{\conp}{\text{\sc NP}}
\newcommand{\pe}{\text{\sc \#PE}}
\newcommand{\shl}{\text{\sc \#L}}
\newcommand{\spp}{\text{\sc span-P}}
\newcommand{\gp}{\text{\sc gap-P}}
\newcommand{\optp}{\text{\sc opt-P}}
\newcommand{\fp}{\text{\sc FP}}
\newcommand{\totp}{\text{\sc TotP}}
\newcommand{\shpspace}{\text{\sc \#PSPACE}}
\newcommand{\fpspace}{\text{\sc FPSPACE}}
\newcommand{\nfpspace}{\text{\sc PSPACE(poly)}}
\newcommand{\acc}{\textbf{acc}}

\newcommand{\CC}{\mathscr{C}}
\newcommand{\KK}{\mathscr{K}}
\newcommand{\FF}{\mathscr{F}}
\newcommand{\GG}{\mathscr{G}}
\newcommand{\LL}{\mathscr{L}}
\newcommand{\QQ}{\mathscr{Q}}
\newcommand{\enc}{{\rm enc}}
\newcommand{\str}{\text{\sc Struct}}
\newcommand{\ostr}{\text{\sc OrdStruct}}
\newcommand{\Func}{\text{\sc Func}}
\newcommand{\res}[2]{#1|_{#2}}

%semiring
\newcommand{\nat}{\mathbb{N}}
\newcommand{\natinf}{\mathbb{N}_\infty}
\newcommand{\trop}{\mathbb{N}_{\min,+}}
\newcommand{\integ}{\mathbb{Z}}
\newcommand{\bln}{\mathbb{B}}
\newcommand{\pwset}[1]{2^{#1}}
\newcommand{\true}{\operatorname{true}}
\newcommand{\false}{\operatorname{false}}

\newcommand{\SR}{\bbS}
\newcommand{\add}{+}
\newcommand{\bigadd}{\sum}
\newcommand{\mult}{\cdot}
\newcommand{\bigmult}{\prod}
\newcommand{\adds}{\oplus}
\newcommand{\bigadds}{\bigoplus}
\newcommand{\mults}{\odot}
\newcommand{\bigmults}{\bigodot}
\newcommand{\zero}{\mathbb{0}}
\newcommand{\one}{\mathbb{1}}

%quantitative logic
\newcommand{\QL}{\operatorname{QL}}
\newcommand{\QMSO}{\operatorname{QMSO}}
\newcommand{\Op}{\operatorname{O}}
\newcommand{\sem}[1]{{\llbracket{}{#1}\rrbracket}}
\newcommand{\pa}[1]{\Pi{#1}.\,}
\newcommand{\pas}{\Pi}
\newcommand{\paq}[1]{\Pi{#1}}
\newcommand{\sa}[1]{\Sigma{#1}.\,}
\newcommand{\sas}{\Sigma}
\newcommand{\saq}[1]{\Sigma{#1}}
\newcommand{\fpa}[1]{\overrightarrow{\prod}{#1}.\:}
\newcommand{\lmid}{\;\mid\;}

% equations and quotes skip
\abovedisplayskip=6pt 
\belowdisplayskip=6pt
\newenvironment{myquote}{\begin{quote}\vspace{-0.75mm}}{\end{quote}\vspace{-0.75mm}}

%tikz definition
\tikzset{
	defaultstyle/.style={>=stealth,semithick, auto,font=\small,
		initial text= {},
		initial distance= {3.5mm},
		accepting distance= {3.5mm}},
	accepting/.style=accepting by arrow,
	nstate/.style={circle, semithick,inner sep=1pt, minimum size=4mm}}

%Turing machine
\newcommand{\tma}{\text{\rm accept}}
\newcommand{\tmr}{\text{\rm reject}}
\newcommand{\tmg}{\text{\rm gap}}
\newcommand{\tmt}{\text{\rm total}}

\newcommand{\support}{\text{\rm support}}



\begin{document}

\title{Descriptive complexity \\
	for counting complexity classes}


% author names and affiliations
% use a multiple column layout for up to three different
% affiliations
\author{\IEEEauthorblockN{Marcelo Arenas}
\IEEEauthorblockA{%Department of Computer Science\\
PUC Chile\\ %Pontificia Universidad Cat\'olica de Chile \\
marenas@ing.puc.cl}
\and
\IEEEauthorblockN{Martin Mu\~noz}
\IEEEauthorblockA{%Department of Computer Science\\
	PUC Chile\\ %Pontificia Universidad Cat\'olica de Chile \\
	mmunos@uc.cl}
\and
\IEEEauthorblockN{Cristian Riveros}
\IEEEauthorblockA{%Department of Computer Science\\
	PUC Chile\\ %Pontificia Universidad Cat\'olica de Chile \\
	cristian.riveros@uc.cl}}

\maketitle


\begin{abstract}
The abstract goes here.
\end{abstract}

\IEEEpeerreviewmaketitle

\section{Introduction}
%!TEX root = main.tex

%\marcelo{Enfatizar el rol fundamental de la logica para obtener los resultados, por ejemplo para obtener cerrado bajo menos uno}
%
%\marcelo{Poner un comentario sobre funciones totales versus parciales, dado que estamos considerando clases con funciones totales}
%
%\marcelo{Enfatizar el rol fundamental de la logica para obtener los resultados, por ejemplo para obtener cerrado bajo menos uno}
%
%Strategy:
%\begin{enumerate}
%	\item Descriptive complexity and application.
%	
%	\item Counting complexity classes. 
%	
%	\item Our contribution in terms of logic.
%	
%	\item sharP and its structure. 
%	
%	\item Syntactic classes with good properties. 
%\end{enumerate}
%
%\cristian{Aqui empieza la intro.}

The goal of descriptive complexity is to measure the complexity of a problem in terms of the logical constructors needed to express it~\cite{immerman1999descriptive}. 
The starting point of this branch of complexity theory is Fagin's theorem, which states that $\np$ is equal to existential second-order logic. Since then, many more complexity classes have been characterized in terms of logics (see \cite{G07} for a survey) and descriptive complexity has found a variety of applications in different areas~\cite{immerman1999descriptive, L04}.
For instance, Fagin's theorem was the key ingredient to define the class {\sc MaxSNP}~\cite{PY91}, which was later shown to be a fundamental class in the study of hardness of approximation \cite{ALMSS98}. 
It is important to mention here that the definition of {\sc MaxSNP} would not have been possible without the machine-independent point of view of descriptive complexity, as pointed out in~\cite{PY91}.

Counting problems differ from decision problems in that the value of a function has to be computed.
More generally, a counting problem corresponds to compute a function $f$ from a set of instances (e.g. graphs, formulae, etc) to natural numbers.\footnote{This value is usually associated to counting the number of solutions
	%for a given instance 
	in a search problem, but here we consider a more general definition.} 
The study of counting problems has given rise to a rich theory of counting complexity classes \cite{HV95,F97,arora2009computational}. Some of these classes are natural counterparts of some classes of decision problems; for example, $\fp$ 
%(resp., $\fpspace$) 
is the class of all functions that can be computed in polynomial time, 
%(resp., polynomial space), 
the natural counterpart of $\ptime$.
% ($\pspace$ resp.). 
However, other function complexity classes have emerged from the need to understand the complexity of some computation problems for which little can be said if their decision counterparts are considered. This is the case of the class $\shp$, a counting complexity class introduced in \cite{Valiant79} to prove that natural problems like counting the number of satisfying assignments of a propositional formula or the number of perfect matchings of a bipartite graph~\cite{Valiant79} are difficult, namely, $\shp$-complete.
Starting from $\shp$,
many more natural 
%the zoo of 
counting complexity classes have been defined, such as 
%was open with other natural counting classes like 
$\shl$, $\spp$ and $\gp$~\cite{HV95,F97}.
%among others~\cite{HV95,F97}.

Although counting problems play a prominent role in computational complexity, descriptive complexity for this type of problems has not been systematically studied and it is not as developed as for the case of decision problems. Insightful characterizations of $\shp$ and some of its extensions have been provided \cite{SalujaST95,ComptonG96}. However, these characterizations do not define function problems in terms of a logic, but instead in terms of some counting problems associated to a logic like $\fo$. Thus, it is not clear how these characterizations can be used to provide a general descriptive complexity framework for counting complexity classes like $\fp$ and $\fpspace$ (the class of functions computable in polynomial space). 
%It should be mentioned that logical definability has also been studied for the case of optimization problems \cite{KT94} and computation over the real numbers \cite{GM95}. As for the previous cases, it is not clear how these approaches can be extended to provide logical characterizations of a variety of function complexity classes. 

In this paper, we propose to study the descriptive complexity of counting complexity classes in terms of Weighted Logics (WL)~\cite{DrosteG07}, a general logical framework that combines Boolean formulae (e.g. in $\fo$ or $\so$) with operations over a fixed semiring (e.g. $\bbN$). 
Specifically, we propose a restriction of WL over natural numbers, called Quantitative Second Order Logics (QSO), and study its expressive power for defining counting complexity classes over ordered structures. 
As a proof of concept, we show that natural syntactical fragments of $\qso$ captures counting complexity classes like $\shp$, $\spp$, $\fp$ and $\fpspace$.
Furthermore, by slightly extending the framework we can prove that $\qso$ can also capture classes like $\gp$ and $\optp$, showing the robustness of our approach.

The next step is to use the machine-independent point of view of $\qso$ to search for subclasses of $\shp$ with some fundamental properties.
%inside $\shp$. 
The question here is, what properties are desirable for a subclass of $\shp$?
First, it is desirable to have a class of counting problems whose associated decision versions are tractable, in the sense that one can decide in $\ptime$ whether the value of the function is greater than $0$. 
In fact, this requirement is crucial in order to have any chance of finding efficient approximation algorithms for a given function (see Section~\ref{sec:syntactic}).
Second, we expect that our class is closed under basic arithmetical operations like sum, multiplication and subtraction by one. 
This is a common topic for counting complexity classes; for example, it is known that $\shp$ is not closed under subtraction by one (under some complexity-theoretical assumption). 
Finally, we want a class with natural complete problems, which characterize all problems in it.
%from the complexity point of view.

In this paper, we give the first results towards defining subclasses of $\shp$ that are robust in terms of approximation, closure properties, and natural complete problems. 
Specifically, we introduce a syntactic hierarchy inside $\shp$, called $\eqso(\fo)$-hierarchy, and we show that it is closely related to the $\fo$-hierarchy introduced in~\cite{SalujaST95}. 
Looking inside the $\eqso(\fo)$-hierarchy, we propose the class $\eqso(\logex{1})$ and show that every function in it has a tractable associated decision version, and it is closed under sum, multiplication, and subtraction by one.
%minus one. 
Unfortunately, it is not clear whether this class admits 
%we cannot show a
a natural complete problem.
% for $\eqso(\logex{1})$.
Thus, 
%Despite of this 
we also introduce a Horn-style syntactic class, inspired by the approach in~\cite{G92}, that has tractable associated decision versions and a natural complete problem.

After studying the 
%internal 
structure of $\shp$, we move beyond $\qso$ by introducing new quantifiers. 
By adding functional variables in top of $\qso$, we introduce a quantitative least fixed point operator to the logic. 
Adding finite recursion to a numerical setting is subtle since functions over natural numbers can easily diverge without finding any fixed point. 
By using the support of the functions, we give a natural halting condition that generalizes the least fixed point operator of Boolean logics.
\martin{creo que la noción de {\em support} no es muy conocida. Yo cambiaría "By using the support of the functions" por algo que no lo d\'e por sabido como "By using the {\em support} operator over functions"}
Then, with a quantitative recursion at hand we show how to capture $\fp$ from a different perspective and, moreover, how to restrict recursion to capture lower complexity classes 
%below $\fp$ 
such as~$\shl$, the counting version of $\nlog$.
%(Nondeterministic Logarithmic-space).

\smallskip

\noindent{\bf Organisation.} The main terminology used in the paper is given in Section~\ref{sec:preliminaries}. Then the logical framework is introduced in Section~\ref{sec:logic}, and it is used to capture standard counting complexity classes in Section~\ref{sec:complexity}. The structure of $\shp$ is studied in Section~\ref{sec:syntactic}. Section~\ref{sec:beyond} is devoted to define recursion in $\qso$, and to show how to capture classes below $\fp$. 
Finally, we give some concluding remarks in Section~\ref{sec:conclusions}. 

\section{Preliminaries} \label{sec:preliminaries}
%!TEX root = main.tex

%In this section, 
%We introduce here the main terminology used in the paper.

\subsection{Second-order logic, LFP and PFP}
A relational signature $\R$ (or just signature) is a finite set $\{R_1, \ldots, R_k\}$, where each $R_i$ ($1 \leq i \leq k$) is a relation name with an associated arity greater than 0, which is denoted by $\arity(R_i)$. A finite structure over $\R$ (or just finite $\R$-structure) is a tuple $\A = \langle A, R_1^\A, \ldots, R_k^\A \rangle$ such that $A$ is a finite set and $R_i^\A \subseteq A^{\arity(R_i)}$ for every $i \in \{1, \ldots, k\}$. Further, an $\R$-structure $\A$ is said to be ordered if $<$ is a binary predicate name in $\R$ and $<^\A$ is a linear order on $A$.
We denote by $\ostr[\R]$ the class of all finite ordered $\R$-structures. 
In this paper we only consider finite ordered structures, so we will usually omit the words finite and ordered when referring to them.

From now on, assume given disjoint infinite sets $\fv$ and $\sv$ of first-order variables and second-order variables, respectively. Notice that every variable in $\sv$ has an associated arity, which is denoted by $\arity(X)$. Then given a  signature $\R$, the set of second-order logic formulae ($\so$-formulae) over $\R$ is given by the following grammar:
\begin{align*}\ 
	\varphi \ &:= \ x = y \ \mid \ R(\bar u) \ \mid \ \top  \ \mid\  
	X(\bar v)  \ \mid
	\neg \varphi \ \mid\ 
	(\varphi \vee \varphi) \ \mid\ 
	\ex{x} \varphi \ \mid\ 
	\ex{X} \varphi
 \end{align*}
where $x,y \in \fv$, $R \in \R$, $\bar u$ is a tuple of (not necessarily distinct) variables from $\fv$ whose length is $\arity(R)$, $\top$ is a reserved symbol to represent a tautology, $X \in \sv$, $\bar v$ is a tuple of (not necessarily distinct) variables from $\fv$ whose length is $\arity(X)$, and $x \in \fv$. 

%\marcelo{Tenemos una definicion muy detallada de $\fo$ y $\so$, no es necesario para esta conferencia. Por otro lado seria bueno mencionar las definiciones de $\Sigma_i$ y $\Pi_i$, aunque son estandar es bueno decir que estamos considerando restricciones de $\fo$.}

%\martin{Decir que $\LL$ incluye tautología y decir que Saluja no lo menciona}

We assume that the reader is familiar with the semantics of $\so$, so we only introduce here some notation that will be used in this paper. 
%To define the semantics of $\so$, we need to introduce some terminology. 
Given a signature $\R$ and an $\R$-structure $\A$ with domain $A$, a first-order assignment $v$ for $\A$ is a total function from $\fv$ to $A$, while a second-order assignment $V$ for $\A$ is a total function with domain $\sv$ that maps each $X \in \sv$ to a subset of $A^{\arity(X)}$. Moreover, given a first-order assignment $v$ for $\A$, $x \in \fv$ and $a \in A$, we denote by $v[a/x]$ a first-order assignment such that $v[a/x](x) = a$ and $v[a/x](y) = v(y)$ for every $y \in \fv$ distinct from $x$. Similarly, given a second-order assignment $V$ for $\A$, $X \in \sv$ and $B  \subseteq A^{\arity(X)}$, we denote by $V[B/X]$ a second-order assignment such that $V[B/X](X) = B$ and $V[B/X](Y) = V(Y)$ for every $Y \in \sv$ distinct from $X$. We use notation $(\A, v, V) \models \varphi$ to indicate that structure $\A$ satisfies $\varphi$ under $v$ and $V$.
% In particular, we have that $(\A, v, V) \models \top$.
%\martin{En vez de decir "for every $y\in \fv$ distinct from $x$" podría ser "for every $y\in\fv\setminus\{x\}$", y lo mismo con $Y\in\sv$.}

In this paper, we consider several fragments or extensions of $\so$ like first-order logic~($\fo$), least fixed point logic (LFP) and partial fixed point logic (PFP) \cite{L04}. Moreover, for every $i \in \N$, we consider the fragment $\Sigma_i$ (resp., $\Pi_i$) of $\fo$, which is the set of $\fo$-formulae of the form 
$\exists \bar x_1 \forall \bar x_2 \cdots \exists \bar x_{i-1} \forall \bar x_{i} \, \psi$ (resp., 
$\forall \bar x_1 \exists \bar x_2 \cdots \forall \bar x_{i-1} \exists \bar x_{i} \, \psi$) if $i$ is even, and of the form
$\exists \bar x_1 \forall \bar x_2 \cdots \forall \bar x_{i-1} \exists \bar x_{i} \, \psi$ (resp., 
$\forall \bar x_1 \exists \bar x_2 \cdots \exists \bar x_{i-1} \forall \bar x_{i} \, \psi$) if $i$ is odd, where $\psi$ is a quantifier-free formula. Finally, we say that a fragment $\L_1$ is contained in a fragment $\L_2$, denoted by $\L_1 \subseteq \L_2$, if for every formula $\varphi$ in $\L_1$, there exists a formula $\psi$ in $\L_2$ such that $\varphi$ is logically equivalent to $\psi$.  Besides, we say that $\L_1$ is properly contained in $\L_2$, denoted by $\L_1 \subsetneq \L_2$, if $\L_1 \subseteq \L_2$ and $\L_2 \not\subseteq \L_1$.


%Assume that $\varphi$ is an $\so$-formula over a signature $\R$. Then given an ordered finite $\R$-structure $\A$ with domain $A$, a first-order assignment $v$ for $\A$ and a second-order assignment $V$ for $\A$, we say that $(\A, v, V)$ satisfies $\varphi$, denoted by $(\A, v, V) \models \varphi$, if: (1) $\varphi$ is the formula $R(x_1, \ldots, x_\ell)$ and $(v(x_1), \ldots, v(x_\ell)) \in R^\A$; (2) $\varphi$ is the formula $X(x_1, \ldots, x_m)$ and $(v(x_1), \ldots, v(x_m)) \in V(X)$; (3) $\varphi$ is the formula $\neg \psi$ and $(\A, v, V)$ does not satisfy $\psi$; (4) $\varphi$ is the formula $(\varphi_1 \vee \varphi_2)$, and $(\A, v, V) \models \varphi_1$ or $(\A, v, V) \models \varphi_2$; (5) $\varphi$ is the formula $\exists x \, \psi$ and there exists $a \in A$ such that $(\A, v[a/x], V) \models \psi$; or (6) $\varphi$ is the formula $\exists X \, \psi$ and there exists $B \subseteq A^{\arity(X)}$ such that $(\A, v, V[B/X]) \models \psi$.  As usual, we consider the propositional operators $\wedge$, $\rightarrow$, and $\leftrightarrow$ that can be obtained from $\vee$ and $\neg$. 
%Moreover, we use the abbreviations $x \not \leq y$ and $x \notin X$ for the negation of the atoms $\leq$ and $\in$. 
%Finally, we consider standard abbreviations of formulae that can be defined in $\so$-logic (actually, $\fo$-logic) like $\first(x) := \fa{x} y \leq x$ and $\last(x) := \fa{x} x \leq y$ to denote the first and last element of the linear order $\leq$, respectively, and $\succesor(x,y) := x \leq y \wedge y \not\leq x \wedge \fa{z} ( z \leq x \vee y \leq z)$ to denote the successor relation.

%\cristian{Agregar que la igualdad $=$ esta en nuestra lógica.}

%\cristian{Agregar acá que usamos las abreviaciones de $<$ o $\subset$.}

%\marcelo{Agregar las definiciones de LFP y PFP}

%\cristian{Explicar acá que significa el containment de lógicas booleans, esto es, la notación $\U{1} \subseteq \LL$.}

\subsection{Counting complexity classes}

We consider several counting complexity classes in this paper, some of them are recalled here~(see \cite{F97,hemaspaandra2013complexity}).
%We define the following function complexity classes: 
$\fp$ is the class of functions $f : \Sigma^* \to \N$ computable in polynomial time, while $\fpspace$ is the class of functions $f : \Sigma^* \to \N$ computable in polynomial space. 
%Moreover, $\nfpspace$ is the class of functions computable in polynomial space with output length bounded by a polynomial. 
Given a nondeterministic Turing Machine (NTM) $M$, let $\tma_M(x)$ be the number of accepting runs of $M$ with input $x$. Then $\shp$ is the class of functions $f$ for which there exists a polynomial-time NTM $M$ such that $f(x) = \tma_M(x)$ for every input $x$, while $\shl$ is the class of functions $f$ for which there exists a logarithmic-space NTM $M$ such that $f(x) = \tma_M(x)$ for every input $x$.  Given an NTM $M$ with output tape, let $\tmo_M(x)$ be the number of distinct outputs of $M$ with input $x$ (notice that  $M$ produces an output if it halts in an accepting state). Then $\spp$ is the  class of functions $f$ for which there exists a polynomial-time NTM $M$ such that $f(x) = \tmo_M(x)$ for every input $x$. Notice that $\shp \subseteq \spp$, and this inclusion is believed to be strict. 

%as the class of functions that on an input are equal to the number of different outputs of an $\np$ transducer on that input. . $\shpspace$ is defined analogously to $\shp$ but in polynomial space. $\nfpspace$ is the class of single-exponentially bounded functions computable in polynomial space. 

%\marcelo{La notacion $\tma_M(x)$ se usa en el paper, hay que definirla aqui. Donde vamos a usar la clase $\gp$? Para definir esta clase tambien necesitamos la notacion $\tmr_M(x)$, podriamos definirla aqui. }

%If $\KK$ is a class of finite structures and $f$ is a function from a signature $\R$ to $\bbD$, we denote by $\res{f}{\KK}$ the restriction of $f$ to $\KK$, that is, a function such that: the domain of $\res{f}{\KK}$ is $\{ \enc(\A) \mid \A \in \str[\R] \cap \KK\}$, and 
%$\res{f}{\KK}(\enc(\A)) = f(\enc(\A))$ for every $\A \in \str[\R] \cap \KK$.


\section{A logic for quantitative functions} \label{sec:logic}
%!TEX root = main.tex

We introduce here the logic framework that we use for studying function complexity classes. 
This logic framework is based on the framework of Weighted Logics~\cite{DrosteG07} (WL) that has been used in the context of weighted automata for studying functions from words (or trees) to semirings. 
We propose here to use the framework of WL over any relational structure and to restrict the semiring to natural numbers. 
The extension to any relational structure will allow us to study general function complexity classes and the restriction to the natural numbers will simplify the notation in this context (see Section~\ref{sec:previous} below for a more detailed discussion).

Given a relational signature $\R$, the set of Quantitative Second-Order logic formulas (or just $\qso$-formulas) over $\R$ is given by the following grammar:
%\[
%\begin{array}{rcl}
%\alpha & := & \varphi \ \mid \ s \ \mid \ (\alpha \add \alpha) \ \mid\ (\alpha \mult \alpha) \ \mid \ \\
%& &  \sa{x} \alpha \ \mid \pa{x} \alpha \ \mid \ \sa{X} \alpha \ \mid \ \pa{X} \alpha 
%\end{array}
%\]
\begin{multline*}
\alpha := \varphi \ \mid \ s \ \mid \ (\alpha \add \alpha) \ \mid\ (\alpha \mult \alpha) \ \mid \\ \sa{x} \alpha \ \mid \ \pa{x} \alpha \ \mid \ \sa{X} \alpha \ \mid \ \pa{X} \alpha 
\end{multline*}
where $\varphi$ is an $\so$-formula over $\R$, $s \in \bbN$, $x \in \fv$ and $X \in \sv$. If $\R$ is not mentioned, $\qso$ refers to the union of the sets of $\qso$ formulas over $\R$, for every relational signature~$\R$.
\marcelo{En la gramatica de la logica deberiamos incluir la formula $\top$ que discutimos, la cual representa true y para la cual se tiene que $\sem{\top}(\A, v, V) = 1$. Les parece?}
 
Note that the syntax of QSO formulas is divided in two levels. 
The first level is composed by $\so$-formulas over $\R$ (called boolean formulas) and the second level is made by counting operators of addition and multiplication. 
For this reason, the quantifiers in $\so$ (e.g. $\exists x$ or $\exists X$) are called boolean quantifiers and the quantifiers that make use of addition and multiplication (e.g. $\Sigma x$ or $\Pi X$) are called {\em quantitative quantifiers}.
Furthermore, $\Sigma x$ and $\Sigma X$ are called first- and second-order sum, and $\Pi x$ and $\Pi X$ the first- and second-order product, respectively.
This division between boolean and quantitative levels is essential for understanding the difference between the logic and the quantitative part. 
Furthermore, this will allow us later to parametrize both levels of the logic in order to capture different function complexity classes.

Let $\R$ be a relational signature, $\A$ be an ordered finite $\R$-structure with domain $A$, $v$ a first-order assignment for $\A$ and $V$ a second-order assignment for $\A$. Then the \emph{evaluation} of a $\qso$-formula $\alpha$ over $(\A, v, V)$ is defined as a function $\sem{\alpha}$ that on input $(\A, v, V)$ returns a number in $\bbN$. Formally, the function $\sem{\alpha}$ is recursively defined in Table~\ref{tab-semantics}.
\begin{table}
	\addtolength{\jot}{0.5em}
	\begin{align*}
	\sem{\varphi}(\A, v, V) & = 
	\begin{cases}
	1 & \mbox{if } (\A, v, V) \models \varphi \\
	0 & \mbox{otherwise}
	\end{cases}\\
	\sem{s}(\A, v, V) & = s \\
	\sem{\alpha_1 \add \alpha_2}(\A, v, V) & = \sem{\alpha_1}(\A, v, V) + \sem{\alpha_2}(\A, v, V)\\
	\sem{\alpha_1 \mult \alpha_2}(\A, v, V) & = \sem{\alpha_1}(\A, v, V) \cdot \sem{\alpha_2}(\A, v, V)\\ 
	\sem{\sa{x} \alpha}(\A, v, V) & = \displaystyle \sum_{a \in A} \sem{\alpha}(\A,v[a/x],V)\\
	\sem{\pa{x} \alpha}(\A, v, V) & = \displaystyle \prod_{a \in A} \sem{\alpha}(\A,v[a/x],V)\\
	\sem{\sa{X} \alpha}(\A, v, V) & = \displaystyle \sum_{B \subseteq A^{\arity(X)}} \sem{\alpha}(\A, v, V[B/X])\\
	\sem{\pa{X} \alpha}(\A, v, V) & = \displaystyle \prod_{B \subseteq A^{\arity(X)}} \sem{\alpha}(\A, v, V[B/X])
	\end{align*}
	\caption{The semantics of QSO formulas.}
	\label{tab-semantics}
	\vspace*{-20pt}
\end{table}
%$$
%\renewcommand{\arraystretch}{1.7}
%\begin{array}{rcl} 
%\sem{\varphi}(\A, v, V) & = & 
%\begin{cases}
%1 & \mbox{if } (\A, v, V) \models \varphi \\
%0 & \mbox{otherwise}
%\end{cases}\\
%\sem{s}(\A, v, V) & = & s \\
%\sem{\alpha_1 \add \alpha_2}(\A, v, V) & = & \sem{\alpha_1}(\A, v, V) + \sem{\alpha_2}(\A, v, V)\\
%\sem{\alpha_1 \mult \alpha_2}(\A, v, V) & = & \sem{\alpha_1}(\A, v, V) \cdot \sem{\alpha_2}(\A, v, V)\\ 
%\sem{\sa{x} \alpha}(\A, v, V) & = & \displaystyle \sum_{a \in A} \sem{\alpha}(\A,v[a/x],V)\\
%\sem{\pa{x} \alpha}(\A, v, V) & = & \displaystyle \prod_{a \in A} \sem{\alpha}(\A,v[a/x],V)\\
%\sem{\sa{X} \alpha}(\A, v, V) & = & \displaystyle \sum_{B \subseteq A^{\arity(X)}} \sem{\alpha}(\A, v, V[B/X])\\
%\sem{\pa{X} \alpha}(\A, v, V) & = & \displaystyle \prod_{B \subseteq A^{\arity(X)}} \sem{\alpha}(\A, v, V[B/X])
%\end{array}
%$$
A $\qso$-formula $\alpha$ is said to be a \emph{sentence} if it does not have any free variable, that is, every variable in $\alpha$ is under the scope of a usual quantifier or a quantitative quantifier. It is important to notice that if $\alpha$ is a $\qso$-sentence over a relational signature $\R$, then for every ordered finite $\R$-structure $\A$, first-order assignments $v_1$, $v_2$ for $\A$ and second-order assignments $V_1$, $V_2$ for $\A$, it holds that $\sem{\alpha}(\A, v_1, V_1) = \sem{\alpha}(\A, v_2, V_2)$.
Thus, in such a case we use the term $\sem{\alpha}(\A)$ to denote $\sem{\alpha}(\A, v, V)$, for some arbitrary first-order assignment $v$ for $\A$ and some arbitrary second-order assignment $V$ for $\A$. 
\begin{example}
Let $\bG = \{E(\cdot,\cdot)\}$ be the vocabulary for graphs and $\fG$ be an ordered finite $\bG$-structure encoding a non-directed graphs. 
Suppose that we want to count the number of triangles in $\fG$. Then this can be defined with the following QSO-formula:
\begin{multline*}
\alpha_1 \ := \ \sa{x} \sa{y} \sa{z} ( E(x,y) \, \wedge \, E(y,z) \, \wedge \, E(z,x) \, \wedge \\
x \leq y \, \wedge \, y \leq z )
\end{multline*}
For $\alpha_1$ we encode a triangle as an increasing sequence of nodes $\{x, y, z\}$ in order to count each triangle once. Then the boolean subformula  $E(x,y) \wedge E(y,z) \wedge E(z,x) \wedge
x \leq y \wedge y \leq z$ is checking the triangle property, by outputting $1$ if $\{x, y, z\}$ forms a triangle in $\fG$ and $0$ otherwise.
Finally, the sum quantifiers in $\alpha_1$ aggregates all the values, counting the number of triangles in $\fG$.

Suppose now that we want to count the number of cliques in~$\fG$ (i.e. set of nodes where every pair of nodes is connected). We can easily define this function with the following formula:
$$
\alpha_2 := \sa{X} \big( \, \fa{x} \fa{y} \left(X(x) \wedge X(y)\right) \rightarrow E(x,y) \, \big)  
$$ 
Similar than for $\alpha_1$, in the boolean subformula of $\alpha_2$ we check whether $X$ is a clique and with the sum quantifier we add one for each clique in $\fG$. 
Note here that, in contrast to $\alpha_1$, in $\alpha_2$ we need a second-order quantifier in the quantitative level.
This is according to the inherent complexity of evaluating each formula: $\alpha_1$ defines an $\fp$ function where $\alpha_2$ defines a $\shp$-hard function~\cite{paper-that-shows-that-this-problem-is-sharpP-hard}.
\end{example}
\begin{example}\label{exa-perm}
For a more involved example that includes multiplication, let $\bM = \{M(\cdot,\cdot)\}$ be the vocabulary for matrices where a structure $\fM$ over $\bM$ encodes a 0-1 matrix $A$ where $M(i,j)$ is true if $A_{i,j} = 1$ and $0$ otherwise. 
Suppose now that we want to compute the permanent of an $n$-by-$n$ 0-1 matrix $A$ which is defined by:
$$
\op{perm}(A) \; = \; \sum_{\sigma \in S_n} \prod_{i=1}^n A_{i, \sigma(i)}  
$$
where $S_n$ is the set of all permutations over $\{1, \ldots, n\}$.
The permanent is a relevant matrix function that has found many applications in different areas~\cite{permanent-applications} and it was one of the first function that was shown to be difficult for counting~\cite{Valiant79} (i.e. $\shp$-complete). 
For a binary relation symbol $S$, let $\op{permut}(S)$ be the $\fo$-formula that is true iff $S$ is a permutation, i.e. bijective function (the definition of $\op{permut}(S)$ is straightforward).
Then a $\qso$-formula for defining the permanent of a matrix is as follows:
\[
\alpha_3 := \sa{S} \op{permut}(S) \cdot \pa{x} (\ex{y} S(x,y) \wedge M(x,y))
\]
Intuitively, the subformula $\beta(s) := \pa{x} (\ex{y} S(x,y) \wedge M(x,y))$ calculates the value $\prod_{i=1}^n A_{i, \sigma(i)}$ whenever $S$ encodes a permutation $\sigma$.
Then the subformula $\op{permut}(S) \cdot \beta(S)$ computes $\beta(S)$ when $S$ is a permutation and $0$ otherwise (i.e. $\op{permut}(S)$ behaves like a filter). 
Finally, the second order sum aggregates these values iterating over all binary relations and calculating the permanent of the matrix.
We would like to end this example by highlighting the similarity of $\alpha_3$ with the permanent formula. 
Indeed, an advantage of $\qso$-formulas is that the first- and second-order quantifiers naturally reflect the operations used to define mathematical formulas in practice.
\end{example}

We consider several fragments of $\qso$, which are obtained by restricting the syntax of the boolean formulas or the use of the quantitative quantifiers.
In this direction, we denote by $\qfo$ the fragment of $\qso$ where second-order sum and product are not allowed. 
For example, $\alpha_1$ is in $\qfo$ and $\alpha_2$ is not.
Another interesting fragment are the formulas of $\qso$ where only sum operators and quantifiers are allowed. 
Formally, we denote by $\eqso$ the fragment of $\qso$ where first- and second-order products (i.e. $\pa{x}$ and $\pa{X}$) are not allowed.
For example, $\alpha_1$ and $\alpha_2$ are formulas of $\eqso$ but $\alpha_3$ is not. 
We also consider fragments of $\qso$ by further restricting the boolean core of the logic.
Let $\LL$ be a fragment of $\so$.
We define the quantitative logic $\qso(\LL)$ to be the fragment of $\qso$ obtained by restricting $\varphi$ to be a formula in $\LL$. 
In general, if $\FF$ is a fragment of $\qso$ we define $\FF(\LL)$ to be the fragment of $\FF$ obtained by restricting boolean formulas to $\LL$.
For example, we can define the fragments $\qfo(\fo)$ or $\eqso(\fo)$ which are subfragments of $\qfo$ and $\eqso$ where $\varphi$ is restricted to $\fo$ formulas. 

\marcelo{IMPORTANTE: Tenemos que decir que $\qfo(\fo)$ permite variables libre de segundo orden en las formulas de primer orden. Hay que aclarar esto tambien para PFP y otras logicas.}

In the next section, we variate $\FF$ and $\LL$ to capture several function complexity classes. Before this, we devote the next subsection to show the connection of $\qso$ with previous frameworks for defining functions over relational structures.

\subsection{Previous frameworks for quantitative functions} \label{sec:previous}

In this subsection we discuss previous frameworks proposed in the literature and how they differ to our approach.
We start by discussing the connection between $\qso$ and weighted logics (WL)~\cite{DrosteG07}. 
As it was previously discussed, $\qso$ is a fragment of WL.
The real difference is that we restrict the semiring used in WL to natural numbers in order to study function complexity classes.
Another difference of WL with our approach is that, as far as we know, this is the first paper to study weighted logics over general signature properly to do descriptive complexity of counting classes. 
Previous works on WL usually restrict the signature of the logic to strings, trees, and other specific structures (see \cite{droste2009handbook} for more examples), and did not study the logic over general structures. 
Furthermore, in this paper we propose further extensions for $\qso$ (see Section~\ref{sec:beyond}) which start to differ from previous approaches in WL.

Another approach that resembles $\qso$ are counting logics~\cite{L04,counting-paper-here}. 
Logics with counting operators usually extend $\fo$ with quantifiers that counts how many ways can be satisfied a formula over a second sort (i.e. natural numbers). 
In contrast to our approach, counting operators are usually used for checking boolean properties over structures and not for outputting values (i.e. they do not define a function).
Although there are some papers that uses this operator for counting~\cite{XXX}, they have not used the logic for capturing function complexity classes.


%\begin{theorem} \label{no-mult}
%	Let $\alpha$ be a $\qso$-formula in some fragment of $\qso$. There is a $\qso$ formula $\alpha'$ in the same fragment such that $\alpha'$ does not have any subformula of the form $(\beta_1 \cdot \beta_2)$ and $\sem{\alpha} = \sem{\alpha'}$.
%\end{theorem}

Finally, the work on~\cite{SalujaST95} and \cite{ComptonG96} is of particular interest for our work. 
In~\cite{SalujaST95}, it was proposed to define functions over structures by using free variables on SO-formulas and count how many extensions of the structure satisfy the formula (i.e. by instantiating the free variables). 
Formally, Saluja et. al \cite{SalujaST95} define a family of counting classes $\#\LL$ for each fragment $\LL$ of $\so$. For a formula $\varphi(\bar{x},\bar{X})$ over $\R$, the function $f_{\varphi(x,X)}$ is defined as
$
f_{\varphi}(\A) = \vert \{\langle \bar{a},\bar{A} \rangle\mid \A\models\varphi(\bar{a},\bar{A})\}\vert.
$
for each $\A\in\ostr[\R]$. Then a function $g:\ostr[\R]\to\nat$ is in $\#\LL$ if there exists an formula $\varphi(\bar{x},\bar{X})$ in $\LL$ such that $f = f_{\varphi}$. 
In~\cite{SalujaST95}, it was shown several results for capturing function complexity classes which are relevant for our work. We discuss and use these results in Sections~\ref{sec:complexity} and~\ref{sec:syntactic}.
Note that for any formula $\varphi(\bar{x},\bar{X})$ it holds that $f_\varphi \equiv \sem{\sa{\bar{X}} \sa{\bar{x}} \varphi(\bar{x},\bar{X})}$, that is, the approach in \cite{SalujaST95} can be seen as a syntactical restriction of $\qso$. 
Thus, the advantage of our approach relies on the flexibility of defining  counting functions by alternating sum with product operators or, moreover, by introducing new quantitative operators (like in Section~\ref{sec:beyond}).
Furthermore, in the next section we show how to capture some interesting classes that could never be captured by following the approach in~\cite{SalujaST95}.


\section{Counting under $\qso$} \label{sec:complexity}
%!TEX root = main.tex

In this section, we show that by syntactically restricting $\qso$ one can capture different counting complexity classes. 
In other words, by using $\qso$ we can extend the theory of descriptive complexity~\cite{immerman1999descriptive} from decision problems to counting problems. 
For this, we first formalize the notion of \emph{capturing} a complexity class of functions.
%, and then show how to capture classes like $\shp$, $\fp$, and $\fpspace$.

Fix a signature $\R = \{R_1, \ldots, R_k\}$ and assume that $\A$ is an ordered $\R$-structure with a domain $A = \{a_1, \ldots, a_n\}$, $R_k =\, <$, and $a_1 <^{\A} a_2 <^{\A} \ldots <^{\A} a_n$. For every $i \in \{1, \ldots, k-1\}$, define the encoding of $R_i^\A$, denoted by $\enc(R_i^\A)$, as the following binary string. Assume that $\ell = \arity(R_i)$ and consider an enumeration of the $\ell$-tuples over $A$ in the lexicographic order induced by $<$. 
%(that is, $(a_1, \ldots, a_1, a_1)$, $(a_1, \ldots, a_1, a_2)$, $\ldots$, $(a_n, \ldots, a_n, a_{n-1})$, $(a_n, \ldots, a_n, a_n)$). 
Then let $\enc(R_i^\A)$ be a binary string of length $n^\ell$ such that the $i$-th bit of $\enc(R_i^\A)$ is 1 if the $i$-th tuple in the previous enumeration belongs to $R_i^\A$, and 0 otherwise. Moreover, define the encoding of $\A$, denoted by $\enc(\A)$, as the string~\cite{L04}:
%following binary string~\cite{L04}:
%\begin{eqnarray*}
	%\enc(\A) & = & 0^n \, 1 \, \enc(R_1^\A) \, \cdots \, \enc(R_k^\A).
%\end{eqnarray*}
$$
0^n \, 1 \, \enc(R_1^\A) \, \cdots \, \enc(R_{k-1}^\A).
$$
\martin{modifique la definicion para que no sea necesario codificar el $<$.}
%We define the class of all $\R$-functions, denoted by $\Func(\R)$, as the class of all functions $f: \ostr \rightarrow \bbN$.
%Given a function complexity class $\CC$ (i.e. $f: \Sigma^* \rightarrow \bbN$ for every $f \in \CC$), we say that a function $f \in \Func(\R)$ can be computed in $\CC$ if there exists $g \in \CC$ such that $f(\A) = g(\enc(\A))$ for every $\A \in \ostr$. 
%Note that the function $g$ outputs $f$ for encodings of structures and can behave arbitrarily otherwise.
We can now formalize the notion of capturing a counting complexity class.
\begin{defi} \label{def:cap}
	Let $\FF$ be a fragment of $\qso$ and $\CC$ a counting complexity class. Then {\em  $\FF$ captures $\CC$ over ordered $\R$-structures} if the  following conditions hold:
	\begin{enumerate}
		\item for every $\alpha \in \FF$, there exists $f \in \CC$ such that $\sem{\alpha}(\A) = f(\enc(\A))$ for every $\A \in \ostr[\R]$. 
		
		\item for every $f \in \CC$, there exists $\alpha \in \FF$ such that   $f(\enc(\A)) = \sem{\alpha}(\A)$ for every $\A \in \ostr[\R]$.
	\end{enumerate} 
	Moreover, {\em $\FF$ captures $\CC$ over ordered structures} if $\FF$ captures~$\CC$ over ordered $\R$-structures for every signature~$\R$. \qed
\end{defi}
%For the sake of simplification, we denote the first condition by $\FF \subseteq \CC$ and the second condition by $\CC \subseteq \FF$.
In Definition~\ref{def:cap}, function $f \in \CC$ and formula $\alpha \in \FF$ must coincide in all the strings that encode ordered $\R$-structures. Notice that this restriction is natural as we want to capture %Since we want to capture 
$\CC$ over a fixed set of structures (e.g. graphs, matrices).
%, it is natural to just consider strings that encodes $\R$-structures. 
Moreover, this restriction is fairly standard in descriptive complexity \cite{immerman1999descriptive,L04}, and it has also been used in the previous work on capturing complexity classes of functions \cite{SalujaST95,ComptonG96}.
%all notions for capturing complexity classes restrict $f \in \CC$ similarly. 

What counting complexity classes can be captured with fragments of $\qso$?
For answering this question, it is reasonable to start with $\shp$, a well-known and widely-studied counting complexity class~\cite{arora2009computational}. 
Since $\shp$ has a strong similarity with $\np$, one could expect a ``Fagin-like'' Theorem~\cite{F75} for this class. 
Actually, in~\cite{SalujaST95} it was shown that the class $\sfo$ captures $\shp$.
In our setting, the class $\sfo$ is contained in $\eqso(\fo)$, which also captures $\shp$ as expected.
 
\begin{prop} \label{prop:capture-shP}
	$\eqso(\fo)$ captures $\shp$ over ordered structures.
\end{prop}
\proof
Saluja et al.~\cite{SalujaST95} proved that $\shp = \sfo$. Hence, given that every function in $\sfo$ can also be defined $\eqso(\fo)$ (see Section~\ref{sec:previous}), we have that every function in $\shp$ can be defined in $\eqso(\fo)$.

For the other direction, let $\alpha\in\eqso(\fo)$ over some signature $\R$, defined by the grammar in \ref{syntax}. Notice that given a $\fo$ formula $\varphi$, checking whether $\A\models\varphi$ can be done in deterministic polynomial time on the size of $\A$. Also, the constant function $s$ for some $s\in\nat$ can be trivially simulated in $\shp$. These facts, together with the closure under exponential sum and polynomial product of $\shp$\cite{F97}, suffice to show that the function represented $\alpha$ is also in $\shp$.

%We construct recursively a $\shp$-machine $M_{\alpha}$ for each $\eqso(\fo)$ formula $\alpha$ over a signature $\R$. This machine, on input $(\A,v,V)$ accepts in $\sem{\alpha}(\A,v,V)$ of its non-deterministic paths for each $(\A,v,V) \in \ostr[\R]^*$. Suppose $\A$ has domain $A$. If $\alpha$ is a $\fo$-formula $\varphi$, then the machine checks if $(\A,v,V)\models\varphi$ deterministically in polynomial time, and accepts if and only if it holds true. If $\alpha$ is a constant $s$, it produces $s$ branches and accepts in all of them. If $\alpha = (\beta \add \gamma)$, then it chooses between 0 or 1, if it is 0 (1), it simulates $M_{\beta}$ ($M_{\gamma}$) on input $(\A,v,V)$. 
%If $\alpha = \sa{x}\beta$, it chooses $a\in A$ non-deterministically and simulates $M_{\beta}$ on input $(\A,v[a/x],V)$.
%If $\alpha = \sa{X}\beta$, it chooses $B\in A^{arity(X)}$ and simulates $M_{\beta}$ on input $(\A,v,V[B/X])$. This covers all possible cases for $\alpha$. Let $\alpha$ be a formula in $\eqso(\fo)$ over a signature $\R$ and let $f$ be a function over $\R$ such that $f(\enc(\A))$ is equal to the accepting paths of $M_{\alpha}$ on input $(\A,v,V)$ for some $(\A,v,V) \in \ostr[\R]^*$. We have that $f$ is a $\shp$-function over $\R$ and $f(\enc(\A)) = \sem{\alpha}(\A)$ for every $\A\in\ostr[\R]$. 
\qed

By following the same approach as~\cite{SalujaST95}, Compton and Gr\"adel~\cite{ComptonG96} show that $\seso$ captures $\spp$, where $\eso$ is the existential fragment of $\so$. As one could expect, if we parametrize $\eqso$ with $\eso$, we can also capture~$\spp$.
\begin{prop} \label{prop:capture-spanP}
	$\eqso(\eso)$ captures $\spp$ over ordered structures.
\end{prop}
\proof
To prove the condition (2), we use the fact that $\spp = \#(\eso)$. The condition holds using the same argument as in Proposition~\ref{prop:capture-shP}.
For condition (1), notice that given an $\eso$ formula $\varphi$, checking whether $\A\models\varphi$ can be done in non-deterministic polynomial time on the size of $\A$\cite{fagin1974generalized}. 
Therefore, a $\spp$ machine for $\varphi$ will simulate the non-deterministic polynomial time machine and produce the same string as output in each accepting non-deterministic run. Furthermore, the constant function $s$ for some $s\in\nat$ can be trivially simulated in $\spp$ and, thus, condition (1) holds analogously to Proposition \ref{prop:capture-shP} since $\spp$ is also closed under exponential sum and polynomial product~\cite{OH93}.
%Similar than the previous proof, we construct recursively a $\spp$ machine $M_{\alpha}$ for each $\eqso(\eso)$ formula $\alpha$ over a signature $\R$. This machine, on input $(\A,v,V)$, non-deterministically produces $\sem{\alpha}(\A,v,V)$ distinct accepting outputs for each $\A \in \ostr[\R]$. Suppose $\A$ has domain $A$. 
%If $\alpha$ is a $\eso$-formula $\varphi$ it checks if $(\A,v,V)\models\varphi$ non-deterministically in polynomial time \cite{F75}, and accepts if and only if the condition holds true. 
%If $\alpha$ is a constant $s$, then the machine produces $s$ branches and accepts in all of them. 
%If $\alpha = (\beta \add \gamma)$, then it chooses between 0 or 1, if it is 0 (1), it simulates $M_{\beta}$ ($M_{\gamma}$) on input $(\A,v,V)$.  
%If $\alpha = \sa{x}\beta$, it chooses $a\in A$ non-deterministically and simulates $M_{\beta}$ on input $(\A,v[a/x],V)$. 
%If $\alpha = \sa{X} \beta$, it chooses $B\in A^{\arity(X)}$ and simulates $M_{\beta}$ on input $(\A,v,V[B/X])$. 
%This covers all possible cases for $\alpha$.
%Additionally, the machine produces a different output on each path. This can be done by printing the trace of all the non-deterministic choices.
%However, when the machine starts checking whether $(\A,v,V)\models\varphi$ for some $\eso$ formula $\varphi$, it stops printing in the output tape. This way the machine produces exactly one output from that point onwards.
%Let $\alpha$ be a formula in $\eqso(\eso)$ over a signature $\R$ and let $f$ be a function over $\R$ such that $f(\enc(\A))$ is equal to the number of accepting outputs of $M_{\alpha}$ on input $(\A,v,V)$ for some $\A \in \ostr[\R]$. 
%We have that $f$ is a $\spp$ function over $\R$ and that $f(\enc(\A)) = \sem{\alpha}(\A)$ for every $\A\in\ostr[\R]$.
\qed
Can we capture $\fp$ by using $\# \LL$ for some fragment $\LL$ of $\so$? A first attempt could be based on the use of a fragment $\LL$ of $\so$ that capture either $\ptime$ or $\nlog$~\cite{G92}. Such an approach fails as $\# \LL$ can encode $\shp$-complete problems in both cases; in the first case, one can encode the problem of counting the number of satisfying assignments of a Horn  propositional formula, while in the second case one can encode the problem of counting the number of satisfying assignments of a 2-CNF propositional formula. A second attempt could then be based on considering a fragment $\LL$ of $\fo$. 
But even if we consider the existential fragment $\Sigma_1$ of $\fo$ the approach fails, as $\# \Sigma_1$ can encode $\shp$-complete problems like counting the number of satisfying assignments of a 3-DNF propositional formula\cite{SalujaST95}. One last attempt could be based on disallowing the use of second-order free variables in $\sfo$. But in this case one 
cannot capture exponential functions definable in $\fp$ such as~$2^n$.
Thus, it is not clear how to capture $\fp$ 
by following the approach proposed in~\cite{SalujaST95}. 
On the other hand, if we consider our framework and move out from $\eqso$, we have other alternatives for counting like first- and second-order products. In fact, the combination of $\qfo$ with $\lfp$ is exactly what we need to capture $\fp$.
\begin{thm} \label{theo:capture-fp}
	$\qfo(\lfp)$ captures $\fp$ over ordered structures.
\end{thm}
\proof
For the first condition, let $\alpha\in\qfo(\lfp)$ over some signature $\R$. Let $f$ be a function over $\R$ defined by the following procedure. Let $\enc(\A)$ be an input, where $\A$ is an ordered structure over $\R$ with domain $A = \{1,\ldots,n\}$. In the procedure we slightly extend the grammar of $\qfo(\lfp)$ to include constants. We replace each first order sum and first order product in $\alpha$ by an expansion using the elements in $A$. This is, $\sa{x} \beta(x)$ is replaced by $(\beta(1)+\cdots+\beta(n))$ and $\pa{x}\beta(x)$ is replaced by $(\beta(1)\cdot\,\cdots\,\cdot\beta(n))$. Then each sub-formula $\varphi\in\lfp$ in $\alpha$ is evaluated in polynomial time and replaced by 1 if $\A\models\varphi$ and by 0 otherwise. The resulting formula is an arithmetic expression of polynomial size (recall that $\alpha$ is fixed) which is evaluated and lastly given as output. Note that $f\in\fp$ and $f(\enc(\A)) = \sem{\alpha}(\A)$.
	
For the second condition, let $f\in \fp$ defined over some signature $\R$.
Let $\ell\in\nat$ be such that for each $\A\in\ostr[\R]$, $\lceil\log_2 f(\enc(\A)) \rceil \leq n^\ell$ (i.e. $n^\ell$ is an upper bound for the output size), where $\A$ has a domain of size $n$.
Let $\bar{x} = (x_1,\ldots,x_{\ell})$.
Consider a procedure that receives $\enc(\A)$ and an assignment $\bar{a}$ to $\bar{x}$. Let $m$ be the position of $\bar{a}$ in the lexicographic order of the tuples in $A^{\ell}$. The procedure then computes the $m$-th bit of $f(\enc(\A))$, from least to most significant. Since this procedure works in polynomial time, it can be described by an $\lfp$ formula $\Phi(\bar{x})$. Then we use
$$
\alpha = \sa{\bar{x}} \Phi(\bar{x})\cdot\varphi(\bar{x}),
$$
where $\varphi(\bar{x}) := \pa{\bar{y}}(\bar{y} < \bar{x} \mapsto 2).$ Note that if $\bar{a} \in A^{\ell}$ is the $m$-th tuple in the given order (starting from 0), then $\sem{\varphi(\bar{a})}(\A) = 2^{m}$. Adding these values for each $\bar{a}\in A^{\ell}$ gives exactly $f(\enc(\A))$. 
In other words, $\Phi(\bar{x})$ simulates the behavior of the $\fp$-machine and the formula $\alpha$ reconstruct the binary output.
Then, $\alpha$ is in $\qfo(\lfp)$ over $\R$ and $\sem{\alpha}(\A) = f(\enc(\A))$.
\qed
%To prove this theorem, 
%capture $\fp$ 
%one first shows that every formula in $\qfo(\lfp)$ can be evaluated in polynomial time. 
%Indeed, $\lfp$ is a polynomial-time logic~\cite{I86,vardi1982complexity}, and the sum and product quantifiers can also be computed in polynomial time. 
%For the other direction, one has to use $\qfo(\lfp)$ to simulate the run of a polynomial time TM $M$ computing a function, in particular using the quantitative quantifiers to reconstruct the natural number returned by $M$ in the output tape. 
%It is important to notice that the alternation between sum and product quantifiers is crucial for this reconstructions and, thus, crucial for capturing $\fp$.

At this point it is natural to ask whether one can extend the previous idea to capture $\fpspace$~\cite{Ladner89}, the class of functions computable in polynomial space. 
Of course, for capturing this class one needs a logical core powerful enough, like $\pfp$, for simulating the run of a polynomial-space TM.
Moreover, 
one also needs more powerful quantitative quantifiers as functions like $2^{2^n}$ can be computed in polynomial space,
so $\eqso$ is not enough for the quantitative layer of a logic for $\fpspace$.
In fact, by considering second-order product we obtain the fragment $\qso(\pfp)$ that captures $\fpspace$. 
\begin{thm} \label{theo:capture-fpspace}
	$\qso(\pfp)$ captures $\fpspace$ over ordered structures.
\end{thm}
\proof
For the first condition of Definition~\ref{def:cap}, notice that each $\pfp$ formula can be evaluated in deterministic polynomial space, the constant function $s$ can be trivially simulated in $\fpspace$, and $\fpspace$ is closed under exponential sum and multiplication. This suffices to show that the condition holds.
For the second condition, the proof is similar than in Theorem~\ref{theo:capture-fp}. Let $f\in \fpspace$ defined over some $\R$ and $\ell\in\nat$ such that $\log_2\left( f(\enc(\A)) \right) \leq 2^{{|\A|}^\ell}$ for every $\A\in\ostr[\R]$  (i.e. $2^{{|\A|}^\ell}$ is an upper bound for the output size). Let $X$ be a second-order variable of arity $\ell$. Consider the linear order induced by $<$ over predicates of arity $\ell$ which can be defined by the following formula:
$$
\varphi_{<}(X,Y) = \ex{\bar{u}}\big[\neg X(\bar{u})\wedge Y(\bar{u})\wedge \fa{\bar{v}}\big(
\bar{u}<\bar{v}\to(X(\bar{u})\iff Y(\bar{v}))\big)\big].
$$
Namely, we use relations to encode numbers with at most $2^{{|\A|}^\ell}$ bits where the empty relation represents $0$ and the total-relation represents $2^{2^{{|\A|}^\ell}}-1$.
Furthermore, we can use a relation~$X$ to index a position in the binary output of $f(\enc(\A))$ as follows.
%Consider a polynomial space machine over the $\R$ that receives as input an $\R$-structure $\A$ and a number $p$ encoded by a relation $X$. Then the machine accepts if, and only if, the $p$-th bit of $f(\enc(\A))$ is $1$. 
Define the language:
\[
L = \{(\A,B)\mid B \subseteq A^{\ell}\text{ and the $B$-th bit of $f(\enc(\A))$ is 1}\}.
\]
Since $L$ is in $\pspace$, it can be specified in $\pfp$ \cite{AbiteboulV89} by a formula $\Phi(X)$ where the free variable $X$ encodes relation $B$ in $L$. Then, similar than the previous proof we define:
$$
\alpha := \sa{X} \Phi(X)\mult  \pa{Y}(\varphi_{<}(Y,X)\mapsto 2).
$$ 
where $\pa{Y}(\varphi_{<}(Y,X)\mapsto 2)$ takes the value $2^m$ if there exist $m$ predicates that are smaller than~$X$ and $\alpha$ reconstruct the output of $f(\enc(\A))$ by simulating $f$ with $\Phi(X)$. Using an analogous argument, we conclude that $\alpha\in\qso(\pfp)$ and $\sem{\alpha}(\A) = f(\enc(\A))$.

\qed
%The proof of the previous theorem follows the same line as for the logical characterization of $\fp$: one shows that each function in $\qso(\pfp)$ can be computed in $\fpspace$ and, conversely, the output of a polynomial-space TM can be reconstructed by using $\pfp$ and quantitative quantifiers.

From the proof of the previous theorem a natural question follows: what happens if we use first-order quantitative quantifiers and $\pfp$?
In~\cite{Ladner89}, Ladner also introduced the class $\nfpspace$ of all functions computed by polynomial-space TMs 
with output length bounded by a polynomial.
Interestingly, if we restrict to FO-quantitative quantifiers we can also capture this class.
\begin{cor} \label{cor:capture-fpspace-poly}
	$\qfo(\pfp)$ captures $\nfpspace$ over ordered structures.
\end{cor}
\proof
For the first condition, let $\alpha\in\qfo(\pfp)$ over some signature $\R$. Let $f$ be a function over $\R$ defined by the following procedure. Let $\enc(\A)$ be an input, where $\A$ is an ordered structure over $\R$ with domain $A = \{1,\ldots,n\}$. In the procedure we slightly extend the grammar of $\qfo(\pfp)$ to include constants. We replace each first order sum and first order product in $\alpha$ by an expansion using the elements in $A$. This is, $\sa{x} \beta(x)$ is replaced by $(\beta(1)+\cdots+\beta(n))$ and $\pa{x}\beta(x)$ is replaced by $(\beta(1)\cdot\,\cdots\,\cdot\beta(n))$. Then each sub-formula $\varphi\in\pfp$ in $\alpha$ is evaluated in polynomial space and replaced by 1 if $\A\models\varphi$ and by 0 otherwise. The resulting formula is an arithmetic expression of polynomial size  (recall that $\alpha$ is fixed) which is evaluated and lastly given as output. Note that $f\in\nfpspace$ and $f(\enc(\A)) = \sem{\alpha}(\A)$.

For the second condition, let $f\in \fpspace$ defined over some signature $\R$.
Let $\ell\in\nat$ be such that for each $\A\in\ostr[\R]$, $\lceil\log_2 f(\enc(\A)) \rceil \leq n^\ell$, where $\A$ has a domain of size $n$.
Let $\bar{x} = (x_1,\ldots,x_{\ell})$.
Consider a procedure that receives $\enc(\A)$ and an assignment $\bar{a}$ to $\bar{x}$. Let $m$ be the position of $\bar{a}$ in the lexicographic order of the tuples in $A^{\ell}$. The procedure then computes the $m$-th bit of $f(\enc(\A))$, from least to most significant. Since this procedure works in polynomial space, it can be described by an $\pfp$ formula $\Phi(\bar{x})$. Then we use
$$
\alpha = \sa{\bar{x}} \Phi(\bar{x})\cdot\varphi(\bar{x}),
$$
where $\varphi(\bar{x}) := \pa{\bar{y}}(\bar{y} < \bar{x} \mapsto 2).$ Note that if $\bar{a} \in A^{\ell}$ is the $m$-th tuple in the given order (starting from 0), then $\sem{\varphi(\bar{a})}(\A) = 2^{m}$. Adding these values for each $\bar{a}\in A^{\ell}$ gives exactly $f(\enc(\A))$. Then, $\alpha$ is in $\qfo(\pfp)$ over $\R$ and $\sem{\alpha}(\A) = f(\enc(\A))$.
\qed

The results of this section validate $\qso$ as an appropriate logical framework for extending the theory of descriptive complexity to counting complexity classes. In the following sections, we provide more arguments for this claim, by considering some fragments of $\eqso$ and, moreover, by showing how to go beyond $\eqso$ to capture other classes.


\section{Exploring the structure of $\shp$ through $\qso$} \label{sec:syntactic}
%!TEX root = main.tex

The class $\shp$ was introduced in \cite{Valiant79} to prove that computing the permanent of a matrix, as defined in Example \ref{exa-perm}, is a difficult problem. More specifically, it was shown in  \cite{Valiant79}  that this problem is $\shp$-complete. As a consequence of this result the problem of computing the number of perfect matchings in a bipartite graph was also shown to be $\shp$-complete. Since then  many counting problems have been proved to be $\shp$-complete \cite{V79b,PB83,P86,L86,BW91,HMRS98,BW05,DS12, PS13,PS14}. Among them, problems having easy decision counterparts play a fundamental role, as a counting problem with a hard decision version is expected to be hard. A first prominent example of such problems is counting the number of perfect matching in a bipartite graph, as it is well-known that the problem of verifying whether there exists a perfect matching in a bipartite graph can be solved in polynomial time. Other prominent examples of such problems include counting the number of: satisfying assignments of a 2-CNF propositional formula \cite{V79b}, satisfying assignments of a DNF propositional formula \cite{DHK05}, simple paths from a source node to a target node in a directed graph \cite{V79b}, extension of a partial order to a linear order \cite{BW91} and Eulerian cycles in an undirected graph \cite{BW05}. 

Counting problems with easy decision versions play a fundamental role in the search of efficient approximations algorithms for functions in $\shp$. A fully-polynomial randomized approximation scheme (FPRAS) for a function $f : \Sigma^* \to \bbN$ is a randomized algorithm ${\cal A} : \Sigma^* \times (0,1) \to \bbN$ such that: (1) for every string $x \in \Sigma^*$ and real value $\varepsilon \in (0,1)$, the probability that $|f(x) - {\cal A}(x,\varepsilon)| \leq \varepsilon \cdot f(x)$ is at least $\frac{3}{4}$, and (2) the running time of ${\cal A}$ is polynomial in the size of $x$ and $1/\varepsilon$ \cite{KL83}. Notably, there exist $\shp$-complete functions that can be efficiently approximated as they admit FPRAS; for instance, there exist FPRAS for the problems of counting the number of satisfying assignments of a DNF propositional formula \cite{KL83} and the number of perfect matchings of a bipartite graph \cite{JSV04}. A key observation here is that if a $\shp$-complete function admits an FPRAS, then its associated decision problem is in the complexity class $\bpp$ (Bounded-Error Probabilistic Polynomial-Time). Hence, under standard complexity-theoretical assumptions we can no hope for an FPRAS for a function in $\shp$ whose decision counterpart is $\np$-complete, and we have to concentrate on the class of counting problems with easy decision versions (in $\bpp$ or in a lower complexity class such as $\ptime$). 

The importance of the class counting problems with easy decision counterparts has motivated the search of robust definitions of classes of functions in $\shp$ with easy decision versions \cite{PagourtzisZ06}. In this section, we use the framework developed in this paper to address this problem. More specifically, we introduce in Section \ref{sec-hier-shp} a hierarchy of 


consider several fragments of $\eqso(\LL)$ where $\LL$ is a boolean logic contained in $\fo$, and we study 

% It should be noted that such class can be directly defined as the set functions f ? #P such that Lf ? P, which is denoted as #Pe in [50, 51]. However, such a definition does not lead to a well-behaved and robust function complexity class. In particular, for every function f ? #P, we have that f + 1 is trivially in #Pe, which is an undesirable property. This has led to the introduction of the more robust class TotP, which is defined as the class of functions f for which there exists a non-deterministic Turing machine M running in polynomial time such that, f(x) is the result of subtracting 1 to the number of (non-necessarily accepting) runs of M with input x [40]. In [51], it is proved that TotP ? #Pe and that TotP has a complete function problem under parsimonious reductions. However, no natural problem is known to be TotP-complete under this type of reductions [51].





In this section we study the fragment of $\eqso(\LL)$ when $\LL$ is a boolean logic contained in $\fo$. We show that by restricting $\LL$ we can find different subclasses below $\shp$ with interesting computational and closure properties. 

\cite{OH93,FH08}

From this point on, for each fragment $\FF$ of $\qso$, we will also use $\FF$ to refer to the class of functions defined by the formulas in $\FF$.

\subsection{A counting hierarchy below $\shp$}
\label{sec-hier-shp}
%!TEX root = syntactic.tex

Inspired by the connection between $\shp$ and $\sfo$, a hierarchy of subclases of $\sfo$ was introduced in~\cite{SalujaST95} 
%studied subclasses of  $\sfo$ syntactically 
by restricting the alternation of quantifiers in Boolean formulas.
%, defining what we call 
Specifically, the \emph{$\sfo$-hierarchy} consists of the 
%they define 
the classes $\E{i}$ and $\U{i}$ for every $i \geq 0$, where $\E{i}$ (resp., $\U{i}$) is defined as $\sfo$ but restricting the formulas used to be in $\loge{i}$ (resp., $\logu{i}$).
%is there exists a for are defined where functions are defined by $\fo$-formulas in $\loge{i}$ and $\logu{i}$, respectively, 
%for every $i \geq 0$. 
By definition ,we have that $\U{0} = \E{0}$. Moreover, it is shown in~\cite{SalujaST95} that:
% this function classes defined a finite hierarchy of the form:
\[
\E{0} \; \subsetneq \; \E{1} \; \subsetneq \; \U{1} \; \subsetneq \; \E{2} \; \subsetneq \; \U{2} \; = \; \#\fo 
\]
In light of the framework introduced in this paper, natural extensions of these classes are obtained by considering 
%in light of the $\eqso$ logic. Specifically, we consider 
%the classes 
$\eqso(\loge{i})$ and $\eqso(\logu{i})$ for every $i \geq 0$, which form the \emph{$\eqso$-hierarchy}.
%, where the boolean logic is restricted to $\loge{i}$ and $\logu{i}$, respectively. 
%We denote each class by $\QE{i}$ and $\QU{i}$ for short. 
Clearly, we have that $\E{i} \subseteq \QE{i}$ and $\U{i} \subseteq \QU{i}$. Indeed, each formula $\varphi(\bar{X}, \bar{x})$ in $\E{i}$ is equivalent to the formula $\sa{\bar X} \sa{\bar x} \varphi(\bar{X}, \bar{x})$ in $\QE{i}$, and likewise for $\U{i}$ and $\QU{i}$.
But what is the exact relationship between these two hierarchies?
%Then it is left to know whether these containments are strict.
%between the sharp and quantitative classes is strict or not. 
%For this, 
To answer this question, we start by introducing two normal forms for $\eqso(\LL)$ that helps us to characterize the expressive power of this quantitative logic.
%study first whether a formula $\alpha$ in $\eqso(\LL)$ can be transformed into a 
%\emph{prenex 
%normal form where sum quantifiers are restricted to be at the beginning of the formula. 
%Formally, we say that 
A formula $\alpha$ in $\eqso(\LL)$ is in \emph{$\LL$-prenex normal form ($\LL$-PNF)} 
%(or just prenex normal form) 
if $\alpha$ is of the form
%\[
%\alpha := 
$\sa{\bar{X}} \sa{\bar{x}} \varphi(\bar{X}, \bar{x})$,
%\]
where $\bar{X}$ and $\bar{x}$ are sequences of zero or more second-order and first-order variables, respectively, and $\varphi(\bar{X}, \bar{x})$ is a formula in $\LL$. Notice that 
%a sentence in $\LL$ is in $\LL$-PNF, while 
a formula $\varphi(\bar{X}, \bar{x})$ in $\#\LL$ is equivalent to the formula $\sa{\bar X} \sa{\bar x} \varphi(\bar{X}, \bar{x})$ in $\LL$-PNF. 
%In particular, 
%%note that a 
%each Boolean formula is in 
%%$\Sigma$-
%prenex normal form.
Moreover, a formula $\alpha$ is in \emph{$\LL$-sum normal form ($\LL$-SNF)} if $\alpha$ is of the form $c + \Sigma_{i=1}^n \alpha_i$, where $c$ is a non-negative constant and each $\alpha_i$ is in 
%$\Sigma$-
$\LL$-PNF.
%prenex normal form.
%The following results shows that 
%%Next we show that 
%each formula in $\eqso(\LL)$ can be converted in sum normal form.
% but not always in prenex normal form.
\begin{theorem}\label{theo-pnf-snf}
Every formula in $\eqso(\LL)$ can be rewritten in $\LL$-SNF.
%sum normal form.
%such that there does not exist a formula in $\Sigma$-prenex normal form in $\QE{1}$ equivalent to $\alpha$.
\end{theorem}
%\martin{Creo que el resultado de que existe una formula en $\QE{1}$ que no se puede pasar a prenex está mas relacionado con el teorema siguiente que con este.}
%By the previous result, we know that there exists logics $\LL$ where no $\Sigma$-prenex normal form exists. 
%Then
Therefore, to unveil the relationship between the $\sfo$-hierarchy and the $\eqso$-hierarchy, we need to understand the boundary between PNF and SNF. We do this in the following theorem. 
%An interesting question at this point is when a formula $\alpha \in \eqso(\LL)$ can be converted in 
%%$\Sigma$-
%prenex normal form. 
%%Of course, this would depend on the expressibility of $\LL$.
%Interestingly, if $\LL$ contains $\logu{1}$, then it can be shown that $\alpha$ can be converted in
%%we can always convert any formula in $\Sigma$-
%%prenex 
%this normal form. 
\begin{theorem}\label{theo-pi1-pnf}
There exists a formula $\alpha$ in $\QE{1}$ that is not equivalent to any formula in $\Sigma_1$-PNF. 
%prenex normal form. 
On the other hand, if $\logu{1} \subseteq \LL$, then 
	%for 
	every formula in
	%$\alpha \in \eqso(\LL)$ 
	$\eqso(\LL)$ can be rewritten in $\LL$-PNF. 
	%there exists a formula $\beta \in \eqso(\LL)$ equivalent to $\alpha$ in $\Sigma$-
%	prenex normal form.
\end{theorem}

\begin{figure*}

\begin{center}
\begin{tabular}{ccc}
$\E{0}$ & $\subsetneq$ & 
\end{tabular}
\end{center}

\caption{The relationship between the $\sfo$-hierarchy and the $\eqso$-hierarchy.\label{fir-sfo-eqso}}
\end{figure*}


As our first result, we show that in terms of containment the $\eqso$-hierarchy behaves as the $\sfo$-hierarchy:
%$\eqso$ also defined a finite hierarchy similar than in~\cite{SalujaST95} that we called the $\eqso$-hierarchy.
\begin{proposition}
\begin{multline*}
\; \QE{0} \; \subsetneq \; \QE{1} \; \subsetneq \; \QU{1} \; \subsetneq \\ \QE{2} \; \subsetneq \; \QU{2} \; = \; \eqso(\fo)
\end{multline*}
\end{proposition}
As our second result, we establish precise connections between 
%A natural question at this point is what is 
%the connection between 
$\E{i}$ and $\U{i}$ and their corresponding classes $\QE{i}$ and $\QU{i}$. 






Theorems \ref{theo-pnf-snf} and \ref{theo-pi1-pnf} are instrumental in answering our question of what is the relationship between the $\sfo$-hierarchy and the $\eqso$-hierarchy. 
%The previous results gives the connection between the hierarchy in \cite{SalujaST95} and the $\eqso$-hierarchy. 
Indeed, if $\LL$ contains $\logu{1}$, then we have that $\sh{\LL}$ is equal to $\eqso(\LL)$ since each formula in $\eqso(\LL)$ can be converted in prefix normal form and, therefore, it is equivalent to a formula in $\sh{\LL}$. 
The following proposition summarizes these results, also including the cases of $\E{0}$ and $\E{1}$.
%Unfortunately, this is not the case for $\loge{0}$ and $\loge{1}$ as the following result shows.
\begin{proposition}
	The classes $\E{0}$ and $\E{1}$ are strictly contained in $\QE{0}$ and $\QE{1}$, respectively. Moreover, the classes $\U{1}$, $\E{2}$, and $\U{2}$ are equivalent with $\QU{1}$, $\QE{2}$, and $\QU{2}$, respectively.
\end{proposition}
The previous result shows that the classes $\QE{i}$ and $\QU{i}$ are more robust than the classes $\E{i}$ and $\U{i}$: they are closed under binary and sum quantifiers but the other not necessarily. 

Now, we study the complexity classes describe by this hierarchies. As the following result shows, $\eqso(\loge{0})$ defines only tractable counting functions and $\eqso(\loge{1})$ intractable counting functions but with an tractable decision problems. 
\begin{proposition} \label{prop:qe0-fp-qe1-totp-fptras}
All functions defined in $\eqso(\loge{0})$ and $\eqso(\loge{1})$ can be computed in $\fp$ and $\totp$, respectively. Furthermore, every function defined in $\eqso(\loge{1})$ has a FPTRAS.
\end{proposition}
Therefore, in terms of counting complexity, the $\eqso$-hierarchy behaves exactly the same as the $\#\fo$-hierarchy.

The next step is to study the closure properties of $\eqso$-hierarchy. 
An advantage of the $\eqso$-hierarchy is that, by its language syntax, all the classes are closed under addition and first and second order sum.
So, the first question is whether the multiplicative operators in $\qso$ can be defined in $\eqso(\LL)$. As the following result shows, if $\LL$ is closed under conjunction, then the binary product can be defined in  $\eqso(\LL)$.
\begin{theorem}\label{theo:binary-prod}
	If $\LL$ is closed under conjunction, then binary product can be defined in $\eqso(\LL)$.
\end{theorem}
The next question is whether the hierarchy is closed under subtraction. Formally, for any pair of functions $f,g$, we define $f - g$ as the function such that $(f - g)(\A) = f(\A)-g(\A)$ whenever $f(\A)>g(\A)$ and $0$ otherwise.
As the next result shows, all classes in the $\eqso$-hierarchy is not closed under subtraction unless ${\sc P} = {\sc NP}$
\begin{theorem} \label{sub-pnp}
If $\eqso(\loge{i})$ or $\eqso(\logu{i})$ is closed under subtraction for $i > 0$, then {\sc P} = {\sc NP}.
\end{theorem}
\cristian{Martin, el resultado que tienes en el apendice se generaliza trivialmente para todas las clases ya que todas contienen la clase $\eqso(\loge{0})$.}

By the previous result, we know that functions in the $\eqso$ hierarchy are unlikely to be closed under subtraction. Then, a natural restriction to this question is to ask whether these classes are closed under subtraction by one, namely, if $\CC$ is a class of functions and $f \in \CC$, is $f-1 \in \CC$ where $1$ is the constant function that outputs $1$ for every structure. 
We do not know $\E{1}$ is closed under subtraction by one. However, if we extend $\logex{1}$ with $\fo$ predicates we can show that this new fragment is closed under subtraction by one.
\begin{theorem} \label{sigmafo-minusone}
	$\eqso(\logex{1})$ is closed under substraction by one.
\end{theorem}


 



% We are interested in, for each $\fo$-fragment $\LL$, the biggest fragment of $\eqso$ that is contained in $\#\LL$.
%Let $\LL = \loge{0}$:
%\begin{theorem} \label{one-sigma-zero}
%	Positive constant functions are not expressible in $\E{0}$
%\end{theorem}
%\begin{corollary}
%	If a fragment of $\eqso(\loge{0})$ is contained in $\E{0}$, its grammar does not allow sole constants.
%\end{corollary}
%\begin{conjecture}
%	Sum is not expressible in $\E{0}$
%\end{conjecture}
%\begin{theorem} \label{mult-sigma-zero}
%	$\sqso(\loge{0})$ with binary product is contained in $\E{0}$.
%\end{theorem}
%\begin{theorem} \label{fo-prod-sigma-zero}
%	If an extension of $\sqso(\loge{0})$ is contained in $\E{0}$, its grammar does not allow first-order product.
%\end{theorem}


%For every logic $\LL$, we define an $\LL$-extended quantifier-free (QF) formula as follows:
%\begin{eqnarray*}
%	\varphi &::=& \alpha, \alpha \text{ is an $\LL$-formula} \ \mid \\
%	&& X_i(x_1,\dots,x_{a_i}), i\in\N \ \mid \ \\
%	&& (\neg \varphi) \ \mid \ (\varphi \wedge \varphi) \ \mid \ (\varphi \vee \varphi).
%\end{eqnarray*}
%
%We define syntactically the fragments $\logex{i}$ and $\logux{i}$ according to the following grammar:
%\begin{align*}
%\logex{0} = \logux{0} &::= \varphi , \varphi \mbox{ is an $\fo$-extended QF formula,} \\
%\logex{i+1} &::= \logux{i} \ \mid \ \exists x\, \logex{i+1}, \\
%\logux{i+1} &::= \logex{i} \ \mid \ \forall x\, \logux{i+1}.
%\end{align*}

%We see that many of the results in Saluja et. al. \cite{SalujaST95} for $\#\LL$ still apply in $\eqso(\LL)$ for a given fragment $\LL$:
%
%\begin{theorem} \label{eqso-sigma-zero-in-fp}
%	For every $\eqso(\loge{0})$ formula $\alpha$ over a signature $\R$, the function $f$ over $\R$ defined as $f(\enc(\A)) = \sem{\alpha}(\A)$ is in $\fp$.
%\end{theorem}
%
%\begin{theorem} \label{eqso-sigma-one-in-eqso-pi-one}
%	For every $\eqso(\loge{1})$ formula $\alpha$ over a signature $\R$ there exists a $\eqso(\logu{1})$ formula $\beta$ over $\R$ such that $\sem{\alpha}(\A) = \sem{\beta}(\A)$ for every $\A\in\ostr[\R]$.
%\end{theorem}
%
%
%The {\em decision problem} associated to a function $f$ is defined by the language $L_f = \{\A \in \str \mid f(\A) > 0\}$.
%
%\begin{theorem} \label{decisionptime}
%	The decision problem associated to a function in $\eqso(\logex{1})$ is in \textsc{P}.
%\end{theorem}

%For a given pair of functions $f,g$, we define $f \dotminus g$ as follows:
%\begin{eqnarray*}
%	(f \dotminus g)(\A) =
%	\begin{cases}
%		f(\A)-g(\A), & \text{if }f(\A)>g(\A) \\
%		0, & \text{if }f(\A) \leq g(\A).
%	\end{cases}
%\end{eqnarray*}
%for every $\L$-structure $\A \in \str$. A function class $\F$ is {\em closed under substraction} if for every pair of functions $f,g \in \F$, it holds that $f \dotminus g \in \F$.
%
%\begin{theorem} \label{sub-pnp}
%	If $\eqso(\loge{1})$ is closed under substraction, then {\sc P} = {\sc NP}.
%\end{theorem}
%
%\begin{theorem} \label{sigma1strict}
%	$\eqso(\loge{1}) \subsetneq \eqso(\logex{1})$
%\end{theorem}
%
%For a given function $f$, we define $f \dotminus 1$ as follows:
%\begin{eqnarray*}
%	f \dotminus 1(\A) =
%	\begin{cases}
%		f(\A)-1, & \text{if }f(\A) > 0 \\
%		0, & \text{if }f(\A) = 0.
%	\end{cases}
%\end{eqnarray*}
%for every $\L$-structure $\A \in \str$. A function class $\F$ is {\em closed under substraction by one} if for every function $f \in \F$, it holds that $f \dotminus 1 \in \F$.
%
%\begin{theorem} \label{sigmafo-minusone}
%	$\eqso(\logex{1})$ is closed under substraction by one.
%\end{theorem}
%
%\begin{theorem} \label{dnf-pars}
%	{\sc \#DNF} is hard for $\eqso(\loge{1})$ under parsimonious reductions. 
%\end{theorem}
%
%\begin{theorem} \label{nplusone-strict}
%	$\U{1}$ with $n$ open first-order variables is properly contained in $\U{1}$ with $n+1$ open first-order variables for $n\in\N$.  
%\end{theorem}

\subsection{Counting hierarchy below $\shp$}


\subsection{Horn Counting Classes}
%!TEX root = main.tex
\newcommand{\pP}{\textit{P}}
\newcommand{\pN}{\textit{N}}
\newcommand{\pV}{\textit{V}}
\newcommand{\pT}{\textit{T}}
\newcommand{\pA}{\textit{A}}
\newcommand{\pNC}{\textit{NC}}
\newcommand{\pD}{\textit{D}}


A positive literal is a formula of the form $X(\x)$, where $X$ is a second-order variable and $\x$ is a tuple of first-order variables, and a negative literal is a formula of the form $\exists \v \, \neg X(\u,\v)$, where $\u$ and $\v$ are tuples of first-order variables. Given a relational signature $\R$, a clause over $\R$ is a formula of the form:
$$
\forall \x \, (\varphi_1 \vee \cdots \vee \varphi_n),
$$
where each $\varphi_i$ ($1 \leq i \leq n$) is either a positive literal, a negative literal or an \fo-formula over $\R$.  A clause is said to be Horn if it contains at most one positive literal, and a formula is said to be Horn if it is a conjunction of Horn clauses over a relational signature $\R$. With this terminology, we define $\uhorn$ as the set of formulas $\psi$ such that $\psi$ is a conjunction of Horn clauses over a relational signature $\R$. 

\begin{proposition}\label{prop-uhorn-pe}
$\eqso(\uhorn) \subseteq \pe$
\end{proposition}

\begin{example} \label{ex-hornsat-esop1}
Let $\R = \{\pP(\cdot,\cdot), \pN(\cdot,\cdot), \pV(\cdot), \pNC(\cdot)\}$. This vocabulary is used as follows to encode a Horn formula. A fact $\pP(c,x)$ indicates that propositional variable $x$ is a disjunct in a clause $c$, while $\pN(c,x)$ indicates that $\neg x$ is a disjunct in $c$. Furthermore, $\pV(x)$ holds if  $x$ is a propositional variable, and $\pNC(c)$ holds if $c$ is a clause containing only negative literals, that is, $c$ is of the form $(\neg x_1 \vee \cdots \vee \neg x_n)$.

To encode $\chsat$, we define an \so-formula $\varphi(\pT)$ over $\R$, where $\pT$ is a unary predicate, such that for every Horn formula $\theta$ encoded by an $\R$-structure $\A$, the number of satisfying assignments of $\theta$ is equal to $\sem{\sa{\pT} \varphi(\pT)}(\A)$. In particular, $\pT(x)$ holds if and only if $x$ is a propositional variable that is assigned value 1.  More specifically, $\varphi(\pT)$ is defined as follows:
\begin{align*}
&\forall x \, (\pT(x) \to \pV(x)) \ \wedge\\
&\forall c \, (\pNC(c) \to \exists x \, (\pN(c,x) \wedge \neg \pT(x))) \ \wedge\\
&\forall c \forall x \, ([\pP(c,x) \wedge \forall y \, (\pN(c,y) \to \pT(y))] \to \pT(x)).
\end{align*}
Given that $\uhorn$ is designed with the goal in mind of capturing $\chsat$, we expect $\varphi(\pT)$ to be a formula in $\uhorn$. However, if we rewrite it as a conjunction of clauses we obtain the following:
\begin{align*}
&\forall x \, (\neg \pT(x) \vee \pV(x)) \ \wedge\\
&\forall c \, (\neg \pNC(c) \vee \exists x \, (\pN(c,x) \wedge \neg \pT(x)))\ \wedge\\
&\forall c \forall x \, (\neg \pP(c,x) \vee \exists y \, (\pN(c,y) \wedge \neg \pT(y)) \vee \pT(x)).
\end{align*}
The resulting formula $\varphi(\pT)$ is not in $\uhorn$, but it can be easily transformed into a formula in this class  by introducing an auxiliary binary predicate $\pA$ defined as follows:
\begin{align*}
\forall c \forall x \, (\neg \pA(c,x) \leftrightarrow [\pN(c,x) \wedge \neg \pT(x)]).
\end{align*}
In this way, we obtain the following formula $\psi(\pT,\pA)$ in $\uhorn$:
\begin{align*}
&\forall x \, (\neg \pT(x) \vee \pV(x)) \ \wedge\\
&\forall c \, (\neg \textit{NC}(c) \vee \exists x \, \neg \textit{A}(c,x)) \ \wedge\\
&\forall c \forall x \, (\neg \textit{P}(c,x) \vee \exists y \, \neg \textit{A}(c,y) \vee \textit{T}(x)) \ \wedge\\
&\forall c \forall x \, (\neg \textit{N}(c,x) \vee \textit{T}(x) \vee \neg \textit{A}(c,x)) \ \wedge\\
&\forall c \forall x \, (\textit{A}(c,x) \vee \textit{N}(c,x)) \ \wedge\\
&\forall c \forall x \, (\textit{A}(c,x) \vee \neg\textit{T}(x)).
\end{align*}
This formula effectively defines $\chsat$
as for every Horn formula $\theta$ encoded by an $\R$-structure $\A$, the number of satisfying assignments of $\theta$ is equal to $\sem{\sa{\pT} \sa{\pA} \psi(\pT,\pA)}(\A)$.  Therefore, we conclude that $\chsat \in \eqso(\uhorn)$. 
\end{example}
We extend the definition of $\uhorn$ to allow existential quantification. More precisely, a formula $\varphi$ is in $\ehorn$ if $\varphi$ is of the form $\exists \bar x \, \psi$ with $\psi$ a Horn formula. Interestingly, it hold that $\cdnf \in \eqso(\ehorn)$ and

\begin{proposition}\label{prop-ehorn-pe}
$\eqso(\ehorn) \subseteq \pe$.
\end{proposition}
A natural question at this point is whether in the definitions of $\uhorn$ and $\ehorn$, it is necessary to allow negative literals of the form $\exists \v \, \neg X(\u,\v)$. The following result shows that it is indeed the case:

\begin{proposition}\label{prop-hsat-not-sigma2}
$\chsat \not\in \eqso(\loge{2})$.
\end{proposition}
We conclude this section by showing that a natural extension of $\chsat$ is $\eqso(\ehorn)$-complete under parsimonious reductions. We define the decision problem:
\begin{multline*}
\dhsat = \{\Phi \mid \Phi \text{ is a disjunction of}\\  \text{Horn formulas and $\Phi$ is satisfiable}\},
\end{multline*}
and the counting problem $\shdhsat$ as a function that counts all satisfying assignments of a formula $\Phi$ that is a disjunction of Horn formulas.

\begin{theorem} \label{sigma2hard}
	$\shdhsat$ is $\eqso(\ehorn)$-complete under parsimonious reductions. 
\end{theorem}


\section{Adding recursion to QSO}\label{sec:beyond}
%!TEX root = main.tex

In the previous sections, we use weighted logics to give a framework for descriptive complexity of functions. Here, we go beyond weighted logics to add unexplored quantifiers at the quantitative level. Specifically, we give the first steps on defining quantitative recursion. This goal is not trivial for two reasons: (1) we want to add recursion over functions and (2) it is not clear what could be the right notion of ``fixed point'' . 
Towards this goal, we show first how to extend $\qso$ with function symbols to later use them to define recursion at a quantitative level. 
Furthermore, we define a natural generalization of (boolean) least fixed point for functions that we called \emph{least s-fixed point} and, as a proof of concept, we show that it captures $\fp$.
Finally,we use our new tool to define an operator for counting paths in a graph, a natural generalization of the transitive closure operator~\cite{immerman1999descriptive}, and show that this captures the expressiveness of~$\shl$.

For defining recursion, we first need to define an extension of $\qso$ with function symbols. Assume that $\fs$ is an infinite set of function symbols, where each $h \in \fs$ has an associated arity, which is denoted by $\arity(h)$. Then the set of $\fqso$ formulas over a relational signature $\R$ is defined by the following grammar:
\begin{multline}
\label{eq-fqso}
	\alpha := \varphi \ \mid \  s \  \mid \  h(x_1, \ldots, x_\ell) \  \mid \
	(\alpha \add \alpha) \  \mid\  (\alpha \mult \alpha) \  \mid \\  
	\sa{x} \alpha \  \mid \
	\pa{x} \alpha \  \mid \
	\sa{X} \alpha \  \mid \
	\pa{X} \alpha,
\end{multline}
where $h \in \fs$, $\arity(h) = \ell$ and $x_1, \ldots, x_\ell$ is a sequence of (non-necessarily distinct) first-order variables. Given an $\R$-structure $\A$ with domain $A$, we say that $F$ is a \emph{function assignment} for $\A$ if for every $h \in \fs$ with $\arity(h) = \ell$, we have that $F(h) :  A^\ell \to \N$. The notion of function assignment is used to extend the semantics of $\qso$ to the case of a quantitative formula of the form $h(x_1, \ldots, x_\ell)$. More precisely, given first-order and second-order assignments $v$ and $V$ for $\A$, respectively, 
we have that:
\begin{eqnarray*}
\sem{h(x_1, \ldots, x_\ell)}(\A,v,V,F) & = & F(h)(v(x_1),\ldots, v(x_\ell)).
\end{eqnarray*}
As for the case of $\qfo$, we define $\fqfo$ disallowing quantifiers $\Sigma X$ and $\Pi X$ in \eqref{eq-fqso}.

It worths noting that function symbols represent in $\fqso$ functions from tuples to numbers and cannot be compared to the classical notion of (boolean) function symbol in first-order logics~\cite{enderton2001mathematical}. 
Furthermore, one could see a function symbol as an ``oracle'' that is instantiated by the function assignment. 
As far as we know, this is the first paper proposing this extension on weighted logics and it would be interesting to study this concept further (i.e. outside recursion).

We define an operator which extends Least Fixed Point Logic (LFP) \cite{I86,vardi1982complexity} over $\fo$ to allow counting. 
%We show how to define this operator in $\qfo(\fo)$ and we discuss how to extend these ideas to $\qso$ towards the end of the section. 
Fix a relational signature $\R$. Then the set of $\rqfo(\fo)$ formulas (Recursive $\qfo$) over $\R$ is defined as an extension of $\qfo(\fo)$ by adding the operator $\clfp{\beta(\x, h)}$ where $\x = (x_1, \ldots, x_\ell)$ is a sequence of $\ell$ distinct first-order variables and $\beta(\x, h)$ is an $\fqfo(\fo)$-formula over $\R$ whose only function symbol is $h$, with $\arity(h) = \ell$. The free variables of the formula $\clfp{\beta(\x,h)}$ are $x_1, \ldots, x_\ell$; in particular, $h$ is not considered to be free in this formula.

Fix an $\R$-structure with domain $A$ and a quantitative formula $\clfp{\beta(\x,h)}$, and assume that $\F$ is the set of functions $f :A^\ell \to \N$. To define the semantics of $\clfp{\beta(\x,h)}$, we first show how $\beta(\x,h)$ can be interpreted as an operator $T_{\beta}$ on $\F$. More precisely, for every $f \in \F$ and tuple $\a = (a_1, \ldots, a_{\ell}) \in A^\ell$, the function $T_{\beta}(f)$ satisfies that:
\begin{eqnarray*}
T_{\beta}(f)(\a) & = & \sem{\beta(\x, h)}(\A,v,F),
\end{eqnarray*}
where $v$ is a first-order assignment  for $\A$ such that $v(x_i) = a_i$ for every $i \in \{1, \ldots, \ell\}$, and $F$ is a function assignment for $\A$ such that $F(h) = f$. 

As for the case of LFP, it would be natural to consider the point-wise partial order $\leq$ on $\F$ defined as $f \leq g$ if, and only if, $f(i) \leq g(i)$ for every $i \in \{1, \ldots, \ell\}$, and let the semantics of $\clfp{\beta(\x,h)}$ be the least fixed point of the operator $T_\beta$. However, $(\F, \leq)$ is not a complete lattice, so we do not have a Knaster-Tarski Theorem ensuring that such a fixed point exists. Instead, we generalize the semantics of LFP as follows. In the definition of the semantics of LFP, an operator $T$ on relations is considered, and the semantics is defined in terms of the least fixed point of $T$, that is, a relation $R$ such that~\cite{I86,vardi1982complexity}: 
\begin{enumerate}
 \item[(a)] $T(R) = R$, and 
 \item[(b)]  $R \subseteq S$ for every $S$ such that $T(S) = S$.  
\end{enumerate}
We can view $T$ as an operator on functions if we consider the characteristic function of a relation. Given a relation $R \subseteq A^\ell$, let $\chi_R$ be its characteristic function, that is $\chi_R(\bar a) = 1$ if $\bar a \in R$, and $\chi_R(\bar a) = 0$ otherwise. Then define an operator $T^\star$ on characteristic functions as $T^\star(\chi_R) = \chi_{T(R)}$. Moreover, we can rewrite the conditions defining a least fixed point of $T$ in terms of the operator $T^\star$ if we consider the notion of support of a function. Given a function $f \in \F$, define the support of $f$, denoted by $\support(f)$, as $\{ \bar a \in A^\ell \mid f(\bar a) > 0 \}$. Then given that $\support(\chi_R) = R$, we have that the conditions (a) and (b) are equivalent to the following conditions on $T^\star$:
%Given a relation $R \subseteq A^\ell$, let $\chi_R$ be its characteristic function, that is $\chi_R(\bar a) = 1$ if $\bar a \in R$, and $\chi_R(\bar a) = 0$ otherwise. Moreover, given a function $f \in \F$, define the support of $f$, denoted by $\support(f)$, as $\{ \bar a \in A^\ell \mid f(\bar a) > 0 \}$. Notice that for a relation $R$ we have that $\support(\chi_R) = R$.  In particular, the conditions defining the least fixed point of $T$ can be rewritten as follows in terms of $T^\star$: 
\begin{enumerate}
	\item[(a)] $\support(T^\star(\chi_R)) = \support(\chi_R)$, and  
	\item[(b)] $\support(\chi_R) \subseteq \support(\chi_S)$ for every $S$ such that  $\support(T^\star(\chi_{S})) = \support(\chi_S)$.  
\end{enumerate}
To define a notion of fixed point for $T_\beta$ we simply generalized these conditions. More precisely, %These conditions can be generalized in the obvious way to the case of the operator $T_\beta$: 
a function $f \in \F$ is a {\em s-fixed point} of $T_{\beta}$ if $\support(T_\beta(f)) = \support(f)$, and $f$ is a {\em least s-fixed point} of $T_{\beta}$ if $f$ is a s-fixed point of $T_\beta$ and for every s-fixed point $g$ of $T_\beta$ it holds that $\support(f) \subseteq \support(g)$. The existence of such fixed point is ensured by the following lemma:
\begin{lemma}\label{lem-support}
If $f,g \in \F$ and $\support(f) \subseteq \support(g)$, then $\support(T_\beta(f)) \subseteq \support(T_\beta(g))$.
\end{lemma}
In fact, as for the case of LFP, this lemma gives us a simple way to compute a least s-fixed point of $T_\beta$. Let $f_0 \in \F$ be a function such that $f_0(\bar a) = 0$ for every $\bar a \in A^\ell$ (i.e. $f_0$ is the only function with empty support), and let function $f_{i+1}$ be defined as $T_\beta(f_i)$ for every $i \in \N$. Then there exists $j \geq 0$ such that $\support(f_j) = \support(T_\beta(f_j))$. Let $k$ be the smallest natural number such that $\support(f_{k}) = \support(T_\beta(f_k))$. We have that $f_k$ is a least s-fixed point of $T_\beta$, which is used to defined the semantics of $\clfp{\beta(\x, h)}$. More specifically, for an arbitrary first-order assignment $v$ for $\A$:
\begin{eqnarray*}
\sem{\clfp{\beta(\x, h)}}(\A,v) & = & f_{k}(v(\x))
\end{eqnarray*}

\begin{example} \label{ex:count-path}
%As an example, 
We would like to define an $\rqfo(\fo)$-formula that, given a directed acyclic graph $G$ with $n$ nodes and a pair of nodes $b$, $c$ in $G$, counts the number of paths of length at most $n$ from $b$ to $c$ in $G$. To this end, assume that graphs are encoded using the relational signature $\R = \{ E(\cdot,\cdot) \}$, and then define formula $\alpha(x, y, f)$ as follows:
$$
%\alpha(x,y,R,\pi) = 
%(\neg \exists zR(z))\cdot(x = y) 
E(x,y) + \sa{z} f(x,z)\cdot E(z,y).
$$
We have that $\clfp{\alpha(x,y,f)}$ defines our counting function. In fact, assume that $\A$ is an $\R$-structure with $n$ elements in its domain encoding an acyclic directed graph. Moreover, assume that $b,c$ are elements of $\A$ and $v$ is a first-order assignment over $\A$ such that $v(x) = b$ and $v(y) = c$. Then we have that $\sem{\alpha(x,y,f)}(\A,v)$ is equal to the  number of paths in $\A$ from $b$ to $c$ of length at most $n$.

Assume now that we need to extend our previous counting function to the case of arbitrary directed graphs. To this end, suppose that $\varphi_{\text{\rm first}}(x)$ and $\varphi_{\text{succ}}(x,y)$ are the $\fo$-formulas for defining the first and successor predicates, respectively, of $<$. Moreover, define formula $\beta(x, y, t, g)$ as follows:
\begin{multline*}
(E(x,y) + \sa{z} g(x,z,t)\cdot E(z,y)) \cdot \varphi_{\text{\rm first}}(t) \ +\\
\sa{t'} \varphi_{\text{succ}}(t',t) \cdot \left(\sa{x'} \sa{y'} g(x',y',t') \right)
\end{multline*}
Then our extended counting function is defined by the following formula:
$$
\sa{t} (\varphi_{\text{\rm first}}(t) \wedge \clfp{\beta(x,y,t,g)}).
$$ 
In fact, the number of paths of length at most $n$ from a node $x$ to a node $y$ is recursively computed by using the formula $(E(x,y) + \sa{z} g(x,z,t)\cdot E(z,y)) \cdot \varphi_{\text{\rm first}}(t)$, which stores this value in $g(x,y,t)$ with $t$ the first element in the domain.  The other formula, the one with the filter $\varphi_{\text{succ}}(t',t)$, is used as a counter that allows to reach a fixed point in the support of function $g$ in $n$ steps.
\end{example}
%
%
%It is well known that least fixed point logic is contained in second-order logic \cite{L04}. In the following theorem we show that the same holds in our case.
%\begin{theorem} \label{so-rec}
%$\rqfo \subseteq \qso$
%%	Given a positive $\fo$ formula $\varphi(\bar{x},R)$ and a $\qfo$ formula $\alpha(\bar{x})$, there exists a $\qso$ formula $\beta(\bar{x})$ such that $\sem{[\alfp\varphi(\bar{x},R)\mid \alpha(\bar{x},R)](\bar{x})} = \sem{\beta(\bar{x})}$.
%\end{theorem} 
Note that, in contrast to $\lfp$-logic, here we do not need to impose any positive restriction to the formula $\beta(\x,h)$.
Indeed, since $\beta$ is constructed from monotones operations (i.e. sum and product) over the natural numbers, the resulting operator $T_{\beta}$ is monotone as well.

Now that a least fixed point operator over functions is defined, the next step is to understand its expressive power.
In the following result we show how lsfp-operator can be used to capture all functions that can be computed in polynomial time.
\begin{theorem} \label{rqfo-fo-cap}
	$\rqfo(\fo)$ captures $\fp$ over the class of ordered structures.
\end{theorem}

%Given a relation signature $\R$, the set of recursive $\qfo$ formulas ($\rqfo$-formulas) is defined by the following grammar:
%%This operator lets us define the set of recursive $\qfo$ formulas over $\R$ ($\rqfo$-formulas) using the following grammar:
%\begin{multline*}
%%	\label{eq-def-rqfo}
%	\alpha := \varphi \ \mid \ s \ \mid \ (\alpha \add \alpha) \ \mid \\ (\alpha \mult \alpha) \ \mid \ \sa{x} \alpha \ \mid \ \pa{x} \alpha \ \mid \ [\alfp \varphi \mid \alpha]
%\end{multline*}
%where $\varphi$ is an $\fo$-formula over $\R$, $s \in \bbN$ and $x \in \fv$.
%
%\marcelo{Vamos a permitir anidacion del operador $\alfp$? Esta gramatica lo permite.}
%
%\begin{theorem} \label{rqfo-fo-cap}
%	$\rqfo(\fo)$ captures $\fp$ over the class of ordered structures.
%\end{theorem}
%

Our last goal in this section is to use the new characterization of $\fp$ to explore classes below it.
It was shown in \cite{I86,I88} that $\fo$ extended with a transitive closure operator captures $\nlog$. 
%Moreover, it was shown in \cite{I83} that $\so$ extended with a transitive closure operator captures $\pspace$. 
Inspired by this work, we show that a restricted version of $\rqfo$ can be used to capture $\shl$, the counting version of $\nlog$. 
Specifically, we use $\rqfo$ to define an operator for counting the number of paths in a directed graph which is exactly what we need for $\shl$. 
%Besides, we show that the same idea can be used to extend $\qso$ allowing to capture harder complexity classes. 

Given a relation signature $\R$, the set of transitive $\qfo$ formulas ($\tqfo$-formulas) is defined as an extension of $\qfo$ with the operator
%\begin{multline}
%	\label{eq-def-tqso}
%	\alpha := \varphi \, \mid \, s \, \mid \, (\alpha \add \alpha) \, \mid\, (\alpha \mult \alpha) \, \mid \, 
%	\sa{x} \alpha \, \mid\, \
%	\pa{x} \alpha \, \mid \\ 
%	\sa{X} \alpha \, \mid \, \pa{X} \alpha \, \mid \, [\pth \psi(\bar{x}, \bar{X},\bar{y}, \bar{Y})],
%\end{multline}
$
[\pth \psi(\bar{x},\bar{y})]
$ where $\psi(\x, \y)$ is an $\fo$-formula over $\R$, $\bar{x} = (x_1, \ldots, x_k)$, $\bar{y} = (y_1, \ldots, y_k)$ are tuples of pairwise distinct first-order variables. The semantics of $[\pth \psi(\bar{x},\bar{y})]$ can easily be defined in terms of $\rqfo(\fo)$ as follows. 
Given an $\R$-structure $\A$ with domain $A$, define a (directed) graph $\cG_{\psi}(\A) = (N,E)$ such that $N = A^k$ and for every pair $\bar b, \bar c \in N$, it holds that $(\bar b, \bar c) \in E$ if, and only if, $\A \models \psi(\bar b, \bar c)$.
Similar than for Example~\ref{ex:count-path}, we can count the paths of length at most $|A^k|$ with the formula $\beta_{\psi(\bar{x},\bar{y})}(\x, \y, \t, g)$:
\begin{multline*}
(\psi(\bar{x},\bar{y}) + \sa{\z} g(\x,\z,\t)\cdot \psi(\z,\y)) \cdot \varphi_{\text{\rm first-lex}}(\t) \ +\\
\sa{\t'} \varphi_{\text{succ-lex}}(\t',\t) \cdot \left(\sa{\x'} \sa{\y'} g(\x',\y',\t') \right)
\end{multline*}
where $\varphi_{\text{\rm first-lex}}$ and $\varphi_{\text{succ-lex}}$ are $\fo$-formulas for defining the first and successor predicates over tuples in $A^k$, following the lexicographic order induced by~$<$.
Then the semantics of the path operator is defined $\rqfo$ formula:
\[
[\pth \psi(\bar{x}, \bar{y})] := \sa{\t} (\varphi_{\text{\rm first}}(\t) \wedge \clfp{\beta_{\psi(\bar{x},\bar{y})}(\x,\y,\t,g)})
\]
In other words, $\sem{[\pth \psi(\bar{x}, \bar{y})]}(\A,v,V)$ counts the number of paths in the graph $\cG_{\psi}(\A)$ whose length is at most~$|A^k|$.
%As for the case of $\qso$, the logic $\tqso(\LL)$ is obtained by restricting $\varphi$ in \eqref{eq-def-tqso} to be a formula in $\LL$. Moreover, the logic $\tqfo$ is obtained by disallowing in \eqref{eq-def-tqso} formulas $\sa{X} \alpha$ and $\pa{X} \alpha$, and by only allowing  first-order free-variables in the formula $\psi$ used in $[\pth \psi]$ in \eqref{eq-def-tqso}. 
As it was previously said, the operator for counting paths is exactly what we need to capture $\shl$.
\begin{theorem} \label{tqfo-shl}
	$\tqfo(\fo)$ captures $\shl$ over the class of ordered structures.
\end{theorem}
The last result perfectly illustrates the benefits of the logical framework for the descriptive complexity of counting functions. 
The distinction of the language between boolean and quantitative allows us to define operators at the later that cannot be defined at the former. 
Indeed, it is not clear how to capture $\shl$ with $\qfo(\tc)$\footnote{$\tc$ corresponds to $\fo$-logic extended with transitive closure}, or any other extension of $\qfo(\fo)$ at the boolean level.
In contrast, with $\rqfo$ one can naturally define the path operator at the quantitative level and easily capture $\shl$.
Therefore, Theorem~\ref{tqfo-shl} is good example of how to go beyond classical logic to have a better understanding of the expressivity of counting.

\begin{comment}

\begin{theorem} \label{tqso-fo-fpsace}
	$\tqso$ and $\tqso(\fo)$ captures $\fpspace$ over the class of ordered structures.
\end{theorem}

\begin{theorem} \label{tqfo-subseteq}
	$\tqfo(\fo) \subseteq \rqfo(\fo)$.
\end{theorem}
Given tuples $\bar x = (x_1, \ldots, x_\ell)$, $\bar y = (y_1, \ldots, y_\ell)$ of pairwise distinct first-order variables, define formula $\varphi_{\text{\rm lex}}(\bar x, \bar y)$ as follows:
$$
\bigvee_{i = 1}^\ell \bigg(\bigwedge_{j = 1}^{i -1} x_i = y_i\bigg) \wedge x_i < y_i.
$$
That is, $\varphi_{\text{\rm lex}}(\bar x, \bar y)$ holds if $\bar x$ is smaller than $\bar y$ in the lexicographic order on tuples with $\ell$ elements induced by the built-in order of each structure. Moreover, define formula $\varphi_{\text{\rm succ}}(\bar x, \bar y)$ as follows:
$$
\varphi_{\text{\rm lex}}(\bar x, \bar y) \wedge \neg \exists \bar z \, (\varphi_{\text{\rm lex}}(\bar x, \bar z) \wedge \varphi_{\text{\rm lex}}(\bar z, \bar y)).
$$
That is, $\varphi_{\text{\rm succ}}(\bar x, \bar y)$ holds if $\bar y$ is the successor of $\bar x$ in the lexicographic order on tuples with $\ell$ elements.
With this notation, define $\tqsos$ to be the restriction of $\tqso$ where each occurrence of the operator ${\bf path}$ is of the form:
$$
[\pth (\psi(\bar{x}, \bar{X},\bar{y}, \bar{Y}) \wedge \varphi_{\text{\rm succ}}(\bar x, \bar y))],
$$
where $\psi(\bar{x}, \bar{X},\bar{y}, \bar{Y})$ satisfies the same conditions as in grammar \eqref{eq-def-tqso}. Then we have that:
\begin{theorem} \label{tqsos-shp}
	$\tqsos(\fo)$ captures $\shp$ over the class of ordered structures.
\end{theorem}

\end{comment}

%
%\marcelo{Puede que este equivocado, pero me parece que teniamos una forma de capturar $\shp$ usando el operator ${\bf path}$. Pero no logro recordar como se hacia esto, y me parece que lo que habiamos escrito antes en esta seccion estaba equivocado: ``$\tqso(\fo)$ captures $\shp$ over the class of ordered structures". Claro que yo puedo estar usando una definicion distinta de $\tqso(\fo)$.}


%We also define the set of transitive $\qso$ formulas ($\tqso$-formulas) using the following grammar:
%\begin{multline*}
%%	\label{eq-def-tqso}
%	\alpha := \varphi \ \mid \ s \ \mid \ (\alpha \add \alpha) \ \mid\ (\alpha \mult \alpha) \ \mid \\ \sa{x} \alpha \ \mid \ \pa{x} \alpha \ \mid \ \sa{X} \alpha \ \mid \ \pa{X} \alpha \ \mid \ [\pth \varphi]
%\end{multline*}
%
%
% 
%We define the operator {\bf path} as follows. Let $\A$ be an ordered structure. Given a formula $\psi(\bar{x},\bar{y})$, where $\vert \bar{x} \vert = \vert \bar{y} \vert = k$ let ${\cal G} = ({\cal V},\cal{E})$ be induced graph over the set of vertices ${\cal V} = A^k$, and for every $\bar{a},\bar{b}\in A^k$ it holds that ${\cal E}(\bar{a},\bar{b})$ if and only if $\A \models \psi(\bar{a},\bar{b})$. To formalize the semantics for this operator, let $n = \vert A^k \vert$.
%For a given first order assignment $v$ and a second order asssignment $V$, let $\bar{a} = v(\bar{x})$ and $\bar{b} = v(\bar{y})$, and $\sem{[\pth\, \psi(\bar{x},\bar{y})]}(\A,v,V)$ will take the value of the number of paths of size less or equal to $n$ from $\bar{a}$ to $\bar{b}$ in the graph ${\cal G}$. This operator lets us define the set of transitive $\qfo$ formulas over $\R$ ($\tqfo$-formulas) using the following grammar:
%\begin{multline*} 
%%	\label{eq-def-tqfo}
%	\alpha := \varphi \ \mid \ s \ \mid \ (\alpha \add \alpha) \ \mid\ (\alpha \mult \alpha) \\ \mid \ \sa{x} \alpha \ \mid \ \pa{x} \alpha \ \mid \ [\pth \varphi]
%\end{multline*}
%where $\varphi$ is an $\fo$-formula over $\R$, $s \in \bbN$ and $x \in \fv$.
%
%We also define the set of transitive $\qso$ formulas ($\tqso$-formulas) using the following grammar:
%\begin{multline*}
%%	\label{eq-def-tqso}
%	\alpha := \varphi \ \mid \ s \ \mid \ (\alpha \add \alpha) \ \mid\ (\alpha \mult \alpha) \ \mid \\ \sa{x} \alpha \ \mid \ \pa{x} \alpha \ \mid \ \sa{X} \alpha \ \mid \ \pa{X} \alpha \ \mid \ [\pth \varphi]
%\end{multline*}
%where $\varphi$ is an $\so$-formula over $\R$, $s \in \bbN$, $x \in \fv$ and $X \in \sv$.
%\begin{theorem} \label{so-rec}
%	Given a positive $\fo$ formula $\varphi(\bar{x},R)$ and a $\qfo$ formula $\alpha(\bar{x})$, there exists a $\qso$ formula $\beta(\bar{x})$ such that $\sem{[\alfp\varphi(\bar{x},R)\mid \alpha(\bar{x},R)](\bar{x})} = \sem{\beta(\bar{x})}$.
%\end{theorem}
%
%\begin{theorem} \label{tqfo-fo-cap}
%	$\tqfo(\fo)$ captures $\shl$ over the class of ordered structures.
%\end{theorem}
%
%\begin{theorem} \label{tqso-fo-cap}
%	$\tqso(\fo)$ captures $\shp$ over the class of ordered structures.
%\end{theorem}




\section{Conclusion}
%!TEX root = main.tex

We proposed a framework based on Weighted Logics to develop a descriptive complexity theory for complexity classes of functions.
%In particular, we show how this framework can be used to capture fundamental counting complexity classes such as $\fp$, $\shp$ and $\fpspace$, among others. Moreover, we use it to define a hierarchy inside $\shp$, identifying counting complexity classes with good closure and approximation properties, and which admit natural complete problems. Finally, by adding recursion to the framework, we show how to capture lower counting complexity classes such as $\shl$.	
We consider the results of this paper as a first step in this direction.
%towards a descriptive complexity theory for complexity classes of functions. �
Consequently, there are several directions for future research, some of which are mentioned here. 
$\totp$ is an interesting counting complexity class as it naturally defines a class of functions in $\shp$ with easy decision counterparts. However, we do not have a logical characterization of this class.
%for which we are missing a logical characterization. 
Similarly, we are missing logical characterizations of other fundamental complexity classes such as $\spanl$~\cite{AlvarezJ93}. We would also like to define a larger syntactic subclass of $\shp$ where each function admits an FPRAS; notice that $\cpm$ is an important problem admitting an FPRAS\cite{JSV04} that is not included in the classes defined in Section \ref{sec-clo}. Moreover, by following the approach proposed in
\cite{I83}, we would like to include second-order free variables in the operator for counting paths introduced in Section \ref{sec:beyond}, in order to have alternative ways to capture $\fpspace$ and even $\shp$. 

Finally, an open problem is to understand the connection of this framework with enumeration complexity classes. For example, in~\cite{durandS11} the enumeration of assignments for first-order formulas was studied following the $\sfo$-hierarchy, and in~\cite{AmarilliBJM17} the efficient enumeration (i.e. constant delay enumeration~\cite{Segoufin13}) was shown for a particular class of circuits. It would be interesting then to adapt the $\eqso(\fo)$-hierarchy for the context of enumeration and identify subclasses that also admit good properties in terms of enumeration.




%\section*{Acknowledgments}
%The authors would like to thank ~\cite{DrosteG07}



\bibliographystyle{IEEEtran}
\bibliography{biblio}

\newpage

\onecolumn
\appendix

\subsection{Notation for the appendix}
For a given signature $\R$, we define $\ostr[\R]^*$ as $$\ostr[\R]^* = \{(\A,v,V) \mid \A\in\ostr[\R]\text{, $v$ ($V$) is a first-order (second-order) assignment for $\A$}  \}.$$
The {\em conditional count} symbol $(\varphi \mapsto \alpha)$ is defined as $(\neg\varphi + (\varphi\cdot\alpha))$ for given $\so$ formula $\varphi$ and $\qso$ formula $\alpha$. Note that for each $(\A,v,V) \in \ostr[\R]^*$, 
$$
\sem{(\varphi \mapsto \alpha)}(\A,v,V) = 
\begin{cases}
\sem{\alpha}(\A,v,V) &\text{if } (\A,v,V)\models\varphi,\\
0 &\text{otherwise}.
\end{cases}
$$
We will use the symbol $<$ also to denote the lexicographic order over same-sized tuples. If $\bar{x} = (x_1,\ldots,x_m)$ and $\bar{y} = (y_1,\ldots,y_m)$ are tuples of first-order variables, we denote $\bar{x} < \bar{y}$ for the formula $\bigvee_{i = 1}^m[\bigwedge_{j = 1}^{i-1}x_j = y_j \wedge x_i < y_i]$. Similarly, we use $=$ to denote equality between tuples, as $\bar{x} = \bar{y}$ denotes $\bigwedge_{i = 1}^m(x_i = y_i)$, and also $\bar{x}\leq\bar{y}$ denotes $\bar{x} < \bar{y} \vee \bar{x} = \bar{y}$. We also denote $\min(\bar{x}) := \forall\bar{y}(\bar{x} \leq \bar{y})$.

If $\bar{x} = (x_1,\ldots,x_m)$ ($\bar{X} = (X_1,\ldots,X_m)$) is a tuple of first-order (second-order) variables, we denote $\sa{\bar{x}}\alpha$ for $\sa{x_1}\cdots\sa{x_m}\alpha$ and $\sa{\bar{X}}\alpha$ for $\sa{X_1}\cdots\sa{X_m}\alpha$ for each $\qso$ formula $\alpha$. We also denote $\length{\bar{x}}$ as the size of $\bar{x}$ ($\length{\bar{X}}$ as the size of $\bar{X}$). In this case, $\length{\bar{x}} = m$ ($\length{\bar{X}} = m$).

\bigskip

\subsection{Proofs from Section~\ref{sec:complexity}}

\medskip

%%% DEMOSTRACION DE SQSO(FO) y #P
\subsection*{Proof of Proposition~\ref{prop:capture-shP}}

We will construct a recursive non-deterministic algorithm $M_{\alpha}$ for each $\eqso(\fo)$ formula $\alpha$ over a signature $\R$. This machine, on input $(\A,v,V)$ accepts in $\sem{\alpha}(\A,v,V)$ of its non-deterministic paths for each $(\A,v,V) \in \ostr[\R]^*$. Suppose $\A$ has domain $A$. If $\alpha$ is a $\fo$-formula $\varphi$, then the algorithm checks if $(\A,v,V)\models\varphi$ deterministically in polynomial time, and accepts if and only if it holds true. If $\alpha$ is a constant $s$, it produces $s$ branches and accepts in all of them. If $\alpha = (\beta \add \gamma)$, then it chooses between 0 or 1, if it is 0 (1), it simulates $M_{\beta}$ ($M_{\gamma}$) on input $(\A,v,V)$. 
%If $\alpha = (\beta \mult \gamma)$, it simulates $M_{\beta}$ on input $(\A,v,V)$ and on each accepting path, it continues simulating $M_{\gamma}$ on input $(\A,v,V)$.
If $\alpha = \sa{x}\beta$, it chooses $a\in A$ and simulates $M_{\beta}$ on input $(\A,v[a/x],V)$.
%If $\alpha = \pa{x}\beta$, it simulates $M_{\beta}$ on input $(\A,v[a/x],V)$ consecutively for each $a\in A$. 
If $\alpha = \sa{X}\beta$, it chooses $B\in A^{arity(X)}$ and simulates $M_{\beta}$ on input $(\A,v,V[B/X])$. This covers all possible cases for $\alpha$. Let $\alpha$ be a formula in $\eqso(\fo)$ over a signature $\R$ and let $f$ be a function over $\R$ such that $f(\enc(\A))$ is equal to the accepting paths of $M_{\alpha}$ on input $(\A,v,V)$ for some $(\A,v,V) \in \ostr[\R]^*$. We have that $f$ is a $\shp$-function over $\R$ and $f(\enc(\A)) = \sem{\alpha}(\A)$ for every $\A\in\ostr[\R]$.

For the other direction, note that Saluja et al.~\cite{SalujaST95} proved that $\shp = \sfo$. 
%We also have that $\sqso(\fo)$ captures $\#\fo$ over ordered structures so for each $f\in \shp$ let $\alpha \in \sqso(\fo)$ be its corresponding formula. 
Since a function in $\sfo$ can also be defined $\eqso(\fo)$ (see Section~\ref{sec:previous}), the condition holds. \qed

\medskip

%%% DEMOSTRACION DE SQSO(ESO) y span-P
\subsection*{Proof of Proposition~\ref{prop:capture-spanP}}

Similar than the previous proof, we will construct a recursive non-deterministic algorithm $M_{\alpha}$ for each $\eqso(\eso)$ formula $\alpha$ over a signature $\R$. This machine, on input $(\A,v,V)$, non-deterministically produces $\sem{\alpha}(\A,v,V)$ distinct accepting outputs for each $(\A,v,V) \in \ostr[\R]^*$. Suppose $\A$ has domain $A$. 
If $\alpha$ is a $\eso$-formula $\varphi$ it checks if $(\A,v,V)\models\varphi$ non-deterministically in polynomial time \cite{fagin1974generalized}, and accepts if and only if the condition holds true. 
If $\alpha$ is a constant $s$, then the algorithm produces $s$ branches and accepts in all of them. 
If $\alpha = (\beta \add \gamma)$, then it chooses between 0 or 1, if it is 0 (1), it simulates $M_{\beta}$ ($M_{\gamma}$) on input $(\A,v,V)$.  
If $\alpha = \sa{x}\beta$, it chooses $a\in A$ and simulates $M_{\beta}$ on input $(\A,v[a/x],V)$. 
If $\alpha = \sa{X} \beta$, it chooses $B\in A^{\arity(X)}$ and simulates $M_{\beta}$ on input $(\A,v,V[B/X])$. 
This covers all possible cases for $\alpha$.
Additionally, the algorithm produces a different output on each path. This can be done by printing the trace of all the non-deterministic choices.
However, when the algorithm starts checking whether $(\A,v,V)\models\varphi$ for some $\eso$ formula $\varphi$, it stops printing in the output tape. This way the algorithm produces exactly one output from that point onwards.
Let $\alpha$ be a formula in $\eqso(\eso)$ over a signature $\R$ and let $f$ be a function over $\R$ such that $f(\enc(\A))$ is equal to the number of accepting outputs of $M_{\alpha}$ on input $(\A,v,V)$ for some $(\A,v,V) \in \ostr[\R]^*$. 
We have that $f$ is a $\spp$ function over $\R$ and that $f(\enc(\A)) = \sem{\alpha}(\A)$ for every $\A\in\ostr[\R]$.

For the other direction, Compton et al.~\cite{ComptonG96} proved that $\spp = \#\eso$. Since a function in $\#\eso$ can also be defined in $\eqso(\eso)$, then $\eqso(\eso)$ captures $\spp$ over ordered structures.

\medskip

%%% DEMOSTRACION DE QFO(LFP) y FP
\subsection*{Proof of Theorem~\ref{theo:capture-fp}}

For the first condition, let $\alpha\in\qfo(\lfp)$ over some signature $\R$. Let $f$ be a function over $\R$ defined by the following procedure. Let $\enc(\A)$ be an input, where $\A$ is an ordered structure over $\R$ with domain $A = \{1,\ldots,n\}$. In the procedure we slightly extend the grammar of $\qfo(\lfp)$ to include constants. We replace each first order sum and first order product in $\alpha$ by an expansion using the elements in $A$. This is, $\sa{x} \beta(x)$ is replaced by $(\beta(1)+\cdots+\beta(n))$ and $\pa{x}\beta(x)$ is replaced by $(\beta(1)\cdot\,\cdots\,\cdot\beta(n))$. Then each sub-formula $\varphi\in\lfp$ in $\alpha$ is evaluated in polynomial time and replaced by 1 if $\A\models\varphi$ and by 0 otherwise. The resulting formula is an arithmetic expression of polynomial size which is evaluated and lastly given as output. Note that $f\in\fp$ and $f(\enc(\A)) = \sem{\alpha}(\A)$.
	
For the second condition, let $f\in \fp$ defined over some signature $\R$.
Let $\ell\in\nat$ be such that for each $\A\in\ostr[\R]$, $\lceil\log_2 f(\enc(\A)) \rceil \leq n^\ell$, where $\A$ has a domain of size $n$.
Let $\bar{x} = (x_1,\ldots,x_{\ell})$.
Consider a procedure that receives $\enc(\A)$ and an assignation $\bar{a}$ to $\bar{x}$. Let $m$ be the position of $\bar{a}$ in the lexicographic order of the tuples in $A^{\ell}$. The procedure then computes the $m$-th bit of $f(\enc(\A))$, from least to most significant. Since this procedure works in polynomial time, it can be described by an $\lfp$ formula $\Phi(\bar{x})$. Then we use
$$
\alpha = \sa{\bar{x}} \Phi(\bar{x})\cdot\varphi(\bar{x}),
$$
where $\varphi(\bar{x}) := \pa{\bar{y}}(\bar{y} < \bar{x} \mapsto 2).$ Note that if $\bar{a} \in A^{\ell}$ is the $m$-th tuple in the given order (starting from 0), then $\sem{\varphi(\bar{a})}(\A) = 2^{m}$. Adding these values for each $\bar{a}\in A^{\ell}$ gives exactly $f(\enc(\A))$. Then, $\alpha$ is in $\qfo(\lfp)$ over $\R$ and $\sem{\alpha}(\A) = f(\enc(\A))$.

\medskip

%%% DEMOSTRACION DE QSO(PFP) y FPSPACE
\subsection*{Proof of Theorem~\ref{theo:capture-fpspace}}

To show how to evaluate a $\qso(\pfp)$-formula, we will construct a recursive non-deterministic algorithm $M_{\alpha}$ for each $\qso(\pfp)$ formula $\alpha$ over a signature $\R$. This machine runs in non-deterministic polynomial space and, on input $(\A,v,V)$, accepts in $\sem{\alpha}(\A,v,V)$ of its non-deterministic paths for each $(\A,v,V) \in \ostr[\R]^*$. Suppose $\A$ has domain $A$. If $\alpha$ is a $\pfp$-formula $\varphi$, then the algorithm checks if $(\A,v,V)\models\varphi$ deterministically in polynomial space~\cite{L04}, and accepts if and only if it holds true. If $\alpha$ is a constant $s$, it produces $s$ branches and accepts in all of them. If $\alpha = (\beta \add \gamma)$, then it chooses between 0 or 1, if it is 0 (1), it simulates $M_{\beta}$ ($M_{\gamma}$) on input $(\A,v,V)$. If $\alpha = (\beta \mult \gamma)$, it simulates $M_{\beta}$ on input $(\A,v,V)$ and on each accepting path, it continues simulating $M_{\gamma}$ on input $(\A,v,V)$.
If $\alpha = \sa{x}\beta$, it chooses $a\in A$ and simulates $M_{\beta}$ on input $(\A,v[a/x],V)$. If $\alpha = \pa{x}\beta$, it simulates $M_{\beta}$ on input $(\A,v[a/x],V)$ consecutively for each $a\in A$. If $\alpha = \sa{X}\beta$, it chooses $B\in A^{arity(X)}$ and simulates $M_{\beta}$ on input $(\A,v,V[B/X])$. If $\alpha = \pa{X}\beta$, it simulates $M_{\beta}$ on input $(\A,v,V[B/X])$ consecutively for each $B\in A^{arity(X)}$. This covers all possible cases for $\alpha$, and each of these steps can be computed in polynomial space. Let $\alpha$ be a formula in $\qso(\pfp)$ over a signature $\R$ and let $f$ be a function over $\R$ such that $f(\enc(\A))$ is equal to the accepting paths of $M_{\alpha}$ on input $(\A,v,V)$ for some $(\A,v,V) \in \ostr[\R]^*$. We have that $f$ is a $\shpspace$ function over $\R$, which implies that $f$ is also a $\fpspace$ function over $\R$, by the fact that $\shpspace = \fpspace$ \cite{Ladner89}, and that $f(\enc(\A)) = \sem{\alpha}(\A)$ for every $\A\in\ostr[\R]$.

For the second condition, let $f\in \fpspace$ defined over some $\R$. Let $\ell\in\nat$ be such that for each $\A\in\ostr[\R]$, $\lceil\log_2 f(\enc(\A)) \rceil \leq 2^{n^\ell}$, where $\A$ has a domain of size $n$. Let $X$ be a second-order variable of arity $\ell$. Consider a linear order over predicates of arity $\ell$ given by the formula 
$$
\varphi_{<}(X,Y) = \exists\bar{u}\big[\neg X(\bar{u})\wedge Y(\bar{u})\wedge \forall\bar{v}\big(
\bar{u}<\bar{v}\to(X(\bar{u})\iff Y(\bar{v}))\big)\big].
$$
Consider a polynomial space algorithm over $\R\cup\{X\}$ that receives a $\R\cup\{X\}$-structure $\A'$. Let $X^{\A'}$ be the $p$-th subset in the given order (we consider 0-indexation). The algorithm computes $f(\enc(\A))$ until $(2^{n^{\ell}}-p)$ characters have been written, but each character of the output is written over the last one. It accepts if and only if the last character written is $1$. Since this algorithm works in polynomial space, it can be described in $\pfp$ \cite{AbiteboulV89} by a formula $\Phi(X)$.

Note that the formula also represents the following.
For each $(\A,v,V)\in\ostr[\R]^*$, let $V(X)$ be the $p$-th subset with respect to this order and let $w$ be a string of size $2^{n^{\ell}}$ which represents the value of $f(\enc(\A))$ in binary. Then $(\A,v,V)\models\Phi(X)$ if and only if the $p$-th bit in $w$ is 1, from least to most significant. We define
$$
\alpha := \sa{X}(\Phi(X)\mult\varphi(X)),
$$ 
where $\varphi(X) = \pa{Y}(\varphi_{<}(Y,X)\mapsto 2)$. Note that for each $B\subseteq A^{\ell}$ such that $B$ is the $m$-th element in the given order, $\sem{\varphi(B)}(\A) = 2^m$. Therefore, $\alpha\in\qso(\pfp)$ and $\sem{\alpha}(\A) = f(\enc(\A))$.

\medskip


%%% DEMOSTRACION DE QFO(PFP) y FPSPACE(poly)
\subsection*{Proof of Corollary~\ref{cor:capture-fpspace-poly}}

For the first condition, let $\alpha\in\qfo(\pfp)$ over some signature $\R$. Let $f$ be a function over $\R$ defined by the following procedure. Let $\enc(\A)$ be an input, where $\A$ is an ordered structure over $\R$ with domain $A = \{1,\ldots,n\}$. In the procedure we slightly extend the grammar of $\qfo(\pfp)$ to include constants. We replace each first order sum and first order product in $\alpha$ by an expansion using the elements in $A$. This is, $\sa{x} \beta(x)$ is replaced by $(\beta(1)+\cdots+\beta(n))$ and $\pa{x}\beta(x)$ is replaced by $(\beta(1)\cdot\,\cdots\,\cdot\beta(n))$. Then each sub-formula $\varphi\in\pfp$ in $\alpha$ is evaluated in polynomial space and replaced by 1 if $\A\models\varphi$ and by 0 otherwise. The resulting formula is an arithmetic expression of polynomial size which is evaluated and lastly given as output. Note that $f\in\nfpspace$ and $f(\enc(\A)) = \sem{\alpha}(\A)$.

For the second condition, let $f\in \fpspace$ defined over some signature $\R$.
Let $\ell\in\nat$ be such that for each $\A\in\ostr[\R]$, $\lceil\log_2 f(\enc(\A)) \rceil \leq n^\ell$, where $\A$ has a domain of size $n$.
Let $\bar{x} = (x_1,\ldots,x_{\ell})$.
Consider a procedure that receives $\enc(\A)$ and an assignation $\bar{a}$ to $\bar{x}$. Let $m$ be the position of $\bar{a}$ in the lexicographic order of the tuples in $A^{\ell}$. The procedure then computes the $m$-th bit of $f(\enc(\A))$, from least to most significant. Since this procedure works in polynomial space, it can be described by an $\pfp$ formula $\Phi(\bar{x})$. Then we use
$$
\alpha = \sa{\bar{x}} \Phi(\bar{x})\cdot\varphi(\bar{x}),
$$
where $\varphi(\bar{x}) := \pa{\bar{y}}(\bar{y} < \bar{x} \mapsto 2).$ Note that if $\bar{a} \in A^{\ell}$ is the $m$-th tuple in the given order (starting from 0), then $\sem{\varphi(\bar{a})}(\A) = 2^{m}$. Adding these values for each $\bar{a}\in A^{\ell}$ gives exactly $f(\enc(\A))$. Then, $\alpha$ is in $\qfo(\pfp)$ over $\R$ and $\sem{\alpha}(\A) = f(\enc(\A))$.


\subsection{Proofs from Section~\ref{sec:syntactic}}
\subsection*{Proof of Proposition \ref{prop:eqso-hierarchy-proper}} %V.1
Classes QE0, QE1, QU1, QE2, QU2 = QFO form a properly contained hierarchy:\vspace{1em}

We give this proof in various parts.

We show here that $\QE{0} \subsetneq \E{1}$. The inclusion is trivial, so we will only prove that it is proper. 



\subsection*{Proof of Theorem \ref{theo:eqso-sum-normal-form}} %V.2
Every formula in $\eqso(\LL)$ can be rewritten in \emph{sum normal form}. Furthermore, there exists a formula $\alpha \in \QE{1}$ such that there does not exist a formula in $\Sigma$-prenex normal form in $\QE{1}$ equivalent to $\alpha$:\vspace{1em}

\subsection*{Proof of Theorem \ref{theo:prenex}}%V.3
If $\logu{1} \subseteq \LL$, then for every formula $\alpha \in \eqso(\LL)$ there exists a formula $\beta \in \eqso(\LL)$ equivalent to $\alpha$ in $\Sigma$-prenex normal form:\vspace{1em}

\subsection*{Proof of Proposition \ref{prop:sharp-Q-rel}}%V.4
The classes $\E{0}$ and $\E{1}$ are strictly contained in $\QE{0}$ and $\QE{1}$, respectively. Moreover, the classes $\U{1}$, $\E{2}$, and $\U{2}$ are equivalent with $\QU{1}$, $\QE{2}$, and $\QU{2}$, respectively:\vspace{1em}

\subsection*{Proof of Proposition \ref{prop:qe0-fp-qe1-totp-fptras}} %V.5
All functions defined in $\eqso(\loge{0})$ and $\eqso(\loge{1})$ can be computed in $\fp$ and $\totp$, respectively. Furthermore, every function defined in $\eqso(\loge{1})$ has a FPTRAS:\vspace{1em}

\subsection*{Proof of Theorem \ref{theo:binary-prod}}%V.6
If $\LL$ is closed under conjunction, then binary product can be defined in $\eqso(\LL)$:\vspace{1em}

\subsection*{Proof of Theorem \ref{sub-pnp}}%V.7
If $\eqso(\loge{i})$ or $\eqso(\logu{i})$ is closed under subtraction for $i > 0$, then {\sc P} = {\sc NP}:\vspace{1em}

Suppose that $\E{1}$ is closed under substraction, that is, for each pair of functions $f,g\in \E{1}$, there exists $h\in\E{1}$ such that $(f\dotminus g)(\A) = h(\A)$ for each $\A\in\str$.

Let $\A = \langle A, S_1^\A, S_2^\A, S_3^\A, S_4^\A, \leq^\A \rangle$ be an $\L-$structure that represents an instance of a 3DNF formula $\Phi$, where $A$ is the set of variables mentioned in $\Phi$, $S_i^\A$ is a ternary relation described as follows, for each $i\in\{1,2,3,4\}$:
\begin{eqnarray*}
	S_1^\A &=& \{(a_1,a_2,a_3)\mid (\neg a_1 \wedge \neg a_2 \wedge \neg a_3) \mbox{ appears as a disjunct in }\Phi\},\\
	S_2^\A &=& \{(a_1,a_2,a_3)\mid ( a_1 \wedge \neg a_2 \wedge \neg a_3) \mbox{ appears as a disjunct in }\Phi\},\\
	S_3^\A &=& \{(a_1,a_2,a_3)\mid ( a_1 \wedge  a_2 \wedge \neg a_3) \mbox{ appears as a disjunct in }\Phi\},\\
	S_4^\A &=& \{(a_1,a_2,a_3)\mid ( a_1 \wedge  a_2 \wedge  a_3) \mbox{ appears as a disjunct in }\Phi\}.
\end{eqnarray*}
Now we define $f_{\#3DNF} = f_{\psi(T)}$ where
\begin{multline*}
\psi(T) = \exists x \exists y \exists z\, [(S_1(x,y,z) \wedge \neg T(x) \wedge \neg T(y) \wedge \neg T(z)) \vee (S_2(x,y,z) \wedge T(x) \wedge \neg T(y) \wedge \neg T(z)) \, \vee \\ (S_3(x,y,z) \wedge T(x) \wedge T(y) \wedge \neg T(z)) \vee (S_4(x,y,z) \wedge T(x) \wedge T(y) \wedge T(z))].
\end{multline*}
Note that $f_{\#3DNF} \in \#\Sigma_1$. Let $f_{all} = f_{\exists x\:\varphi(x,X)}$, where
$$
\varphi(x,X) = (T(x) \vee \neg T(x)).
$$
Note that $f_{all}$ counts every possible truth assignment (satisfying or not) to a 3DNF formula. Given that $f_{\#3DNF}, f_{all} \in \E{1}$, we have by our initial assumption that $f_{all}-f_{\#3DNF} \in \E{1}$. Let $h\in\E{1}$ be such that $h = f_{all}-f_{\#3DNF}$. For each structure $\A$ that represents a 3DNF formula $\psi$, it holds that $h(\A) = f_{all}(\A)-f_{\#3DNF}(\A) = 0$ if and only if $\psi$ is a tautology, so the decision version $L_h$ of $f_{all}-f_{\#3DNF}$ is $\conp$-complete. However, as we showed previously in Theorem \ref{decisionptime}, since $h\in\E{1}$, we have that $L_h \in \ptime$. Then, $\conp \subseteq \ptime$, from which we conclude that $\ptime = \np$.


\subsection*{Proof of Theorem \ref{sigmafo-minusone}}%V.8

Let $\alpha$ be a $\eqso(\logex{1})$ formula over a signature $\R$. We will define a $\eqso(\logex{1})$ formula $\kappa(\alpha)$ such that for each finite structure $A$ over $\R$: $\sem{\kappa(\alpha)}(\A) = \sem{\alpha}(\A) \dotminus 1$. Let $\tau(\alpha) = \beta_1\add\cdots\add\beta_n$ be defined as in \ref{eqso-sigma-zero-in-fp}, that is, each $\beta_i$ is equal to $1$, $\varphi$, $\sa{\bar{x}}1$, $\sa{\bar{x}}\varphi$, $\sa{\bar{X}}1$, $\sa{\bar{X}}\varphi$, $\sa{\bar{X}}\sa{\bar{x}}1$ or $\sa{\bar{X}}\sa{\bar{x}}\varphi$, where $\varphi$ is a formula in $\logex{1}$ and $\bar{x}$ ($\bar{X}$) is a tuple of first-order (second-order) variables. We further assume that each $\beta_i$ is either $\sa{\bar{x}}\varphi$ or $\sa{\bar{X}}\sa{\bar{x}}\varphi$, since each $1$, $\varphi$ and $\sa{\bar{X}}\varphi$ can be replaced by the formulas $\sa{x}\min(x)$, $\sa{x}(\varphi\wedge\min(x))$ and $\sa{x}\sa{\bar{X}}(\varphi\wedge\min(x))$ respectively, while maintaining the value of $\sem{\alpha}(\A)$. 

The intuition behind our reasoning is separated in three points: (1) For each $\beta_i$ of the form $\sa{\bar{x}}\varphi$, the formula $\kappa(\beta_i)$ will count every tuple $\bar{x}$ that satisfies $\varphi$ except for the lexicographically smallest one. (2) For each $\beta_i$ of the form $\sa{\bar{X}}\sa{\bar{x}}\varphi$, the formula $\kappa(\beta_i)$ will isolate the smallest $\bar{X}$ that satisfies $\varphi$, and exclude the lexicographically smallest tuple $\bar{x}$ that satisfies $\varphi(\bar{X})$. And (3) if $\alpha = (\beta\add\sa{\bar{x}}\varphi)$ or $\alpha = (\beta\add\sa{\bar{X}}\sa{\bar{x}}\varphi)$, the formula $\kappa(\alpha)$ will exclude the lexicographically smallest tuple that satisfies $\varphi$ if and only if $\beta$ is equal to 0.

\vspace{1em}

For each $\alpha$ in $\eqso(\logex{1})$ such that $\alpha = \sa{\bar{x}}\exists\bar{y}\,\varphi(\bar{x},\bar{y})$ for some quantifier-free formula $\varphi$, we define $\kappa(\alpha) = \sa{\bar{x}}\exists\bar{y}[\varphi(\bar{x},\bar{y})\wedge\exists\bar{z}(\varphi(\bar{z},\bar{y})\wedge\bar{z} < \bar{x})]$, which is in $\eqso(\logex{1})$ and fulfils the desired condition.

\vspace{1em}

For each $\alpha$ in $\eqso(\logex{1})$ such that $\alpha = \sa{\bar{X}}\sa{\bar{x}}\exists\bar{y}\,\varphi(\bar{X},\bar{x},\bar{y})$ for some quantifier-free formula $\varphi$, we define $\kappa(\alpha)$ procedurally as follows: Let $\bar{x} = (x_1,\ldots,x_{\length(\bar{x})})$ and $\bar{X} = (X_1,\ldots,X_{\length(\bar{X})})$. We suppose that $\varphi$ is in a DNF form that leaves formulas that do not mention $\bar{X}$ as is. If it is not, we convert $\varphi$ to this form with a standard transformation algorithm. Let $\varphi(\bar{X},\bar{x},\bar{y}) = \bigvee_{i = 1}^m\varphi_i(\bar{X},\bar{x},\bar{y})$ where each $\varphi_i$ has the form:
$$
\varphi_i(\bar{X},\bar{x},\bar{y}) = (\text{conjunction of $X_j$'s}) \wedge (\text{conjunction of $\neg X_j$'s})  \wedge (\text{an $\fo$ formula that does not mention any $X_j$}).
$$
Define $\varphi_i^{+}$, $\varphi_i^{-}$ and $\varphi_i^{\fo}$ as the formulas mentioned above. 

In our procedure we assume that in $\varphi_i^{+}\wedge\varphi_i^{-}$ every first-order variable from $\bar{x},\bar{y}$ is mentioned at most once. If not, we add new first-order variables from $\fv$ to $\bar{y}$ so that no variable is repeated, replace them in $\varphi_i^{+}\wedge\varphi_i^{-}$ and add their respective equalities in $\varphi_i^{\fo}$ so that the new formula is equivalent. For example, if $\bar{x} = x$, $\bar{y} = y$ and $\varphi_i = X(x,y)\wedge \neg X(x,x) \wedge x < y$, then we redefine $\bar{y} = (y,v_1,v_2,v_3,v_4)$ and $\varphi_i := X(v_1,v_2) \wedge \neg X(v_3,v_4) \wedge x < y \wedge v_1 = x \wedge v_2 = y \wedge v_3 = x \wedge v_4 = x.$ 

Furthermore, we assume that $\varphi_i^{\fo}$ defines an ordered partition of the variables in the tuple $(\bar{x},\bar{y})$. For example, let $\bar{x} = (x_1,x_2,x_3,x_4)$. A possible ordered partition would be defined by the formula $\theta(\bar{x}) = x_2 < x_1 \wedge x_1 = x_4 \wedge x_4 < x_3$. On the other hand, the formula $\theta'(\bar{x}) = x_1 < x_2 \wedge x_1 < x_4 \wedge x_2 = x_3$ does not define an ordered partition since both $\{x_1\}<\{x_2,x_3\}<\{x_4\}$ and $\{x_1\} < \{x_2,x_3,x_4\}$ satisfy $\theta'$.
For a given $k$, we define $\cB_k$ as the number of possible ordered partitions for a set of size $k$. 
If $\varphi_i^{\fo}$ does not define an ordered partition, then for $1 \leq j \leq \cB_{\length(\bar{x},\bar{y})}$ 
let $\theta^j(\bar{x},\bar{y})$ be a formula that defines an ordered partition over $(\bar{x},\bar{y})$. The formula $\varphi_i(\bar{X},\bar{x},\bar{y})$ is converted into $\bigvee_{j = 1}^{\cB_{\length(\bar{x},\bar{y})}}\varphi_i(\bar{X},\bar{x},\bar{y}) \wedge \theta^j(\bar{x},\bar{y})$. Consider each of these disjuncts as a new $\varphi_i$.

After the previous assumptions, we do the following: for each $X_j\in \bar{X}$ we check every instance of $X_j(\bar{w})$ in $\varphi^{+}_i$ and every instance of $X_j(\bar{z})$ in $\varphi^{-}_i$, where $\bar{w}$ and $\bar{z}$ are subtuples of $(\bar{x},\bar{y})$. If the ordered partition in $\varphi^{\fo}_i$ satisfies $\bar{w} = \bar{z}$, the entire formula $\varphi_i$ is removed from $\varphi$.

Now let $\bar{v}$ be the tuple of every first-order variable mentioned in $\varphi_i^{+}$ and let $\bar{u}$ be such that $(\bar{x},\bar{y}) = (\bar{u},\bar{v})$. We define a formula $\mu_i$ that is satisfied only by the lexicographically smallest $\bar{v}$ that satisfies $\varphi_i^{\fo}$ for some $\bar{u}$:
$$
\mu_i(\bar{v}) = \exists\bar{u}\,\varphi_i^{\fo}(\bar{u},\bar{v})\wedge\forall\bar{u}'\forall\bar{v}'(\varphi_i^{\fo}(\bar{u}',\bar{v}')\to\bar{v}\leq\bar{v}').
$$
The following is a pivotal formula for this proof. Consider some ordered finite structure $\A$ over $\R$. Given our conditions for $\varphi_i$, if $\A\models\varphi_i$ and $\A\models\varphi^{\fo}_i(\bar{u},\bar{v})$ then for that $\bar{v}$ we can define a $\bar{X}$ that has exactly the tuples mentioned in $\bar{v}$ and nothing else. This is {\em the smallest $\bar{X}$ that satisfies $\varphi_i$ over $\bar{v}$}. And if $\bar{v}$ is the lexicographically smallest such that $\A\models\varphi_i(\bar{v})$, then it is the {\em the smallest $\bar{X}$ that satisfies $\varphi_i$}. The following formula is satisfied by every pair $(\bar{X},\bar{x})$, except for the one formed by the smallest $\bar{X}$ that satisfies $\varphi_i$ and the lexicographically smallest $\bar{x}$ that satisfies $\varphi_i(\bar{X})$. And if no pair satisfies $\varphi_i$, the following formula is satisfied by every pair.
$$
\psi_i(\bar{X},\bar{x}) = \exists\bar{v}(\mu_i(\bar{v})\wedge(\neg\varphi^{+}_i(\bar{X},\bar{v})\vee\bigvee_{X \in \bar{X}} \exists\bar{z}(X(\bar{z}) \wedge \bigwedge\limits_{\substack{\text{instances of }X(\bar{w}) \\ \text{in }\varphi^{+}_i(\bar{X},\bar{v})}}\bar{w}\neq\bar{z}))) \vee \exists\bar{x}'(\exists\bar{y}'\varphi_i(\bar{X},\bar{x}',\bar{y}')\wedge \bar{x}'<\bar{x})\vee\neg\exists\bar{v}\mu_i(\bar{v}) .
$$
\begin{claim}
	For a given ordered structure $\A$ such that $\A\models\exists\bar{X}\exists\bar{x}\exists\bar{y}\,\varphi_i(\bar{X},\bar{x},\bar{y})$, there is an assignment $(\bar{B},\bar{b})$ to $(\bar{X},\bar{x})$ that satisfies the following conditions (1) $\A\models\exists\bar{y}\,\varphi_i(\bar{B},\bar{b},\bar{y})$, (2) $\A\not\models\psi_i(\bar{B},\bar{b})$ and (3) this is the only assignment to $(\bar{X},\bar{x})$ that satisfies (1) and (2).
\end{claim}
\begin{proof}
	proof here.
\end{proof}
The following claim identifies a crucial condition that is verifiable with a $\fo$ formula. 
\begin{claim}
	Let $\varphi_i(\bar{X},\bar{x},\bar{y}) = \varphi^{\fo}_i(\bar{x},\bar{y}) \wedge \varphi^{-}_i(\bar{X},\bar{x},\bar{y}) \wedge \varphi^{+}_i(\bar{X},\bar{x},\bar{y})$ be a $\logex{0}$ formula that satisfies all the mentioned assumptions. For a given ordered structure $\A$ it holds that $\A\models\exists\bar{x}\exists\bar{y}\,\varphi^{\fo}_i(\bar{x},\bar{y})$ iff $\A\models\exists\bar{X}\exists\bar{x}\exists\bar{y}\,\varphi_i(\bar{X},\bar{x},\bar{y})$.
\end{claim}
\begin{proof}
	Let $\A$ be an ordered structure with domain $A$ and let $\bar{a}\in A^{\length(\bar{y})}$ and $\bar{b}\in A^{\length(\bar{x})}$ such that $\A\models\varphi^{\fo}_i(\bar{b},\bar{a})$. Define $\bar{B} = (B_1,\ldots,B_{\length(\bar{X})})$ as $B_j = \{\bar{c}\mid\bar{c}\text{ is a subtuple of $(\bar{b},\bar{a})$ and $X_j(\bar{c})$ is mentioned in $\varphi^{+}_i(\bar{X},\bar{b},\bar{a})$}\}$. Towards a contradiction, suppose that $\A\not\models\varphi_i(\bar{B},\bar{b},\bar{a})$. By the choice of $\bar{a}$ and $\bar{b}$, and construction of $\bar{B}$ it is clear that $\A\models\varphi^{\fo}_i(\bar{b},\bar{a})\wedge\varphi^{+}_i(\bar{B},\bar{b},\bar{a})$, so we have that $\A\not\models\varphi^{-}_i(\bar{B},\bar{b},\bar{a})$. Let $B_t(\bar{c})$ be such that $X_t(\bar{c})$ is mentioned in $\varphi^{-}_i(\bar{X},\bar{b},\bar{a})$ and $\A\not\models\neg B_t(\bar{c})$. This is, $\bar{c}\in B_t$. By the construction of $\bar{B}$ we have that $X_t(\bar{c})$ is mentioned in $\varphi^{+}_i(\bar{X},\bar{b},\bar{a})$. From our assumptions, we have that (1) every first-order variable is mentioned at most once in $\varphi^{-}_i(\bar{X},\bar{x},\bar{y}) \wedge \varphi^{+}_i(\bar{X},\bar{x},\bar{y})$, (2) $\varphi^{\fo}_i$ defines an ordered partition over $(\bar{x},\bar{y})$, and (3) there are no $X_j(\bar{w})$ in $\varphi^{+}_i$ and $X_j(\bar{z})$ in $\varphi^{-}_i$ such that the ordered partition defined by $\varphi^{\fo}_i$ satisfies $\bar{x} = \bar{z}$. Since $\bar{c}$ is a subtuple of $(\bar{b},\bar{a})$ that satisfies $\varphi^{\fo}_i$, and $X_t(\bar{w})$ mentioned in $\varphi^{+}_i$ and $X_t(\bar{z})$ mentioned in $\varphi^{-}_i$ were both assigned the value $\bar{c}$, then the ordered partition satisfies $\bar{w} = \bar{z}$, which follows to a contradiction.
\end{proof}
We define $\chi_i = \exists\bar{x}\exists\bar{y}\,\varphi^{\fo}_i(\bar{x},\bar{y})$. Now recall that $\varphi = \bigvee_{i = 1}^m\varphi_i(\bar{X},\bar{x},\bar{y})$, where each $\varphi_i$ satisfies all the previous assumptions. For each $\varphi_i$ we define:
$$
\varphi_i'(\bar{X},\bar{x},\bar{y}) = \varphi_i(\bar{X},\bar{x},\bar{y})\wedge\psi_1(\bar{X},\bar{x})\wedge(\chi_1\vee\psi_2(\bar{X},\bar{x}))\wedge(\chi_1\vee\chi_2\vee\psi_2(\bar{X},\bar{x}))\wedge\cdots\wedge(
\bigvee_{j = 1}^{j = i-1}\chi_j\vee\psi_i(\bar{X},\bar{x})),
$$
and lastly, $\kappa(\alpha)$ is defined as $\kappa(\alpha) = \sa{\bar{X}}\sa{\bar{x}}\exists\bar{y}\bigvee_{i = 1}^m\varphi_i'(\bar{X},\bar{x},\bar{y})$.

\vspace{1em}

For each $\alpha$ in $\eqso(\logex{1})$ such that $\alpha = (\beta + \sa{\bar{X}}\sa{\bar{x}}\exists\bar{y}\,\varphi(\bar{X},\bar{x},\bar{y}))$ for some algebraic formula $\beta$ and some quantifier-free formula $\varphi$, we define $\kappa(\alpha)$ as follows: First, perform the same transformations to $\varphi(\bar{X},\bar{x},\bar{y})$ as in the previous case. Let $\varphi = \bigvee_{i = 1}^m\varphi_i(\bar{X},\bar{x},\bar{y})$ where each $\varphi_i$ satisfies the previous assumptions. We also use the previously defined formulas $\chi_i$ and $\psi_i$. 

We recursively define a function $\lambda$ as follows. If $\alpha = \sa{\bar{x}}\exists\bar{y}\varphi(\bar{x},\bar{y})$, then $\lambda(\alpha) = \exists\bar{x}'\exists\bar{y}'\varphi(\bar{x}',\bar{y}')$. If $\alpha = \sa{\bar{X}}\sa{\bar{x}}\exists\bar{y}\,\varphi(\bar{X},\bar{x},\bar{y})$, then let $\varphi = \bigvee_{i = 1}^{m}\varphi_i$ and $\lambda(\alpha) = \chi_1\vee\cdots\vee\chi_m$ as each $\chi_i$ was previously defined. If $\alpha = (\beta + \gamma)$, then $\lambda(\alpha) = \lambda(\beta) \vee \lambda(\gamma)$. Note that for a given ordered structure $\A$, then $\A\models\lambda(\alpha)$ if and only if $\sem{\alpha}(\A) > 0$.

Now, for each $\varphi_i$ we define:
$$
\varphi_i'(\bar{X},\bar{x},\bar{y}) = \varphi_i(\bar{X},\bar{x},\bar{y})\wedge[\lambda(\beta)\vee[\psi_1(\bar{X},\bar{x})\wedge(\chi_1\vee\psi_2(\bar{X},\bar{x}))\wedge(\chi_1\vee\chi_2\vee\psi_2(\bar{X},\bar{x}))\wedge\cdots\wedge(
\bigvee_{j = 1}^{j = i-1}\chi_j\vee\psi_i(\bar{X},\bar{x}))]].
$$
And lastly $\kappa(\alpha)$ is defined as $\kappa(\alpha) = \kappa(\beta) + \sa{\bar{X}}\sa{\bar{x}}\exists\bar{y}\bigvee_{i = 1}^m\varphi_i'(\bar{X},\bar{x},\bar{y})$, which is in $\eqso(\logex{1})$ and satisfies the desired condition.

\subsection*{Proof of Proposition \ref{prop:uhorn-pe}} %V.9

\subsection*{Proof of Proposition \ref{prop:ehorn-pe}} %V.11

\subsection*{Proof of Proposition \ref{prop:hsat-not-sigma2}} %V.12
$\eqso(\uhorn) \subseteq \totp$:\vspace{1em}

\subsection*{Proof of Theorem \ref{sigma2hard}} %V.13

First we prove that $\shdhsat$ is in $\eqso(\ehorn)$. Recall that each instance of $\shdhsat$ is a disjunction of Horn formulas. Let $\R = \{\pP(\cdot,\cdot), \pN(\cdot,\cdot), \pV(\cdot), \pNC(\cdot), \pD(\cdot,\cdot)\}$. Each symbol in this vocabulary is used to indicate the same as in Example \ref{ex-hornsat-esop1}, with the addition of $\pD(d,c)$ which indicates that $c$ is a clause in the formula $d$. Recall that the formula
\begin{align*}
&\forall x \, (\neg \pT(x) \vee \pV(x)) \ \wedge\\
&\forall c \, (\neg \textit{NC}(c) \vee \exists x \, \neg \textit{A}(c,x)) \ \wedge\\
&\forall c \forall x \, (\neg \textit{P}(c,x) \vee \exists y \, \neg \textit{A}(c,y) \vee \textit{T}(x)) \ \wedge\\
&\forall c \forall x \, (\neg \textit{N}(c,x) \vee \textit{T}(x) \vee \neg \textit{A}(c,x)) \ \wedge\\
&\forall c \forall x \, (\textit{A}(c,x) \vee \textit{N}(c,x)) \ \wedge\\
&\forall c \forall x \, (\textit{A}(c,x) \vee \neg\textit{T}(x)).
\end{align*}
defines $\chsat$. We obtain the following formula $\psi(T,A)$ in $\ehorn$:
\begin{align*}
\exists d[&\forall x \, (\neg \pT(x) \vee \pV(x)) \ \wedge\\
&\forall c \, (\neg \pD(c,d)\vee \neg \textit{NC}(c) \vee \exists x \, \neg \textit{A}(c,x)) \ \wedge\\
&\forall c \forall x \, (\neg \pD(c,d)\vee\neg \textit{P}(c,x) \vee \exists y \, \neg \textit{A}(c,y) \vee \textit{T}(x)) \ \wedge\\
&\forall c \forall x \, (\neg \pD(c,d)\vee\neg \textit{N}(c,x) \vee \textit{T}(x) \vee \neg \textit{A}(c,x)) \ \wedge\\
&\forall c \forall x \, (\neg \pD(c,d)\vee\textit{A}(c,x) \vee \textit{N}(c,x)) \ \wedge\\
&\forall c \forall x \, (\neg \pD(c,d)\vee\textit{A}(c,x) \vee \neg\textit{T}(x))]
\end{align*}
effectively defines $\chsat$ as for every disjunction of Horn formulas $\theta = \theta_1\vee\cdots\vee\theta_m$ encoded by an $\R$-structure $\A$, the number of satisfying assignments of $\theta$ is equal to $\sem{\sa{\pT} \sa{\pA} \psi(\pT,\pA)}(\A)$.  Therefore, we conclude that $\shdhsat \in \eqso(\ehorn)$.

\vspace{1em}
We will now prove that $\shdhsat$ is hard for $\eqso$ over a signature $\R$ under parsimonious reductions. For each $\eqso(\ehorn)$ formula $\alpha$ over $\R$, we will define a polynomial-time procedure that computes a function $g_{\alpha}$. This function receives a finite $\R$-structure $\A$ and outputs an instance of $\shdhsat$ such that $\sem{\alpha}(\A) = \shdhsat(g_{\alpha}(\A))$. We suppose that $\alpha$ is in sum normal form and:
$$
\alpha = c + \sum_{i = 1}^{\text{\#clauses}} \sa{\bar{X}}\sa{\bar{x}}\exists\bar{y}\bigwedge_{j = 1}^{n}\forall\bar{z}\,\varphi^i_j(\bar{X},\bar{x},\bar{y},\bar{z}),
$$
where each $\varphi^i_j$ is a Horn clause.                                                                

Consider a finite $\R$-structure $\A$ with domain $A$. To simplify the proof, we extend our grammar to allow first-order constants. Consider each tuple $\bar{a}\in A^{\length(\bar{x})}$, each $\bar{b}\in A^{\length(\bar{y})}$ and each $\bar{c}\in A^{\length(\bar{z})}$ as a tuple of first-order constants. The following formula defines the same function as $\alpha$:
$$
c + \sum_{i = 1}^{\#clauses} \sum_{\bar{a}\in A^{\length(\bar{x})}} \sa{\bar{X}}\bigvee_{\bar{b}\in A^{\length(\bar{y})}}\bigwedge_{j = 1}^{n}\bigwedge_{\bar{c}\in A^{\length(\bar{z})}}\varphi^i_j(\bar{X},\bar{a},\bar{b},\bar{c}).
$$
Note that each $\fo$ formula over $(\bar{x},\bar{y},\bar{z})$ in each $\varphi^i_j$ has no free variables. Therefore, we can evaluate each of these in polynomial time and replace them by $\perp$ and $\top$ where it corresponds. Each $\varphi^i_j$ will be of the form $\perp \vee\, \chi^i_j(\bar{X})$ or $\top \vee \chi^i_j(\bar{X})$ where $\chi^i_j$ is a disjunction of $\neg X_{\ell}$'s and at most one $X_{\ell}$. The formulas of the form $\top \vee \chi^i_j(\bar{X})$ can be removed entirely, and the formulas of the form $\perp \vee\, \chi^i_j(\bar{X})$ can be replaced by $\chi^i_j(\bar{X})$. We obtain the formula
$$
c + \sum_{i = 1}^{m}\sa{\bar{X}}\bigvee_{j = 1}^{\#d}\bigwedge_{k = 1}^{\#c}\psi^{i}_{j,k}(\bar{X})
$$
where every $\psi^{i}_{j,k}(\bar{X})$ is a disjunction of $\neg X_{\ell}$'s and zero or one $X_{\ell}$.

Our idea for the rest of the proof is to define $g_{\alpha}$ for each $\alpha = \sa{\bar{X}}\bigvee_{j = 1}^{\#d}\bigwedge_{k = 1}^{\#c}\psi^{i}_{j,k}(\bar{X})$, for $\alpha = c$ and for $\alpha = \beta_1 + \cdots + \beta_m$ where each $\beta_i$ is in one of the two previous cases.

If $\alpha$ is equal to $\sa{\bar{X}}\bigvee_{j = 1}^{\#d}\bigwedge_{k = 1}^{\#c}\psi_{j,k}(\bar{X})$ where $\psi_{j,k}(\bar{X})$ is a disjunction of $\neg X_{\ell}$'s and zero or one $X_{\ell}$, then we obtain the {\bf propositional formula} $g_{\alpha}(\A) = \bigvee_{j = 1}^{\#d}\bigwedge_{k = 1}^{\#c}\psi_{j,k}(\bar{X})$ over the propositional alphabet $\{X(\bar{e}) \mid X \in \bar{X} \text{ and } \bar{e}\in A^{\arity(X)} \}$ which has exactly $\sem{\alpha}(\A)$ satisfying assignments and is precisely a disjunction of Horn formulas.

If $\alpha$ is equal to a constant $c$, then we define $g_{\alpha}(\A)$ as the following formula that has exactly $c$ satisfying assignments:
$$
g_{\alpha}(\A) = \bigvee_{i = 1}^{c}\neg t_1 \wedge \cdots \wedge \neg t_{i-1} \wedge t_i \wedge \neg t_{i+1} \wedge \cdots \wedge \neg p_c.
$$ 
If $\alpha = \beta_1 + \cdots + \beta_m$, let $g_{\beta_i}(\A) = \bigvee_{j = 1}^{\#d}\bigwedge_{k = 1}^{\#c}\theta^i_{j,k}$ for each $\beta_i$ where each $\theta^i_{j,k}$ is a Horn clause. Let $\Theta_i = g_{\beta_i}(\A)$. We rename the variables in each $\Theta_i$ so none of them are mentioned in any other $\Theta_j$. We add $m$ new variables $t_1,\ldots,t_m$ and we define:
\begin{align*}
g_{\alpha}(\A) = &\bigvee_{i = 1}^{\#d}(\bigwedge_{j = 1}^{\#c}\theta^1_{i,j} \wedge (\bigwedge\limits_{\substack{\text{each } t\\ \text{ mentioned in}\\ \Theta_2,\ldots,\Theta_{m}}}t) \wedge (t_1 \wedge \bigwedge_{\ell = 2}^{m} \neg t_{\ell})) \vee \\ 
&\bigvee_{i = 1}^{\#d}(\bigwedge_{j = 1}^{\#c}\theta^2_{i,j} \wedge (\bigwedge\limits_{\substack{\text{each $t$}\\ \text{ mentioned in}\\ \Theta_1,\Theta_3,\ldots,\Theta_{m}}}t) \wedge (t_2 \wedge \bigwedge\limits_{\substack{\ell = 1 \\ \ell \neq 2}}^{m} \neg t_{\ell})) \vee \cdots \vee\\ 
&\bigvee_{i = 1}^{\#d}(\bigwedge_{j = 1}^{\#c}\theta^m_{i,j} \wedge (\bigwedge\limits_{\substack{\text{each } t\\ \text{ mentioned in}\\ \Theta_2,\ldots,\Theta_{m-1}}}t) \wedge (t_m \wedge \bigwedge_{\ell = 1}^{m-1} \neg t_{\ell})).
\end{align*}
The formula is a disjunction of Horn formulas, and the number of satisfying assignments for this formula is exactly the sum of satisfying assignments for each $g_{\beta_i}(\A)$. This, at the same time, is equal to $\sem{\alpha}(\A)$. This covers all possible cases for $\alpha$, and the entire procedure takes polynomial time.

\subsection{Proofs from Section~\ref{sec:beyond}}
\subsection*{Proof of Theorem \ref{lem-support}}



\subsection*{Proof of Theorem \ref{rqfo-fo-cap}}

Let $\R$ be a relational signature. First we prove the first condition in Definition \ref{def:cap}. Let $\alpha$ be a formula in $\tqfo(\fo)$.
We will recursively construct a deterministic polynomial-space algorithm $M_{\alpha}$ that on input $(\enc(\A),\enc(v),\enc(F))$, where $v$ is a first-order assignment and $F$ is a function assignment, outputs the value $\sem{\alpha}(\A,v)$.
From this point on we use $\A$, $v$ and $F$ to denote $\enc(\A)$, $\enc(v)$ and $\enc(F)$ respectively.
Suppose the domain of $\A$ is $A = \{1,\ldots,n\}$.
If $\alpha = \varphi$, a formula in $\fo$, we check if $(\A,v,F)\models\varphi$ in deterministic polynomial-space space, and output 1.
If $\alpha = s$, we write $s$ as output.
If $\alpha = (\alpha_1 + \alpha_2)$, we compute $M_{\alpha_1}$ and $M_{\alpha_2}$ on input $(\A,v,F)$ and give the sum of the values as output.
If $\alpha = (\alpha_1\cdot\alpha_2)$, we compute $M_{\alpha_1}$ and $M_{\alpha_2}$ on input $(\A,v,F)$ and give the product of the values as output.
If $\alpha = \sa{x}\beta$, we iterate for each $a\in A$, compute $M_{\beta}$ on input $(\A,v[x/a],F)$, and give the sum of all values as output.
If $\alpha = \pa{x}\beta$, we iterate for each $a\in A$, compute $M_{\beta}$ on input $(\A,v[x/a],F)$, and give the product of all values as output.
If $\alpha = \clfp{\beta(\bar{x},h)}$, we compute each of the functions $f_0,\ldots,f_k$ iteratively. We start from the null function $f_0$ and compute $f_i$ on each step. If at any point $\support(f_i) = \support(f_{i+1})$, then we stop iterating and output $f_i(v(\bar{x}))$.
This ends the construction of the algorithm.
Consider $f$ as the $\fp$ function associated to this procedure and we have that for each finite $\R$-structure $\A$: $f(\enc(\A)) = \sem{\alpha}(\A)$.

\vspace{1em}
Now we prove the second condition in Definition \ref{def:cap}. Let $f\in\fp$ be defined over $\R$. Recall that in the proof for Theorem \ref{theo:capture-fp} we constructed a formula $\alpha = \sa{\bar{x}}\Phi(\bar{x})\mult\varphi(\bar{x})$ in $\qfo(\lfp)$ where $\Phi(\bar{x})$ is a formula in $\lfp$ extended with constants over the signature $\R \cup \{\bar{x}\}$. We use the construction used by Gradel in \cite{G07} to capture $\ptime$ with $\lfp$. Consider the function defined by the operator $\clfp{\varphi(\bar{x},h)}(\bar{x})$. This corresponds exactly to the function:
$$
f(\A,v(\bar{x})) 
\begin{cases}
1 &(\A,v)\models [{\bf lfp}\,{\varphi(\bar{x},R)}](\bar{x}), \\
0 &\text{ in other case.}
\end{cases}
$$
Then, each formula $[{\bf lfp}\,{\varphi(\bar{x},R)}]$ in $\Phi$ can be replaced by $\clfp{\varphi(\bar{x},h)}$ to obtain an equivalent formula $\alpha'$. We have that $\alpha'$ is in $\rqfo(\fo)$ and for each $\A\in\ostr[\R]$ it holds that $\sem{\alpha'}(\A) = f(\enc(\A))$.


\subsection*{Proof of Theorem \ref{tqfo-shl}}

Let $\R$ be some relational signature.  First we address the first condition in the Definition \ref{def:cap}. Let $\alpha$ be a formula in $\tqfo(\fo)$. We will construct a nondeterministic logspace algorithm $M_{\alpha}$ that on input $\enc(\A)$, where a first-order assignment $v$ is being stored in memory, accepts in $\sem{\alpha}(\A)$ paths. Suppose the domain of $\A$ is $A = \{1,\ldots,n\}$. The algorithm needs $c\cdot\log_2(n)$ bits of memory to store $v$, where $c$ is the total number of first-order variables in $\alpha$. If $\alpha = \varphi$, we check if $(\A,v)\models\varphi$ in deterministic logarithmic space, and accept if and only if it does. If $\alpha = s$, we generate $s$ branches and accept in all of them. If $\alpha = (\alpha_1 + \alpha_2)$, we simulate $M_{\alpha_1}$ and $M_{\alpha_2}$ on separate branches. If $\alpha = (\alpha_1\cdot\alpha_2)$, we simulate $\alpha_1$ and if it accepts, instead of doing so, we simulate $\alpha_2$. If $\alpha = \sa{x}\beta$, for each $a\in A$ we generate a different branch where we simulate $M_{\beta}$ while storing $v[a/x]$. If $\alpha = \pa{x}\beta$, we simulate $M_{\beta}$ while storing $v[1/n]$, and on each accepting branch, instead of accepting we replace the assignment on $x$ to 2, to simulate $M_{\beta}$ while storing $v[2/x]$, and so on. If $\alpha = [\pth \varphi(\bar{x},\bar{y})]$ where $\varphi$ is an $\fo$ formula, we simulate the $\shl$ procedure that counts the number of paths for a graph of a given size. This procedure starts by setting $\bar{a} = v(\bar{x})$. On each iteration, nondeterministically chooses an assignment $\bar{a}$ for $\bar{x}$, continues if $(\A,v)\models\varphi(\bar{a}',\bar{a})$ where $\bar{a}'$ is the previously chosen value for $\bar{a}$, and rejects otherwise. If at any point we obtain that $\bar{a} = v(\bar{y})$, we generate an accepting branch, and continue simulating the procedure in the current branch. We simulate $n^{\length(\bar{x})}$ iterations of the procedure, and this generates exactly $\sem{[\pth \varphi(\bar{x},\bar{y})]}(\A,v)$ accepting branches. This ends the construction of the algorithm. Consider $f$ as the $\shl$ function associated to this procedure and we have that for each finite $\R$-structure $\A$: $f(\enc(\A)) = \sem{\alpha}(\A)$.

\vspace{1em}
For the second condition, let $f \in \shl$ defined over $\R$. We will address the case where $\R$ contains only one binary predicate $E$, and the rest of the cases can be deduced from this. Let $M$ be a non-deterministic logspace machine such that $f(\enc(\A)) = \acc_M(\enc(\A))$ for each $\A \in \ostr[\R]$. Suppose ${\cal Q} = \{q_1,\ldots,q_{\ell}\}$ is the set of states of $M$, where $q_1$ is the initial state, $q_{\ell}$ is only final state of $M$ and suppose $\Delta \subseteq Q \times \{0,1\} \times \{0,1\} \times Q \times \{-1,=,+1\} \times \{0,1\} \times \{-1,=,+1\}$ is the set of transitions in $M$. We also suppose that the machine stops when it reaches a final state. Formally, for each $a,b,c\in\{0,1\}$, each $op_1,op_2\in\{-1,=,+1\}$ and state $q\in Q$, we have $(q_{\ell},a,b,op_1,c,op_2,q)\not\in\Delta$. Let $n = \vert A \vert$ and let $w = \enc(\A) \in \{0,1\}^{n^2}$. We assume that $M$ with input $w$ uses space $s_M(w) < c\cdot\log(n)$ and furthermore, $s_M(w) < n-2$. We notate $M(w)$ as the graph of configurations of $M$ running on input $w$.

We represent configurations with a tuple of fixed size. The formula $\varphi(\bar{x},\bar{y})$ describes a procedure that given a configuration generates a possible next configuration. The formula $\varphi_I(\bar{x})$ describes that $\bar{x}$ is the initial configuration of $M(w)$. The formula $\varphi_F(\bar{x})$ describes that $\bar{x}$ is an accepting (final) configuration of $M(w)$. The formula we construct is:
$$
\alpha = \sa{\bar{x}}\sa{\bar{y}}([\pth \varphi(\bar{x},\bar{y})]\cdot \varphi_I(\bar{x})\cdot\varphi_F(\bar{y})).
$$

To illustrate our idea, we will show a simplified example. Consider a machine $M$ that works in exactly $\log_2(n)$ space and only allows 0 or 1 in the working tape. Consider an input $\A$ of size 16 (that is, $A = \{0,\ldots,9,A,\ldots,F\}$). Let some configuration $s$ have 0011 in the working tape, the head in the input tape is in position 26, and the head in the input tape is in position 2 (we consider 0-indexed positions). Also, $Q = \{q_1,\ldots,q_5\}$ and the current state is $q_3$.

As a first approach, we will use a 9-tuple $\bar{a} = (a_1,\ldots,a_9)$ to represent $s$. That is, $(a_1,a_2) = (1,A)$ represent the position of the head in the input tape (since 1A equal to 26 in base 16), $a_3 = 2$ represents the position of the head in the working tape, $a_4 = C$ (1100b in base 16) represents the content of the working tape, and $(a_5,\ldots,a_9) = (0,0,1,0,0)$ represents the current state. Then $\bar{a} = (1,A,2,C,0,0,1,0,0)$ will represent $s$.

\newcommand\algx{\mathtt{x}}
\newcommand\algy{\mathtt{y}}
\newcommand\algz{\mathtt{z}}
\newcommand\algu{\mathtt{u}}
\newcommand\algv{\mathtt{v}}
\newcommand\algi{\mathtt{i}}
\newcommand\algj{\mathtt{j}}


The problem that arises from this representation, is that to describe a transition in $M$ we need to read an arbitrary character in the working tape. (In the example, this translates to obtaining the $a_3$-th bit in $a_4$. Furthermore, to represent the next configuration, we need compute $a_4$ with the $a_3$-th bit flipped.) This is generally not possible to describe with an $\fo$ formula. To deal with this issue, consider the procedure defined in Algorithm \ref{switch1to0}. (In the example the procedure would receive $\algx = a_4$ and $\algi = a_3$.)

\begin{algorithm}
	\caption{If the $\algi$-th bit in $\algx$ is 1 replace it by 0 and return the result}
	\label{switch1to0}
	\begin{algorithmic}
		\State $\algu \gets \algx,\; \algj \gets \algi$ \Comment{Get the $\algi$-th bit on $\algx$ and store it in $\algu$}
		\While{$\algj > 0$}
		\State $\algv \gets 0$
		\While{$\algu > 1$}
		\State $\algu \gets \algu-2,\; \algv \gets \algv+1$
		\EndWhile
		\State $\algu\gets \algv,\; \algj \gets \algj-1$
		\EndWhile
		\While{$\algu > 1$}
		\State $\algu \gets \algu-2$
		\EndWhile
		\State $\textbf{assert } \algu = 1$ \Comment{If $\algu \neq 1$ simply stop}	
		\State $\algy \gets 1$ \Comment{Compute $2^{\algi}$ and store it in $\algy$}
		\While{$\algi > 0$}
		\State $\algz \gets 0$
		\While{$\algy > 0$}
		\State $\algz \gets \algz+2,\; \algy \gets \algy-1$
		\EndWhile
		\State $\algi \gets \algi-1,\; \algy \gets \algz$
		\EndWhile
		\While{$\algy > 0$} \Comment{Subtract $\algy$ from $\algx$}
		\State $\algx \gets \algx-1,\; y \gets \algy-1$
		\EndWhile
		\State \Return $\algx$.
	\end{algorithmic}
\end{algorithm}	
Each of the instructions can be expressed with $\fo$, so our strategy is to use the $\pth$ operator to simulate the algorithm and then we can describe a transition using the processed value of $a_4$. This procedure simulates a transition that writes 1 in the cell where it read a 0. We call this a $1 \to 0$ transition. At the end of the proof we provide in detail three more procedures that simulate a $0\to 0$ transition, a $0\to 1$ transition, and a $1\to 1$ transition. The rest of the proof only addresses the case where we are simulating a $1\to 0$ transition, and the rest of the cases can be described analogously.

We will now describe how to simulate both the procedure and the transition. A procedure tuple $\bar{p} = (a_1,\ldots,a_{3+c+\ell},b_1,b_2,c_1,c_2,c_3,d_1,\ldots,d_{5c+2})$ represents the current configuration of $M(w)$ in $a_1,\ldots,a_{2+c+\ell}$, the values that will be read and written in the working tape in $b_1,b_2$, the instruction pointer in $c_1,c_2,c_3$ and the values stored in memory in $d_1,\ldots,d_{10c+2}$. In detail:
\begin{enumerate}
	\item $a_1,a_2$ and $a_3$ represent the position of the head in the input tape and the working tape, respectively, $a_4,\ldots,a_{3+c}$ represent the content of the working tape and $a_{4+c},\ldots,a_{3+c+\ell}$ represent the current state in the current configuration that is being processed.
	\item $b_1$ and $b_2$ are equal to the value that is being read in the working tape and the value that will be written in the working tape respectively. These values also indicate which algorithm is being simulated.
	\item $c_1,c_2,c_2$ represent the instruction pointer in the procedure. Only 8 different instructions are needed in the simulation.
	\item The variables $\algx,\algy,\algz,\algu,\algv$ need $c$ elements each to be represented and $\algi,\algj$ need only one. We map $(d_1\ldots,d_{c}) \to \algx$, $(d_{c+1}\ldots,d_{2c}) \to \algy$,
	$(d_{2c+1}\ldots,d_{3c}) \to \algz$, $(d_{3c+1}\ldots,d_{4c}) \to \algu$,
	$(d_{4c+1}\ldots,d_{5c}) \to \algv$, $d_{5c+1} \to \algi$ and $d_{5c+2}\to \algj$.
\end{enumerate}
For each transition $\delta \in \Delta \subseteq Q \times \{0,1\} \times \{0,1\} \times Q \times \{-1,=,+1\} \times \{0,1\} \times \{-1,=,+1\}$ we define a formula $\varphi_{\delta}(\bar{x},\bar{s},\bar{w},\bar{u},\bar{y},\bar{t},\bar{z},\bar{v})$, where $\bar{x} = (x_1,\ldots,x_{3+c+\ell})$, $\bar{s} = (s_1,s_2)$, $\bar{w} = (w_1,w_2,w_3)$, $\bar{u} = (u_1,\ldots,u_{5c+2})$, $\bar{y} = (y_1,\ldots,y_{3+c+\ell})$, $\bar{t} = (t_1,t_2)$, $\bar{z} = (z_1,z_2,z_3)$ and $\bar{v} = (v_1,\ldots,v_{5c+2})$. The tuples $\bar{x}$ and $\bar{y}$ represent the current and next configuration of $M$ respectively, $\bar{s}$ and $\bar{t}$ indicate which algorithm is being simulated, $\bar{w}$ and $\bar{z}$ represent the current and next instruction of the algorithm, $\bar{u}$ and $\bar{v}$ represent the current and next values in memory. We will describe the formula part by part. Suppose $\delta = (q_i,a,1,q_j,op_1,0,op_2)$, so we have to simulate Algorithm \ref{switch1to0}.

To significantly improve the readability of the construction, we define the following tuples:
\begin{equation*}
\begin{aligned}
	\bar{x}_{\text{h-in}} &= (x_1,x_2), \\
	x_{\text{h-w}} &= x_3, \\
	\bar{x}_{\text{tape}} &= (x_4,\ldots,x_{3+c}),\\
	\bar{x}_{\text{state}} &= (x_{4+c},\ldots,x_{3+c+\ell}),\\
	\bar{u}_{\algx} &= (u_1,\ldots,u_c),\\
	\bar{u}_{\algy} &= (u_{c+1},\ldots,u_{2c}),\\
	\bar{u}_{\algz} &= (u_{2c+1},\ldots,u_{3c}),\\
	\bar{u}_{\algu} &= (u_{3c+1},\ldots,u_{4c}),\\
	\bar{u}_{\algv} &= (u_{4c+1},\ldots,u_{5c}),\\
	u_{\algi} &= u_{5c+1},\\
	u_{\algj} &= u_{5c+2},
\end{aligned}
\hspace{1em}
\begin{aligned}
	\bar{y}_{\text{h-in}} &= (y_1,y_2), \\
	y_{\text{h-w}} &= y_3, \\
	\bar{y}_{\text{tape}} &= (y_4,\ldots,y_{3+c}),\\
	\bar{x}_{\text{state}} &= (x_{4+c},\ldots,x_{3+c+\ell}),\\
	\bar{v}_{\algx} &= (v_1,\ldots,v_c),\\
	\bar{v}_{\algy} &= (v_{c+1},\ldots,v_{2c}),\\
	\bar{v}_{\algz} &= (v_{2c+1},\ldots,v_{3c}),\\
	\bar{v}_{\algu} &= (v_{3c+1},\ldots,v_{4c}),\\
	\bar{v}_{\algv} &= (v_{4c+1},\ldots,v_{5c}),\\
	v_{\algi} &= v_{5c+1},\\
	v_{\algj} &= v_{5c+2}.
\end{aligned}
\end{equation*}
We also define some auxiliary formulas:
\begin{align*}
\gamma_{0}(\bar{x}) &= \neg\exists\bar{y}(\bar{y}<\bar{x}),\\
\gamma_{1}(\bar{x}) &= \exists\bar{y}(\gamma_{0}(\bar{y})\wedge \bar{y} < \bar{x} \wedge \neg\exists\bar{z}(\bar{y}<\bar{z}\wedge\bar{z}<\bar{x}))\\
\gamma_{+1}(\bar{x},\bar{y}) &= \bar{x} < \bar{y} \wedge \neg\exists \bar{z}(\bar{x}<\bar{z} \wedge \bar{z}<\bar{y}), \\
\gamma_{-1}(\bar{x},\bar{y}) &= \gamma_{+1}(\bar{y},\bar{x}),\\
\gamma_{=}(\bar{x},\bar{y}) &= \bar{x} = \bar{y} \\
\gamma_{+2}(\bar{x},\bar{y}) &= \exists\bar{z}(\gamma_{+1}(\bar{x},\bar{z}) \wedge \gamma_{+1}(\bar{z},\bar{y})),\\
\gamma_{-2}(\bar{x},\bar{y}) &= \gamma_{+2}(\bar{y},\bar{x}),\\
\gamma_{i,j}(x,y) &= \gamma_i(x) \wedge \gamma_j(y),\text{ for $i,j \in\{0,1\}$}\\
\varphi^b_k(x_1,x_2,x_3) &= \gamma_{a_1}(x_1) \wedge \gamma_{a_2}(x_2) \wedge \gamma_{a_3}(x_3),\text{ for each $k \leq 7$, where $a_1a_2a_3$ is the value of $k$ in binary}, \\
\varphi^q_i(x_1,\ldots,x_{\ell}) &= \gamma_0(x_1) \wedge \cdots \wedge \gamma_0(x_{i-1}) \wedge \gamma_1(x_i) \wedge \gamma_0(x_{i+1}) \wedge \cdots \wedge \gamma_0(x_{\ell}), \text{ for each } q_i\in Q\\
\varphi^E_0(x_1,x_2) &= \neg E(x_1,x_2),\\		\varphi^E_1(x_1,x_2) &= E(x_1,x_2),\\
\end{align*}

We start from instruction 0, which means that the procedure has not started yet and every value in the tuple is 0 except for the configuration values. It also initializes all the values in the tuple to 0 except for $\algx,\algu,\algi,\algj$.
\begin{align*}
\varphi^{0,1}_{\delta}(\bar{x},\bar{s},\bar{w},\bar{u},\bar{y},\bar{t},\bar{z},\bar{v}) = \,&\gamma_{0,0}(\bar{s})\wedge\varphi^b_0(\bar{w}) \wedge \\ &\gamma_{1,0}(\bar{t}) \wedge \varphi^b_1(\bar{z}) \wedge 
\bar{v}_{\algx} = \bar{x}_{\text{tape}} \wedge 
\gamma_0(\bar{v}_{\algy}) \wedge 
\gamma_0(\bar{v}_{\algz}) \wedge 
\bar{v}_{\algu} = \bar{x}_{\text{tape}} \wedge 
\gamma_0(\bar{v}_{\algv}) \wedge 
v_{\algi} = x_{\text{h-w}} \wedge v_{\algj} = x_{\text{h-w}}.
\end{align*}
Instruction 1 which checks whether the value of $\algj$ is more than 0 or not, and then proceeds to instruction 2 or 3 on each case.
\begin{align*}
\varphi^{1,2}_{\delta}(\bar{x},\bar{s},\bar{w},\bar{u},\bar{y},\bar{t},\bar{z},\bar{v}) = 
\,&\gamma_{1,0}(\bar{s}) \wedge \varphi^b_1(\bar{w}) \wedge \neg \gamma_0(u_{\algj}) \wedge \\ &\gamma_{1,0}(\bar{t}) \wedge \varphi^b_2(\bar{z}) \wedge \bar{u} = \bar{v}, \\
\varphi^{1,3}_{\delta}(\bar{x},\bar{s},\bar{w},\bar{u},\bar{y},\bar{t},\bar{z},\bar{v}) = 
\,&\gamma_{1,0}(\bar{s}) \wedge \varphi^b_1(\bar{w}) \wedge \gamma_0(u_{\algj}) \wedge \\ &\gamma_{1,0}(\bar{t}) \wedge \varphi^b_3(\bar{z}) \wedge \bar{u} = \bar{v}.
\end{align*}
Instruction 2 checks the value of $\algu$. If it is $> 1$ then it subtracts 2 from $\algu$ and adds 1 to $\algv$, then repeats instruction 2. If it is equal to 0 or 1, then moves the value of $\algv$ to $\algu$, subtracts 1 from $\algj$ and goes back to instruction 1.
\begin{align*}
\varphi^{2,2}_{\delta}(\bar{x},\bar{s},\bar{w},\bar{u},\bar{y},\bar{t},\bar{z},\bar{v}) = 
\,&\gamma_{1,0}(\bar{s}) \wedge \varphi^b_2(\bar{w}) \wedge \neg \gamma_0(\bar{u}_{\algu}) \wedge \neg \gamma_1(\bar{u}_{\algu}) \wedge \\ &\gamma_{1,0}(\bar{t}) \wedge \varphi^b_2(\bar{z}) \wedge
\bar{u}_{\algx} = \bar{v}_{\algx} \wedge
\bar{u}_{\algy} = \bar{v}_{\algy} \wedge
\bar{u}_{\algz} = \bar{v}_{\algz} \wedge
\gamma_{-2}(\bar{u}_{\algu},\bar{v}_{\algu}) \wedge
\gamma_{+1}(\bar{u}_{\algv},\bar{v}_{\algv}) \wedge u_{\algi} = v_{\algi} \wedge u_{\algj} = v_{\algj},\\
\varphi^{2,1}_{\delta}(\bar{x},\bar{s},\bar{w},\bar{u},\bar{y},\bar{t},\bar{z},\bar{v}) = \,&\gamma_{1,0}(\bar{s}) \wedge \varphi^b_2(\bar{w}) \wedge ( \gamma_0(\bar{u}_{\algu}) \vee \gamma_1(\bar{u}_{\algu})) \wedge \\ 
&\gamma_{1,0}(\bar{t}) \wedge \varphi^b_1(\bar{z}) \wedge
\bar{u}_{\algx} = \bar{v}_{\algx} \wedge
\bar{u}_{\algy} = \bar{v}_{\algy} \wedge
\bar{u}_{\algz} = \bar{v}_{\algz} \wedge
\bar{u}_{\algv} = \bar{v}_{\algu} \wedge
\gamma_0(\bar{v}_{\algv}) \wedge
u_{\algi} = v_{\algi} \wedge \gamma_{-1}(u_{\algj},v_{\algj}).	
\end{align*}
Instruction 3 calculates the value of $\algu\mod 2$, that is, it repeats instruction 3 until the value of $\algu$ is equal to 0 or 1. On each iteration, it subtracts 2 from $\algu$. Moreover, if the value of $\algu$ at the end of the iterations is not 1 then there is no step defined.
\begin{align*}
\varphi^{3,3}_{\delta}(\bar{x},\bar{s},\bar{w},\bar{u},\bar{y},\bar{t},\bar{z},\bar{v}) = \,&\gamma_{1,0}(\bar{s}) \wedge \varphi^b_3(\bar{w}) \wedge \neg \gamma_0(\bar{u}_{\algu}) \wedge \neg \gamma_1(\bar{u}_{\algu}) \wedge \\ &\gamma_{1,0}(\bar{t}) \wedge 	\varphi^b_3(\bar{z}) \wedge
\bar{u}_{\algx} = \bar{v}_{\algx} \wedge
\bar{u}_{\algy} = \bar{v}_{\algy} \wedge
\bar{u}_{\algz} = \bar{v}_{\algz} \wedge
\gamma_{-2}(\bar{u}_{\algu},\bar{v}_{\algu}) \wedge
\bar{u}_{\algv} = \bar{v}_{\algv} \wedge
u_{\algi} = v_{\algi} \wedge u_{\algj} = v_{\algj},\\
\varphi^{3,3}_{\delta}(\bar{x},\bar{s},\bar{w},\bar{u},\bar{y},\bar{t},\bar{z},\bar{v}) = 
\,&\gamma_{1,0}(\bar{s}) \wedge \varphi^b_3(\bar{w}) \wedge \gamma_1(\bar{u}_{\algu}) \wedge\\ &\gamma_{1,0}(\bar{t}) \wedge 	\varphi^b_4(\bar{z}) \wedge
\bar{u}_{\algx} = \bar{v}_{\algx} \wedge
\gamma_1(\bar{v}_{\algy}) \wedge
\bar{u}_{\algz} = \bar{v}_{\algz} \wedge
\bar{u}_{\algu} = \bar{v}_{\algu} \wedge
\bar{u}_{\algv} = \bar{v}_{\algv} \wedge
u_{\algi} = v_{\algi} \wedge u_{\algj} = v_{\algj}
\end{align*}
Instruction 4 checks the value of $\algi$. If it is not 0 then goes to instruction 5 and if is 0 then goes to instruction 6. Moreover it initializes the value of $\algz$ to 0 (which was 0 all along.)
\begin{align*}
\varphi^{4,5}_{\delta}(\bar{x},\bar{s},\bar{w},\bar{u},\bar{y},\bar{t},\bar{z},\bar{v}) = \,&\gamma_{1,0}(\bar{s}) \wedge \varphi^b_4(\bar{w}) \wedge \neg \gamma_0(u_{\algi}) \wedge \\ &\gamma_{1,0}(\bar{t}) \wedge 	\varphi^b_5(\bar{z}) \wedge \bar{u} = \bar{v}, \\
\varphi^{4,6}_{\delta}(\bar{x},\bar{s},\bar{w},\bar{u},\bar{y},\bar{t},\bar{z},\bar{v}) = &\gamma_{1,0}(\bar{s}) \wedge \varphi^b_4(\bar{w}) \wedge \gamma_0(u_{\algi}) \wedge \\ &\gamma_{1,0}(\bar{t}) \wedge 	\varphi^b_6(\bar{z}) \wedge \bar{u} = \bar{v}.
\end{align*}
Instruction 5 checks the value of $\algy$. If it is more than 0 then it adds 2 to $\algz$ and subtracts 1 from $\algy$, then repeats instruction 2. If it is not, then copies the value of $\algz$ to $\algy$ and subtracts 1 from $\algi$ and returns to instruction 4.
\begin{align*}
\varphi^{5,5}_{\delta}(\bar{x},\bar{s},\bar{w},\bar{u},\bar{y},\bar{t},\bar{z},\bar{v}) = \,&\gamma_{1,0}(\bar{s}) \wedge \varphi^b_5(\bar{w}) \wedge \neg \gamma_0(\bar{u}_{\algy}) \wedge \\ &\gamma_{1,0}(\bar{t}) \wedge \varphi^b_5(\bar{z}) \wedge
\bar{u}_{\algx} = \bar{v}_{\algx} \wedge
\gamma_{-1}(\bar{u}_{\algy},\bar{v}_{\algy}) \wedge
\gamma_{+2}(\bar{u}_{\algz},\bar{v}_{\algz})\, \wedge \bar{u}_{\algu} = \bar{v}_{\algu} \wedge
\bar{u}_{\algv} = \bar{v}_{\algv} \wedge
u_{\algi} = v_{\algi} \wedge u_{\algj} = v_{\algj},\\
\varphi^{5,4}_{\delta}(\bar{x},\bar{s},\bar{w},\bar{u},\bar{y},\bar{t},\bar{z},\bar{v}) = \,&\gamma_{1,0}(\bar{s}) \wedge \varphi^b_5(\bar{w}) \wedge \gamma_0(\bar{u}_{\algy}) \wedge \\ &\gamma_{1,0}(\bar{t}) \wedge \varphi^b_4(\bar{z}) \wedge
\bar{u}_{\algx} = \bar{v}_{\algx} \wedge
\bar{u}_{\algz} = \bar{v}_{\algy} \wedge
\gamma_0(\bar{v}_{\algz}) \wedge
\bar{u}_{\algu} = \bar{v}_{\algu} \wedge
\bar{u}_{\algv} = \bar{v}_{\algv} \wedge
\gamma_{-1}(u_{\algi},v_{\algi}) \wedge u_{\algj} = v_{\algj}
\end{align*}
Instruction 6 checks the value of $\algy$. If it is more than 0, then subtracts 1 from $\algx$ and $\algy$ and repeats instruction 6. If it is not, then goes to instruction 7.
\begin{align*}
\varphi^{6,6}_{\delta}(\bar{x},\bar{s},\bar{w},\bar{u},\bar{y},\bar{t},\bar{z},\bar{v}) = \,&\gamma_{1,0}(\bar{s}) \wedge \varphi^b_6(\bar{w}) \wedge \neg \gamma_0(\bar{u}_{\algy}) \wedge \\ &\gamma_{1,0}(\bar{t}) \wedge \varphi^b_6(\bar{z}) \wedge
\gamma_{-1}(\bar{u}_{\algx},\bar{v}_{\algx}) \wedge \gamma_{-1}(\bar{u}_{\algy},\bar{v}_{\algy}) \wedge \bar{u}_{\algu} = \bar{v}_{\algu} \wedge \bar{u}_{\algv} = \bar{v}_{\algv} \\
\varphi^{6,7}_{\delta}(\bar{x},\bar{s},\bar{w},\bar{u},\bar{y},\bar{t},\bar{z},\bar{v}) = \,&\gamma_{1,0}(\bar{s}) \wedge \varphi^b_6(\bar{w}) \wedge \gamma_0(\bar{u}_{\algy}) \wedge \\ &\gamma_{1,0}(\bar{t}) \wedge \varphi^b_7(\bar{z}) \wedge \bar{u} = \bar{v}.
\end{align*}
Instruction 7 stores the value of $\algx$ after the corresponding bit has been switched. Then we can define $\gamma_{\delta}$ which also simulates the actual transition. If $\algu$ equals 1, then copy what is stored in $\algx$ to $a_4,\ldots,a_{3+c}$, go from state $q_i$ to state $q_j$, and move the heads to their corresponding positions. Recall that $op_1,op_2\in\{+1,=,-1\}$
\begin{align*}
\gamma_{\delta}(\bar{x},\bar{s},\bar{w},\bar{u},\bar{y},\bar{t},\bar{z},\bar{v}) = 	[&\bar{x} = \bar{y} \wedge (\varphi^{0,1}(\bar{x},\bar{s},\bar{w},\bar{u},\bar{y},\bar{t},\bar{z},\bar{v}) \vee \\ &\varphi^{1,2}(\bar{x},\bar{s},\bar{w},\bar{u},\bar{y},\bar{t},\bar{z},\bar{v}) \vee \varphi^{1,3}(\bar{x},\bar{s},\bar{w},\bar{u},\bar{y},\bar{t},\bar{z},\bar{v}) \vee \\ &\varphi^{2,2}(\bar{x},\bar{s},\bar{w},\bar{u},\bar{y},\bar{t},\bar{z},\bar{v}) \vee \varphi^{2,1}(\bar{x},\bar{s},\bar{w},\bar{u},\bar{y},\bar{t},\bar{z},\bar{v}) \vee \\ &\varphi^{3,3}(\bar{x},\bar{s},\bar{w},\bar{u},\bar{y},\bar{t},\bar{z},\bar{v}) \vee \varphi^{3,4}(\bar{x},\bar{s},\bar{w},\bar{u},\bar{y},\bar{t},\bar{z},\bar{v}) \vee \\ &\varphi^{4,5}(\bar{x},\bar{s},\bar{w},\bar{u},\bar{y},\bar{t},\bar{z},\bar{v}) \vee \varphi^{4,6}(\bar{x},\bar{s},\bar{w},\bar{u},\bar{y},\bar{t},\bar{z},\bar{v}) \vee \\ &\varphi^{5,5}(\bar{x},\bar{s},\bar{w},\bar{u},\bar{y},\bar{t},\bar{z},\bar{v}) \vee  \varphi^{5,6}(\bar{x},\bar{s},\bar{w},\bar{u},\bar{y},\bar{t},\bar{z},\bar{v}) \vee \\ &\varphi^{6,6}(\bar{x},\bar{s},\bar{w},\bar{u},\bar{y},\bar{t},\bar{z},\bar{v}) \vee \varphi^{6,7}(\bar{x},\bar{s},\bar{w},\bar{u},\bar{y},\bar{t},\bar{z},\bar{v}))] \, \vee \\
[\varphi^E_a(\bar{x}_{\text{h-in}}) \wedge &\gamma_{1,0}(\bar{s}) \wedge \varphi^b_7(\bar{w}) \wedge \gamma_1(\bar{u}_{\algu}) \wedge \\ &\gamma_{0,0}(\bar{t}) \wedge \varphi^b_0(\bar{z}) \wedge \bar{u} = \bar{v} \wedge
\gamma_{op_1}(\bar{x}_{\text{h-in}},\bar{y}_{\text{h-in}}) \wedge \gamma_{op_2}(x_{\text{h-w}},y_{\text{h-w}}) \wedge \bar{y}_{\text{tape}} = \bar{u}_{\algx} \wedge \varphi^q_i(\bar{x}_{\text{state}}) \wedge \varphi^q_j(\bar{y}_{\text{state}})].
\end{align*}

Note that we also need to specify that the program we are following is Algorithm \ref{switch1to0} so we store $1,0$ in $b_1,b_2$ all along the procedure. We describe the three other algorithms that compute the switches from $0\to 0$, $0\to 1$ and $1\to 1$ (Algorithms \ref{switch0to0}, \ref{switch0to1} and \ref{switch1to1}.)
For the other three cases, where $\delta = (q_i,a,0,q_j,op_1,0,op_2)$, $\delta = (q_i,a,0,q_j,op_1,1,op_2)$ and $\delta = (q_i,a,1,q_j,op_1,1,op_2)$, $\gamma_{\delta}(\bar{x},\bar{s},\bar{w},\bar{u},\bar{y},\bar{t},\bar{z},\bar{v})$ is defined analogously. Then, $\varphi(\bar{x},\bar{s},\bar{w},\bar{u},\bar{y},\bar{t},\bar{z},\bar{v})$ is defined as:
$$
\varphi(\bar{x},\bar{s},\bar{w},\bar{u},\bar{y},\bar{t},\bar{z},\bar{v}) = \bigvee_{\delta \in \Delta} \gamma_{\delta}(\bar{x},\bar{s},\bar{w},\bar{u},\bar{y},\bar{t},\bar{z},\bar{v}).
$$
Lastly we define $\varphi_I$ and $\varphi_F$:
\begin{align*}
\varphi_I(\bar{x},\bar{s},\bar{w},\bar{u}) &= \gamma_0(\bar{x}_{\text{Head-1}}) \wedge \gamma_0(x_{\text{Head-2}}) \wedge \gamma_0(\bar{x}_{\text{Tape}}) \wedge \varphi^{q}_1(\bar{x}_{\text{State}})\wedge \gamma_{0,0}(\bar{s})\wedge \varphi^b_0(\bar{w}) \wedge\gamma_0(\bar{u}). \\
\varphi_F(\bar{x},\bar{s},\bar{w},\bar{u}) &= \varphi^q_{\ell}(\bar{x}_{\text{State}}) \wedge \gamma_{0,0}(\bar{s}) \wedge \varphi^b_0(\bar{w}) \wedge\gamma_0(\bar{u}),
\end{align*}
and then $\sem{\alpha}(\A) = \sem{\sa{\bar{x}}\sa{\bar{y}}([\pth \varphi(\bar{x},\bar{y})]\cdot \varphi_I(\bar{x})\cdot\varphi_F(\bar{y}))}(\A) = \acc_M(\A)$.

\begin{algorithm}
	\caption{If the $i$-th bit in $x$ is 0 return $x$} \label{switch0to0}
	\begin{algorithmic}
		\State $u \gets x,\; j \gets i$ \Comment{Get the $i$-th bit on $x$ and store it in $u$}
		\While{$j > 0$}
		\State $v \gets 0$
		\While{$u > 1$}
		\State $u \gets u-2,\; v \gets v+1$
		\EndWhile
		\State $u\gets v,\; j \gets j-1$
		\EndWhile
		\While{$u > 1$}
		\State $u \gets u-2$
		\EndWhile
		\State $\textbf{assert } u = 0$ \Comment{If $u \neq 0$ simply stop}	
		\State \Return $x$.
	\end{algorithmic}
\end{algorithm}

\begin{algorithm}
	\caption{If the $i$-th bit in $x$ is 0 replace it by 1 and return the result}
	\label{switch0to1}
	\begin{algorithmic}
		\State $u \gets x,\; j \gets i$ \Comment{Get the $i$-th bit on $x$ and store it in $u$}
		\While{$j > 0$}
		\State $v \gets 0$
		\While{$u > 1$}
		\State $u \gets u-2,\; v \gets v+1$
		\EndWhile
		\State $u\gets v,\; j \gets j-1$
		\EndWhile
		\While{$u > 1$}
		\State $u \gets u-2$
		\EndWhile
		\State $\textbf{assert } u = 0$ \Comment{If $u \neq 0$ simply stop}	
		\State $y \gets 1$ \Comment{Compute $2^i$ and store it in $y$}
		\While{$i > 0$}
		\State $z \gets 0$
		\While{$y > 0$}
		\State $z \gets z+2,\; y \gets y-1$
		\EndWhile
		\State $i \gets i-1,\; y \gets z$
		\EndWhile
		\While{$y > 0$} \Comment{Add $y$ to $x$}
		\State $x \gets x+1,\; y \gets y-1$
		\EndWhile
		\State \Return $x$.
	\end{algorithmic}
\end{algorithm}	

\begin{algorithm}
	\caption{If the $i$-th bit in $x$ is 1 return $x$}
	\label{switch1to1}
	\begin{algorithmic}
		\State $u \gets x,\; j \gets i$ \Comment{Get the $i$-th bit on $x$ and store it in $u$}
		\While{$j > 0$}
		\State $v \gets 0$
		\While{$u > 1$}
		\State $u \gets u-2,\; v \gets v+1$
		\EndWhile
		\State $u\gets v,\; j \gets j-1$
		\EndWhile
		\While{$u > 1$}
		\State $u \gets u-2$
		\EndWhile
		\State $\textbf{assert } u = 1$ \Comment{If $u \neq 0$ simply stop}	
		\State \Return $x$.
	\end{algorithmic}
\end{algorithm}


%
%
%
%
%
%
%\medskip
%
%\subsection*{Proof of Theorem \ref{tqso-fo-fpsace}}
%
%We separate the proof in two parts. Let $\R$ be a relational signature. First we prove that for every formula $\alpha$ in $\tqso$ over $\R$ there exists a function $f\in\shpspace$ such that $\sem{\alpha}(\A) = f(\enc(\A))$ for every $\A\in\ostr[\R]$. Then we prove that for every function $f\in \fpspace$ over $\R$ there exists a $\tqso(\fo)$ formula $\alpha$ such that $f(\enc(\A)) = \sem{\alpha}(\A)$ for every $\A\in\ostr[\R]$. By the inclusion of $\tqso(\fo)\subseteq\tqso$ and the equality $\shpspace = \fpspace$, this proves that both $\tqso$ and $\tqso(\fo)$ capture $\fpspace$ over ordered structures.
%
%\vspace{1em}
%For the first part, let $\alpha$ be a formula in $\tqso$ over $\R$. 
%We will construct a nondeterministic polynomial-space algorithm $M_{\alpha}$ that on input $(\enc(\A),\enc(v),\enc(V))$, accepts in $\sem{\alpha}(\A,v,V)$ paths, for each $(\A,v,V)\in\ostr[\R]^*$. Let $A = \{1,\ldots,n\}$  be the domain of $\A$. 
%First-order assignments are encoded as a simple mapping from every first-order variable mentioned in $\alpha$ to an element in $A$. 
%Second order assignments are encoded in polynomial space as a mapping from every second-order variable $X$ to a subset of $A^{\arity(X)}$. We now begin the construction of the algorithm. 
%If $\alpha = \varphi$, a $\so$ formula, we check if $(\A,v,V)\models\varphi$ in deterministic polynomial space, and accept if and only if it holds. 
%If $\alpha = s$, we generate $s$ branches and accept in all of them. 
%If $\alpha = (\alpha_1 + \alpha_2)$, we simulate $M_{\alpha_1}$ and $M_{\alpha_2}$, both on input $(A,v,V)$, on separate branches. 
%If $\alpha = (\alpha_1\cdot\alpha_2)$, we simulate $\alpha_1$ on input $(A,v,V)$ and if it accepts, instead of doing so, we simulate $\alpha_2$ on input $(A,v,V)$. 
%If $\alpha = \sa{x}\beta$, for each $a\in A$ we generate a different branch where we simulate $M_{\beta}$ on input $(A,v[a/x],V)$.
%If $\alpha = \pa{x}\beta$, we simulate $M_{\beta}$ on input $(A,v[1/x],V)$, and on each accepting branch, instead of accepting we simulate $M_{\beta}$ on input $(A,v[2/x],V)$, and so on. 
%If $\alpha = \sa{X}\beta$, for each $B\subseteq A^{\arity(X)}$ we generate a different branch where we simulate $M_{\beta}$ on input $(A,v,V[B/X])$.
%If $\alpha = \pa{X}\beta$, we simulate $M_{\beta}$ on input $(\A,v,V[B/X])$ consecutively for each $B\subseteq A^{\arity(X)}$. 
%If $\alpha = [\pth \varphi(\bar{x},\bar{X},\bar{y},\bar{Y})]$ where $\varphi$ is an $\so$ formula, we simulate the procedure that counts the number of paths for a graph of a given size. This procedure starts with $(\bar{b},\bar{B}) = (v(\bar{x}),V(\bar{X}))$, then on each iteration, nondeterministically chooses an assignment $(\bar{a},\bar{B})$ for $(\bar{x},\bar{X})$, and checks in polynomial space if $(\A,v,V)\models\varphi(\bar{a}',\bar{B}',\bar{a},\bar{B})$, where $(\bar{a}',\bar{B}')$ is the previously chosen value. If it holds, we continue, and otherwise we reject. If at any point we obtain that $(\bar{b}, \bar{B}) = (v(\bar{y}),V(\bar{Y}))$ we generate a new accepting branch, and we continue with the procedure.
%We compute $n^{\length(\bar{x})}\cdot \prod_{X\in\bar{X}} 2^{A^{\arity(X)}}$ iterations of the procedure, which generates exactly $\sem{[\pth \varphi(\bar{x},\bar{X},\bar{y},\bar{Y})]}(\A,v,V)$ accepting branches. 
%This ends the construction of the algorithm. 
%Consider $f$ as the $\shpspace$ function associated to this procedure and we have that for each finite $\R$-structure $\A$: $f(\enc(\A)) = \sem{\alpha}(\A)$.
%
%\vspace{1em}
%For the second part let $f\in\fpspace$ over some relational signature $\R$. We use our proof for theorem \ref{theo:capture-fpspace} and recall the formula
%$$
%\alpha := \sa{X}(\Phi(X)\mult\gamma(X))
%$$
%that satisfies $\sem{\alpha}(\A) = f(\enc(\A))$ for each $\R$-structure $\A$, where $\Phi(X)$ is a $\pfp$ formula that models a $\pspace$ language over the encoding of structures in $\ostr[\R\cup\{X\}]$. The typical reduction  for this \cite{G07} uses a first-order over $\R\cup\{X,R\}$ formula $\psi(R,\bar{x})$ and describes $\Phi = \exists\bar{y}\,[\mathbf{pfp}\,R(\bar{x}):\psi(R,\bar{x})](\bar{y})$ (note that $X$ might be mentioned in $\psi$). We will provide an equivalent formula $\zeta(X)$ in $\tqso(\fo)$:
%$$
%\zeta(X) := \sa{R}\sa{S}\sa{Y}[\pth\varphi(X,R,Y,S)]\mult\forall\bar{x}(X(\bar{x})\leftrightarrow Y(\bar{x}))\mult\forall\bar{y}\neg R(\bar{y})\mult\forall\bar{y}(S(\bar{y})\leftrightarrow\psi(X,S,\bar{y}))\mult\exists\bar{y}S(\bar{y}),
%$$
%where $\varphi$ is defined as:
%$$
%\varphi(X,R,Y,S) := \forall\bar{x}(X(\bar{x})\leftrightarrow Y(\bar{x}))\wedge\forall\bar{y}(S(\bar{y})\leftrightarrow\psi(X,R,\bar{y}))\wedge\neg\forall\bar{y}(R(\bar{y})\leftrightarrow\psi(X,R,\bar{y})).
%$$
%The formula $\neg\forall\bar{y}(R(\bar{y})\leftrightarrow\psi(R,\bar{y}))$ prevents $R$ from already being a fixed point. This way, the operator finds a fixed point $S$ and then stops iterating. It can be checked that in the graph described for this operator, each node has out-degree and in-degree at most 1. And so, $\sem{\zeta(X)}(\A,v,V)$ takes the value 0 or 1 for each $(\A,v,V)\in\ostr[\R]^*$. Then the formula
%$$
%\alpha := \sa{X}(\zeta(X)\mult\gamma(X))
%$$
%is in $\tqso(\fo)$ and satisfies $\sem{\alpha}(\A) = f(\enc(\A))$ for each $\A\in\ostr[\R]$.
%
%\subsection*{Proof of Theorem \ref{tqsos-shp}}
%
%
%\vspace{1em}
%Let $\R$ be a signature. We will first prove the first condition in Definition \ref{def:cap}. Let $\alpha$ be a formula in $\tqsos(\fo)$ over $\R$. 
%We will construct a nondeterministic polynomial-space algorithm $M_{\alpha}$ that on input $(\enc(\A),\enc(v),\enc(V))$, accepts in $\sem{\alpha}(\A,v,V)$ paths, for each $(\A,v,V)\in\ostr[\R]^*$. Let $A = \{1,\ldots,n\}$  be the domain of $\A$. 
%First-order assignments are encoded as a simple mapping from every first-order variable mentioned in $\alpha$ to an element in $A$. 
%Second order assignments are encoded in polynomial space as a mapping from every second-order variable $X$ to a subset of $A^{\arity(X)}$. We now begin the construction of the algorithm. 
%If $\alpha = \varphi$, a $\so$ formula, we check if $(\A,v,V)\models\varphi$ in deterministic polynomial space, and accept if and only if it holds. 
%If $\alpha = s$, we generate $s$ branches and accept in all of them. 
%If $\alpha = (\alpha_1 + \alpha_2)$, we simulate $M_{\alpha_1}$ and $M_{\alpha_2}$, both on input $(A,v,V)$, on separate branches. 
%If $\alpha = (\alpha_1\cdot\alpha_2)$, we simulate $\alpha_1$ on input $(A,v,V)$ and if it accepts, instead of doing so, we simulate $\alpha_2$ on input $(A,v,V)$. 
%If $\alpha = \sa{x}\beta$, for each $a\in A$ we generate a different branch where we simulate $M_{\beta}$ on input $(A,v[a/x],V)$.
%If $\alpha = \pa{x}\beta$, we simulate $M_{\beta}$ on input $(A,v[1/x],V)$, and on each accepting branch, instead of accepting we simulate $M_{\beta}$ on input $(A,v[2/x],V)$, and so on. 
%If $\alpha = \sa{X}\beta$, for each $B\subseteq A^{\arity(X)}$ we generate a different branch where we simulate $M_{\beta}$ on input $(A,v,V[B/X])$.
%If $\alpha = [\pth \varphi(\bar{x},\bar{X},\bar{y},\bar{Y})]$ where $\varphi$ is an $\so$ formula, we simulate the procedure that counts the number of paths for a graph of a given size. This procedure starts with $(\bar{b},\bar{B}) = (v(\bar{x}),V(\bar{X}))$, then on each iteration, nondeterministically chooses an assignment $(\bar{a},\bar{B})$ for $(\bar{x},\bar{X})$, and checks in polynomial space if $(\A,v,V)\models\varphi(\bar{a}',\bar{B}',\bar{a},\bar{B})$, where $(\bar{a}',\bar{B}')$ is the previously chosen value. If it holds, we continue, and otherwise we reject. If at any point we obtain that $(\bar{b}, \bar{B}) = (v(\bar{y}),V(\bar{Y}))$ we generate a new accepting branch, and we continue with the procedure.
%We compute $n^{\length(\bar{x})}\cdot \prod_{X\in\bar{X}} 2^{A^{\arity(X)}}$ iterations of the procedure, which generates exactly $\sem{[\pth \varphi(\bar{x},\bar{X},\bar{y},\bar{Y})]}(\A,v,V)$ accepting branches. 
%This ends the construction of the algorithm. 
%Consider $f$ as the $\shpspace$ function associated to this procedure and we have that for each finite $\R$-structure $\A$: $f(\enc(\A)) = \sem{\alpha}(\A)$.
%
%\vspace{1em}
%For the second part let $f\in\shp$ over some relational signature $\R$.
%



\end{document}


