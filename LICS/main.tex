\documentclass[conference]{IEEEtran}

\usepackage{cite}

\usepackage[utf8]{inputenc}	
\usepackage{amsmath}
\usepackage{amsfonts}
\usepackage{wrapfig}
%\usepackage{amssymb}
\usepackage{stmaryrd}
\usepackage{thmtools}
\usepackage{bbold}
\usepackage{multicol}
%\usepackage{MnSymbol}
\usepackage{tikz}
\usetikzlibrary{arrows,automata}
\usepackage{calc}
\usepackage{mathabx}
\usepackage[textwidth=2cm,textsize=small]{todonotes}

\usepackage{array}
\usepackage[caption=false,font=normalsize,labelfont=sf,textfont=sf]{subfig}
\usepackage{fixltx2e}
\usepackage{stfloats}
\usepackage{url}

\usetikzlibrary{chains,fit,shapes}
\usetikzlibrary{arrows,positioning} 

\tikzset{
    rect/.style={
           rectangle,
           rounded corners,
           draw=black, 
           thick,
           text centered},
    rectw/.style={
           rectangle,
           rounded corners,
           draw=white, 
           thick,
           text centered},
    sq/.style={
           rectangle,
           draw=black, 
           thick,
           text centered},
    sqw/.style={
           rectangle,
           draw=white, 
           thick,
           text centered},
    arrout/.style={
           ->,
           -latex,
           thick,
           },
    arrin/.style={
           <-,
           latex-,
           thick,
           },
    arrd/.style={
           <->,
           >=latex,
           thick,
           },
    arrw/.style={
           thick,
           }
}


\hyphenation{op-tical net-works semi-conduc-tor}

% commands
\newcommand{\marcelo}[1]{\todo[size=\scriptsize, color=blue!40]{#1}}
\newcommand{\cristian}[1]{\todo[size=\scriptsize,color=green!40]{#1}}

%logic
\newcommand{\fo}{{\rm FO}}
\newcommand{\so}{{\rm SO}}
\newcommand{\lfp}{{\rm LFP}}
\newcommand{\alfp}{{\bf alfp}}
\newcommand{\pth}{{\bf path}}
\newcommand{\tc}{{\rm TC}}
\newcommand{\dtc}{{\rm DTC}}
\newcommand{\pfp}{{\rm PFP}}
\newcommand{\eso}{\exists\so}
\newcommand{\first}{\operatorname{first}}
\newcommand{\last}{\operatorname{last}}
\newcommand{\succesor}{\operatorname{succ}}
\newcommand{\partition}{\operatorname{partition}}

\newcommand{\R}{\mathbf{R}}
\newcommand{\A}{\mathfrak{A}}
\newcommand{\all}{\text{\sc All}}
\newcommand{\allo}{\text{\sc AllOrd}}
\newcommand{\qso}{{\rm QSO}}
\newcommand{\qfo}{{\rm QFO}}
\newcommand{\eqso}{\Sigma\qso}
\newcommand{\fv}{\mathbf{FV}}
\newcommand{\sv}{\mathbf{SV}}
\newcommand{\arity}{{\rm arity}}
\newcommand{\shp}{\text{\sc \#P}}
\newcommand{\shl}{\text{\sc \#L}}
\newcommand{\spp}{\text{\sc span-P}}
\newcommand{\gp}{\text{\sc gap-P}}
\newcommand{\fp}{\text{\sc FP}}
\newcommand{\totp}{\text{\sc TotP}}
\newcommand{\fpspace}{\text{\sc FPSPACE}}
\newcommand{\nfpspace}{\natural\text{\sc PSPACE}}
\newcommand{\CC}{\mathscr{C}}
\newcommand{\KK}{\mathscr{K}}
\newcommand{\FF}{\mathscr{F}}
\newcommand{\GG}{\mathscr{G}}
\newcommand{\LL}{\mathscr{L}}
\newcommand{\QQ}{\mathscr{Q}}
\newcommand{\enc}{{\rm enc}}
\newcommand{\str}{\text{\sc Struct}}
\newcommand{\ostr}{\text{\sc OrdStruct}}
\newcommand{\res}[2]{#1|_{#2}}

%semiring
\newcommand{\nat}{\mathbb{N}}
\newcommand{\natinf}{\mathbb{N}_\infty}
\newcommand{\trop}{\mathbb{N}_{\min,+}}
\newcommand{\integ}{\mathbb{Z}}
\newcommand{\bln}{\mathbb{B}}
\newcommand{\pwset}[1]{2^{#1}}
\newcommand{\true}{\operatorname{true}}
\newcommand{\false}{\operatorname{false}}

\newcommand{\SR}{\bbS}
\newcommand{\add}{+}
\newcommand{\bigadd}{\sum}
\newcommand{\mult}{\cdot}
\newcommand{\bigmult}{\prod}
\newcommand{\adds}{\oplus}
\newcommand{\bigadds}{\bigoplus}
\newcommand{\mults}{\odot}
\newcommand{\bigmults}{\bigodot}
\newcommand{\zero}{\mathbb{0}}
\newcommand{\one}{\mathbb{1}}

%quantitative logic
\newcommand{\QL}{\operatorname{QL}}
\newcommand{\QMSO}{\operatorname{QMSO}}
\newcommand{\Op}{\operatorname{O}}
\newcommand{\sem}[1]{{\lsem{}{#1}\rsem}}
\newcommand{\pa}[1]{\Pi{#1}.\,}
\newcommand{\pas}{\Pi}
\newcommand{\paq}[1]{\Pi{#1}}
\newcommand{\sa}[1]{\Sigma{#1}.\,}
\newcommand{\sas}{\Sigma}
\newcommand{\saq}[1]{\Sigma{#1}}
\newcommand{\fpa}[1]{\overrightarrow{\prod}{#1}.\:}
\newcommand{\lmid}{\;\mid\;}

% equations and quotes skip
\abovedisplayskip=6pt 
\belowdisplayskip=6pt
\newenvironment{myquote}{\begin{quote}\vspace{-0.75mm}}{\end{quote}\vspace{-0.75mm}}

%tikz definition
\tikzset{
defaultstyle/.style={>=stealth,semithick, auto,font=\small,
initial text= {},
initial distance= {3.5mm},
accepting distance= {3.5mm}},
accepting/.style=accepting by arrow,
nstate/.style={circle, semithick,inner sep=1pt, minimum size=4mm}}





\begin{document}

\title{Descriptive complexity \\
	for counting complexity classes}


% author names and affiliations
% use a multiple column layout for up to three different
% affiliations
\author{\IEEEauthorblockN{Marcelo Arenas}
\IEEEauthorblockA{%Department of Computer Science\\
PUC Chile\\ %Pontificia Universidad Cat\'olica de Chile \\
marenas@ing.puc.cl}
\and
\IEEEauthorblockN{Martin Mu\~noz}
\IEEEauthorblockA{%Department of Computer Science\\
	PUC Chile\\ %Pontificia Universidad Cat\'olica de Chile \\
	mmunos@uc.cl}
\and
\IEEEauthorblockN{Cristian Riveros}
\IEEEauthorblockA{%Department of Computer Science\\
	PUC Chile\\ %Pontificia Universidad Cat\'olica de Chile \\
	cristian.riveros@uc.cl}}

\maketitle


\begin{abstract}
Descriptive Complexity has been very successful in characterizing complexity classes of decision problems in terms of the properties definable in some logics. However, descriptive complexity for counting complexity classes, such as $\fp$ and $\shp$, has not been systematically studied, and it is not as developed as its decision counterpart. In this paper, we propose a framework based on Weighted Logics to address this issue. Specifically, by focusing on the natural numbers we obtained a logic called Quantitative Second Order Logics (QSO), and show how some of its fragments can be used to capture fundamental counting complexity classes such as $\fp$, $\shp$ and $\fpspace$, among others. We also use QSO to define a hierarchy inside $\shp$, identifying counting complexity classes with good closure and approximation properties, and which admit natural complete problems. Finally, we add recursion to QSO, and show how this extension naturally captures lower counting complexity classes such as $\shl$.	
	
%The goal of descriptive complexity is to measure the complexity of a problem in terms of the logical constructors needed to express it. 
%The starting point of this branch of complexity theory is Fagin's theorem, which states that $\np$ is equal to existential second-order logic. Since then, many more decision complexity classes have been characterized in terms of the properties definable in a logic. 
%Unfortunately, descriptive complexity for function complexity classes (e.g. $\shp$, $\fp$) has not been systematically studied and it is not as developed as for the case of decision problems.
%
%In this paper, we propose to study the descriptive complexity of function complexity classes in terms of Weighted Logics (WL), a general logical framework that combines boolean formulas (e.g. FO, SO) with operations over a fix semirings (e.g. $\bbN$). 
%Specifically, we propose to restrict WL over natural numbers, called Quantitative Second Order Logics (QSO), and study its expressive power for defining function complexity classes over general structures. 
%Interestingly, we show that fragments of QSO naturally captures classes like $\shp$, $\fp$ and $\fpspace$ among others. 
%Then we study the structure of $\eqso$, a fragment of $\qso$ that captures $\shp$, by refining the hierarchy studied by Saluja et al. 
%In particular, we show fragments of $\eqso$ that have good properties in terms of decidability, closure properties, and natural complete problems. 
%Finally, we go beyond QSO by adding recursion over functions. We show that this new logic naturally captures $\fp$ and subclasses like $\shl$.


\end{abstract}

\IEEEpeerreviewmaketitle

\section{Introduction}
%!TEX root = main.tex

%\marcelo{Enfatizar el rol fundamental de la logica para obtener los resultados, por ejemplo para obtener cerrado bajo menos uno}
%
%\marcelo{Poner un comentario sobre funciones totales versus parciales, dado que estamos considerando clases con funciones totales}
%
%\marcelo{Enfatizar el rol fundamental de la logica para obtener los resultados, por ejemplo para obtener cerrado bajo menos uno}
%
%Strategy:
%\begin{enumerate}
%	\item Descriptive complexity and application.
%	
%	\item Counting complexity classes. 
%	
%	\item Our contribution in terms of logic.
%	
%	\item sharP and its structure. 
%	
%	\item Syntactic classes with good properties. 
%\end{enumerate}
%
%\cristian{Aqui empieza la intro.}

The goal of descriptive complexity is to measure the complexity of a problem in terms of the logical constructors needed to express it~\cite{immerman1999descriptive}. 
The starting point of this branch of complexity theory is Fagin's theorem, which states that $\np$ is equal to existential second-order logic. Since then, many more complexity classes have been characterized in terms of logics (see \cite{G07} for a survey) and descriptive complexity has found a variety of applications in different areas~\cite{immerman1999descriptive, L04}.
For instance, Fagin's theorem was the key ingredient to define the class {\sc MaxSNP}~\cite{PY91}, which was later shown to be a fundamental class in the study of hardness of approximation \cite{ALMSS98}. 
It is important to mention here that the definition of {\sc MaxSNP} would not have been possible without the machine-independent point of view of descriptive complexity, as pointed out in~\cite{PY91}.

Counting problems differ from decision problems in that the value of a function has to be computed.
More generally, a counting problem corresponds to compute a function $f$ from a set of instances (e.g. graphs, formulae, etc) to natural numbers.\footnote{This value is usually associated to counting the number of solution 
	%for a given instance 
	in a search problem, but here we consider a more general definition.} 
The study of counting problems has given rise to a rich theory of counting complexity classes \cite{HV95,F97,arora2009computational}. Some of these classes are natural counterparts of some classes of decision problems; for example, $\fp$ 
%(resp., $\fpspace$) 
is the class of all functions that can be computed in polynomial time, 
%(resp., polynomial space), 
the natural counterpart of $\ptime$.
% ($\pspace$ resp.). 
However, other function complexity classes have emerged from the need to understand the complexity of some computation problems for which little can be said if their decision counterparts are considered. This is the case of the class $\shp$, a counting complexity class introduced in \cite{Valiant79} to prove that natural problems like counting the number of satisfying assignments of a propositional formula or the number of perfect matchings of a bipartite graph~\cite{Valiant79} are difficult, namely, $\shp$-complete.
Starting from $\shp$,
many more natural 
%the zoo of 
counting complexity classes have been defined, such as 
%was open with other natural counting classes like 
$\shl$, $\spp$ and $\gp$~\cite{HV95,F97}.
%among others~\cite{HV95,F97}.

Although counting problems play a prominent role in computational complexity, descriptive complexity for this type of problems has not been systematically studied and it is not as developed as for the case of decision problems. Insightful characterizations of $\shp$ and some of its extensions have been provided \cite{SalujaST95,ComptonG96}. However, these characterizations do not define function problems in terms of a logic, but instead in terms of some counting problems associated to a logic like $\fo$. Thus, it is not clear how these characterizations can be used to provide a general descriptive complexity framework for counting complexity classes like $\fp$ and $\fpspace$ (the class of functions computable in polynomial space). 
%It should be mentioned that logical definability has also been studied for the case of optimization problems \cite{KT94} and computation over the real numbers \cite{GM95}. As for the previous cases, it is not clear how these approaches can be extended to provide logical characterizations of a variety of function complexity classes. 

In this paper, we propose to study the descriptive complexity of counting complexity classes in terms of Weighted Logics (WL)~\cite{DrosteG07}, a general logical framework that combines Boolean formulae (e.g. in $\fo$ or $\so$) with operations over a fix semirings (e.g. $\bbN$). 
Specifically, we propose to restrict WL over natural numbers, called Quantitative Second Order Logics (QSO), and study its expressive power for defining counting complexity classes over general structures. 
As a proof of concept, we show that natural syntactical fragments of $\qso$ captures counting complexity classes like $\shp$, $\spp$, $\fp$ and $\fpspace$.
Furthermore, by slightly extending the framework we can prove that $\qso$ can also capture classes like $\gp$ and $\optp$, showing the robustness of our approach.

The next step is to use the machine-independent point of view of $\qso$ to search for subclasses of $\shp$ with some fundamental properties.
%inside $\shp$. 
The question here is, what properties are desirable for a subclass of $\shp$?
First, it is desirable to have a class of counting problems whose associated decision versions are tractable, in the sense that one can decide in $\ptime$ whether the value of the function is greater than $0$. 
In fact, this requirement is crucial in order to have any chance of finding efficient approximation algorithms for a given function (see Section~\ref{sec:syntactic}).
Second, we expect that our class is closed under basic arithmetical operations like sum, multiplication and subtraction by one. 
This is a common topic for counting complexity classes; for example, it is known that $\shp$ is not closed under subtraction by one (under some complexity-theoretical assumption). 
Finally, we want a class with natural complete problems, which characterize all problems in it.
%from the complexity point of view.

In this paper, we give the first results towards defining subclasses of $\shp$ that are robust in terms of approximation, closure properties, and natural complete problems. 
Specifically, we introduce a syntactic hierarchy inside $\shp$, called $\eqso(\fo)$-hierarchy, and we show that it is closely related to the $\fo$-hierarchy introduced in~\cite{SalujaST95}. 
Looking inside the $\eqso(\fo)$-hierarchy, we propose the class $\eqso(\logex{1})$ and show that every function in it has a tractable associated decision version, and it is closed under sum, multiplication, and subtraction by one.
%minus one. 
Unfortunately, it is not clear whether this class admits 
%we cannot show a
a natural complete problem.
% for $\eqso(\logex{1})$.
Thus, 
%Despite of this 
we also introduce a Horn-style syntactic class, inspired by the approach in~\cite{G92}, that has tractable associated decision versions and a natural complete problem.

After studying the 
%internal 
structure of $\shp$, we move beyond $\qso$ by introducing new quantifiers. 
By adding functional variables in top of $\qso$, we introduce a quantitative least fixed point operator to the logic. 
Adding finite recursion to a numerical setting is subtle since functions over natural numbers can easily diverge without finding any fixed point. 
By using the support of the functions, we give a natural halting condition that generalizes the least fixed point operator of Boolean logics. 
Then, with a quantitative recursion at hand we show how to capture $\fp$ from a different perspective and, moreover, how to restrict recursion to capture lower complexity classes 
%below $\fp$ 
such as~$\shl$, the counting version of $\nlog$.
%(Nondeterministic Logarithmic-space).

\smallskip

\noindent{\bf Organisation.} The main terminology used in the paper is given in Section~\ref{sec:preliminaries}. Then the logical framework is introduced in Section~\ref{sec:logic}, and it is used to capture standard counting complexity classes in Section~\ref{sec:complexity}. The structure of $\shp$ is studied in Section~\ref{sec:syntactic}. Section~\ref{sec:beyond} is devoted to define recursion in $\qso$, and to show how to capture classes below $\fp$. 
Finally, we give some concluding remarks in Section~\ref{sec:conclusions}. 

\section{Preliminaries} \label{sec:preliminaries}
%!TEX root = main.tex

A $\mathbb{N}$ relational signature $\R$ is a finite set $\{R_1, \ldots, R_k\}$, where each $R_i$ ($1 \leq i \leq k$) is a relation name with an associate arity greater than 0, which is denoted by $\arity(R_i)$. A finite structure over $\R$ (or just finite $\R$-structure) is a tuple $\A = \langle A, R_1^\A, \ldots, R_k^\A \rangle$ such that $A$ is a finite set and $R_i^\A \subseteq A^{\arity(R_i)}$ for every $i \in \{1, \ldots, k\}$. A finite $\R$-structure $\A$ is said to be ordered if $<$ is a binary predicate name in $\R$ and $<^\A$ is a linear order on $A$. Let $\str[\R]$ be the class of all finite $\R$-structures and $\ostr[\R]$ be the class of all ordered finite $\R$-structures. 

From now on, assume given disjoint infinite sets $\fv$ and $\sv$ of first-order variables and second-order variables, respectively. Notice that every variable in $\sv$ has an associated arity, which is denoted by $\arity(X)$. Then given a relational signature $\R$, the set of second-order logic formulas ($\so$-formulas) over $\R$ is given by the following grammar:
\begin{eqnarray*}\ 
	\varphi &:=& R(\bar u) \ \mid\  
	X(\bar v)  \ \mid\ 
	\neg \varphi \ \mid\ 
	(\varphi \vee \varphi) \ \mid\ 
	\exists x \, \varphi \ \mid\ 
	\exists X \, \varphi
\end{eqnarray*}
where $R \in \R$, $\bar u$ is a tuple of (non-necessarily distinct) variables from $\fv$ whose length is $\arity(R)$, $X \in \sv$, $\bar v$ is a tuple of (non-necessarily distinct) variables from $\fv$ whose length is $\arity(X)$, and $x \in \fv$. 

To define the semantics of $\so$, we need to introduce some terminology. Given a relational signature $\R$ and a finite $\R$-structure $\A$ with domain $A$, a first-order assignment $v$ for $\A$ is a total function from $\fv$ to $A$, while a second-order assignment $V$ for $\A$ is a total function with domain $\sv$ that maps each $X \in \sv$ to a subset of $A^{\arity(X)}$. Moreover, given a first-order assignment $v$ for $\A$, $x \in \fv$ and $a \in A$, we denote by $v[a/x]$ a first-order assignment such that $v[a/x](x) = a$ and $v[a/x](y) = v(y)$ for every $y \in \fv$ distinct from $x$. Similarly, given a second-order assignment $V$ for $\A$, $X \in \sv$ and $B  \subseteq A^{\arity(X)}$, we denote by $V[B/X]$ a second-order assignment such that $V[B/X](X) = B$ and $V[B/X](Y) = V(Y)$ for every $Y \in \sv$ distinct from $X$. 

Assume that $\varphi$ is an $\so$-formula over a signature $\R$. Then given a finite $\R$-structure $\A$ with domain $A$, a first-order assignment $v$ for $\A$ and a second-order assignment $V$ for $\A$, we say that $(\A, v, V)$ satisfies $\varphi$, denoted by $(\A, v, V) \models \varphi$, if: (1) $\varphi$ is the formula $R(x_1, \ldots, x_\ell)$ and $(v(x_1), \ldots, v(x_\ell)) \in R^\A$; (2) $\varphi$ is the formula $X(x_1, \ldots, x_m)$ and $(v(x_1), \ldots, v(x_m)) \in V(X)$; (3) $\varphi$ is the formula $\neg \psi$ and it not the case that $(\A, v, V) \models \psi$; (4) $\varphi$ is the formula $(\varphi_1 \vee \varphi_2)$, and $(\A, v, V) \models \varphi_1$ or $(\A, v, V) \models \varphi_2$; (5) $\varphi$ is the formula $\exists x \, \psi$ and there exists $a \in A$ such that $(\A, v[a/x], V) \models \psi$; or (6) $\varphi$ is the formula $\exists X \, \psi$ and there exists $B \subseteq A^{\arity(X)}$ such that $(\A, v, V[B/X]) \models \psi$.  As usual, we consider the propositional operators $\wedge$, $\rightarrow$, and $\leftrightarrow$ that can be obtained from $\vee$ and $\neg$. 
%Moreover, we use the abbreviations $x \not \leq y$ and $x \notin X$ for the negation of the atoms $\leq$ and $\in$. 
%Finally, we consider standard abbreviations of formulas that can be defined in $\so$-logic (actually, $\fo$-logic) like $\first(x) := \fa{x} y \leq x$ and $\last(x) := \fa{x} x \leq y$ to denote the first and last element of the linear order $\leq$, respectively, and $\succesor(x,y) := x \leq y \wedge y \not\leq x \wedge \fa{z} ( z \leq x \vee y \leq z)$ to denote the successor relation.

\marcelo{Agregar las definiciones de LFP y PFP}
\subsection{Function complexity classes}
Assume that $\R = \{R_1, \ldots, R_k\}$ is a relational signature and $\A$ is a finite $\R$-structure with a domain $A$ containing $n$ elements, and assume that  $<$ is a linear order on $A$, say $a_1 < a_2 < \ldots < a_n$. For every $i \in \{1, \ldots, k\}$, define the encoding of $R_i^\A$, denoted by $\enc(R_i^\A)$, as the following binary string. Assume that $\ell = \arity(R_i)$ and consider an enumeration of the $\ell$-tuples over $A$ in the lexicographic order induced by $<$ (that is, $(a_1, \ldots, a_1, a_1)$, $(a_1, \ldots, a_1, a_2)$, $\ldots$, $(a_n, \ldots, a_n, a_{n-1})$, $(a_n, \ldots, a_n, a_n)$). Then let $\enc(R_i^\A)$ be a binary string of length $n^\ell$ such that the $i$-th bit of $\enc(R_i^\A)$ is 1 if the $i$-th tuple in the previous enumeration belongs to $R_i^\A$, and 0 otherwise. Moreover, define the encoding of $\A$, denoted by $\enc(\A)$, as the following binary string~\cite{L04}:
\begin{eqnarray*}
	\enc(\A) & = & 0^n \, 1 \, \enc(R_1^\A) \, \cdots \, \enc(R_k^\A).
\end{eqnarray*}
Given a set $\bbD$ and a relational signature $\R$, a function $f$ is said to be from $\R$ to $\bbD$ if $f$ is a total function from $\{\enc(\A) \mid \A \in \str[\R]\}$ to $\bbD$. Moreover, a set $\CC$ of functions over $\bbD$ is said to be a {\em function complexity class over $\bbD$} if every $f \in \CC$ is a function from some relational signature $\R$ to $\bbD$.

\marcelo{Ya no consideramos clases de estructuras, hay que revisar que esta seccion tenga una notacion consistente}

%If $\KK$ is a class of finite structures and $f$ is a function from a relational signature $\R$ to $\bbD$, we denote by $\res{f}{\KK}$ the restriction of $f$ to $\KK$, that is, a function such that: the domain of $\res{f}{\KK}$ is $\{ \enc(\A) \mid \A \in \str[\R] \cap \KK\}$, and 
%$\res{f}{\KK}(\enc(\A)) = f(\enc(\A))$ for every $\A \in \str[\R] \cap \KK$.

\section{A logic for quantitative functions} \label{sec:logic}
%!TEX root = main.tex

We introduce here the logical framework that we use for studying counting complexity classes. 
This framework is based on the framework of Weighted Logics (WL)~\cite{DrosteG07}  that has been used in the context of weighted automata for studying functions from words (or trees) to semirings. 
We propose here to use the framework of WL over any relational structure and to restrict the semiring to natural numbers. 
The extension to any relational structure will allow us to study general counting complexity classes and the restriction to the natural numbers will simplify the notation in this context (see Section~\ref{sec:previous} for a more detailed discussion).

Given a relational signature $\R$, the set of Quantitative Second-Order logic formulae (or just $\qso$-formulae) over $\R$ is given by the following grammar:
%\[
%\begin{array}{rcl}
%\alpha & := & \varphi \ \mid \ s \ \mid \ (\alpha \add \alpha) \ \mid\ (\alpha \mult \alpha) \ \mid \ \\
%& &  \sa{x} \alpha \ \mid \pa{x} \alpha \ \mid \ \sa{X} \alpha \ \mid \ \pa{X} \alpha 
%\end{array}
%\]
\begin{align}
\alpha \ &:= \ \varphi \ \mid \ s \ \mid \ (\alpha \add \alpha) \ \mid\ (\alpha \mult \alpha) \ \mid \sa{x} \alpha \ \mid \ \pa{x} \alpha \ \mid \ \sa{X} \alpha \ \mid \ \pa{X} \alpha \label{syntax} 
\end{align}
where $\varphi$ is an $\so$-formula over $\R$, $s \in \bbN$, $x \in \fv$ and $X \in \sv$. Moreover, if $\R$ is not mentioned, then $\qso$ refers to the set of $\qso$ formulae over all possible relational signatures.
%\marcelo{En la gramatica de la logica deberiamos incluir la formula $\top$ que discutimos, la cual representa true y para la cual se tiene que $\sem{\top}(\A, v, V) = 1$. Les parece?}

The syntax of QSO formulae is divided in two levels. 
The first level is composed by $\so$-formulae over $\R$ (called Boolean formulae) and the second level is made by counting operators of addition and multiplication. 
For this reason, the quantifiers in $\so$ (e.g. $\exists x$ or $\exists X$) are called Boolean quantifiers and the quantifiers that make use of addition and multiplication (e.g. $\Sigma x$ or $\Pi X$) are called {\em quantitative quantifiers}.
Furthermore, $\Sigma x$ and $\Sigma X$ are called first- and second-order sum, whereas $\Pi x$ and $\Pi X$ are called first- and second-order product, respectively.
%This division between Boolean and quantitative level is essential for understanding the difference between the logic and the quantitative part. 
This separation between the Boolean and quantitative levels is essential for understanding the difference between the logic and the quantitative parts of the framework.
%\martin{creo que eso quer\'ia decir esta oraci\'on}
%\marcelo{OK}
Furthermore, this will later allow us to parametrize both levels of the logic in order to capture different counting complexity classes.

\begin{table}
	\addtolength{\jot}{0.5em}
	\begin{align*}
	\sem{\varphi}(\A, v, V) & = 
	\begin{cases}
	1 & \mbox{if } (\A, v, V) \models \varphi \\
	0 & \mbox{otherwise}
	\end{cases}\\
	\sem{s}(\A, v, V) & = s \\
	\sem{\alpha_1 \add \alpha_2}(\A, v, V) & = \sem{\alpha_1}(\A, v, V) + \sem{\alpha_2}(\A, v, V)\\
	\sem{\alpha_1 \mult \alpha_2}(\A, v, V) & = \sem{\alpha_1}(\A, v, V) \cdot \sem{\alpha_2}(\A, v, V)\\ 
	\sem{\sa{x} \alpha}(\A, v, V) & = \displaystyle \sum_{a \in A} \sem{\alpha}(\A,v[a/x],V)\\
	\sem{\pa{x} \alpha}(\A, v, V) & = \displaystyle \prod_{a \in A} \sem{\alpha}(\A,v[a/x],V)\\
	\sem{\sa{X} \alpha}(\A, v, V) & = \displaystyle \sum_{B \subseteq A^{\arity(X)}} \sem{\alpha}(\A, v, V[B/X])\\
	\sem{\pa{X} \alpha}(\A, v, V) & = \displaystyle \prod_{B \subseteq A^{\arity(X)}} \sem{\alpha}(\A, v, V[B/X])
	\end{align*}
	\caption{The semantics of QSO formulae.}
	\label{tab-semantics}
\end{table}
Let $\R$ be a signature, $\A$ an $\R$-structure with domain $A$, $v$ a first-order assignment for $\A$ and $V$ a second-order assignment for $\A$. Then the \emph{evaluation} of a $\qso$-formula $\alpha$ over $(\A, v, V)$ is defined as a function $\sem{\alpha}$ that on input $(\A, v, V)$ returns a number in $\bbN$. Formally, the function $\sem{\alpha}$ is recursively defined in Table~\ref{tab-semantics}.
A $\qso$-formula $\alpha$ is said to be a \emph{sentence} if it does not have any free variable, that is, every variable in $\alpha$ is under the scope of a usual quantifier or a quantitative quantifier. It is important to notice that if $\alpha$ is a $\qso$-sentence over a signature $\R$, then for every $\R$-structure $\A$, first-order assignments $v_1$, $v_2$ for $\A$ and second-order assignments $V_1$, $V_2$ for $\A$, it holds that $\sem{\alpha}(\A, v_1, V_1) = \sem{\alpha}(\A, v_2, V_2)$.
Thus, in such a case we use the term $\sem{\alpha}(\A)$ to denote $\sem{\alpha}(\A, v, V)$, for some arbitrary first-order assignment $v$ for $\A$ and some arbitrary second-order assignment $V$ for $\A$. 
\begin{exa}\label{ex:cliques}
Let $\bG = \{E(\cdot,\cdot),<\}$ be the vocabulary for graphs and $\fG$ be an ordered $\bG$-structure encoding a non-directed graph. 
Suppose that we want to count the number of triangles in $\fG$. Then this can be defined as follows:
\begin{align*}
\alpha_1 \ &:= \ \sa{x} \sa{y} \sa{z} ( E(x,y) \, \wedge \, E(y,z) \, \wedge \, E(z,x) \, \wedge x < y \, \wedge \, y < z )
\end{align*}
We encode a triangle in $\alpha_1$ as an increasing sequence of nodes $\{x, y, z\}$, in order to count each triangle once. Then the Boolean subformula  $E(x,y) \wedge E(y,z) \wedge E(z,x) \wedge
x < y \wedge y < z$ is checking the triangle property, by returning $1$ if $\{x, y, z\}$ forms a triangle in $\fG$ and $0$ otherwise.
Finally, the sum quantifiers in $\alpha_1$ aggregate all the values, counting the number of triangles in $\fG$.

Suppose now that we want to count the number of cliques in~$\fG$. We can define this function with the following formula:
\[
\alpha_2 \ := \ \sa{X} \clique(X),
\] 
where $\clique(X) := \fa{x} \fa{y} ((X(x) \wedge X(y) \wedge x \neq y)  \rightarrow E(x,y))$.
In the Boolean sub-formula of $\alpha_2$ we check whether $X$ is a clique, and with the sum quantifier we add one for each clique in $\fG$. 
But in contrast to $\alpha_1$, 
in $\alpha_2$ we need a second-order quantifier in the quantitative level.
This is according to the
complexity of evaluating each formula:
$\alpha_1$ defines an $\fp$-function while $\alpha_2$ defines a $\shp$-complete function. \qed
\end{exa}
\begin{exa}\label{exa-perm}
For an example that includes multiplication, let $\bM = \{M(\cdot,\cdot),<\}$ be a vocabulary for storing 0-1 matrices; in particular, a structure $\fM$ over $\bM$ encodes a 0-1 matrix $A$ as follows: if $A[i,j] = 1$, then $M(i,j)$ is true, otherwise $M(i,j)$ is false.
Suppose now that we want to compute the permanent of an $n$-by-$n$ 0-1 matrix $A$, that is:
\begin{align*}
\op{perm}(A) &= \sum_{\sigma \in S_n} \prod_{i=1}^n A[i, \sigma(i)],  
\end{align*}
where $S_n$ is the set of all permutations over $\{1, \ldots, n\}$.
The permanent is a fundamental function on matrices that has found many applications;
% in different areas,
%~\cite{permanent-applications},
in fact, showing that this function is hard to compute was one of the main motivations behind the definition of the class $\shp$~\cite{Valiant79}.
%and it was one of the first function that was shown to be difficult for counting~\cite{Valiant79} (i.e. $\shp$-complete). 

To define the permanent of a 0-1 matrix in $\qso$, assume that for a binary relation symbol $S$, $\op{permut}(S)$ is an $\fo$-formula that is true if, and only if, $S$ is a permutation, namely, a total bijective function (the definition of $\op{permut}(S)$ is straightforward).
%\martin{abreviar $\op{permutation}$ a $\op{permut}$ ahorra un par de lineas}
Then the following is a $\qso$-formula defining the permanent of a matrix:
\[
\alpha_3 := \sa{S} \op{permut}(S) \cdot \pa{x} (\ex{y} S(x,y) \wedge M(x,y)).
\]
Intuitively, the subformula $\beta(S) := \pa{x} (\ex{y} S(x,y) \wedge M(x,y))$ calculates the value \linebreak $\prod_{i=1}^n A[i, \sigma(i)]$ whenever $S$ encodes a permutation $\sigma$.
Moreover, the subformula $\op{permut}(S) \cdot \beta(S)$ returns $\beta(S)$ when $S$ is a permutation, and returns $0$ otherwise (i.e. $\op{permut}(S)$ behaves like a filter). 
Finally, the second order sum aggregates these values iterating over all binary relations and calculating the permanent of the matrix.
We would like to finish with this example by highlighting the similarity of $\alpha_3$ to the permanent formula. 
Indeed, an advantage of $\qso$-formulae is that the first- and second-order quantifiers in the quantitative level naturally reflect the operations used to define mathematical formulae. \qed
\end{exa}

We consider several fragments of $\qso$, which are obtained by restricting the syntax of the Boolean formulae or the use of the quantitative quantifiers, and we consider some extensions that are obtained by adding recursive operators to $\qso$.
In this regard, we denote by $\qfo$ the fragment of $\qso$ where second-order sum and product are not allowed. 
For instance, for the $\qso$-formulae defined in Example \ref{ex:cliques}, we have that $\alpha_1$ is in $\qfo$ and $\alpha_2$ is not.
Another interesting fragment of $\qso$ consists of the $\qso$-formulae where only sum operators and sum quantifiers are allowed. 
Formally, we denote by $\eqso$ the fragment of $\qso$ where first- and second-order products (i.e. $\pa{x}$ and $\pa{X}$) are not allowed.
For example, $\alpha_1$ and $\alpha_2$ in Example \ref{ex:cliques} are formulae of $\eqso$, while $\alpha_3$ in Example \ref{exa-perm} is not. 
We also consider fragments of $\qso$ by further restricting the Boolean part of the logic.
If $\LL$ is a fragment of $\so$, then we define the quantitative logic $\qso(\LL)$ to be the fragment of $\qso$ obtained by restricting $\varphi$ in \eqref{syntax} to be a formula in $\LL$. Moreover, we also restrict other fragments of $\qso$ by using the same idea. 
For example, we define $\qfo(\fo)$ to be the fragment of $\qfo$ obtained by restricting $\varphi$ in \eqref{syntax} to be an $\fo$-formula, and likewise for $\eqso(\fo)$.

In the following section, we use different fragments or extensions of $\qso$ to capture counting complexity classes. But before doing this, we show the connection of $\qso$ to previous frameworks for defining functions over relational structures.

\subsection{Previous frameworks for quantitative functions} \label{sec:previous}

In this section, we discuss some previous frameworks proposed in the literature and how they differ from our approach.
We start by discussing the connection between $\qso$ and weighted logics (WL)~\cite{DrosteG07}. 
As it was previously discussed, $\qso$ is a fragment of WL.
The main difference is that we restrict the semiring used in WL to natural numbers in order to study counting complexity classes.
Another difference between WL and our approach is that, to the best of our knowledge, this is the first paper to study weighted logics over general relational signatures, in order  to do descriptive complexity for counting complexity classes. 
Previous works on WL usually restrict the signature of the logic to strings, trees, and other specific structures (see \cite{droste2009handbook} for more examples), and they did not study the logic over general structures. 
Furthermore, in this paper we propose further extensions for $\qso$ (see Section~\ref{sec:beyond}) which differ from previous approaches in WL.

Another approach that resembles $\qso$ are logics with counting~\cite{IL90,E97,GG98,L04}, which include operators that extend $\fo$ with quantifiers that allow to count in how many ways a formula  is satisfied (the result of this counting is a value of a second sort, in this case the  natural numbers). 
In contrast to our approach, counting operators are usually used for checking Boolean properties over structures and not for producing values (i.e. they do not define a function).
In particular, we are not aware of any paper that uses this approach for capturing counting complexity classes.

Finally, the work in~\cite{SalujaST95,ComptonG96,0001HKV16} is of particular interest for our research. 
In~\cite{SalujaST95}, it was proposed to define a function over a structure by using free variables in an SO-formula; in particular, the function is defined by the number of instantiations of the free variables that are satisfied by the structure.
Formally, Saluja et. al \cite{SalujaST95} define a family of counting classes $\#\LL$ for a fragment $\LL$ of $\fo$. For a formula $\varphi(\bar{x},\bar{X})$ over $\R$, the function $f_{\varphi(\bar x, \bar X)}$ is defined as
$
f_{\varphi(\bar x, \bar X)}(\A) = \vert \{(\bar{a},\bar{A}) \mid \A\models\varphi(\bar{a},\bar{A})\}\vert
$
for every $\A\in\ostr[\R]$. Then a function $g\colon \ostr[\R]\to\nat$ is in $\#\LL$ if there exists a formula $\varphi(\bar{x},\bar{X})$ in $\LL$ such that $g = f_{\varphi(\bar x, \bar X)}$.
In~\cite{SalujaST95}, they proved several results about capturing counting complexity classes which are relevant for our work. We discuss and use these results in Sections~\ref{sec:complexity} and~\ref{sec:syntactic}.
Notice that for every formula $\varphi(\bar{x},\bar{X})$, it holds that $f_{\varphi(\bar{x},\bar{X})}$ is the same function as $\sem{\sa{\bar{X}} \sa{\bar{x}} \varphi(\bar{x},\bar{X})}$, that is, the approach in \cite{SalujaST95} can be seen as a syntactical restriction of our approach based on $\qso$. 
Thus, the advantage of our approach relies on the flexibility to define functions by alternating sum with product operators and, moreover, by introducing new quantitative operators (see Section~\ref{sec:beyond}).
Furthermore, we show in the next section how to capture some classes that cannot be captured by following the approach in~\cite{SalujaST95}.


\section{Counting under $\qso$} \label{sec:complexity}
%!TEX root = main.tex

In this section, we show that by syntactically restricting $\qso$ one can capture different counting complexity classes. 
In other words, by using $\qso$ we can extend the theory of descriptive complexity~\cite{immerman1999descriptive} from decision problems to counting problems. 
For this, we first formalize the notion of \emph{capturing} a complexity class of functions.
%, and then show how to capture classes like $\shp$, $\fp$, and $\fpspace$.

Fix a signature $\R = \{R_1, \ldots, R_k\}$ and assume that $\A$ is an ordered $\R$-structure with a domain $A = \{a_1, \ldots, a_n\}$, $R_k =\, <$, and $a_1 <^{\A} a_2 <^{\A} \ldots <^{\A} a_n$. For every $i \in \{1, \ldots, k-1\}$, define the encoding of $R_i^\A$, denoted by $\enc(R_i^\A)$, as the following binary string. Assume that $\ell = \arity(R_i)$ and consider an enumeration of the $\ell$-tuples over $A$ in the lexicographic order induced by $<$. 
%(that is, $(a_1, \ldots, a_1, a_1)$, $(a_1, \ldots, a_1, a_2)$, $\ldots$, $(a_n, \ldots, a_n, a_{n-1})$, $(a_n, \ldots, a_n, a_n)$). 
Then let $\enc(R_i^\A)$ be a binary string of length $n^\ell$ such that the $i$-th bit of $\enc(R_i^\A)$ is 1 if the $i$-th tuple in the previous enumeration belongs to $R_i^\A$, and 0 otherwise. Moreover, define the encoding of $\A$, denoted by $\enc(\A)$, as the string~\cite{L04}:
%following binary string~\cite{L04}:
%\begin{eqnarray*}
	%\enc(\A) & = & 0^n \, 1 \, \enc(R_1^\A) \, \cdots \, \enc(R_k^\A).
%\end{eqnarray*}
$$
0^n \, 1 \, \enc(R_1^\A) \, \cdots \, \enc(R_{k-1}^\A).
$$
\martin{modifique la definicion para que no sea necesario codificar el $<$.}
\marcelo{OK}
%We define the class of all $\R$-functions, denoted by $\Func(\R)$, as the class of all functions $f: \ostr \rightarrow \bbN$.
%Given a function complexity class $\CC$ (i.e. $f: \Sigma^* \rightarrow \bbN$ for every $f \in \CC$), we say that a function $f \in \Func(\R)$ can be computed in $\CC$ if there exists $g \in \CC$ such that $f(\A) = g(\enc(\A))$ for every $\A \in \ostr$. 
%Note that the function $g$ outputs $f$ for encodings of structures and can behave arbitrarily otherwise.
We can now formalize the notion of capturing a counting complexity class.
\begin{defi} \label{def:cap}
	Let $\FF$ be a fragment of $\qso$ and $\CC$ a counting complexity class. Then {\em  $\FF$ captures $\CC$ over ordered $\R$-structures} if the  following conditions hold:
	\begin{enumerate}
		\item for every $\alpha \in \FF$, there exists $f \in \CC$ such that $\sem{\alpha}(\A) = f(\enc(\A))$ for every $\A \in \ostr[\R]$. 
		
		\item for every $f \in \CC$, there exists $\alpha \in \FF$ such that   $f(\enc(\A)) = \sem{\alpha}(\A)$ for every $\A \in \ostr[\R]$.
	\end{enumerate} 
	Moreover, {\em $\FF$ captures $\CC$ over ordered structures} if $\FF$ captures~$\CC$ over ordered $\R$-structures for every signature~$\R$. \qed
\end{defi}
%For the sake of simplification, we denote the first condition by $\FF \subseteq \CC$ and the second condition by $\CC \subseteq \FF$.
In Definition~\ref{def:cap}, function $f \in \CC$ and formula $\alpha \in \FF$ must coincide in all the strings that encode ordered $\R$-structures. Notice that this restriction is natural as we want to capture %Since we want to capture 
$\CC$ over a fixed set of structures (e.g. graphs, matrices).
%, it is natural to just consider strings that encodes $\R$-structures. 
Moreover, this restriction is fairly standard in descriptive complexity \cite{immerman1999descriptive,L04}, and it has also been used in the previous work on capturing complexity classes of functions \cite{SalujaST95,ComptonG96}.
%all notions for capturing complexity classes restrict $f \in \CC$ similarly. 

What counting complexity classes can be captured with fragments of $\qso$?
For answering this question, it is reasonable to start with $\shp$, a well-known and widely-studied counting complexity class~\cite{arora2009computational}. 
Since $\shp$ has a strong similarity with $\np$, one could expect a ``Fagin-like'' Theorem~\cite{F75} for this class. 
Actually, in~\cite{SalujaST95} it was shown that the class $\sfo$ captures $\shp$.
In our setting, the class $\sfo$ is contained in $\eqso(\fo)$, which also captures $\shp$ as expected.
 
\begin{prop} \label{prop:capture-shP}
	$\eqso(\fo)$ captures $\shp$ over ordered structures.
\end{prop}
\proof
We briefly explain how the two conditions of Definition~\ref{def:cap} are satisfied. First, for condition (2) Saluja et al. proved that $\shp = \sfo$\cite{SalujaST95}. Hence, given that every function in $\sfo$ can be trivially defined as a formula in $\eqso(\fo)$ (see Section~\ref{sec:previous}), condition~(2) holds.
For condition (1), let $\alpha\in\eqso(\fo)$ over some signature $\R$. Given an $\fo$ formula $\varphi$, checking whether $\A\models\varphi$ can be done in deterministic polynomial time on the size of $\A$ and any constant function $s$ can be trivially simulated in $\shp$. These facts, together with the closures under exponential sum and polynomial product of $\shp$\cite{F97}, suffice to show that the function represented by $\alpha$ is in $\shp$.
%We construct recursively a $\shp$-machine $M_{\alpha}$ for each $\eqso(\fo)$ formula $\alpha$ over a signature $\R$. This machine, on input $(\A,v,V)$ accepts in $\sem{\alpha}(\A,v,V)$ of its non-deterministic paths for each $(\A,v,V) \in \ostr[\R]^*$. Suppose $\A$ has domain $A$. If $\alpha$ is a $\fo$-formula $\varphi$, then the machine checks if $(\A,v,V)\models\varphi$ deterministically in polynomial time, and accepts if and only if it holds true. If $\alpha$ is a constant $s$, it produces $s$ branches and accepts in all of them. If $\alpha = (\beta \add \gamma)$, then it chooses between 0 or 1, if it is 0 (1), it simulates $M_{\beta}$ ($M_{\gamma}$) on input $(\A,v,V)$. 
%If $\alpha = \sa{x}\beta$, it chooses $a\in A$ non-deterministically and simulates $M_{\beta}$ on input $(\A,v[a/x],V)$.
%If $\alpha = \sa{X}\beta$, it chooses $B\in A^{arity(X)}$ and simulates $M_{\beta}$ on input $(\A,v,V[B/X])$. This covers all possible cases for $\alpha$. Let $\alpha$ be a formula in $\eqso(\fo)$ over a signature $\R$ and let $f$ be a function over $\R$ such that $f(\enc(\A))$ is equal to the accepting paths of $M_{\alpha}$ on input $(\A,v,V)$ for some $(\A,v,V) \in \ostr[\R]^*$. We have that $f$ is a $\shp$-function over $\R$ and $f(\enc(\A)) = \sem{\alpha}(\A)$ for every $\A\in\ostr[\R]$.
 
\qed

By following the same approach as~\cite{SalujaST95}, Compton and Gr\"adel~\cite{ComptonG96} show that $\seso$ captures $\spp$, where $\eso$ is the existential fragment of $\so$. As one could expect, if we parametrize $\eqso$ with $\eso$, we can also capture~$\spp$.
\begin{prop} \label{prop:capture-spanP}
	$\eqso(\eso)$ captures $\spp$ over ordered structures.
\end{prop}
\proof
Similar than the previous proof, we construct recursively a $\spp$ machine $M_{\alpha}$ for each $\eqso(\eso)$ formula $\alpha$ over a signature $\R$. This machine, on input $(\A,v,V)$, non-deterministically produces $\sem{\alpha}(\A,v,V)$ distinct accepting outputs for each $(\A,v,V) \in \ostr[\R]^*$. Suppose $\A$ has domain $A$. 
If $\alpha$ is a $\eso$-formula $\varphi$ it checks if $(\A,v,V)\models\varphi$ non-deterministically in polynomial time \cite{F75}, and accepts if and only if the condition holds true. 
If $\alpha$ is a constant $s$, then the machine produces $s$ branches and accepts in all of them. 
If $\alpha = (\beta \add \gamma)$, then it chooses between 0 or 1, if it is 0 (1), it simulates $M_{\beta}$ ($M_{\gamma}$) on input $(\A,v,V)$.  
If $\alpha = \sa{x}\beta$, it chooses $a\in A$ non-deterministically and simulates $M_{\beta}$ on input $(\A,v[a/x],V)$. 
If $\alpha = \sa{X} \beta$, it chooses $B\in A^{\arity(X)}$ and simulates $M_{\beta}$ on input $(\A,v,V[B/X])$. 
This covers all possible cases for $\alpha$.
Additionally, the machine produces a different output on each path. This can be done by printing the trace of all the non-deterministic choices.
However, when the machine starts checking whether $(\A,v,V)\models\varphi$ for some $\eso$ formula $\varphi$, it stops printing in the output tape. This way the machine produces exactly one output from that point onwards.
Let $\alpha$ be a formula in $\eqso(\eso)$ over a signature $\R$ and let $f$ be a function over $\R$ such that $f(\enc(\A))$ is equal to the number of accepting outputs of $M_{\alpha}$ on input $(\A,v,V)$ for some $(\A,v,V) \in \ostr[\R]^*$. 
We have that $f$ is a $\spp$ function over $\R$ and that $f(\enc(\A)) = \sem{\alpha}(\A)$ for every $\A\in\ostr[\R]$.

For the other direction, Compton et al.~\cite{ComptonG96} proved that $\spp = \#\eso$. Since a function in $\#\eso$ can also be defined in $\eqso(\eso)$, then $\eqso(\eso)$ captures $\spp$ over ordered structures.
\qed
Can we capture $\fp$ by using $\# \LL$ for some fragment $\LL$ of $\so$? A first attempt could be based on the use of a fragment $\LL$ of $\so$ that captures either $\ptime$ or $\nlog$~\cite{G92}. Such an approach fails as $\# \LL$ can encode $\shp$-complete problems in both cases; in the first case, one can encode the problem of counting the number of satisfying assignments of a Horn  propositional formula, while in the second case one can encode the problem of counting the number of satisfying assignments of a 2-CNF propositional formula. A second attempt could then be based on considering a fragment $\LL$ of $\fo$. 
But even if we consider the existential fragment $\Sigma_1$ of $\fo$ the approach fails, as $\# \Sigma_1$ can encode $\shp$-complete problems like counting the number of satisfying assignments of a 3-DNF propositional formula\cite{SalujaST95}. One last attempt could be based on disallowing the use of second-order free variables in $\sfo$. But in this case one 
cannot capture exponential functions definable in $\fp$ such as~$2^n$.
Thus, it is not clear how to capture $\fp$ 
by following the approach proposed in~\cite{SalujaST95}. 
On the other hand, if we consider our framework and move out from $\eqso$, we have other alternatives for counting like first- and second-order products. In fact, the combination of $\qfo$ with $\lfp$ is exactly what we need to capture $\fp$.
\begin{thm} \label{theo:capture-fp}
	$\qfo(\lfp)$ captures $\fp$ over ordered structures.
\end{thm}
\proof
For the first condition, let $\alpha\in\qfo(\lfp)$ over some signature $\R$, defined by the grammar in \ref{syntax}. Notice that for each $\lfp$ formula $\varphi$,  checking whether $\A\models\varphi$ can be done in deterministic polynomial time on the size of $\A$ [cite here]. Also, the constant function $s$ can be trivially simulated in $\fp$. These facts, together with closure properties of $\fp$ of polynomial sum and product [cite here?] suffice to show that the function represented by $\alpha$ is in $\fp$.
	
For the second condition, let $f\in \fp$ and consider some signature $\R$.
Let $\ell\in\nat$ be such that for each $\A\in\ostr[\R]$, $\lceil\log_2 f(\enc(\A)) \rceil \leq \size{\A}^\ell$ (i.e. $\size{\A}^\ell$ is an upper bound for the output size).
Let $\bar{x} = (x_1,\ldots,x_{\ell})$.
Define a language
\[
L = \{(\A,a_1,\ldots,a_{\ell})\mid a_1,\ldots,a_{\ell}\in A \text{ and the } (a_1,\ldots,a_{\ell})\text{-th bit of }f(\enc(\A))\text{ is 1}\}.
\]

%Consider a procedure that receives $\enc(\A)$ and an assignment $\bar{a}$ to $\bar{x}$. Let $m$ be the position of $\bar{a}$ in the lexicographic order of the tuples in $A^{\ell}$. The procedure then computes the $m$-th bit of $f(\enc(\A))$, from least to most significant. 
Since this language is in $\ptime$, by \cite{I86} there exists a formula $\Phi(\bar{x})$ in $\lfp$ such that $\A\models\Phi(\bar{a})$ if and only if $(\A,\bar{a})\in L$. 
Then we use
$$
\alpha = \sa{\bar{x}} \Phi(\bar{x})\cdot\varphi(\bar{x}),
$$
where $\varphi(\bar{x}) := \pa{\bar{y}}(\bar{y} < \bar{x} \mapsto 2)$. This formula takes the value $2^m$ if there exists $m$ tuples in $A^{\ell}$ that are smaller than $\bar{x}$. Adding these values for each $\bar{a}\in A^{\ell}$ gives exactly $f(\enc(\A))$. 
In other words, $\Phi(\bar{x})$ simulates the behavior of the $\fp$-machine and the formula $\alpha$ reconstruct the binary output.
Then, $\alpha$ is in $\qfo(\lfp)$ over $\R$ and $\sem{\alpha}(\A) = f(\enc(\A))$.
\qed
%To prove this theorem, 
%capture $\fp$ 
%one first shows that every formula in $\qfo(\lfp)$ can be evaluated in polynomial time. 
%Indeed, $\lfp$ is a polynomial-time logic~\cite{I86,vardi1982complexity}, and the sum and product quantifiers can also be computed in polynomial time. 
%For the other direction, one has to use $\qfo(\lfp)$ to simulate the run of a polynomial time TM $M$ computing a function, in particular using the quantitative quantifiers to reconstruct the natural number returned by $M$ in the output tape. 
%It is important to notice that the alternation between sum and product quantifiers is crucial for this reconstructions and, thus, crucial for capturing $\fp$.

At this point it is natural to ask whether one can extend the previous idea to capture $\fpspace$~\cite{Ladner89}, the class of functions computable in polynomial space. 
Of course, for capturing this class one needs a logical core powerful enough, like $\pfp$, for simulating the run of a polynomial-space TM.
Moreover, 
one also needs more powerful quantitative quantifiers as functions like $2^{2^n}$ can be computed in polynomial space,
so $\eqso$ is not enough for the quantitative layer of a logic for $\fpspace$.
In fact, by considering second-order product we obtain the fragment $\qso(\pfp)$ that captures $\fpspace$. 
\begin{thm} \label{theo:capture-fpspace}
	$\qso(\pfp)$ captures $\fpspace$ over ordered structures.
\end{thm}
\proof
%!TEX root = main.tex

For the first condition of Definition~\ref{def:cap}, notice that each $\pfp$ formula can be evaluated in deterministic polynomial space, the constant function $s$ can be trivially simulated in $\fpspace$, and $\fpspace$ is closed under exponential sum and multiplication. This suffices to show that the condition holds.
For the second condition, the proof is similar to the proof of Theorem~\ref{theo:capture-fp}. Let $f\in \fpspace$ defined over some $\R$ and $\ell\in\nat$ such that $\log_2\left( f(\enc(\A)) \right) \leq 2^{{|\A|}^\ell}$ for every $\A\in\ostr[\R]$  (i.e. $2^{{|\A|}^\ell}$ is an upper bound for the size of the output). Let $X$ be a second-order variable of arity $\ell$. Consider the linear order induced by $<$ over predicates of arity $\ell$ which can be defined by the following formula:
$$
\varphi_{<}(X,Y) = \ex{\bar{u}}\big[\neg X(\bar{u})\wedge Y(\bar{u})\wedge \fa{\bar{v}}\big(
\bar{u}<\bar{v}\to(X(\bar{u})\iff Y(\bar{v}))\big)\big].
$$
Namely, we use predicates to encode a number that will have most $2^{{|\A|}^\ell}$ bits. We define this encoding through the function $\tau\colon 2^{A^\ell}\to\nat$, such that $\tau(B)$ is equal to the number of predicates in $2^{A^\ell}$ that are smaller than $B$ with respect to the induced order. For example, we have that $\tau(\emptyset) = 0$ and $\tau(A^{\ell}) = 2^{{|\A|}^\ell}-1$. Furthermore, we can use a relation~$X$ to index a position in the binary output of $f(\enc(\A))$ as follows.
%Consider a polynomial space machine over the $\R$ that receives as input an $\R$-structure $\A$ and a number $p$ encoded by a relation $X$. Then the machine accepts if, and only if, the $p$-th bit of $f(\enc(\A))$ is $1$. 
Define the language:
\[
L = \{(\A,B)\mid B \subseteq A^{\ell}\text{ and the $\tau(B)$-th bit of $f(\enc(\A))$ is 1}\}.
\]
Since $L$ is in $\pspace$, it can be specified in $\pfp$ \cite{AbiteboulV89} by a formula $\Phi(X)$ such that $\A\models\Phi(B)$ if and only if $(\A,B)\in L$. Then, similarly as for the previous proof we define:
$$
\alpha := \sa{X} \Phi(X)\mult  \pa{Y}(\varphi_{<}(Y,X)\mapsto 2).
$$ 
where $\pa{Y}(\varphi_{<}(Y,X)\mapsto 2)$ takes the value $2^{\tau(X)}$ and $\alpha$ reconstructs the output of $f(\enc(\A))$. Using an argument analogous to the previous proof, we conclude that $\alpha\in\qso(\pfp)$ and $\sem{\alpha}(\A) = f(\enc(\A))$.
%\martin{Reescrib\'i varias l\'ineas de esta demostraci\'on}

\qed
%The proof of the previous theorem follows the same line as for the logical characterization of $\fp$: one shows that each function in $\qso(\pfp)$ can be computed in $\fpspace$ and, conversely, the output of a polynomial-space TM can be reconstructed by using $\pfp$ and quantitative quantifiers.

From the proof of the previous theorem a natural question follows: what happens if we use first-order quantitative quantifiers and $\pfp$?
In~\cite{Ladner89}, Ladner also introduced the class $\nfpspace$ of all functions computed by polynomial-space TMs 
with output length bounded by a polynomial.
Interestingly, if we restrict to FO-quantitative quantifiers we can also capture this class.
\begin{cor} \label{cor:capture-fpspace-poly}
	$\qfo(\pfp)$ captures $\nfpspace$ over ordered structures.
\end{cor}
\proof
In this proof, both conditions are analogous to Theorem~\ref{theo:capture-fp} and~\ref{theo:capture-fpspace}. For the first condition, each $\pfp$ formula $\varphi$ can be evaluated in $\pspace$ and the class is closed under first sum and product. For the second condition, we use the same language $L$ defined in the proof of Theorem~\ref{theo:capture-fp}, which in this case is in $\pspace$. The same construction of $\alpha$, which in turn is in $\qfo(\pfp)$, is used to show that the condition holds.
\qed

The results of this section validate $\qso$ as an appropriate logical framework for extending the theory of descriptive complexity to counting complexity classes. In the following sections, we provide more arguments for this claim, by considering some fragments of $\eqso$ and, moreover, by showing how to go beyond $\eqso$ to capture other classes.


\section{Exploring the structure of $\shp$ through $\qso$} \label{sec:syntactic}
%!TEX root = main.tex

The class $\shp$ was introduced in \cite{Valiant79} to prove that computing the permanent of a matrix, as defined in Example \ref{exa-perm}, is a difficult problem. More specifically, it was shown in  \cite{Valiant79}  that this problem is $\shp$-complete. As a consequence of this result the problem of computing the number of perfect matchings in a bipartite graph was also shown to be $\shp$-complete. Since then  many counting problems have been proved to be $\shp$-complete \cite{V79b,PB83,P86,L86,BW91,HMRS98,BW05,DS12, PS13,PS14}. Among them, problems having easy decision counterparts play a fundamental role, as a counting problem with a hard decision version is expected to be hard. A first prominent example of such problems is counting the number of perfect matching in a bipartite graph, as it is well-known that the problem of verifying whether there exists a perfect matching in a bipartite graph can be solved in polynomial time. Other prominent examples of such problems include counting the number of: satisfying assignments of a 2-CNF propositional formula \cite{V79b}, satisfying assignments of a DNF propositional formula \cite{DHK05}, simple paths from a source node to a target node in a directed graph \cite{V79b}, extensions of a partial order to a linear order \cite{BW91} and Eulerian cycles in an undirected graph \cite{BW05}. 

Counting problems with easy decision versions play a fundamental role in the search of efficient approximations algorithms for functions in $\shp$. A fully-polynomial randomized approximation scheme (FPRAS) for a function $f : \Sigma^* \to \bbN$ is a randomized algorithm ${\cal A} : \Sigma^* \times (0,1) \to \bbN$ such that: (1) for every string $x \in \Sigma^*$ and real value $\varepsilon \in (0,1)$, the probability that $|f(x) - {\cal A}(x,\varepsilon)| \leq \varepsilon \cdot f(x)$ is at least $\frac{3}{4}$, and (2) the running time of ${\cal A}$ is polynomial in the size of $x$ and $1/\varepsilon$ \cite{KL83}. Notably, there exist $\shp$-complete functions that can be efficiently approximated as they admit FPRAS; for instance, there exist FPRAS for the problems of counting the number of satisfying assignments of a DNF propositional formula \cite{KL83} and the number of perfect matchings of a bipartite graph \cite{JSV04}. A key observation here is that if a $\shp$-complete function admits an FPRAS, then its associated decision problem is in the complexity class $\bpp$ (Bounded-Error Probabilistic Polynomial-Time). Hence, under standard complexity-theoretical assumptions we cannot hope for an FPRAS for a function in $\shp$ whose decision counterpart is $\np$-complete, and we have to concentrate on the class of counting problems with easy decision versions (in $\bpp$ or in a lower complexity class such as $\ptime$). 

The importance of the class counting problems with easy decision counterparts has motivated the search of robust definitions of classes of functions in $\shp$ with easy decision versions \cite{PagourtzisZ06}. In this section, we use the framework developed in this paper to address this problem. More specifically, we introduce in Section \ref{sec-hier-shp} a hierarchy of 


consider several fragments of $\eqso(\LL)$ where $\LL$ is a boolean logic contained in $\fo$, and we study 

% It should be noted that such class can be directly defined as the set functions f ? #P such that Lf ? P, which is denoted as #Pe in [50, 51]. However, such a definition does not lead to a well-behaved and robust function complexity class. In particular, for every function f ? #P, we have that f + 1 is trivially in #Pe, which is an undesirable property. This has led to the introduction of the more robust class TotP, which is defined as the class of functions f for which there exists a non-deterministic Turing machine M running in polynomial time such that, f(x) is the result of subtracting 1 to the number of (non-necessarily accepting) runs of M with input x [40]. In [51], it is proved that TotP ? #Pe and that TotP has a complete function problem under parsimonious reductions. However, no natural problem is known to be TotP-complete under this type of reductions [51].





In this section we study the fragment of $\eqso(\LL)$ when $\LL$ is a boolean logic contained in $\fo$. We show that by restricting $\LL$ we can find different subclasses below $\shp$ with interesting computational and closure properties. 

\cite{OH93,FH08}

From this point on, for each fragment $\FF$ of $\qso$, we will also use $\FF$ to refer to the class of functions defined by the formulas in $\FF$.

\subsection{A counting hierarchy below $\shp$}
\label{sec-hier-shp}
Saluja et. al \cite{DBLP:journals/jcss/SalujaST95} define a family of counting classes $\#\cL$ for each fragment $\cL$ of $\fo$. For a formula $\varphi(x,X)$, the function $f_{\varphi(x,X)}$ is defined as
\[
f_{\varphi(x,X)}(\A) = \vert \{\langle e,P\rangle\mid \A\models\varphi(e,P)\}\vert.
\]
for each $\A\in\str$. A function $f:\Sigma^*\to\nat$ is in $\#\cL$ if there exists an $\cL$ formula $\varphi(x,X)$ such that $f = f_{\varphi(x,X)}$.

\begin{theorem} \label{saluja-eq}
	$\eqso(\cL) = \#\cL$ for each fragment $\L$ of $\fo$.
\end{theorem}

For every logic $\cL$, we define an $\cL$-extended quantifier-free (QF) formula as follows:
\begin{eqnarray*}
	\varphi &::=& \alpha, \alpha \text{ is an $\cL$-formula} \ \mid \\
	&& X_i(x_1,\dots,x_{a_i}), i\in\N \ \mid \ \\
	&& (\neg \varphi) \ \mid \ (\varphi \wedge \varphi) \ \mid \ (\varphi \vee \varphi).
\end{eqnarray*}

We define syntactically the fragments $\logex{i}$ and $\logux{i}$ according to the following grammar:
\begin{align*}
\logex{0} = \logux{0} &::= \varphi , \varphi \mbox{ is an $\fo$-extended QF formula,} \\
\logex{i+1} &::= \logux{i} \ \mid \ \exists x\, \logex{i+1}, \\
\logux{i+1} &::= \logex{i} \ \mid \ \forall x\, \logux{i+1}.
\end{align*}

\begin{theorem} \label{fp1}
	$\qfo(\logex{0}) \subseteq$ {\sc FP}.
\end{theorem}

The {\em decision problem} associated to a function $f$ is defined by the language $L_f = \{\A \in \str \mid f(\A) > 0\}$.

\begin{theorem} \label{decisionptime}
	The decision problem associated to a function in $\eqso(\logex{1})$ is in \textsc{P}.
\end{theorem}

For a given pair of functions $f,g$, we define $f \dotminus g$ as follows:
\begin{eqnarray*}
	(f \dotminus g)(\A) =
	\begin{cases}
		f(\A)-g(\A), & \text{if }f(\A)>g(\A) \\
		0, & \text{if }f(\A) \leq g(\A).
	\end{cases}
\end{eqnarray*}
for every $\L$-structure $\A \in \str$. A function class $\F$ is {\em closed under substraction} if for every pair of functions $f,g \in \F$, it holds that $f \dotminus g \in \F$.

\begin{theorem} \label{sub-pnp}
	If $\eqso(\loge{1})$ is closed under substraction, then {\sc P} = {\sc NP}.
\end{theorem}

\begin{theorem} \label{sigma1strict}
	$\eqso(\loge{1}) \subsetneq \eqso(\logex{1})$
\end{theorem}

For a given function $f$, we define $f \dotminus 1$ as follows:
\begin{eqnarray*}
	f \dotminus 1(\A) =
	\begin{cases}
		f(\A)-1, & \text{if }f(\A) > 0 \\
		0, & \text{if }f(\A) = 0.
	\end{cases}
\end{eqnarray*}
for every $\L$-structure $\A \in \str$. A function class $\F$ is {\em closed under substraction by one} if for every function $f \in \F$, it holds that $f \dotminus 1 \in \F$.

\begin{theorem} \label{sigmafo-minusone}
	$\eqso(\logex{1})$ is closed under substraction by one.
\end{theorem}

\begin{theorem} \label{dnf-pars}
	{\sc \#DNF} is hard for $\eqso(\loge{1})$ under parsimonious reductions. 
\end{theorem}

\begin{theorem} \label{nplusone-strict}
	$\U{1}$ with $n$ open first-order variables is properly contained in $\U{1}$ with $n+1$ open first-order variables for $n\in\N$.  
\end{theorem}

\subsection{Counting hierarchy below $\shp$}


\subsection{Horn Counting Classes}
%!TEX root = main.tex
\newcommand{\pP}{\textit{P}}
\newcommand{\pN}{\textit{N}}
\newcommand{\pV}{\textit{V}}
\newcommand{\pT}{\textit{T}}
\newcommand{\pA}{\textit{A}}
\newcommand{\pNC}{\textit{NC}}
\newcommand{\pD}{\textit{D}}


A positive literal is a formula of the form $X(\x)$, where $X$ is a second-order variable and $\x$ is a tuple of first-order variables, and a negative literal is a formula of the form $\exists \v \, \neg X(\u,\v)$, where $\u$ and $\v$ are tuples of first-order variables. Given a relational signature $\R$, a clause over $\R$ is a formula of the form:
$$
\forall \x \, (\varphi_1 \vee \cdots \vee \varphi_n),
$$
where each $\varphi_i$ ($1 \leq i \leq n$) is either a positive literal, a negative literal or an \fo-formula over $\R$.  A clause is said to be Horn if it contains at most one positive literal, and a formula is said to be Horn if it is a conjunction of Horn clauses over a relational signature $\R$. With this terminology, we define $\uhorn$ as the set of formulas $\psi$ such that $\psi$ is a conjunction of Horn clauses over a relational signature $\R$. 

\begin{proposition}\label{prop:uhorn-pe}
$\eqso(\uhorn) \subseteq \totp$
\end{proposition}

\begin{example} \label{ex-hornsat-esop1}
Let $\R = \{\pP(\cdot,\cdot), \pN(\cdot,\cdot), \pV(\cdot), \pNC(\cdot)\}$. This vocabulary is used as follows to encode a Horn formula. A fact $\pP(c,x)$ indicates that propositional variable $x$ is a disjunct in a clause $c$, while $\pN(c,x)$ indicates that $\neg x$ is a disjunct in $c$. Furthermore, $\pV(x)$ holds if  $x$ is a propositional variable, and $\pNC(c)$ holds if $c$ is a clause containing only negative literals, that is, $c$ is of the form $(\neg x_1 \vee \cdots \vee \neg x_n)$.

To encode $\chsat$, we define an \so-formula $\varphi(\pT)$ over $\R$, where $\pT$ is a unary predicate, such that for every Horn formula $\theta$ encoded by an $\R$-structure $\A$, the number of satisfying assignments of $\theta$ is equal to $\sem{\sa{\pT} \varphi(\pT)}(\A)$. In particular, $\pT(x)$ holds if and only if $x$ is a propositional variable that is assigned value 1.  More specifically, $\varphi(\pT)$ is defined as follows:
\begin{align*}
&\forall x \, (\pT(x) \to \pV(x)) \ \wedge\\
&\forall c \, (\pNC(c) \to \exists x \, (\pN(c,x) \wedge \neg \pT(x))) \ \wedge\\
&\forall c \forall x \, ([\pP(c,x) \wedge \forall y \, (\pN(c,y) \to \pT(y))] \to \pT(x)).
\end{align*}
Given that $\uhorn$ is designed with the goal in mind of capturing $\chsat$, we expect $\varphi(\pT)$ to be a formula in $\uhorn$. However, if we rewrite it as a conjunction of clauses we obtain the following:
\begin{align*}
&\forall x \, (\neg \pT(x) \vee \pV(x)) \ \wedge\\
&\forall c \, (\neg \pNC(c) \vee \exists x \, (\pN(c,x) \wedge \neg \pT(x)))\ \wedge\\
&\forall c \forall x \, (\neg \pP(c,x) \vee \exists y \, (\pN(c,y) \wedge \neg \pT(y)) \vee \pT(x)).
\end{align*}
The resulting formula $\varphi(\pT)$ is not in $\uhorn$, but it can be easily transformed into a formula in this class  by introducing an auxiliary binary predicate $\pA$ defined as follows:
\begin{align*}
\forall c \forall x \, (\neg \pA(c,x) \leftrightarrow [\pN(c,x) \wedge \neg \pT(x)]).
\end{align*}
In this way, we obtain the following formula $\psi(\pT,\pA)$ in $\uhorn$:
\begin{align*}
&\forall x \, (\neg \pT(x) \vee \pV(x)) \ \wedge\\
&\forall c \, (\neg \textit{NC}(c) \vee \exists x \, \neg \textit{A}(c,x)) \ \wedge\\
&\forall c \forall x \, (\neg \textit{P}(c,x) \vee \exists y \, \neg \textit{A}(c,y) \vee \textit{T}(x)) \ \wedge\\
&\forall c \forall x \, (\neg \textit{N}(c,x) \vee \textit{T}(x) \vee \neg \textit{A}(c,x)) \ \wedge\\
&\forall c \forall x \, (\textit{A}(c,x) \vee \textit{N}(c,x)) \ \wedge\\
&\forall c \forall x \, (\textit{A}(c,x) \vee \neg\textit{T}(x)).
\end{align*}
This formula effectively defines $\chsat$
as for every Horn formula $\theta$ encoded by an $\R$-structure $\A$, the number of satisfying assignments of $\theta$ is equal to $\sem{\sa{\pT} \sa{\pA} \psi(\pT,\pA)}(\A)$.  Therefore, we conclude that $\chsat \in \eqso(\uhorn)$. 
\end{example}
We extend the definition of $\uhorn$ to allow existential quantification. More precisely, a formula $\varphi$ is in $\ehorn$ if $\varphi$ is of the form $\exists \bar x \, \psi$ with $\psi$ a Horn formula. Interestingly, it hold that $\cdnf \in \eqso(\ehorn)$ and

\begin{proposition}\label{prop:ehorn-pe}
$\eqso(\ehorn) \subseteq \totp$.
\end{proposition}
A natural question at this point is whether in the definitions of $\uhorn$ and $\ehorn$, it is necessary to allow negative literals of the form $\exists \v \, \neg X(\u,\v)$. The following result shows that it is indeed the case:

\begin{proposition}\label{prop:hsat-not-sigma2}	
$\chsat \not\in \eqso(\loge{2})$.
\end{proposition}
We conclude this section by showing that a natural extension of $\chsat$ is $\eqso(\ehorn)$-complete under parsimonious reductions. We define the decision problem:
\begin{multline*}
\dhsat = \{\Phi \mid \Phi \text{ is a disjunction of}\\  \text{Horn formulas and $\Phi$ is satisfiable}\},
\end{multline*}
and the counting problem $\shdhsat$ as a function that counts all satisfying assignments of a formula $\Phi$ that is a disjunction of Horn formulas.

\begin{theorem} \label{sigma2hard}
	$\shdhsat$ is $\eqso(\ehorn)$-complete under parsimonious reductions. 
\end{theorem}


\section{Adding recursion to QSO}\label{sec:beyond}
%!TEX root = main.tex

\subsection{Transitive Logics}

It was shown in \cite{I86,I88} that first-order logic extended with a transitive closure operator captures $\nlog$. Inspired by this work, we extend the definition of $\qfo$ with an operator for counting the number of paths in a directed graph, and we show that it can be used to capture $\shl$. Besides, we show that the same idea can be used to extend $\qso$ allowing to capture harder complexity classes. 

Given a relation signature $\R$, the set of transitive $\qso$ formulas ($\tqso$-formulas) is defined by the following grammar:
\begin{multline}
	\label{eq-def-tqso}
	\alpha := \varphi \, \mid \, s \, \mid \, (\alpha \add \alpha) \, \mid\, (\alpha \mult \alpha) \, \mid \, 
	\sa{x} \alpha \, \mid\, \
	\pa{x} \alpha \, \mid \\ 
	\sa{X} \alpha \, \mid \, \pa{X} \alpha \, \mid \, [\pth \psi(\bar{x}, \bar{X},\bar{y}, \bar{Y})],
\end{multline}
where $\varphi$ is an $\so$-formula over $\R$, $\psi(\bar{x},\bar{X},\bar{y},\bar{Y})$ is an $\fo$-formula over $\R$, $\bar{x} = (x_1, \ldots, x_k)$, $\bar{y} = (y_1, \ldots, y_k)$ are tuples of pairwise distinct first-order variables, and $\bar{X} = (X_1, \ldots, X_\ell)$, $\bar{Y} = (Y_1, \ldots, Y_\ell)$ are tuples of pairwise distinct second-order variables such that $\arity(X_i) = \arity(Y_i) = m_i$ for every $i \in \{1, \ldots, \ell\}$. The semantics of $[\pth \psi(\bar{x},\bar{X},\bar{y}.\bar{Y})]$ is defined as follows. Given an $\R$-structure $\A$, define a (directed) graph $\cG_{\psi}(\A) = (N,E)$ such that $N = A^k \times \prod_{i=1}^\ell 2^{A^{m_i}}$, where $A$ is the domain of $\A$, and for every pair $(\bar b, \bar B)$, $(\bar c, \bar C)$ of elements of $N$, it holds that $((\bar b, \bar B), (\bar c, \bar C)) \in E$ if and only if $\A \models \psi(\bar b, \bar{B}, \bar c, \bar{C})$. Then given first-order and second-order assignments $v$, $V$ for $\A$, we have that $\sem{[\pth \psi(\bar{x},\bar{X}, \bar{y}, \bar{Y})]}(\A,v,V)$ is the number of paths in $\cG_\psi(\A)$ from $(v(\bar x), V(\bar X))$ to $(v(\bar y), V(\bar Y))$ whose length is at most $n = |N|$.

As for the case of $\qso$, the logic $\tqso(\LL)$ is obtained by restricting $\varphi$ in \eqref{eq-def-tqso} to be a formula in $\LL$. Moreover, the logic $\tqfo$ is obtained by disallowing in \eqref{eq-def-tqso} formulas $\sa{X} \alpha$ and $\pa{X} \alpha$, and by only allowing  first-order free-variables in the formula $\psi$ used in $[\pth \psi]$ in \eqref{eq-def-tqso}. With this notation, we have the following results:


\begin{theorem} \label{tqfo-shl}
	$\tqfo(\fo)$ captures $\shl$ over the class of ordered structures.
\end{theorem}

\begin{theorem} \label{tqso-fo-fpsace}
	$\tqso$ and $\tqso(\fo)$ captures $\fpspace$ over the class of ordered structures.
\end{theorem}

\marcelo{Puede que este equivocado, pero me parece que teniamos una forma de capturar $\shp$ usando el operator ${\bf path}$. Pero no logro recordar como se hacia esto, y me parece que lo que habiamos escrito antes en esta seccion estaba equivocado: ``$\tqso(\fo)$ captures $\shp$ over the class of ordered structures". Claro que yo puedo estar usando una definicion distinta de $\tqso(\fo)$.}


%We also define the set of transitive $\qso$ formulas ($\tqso$-formulas) using the following grammar:
%\begin{multline*}
%%	\label{eq-def-tqso}
%	\alpha := \varphi \ \mid \ s \ \mid \ (\alpha \add \alpha) \ \mid\ (\alpha \mult \alpha) \ \mid \\ \sa{x} \alpha \ \mid \ \pa{x} \alpha \ \mid \ \sa{X} \alpha \ \mid \ \pa{X} \alpha \ \mid \ [\pth \varphi]
%\end{multline*}
%
%
% 
%We define the operator {\bf path} as follows. Let $\A$ be an ordered structure. Given a formula $\psi(\bar{x},\bar{y})$, where $\vert \bar{x} \vert = \vert \bar{y} \vert = k$ let ${\cal G} = ({\cal V},\cal{E})$ be induced graph over the set of vertices ${\cal V} = A^k$, and for every $\bar{a},\bar{b}\in A^k$ it holds that ${\cal E}(\bar{a},\bar{b})$ if and only if $\A \models \psi(\bar{a},\bar{b})$. To formalize the semantics for this operator, let $n = \vert A^k \vert$.
%For a given first order assignment $v$ and a second order asssignment $V$, let $\bar{a} = v(\bar{x})$ and $\bar{b} = v(\bar{y})$, and $\sem{[\pth\, \psi(\bar{x},\bar{y})]}(\A,v,V)$ will take the value of the number of paths of size less or equal to $n$ from $\bar{a}$ to $\bar{b}$ in the graph ${\cal G}$. This operator lets us define the set of transitive $\qfo$ formulas over $\R$ ($\tqfo$-formulas) using the following grammar:
%\begin{multline*} 
%%	\label{eq-def-tqfo}
%	\alpha := \varphi \ \mid \ s \ \mid \ (\alpha \add \alpha) \ \mid\ (\alpha \mult \alpha) \\ \mid \ \sa{x} \alpha \ \mid \ \pa{x} \alpha \ \mid \ [\pth \varphi]
%\end{multline*}
%where $\varphi$ is an $\fo$-formula over $\R$, $s \in \bbN$ and $x \in \fv$.
%
%We also define the set of transitive $\qso$ formulas ($\tqso$-formulas) using the following grammar:
%\begin{multline*}
%%	\label{eq-def-tqso}
%	\alpha := \varphi \ \mid \ s \ \mid \ (\alpha \add \alpha) \ \mid\ (\alpha \mult \alpha) \ \mid \\ \sa{x} \alpha \ \mid \ \pa{x} \alpha \ \mid \ \sa{X} \alpha \ \mid \ \pa{X} \alpha \ \mid \ [\pth \varphi]
%\end{multline*}
%where $\varphi$ is an $\so$-formula over $\R$, $s \in \bbN$, $x \in \fv$ and $X \in \sv$.
%\begin{theorem} \label{so-rec}
%	Given a positive $\fo$ formula $\varphi(\bar{x},R)$ and a $\qfo$ formula $\alpha(\bar{x})$, there exists a $\qso$ formula $\beta(\bar{x})$ such that $\sem{[\alfp\varphi(\bar{x},R)\mid \alpha(\bar{x},R)](\bar{x})} = \sem{\beta(\bar{x})}$.
%\end{theorem}
%
%\begin{theorem} \label{tqfo-fo-cap}
%	$\tqfo(\fo)$ captures $\shl$ over the class of ordered structures.
%\end{theorem}
%
%\begin{theorem} \label{tqso-fo-cap}
%	$\tqso(\fo)$ captures $\shp$ over the class of ordered structures.
%\end{theorem}


\subsection{Recursive Logics}

We define an operator which extends least fixed point logic \cite{I86,vardi1982complexity} to allow counting. 
Fix a relational signature $\R$. Then the set of $\rqfo$ formulas over $\R$ is defined by the following grammar:
\begin{multline*}
	\alpha := \varphi \, \mid \, s \, \mid \, (\alpha \add \alpha) \, \mid\, (\alpha \mult \alpha) \, \mid \, 
	\sa{x} \alpha \, \mid\\ 
	\pa{x} \alpha \, \mid\,
	\clfp{\psi(x_1,\ldots,x_k,R)}{\beta(y_1, \ldots, y_\ell, R, \pi)},
\end{multline*}
where $R$ is second-order variable of arity $k$, $\psi(x_1, \ldots, x_k, R)$ is an $\fo$-formula over $(\R \cup \{R\})$ that is positive on $R$ and $\beta(y_1, \ldots, y_\ell, R, \pi)$ is a $\qfo$-formula over $\R$ including a fresh symbol $\pi$ that represents a function of arity $\ell$, that is, every atomic formula in $\beta(y_1, \ldots, y_\ell, R, \pi)$ is either an $\fo$-formula over $\R$, a constant $s \in \N$ or a term of the form $\pi(u_1, \ldots, u_\ell)$ with $u_1, \ldots, u_\ell$ non-necessarily distinct first-order variables. 
The free variables of the formula $\clfp{\psi(x_1,\ldots,x_k,R)}{\beta(y_1, \ldots, y_\ell, R, \pi)}$
are $y_1, \ldots, y_\ell$, in particular it does not have any second-order free variable.

We need to introduce some terminology to define the semantics of the formula $\clfp{\psi(x_1,\ldots,x_k,R)}{\beta(y_1, \ldots, y_\ell, R, \pi)}$
Given a $\qfo$-formula $\gamma$ with $\ell$ first-order free variables and no second-order free variable, define $\beta[\gamma/\pi]$ to be the $\qfo$-formula obtained from $\beta$ by replacing every occurrence of a term $\pi(u_1, \ldots, u_\ell)$ by $\gamma(u_1, \ldots, u_\ell)$.
Let $\A$ be an $\R$-structure with domain $A$. Then for every $a \in A$, let $\varphi_a(x)$ be an $\fo$-formula such that $\A \models \varphi_a(b)$ if and only if $b = a$. Formally, assuming that $a$ is the $p$-th element of $A$ according to the linear order $<^\A$, we have that:
%then $\varphi_a(x) = \forall y(x < y \vee x = y)$
%for every natural number $i \geq 1$, let $\varphi_i(x)$ be an $\fo$-formula such that $\varphi_i(a)$ holds in an $\R$-structure $\A$ if and only if $a$ is the $i$-th element in the linear order  $<^{\A}$,
%that is, $\varphi_1(x) = \forall y(x < y \vee x = y)$ and for every $p > 1$:
\begin{multline*}
\varphi_a(x) \ = \ \exists x_1 \cdots \exists x_{p-1}\bigg[\bigwedge_{i =1}^{p-2}x_i < x_{i+1} \wedge\,\\  
x_{p-1} < x  \wedge \forall y(y < x \to \bigvee_{i = 1}^{p-1} y = x_i)\bigg].
\end{multline*}
Finally, given a function $f : A^\ell \to \N$, define $\qfo$-formula $\delta_{f,\A}(z_1, \ldots, z_\ell)$ as follows:
\begin{multline*}
 \mathop{+}_{(a_1,\ldots,a_{\ell})\in A^{\ell}} \ \varphi_{a_1}(z_1) \cdot \ldots \cdot \varphi_{a_{\ell}}(z_{\ell})\cdot f(a_1,\ldots,a_{\ell}), 
\end{multline*}
where $z_1, \ldots, z_\ell$ is a sequence of pairwise distinct first-order variables.
We have that $\delta_{f,\A}(z_1, \ldots, z_\ell)$ encodes $f$ in the structure $\A$ since:
\begin{eqnarray*}
\sem{\delta_{f,\A}(z_1, \ldots, z_\ell)}(\A,v)  & = & f(v(z_1), \ldots, v(z_\ell)).
\end{eqnarray*}
%Recall that the least fixed point operator is defined by a formula $\psi(x_1,\ldots,x_k,R)$ that is positive on $R$, where $R$  is a predicate of arity $k$. For a structure $\A$ with domain $A$, the operator $T_{\varphi}:2^{A^k} \to 2^{A^k}$ is defined as $T_{\varphi}(X) = \{(a_1,\ldots,a_k)\mid (\A,X)\models \psi(a_1,\ldots,a_k,R) \}$, for each $X\subseteq A^k$. Let $T_0 = \emptyset$ and $T_{i+1} = T_{\varphi}(T_i)$ for each $i \in \nat$. Note that there exists $n\in \nat$ such that $T_{n+1} = T_n$ because $R$ is positive in $\psi(x_1,\ldots,x_k,R)$. Then the evaluation of $[\lfpop\psi(x_1,\ldots,x_k,R)]$ is defined by the fixed point $T_n$, that is, for every $(a_1,\ldots,a_k)\in A^k$, it holds that $\A\models[\lfpop\psi(x_1,\ldots,x_k,R)](a_1,\ldots,a_k)$ if and only if $(a_1,\ldots,a_k) \in T_n$.
We are now ready to define the semantics of the formula $\clfp{\psi(x_1,\ldots,x_k,R)}{\beta(y_1,\ldots,y_{\ell},R,\pi)}$,  whose free variables are $y_1$, $\ldots$, $y_\ell$. Assume that $\A$ is an $\R$-structure with domain $A$. Following the definition of the semantics of least fixed point logic \cite{I86,vardi1982complexity}, define an operator $T_{\varphi}:2^{A^k} \to 2^{A^k}$ such that:
$$
T_{\varphi}(X)  =  \{(a_1,\ldots,a_k) \in A^k \mid \A \models \psi(a_1,\ldots,a_k,X) \},
$$
for every $X \subseteq A^k$. This operator is used to defined 
a sequence $T_0 \subsetneq T_1 \subsetneq \cdots \subsetneq T_n \subseteq A^k$  such that $n \geq 0$, $T_0 = \emptyset$, $T_{i+1} = T_{\varphi}(T_i)$ for every $i \in \{0, \ldots, n-1\}$, and $T_n = T_\varphi(T_n)$. We know that such a sequence exists as formula $\psi(x_1,\ldots,x_k,R)$ is positive on $R$. Now, the sequence $T_0 \subsetneq T_1 \subsetneq \cdots \subsetneq T_n$ is used to defined  a sequence of functions $f_0,f_1,\ldots,f_n:A^{\ell}\to\nat$. Formally, we have that $f_0(a_1, \ldots, a_\ell) = 0$ for every $(a_1, \ldots, a_\ell) \in A^\ell$. Moreover, assuming that $(a_1, \ldots, a_\ell) \in A^\ell$, $v$ is a first-order assignment for $\A$ such that $v(y_i) = a_i$ for every $i \in \{1, \ldots, \ell\}$, and that $V$ is a second-order assignment for $\A$ such that $V(R) = T_{i+1}$, we have that:
\begin{eqnarray*}
f_{i+1}(a_1, \ldots, a_\ell) & = & \sem{\beta[\delta_{f_i,\A}/\pi]}(\A,v,V).
\end{eqnarray*}
%, for each $T_i \in \T$, let $V$ be a second-order assignment for $\A$ that assigns $T_i$ to $R$, let $\zeta_i: A^{\ell}\to\nat$ be such that for each $(a_1,\ldots,a_{\ell})\in A^{\ell}$ it holds $\zeta_i(a_1,\ldots,a_{\ell}) = \sem{\alpha\mid_{\pi(u_1,\ldots,u_{\ell})\to \beta_i(u_1,\ldots,u_{\ell})}}(\A,v,V)$, where $v$ is a first-order assignment for $\A$ that satisfies $a_i = v(x_i)$ for each free $x_i$ in $\alpha$.
Finally, the semantics of the formula $\clfp{\psi(x_1,\ldots,x_k,R)}{\beta(y_1,\ldots,y_{\ell},R,\pi)}$ is defined by means of function $f_n$:
%For a given first order assignment $v$ and a second order assignment $V$, let $a_i = v(x_i)$, the operator is then evaluated as:
\begin{align*}
&\llbracket [{\bf lfp} \, \psi(x_1,\ldots,x_k,R) \,|\\
&\hspace{10mm}\beta(y_1,\ldots,y_{\ell},R,\pi)]\rrbracket(\A,v) \ = \
f_n(v(y_1),\ldots,v(y_{\ell})).
\end{align*}

\begin{example}
%As an example, 
We would like to define a formula that, given a graph $G$ with $n$ nodes and a pair of nodes $b$, $c$ in $G$, counts the number of paths of length at most $n$ from $b$ to $c$ in $G$.
To this end, assume that $\R = \{ E(\cdot,\cdot) \}$, and define $\psi(x,R)$ as follows:
\begin{eqnarray*}
%\psi(x,R) = 
\forall y(x < y \vee x = y) \vee \exists z(R(z) \wedge \varphi_{succ}(z,x)),
\end{eqnarray*}
where $\varphi_{succ}(x,y)$ is a formula that is satisfied by pairs $(x,y)$ that are consecutive in the order $<$. That is, $\varphi_{succ}(x,y) = x < y \wedge \neg \exists z \, (x < z \wedge z < y)$. Moreover, define formula $\beta(x,y,R,\pi)$ as follows:
$$
%\alpha(x,y,R,\pi) = 
%(\neg \exists zR(z))\cdot(x = y) 
E(x,y) + \sa{z} [\pi(x,z)\cdot E(z,y)].
$$
Then we have that $\clfp{\psi(x,R)}{\beta(x,y,R,\pi)}$ defines our counting function. In fact, assume that $\A$ is an $\R$-structure with $n$ elements in its domain, $b,c$ are elements of $\A$ and $v$ is a first-order assignment over $\A$ such that $v(x) = b$ and $v(y) = c$. Then we have that $\sem{\clfp{\psi(x,R)}{\beta(x,y,R,\pi)}}(\A,v)$ is equal to the  number of paths in $\A$ from $b$ to $c$ of length at most $n$.
\end{example}

\marcelo{Deberiamos mostrar como el ejemplo anterior se generaliza para definir $[\pth \psi(\bar x, \bar y)]$ en terminos del operador ${\bf lfp}$.}

It is well known that least fixed point logic is contained in second-order logic \cite{L04}. In the following theorem we show that the same holds in our case.
\begin{theorem} \label{so-rec}
$\rqfo \subseteq \qso$
%	Given a positive $\fo$ formula $\varphi(\bar{x},R)$ and a $\qfo$ formula $\alpha(\bar{x})$, there exists a $\qso$ formula $\beta(\bar{x})$ such that $\sem{[\alfp\varphi(\bar{x},R)\mid \alpha(\bar{x},R)](\bar{x})} = \sem{\beta(\bar{x})}$.
\end{theorem}

Finally, the following is the main result of this section:
\begin{theorem} \label{rqfo-fo-cap}
	$\rqfo(\fo)$ captures $\fp$ over the class of ordered structures.
\end{theorem}

\marcelo{Notese que estoy permitiendo en la formula $\beta(y_1, \ldots, y_\ell,R,\pi)$ tener subformulas $\pa{x} \alpha$. Esta bien el resultado con esto? O tenemos que eliminar estas formulas?}

%Given a relation signature $\R$, the set of recursive $\qfo$ formulas ($\rqfo$-formulas) is defined by the following grammar:
%%This operator lets us define the set of recursive $\qfo$ formulas over $\R$ ($\rqfo$-formulas) using the following grammar:
%\begin{multline*}
%%	\label{eq-def-rqfo}
%	\alpha := \varphi \ \mid \ s \ \mid \ (\alpha \add \alpha) \ \mid \\ (\alpha \mult \alpha) \ \mid \ \sa{x} \alpha \ \mid \ \pa{x} \alpha \ \mid \ [\alfp \varphi \mid \alpha]
%\end{multline*}
%where $\varphi$ is an $\fo$-formula over $\R$, $s \in \bbN$ and $x \in \fv$.
%
%\marcelo{Vamos a permitir anidacion del operador $\alfp$? Esta gramatica lo permite.}
%
%\begin{theorem} \label{rqfo-fo-cap}
%	$\rqfo(\fo)$ captures $\fp$ over the class of ordered structures.
%\end{theorem}
%


\section{Conclusion}\label{sec:conclusions}
%!TEX root = main.tex

Here comes the conclusions.


%\section*{Acknowledgments}
%The authors would like to thank ~\cite{DrosteG07}



\bibliographystyle{IEEEtranS}
\bibliography{biblio}

\newpage

\onecolumn
\appendix

\subsection{Notation for the appendix}
For a given signature $\R$, we define $\ostr[\R]^*$ as $$\ostr[\R]^* = \{(\A,v,V) \mid \A\in\ostr[\R]\text{, $v$ ($V$) is a first-order (second-order) assignment for $\A$}  \}.$$
The {\em conditional count} symbol $(\varphi \mapsto \alpha)$ is defined as $(\neg\varphi + (\varphi\cdot\alpha))$ for given $\so$ formula $\varphi$ and $\qso$ formula $\alpha$. Note that for each $(\A,v,V) \in \ostr[\R]^*$, 
$$
\sem{(\varphi \mapsto \alpha)}(\A,v,V) = 
\begin{cases}
\sem{\alpha}(\A,v,V) &\text{if } (\A,v,V)\models\varphi,\\
0 &\text{otherwise}.
\end{cases}
$$
We will use the symbol $<$ also to denote the lexicographic order over same-sized tuples. If $\bar{x} = (x_1,\ldots,x_m)$ and $\bar{y} = (y_1,\ldots,y_m)$ are tuples of first-order variables, we denote $\bar{x} < \bar{y}$ for the formula $\bigvee_{i = 1}^m[\bigwedge_{j = 1}^{i-1}x_j = y_j \wedge x_i < y_i]$. Similarly, we use $=$ to denote equality between tuples, as $\bar{x} = \bar{y}$ denotes $\bigwedge_{i = 1}^m(x_i = y_i)$, and also $\bar{x}\leq\bar{y}$ denotes $\bar{x} < \bar{y} \vee \bar{x} = \bar{y}$. We also denote $\min(\bar{x}) := \forall\bar{y}(\bar{x} \leq \bar{y})$.

If $\bar{x} = (x_1,\ldots,x_m)$ ($\bar{X} = (X_1,\ldots,X_m)$) is a tuple of first-order (second-order) variables, we denote $\sa{\bar{x}}\alpha$ for $\sa{x_1}\cdots\sa{x_m}\alpha$ and $\sa{\bar{X}}\alpha$ for $\sa{X_1}\cdots\sa{X_m}\alpha$ for each $\qso$ formula $\alpha$. We also denote $\length{\bar{x}}$ as the size of $\bar{x}$ ($\length{\bar{X}}$ as the size of $\bar{X}$). In this case, $\length{\bar{x}} = m$ ($\length{\bar{X}} = m$).

\bigskip

\subsection{Proofs from Section~\ref{sec:complexity}}

\medskip

%%% DEMOSTRACION DE SQSO(FO) y #P
\subsection*{Proof of Proposition~\ref{prop:capture-shP}}

We will construct a recursive non-deterministic algorithm $M_{\alpha}$ for each $\eqso(\fo)$ formula $\alpha$ over a signature $\R$. This machine, on input $(\A,v,V)$ accepts in $\sem{\alpha}(\A,v,V)$ of its non-deterministic paths for each $(\A,v,V) \in \ostr[\R]^*$. Suppose $\A$ has domain $A$. If $\alpha$ is a $\fo$-formula $\varphi$, then the algorithm checks if $(\A,v,V)\models\varphi$ deterministically in polynomial time, and accepts if and only if it holds true. If $\alpha$ is a constant $s$, it produces $s$ branches and accepts in all of them. If $\alpha = (\beta \add \gamma)$, then it chooses between 0 or 1, if it is 0 (1), it simulates $M_{\beta}$ ($M_{\gamma}$) on input $(\A,v,V)$. 
%If $\alpha = (\beta \mult \gamma)$, it simulates $M_{\beta}$ on input $(\A,v,V)$ and on each accepting path, it continues simulating $M_{\gamma}$ on input $(\A,v,V)$.
If $\alpha = \sa{x}\beta$, it chooses $a\in A$ and simulates $M_{\beta}$ on input $(\A,v[a/x],V)$.
%If $\alpha = \pa{x}\beta$, it simulates $M_{\beta}$ on input $(\A,v[a/x],V)$ consecutively for each $a\in A$. 
If $\alpha = \sa{X}\beta$, it chooses $B\in A^{arity(X)}$ and simulates $M_{\beta}$ on input $(\A,v,V[B/X])$. This covers all possible cases for $\alpha$. Let $\alpha$ be a formula in $\eqso(\fo)$ over a signature $\R$ and let $f$ be a function over $\R$ such that $f(\enc(\A))$ is equal to the accepting paths of $M_{\alpha}$ on input $(\A,v,V)$ for some $(\A,v,V) \in \ostr[\R]^*$. We have that $f$ is a $\shp$-function over $\R$ and $f(\enc(\A)) = \sem{\alpha}(\A)$ for every $\A\in\ostr[\R]$.

For the other direction, note that Saluja et al.~\cite{SalujaST95} proved that $\shp = \sfo$. 
%We also have that $\sqso(\fo)$ captures $\#\fo$ over ordered structures so for each $f\in \shp$ let $\alpha \in \sqso(\fo)$ be its corresponding formula. 
Since a function in $\sfo$ can also be defined $\eqso(\fo)$ (see Section~\ref{sec:previous}), the condition holds. \qed

\medskip

%%% DEMOSTRACION DE SQSO(ESO) y span-P
\subsection*{Proof of Proposition~\ref{prop:capture-spanP}}

Similar than the previous proof, we will construct a recursive non-deterministic algorithm $M_{\alpha}$ for each $\eqso(\eso)$ formula $\alpha$ over a signature $\R$. This machine, on input $(\A,v,V)$, non-deterministically produces $\sem{\alpha}(\A,v,V)$ distinct accepting outputs for each $(\A,v,V) \in \ostr[\R]^*$. Suppose $\A$ has domain $A$. 
If $\alpha$ is a $\eso$-formula $\varphi$ it checks if $(\A,v,V)\models\varphi$ non-deterministically in polynomial time \cite{fagin1974generalized}, and accepts if and only if the condition holds true. 
If $\alpha$ is a constant $s$, then the algorithm produces $s$ branches and accepts in all of them. 
If $\alpha = (\beta \add \gamma)$, then it chooses between 0 or 1, if it is 0 (1), it simulates $M_{\beta}$ ($M_{\gamma}$) on input $(\A,v,V)$.  
If $\alpha = \sa{x}\beta$, it chooses $a\in A$ and simulates $M_{\beta}$ on input $(\A,v[a/x],V)$. 
If $\alpha = \sa{X} \beta$, it chooses $B\in A^{\arity(X)}$ and simulates $M_{\beta}$ on input $(\A,v,V[B/X])$. 
This covers all possible cases for $\alpha$.
Additionally, the algorithm produces a different output on each path. This can be done by printing the trace of all the non-deterministic choices.
However, when the algorithm starts checking whether $(\A,v,V)\models\varphi$ for some $\eso$ formula $\varphi$, it stops printing in the output tape. This way the algorithm produces exactly one output from that point onwards.
Let $\alpha$ be a formula in $\eqso(\eso)$ over a signature $\R$ and let $f$ be a function over $\R$ such that $f(\enc(\A))$ is equal to the number of accepting outputs of $M_{\alpha}$ on input $(\A,v,V)$ for some $(\A,v,V) \in \ostr[\R]^*$. 
We have that $f$ is a $\spp$ function over $\R$ and that $f(\enc(\A)) = \sem{\alpha}(\A)$ for every $\A\in\ostr[\R]$.

For the other direction, Compton et al.~\cite{ComptonG96} proved that $\spp = \#\eso$. Since a function in $\#\eso$ can also be defined in $\eqso(\eso)$, then $\eqso(\eso)$ captures $\spp$ over ordered structures.

\medskip

%%% DEMOSTRACION DE QFO(LFP) y FP
\subsection*{Proof of Theorem~\ref{theo:capture-fp}}

For the first condition, let $\alpha\in\qfo(\lfp)$ over some signature $\R$. Let $f$ be a function over $\R$ defined by the following procedure. Let $\enc(\A)$ be an input, where $\A$ is an ordered structure over $\R$ with domain $A = \{1,\ldots,n\}$. In the procedure we slightly extend the grammar of $\qfo(\lfp)$ to include constants. We replace each first order sum and first order product in $\alpha$ by an expansion using the elements in $A$. This is, $\sa{x} \beta(x)$ is replaced by $(\beta(1)+\cdots+\beta(n))$ and $\pa{x}\beta(x)$ is replaced by $(\beta(1)\cdot\,\cdots\,\cdot\beta(n))$. Then each sub-formula $\varphi\in\lfp$ in $\alpha$ is evaluated in polynomial time and replaced by 1 if $\A\models\varphi$ and by 0 otherwise. The resulting formula is an arithmetic expression of polynomial size which is evaluated and lastly given as output. Note that $f\in\fp$ and $f(\enc(\A)) = \sem{\alpha}(\A)$.
	
For the second condition, let $f\in \fp$ defined over some signature $\R$.
Let $\ell\in\nat$ be such that for each $\A\in\ostr[\R]$, $\lceil\log_2 f(\enc(\A)) \rceil \leq n^\ell$, where $\A$ has a domain of size $n$.
Let $\bar{x} = (x_1,\ldots,x_{\ell})$.
Consider a procedure that receives $\enc(\A)$ and an assignation $\bar{a}$ to $\bar{x}$. Let $m$ be the position of $\bar{a}$ in the lexicographic order of the tuples in $A^{\ell}$. The procedure then computes the $m$-th bit of $f(\enc(\A))$, from least to most significant. Since this procedure works in polynomial time, it can be described by an $\lfp$ formula $\Phi(\bar{x})$. Then we use
$$
\alpha = \sa{\bar{x}} \Phi(\bar{x})\cdot\varphi(\bar{x}),
$$
where $\varphi(\bar{x}) := \pa{\bar{y}}(\bar{y} < \bar{x} \mapsto 2).$ Note that if $\bar{a} \in A^{\ell}$ is the $m$-th tuple in the given order (starting from 0), then $\sem{\varphi(\bar{a})}(\A) = 2^{m}$. Adding these values for each $\bar{a}\in A^{\ell}$ gives exactly $f(\enc(\A))$. Then, $\alpha$ is in $\qfo(\lfp)$ over $\R$ and $\sem{\alpha}(\A) = f(\enc(\A))$.

\medskip

%%% DEMOSTRACION DE QSO(PFP) y FPSPACE
\subsection*{Proof of Theorem~\ref{theo:capture-fpspace}}

To show how to evaluate a $\qso(\pfp)$-formula, we will construct a recursive non-deterministic algorithm $M_{\alpha}$ for each $\qso(\pfp)$ formula $\alpha$ over a signature $\R$. This machine runs in non-deterministic polynomial space and, on input $(\A,v,V)$, accepts in $\sem{\alpha}(\A,v,V)$ of its non-deterministic paths for each $(\A,v,V) \in \ostr[\R]^*$. Suppose $\A$ has domain $A$. If $\alpha$ is a $\pfp$-formula $\varphi$, then the algorithm checks if $(\A,v,V)\models\varphi$ deterministically in polynomial space~\cite{L04}, and accepts if and only if it holds true. If $\alpha$ is a constant $s$, it produces $s$ branches and accepts in all of them. If $\alpha = (\beta \add \gamma)$, then it chooses between 0 or 1, if it is 0 (1), it simulates $M_{\beta}$ ($M_{\gamma}$) on input $(\A,v,V)$. If $\alpha = (\beta \mult \gamma)$, it simulates $M_{\beta}$ on input $(\A,v,V)$ and on each accepting path, it continues simulating $M_{\gamma}$ on input $(\A,v,V)$.
If $\alpha = \sa{x}\beta$, it chooses $a\in A$ and simulates $M_{\beta}$ on input $(\A,v[a/x],V)$. If $\alpha = \pa{x}\beta$, it simulates $M_{\beta}$ on input $(\A,v[a/x],V)$ consecutively for each $a\in A$. If $\alpha = \sa{X}\beta$, it chooses $B\in A^{arity(X)}$ and simulates $M_{\beta}$ on input $(\A,v,V[B/X])$. If $\alpha = \pa{X}\beta$, it simulates $M_{\beta}$ on input $(\A,v,V[B/X])$ consecutively for each $B\in A^{arity(X)}$. This covers all possible cases for $\alpha$, and each of these steps can be computed in polynomial space. Let $\alpha$ be a formula in $\qso(\pfp)$ over a signature $\R$ and let $f$ be a function over $\R$ such that $f(\enc(\A))$ is equal to the accepting paths of $M_{\alpha}$ on input $(\A,v,V)$ for some $(\A,v,V) \in \ostr[\R]^*$. We have that $f$ is a $\shpspace$ function over $\R$, which implies that $f$ is also a $\fpspace$ function over $\R$, by the fact that $\shpspace = \fpspace$ \cite{Ladner89}, and that $f(\enc(\A)) = \sem{\alpha}(\A)$ for every $\A\in\ostr[\R]$.

For the second condition, let $f\in \fpspace$ defined over some $\R$. Let $\ell\in\nat$ be such that for each $\A\in\ostr[\R]$, $\lceil\log_2 f(\enc(\A)) \rceil \leq 2^{n^\ell}$, where $\A$ has a domain of size $n$. Let $X$ be a second-order variable of arity $\ell$. Consider a linear order over predicates of arity $\ell$ given by the formula 
$$
\varphi_{<}(X,Y) = \exists\bar{u}\big[\neg X(\bar{u})\wedge Y(\bar{u})\wedge \forall\bar{v}\big(
\bar{u}<\bar{v}\to(X(\bar{u})\iff Y(\bar{v}))\big)\big].
$$
Consider a polynomial space algorithm over $\R\cup\{X\}$ that receives a $\R\cup\{X\}$-structure $\A'$. Let $X^{\A'}$ be the $p$-th subset in the given order (we consider 0-indexation). The algorithm computes $f(\enc(\A))$ until $(2^{n^{\ell}}-p)$ characters have been written, but each character of the output is written over the last one. It accepts if and only if the last character written is $1$. Since this algorithm works in polynomial space, it can be described in $\pfp$ \cite{AbiteboulV89} by a formula $\Phi(X)$.

Note that the formula also represents the following.
For each $(\A,v,V)\in\ostr[\R]^*$, let $V(X)$ be the $p$-th subset with respect to this order and let $w$ be a string of size $2^{n^{\ell}}$ which represents the value of $f(\enc(\A))$ in binary. Then $(\A,v,V)\models\Phi(X)$ if and only if the $p$-th bit in $w$ is 1, from least to most significant. We define
$$
\alpha := \sa{X}(\Phi(X)\mult\varphi(X)),
$$ 
where $\varphi(X) = \pa{Y}(\varphi_{<}(Y,X)\mapsto 2)$. Note that for each $B\subseteq A^{\ell}$ such that $B$ is the $m$-th element in the given order, $\sem{\varphi(B)}(\A) = 2^m$. Therefore, $\alpha\in\qso(\pfp)$ and $\sem{\alpha}(\A) = f(\enc(\A))$.

\medskip


%%% DEMOSTRACION DE QFO(PFP) y FPSPACE(poly)
\subsection*{Proof of Corollary~\ref{cor:capture-fpspace-poly}}

For the first condition, let $\alpha\in\qfo(\pfp)$ over some signature $\R$. Let $f$ be a function over $\R$ defined by the following procedure. Let $\enc(\A)$ be an input, where $\A$ is an ordered structure over $\R$ with domain $A = \{1,\ldots,n\}$. In the procedure we slightly extend the grammar of $\qfo(\pfp)$ to include constants. We replace each first order sum and first order product in $\alpha$ by an expansion using the elements in $A$. This is, $\sa{x} \beta(x)$ is replaced by $(\beta(1)+\cdots+\beta(n))$ and $\pa{x}\beta(x)$ is replaced by $(\beta(1)\cdot\,\cdots\,\cdot\beta(n))$. Then each sub-formula $\varphi\in\pfp$ in $\alpha$ is evaluated in polynomial space and replaced by 1 if $\A\models\varphi$ and by 0 otherwise. The resulting formula is an arithmetic expression of polynomial size which is evaluated and lastly given as output. Note that $f\in\nfpspace$ and $f(\enc(\A)) = \sem{\alpha}(\A)$.

For the second condition, let $f\in \fpspace$ defined over some signature $\R$.
Let $\ell\in\nat$ be such that for each $\A\in\ostr[\R]$, $\lceil\log_2 f(\enc(\A)) \rceil \leq n^\ell$, where $\A$ has a domain of size $n$.
Let $\bar{x} = (x_1,\ldots,x_{\ell})$.
Consider a procedure that receives $\enc(\A)$ and an assignation $\bar{a}$ to $\bar{x}$. Let $m$ be the position of $\bar{a}$ in the lexicographic order of the tuples in $A^{\ell}$. The procedure then computes the $m$-th bit of $f(\enc(\A))$, from least to most significant. Since this procedure works in polynomial space, it can be described by an $\pfp$ formula $\Phi(\bar{x})$. Then we use
$$
\alpha = \sa{\bar{x}} \Phi(\bar{x})\cdot\varphi(\bar{x}),
$$
where $\varphi(\bar{x}) := \pa{\bar{y}}(\bar{y} < \bar{x} \mapsto 2).$ Note that if $\bar{a} \in A^{\ell}$ is the $m$-th tuple in the given order (starting from 0), then $\sem{\varphi(\bar{a})}(\A) = 2^{m}$. Adding these values for each $\bar{a}\in A^{\ell}$ gives exactly $f(\enc(\A))$. Then, $\alpha$ is in $\qfo(\pfp)$ over $\R$ and $\sem{\alpha}(\A) = f(\enc(\A))$.


\subsection{Proofs from Section~\ref{sec:syntactic}}
\subsection*{Proof of Theorem \ref{theo-pnf-snf}}
Recall that a formula in $\eqso(\LL)$ is on the following grammar:
$$
\alpha = \varphi \ \mid \ s \ \mid \ (\alpha + \alpha) \ \mid \ \sa{x} \alpha \ \mid \ \sa{X} \alpha,
$$
where $\varphi$ is a formula in $\LL$ and $s\in\nat$. We will construct a recursive function $\tau$ such that for every $\eqso(\LL)$ formula $\alpha$, it outputs an equivalent formula $\tau(\alpha)$ which is in $\LL$-SNF. If $\alpha = \varphi$, then we define $\tau(\alpha) = \alpha$ which is clearly equivalent and already in $\LL$-SNF. If $\alpha = s$, then we define $(\top \add \cdots \add \top)$ ($s$ times), which also satisfies the condition. We assume that for every sub-formula $\beta$ of $\alpha$, $\tau(\beta)$ is an equivalent formula in $\LL$-SNF. If $\alpha = (\alpha_1 + \alpha_2)$, then we define $\tau(\alpha) = (\tau(\alpha_1) + \tau(\alpha_2))$. If $\alpha = \sa{x}\beta$, then let $\tau(\beta) = \sum_{i = 1}^{n}\beta_i$ where each $\beta_i$ is in $\LL$-PNF. We have that each $\beta_i$ also is in $\LL$-PNF, and then we define $\tau(\alpha) = \sum_{i = 1}^{m}\sa{x}\beta_i$ which satisfies the condition. If $\alpha = \sa{X}\beta$, then we proceed analogously as in the previous case. This covers all possible cases for $\alpha$ and we conclude the proof by taking $\tau(\alpha)$ as the desired rewrite of $\alpha$.












\subsection*{Proof of theorem \ref{theo-pi1-pnf}}
We divide the proof in three parts.

\vspace{1em}
First, we prove that the formula $\alpha_{0} = \sa{X(\cdot)}1 + 1$ is not equivalent to any formula in $\loge{0}$-PNF. Suppose that there exists some formula $\alpha = \sa{\bar{X}}\sa{\bar{x}}\varphi(\bar{X},\bar{x})$ in $\loge{0}$-PNF that is equivalent to the $\eqso(\loge{0})$ formula $\sa{X(\cdot)}1 + 1$. Note that this defines the function $2^{n}+1$, where $n$ is the size of the input structure. Then, note that if $\varphi(\bar{X},\bar{x})$ is not satisfiable, then the function is null. Note also that if $\length(\bar{X}) = \length(\bar{x}) = 0$, then the function defined by $\alpha$ is never greater than 1. Now note that $\sa{\bar{X}}\sa{\bar{x}}\varphi(\bar{X},\bar{x})$ is equivalent to a function in $\E{0}$. The authors in \cite{SalujaST95} proved that if $\length(\bar{X}) > 0$, then the function defined by the formula, for big enough structures, is always even. And since $\sa{\bar{x}}\varphi(\bar{x})$ defines a polynomially bounded function, we follow to a contradiction.

\vspace{1em}
Second, we prove that the formula $\alpha_{1} = 2$ is not equivalent to any formula in $\loge{1}$-PNF. Suppose that there exists some formula $\alpha = \sa{\bar{X}}\sa{\bar{x}}\exists\bar{y}\, \varphi(\bar{X},\bar{x},\bar{y})$ in $\loge{1}$-PNF that is equivalent to the $\eqso(\loge{1})$ formula $2$. Suppose that $\varphi$ is quantifier-free. Note first that if $\length(\bar{X}) = \length(\bar{x}) = 0$, then the function defined by $\alpha$ is never greater than 1. Consider an ordered structure $\A$. Since $\sem{\alpha}(\A) = 2$, there exist at least two assignations $(\bar{B}_1,\bar{b}_1,\bar{a}_1)$, $(\bar{B}_2,\bar{b}_2,\bar{a}_2)$ to $(\bar{X},\bar{x},\bar{y})$ such that for both, $\A\models\varphi(\bar{B}_i,\bar{b}_i,\bar{a}_i)$. Now consider the ordered structure $\A'$ that is obtained by duplicating $\A$. This is, each half of $\A'$ is isomorphic to $\A$. Note that $\A'\models\varphi(\bar{B}_i,\bar{b}_i,\bar{a}_i)$ for $i = 1,2$ and there exists a third assignation $(\bar{B}_1',\bar{b}_1',\bar{a}_1')$ that is identical to $(\bar{B}_1,\bar{b}_1,\bar{a}_1)$ but in the other half to the structure such that $\A'\models\varphi(\bar{B}_1',\bar{b}_1',\bar{a}_1')$. We have that $\sem{\alpha}(\A) \geq 3$, and we follow to a contradiction.

\vspace{1em}
We now show that if $\LL$ contains $\logu{1}$ and is closed under conjunction and disjunction, then for every formula $\alpha$ in $\eqso(\LL)$ there is an equivalent formula $\beta$ in $\LL$-PNF. As in Theorem \ref{theo-pnf-snf}, we show a recursive function $\tau$ that produces such formula. As we showed, there exists an equivalent formula in $\LL$-SNF, so we assume that $\alpha$ is in that form. Let $\alpha = \sum_{i = 1}^n \alpha_i$ where each $\alpha_i$ is in $\LL$-SNF. Let $\alpha_i = \sa{\bar{X}}\sa{\bar{x}}\varphi_i(\bar{X},\bar{x})$. We replace each $\alpha_i$ for the equivalent formula $\sa{\bar{X}}\sa{Y}\sa{\bar{x}}\sa{y}(\varphi_i(\bar{X},\bar{x})\wedge\forall z\,Y(z) \wedge \forall z(y \leq z))$ to obtain that $\length(\bar{X}) > 0$ and $\length(\bar{x}) > 0$. 

Now we begin describing the function $\tau$. If $\alpha = \sa{\bar{X}}\sa{\bar{x}}\varphi(\bar{X},\bar{x})$, then the formula is already in $\LL$-PNF so we define $\tau(\alpha) = \alpha$. If $\alpha = \alpha_1 + \alpha_2$, then we assume that $\tau(\alpha_1) = \sa{\bar{X}}\sa{\bar{x}}\varphi(\bar{X},\bar{x})$ and $\tau(\alpha_2) = \sa{\bar{X}}\sa{\bar{x}}\psi(\bar{X},\bar{x})$. The construction that we will provide for this function works by identifying a ``first'' assignment for both $(\bar{X},\bar{x})$ and $(\bar{Y},\bar{y})$ and a ``last'' assignment for both $(\bar{X},\bar{x})$ and $(\bar{Y},\bar{y})$. These are identified by the following formulas:
\begin{align*}
\gamma_{\text{first}}(\bar{X},\bar{x}) &= \bigwedge_{i = 1}^{\length(\bar{X})} \forall\bar{z}\neg X_i(\bar{z}) \wedge \forall\bar{z}(\bar{x}\leq\bar{z}), \\
\gamma_{\text{last}}(\bar{X},\bar{x}) &= \bigwedge_{i = 1}^{\length(\bar{X})} \forall\bar{z} X_i(\bar{z}) \wedge \forall\bar{z}(\bar{z}\leq\bar{x}).
\end{align*}
In other words, the ``first'' assignment is the one where every second-order predicate is empty and the first-order assignment is the lexicographically smallest, and the ``last'' assignment is the one where every second-order predicate is full and the first-order assignment is the lexicographically greatest. We also need to identify assignments that are not first and are not last. We do this by negating the two formulas above, and we use a concatenation of the quantified first-order variables:
\begin{align*}
\gamma_{\text{not first}}(\bar{X},\bar{x}) &= \exists\bar{z}(\bar{z}_0 < \bar{x} \vee \bigvee_{i = 1}^{\length(\bar{X})}X(\bar{z}_i)), \\
\gamma_{\text{not last}}(\bar{X},\bar{x}) &= \exists\bar{z}(\bar{x} < \bar{z}_0 \vee \bigvee_{i = 1}^{\length(\bar{X})}\neg X(\bar{z}_i)),
\end{align*}
where $\bar{z} = (\bar{z}_0,\bar{z}_1,\ldots,\bar{z}_{\length(\bar{X})})$. The following formula is equivalent to $\alpha$:
\begin{align}
\sa{\bar{X}}\sa{\bar{x}}\sa{\bar{Y}}\sa{\bar{y}}[&(\varphi(\bar{X},\bar{x})\wedge\gamma_{\text{not first}}(\bar{X},\bar{x})\wedge\gamma_{\text{first}}(\bar{Y},\bar{y}))\vee\\
&(\varphi(\bar{X},\bar{x})\wedge\gamma_{\text{first}}(\bar{X},\bar{x})\wedge\gamma_{\text{last}}(\bar{Y},\bar{y}))\vee\\
&(\psi(\bar{Y},\bar{y})\wedge\gamma_{\text{first}}(\bar{X},\bar{x})\wedge\gamma_{\text{not last}}(\bar{Y},\bar{y}))\vee\\
&(\psi(\bar{Y},\bar{y})\wedge\gamma_{\text{last}}(\bar{X},\bar{x})\wedge\gamma_{\text{last}}(\bar{Y},\bar{y}))].
\end{align}
To show that the formula is indeed equivalent to $\alpha$, note three facts. First, the formulas in lines (2) and (3) form a partition over the assignments of $(\bar{X},\bar{x})$, while fixing an assignment for $(\bar{Y},\bar{y})$. Second, the formulas in lines (4) and (5) form a partition over the assignments of $(\bar{Y},\bar{y})$, while fixing an assignment for $(\bar{X},\bar{x})$. And third, the four lines define pairwise disjoint assignments for $(\bar{X},\bar{x}),(\bar{Y},\bar{y})$.

However, the formula is not yet in the correct form. By expanding it we obtain:
\begin{align*}
\sa{\bar{X}}\sa{\bar{x}}\sa{\bar{Y}}\sa{\bar{y}}[&(\varphi(\bar{X},\bar{x})\wedge\exists\bar{v}(\bar{v}_0 < \bar{x} \vee \bigvee_{i = 1}^{\length(\bar{X})}X(\bar{v}_i))\wedge\bigwedge_{i = 1}^{\length(\bar{Y})} \forall\bar{z}\neg Y_i(\bar{z}) \wedge \forall\bar{z}(\bar{y}\leq\bar{z})\vee\\
&(\varphi(\bar{X},\bar{x})\wedge\bigwedge_{i = 1}^{\length(\bar{X})} \forall\bar{z}\neg X_i(\bar{z}) \wedge \forall\bar{z}(\bar{x}\leq\bar{z})\wedge\bigwedge_{i = 1}^{\length(\bar{Y})} \forall\bar{z} Y_i(\bar{z}) \wedge \forall\bar{z}(\bar{z}\leq\bar{y}))\vee\\
&(\psi(\bar{Y},\bar{y})\wedge\bigwedge_{i = 1}^{\length(\bar{X})} \forall\bar{z}\neg X_i(\bar{z}) \wedge \forall\bar{z}(\bar{x}\leq\bar{z})\wedge\exists\bar{w}(\bar{y} < \bar{w}_0 \vee \bigvee_{i = 1}^{\length(\bar{Y})}\neg Y(\bar{w}_i))\vee\\
&(\psi(\bar{Y},\bar{y})\wedge\bigwedge_{i = 1}^{\length(\bar{X})} \forall\bar{z} X_i(\bar{z}) \wedge \forall\bar{z}(\bar{z}\leq\bar{x})\wedge\bigwedge_{i = 1}^{\length(\bar{Y})} \forall\bar{z} Y_i(\bar{z}) \wedge \forall\bar{z}(\bar{z}\leq\bar{y}))].
\end{align*}
To construct an equivalent formula that is in the correct form, we define $\bar{u} = (\bar{v},\bar{w})$ and we replace the first-order quantifiers by a first-sum and count the first assignment to $\bar{v}$ and $\bar{w}$ that satisfies the formula. A similar construction was used in \cite{SalujaST95}. We obtain:
\begin{align*}
\sa{\bar{X}}\sa{\bar{Y}}\sa{\bar{x}}\sa{\bar{y}}\sa{\bar{u}}[&(\varphi(\bar{X},\bar{x})\wedge(\bar{v}_0 < \bar{x} \vee \bigvee_{i = 1}^{\length(\bar{X})}X(\bar{v}_i))\wedge\forall\bar{u}'((\bar{v}_0' < \bar{x} \vee \bigvee_{i = 1}^{\length(\bar{X})}X(\bar{v}_i'))\to\bar{u}\leq\bar{u}') \wedge\bigwedge_{i = 1}^{\length(\bar{Y})} \forall\bar{z}\neg Y_i(\bar{z}) \wedge \forall\bar{z}(\bar{y}\leq\bar{z})\vee\\
&(\varphi(\bar{X},\bar{x})\wedge\bigwedge_{i = 1}^{\length(\bar{X})} \forall\bar{z}\neg X_i(\bar{z}) \wedge \forall\bar{z}(\bar{x}\leq\bar{z})\wedge\bigwedge_{i = 1}^{\length(\bar{Y})} \forall\bar{z} Y_i(\bar{z}) \wedge \forall\bar{z}(\bar{z}\leq\bar{y})\wedge\forall\bar{u}'(\bar{u}\leq\bar{u}'))\vee\\
&(\psi(\bar{Y},\bar{y})\wedge\bigwedge_{i = 1}^{\length(\bar{X})} \forall\bar{z}\neg X_i(\bar{z}) \wedge \forall\bar{z}(\bar{x}\leq\bar{z})\wedge(\bar{y} < \bar{w}_0 \vee \bigvee_{i = 1}^{\length(\bar{Y})}\neg Y(\bar{w}_i))\wedge\forall\bar{u}'(\bar{y} < \bar{w}_0' \vee \bigvee_{i = 1}^{\length(\bar{Y})}\neg Y(\bar{w}_i'))\to\bar{u}\leq\bar{u}')\vee\\
&(\psi(\bar{Y},\bar{y})\wedge\bigwedge_{i = 1}^{\length(\bar{X})} \forall\bar{z} X_i(\bar{z}) \wedge \forall\bar{z}(\bar{z}\leq\bar{x})\wedge\bigwedge_{i = 1}^{\length(\bar{Y})} \forall\bar{z} Y_i(\bar{z}) \wedge \forall\bar{z}(\bar{z}\leq\bar{y})\wedge\forall\bar{u}'(\bar{u}\leq\bar{u}'))].
\end{align*}
This covers all possible cases for $\alpha$.

Finally, consider a $\eqso(\LL)$ formula $\alpha$ in $\LL$-SNF. If $\alpha = \sum_{i = 1}^n\alpha_i$, then consider $\alpha = (\alpha_1 + (\sum_{i = 2}^n\alpha_i))$. Then we use $\tau(\alpha)$ as the rewrite of $\alpha$, which satisfies the condition in the hypothesis.










\subsection*{Proof of Theorem \ref{prop-rest}}
We give this proof in three parts.

\vspace{1em}
First, we show that $\QE{0} \not\subseteq \E{1}$. By contradiction, suppose that there is a $\QE{0}$ formula $\alpha$ over some signature $\R$ such that defines the following function. For every finite $\R$-structure with $n$ elements, and where every predicate in $\R$ is empty, $\alpha(\enc(\A)) = n - 1$. We use the following claim.
\begin{claim}
Let $\alpha = \sa{\bar{x}}\varphi(\bar{x})$	where $\varphi$ is quantifier free. Then the function defined by $\alpha$ is either null, greater or equal to $n$, or is in $\Omega(n^2)$.
\end{claim}
\begin{proof}
Suppose that the function defined by $\alpha$ is not $0$ and that $\varphi$ is in DNF. Furthermore, suppose $\bar{x} = (x_1,\ldots,x_{\length(\bar{x})})$. Then $\alpha = \sa{\bar{x}} \varphi_1(\bar{x}) \vee \cdots \vee \varphi_n(\bar{x})$. Since $\alpha$ is not null, then some $\varphi_i$ must be satisfiable. This is, the function defined by $\sa{\bar{x}}\varphi(\bar{x})$ is not null. We will prove by induction on $\length(\bar{x})$ that the function defined by $\sa{\bar{x}}\varphi(\bar{x})$ is either greater or equal to $n$, or in $\Omega(n^2)$. We address the case $\length(\bar{x})= 1$, then $\alpha = \sa{x}\bigwedge\psi(x)$. If any $\psi(x) = (x = x)$ or $\neg(x < x)$, then we can eliminate it and we obtain the same function. If any $\psi = (x < x)$ or $\neg(x=x)$, then the function becomes null. If $\psi(x) = R(x,\ldots,x)$ for some $R\in\R$ the function becomes null for the structures we are considering. If $\psi(x) = \neg R(x,\ldots,x)$, we can eliminate it and for the structures we are considering we obtain the same function. The only possible $\alpha$ left is $\alpha = \sa{x}\top$ which is equal to the function $n$. This covers all possible cases for $\length(\bar{x}) = 1$. Now suppose that it holds for $\length(\bar{x}) = k$ and suppose $\alpha = \sa{\bar{x}}\bigwedge\psi(\bar{x})$ for $\length(\bar{x}) = k+1$. If any $\psi(\bar{x}) = (x_i = x_j)$ where $i \neq j$, then $\alpha$ describes the same function as $\alpha$ where $x_j$ has been replaced by $x_i$. In this formula the tuple of first-order variables has $k$ elements so the function it describes if one of the mentioned in the hypothesis. If $i = j$, then we can eliminate it and obtain the same function. If any $\psi(\bar{x}) = R(\bar{v})$ or $\neg R(\bar{v})$ where $\bar{v}$ is a sub-tuple of $\bar{x}$ then we can either eliminate it or the function becomes null, following the same argument as in the case $\length(\bar{x}) = 1$. If any $\psi(\bar{x}) = \neg(x_i = x_j)$ or $(x_i < x_j)$ where $i = j$, then the function becomes null. If any $\psi(\bar{x}) = \neg(x_i < x_j)$ where $i = j$, we can eliminate it. The remaining formulas in $\bigwedge\psi(\bar{x})$ are either $\neg(x_i = x_j)$, $(x_i<x_j)$ or $\neg(x_i<x_j)$. If the formula violates transitivity in $<$ (for example, $x < y \wedge y < z \wedge z < x$), then the function $\alpha$ describes is null. Therefore, there is some order over $\bar{x}$ that satisfies $\bigwedge\psi(\bar{x})$. Consider the formula that describes this order (like $x_1 < x_3 \wedge x_3 < x_4 \wedge x_4 < x_2$). The function $\alpha$ describes is greater or equal to the one this formula describes, which is exactly $\binom{n}{\length(\bar{x})}$ which is in $\Omega(n^{\length(\bar{x})}) \subseteq \Omega(n^2)$ if $\length(\bar{x}) > 1$. This concludes the proof of the claim.
\end{proof}
We suppose that $\alpha$ is in SNF, this is, $\alpha = \sum_{i = 1}^n\alpha_i$. Since $\alpha$ is not null, consider some $\alpha_i$ that describes a non-null function. Let $\alpha_i = \sa{\bar{X}}\sa{\bar{x}}\varphi(\bar{X},\bar{x})$, where $\varphi$ is quantifier-free. Note that if $\length(\bar{X}) > 0$, then the function $\alpha$ describes is in $\Omega(2^n)$, as it was proven by the authors in \cite{SalujaST95}. We have that $\alpha_i = \sa{\bar{x}}\varphi(\bar{x})$, as we proved in the claim, describes either some function greater or equal to $n$, or in $\Omega(n^2)$, which leads to a contradiction. Lastly, note that the formula $\sa{x}\exists y(x < y)$ is in $\E{0}$ and describes the function $n-1$, which concludes the proof.

\vspace{1em}
Now we show that $\E{1}\not\subseteq\QE{0}$. In Theorem \ref{theo-pi1-pnf} we proved that there is no formula in $\loge{1}$-PNF equivalent to the formula $\alpha = 2$. Every formula in $\E{1}$ can be expressed in $\loge{1}$-PNF, which implies that $2 \in \QE{0}$ and $2 \not\in \E{1}$.

\vspace{1em}
Lastly, we prove that $\eqso(\loge{1})\subsetneq\eqso(\logu{1})$. For inclusion, let $\alpha$ be a formula in $\eqso(\loge{1})$. Suppose that it is in $\loge{1}$-SNF. This is, $\alpha = c + \sum_{i = 1}^{n}\alpha_i$. Let $\alpha_i = \sa{\bar{X}}\sa{\bar{x}}\exists\bar{y}\,\varphi_i(\bar{X},\bar{x},\bar{y})$, where $\varphi_i$ is quantifier-free, for each $\alpha_i$. We use the same construction used in \cite{SalujaST95}, and we obtain that the formula $\exists\bar{y}\,\varphi_i(\bar{X},\bar{x},\bar{y})$ is equivalent to $\sa{\bar{y}}\,\varphi_i(\bar{X},\bar{x},\bar{y}) \wedge \forall\bar{y}'(\varphi_i(\bar{X},\bar{x},\bar{y}')\to\bar{y}\leq\bar{y}')$ for every assignment to $(\bar{X},\bar{x})$. We do this replacement for each $\alpha_i$ and we obtain an equivalent formula in $\eqso(\logu{1})$.

To prove that the inclusion is proper, consider the $\eqso(\logu{1})$ formula $\sa{x}\forall y(y = x)$. This formula defines the following function that takes an ordered structure $\A$ as input:
$$
\sem{\alpha}(\A) = 
\begin{cases}
1 &\A \text{ has one element}\\
0 &\text{ otherwise}.
\end{cases}
$$
Suppose that there exists an equivalent formula $\alpha$ in $\eqso(\loge{1})$. Also, suppose that it is in $\L$-PNF, so $\alpha = c + \sum_{i = 1}^n\sa{\bar{X}}\sa{\bar{x}}\exists\bar{y}\varphi_i(\bar{X},\bar{x},\bar{y})$. Since $\alpha$ takes the value 0 for some structures, $c$ must be 0. Consider a structure with one element $\mathfrak{1}$. We have that for some $i$, there exists an assignment $(\bar{B},\bar{b},\bar{a})$ for $(\bar{X},\bar{x},\bar{y})$ such that $\mathfrak{1}\models\varphi_i(\bar{B},\bar{b},\bar{a})$. Consider now the structure $\mathfrak{2}$ that is obtained by duplicating $\mathfrak{1}$, as we did for Theorem \ref{theo-pi1-pnf}. Note that $\mathfrak{2}\models\varphi_i(\bar{B},\bar{b},\bar{a})$, which implies that $\sem{\alpha}(\mathfrak{2}) \geq 1$, which leads to a contradiction.










\subsection*{Proof of Proposition \ref{prop-e1-nc}}

Towards a contradiction, assume that the statement is false. This is, that $\E{1}$ is closed under binary sum. Consider the formula $\sa{x}(x = x)$ which is in $\E{1}$ over some signature $\R$. For every finite $\R$-structure $\A$ with $n$ elements, and where every predicate in $\R$ is empty, $\alpha(\enc(\A)) = n$. From our assumption, there exists some formula in $\E{1}$ equivalent to the formula $\alpha \add \alpha$, which describes the function $2n$. Let $\sa{\bar{X}}\sa{\bar{x}}\exists\bar{y}\,\varphi(\bar{X},\bar{x},\bar{y})$ be this formula, where $\varphi$ is quantifier-free. Note that the function defined by this formula is equal or greater than the one defined by $\sa{\bar{X}}\sa{\bar{x}}\sa{\bar{y}}\,\varphi(\bar{X},\bar{x},\bar{y})$ divided by a polynomial factor. More specifically, for each ordered structure $\A$ with domain $A$, we have the following inequality:
$$
\sem{\sa{\bar{X}}\sa{\bar{x}}\exists\bar{y}\,\varphi(\bar{X},\bar{x},\bar{y})}(\A) \cdot \vert A \vert^{\length(\bar{y})} \geq \sem{\sa{\bar{X}}\sa{\bar{x}}\sa{\bar{y}}\,\varphi(\bar{X},\bar{x},\bar{y})}(\A)
$$
Note that the formula $\sa{\bar{X}}\sa{\bar{x}}\sa{\bar{y}}\,\varphi(\bar{X},\bar{x},\bar{y})$ defines a function in $\E{0}$. It was shown by the authors in \cite{SalujaST95} that every function in $\E{0}$ grows exponentially over the size of the structure for large enough structures, when $\length(\bar{X}) > 0$. This function divided by a polynomial factor still grows exponentially. Therefore, for $\sa{\bar{X}}\sa{\bar{x}}\exists\bar{y}\,\varphi(\bar{X},\bar{x},\bar{y})$ we have that $\length(\bar{X}) = 0$.

Now, for the formula $\sa{\bar{x}}\exists\bar{y}\,\varphi(\bar{x},\bar{y})$ consider a structure $\mathfrak{1}$ with only one element $a$. We have that $\sem{\sa{\bar{x}}\exists\bar{y}\,\varphi(\bar{x},\bar{y})}(\mathfrak{1}) = 2$, but the only possible assignment to $\bar{x}$ is the tuple $(a,\ldots,a)$ so $\sem{\sa{\bar{x}}\exists\bar{y}\,\varphi(\bar{x},\bar{y})}(\mathfrak{1}) \leq 1$, which follows to a contradiction.










\subsection*{Proof of Proposition \ref{prop:qe0-fp-qe1-totp-fptras}}

We separate this proof in three parts

\vspace{1em}
The authors in \cite{SalujaST95} proved that there exists a {\em product reduction} from every function in $\E{1}$ to a restricted version of $\cdnf$. This is, if $\alpha\in\E{1}$ over some signature $\R$, there exist polynomially computable functions $g:\ostr[\R]\to\ostr[\R_{\text{DNF}}]$ and $h:\nat\to\nat$ such that for every finite $\R$-structure $\A$ with domain $A$, it holds that $\sem{\alpha}(\A) = \cdnf(\enc(g(\A)))\cdot h(\vert A \vert)$. We base our proof on this fact.

For the first part, let $\alpha$ be a $\eqso(\loge{1})$ formula and assume that it is in $\loge{1}$-SNF. This is, $\alpha = \sum_{i = 1}^n\alpha_i$ where each $\alpha_i$ is in $\loge{1}$-PNF. Consider the following nondeterministic procedure that on input $\enc(\A)$ generates $\sem{\alpha}(\A)$ branches. For each $\alpha_i = \varphi$, where $\varphi$ is a $\loge{1}$ formula, it checks if $\A\models\varphi$ in polynomial time and generates a new branch if that is the case. For each $\alpha_i = \sa{\bar{X}}\sa{\bar{x}}\varphi$, this formula is also in $\E{1}$. We use the reduction to $\cdnf$ provided in \cite{SalujaST95} and we obtain $g(\enc(\A))$, which is an instance to $\cdnf$. Since $\cdnf$ is also in $\totp$ \cite{PagourtzisZ06}, we simulate the corresponding nondeterministic procedure that generates exactly $\cdnf(\enc(g(\A)))$ branches. Since, $\fp\subseteq\totp$\cite{PagourtzisZ06}, each polynomially computable function is also in $\totp$, and then on each of these branches we simulate the corresponding nondeterministic procedure to generate $h(\vert A \vert)$ more. The number of branches for each $\alpha_i$ is $\sem{\alpha_i}(\A) = \cdnf(\enc(g(\A)))\cdot h(\vert A \vert)$, and the total number of branches in the procedure amounts to $\sem{\alpha}(\A)$. We conclude that $\alpha\in\totp$.

\vspace{1em}
For the second part, let $\alpha$ be a $\eqso(\loge{1})$ formula and assume that it is in $\loge{1}$-SNF. This is, $\alpha = \sum_{i = 1}^n\alpha_i$ where each $\alpha_i$ is in $\loge{1}$-PNF. Note that each $\alpha_i$ that is equal to some $\loge{1}$ formula $\varphi$ has an FPRAS given by the procedure that simply checks if $\A\models\varphi$ and returns 1 if it does and 0 otherwise. Also, each remaining $\alpha_i$ has an FPRAS since $\alpha_i\in \E{1}$ \cite{SalujaST95}. If two functions have an FPRAS, then their sum also has one given by the procedure that simulates them both and sums the results. We conclude that $\alpha$ has an FPRAS.

\vspace{2em}
For the third part, we will prove that $\eqso(\loge{1})$ is closed under sum and multiplication. Since $\eqso(\loge{1})$ is closed under sum by definition, we focus only in proving that the class is closed under multiplication. We prove this for a more general case for $\eqso(\LL)$ where $\LL$ is a fragment of $\so$.

Let $\LL$ be a fragment {\bf closed under conjunction}. We will define a recursive function $\tau$ that receives a formula $\alpha$ over the grammar of $\eqso(\LL)$ extended by binary product, and outputs an equivalent formula $\tau(\alpha)$ over the unextended grammar of $\eqso(\LL)$. In fact, the formula $\tau(\alpha)$ is in $\LL$-SNF. First we replace each constant $s$ in $\alpha$ for $(\top \add \cdot \add \top)$ ($s$ times). If $\alpha = \varphi$, then we define $\tau(\alpha) = \alpha$. We assume that for every $\beta$ that has less algebraic operators than $\alpha$, $\tau(\beta)$ is in $\LL$-SNF. If $\alpha = (\alpha_1 + \alpha_2)$ then we define $\tau(\alpha) = \tau(\alpha_1) + \tau(\alpha_2)$. If $\alpha = \sa{x}\beta$ or $\alpha = \sa{X}\beta$, then we define $\tau(\alpha)$ as the formula in $\LL$-SNF that is equivalent to $\sa{x}\tau(\beta)$ and to $\tau(\alpha) = \sa{X}\tau(\beta)$, respectively. If $\alpha = (\alpha_1 \cdot \alpha_2)$, we assume that each $\alpha_i$ is in $\LL$-SNF. We identify three cases. (1) Some $\alpha_i$ is equal to $\sum_{j = 1}^n\beta_j$ for $n > 1$. Suppose wlog. that it is $\alpha_1$. We then define $\tau(\alpha) = \sum_{j = 1}^n\tau(\beta_j\cdot\alpha_2)$. In the following cases, $\alpha_1$ and $\alpha_2$ are in $\LL$-SNF. (2) If some $\alpha_i$ is equal to $\sa{X}\beta$ or $\sa{x}\beta$, we define $\tau(\alpha)$ as the $\LL$-SNF formula that is equivalent to $\sa{x}\tau(\beta\cdot\alpha_2)$ and $\sa{X}\tau(\beta\cdot\alpha_2)$, respectively. The remaining case is (3) $\alpha_1 = \varphi_1$ and $\alpha_2 = \varphi_2$ where each $\varphi$ is an $\LL$ formula. Then we define $\tau(\alpha) = \varphi_1 \wedge \varphi_2$. This covers all possible cases for $\alpha$. 

For every pair of formulas $\alpha,\beta$ in $\eqso(\LL)$, we have that their multiplication $(\alpha\cdot\beta)$ is a formula in the grammar $\eqso(\LL)$ extended by binary product, and so, there exists an equivalent formula $\tau(\alpha\cdot\beta)$ which is in unextended $\eqso(\LL)$. Since $\loge{1}$ is closed under conjunction, this also holds for $\eqso(\loge{1})$. This concludes the proof.










\subsection*{Proof of Proposition \ref{pi-minusone}}

Let $\LL$ be a fragment of $\fo$ that contains $\logu{1}$. Then we have that every function in $\U{1}$ is expressible in $\eqso(\LL)$. In particular, $\ctcnf \in \eqso(\LL)$. Suppose that $\eqso(\LL)$ is closed under subtraction by one. Then, the function $\ctcnf-1$, which counts the number of satisfying assignments of a 3-CNF formula minus one, is also in $\eqso(\LL)$. Note that $\eqso(\LL) \subseteq \eqso(\fo) = \shp$. We have that $\ctcnf$ is $\shp$-complete under parsimonious reductions\footnote{It can be easily verified that the standard reduction from SAT to 3-CNF (or 3-SAT) preserves the number of satisfying assignments}. Now, let $f$ be a function in $\shp$, and consider the nondeterministic polynomial-time procedure that on input $\enc(\A)$ computes the corresponding reduction to $\ctcnf$, name it $g(\enc(\A))$, and simulates the $\shp$ procedure for $\ctcnf-1$ on input $g(\enc(\A))$. We have that this is a $\shp$ procedure that computes $f-1$, from which we conclude that $\shp$ is closed under subtraction by one.










\subsection*{Proof of Theorem \ref{sigmafo-minusone}}

We divide the proof in two parts.

\vspace{1em}
First, note that for every fragment $\LL$ of $\so$ the class $\eqso(\LL)$ is closed under sum by definition, and if $\LL$ is closed under conjunction, like $\logex{1}$ is, $\eqso(\LL)$ is closed under product. We are only left to prove that $\eqso(\logex{1})$ is closed under subtraction by one.

Let $\alpha$ be a $\eqso(\logex{1})$ formula over a signature $\R$. We will define a $\eqso(\logex{1})$ formula $\kappa(\alpha)$ such that for each finite structure $A$ over $\R$: $\sem{\kappa(\alpha)}(\A) = \sem{\alpha}(\A) \dotminus 1$. We suppose that $\alpha$ is in $\logex{1}$-SNF and $\alpha = \sum_{i = 1}^{n}\sa{\bar{X}}\sa{\bar{x}}\varphi_i$ where each $\varphi_i$ is in $\logex{1}$. Moreover, we assume that $\length(\bar{x}) > 0$ since we can replace each $\varphi_i$ for the equivalent formula $\sa{y}\varphi_i\wedge\min(y)$.

The intuition behind our reasoning is separated in three points: (1) For each $\beta_i$ of the form $\sa{\bar{x}}\varphi$, the formula $\kappa(\beta_i)$ will count every tuple $\bar{x}$ that satisfies $\varphi$ except for the lexicographically smallest one. (2) For each $\beta_i$ of the form $\sa{\bar{X}}\sa{\bar{x}}\varphi$, the formula $\kappa(\beta_i)$ will isolate the smallest $\bar{X}$ that satisfies $\varphi$, and exclude the lexicographically smallest tuple $\bar{x}$ that satisfies $\varphi(\bar{X})$. And (3) if $\alpha = (\beta\add\sa{\bar{x}}\varphi)$ or $\alpha = (\beta\add\sa{\bar{X}}\sa{\bar{x}}\varphi)$, the formula $\kappa(\alpha)$ will exclude the lexicographically smallest tuple that satisfies $\varphi$ if and only if $\beta$ is equal to 0.

\vspace{1em}

For each $\alpha$ in $\eqso(\logex{1})$ such that $\alpha = \sa{\bar{x}}\exists\bar{y}\,\varphi(\bar{x},\bar{y})$ for some quantifier-free formula $\varphi$, we define $\kappa(\alpha) = \sa{\bar{x}}\exists\bar{y}[\varphi(\bar{x},\bar{y})\wedge\exists\bar{z}(\varphi(\bar{z},\bar{y})\wedge\bar{z} < \bar{x})]$, which is in $\eqso(\logex{1})$ and fulfils the desired condition.

\vspace{1em}

For each $\alpha$ in $\eqso(\logex{1})$ such that $\alpha = \sa{\bar{X}}\sa{\bar{x}}\exists\bar{y}\,\varphi(\bar{X},\bar{x},\bar{y})$ for some quantifier-free formula $\varphi$, we define $\kappa(\alpha)$ procedurally as follows: Let $\bar{x} = (x_1,\ldots,x_{\length(\bar{x})})$ and $\bar{X} = (X_1,\ldots,X_{\length(\bar{X})})$. We suppose that $\varphi$ is in a DNF form that leaves formulas that do not mention $\bar{X}$ intact. If it is not, we convert $\varphi$ to this form with a standard transformation algorithm. Let $\varphi(\bar{X},\bar{x},\bar{y}) = \bigvee_{i = 1}^m\varphi_i(\bar{X},\bar{x},\bar{y})$ where each $\varphi_i$ has the form:
$$
\varphi_i(\bar{X},\bar{x},\bar{y}) = (\text{conjunction of $X_j$'s}) \wedge (\text{conjunction of $\neg X_j$'s})  \wedge (\text{an $\fo$ formula that does not mention any $X_j$}).
$$
Define $\varphi_i^{+}$, $\varphi_i^{-}$ and $\varphi_i^{\fo}$ as the formulas mentioned above. 

In our procedure we assume that in $\varphi_i^{+}\wedge\varphi_i^{-}$ every first-order variable from $\bar{x},\bar{y}$ is mentioned at most once. If not, we add new first-order variables from $\fv$ to $\bar{y}$ so that no variable is repeated, replace them in $\varphi_i^{+}\wedge\varphi_i^{-}$ and add their respective equalities in $\varphi_i^{\fo}$ so that the new formula is equivalent. For example, if $\bar{x} = x$, $\bar{y} = y$ and $\varphi_i = X(x,y)\wedge \neg X(x,x) \wedge x < y$, then we redefine $\bar{y} = (y,v_1,v_2,v_3,v_4)$ and $\varphi_i := X(v_1,v_2) \wedge \neg X(v_3,v_4) \wedge x < y \wedge v_1 = x \wedge v_2 = y \wedge v_3 = x \wedge v_4 = x.$ 

Furthermore, we assume that $\varphi_i^{\fo}$ defines an ordered partition of the variables in the tuple $(\bar{x},\bar{y})$. For example, let $\bar{x} = (x_1,x_2,x_3,x_4)$. A possible ordered partition would be defined by the formula $\theta(\bar{x}) = x_2 < x_1 \wedge x_1 = x_4 \wedge x_4 < x_3$. On the other hand, the formula $\theta'(\bar{x}) = x_1 < x_2 \wedge x_1 < x_4 \wedge x_2 = x_3$ does not define an ordered partition since both $\{x_1\}<\{x_2,x_3\}<\{x_4\}$ and $\{x_1\} < \{x_2,x_3,x_4\}$ satisfy $\theta'$.
For a given $k$, we define $\cB_k$ as the number of possible ordered partitions for a set of size $k$. 
If $\varphi_i^{\fo}$ does not define an ordered partition, then for $1 \leq j \leq \cB_{\length(\bar{x},\bar{y})}$ 
let $\theta^j(\bar{x},\bar{y})$ be a formula that defines an ordered partition over $(\bar{x},\bar{y})$. The formula $\varphi_i(\bar{X},\bar{x},\bar{y})$ is converted into $\bigvee_{j = 1}^{\cB_{\length(\bar{x},\bar{y})}}\varphi_i(\bar{X},\bar{x},\bar{y}) \wedge \theta^j(\bar{x},\bar{y})$. Consider each of these disjuncts as a new $\varphi_i$.

After the previous assumptions, we do the following: for each $X_j\in \bar{X}$ we check every instance of $X_j(\bar{w})$ in $\varphi^{+}_i$ and every instance of $X_j(\bar{z})$ in $\varphi^{-}_i$, where $\bar{w}$ and $\bar{z}$ are subtuples of $(\bar{x},\bar{y})$. If the ordered partition in $\varphi^{\fo}_i$ satisfies $\bar{w} = \bar{z}$, the entire formula $\varphi_i$ is removed from $\varphi$.

Now let $\bar{v}$ be the tuple of every first-order variable mentioned in $\varphi_i^{+}$ and let $\bar{u}$ be such that $(\bar{x},\bar{y}) = (\bar{u},\bar{v})$. We define a formula $\mu_i$ that is satisfied only by the lexicographically smallest $\bar{v}$ that satisfies $\varphi_i^{\fo}$ for some $\bar{u}$:
$$
\mu_i(\bar{v}) = \exists\bar{u}\,\varphi_i^{\fo}(\bar{u},\bar{v})\wedge\forall\bar{u}'\forall\bar{v}'(\varphi_i^{\fo}(\bar{u}',\bar{v}')\to\bar{v}\leq\bar{v}').
$$
The following is a pivotal formula for this proof. Consider some ordered finite structure $\A$ over $\R$. Given our conditions for $\varphi_i$, if $\A\models\varphi_i$ and $\A\models\varphi^{\fo}_i(\bar{u},\bar{v})$ then for that $\bar{v}$ we can define a $\bar{X}$ that has exactly the tuples mentioned in $\bar{v}$ and nothing else. This is {\em the smallest $\bar{X}$ that satisfies $\varphi_i$ over $\bar{v}$}. And if $\bar{v}$ is the lexicographically smallest such that $\A\models\varphi_i(\bar{v})$, then it is the {\em the smallest $\bar{X}$ that satisfies $\varphi_i$}. The following formula is satisfied by every pair $(\bar{X},\bar{x})$, except for the one formed by the smallest $\bar{X}$ that satisfies $\varphi_i$ and the lexicographically smallest $\bar{x}$ that satisfies $\varphi_i(\bar{X})$. And if no pair satisfies $\varphi_i$, the following formula is satisfied by every pair.
$$
\psi_i(\bar{X},\bar{x}) = \exists\bar{v}(\mu_i(\bar{v})\wedge(\neg\varphi^{+}_i(\bar{X},\bar{v})\vee\bigvee_{X \in \bar{X}} \exists\bar{z}(X(\bar{z}) \wedge \bigwedge\limits_{\substack{\text{instances of }X(\bar{w}) \\ \text{in }\varphi^{+}_i(\bar{X},\bar{v})}}\bar{w}\neq\bar{z}))) \vee \exists\bar{x}'(\exists\bar{y}'\varphi_i(\bar{X},\bar{x}',\bar{y}')\wedge \bar{x}'<\bar{x})\vee\neg\exists\bar{v}\mu_i(\bar{v}) .
$$
\begin{lemma}
	For a given ordered structure $\A$ such that $\A\models\exists\bar{X}\exists\bar{x}\exists\bar{y}\,\varphi_i(\bar{X},\bar{x},\bar{y})$, there is an assignment $(\bar{B},\bar{b})$ to $(\bar{X},\bar{x})$ that satisfies the following conditions (1) $\A\models\exists\bar{y}\,\varphi_i(\bar{B},\bar{b},\bar{y})$, (2) $\A\not\models\psi_i(\bar{B},\bar{b})$ and (3) this is the only assignment to $(\bar{X},\bar{x})$ that satisfies (1) and (2).
\end{lemma}
\begin{proof}
	Excusing the nested proof, we have a claim that identifies a crucial condition that is verifiable with a $\fo$ formula. 
	\begin{claim}
		Let $\varphi_i(\bar{X},\bar{x},\bar{y}) = \varphi^{\fo}_i(\bar{x},\bar{y}) \wedge \varphi^{-}_i(\bar{X},\bar{x},\bar{y}) \wedge \varphi^{+}_i(\bar{X},\bar{x},\bar{y})$ be a $\logex{0}$ formula that satisfies all the mentioned assumptions. For a given ordered structure $\A$ it holds that $\A\models\exists\bar{x}\exists\bar{y}\,\varphi^{\fo}_i(\bar{x},\bar{y})$ iff $\A\models\exists\bar{X}\exists\bar{x}\exists\bar{y}\,\varphi_i(\bar{X},\bar{x},\bar{y})$.
	\end{claim}
	\begin{proof}
		Let $\A$ be an ordered structure with domain $A$ and let $\bar{a}\in A^{\length(\bar{y})}$ and $\bar{b}\in A^{\length(\bar{x})}$ such that $\A\models\varphi^{\fo}_i(\bar{b},\bar{a})$. Define $\bar{B} = (B_1,\ldots,B_{\length(\bar{X})})$ as $B_j = \{\bar{c}\mid\bar{c}\text{ is a subtuple of $(\bar{b},\bar{a})$ and $X_j(\bar{c})$ is mentioned in $\varphi^{+}_i(\bar{X},\bar{b},\bar{a})$}\}$. Towards a contradiction, suppose that $\A\not\models\varphi_i(\bar{B},\bar{b},\bar{a})$. By the choice of $\bar{a}$ and $\bar{b}$, and construction of $\bar{B}$ it is clear that $\A\models\varphi^{\fo}_i(\bar{b},\bar{a})\wedge\varphi^{+}_i(\bar{B},\bar{b},\bar{a})$, so we have that $\A\not\models\varphi^{-}_i(\bar{B},\bar{b},\bar{a})$. Let $B_t(\bar{c})$ be such that $X_t(\bar{c})$ is mentioned in $\varphi^{-}_i(\bar{X},\bar{b},\bar{a})$ and $\A\not\models\neg B_t(\bar{c})$. This is, $\bar{c}\in B_t$. By the construction of $\bar{B}$ we have that $X_t(\bar{c})$ is mentioned in $\varphi^{+}_i(\bar{X},\bar{b},\bar{a})$. From our assumptions, we have that (1) every first-order variable is mentioned at most once in $\varphi^{-}_i(\bar{X},\bar{x},\bar{y}) \wedge \varphi^{+}_i(\bar{X},\bar{x},\bar{y})$, (2) $\varphi^{\fo}_i$ defines an ordered partition over $(\bar{x},\bar{y})$, and (3) there are no $X_j(\bar{w})$ in $\varphi^{+}_i$ and $X_j(\bar{z})$ in $\varphi^{-}_i$ such that the ordered partition defined by $\varphi^{\fo}_i$ satisfies $\bar{x} = \bar{z}$. Since $\bar{c}$ is a subtuple of $(\bar{b},\bar{a})$ that satisfies $\varphi^{\fo}_i$, and $X_t(\bar{w})$ mentioned in $\varphi^{+}_i$ and $X_t(\bar{z})$ mentioned in $\varphi^{-}_i$ were both assigned the value $\bar{c}$, then the ordered partition satisfies $\bar{w} = \bar{z}$, which follows to a contradiction.
	\end{proof}
	We give the proof for the base theorem. Let $\A$ be an ordered structure with domain $A$. We clearly have that $\A\models\exists\bar{x}\exists\bar{y}\,\varphi^{\fo}_i(\bar{x},\bar{y})$, so let $\bar{a}\in A^{\length(\bar{y})}$ and $\bar{b}\in A^{\length(\bar{x})}$ such that $\A\models\mu_i(\bar{b},\bar{a})$. We use the same construction in the claim and we obtain the assignment $(\bar{B},\bar{b},\bar{a})$ which satisfies the three conditions.
\end{proof}

We define $\chi_i = \exists\bar{x}\exists\bar{y}\,\varphi^{\fo}_i(\bar{x},\bar{y})$. Now recall that $\varphi = \bigvee_{i = 1}^m\varphi_i(\bar{X},\bar{x},\bar{y})$, where each $\varphi_i$ satisfies all the previous assumptions. For each $\varphi_i$ we define:
$$
\varphi_i'(\bar{X},\bar{x},\bar{y}) = \varphi_i(\bar{X},\bar{x},\bar{y})\wedge\psi_1(\bar{X},\bar{x})\wedge(\chi_1\vee\psi_2(\bar{X},\bar{x}))\wedge(\chi_1\vee\chi_2\vee\psi_2(\bar{X},\bar{x}))\wedge\cdots\wedge(
\bigvee_{j = 1}^{j = i-1}\chi_j\vee\psi_i(\bar{X},\bar{x})),
$$
and lastly, $\kappa(\alpha)$ is defined as $\kappa(\alpha) = \sa{\bar{X}}\sa{\bar{x}}\exists\bar{y}\bigvee_{i = 1}^m\varphi_i'(\bar{X},\bar{x},\bar{y})$.

\vspace{1em}

For each $\alpha$ in $\eqso(\logex{1})$ such that $\alpha = (\beta + \sa{\bar{X}}\sa{\bar{x}}\exists\bar{y}\,\varphi(\bar{X},\bar{x},\bar{y}))$ for some algebraic formula $\beta$ and some quantifier-free formula $\varphi$, we define $\kappa(\alpha)$ as follows: First, perform the same transformations to $\varphi(\bar{X},\bar{x},\bar{y})$ as in the previous case. Let $\varphi = \bigvee_{i = 1}^m\varphi_i(\bar{X},\bar{x},\bar{y})$ where each $\varphi_i$ satisfies the previous assumptions. We also use the previously defined formulas $\chi_i$ and $\psi_i$. 

We recursively define a function $\lambda$ as follows. If $\alpha = \sa{\bar{x}}\exists\bar{y}\varphi(\bar{x},\bar{y})$, then $\lambda(\alpha) = \exists\bar{x}'\exists\bar{y}'\varphi(\bar{x}',\bar{y}')$. If $\alpha = \sa{\bar{X}}\sa{\bar{x}}\exists\bar{y}\,\varphi(\bar{X},\bar{x},\bar{y})$, then let $\varphi = \bigvee_{i = 1}^{m}\varphi_i$ and $\lambda(\alpha) = \chi_1\vee\cdots\vee\chi_m$ as each $\chi_i$ was previously defined. If $\alpha = (\beta + \gamma)$, then $\lambda(\alpha) = \lambda(\beta) \vee \lambda(\gamma)$. Note that for a given ordered structure $\A$, then $\A\models\lambda(\alpha)$ if and only if $\sem{\alpha}(\A) > 0$.

Now, for each $\varphi_i$ we define:
$$
\varphi_i'(\bar{X},\bar{x},\bar{y}) = \varphi_i(\bar{X},\bar{x},\bar{y})\wedge[\lambda(\beta)\vee[\psi_1(\bar{X},\bar{x})\wedge(\chi_1\vee\psi_2(\bar{X},\bar{x}))\wedge(\chi_1\vee\chi_2\vee\psi_2(\bar{X},\bar{x}))\wedge\cdots\wedge(
\bigvee_{j = 1}^{j = i-1}\chi_j\vee\psi_i(\bar{X},\bar{x}))]].
$$
And lastly $\kappa(\alpha)$ is defined as $\kappa(\alpha) = \kappa(\beta) + \sa{\bar{X}}\sa{\bar{x}}\exists\bar{y}\bigvee_{i = 1}^m\varphi_i'(\bar{X},\bar{x},\bar{y})$, which is in $\eqso(\logex{1})$ and satisfies the desired condition.

\vspace{1em}
For the second part, for each $\alpha$ in $\eqso(\logex{0})$ over a signature $\R$ we will define a formula $\lambda(\alpha)$ in $\eqso(\loge{0})$ over a signature $\R_{\alpha}$, and a function $g_{\alpha}$ that receives a finite $\R$-structure $\A$ and outputs a finite $\R_{\alpha}$-structure $g_{\alpha}(\A)$. Let $\alpha$ be in $\eqso(\logex{0})$. The signature $\R_{\alpha}$ is defined by adding the symbol $R_{\psi}$ for every $\fo$ formula $\psi(\bar{z})$. Each symbol $R_{\psi}$ represents a predicate with arity $\length(\bar{z})$. Then, $\lambda(\alpha)$ is defined by replacing every $\fo$ formula $\psi(\bar{z})$ for $R_{\psi}(\bar{z})$. We now define the function $g_{\alpha}$. Let $\A$ be a $\R$-structure with domain $A$. Let $\A'$ be a $\R_{\alpha}$-structure obtained by copying $\A$ and leaving each $R_{\psi}^{\A}$ empty. For each $\fo$-formula $\psi(\z)$ with $\length(\bar{z})$ open first-order variables, we iterate for every tuple $\bar{a} \in A^{\length(\bar{z})}$. If $\A\models\psi(\bar{a})$, then the tuple $\bar{a}$ is added to $R_{\psi}^{\A'}$. This concludes the construction of $\A'$. Note that the number of $\fo$ subformulas and each arity and tuple size is fixed to $\alpha$, so the entire procedure takes polynomial time over the size of the structure (and $g_{\alpha}(\A)$ has polynomial size over $\A$). We define $g_{\alpha}(\A) = \A'$ and we have that for each finite $\R$-structure $\A$: $\sem{\alpha}(\A) = \sem{\lambda(\alpha)}(g_{\alpha}(\A))$. Therefore, we have a parsimonious reduction from $\alpha$ to $\lambda(\alpha)$, which is in $\eqso(\loge{1})$.

To show that $\alpha$ is in $\totp$, consider a procedure that takes an input $\A$, converts it to $g_{\alpha}(\A)$ and simulates the $\totp$ procedure for $\lambda(\alpha)$ on input $g_{\alpha}(\A)$. This procedure generates exactly $\sem{\alpha}(\A) = \sem{\lambda(\alpha)}(g_{\alpha}(\A))$ branches, and therefore $\alpha$ is in $\totp$.

To show that $\alpha$ has an FPRAS, consider a procedure that takes an input $\A$, converts it to $g_{\alpha}(\A)$ and simulates the FPRAS for $\lambda(\alpha)$. This procedure also takes polynomial time and satisfies the condition.










\subsection*{Proof of Theorem \ref{sub-pnp}}

Towards a contradiction, suppose that any of the classes $\E{1}$, $\eqso(\loge{1})$, and $\eqso(\logex{1})$ is closed under subtraction.

Let $\R$ include the symbols $S_1, S_2, S_3, S_4$, which describe the following properties. If a finite $\R$-structure $\A$ defines a 3DNF formula $\Phi$, then its domain is the set of variables mentioned in $\Phi$, and for each $i = 1,2,3,4$:
\begin{align*}
	S_1^\A &= \{(a_1,a_2,a_3)\mid (\neg a_1 \wedge \neg a_2 \wedge \neg a_3) \mbox{ appears as a disjunct in }\Phi\},\\
	S_2^\A &= \{(a_1,a_2,a_3)\mid ( a_1 \wedge \neg a_2 \wedge \neg a_3) \mbox{ appears as a disjunct in }\Phi\},\\
	S_3^\A &= \{(a_1,a_2,a_3)\mid ( a_1 \wedge  a_2 \wedge \neg a_3) \mbox{ appears as a disjunct in }\Phi\},\\
	S_4^\A &= \{(a_1,a_2,a_3)\mid ( a_1 \wedge  a_2 \wedge  a_3) \mbox{ appears as a disjunct in }\Phi\}.
\end{align*}
Now we define $f_{\#3DNF} = f_{\psi(T)}$ where
\begin{multline*}
\psi(T) = \exists x \exists y \exists z\, [(S_1(x,y,z) \wedge \neg T(x) \wedge \neg T(y) \wedge \neg T(z)) \vee (S_2(x,y,z) \wedge T(x) \wedge \neg T(y) \wedge \neg T(z)) \, \vee \\ (S_3(x,y,z) \wedge T(x) \wedge T(y) \wedge \neg T(z)) \vee (S_4(x,y,z) \wedge T(x) \wedge T(y) \wedge T(z))].
\end{multline*}
Note that $f_{\#3DNF} \in \#\Sigma_1$. Let $f_{all} = f_{\exists x\:\varphi(x,X)}$, where
$$
\varphi(x,X) = (T(x) \vee \neg T(x)).
$$
Note that $f_{all}$ counts every possible truth assignment (satisfying or not) to a 3DNF formula. Given that $f_{\#3DNF}, f_{all} \in \E{1} \subseteq \eqso(\loge{1}) \subseteq \eqso(\logex{1}) \subseteq \totp$, and at least one of the classes is closed under subtraction, the function $f_{\#3DNF} - f_{all}$ is in $\totp$. However, note that for each $\R$-structure $\A$ that represents a formula $\Phi$, $(f_{\#3DNF} - f_{all})(\A) = 0$ if an only if $\Phi$ is a tautology. The decision version of this function is the $\np$-complete problem $\textsc{Tautology}$. And since the function is in $\totp$, its decision version is also in $\ptime$. We conclude that $\np \subseteq \ptime$.










\subsection*{Proof of Proposition \ref{prop:ehorn-pe}}
Pagourtzis and Zachos mention a $\totp$ procedure that computes the number of satisfying assignments of a DNF formula \cite{PagourtzisZ06}. This procedure can be easily extended to receive Horn formulas, and furthermore, a disjunction of Horn formulas. We can deduce that $\shdhsat$ is in $\totp$.

As we show in Proposition \ref{sigma2hard}, $\shdhsat$ is complete for $\eqso(\ehorn)$ under parsimonious reductions. Let $\alpha$ be a formula in $\eqso(\ehorn)$ and let $g_{\alpha}$ be the reduction to $\shdhsat$. The $\totp$ procedure we construct, for each input $\enc(\A)$, is simply to compute $g_{\alpha}(\enc(\A))$, and then simulate the $\totp$ procedure for $\shdhsat$ on input $g_{\alpha}(\enc(\A))$. We conclude that $\alpha$ is in $\totp$.










\subsection*{Proof of Proposition \ref{prop:hsat-not-sigma2}} %V.12
We use a similar proof to the one provided by the authors in \cite{SalujaST95} to separate the classes $\E{2}$ and $\U{2}$. Suppose that the statement is false, this is, $\chsat \in \eqso(\loge{2})$. We consider the signature $\R$ that we used as the encoding for a Horn formula (Example \ref{ex-hornsat-esop1}) and that the formula $\alpha \in \eqso(\loge{2})$ follows the encoding in the same way. From what we proved in Theorem \ref{theo-pnf-snf}, we have that every formula in $\eqso(\loge{2})$ can be rewritten in $\loge{2}$-PNF, so we assume that $\alpha$ is in this form. Let $\alpha = \sa{\bar{X}}\sa{\bar{x}}\exists\bar{y}\,\forall\bar{z}\,\varphi(\bar{X},\bar{x},\bar{y},\bar{z})$. Consider the following Horn formula $\Phi$:
$$
\Phi = p \wedge \bigwedge_{i = 1}^n (t_i \wedge p \to q) \wedge \neg q,
$$
where $n = \length(\bar{x}) + \length(\bar{y}) + 1$. Let $\A$ be the encoding of this formula. In the encoding, each variable appears as an element in the domain of $\A$. This formula is satisfiable, so $\sem{\alpha}(\A) \geq 1$. Let $(\bar{B},\bar{b},\bar{a})$ be an assignment to $(\bar{X},\bar{x},\bar{y})$ such that $\A\models\forall\bar{z}\,\varphi(\bar{B},\bar{b},\bar{a},\bar{z})$. Let $t_{\ell}$ be such that it does not appear in $\bar{b}$ or $\bar{a}$. Consider the induced substructure $\A'$ that is obtained by removing $t_{\ell}$ from $\A$ and $\bar{B}'$ as the subset of $\bar{B}$ obtained by deleting each appearance of $t_{\ell}$ in $\bar{B}$. We have that $\A'\models\forall\bar{z}\,\varphi(\bar{B},\bar{b},\bar{a},\bar{z})$. This is because each subformula of the form $\exists y \neg B_i$ is still true, and universal formulas are monotone over induced substructures. It follows that $\sem{\alpha}(\A') \geq 1$ which is not possible since $\A'$ encodes the formula
$$
\Phi' = p \wedge \bigwedge_{i = 1}^{\ell-1} (t_i \wedge p \to q) \wedge (p\to q) \wedge \bigwedge_{i = \ell+1}^{n} (t_i \wedge p \to q) \wedge \neg q,
$$
which is unsatisfiable. We arrive to a contradiction and we conclude that $\chsat$ is not in $\eqso(\ehorn)$.








\subsection*{Proof of Theorem \ref{sigma2hard}} %V.13

First we prove that $\shdhsat$ is in $\eqso(\ehorn)$. Recall that each instance of $\shdhsat$ is a disjunction of Horn formulas. Let $\R = \{\pP(\cdot,\cdot), \pN(\cdot,\cdot), \pV(\cdot), \pNC(\cdot), \pD(\cdot,\cdot)\}$. Each symbol in this vocabulary is used to indicate the same as in Example \ref{ex-hornsat-esop1}, with the addition of $\pD(d,c)$ which indicates that $c$ is a clause in the formula $d$. Recall that the formula
\begin{align*}
&\forall x \, (\neg \pT(x) \vee \pV(x)) \ \wedge\\
&\forall c \, (\neg \textit{NC}(c) \vee \exists x \, \neg \textit{A}(c,x)) \ \wedge\\
&\forall c \forall x \, (\neg \textit{P}(c,x) \vee \exists y \, \neg \textit{A}(c,y) \vee \textit{T}(x)) \ \wedge\\
&\forall c \forall x \, (\neg \textit{N}(c,x) \vee \textit{T}(x) \vee \neg \textit{A}(c,x)) \ \wedge\\
&\forall c \forall x \, (\textit{A}(c,x) \vee \textit{N}(c,x)) \ \wedge\\
&\forall c \forall x \, (\textit{A}(c,x) \vee \neg\textit{T}(x)).
\end{align*}
defines $\chsat$. We obtain the following formula $\psi(T,A)$ in $\ehorn$:
\begin{align*}
\exists d[&\forall x \, (\neg \pT(x) \vee \pV(x)) \ \wedge\\
&\forall c \, (\neg \pD(c,d)\vee \neg \textit{NC}(c) \vee \exists x \, \neg \textit{A}(c,x)) \ \wedge\\
&\forall c \forall x \, (\neg \pD(c,d)\vee\neg \textit{P}(c,x) \vee \exists y \, \neg \textit{A}(c,y) \vee \textit{T}(x)) \ \wedge\\
&\forall c \forall x \, (\neg \pD(c,d)\vee\neg \textit{N}(c,x) \vee \textit{T}(x) \vee \neg \textit{A}(c,x)) \ \wedge\\
&\forall c \forall x \, (\neg \pD(c,d)\vee\textit{A}(c,x) \vee \textit{N}(c,x)) \ \wedge\\
&\forall c \forall x \, (\neg \pD(c,d)\vee\textit{A}(c,x) \vee \neg\textit{T}(x))]
\end{align*}
effectively defines $\chsat$ as for every disjunction of Horn formulas $\theta = \theta_1\vee\cdots\vee\theta_m$ encoded by an $\R$-structure $\A$, the number of satisfying assignments of $\theta$ is equal to $\sem{\sa{\pT} \sa{\pA} \psi(\pT,\pA)}(\A)$.  Therefore, we conclude that $\shdhsat \in \eqso(\ehorn)$.

\vspace{1em}
We will now prove that $\shdhsat$ is hard for $\eqso$ over a signature $\R$ under parsimonious reductions. For each $\eqso(\ehorn)$ formula $\alpha$ over $\R$, we will define a polynomial-time procedure that computes a function $g_{\alpha}$. This function receives a finite $\R$-structure $\A$ and outputs an instance of $\shdhsat$ such that $\sem{\alpha}(\A) = \shdhsat(g_{\alpha}(\A))$. We suppose that $\alpha$ is in sum normal form and:
$$
\alpha = c + \sum_{i = 1}^{\text{\#clauses}} \sa{\bar{X}}\sa{\bar{x}}\exists\bar{y}\bigwedge_{j = 1}^{n}\forall\bar{z}\,\varphi^i_j(\bar{X},\bar{x},\bar{y},\bar{z}),
$$
where each $\varphi^i_j$ is a Horn clause.                                                                

Consider a finite $\R$-structure $\A$ with domain $A$. To simplify the proof, we extend our grammar to allow first-order constants. Consider each tuple $\bar{a}\in A^{\length(\bar{x})}$, each $\bar{b}\in A^{\length(\bar{y})}$ and each $\bar{c}\in A^{\length(\bar{z})}$ as a tuple of first-order constants. The following formula defines the same function as $\alpha$:
$$
c + \sum_{i = 1}^{\#clauses} \sum_{\bar{a}\in A^{\length(\bar{x})}} \sa{\bar{X}}\bigvee_{\bar{b}\in A^{\length(\bar{y})}}\bigwedge_{j = 1}^{n}\bigwedge_{\bar{c}\in A^{\length(\bar{z})}}\varphi^i_j(\bar{X},\bar{a},\bar{b},\bar{c}).
$$
Note that each $\fo$ formula over $(\bar{x},\bar{y},\bar{z})$ in each $\varphi^i_j$ has no free variables. Therefore, we can evaluate each of these in polynomial time and replace them by $\perp$ and $\top$ where it corresponds. Each $\varphi^i_j$ will be of the form $\perp \vee\, \chi^i_j(\bar{X})$ or $\top \vee \chi^i_j(\bar{X})$ where $\chi^i_j$ is a disjunction of $\neg X_{\ell}$'s and at most one $X_{\ell}$. The formulas of the form $\top \vee \chi^i_j(\bar{X})$ can be removed entirely, and the formulas of the form $\perp \vee\, \chi^i_j(\bar{X})$ can be replaced by $\chi^i_j(\bar{X})$. We obtain the formula
$$
c + \sum_{i = 1}^{m}\sa{\bar{X}}\bigvee_{j = 1}^{\#d}\bigwedge_{k = 1}^{\#c}\psi^{i}_{j,k}(\bar{X})
$$
where every $\psi^{i}_{j,k}(\bar{X})$ is a disjunction of $\neg X_{\ell}$'s and zero or one $X_{\ell}$.

Our idea for the rest of the proof is to define $g_{\alpha}$ for each $\alpha = \sa{\bar{X}}\bigvee_{j = 1}^{\#d}\bigwedge_{k = 1}^{\#c}\psi^{i}_{j,k}(\bar{X})$, for $\alpha = c$ and for $\alpha = \beta_1 + \cdots + \beta_m$ where each $\beta_i$ is in one of the two previous cases.

If $\alpha$ is equal to $\sa{\bar{X}}\bigvee_{j = 1}^{\#d}\bigwedge_{k = 1}^{\#c}\psi_{j,k}(\bar{X})$ where $\psi_{j,k}(\bar{X})$ is a disjunction of $\neg X_{\ell}$'s and zero or one $X_{\ell}$, then we obtain the {\bf propositional formula} $g_{\alpha}(\A) = \bigvee_{j = 1}^{\#d}\bigwedge_{k = 1}^{\#c}\psi_{j,k}(\bar{X})$ over the propositional alphabet $\{X(\bar{e}) \mid X \in \bar{X} \text{ and } \bar{e}\in A^{\arity(X)} \}$ which has exactly $\sem{\alpha}(\A)$ satisfying assignments and is precisely a disjunction of Horn formulas.

If $\alpha$ is equal to a constant $c$, then we define $g_{\alpha}(\A)$ as the following formula that has exactly $c$ satisfying assignments:
$$
g_{\alpha}(\A) = \bigvee_{i = 1}^{c}\neg t_1 \wedge \cdots \wedge \neg t_{i-1} \wedge t_i \wedge \neg t_{i+1} \wedge \cdots \wedge \neg p_c.
$$ 
If $\alpha = \beta_1 + \cdots + \beta_m$, let $g_{\beta_i}(\A) = \bigvee_{j = 1}^{\#d}\bigwedge_{k = 1}^{\#c}\theta^i_{j,k}$ for each $\beta_i$ where each $\theta^i_{j,k}$ is a Horn clause. Let $\Theta_i = g_{\beta_i}(\A)$. We rename the variables in each $\Theta_i$ so none of them are mentioned in any other $\Theta_j$. We add $m$ new variables $t_1,\ldots,t_m$ and we define:
\begin{align*}
g_{\alpha}(\A) = &\bigvee_{i = 1}^{\#d}(\bigwedge_{j = 1}^{\#c}\theta^1_{i,j} \wedge (\bigwedge\limits_{\substack{\text{each } t\\ \text{ mentioned in}\\ \Theta_2,\ldots,\Theta_{m}}}t) \wedge (t_1 \wedge \bigwedge_{\ell = 2}^{m} \neg t_{\ell})) \vee \\ 
&\bigvee_{i = 1}^{\#d}(\bigwedge_{j = 1}^{\#c}\theta^2_{i,j} \wedge (\bigwedge\limits_{\substack{\text{each $t$}\\ \text{ mentioned in}\\ \Theta_1,\Theta_3,\ldots,\Theta_{m}}}t) \wedge (t_2 \wedge \bigwedge\limits_{\substack{\ell = 1 \\ \ell \neq 2}}^{m} \neg t_{\ell})) \vee \cdots \vee\\ 
&\bigvee_{i = 1}^{\#d}(\bigwedge_{j = 1}^{\#c}\theta^m_{i,j} \wedge (\bigwedge\limits_{\substack{\text{each } t\\ \text{ mentioned in}\\ \Theta_2,\ldots,\Theta_{m-1}}}t) \wedge (t_m \wedge \bigwedge_{\ell = 1}^{m-1} \neg t_{\ell})).
\end{align*}
The formula is a disjunction of Horn formulas, and the number of satisfying assignments for this formula is exactly the sum of satisfying assignments for each $g_{\beta_i}(\A)$. This, at the same time, is equal to $\sem{\alpha}(\A)$. This covers all possible cases for $\alpha$, and the entire procedure takes polynomial time.

\subsection{Proofs from Section~\ref{sec:beyond}}
\subsection*{Proof of Theorem \ref{tqfo-shl}}

Let $\R$ be some relational signature.  First we address the first condition in the Definition \ref{def:cap}. Let $\alpha$ be a formula in $\tqfo(\fo)$. We will construct a nondeterministic logspace algorithm $M_{\alpha}$ that on input $\enc(\A)$, where a first-order assignment $v$ is being stored in memory, accepts in $\sem{\alpha}(\A)$ paths. Suppose the domain of $\A$ is $A = \{1,\ldots,n\}$. The algorithm needs $c\cdot\log_2(n)$ bits of memory to store $v$, where $c$ is the total number of first-order variables in $\alpha$. If $\alpha = \varphi$, we check if $(\A,v)\models\varphi$ in deterministic logarithmic space, and accept if and only if it does. If $\alpha = s$, we generate $s$ branches and accept in all of them. If $\alpha = (\alpha_1 + \alpha_2)$, we simulate $M_{\alpha_1}$ and $M_{\alpha_2}$ on separate branches. If $\alpha = (\alpha_1\cdot\alpha_2)$, we simulate $\alpha_1$ and if it accepts, instead of doing so, we simulate $\alpha_2$. If $\alpha = \sa{x}\beta$, for each $a\in A$ we generate a different branch where we simulate $M_{\beta}$ while storing $v[a/x]$. If $\alpha = \pa{x}\beta$, we simulate $M_{\beta}$ while storing $v[1/n]$, and on each accepting branch, instead of accepting we replace the assignment on $x$ to 2, to simulate $M_{\beta}$ while storing $v[2/x]$, and so on. If $\alpha = [\pth \varphi(\bar{x},\bar{y})]$ where $\varphi$ is an $\fo$ formula, we simulate the $\shl$ procedure that counts the number of paths for a graph of a given size. This procedure, on each iteration, nondeterministically chooses an assignment $\bar{a}$ for $\bar{x}$, continues if $(\A,v)\models\varphi(\bar{a}',\bar{a})$ where $\bar{a}'$ is the previously chosen value, and rejects otherwise. This is repeated $n^{\length(\bar{x})}$ times, and it generates exactly $\sem{[\pth \varphi(\bar{x},\bar{y})]}(\A,v)$ accepting branches. This ends the construction of the algorithm. Consider $f$ as the $\shl$ function associated to this procedure and we have that for each finite $\R$-structure $\A$: $f(\enc(\A)) = \sem{\alpha}(\A)$.

\vspace{1em}
For the second condition, let $f \in \shl$ defined over some signature $\R$. We will address the case where $\R$ contains only one binary predicate $E$, and the rest of the cases can be deduced from this. Let $M$ be a non-deterministic logspace machine such that $f(\enc(\A)) = \acc_M(\enc(\A))$ for each $\A \in \ostr[\R]$. Suppose ${\cal Q} = \{q_1,\ldots,q_{\ell}\}$ is the set of states of $M$, where $q_1$ is the initial state, and $q_{\ell}$ is only final state of $M$. Let $n = \vert A \vert$ and let $w = \enc(\A) \in \{0,1\}^{n^2}$. We assume that $M$ with input $w$ uses space $s_M(w) < c\cdot\log(n)$ and furthermore, $s_M(w) < n-2$. We notate $M(w)$ as the graph of configurations of $M$ running on input $w$.

We represent configurations with a tuple of fixed size. The formula $\varphi(\bar{x},\bar{y})$ describes a procedure that given a configuration generates a possible next configuration. The formula $\varphi_I(\bar{x})$ describes that $\bar{x}$ is the initial configuration of $M(w)$. The formula $\varphi_F(\bar{x})$ describes that $\bar{x}$ is an accepting (final) configuration of $M(w)$. The formula we construct is:
$$
\alpha = \sa{\bar{x}}\sa{\bar{y}}([\pth \varphi(\bar{x},\bar{y})]\cdot \varphi_I(\bar{x})\cdot\varphi_F(\bar{y})).
$$

To illustrate our idea, we will show a simplified example. Consider a machine $M$ that works in exactly $\log_2(n)$ space and only allows 0 or 1 in the working tape. Consider an input $\A$ of size 16 (that is, $A = \{0,\ldots,9,A,\ldots,F\}$). Let some configuration $s$ have 0011 in the working tape, the head in the input tape is in position 26, and the head in the input tape is in position 2 (we consider 0-indexed positions). Also, $Q = \{q_1,\ldots,q_5\}$ and the current state is $q_3$.

As a first approach, we will use a 9-tuple $\bar{a} = (a_1,\ldots,a_9)$ to represent $s$. That is, $(a_1,a_2) = (1,A)$ represent the position of the head in the input tape (since 1A equal to 26 in base 16), $a_3 = 2$ represents the position of the head in the working tape, $a_4 = C$ (1100b in base 16) represents the content of the working tape, and $(a_5,\ldots,a_9) = (0,0,1,0,0)$ represents the current state. Then $\bar{a} = (1,A,2,C,0,0,1,0,0)$ will represent $s$.

\newcommand\algx{\mathtt{x}}
\newcommand\algy{\mathtt{y}}
\newcommand\algz{\mathtt{z}}
\newcommand\algu{\mathtt{u}}
\newcommand\algv{\mathtt{v}}
\newcommand\algi{\mathtt{i}}
\newcommand\algj{\mathtt{j}}


The problem that arises from this representation, is that to describe a transition in $M$ we need to read an arbitrary character in the working tape. In the example, this translates to obtaining the $a_3$-th bit in $a_4$. Furthermore, to represent the following configuration, we need compute $a_4$ with the $a_3$-th bit flipped. This is generally not possible to describe with an $\fo$ formula. To deal with this issue, consider the following procedure. (In the example it would receive $\algx = a_4$ and $\algi = a_3$.)

\begin{algorithm}
	\caption{If the $\algi$-th bit in $\algx$ is 1 replace it by 0 and return the result}
	\label{switch1to0}
	\begin{algorithmic}
		\State $\algu \gets \algx,\; \algj \gets \algi$ \Comment{Get the $\algi$-th bit on $\algx$ and store it in $\algu$}
		\While{$\algj > 0$}
		\State $\algv \gets 0$
		\While{$\algu > 1$}
		\State $\algu \gets \algu-2,\; \algv \gets \algv+1$
		\EndWhile
		\State $\algu\gets \algv,\; \algj \gets \algj-1$
		\EndWhile
		\While{$\algu > 1$}
		\State $\algu \gets \algu-2$
		\EndWhile
		\State $\textbf{assert } \algu = 1$ \Comment{If $\algu \neq 1$ simply stop}	
		\State $\algy \gets 1$ \Comment{Compute $2^{\algi}$ and store it in $\algy$}
		\While{$\algi > 0$}
		\State $\algz \gets 0$
		\While{$\algy > 0$}
		\State $\algz \gets \algz+2,\; \algy \gets \algy-1$
		\EndWhile
		\State $\algi \gets \algi-1,\; \algy \gets \algz$
		\EndWhile
		\While{$\algy > 0$} \Comment{Subtract $\algy$ from $\algx$}
		\State $\algx \gets \algx-1,\; y \gets \algy-1$
		\EndWhile
		\State \Return $\algx$.
	\end{algorithmic}
\end{algorithm}	
Each of the instructions can be expressed with $\fo$, so our strategy is to use the $\pth$ operator to simulate the algorithm and then we can describe a transition using the processed value of $a_4$. This procedure simulates a transition that writes 1 in the cell where it read a 0. We call this a $1 \to 0$ transition. At the end of the proof we provide in detail three more procedures that simulate a $0\to 0$ transition, a $0\to 1$ transition, and a $1\to 1$ transition. The rest of the proof only addresses the case where we are simulating a $1\to 0$ transition, and the rest of the cases can be described analogously.

We will now describe how to simulate both the procedure and the transition. A procedure tuple $\bar{p} = (a_1,\ldots,a_{3+c+\ell},b_1,b_2,c_1,c_2,c_3,d_1,\ldots,d_{5c+2})$ represents the current configuration of $M(w)$ in $a_1,\ldots,a_{2+c+\ell}$, the values that will be read and written in the working tape in $b_1,b_2$, the instruction pointer in $c_1,c_2,c_3$ and the values stored in memory in $d_1,\ldots,d_{10c+2}$. In detail:
\begin{enumerate}
	\item $a_1,a_2$ and $a_3$ represent the position of the head in the input tape and the working tape, respectively, $a_4,\ldots,a_{3+c}$ represent the content of the working tape and $a_{4+c},\ldots,a_{3+c+\ell}$ represent the current state in the current configuration that is being processed.
	\item $b_1$ and $b_2$ are equal to the value that is being read in the working tape and the value that will be written in the working tape respectively. These values also indicate which algorithm is being simulated.
	\item $c_1,c_2,c_2$ represent the instruction pointer in the procedure. Only 8 different instructions are needed in the simulation.
	\item The variables $\algx,\algy,\algz,\algu,\algv$ need $c$ elements each to be represented and $\algi,\algj$ need only one. We map $(d_1\ldots,d_{c}) \to \algx$, $(d_{c+1}\ldots,d_{2c}) \to \algy$,
	$(d_{2c+1}\ldots,d_{3c}) \to \algz$, $(d_{3c+1}\ldots,d_{4c}) \to \algu$,
	$(d_{4c+1}\ldots,d_{5c}) \to \algv$, $d_{5c+1} \to \algi$ and $d_{5c+2}\to \algj$.
\end{enumerate}
For each transition $\delta \in \Delta \subseteq Q \times \{0,1\} \times \{0,1\} \times Q \times \{-1,=,+1\} \times \{0,1\} \times \{-1,=,+1\}$ we define a formula $\varphi_{\delta}(\bar{x},\bar{s},\bar{w},\bar{u},\bar{y},\bar{t},\bar{z},\bar{v})$, where $\bar{x} = (x_1,\ldots,x_{3+c+\ell})$, $\bar{s} = (s_1,s_2)$, $\bar{w} = (w_1,w_2,w_3)$, $\bar{u} = (u_1,\ldots,u_{5c+2})$, $\bar{y} = (y_1,\ldots,y_{3+c+\ell})$, $\bar{t} = (t_1,t_2)$, $\bar{z} = (z_1,z_2,z_3)$ and $\bar{v} = (v_1,\ldots,v_{5c+2})$. The tuples $\bar{x}$ and $\bar{y}$ represent the current and next configuration of $M$ respectively, $\bar{s}$ and $\bar{t}$ indicate which algorithm is being simulated, $\bar{w}$ and $\bar{z}$ represent the current and next instruction of the algorithm, $\bar{u}$ and $\bar{v}$ represent the current and next values in memory. We will describe the formula part by part. Suppose $\delta = (q_i,a,1,q_j,op_1,0,op_2)$, so we have to simulate Algorithm \ref{switch1to0}.

To significantly improve the readability of the construction, we define the following tuples:
\begin{equation*}
\begin{aligned}
	\bar{x}_{\text{h-in}} &= (x_1,x_2), \\
	x_{\text{h-w}} &= x_3, \\
	\bar{x}_{\text{tape}} &= (x_4,\ldots,x_{3+c}),\\
	\bar{x}_{\text{state}} &= (x_{4+c},\ldots,x_{3+c+\ell}),\\
	\bar{u}_{\algx} &= (u_1,\ldots,u_c),\\
	\bar{u}_{\algy} &= (u_{c+1},\ldots,u_{2c}),\\
	\bar{u}_{\algz} &= (u_{2c+1},\ldots,u_{3c}),\\
	\bar{u}_{\algu} &= (u_{3c+1},\ldots,u_{4c}),\\
	\bar{u}_{\algv} &= (u_{4c+1},\ldots,u_{5c}),\\
	u_{\algi} &= u_{5c+1},\\
	u_{\algj} &= u_{5c+2},
\end{aligned}
\hspace{1em}
\begin{aligned}
	\bar{y}_{\text{h-in}} &= (y_1,y_2), \\
	y_{\text{h-w}} &= y_3, \\
	\bar{y}_{\text{tape}} &= (y_4,\ldots,y_{3+c}),\\
	\bar{x}_{\text{state}} &= (x_{4+c},\ldots,x_{3+c+\ell}),\\
	\bar{v}_{\algx} &= (v_1,\ldots,v_c),\\
	\bar{v}_{\algy} &= (v_{c+1},\ldots,v_{2c}),\\
	\bar{v}_{\algz} &= (v_{2c+1},\ldots,v_{3c}),\\
	\bar{v}_{\algu} &= (v_{3c+1},\ldots,v_{4c}),\\
	\bar{v}_{\algv} &= (v_{4c+1},\ldots,v_{5c}),\\
	v_{\algi} &= v_{5c+1},\\
	v_{\algj} &= v_{5c+2}.
\end{aligned}
\end{equation*}
We also define some auxiliary formulas:
\begin{align*}
\gamma_{0}(\bar{x}) &= \neg\exists\bar{y}(\bar{y}<\bar{x}),\\
\gamma_{1}(\bar{x}) &= \exists\bar{y}(\gamma_{0}(\bar{y})\wedge \bar{y} < \bar{x} \wedge \neg\exists\bar{z}(\bar{y}<\bar{z}\wedge\bar{z}<\bar{x}))\\
\gamma_{+1}(\bar{x},\bar{y}) &= \bar{x} < \bar{y} \wedge \neg\exists \bar{z}(\bar{x}<\bar{z} \wedge \bar{z}<\bar{y}), \\
\gamma_{-1}(\bar{x},\bar{y}) &= \gamma_{+1}(\bar{y},\bar{x}),\\
\gamma_{=}(\bar{x},\bar{y}) &= \bar{x} = \bar{y} \\
\gamma_{+2}(\bar{x},\bar{y}) &= \exists\bar{z}(\gamma_{+1}(\bar{x},\bar{z}) \wedge \gamma_{+1}(\bar{z},\bar{y})),\\
\gamma_{-2}(\bar{x},\bar{y}) &= \gamma_{+2}(\bar{y},\bar{x}),\\
\gamma_{i,j}(x,y) &= \gamma_i(x) \wedge \gamma_j(y),\text{ for $i,j \in\{0,1\}$}\\
\varphi^b_k(x_1,x_2,x_3) &= \gamma_{a_1}(x_1) \wedge \gamma_{a_2}(x_2) \wedge \gamma_{a_3}(x_3),\text{ for each $k \leq 7$, where $a_1a_2a_3$ is the value of $k$ in binary}, \\
\varphi^q_i(x_1,\ldots,x_{\ell}) &= \gamma_0(x_1) \wedge \cdots \wedge \gamma_0(x_{i-1}) \wedge \gamma_1(x_i) \wedge \gamma_0(x_{i+1}) \wedge \cdots \wedge \gamma_0(x_{\ell}), \text{ for each } q_i\in Q\\
\varphi^E_0(x_1,x_2) &= \neg E(x_1,x_2),\\		\varphi^E_1(x_1,x_2) &= E(x_1,x_2),\\
\end{align*}

We start from instruction 0, which means that the procedure has not started yet and every value in the tuple is 0 except for the configuration values. It also initializes all the values in the tuple to 0 except for $\algx,\algu,\algi,\algj$.
\begin{align*}
\varphi^{0,1}_{\delta}(\bar{x},\bar{s},\bar{w},\bar{u},\bar{y},\bar{t},\bar{z},\bar{v}) = \,&\gamma_{0,0}(\bar{s})\wedge\varphi^b_0(\bar{w}) \wedge \\ &\gamma_{1,0}(\bar{t}) \wedge \varphi^b_1(\bar{z}) \wedge 
\bar{v}_{\algx} = \bar{x}_{\text{tape}} \wedge 
\gamma_0(\bar{v}_{\algy}) \wedge 
\gamma_0(\bar{v}_{\algz}) \wedge 
\bar{v}_{\algu} = \bar{x}_{\text{tape}} \wedge 
\gamma_0(\bar{v}_{\algv}) \wedge 
v_{\algi} = x_{\text{h-w}} \wedge v_{\algj} = x_{\text{h-w}}.
\end{align*}
Instruction 1 which checks whether the value of $\algj$ is more than 0 or not, and then proceeds to instruction 2 or 3 on each case.
\begin{align*}
\varphi^{1,2}_{\delta}(\bar{x},\bar{s},\bar{w},\bar{u},\bar{y},\bar{t},\bar{z},\bar{v}) = 
\,&\gamma_{1,0}(\bar{s}) \wedge \varphi^b_1(\bar{w}) \wedge \neg \gamma_0(u_{\algj}) \wedge \\ &\gamma_{1,0}(\bar{t}) \wedge \varphi^b_2(\bar{z}) \wedge \bar{u} = \bar{v}, \\
\varphi^{1,3}_{\delta}(\bar{x},\bar{s},\bar{w},\bar{u},\bar{y},\bar{t},\bar{z},\bar{v}) = 
\,&\gamma_{1,0}(\bar{s}) \wedge \varphi^b_1(\bar{w}) \wedge \gamma_0(u_{\algj}) \wedge \\ &\gamma_{1,0}(\bar{t}) \wedge \varphi^b_3(\bar{z}) \wedge \bar{u} = \bar{v}.
\end{align*}
Instruction 2 checks the value of $\algu$. If it is $> 1$ then it subtracts 2 from $\algu$ and adds 1 to $\algv$, then repeats instruction 2. If it is equal to 0 or 1, then moves the value of $\algv$ to $\algu$, subtracts 1 from $\algj$ and goes back to instruction 1.
\begin{align*}
\varphi^{2,2}_{\delta}(\bar{x},\bar{s},\bar{w},\bar{u},\bar{y},\bar{t},\bar{z},\bar{v}) = 
\,&\gamma_{1,0}(\bar{s}) \wedge \varphi^b_2(\bar{w}) \wedge \neg \gamma_0(\bar{u}_{\algu}) \wedge \neg \gamma_1(\bar{u}_{\algu}) \wedge \\ &\gamma_{1,0}(\bar{t}) \wedge \varphi^b_2(\bar{z}) \wedge
\bar{u}_{\algx} = \bar{v}_{\algx} \wedge
\bar{u}_{\algy} = \bar{v}_{\algy} \wedge
\bar{u}_{\algz} = \bar{v}_{\algz} \wedge
\gamma_{-2}(\bar{u}_{\algu},\bar{v}_{\algu}) \wedge
\gamma_{+1}(\bar{u}_{\algv},\bar{v}_{\algv}) \wedge u_{\algi} = v_{\algi} \wedge u_{\algj} = v_{\algj},\\
\varphi^{2,1}_{\delta}(\bar{x},\bar{s},\bar{w},\bar{u},\bar{y},\bar{t},\bar{z},\bar{v}) = \,&\gamma_{1,0}(\bar{s}) \wedge \varphi^b_2(\bar{w}) \wedge ( \gamma_0(\bar{u}_{\algu}) \vee \gamma_1(\bar{u}_{\algu})) \wedge \\ 
&\gamma_{1,0}(\bar{t}) \wedge \varphi^b_1(\bar{z}) \wedge
\bar{u}_{\algx} = \bar{v}_{\algx} \wedge
\bar{u}_{\algy} = \bar{v}_{\algy} \wedge
\bar{u}_{\algz} = \bar{v}_{\algz} \wedge
\bar{u}_{\algv} = \bar{v}_{\algu} \wedge
\gamma_0(\bar{v}_{\algv}) \wedge
u_{\algi} = v_{\algi} \wedge \gamma_{-1}(u_{\algj},v_{\algj}).	
\end{align*}
Instruction 3 calculates the value of $\algu\mod 2$, that is, it repeats instruction 3 until the value of $\algu$ is equal to 0 or 1. On each iteration, it subtracts 2 from $\algu$. Moreover, if the value of $\algu$ at the end of the iterations is not 1 then there is no step defined.
\begin{align*}
\varphi^{3,3}_{\delta}(\bar{x},\bar{s},\bar{w},\bar{u},\bar{y},\bar{t},\bar{z},\bar{v}) = \,&\gamma_{1,0}(\bar{s}) \wedge \varphi^b_3(\bar{w}) \wedge \neg \gamma_0(\bar{u}_{\algu}) \wedge \neg \gamma_1(\bar{u}_{\algu}) \wedge \\ &\gamma_{1,0}(\bar{t}) \wedge 	\varphi^b_3(\bar{z}) \wedge
\bar{u}_{\algx} = \bar{v}_{\algx} \wedge
\bar{u}_{\algy} = \bar{v}_{\algy} \wedge
\bar{u}_{\algz} = \bar{v}_{\algz} \wedge
\gamma_{-2}(\bar{u}_{\algu},\bar{v}_{\algu}) \wedge
\bar{u}_{\algv} = \bar{v}_{\algv} \wedge
u_{\algi} = v_{\algi} \wedge u_{\algj} = v_{\algj},\\
\varphi^{3,3}_{\delta}(\bar{x},\bar{s},\bar{w},\bar{u},\bar{y},\bar{t},\bar{z},\bar{v}) = 
\,&\gamma_{1,0}(\bar{s}) \wedge \varphi^b_3(\bar{w}) \wedge \gamma_1(\bar{u}_{\algu}) \wedge\\ &\gamma_{1,0}(\bar{t}) \wedge 	\varphi^b_4(\bar{z}) \wedge
\bar{u}_{\algx} = \bar{v}_{\algx} \wedge
\gamma_1(\bar{v}_{\algy}) \wedge
\bar{u}_{\algz} = \bar{v}_{\algz} \wedge
\bar{u}_{\algu} = \bar{v}_{\algu} \wedge
\bar{u}_{\algv} = \bar{v}_{\algv} \wedge
u_{\algi} = v_{\algi} \wedge u_{\algj} = v_{\algj}
\end{align*}
Instruction 4 checks the value of $\algi$. If it is not 0 then goes to instruction 5 and if is 0 then goes to instruction 6. Moreover it initializes the value of $\algz$ to 0 (which was 0 all along.)
\begin{align*}
\varphi^{4,5}_{\delta}(\bar{x},\bar{s},\bar{w},\bar{u},\bar{y},\bar{t},\bar{z},\bar{v}) = \,&\gamma_{1,0}(\bar{s}) \wedge \varphi^b_4(\bar{w}) \wedge \neg \gamma_0(u_{\algi}) \wedge \\ &\gamma_{1,0}(\bar{t}) \wedge 	\varphi^b_5(\bar{z}) \wedge \bar{u} = \bar{v}, \\
\varphi^{4,6}_{\delta}(\bar{x},\bar{s},\bar{w},\bar{u},\bar{y},\bar{t},\bar{z},\bar{v}) = &\gamma_{1,0}(\bar{s}) \wedge \varphi^b_4(\bar{w}) \wedge \gamma_0(u_{\algi}) \wedge \\ &\gamma_{1,0}(\bar{t}) \wedge 	\varphi^b_6(\bar{z}) \wedge \bar{u} = \bar{v}.
\end{align*}
Instruction 5 checks the value of $\algy$. If it is more than 0 then it adds 2 to $\algz$ and subtracts 1 from $\algy$, then repeats instruction 2. If it is not, then copies the value of $\algz$ to $\algy$ and subtracts 1 from $\algi$ and returns to instruction 4.
\begin{align*}
\varphi^{5,5}_{\delta}(\bar{x},\bar{s},\bar{w},\bar{u},\bar{y},\bar{t},\bar{z},\bar{v}) = \,&\gamma_{1,0}(\bar{s}) \wedge \varphi^b_5(\bar{w}) \wedge \neg \gamma_0(\bar{u}_{\algy}) \wedge \\ &\gamma_{1,0}(\bar{t}) \wedge \varphi^b_5(\bar{z}) \wedge
\bar{u}_{\algx} = \bar{v}_{\algx} \wedge
\gamma_{-1}(\bar{u}_{\algy},\bar{v}_{\algy}) \wedge
\gamma_{+2}(\bar{u}_{\algz},\bar{v}_{\algz})\, \wedge \bar{u}_{\algu} = \bar{v}_{\algu} \wedge
\bar{u}_{\algv} = \bar{v}_{\algv} \wedge
u_{\algi} = v_{\algi} \wedge u_{\algj} = v_{\algj},\\
\varphi^{5,4}_{\delta}(\bar{x},\bar{s},\bar{w},\bar{u},\bar{y},\bar{t},\bar{z},\bar{v}) = \,&\gamma_{1,0}(\bar{s}) \wedge \varphi^b_5(\bar{w}) \wedge \gamma_0(\bar{u}_{\algy}) \wedge \\ &\gamma_{1,0}(\bar{t}) \wedge \varphi^b_4(\bar{z}) \wedge
\bar{u}_{\algx} = \bar{v}_{\algx} \wedge
\bar{u}_{\algz} = \bar{v}_{\algy} \wedge
\gamma_0(\bar{v}_{\algz}) \wedge
\bar{u}_{\algu} = \bar{v}_{\algu} \wedge
\bar{u}_{\algv} = \bar{v}_{\algv} \wedge
\gamma_{-1}(u_{\algi},v_{\algi}) \wedge u_{\algj} = v_{\algj}
\end{align*}
Instruction 6 checks the value of $\algy$. If it is more than 0, then subtracts 1 from $\algx$ and $\algy$ and repeats instruction 6. If it is not, then goes to instruction 7.
\begin{align*}
\varphi^{6,6}_{\delta}(\bar{x},\bar{s},\bar{w},\bar{u},\bar{y},\bar{t},\bar{z},\bar{v}) = \,&\gamma_{1,0}(\bar{s}) \wedge \varphi^b_6(\bar{w}) \wedge \neg \gamma_0(\bar{u}_{\algy}) \wedge \\ &\gamma_{1,0}(\bar{t}) \wedge \varphi^b_6(\bar{z}) \wedge
\gamma_{-1}(\bar{u}_{\algx},\bar{v}_{\algx}) \wedge \gamma_{-1}(\bar{u}_{\algy},\bar{v}_{\algy}) \wedge \bar{u}_{\algu} = \bar{v}_{\algu} \wedge \bar{u}_{\algv} = \bar{v}_{\algv} \\
\varphi^{6,7}_{\delta}(\bar{x},\bar{s},\bar{w},\bar{u},\bar{y},\bar{t},\bar{z},\bar{v}) = \,&\gamma_{1,0}(\bar{s}) \wedge \varphi^b_6(\bar{w}) \wedge \gamma_0(\bar{u}_{\algy}) \wedge \\ &\gamma_{1,0}(\bar{t}) \wedge \varphi^b_7(\bar{z}) \wedge \bar{u} = \bar{v}.
\end{align*}
Instruction 7 stores the value of $\algx$ after the corresponding bit has been switched. Then we can define $\gamma_{\delta}$ which also simulates the actual transition. If $\algu$ equals 1, then copy what is stored in $\algx$ to $a_4,\ldots,a_{3+c}$, go from state $q_i$ to state $q_j$, and move the heads to their corresponding positions. Recall that $op_1,op_2\in\{+1,=,-1\}$
\begin{align*}
\gamma_{\delta}(\bar{x},\bar{s},\bar{w},\bar{u},\bar{y},\bar{t},\bar{z},\bar{v}) = 	[&\bar{x} = \bar{y} \wedge (\varphi^{0,1}(\bar{x},\bar{s},\bar{w},\bar{u},\bar{y},\bar{t},\bar{z},\bar{v}) \vee \\ &\varphi^{1,2}(\bar{x},\bar{s},\bar{w},\bar{u},\bar{y},\bar{t},\bar{z},\bar{v}) \vee \varphi^{1,3}(\bar{x},\bar{s},\bar{w},\bar{u},\bar{y},\bar{t},\bar{z},\bar{v}) \vee \\ &\varphi^{2,2}(\bar{x},\bar{s},\bar{w},\bar{u},\bar{y},\bar{t},\bar{z},\bar{v}) \vee \varphi^{2,1}(\bar{x},\bar{s},\bar{w},\bar{u},\bar{y},\bar{t},\bar{z},\bar{v}) \vee \\ &\varphi^{3,3}(\bar{x},\bar{s},\bar{w},\bar{u},\bar{y},\bar{t},\bar{z},\bar{v}) \vee \varphi^{3,4}(\bar{x},\bar{s},\bar{w},\bar{u},\bar{y},\bar{t},\bar{z},\bar{v}) \vee \\ &\varphi^{4,5}(\bar{x},\bar{s},\bar{w},\bar{u},\bar{y},\bar{t},\bar{z},\bar{v}) \vee \varphi^{4,6}(\bar{x},\bar{s},\bar{w},\bar{u},\bar{y},\bar{t},\bar{z},\bar{v}) \vee \\ &\varphi^{5,5}(\bar{x},\bar{s},\bar{w},\bar{u},\bar{y},\bar{t},\bar{z},\bar{v}) \vee  \varphi^{5,6}(\bar{x},\bar{s},\bar{w},\bar{u},\bar{y},\bar{t},\bar{z},\bar{v}) \vee \\ &\varphi^{6,6}(\bar{x},\bar{s},\bar{w},\bar{u},\bar{y},\bar{t},\bar{z},\bar{v}) \vee \varphi^{6,7}(\bar{x},\bar{s},\bar{w},\bar{u},\bar{y},\bar{t},\bar{z},\bar{v}))] \, \vee \\
[\varphi^E_a(\bar{x}_{\text{h-in}}) \wedge &\gamma_{1,0}(\bar{s}) \wedge \varphi^b_7(\bar{w}) \wedge \gamma_1(\bar{u}_{\algu}) \wedge \\ &\gamma_{0,0}(\bar{t}) \wedge \varphi^b_0(\bar{z}) \wedge \bar{u} = \bar{v} \wedge
\gamma_{op_1}(\bar{x}_{\text{h-in}},\bar{y}_{\text{h-in}}) \wedge \gamma_{op_2}(x_{\text{h-w}},y_{\text{h-w}}) \wedge \bar{y}_{\text{tape}} = \bar{u}_{\algx} \wedge \varphi^q_i(\bar{x}_{\text{state}}) \wedge \varphi^q_j(\bar{y}_{\text{state}})].
\end{align*}

Note that we also need to specify that the program we are following is Algorithm \ref{switch1to0} so we store $1,0$ in $b_1,b_2$ all along the procedure. We describe the three other algorithms that compute the switches from $0\to 0$, $0\to 1$ and $1\to 1$ (Algorithms \ref{switch0to0}, \ref{switch0to1} and \ref{switch1to1}.)
For the other three cases, where $\delta = (q_i,a,0,q_j,op_1,0,op_2)$, $\delta = (q_i,a,0,q_j,op_1,1,op_2)$ and $\delta = (q_i,a,1,q_j,op_1,1,op_2)$, $\gamma_{\delta}(\bar{x},\bar{s},\bar{w},\bar{u},\bar{y},\bar{t},\bar{z},\bar{v})$ is defined analogously. Then, $\varphi(\bar{x},\bar{s},\bar{w},\bar{u},\bar{y},\bar{t},\bar{z},\bar{v})$ is defined as:
$$
\varphi(\bar{x},\bar{s},\bar{w},\bar{u},\bar{y},\bar{t},\bar{z},\bar{v}) = \bigvee_{\delta \in \Delta} \gamma_{\delta}(\bar{x},\bar{s},\bar{w},\bar{u},\bar{y},\bar{t},\bar{z},\bar{v}).
$$
Lastly we define $\varphi_I$ and $\varphi_F$:
\begin{align*}
\varphi_I(\bar{x},\bar{s},\bar{w},\bar{u}) &= \gamma_0(\bar{x}_{\text{Head-1}}) \wedge \gamma_0(x_{\text{Head-2}}) \wedge \gamma_0(\bar{x}_{\text{Tape}}) \wedge \varphi^{q}_1(\bar{x}_{\text{State}})\wedge \gamma_{0,0}(\bar{s})\wedge \varphi^b_0(\bar{w}) \wedge\gamma_0(\bar{u}). \\
\varphi_F(\bar{x},\bar{s},\bar{w},\bar{u}) &= \varphi^q_{\ell}(\bar{x}_{\text{State}}) \wedge \gamma_{0,0}(\bar{s}) \wedge \varphi^b_0(\bar{w}) \wedge\gamma_0(\bar{u}),
\end{align*}
and then $\sem{\alpha}(\A) = \sem{\sa{\bar{x}}\sa{\bar{y}}([\pth \varphi(\bar{x},\bar{y})]\cdot \varphi_I(\bar{x})\cdot\varphi_F(\bar{y}))}(\A) = \acc_M(\A)$.

\begin{algorithm}
	\caption{If the $i$-th bit in $x$ is 0 return $x$} \label{switch0to0}
	\begin{algorithmic}
		\State $u \gets x,\; j \gets i$ \Comment{Get the $i$-th bit on $x$ and store it in $u$}
		\While{$j > 0$}
		\State $v \gets 0$
		\While{$u > 1$}
		\State $u \gets u-2,\; v \gets v+1$
		\EndWhile
		\State $u\gets v,\; j \gets j-1$
		\EndWhile
		\While{$u > 1$}
		\State $u \gets u-2$
		\EndWhile
		\State $\textbf{assert } u = 0$ \Comment{If $u \neq 0$ simply stop}	
		\State \Return $x$.
	\end{algorithmic}
\end{algorithm}

\begin{algorithm}
	\caption{If the $i$-th bit in $x$ is 0 replace it by 1 and return the result}
	\label{switch0to1}
	\begin{algorithmic}
		\State $u \gets x,\; j \gets i$ \Comment{Get the $i$-th bit on $x$ and store it in $u$}
		\While{$j > 0$}
		\State $v \gets 0$
		\While{$u > 1$}
		\State $u \gets u-2,\; v \gets v+1$
		\EndWhile
		\State $u\gets v,\; j \gets j-1$
		\EndWhile
		\While{$u > 1$}
		\State $u \gets u-2$
		\EndWhile
		\State $\textbf{assert } u = 0$ \Comment{If $u \neq 0$ simply stop}	
		\State $y \gets 1$ \Comment{Compute $2^i$ and store it in $y$}
		\While{$i > 0$}
		\State $z \gets 0$
		\While{$y > 0$}
		\State $z \gets z+2,\; y \gets y-1$
		\EndWhile
		\State $i \gets i-1,\; y \gets z$
		\EndWhile
		\While{$y > 0$} \Comment{Add $y$ to $x$}
		\State $x \gets x+1,\; y \gets y-1$
		\EndWhile
		\State \Return $x$.
	\end{algorithmic}
\end{algorithm}	

\begin{algorithm}
	\caption{If the $i$-th bit in $x$ is 1 return $x$}
	\label{switch1to1}
	\begin{algorithmic}
		\State $u \gets x,\; j \gets i$ \Comment{Get the $i$-th bit on $x$ and store it in $u$}
		\While{$j > 0$}
		\State $v \gets 0$
		\While{$u > 1$}
		\State $u \gets u-2,\; v \gets v+1$
		\EndWhile
		\State $u\gets v,\; j \gets j-1$
		\EndWhile
		\While{$u > 1$}
		\State $u \gets u-2$
		\EndWhile
		\State $\textbf{assert } u = 1$ \Comment{If $u \neq 0$ simply stop}	
		\State \Return $x$.
	\end{algorithmic}
\end{algorithm}










\subsection*{Proof of Theorem \ref{tqso-fo-fpsace}}

We separate the proof in two parts. Let $\R$ be a relational signature. First we prove that for every formula $\alpha$ in $\tqso$ over $\R$ there exists a function $f\in\shpspace$ such that $\sem{\alpha}(\A) = f(\enc(\A))$ for every $\A\in\ostr[\R]$. Then we prove that for every function $f\in \fpspace$ over $\R$ there exists a $\tqso(\fo)$ formula $\alpha$ such that $f(\enc(\A)) = \sem{\alpha}(\A)$ for every $\A\in\ostr[\R]$. By the inclusion of $\tqso(\fo)\subseteq\tqso$ and the equality $\shpspace = \fpspace$, this proves that both $\tqso$ and $\tqso(\fo)$ capture $\fpspace$ over ordered structures.

\vspace{1em}
For the first part, let $\alpha$ be a formula in $\tqso$ over $\R$. 
We will construct a nondeterministic polynomial-space algorithm $M_{\alpha}$ that on input $(\enc(\A),\enc(v),\enc(V))$, accepts in $\sem{\alpha}(\A,v,V)$ paths, for each $(\A,v,V)\in\ostr[\R]^*$. Let $A = \{1,\ldots,n\}$  be the domain of $\A$. 
First-order assignments are encoded as a simple mapping from every first-order variable mentioned in $\alpha$ to an element in $A$. 
Second order assignments are encoded in polynomial space as a mapping from every second-order variable $X$ to a subset of $A^{\arity(X)}$. We now begin the construction of the algorithm. 
If $\alpha = \varphi$, a $\so$ formula, we check if $(\A,v,V)\models\varphi$ in deterministic polynomial space, and accept if and only if it holds. 
If $\alpha = s$, we generate $s$ branches and accept in all of them. 
If $\alpha = (\alpha_1 + \alpha_2)$, we simulate $M_{\alpha_1}$ and $M_{\alpha_2}$, both on input $(A,v,V)$, on separate branches. 
If $\alpha = (\alpha_1\cdot\alpha_2)$, we simulate $\alpha_1$ on input $(A,v,V)$ and if it accepts, instead of doing so, we simulate $\alpha_2$ on input $(A,v,V)$. 
If $\alpha = \sa{x}\beta$, for each $a\in A$ we generate a different branch where we simulate $M_{\beta}$ on input $(A,v[a/x],V)$.
If $\alpha = \pa{x}\beta$, we simulate $M_{\beta}$ on input $(A,v[1/x],V)$, and on each accepting branch, instead of accepting we simulate $M_{\beta}$ on input $(A,v[2/x],V)$, and so on. 
If $\alpha = \sa{X}\beta$, for each $B\subseteq A^{\arity(X)}$ we generate a different branch where we simulate $M_{\beta}$ on input $(A,v,V[B/X])$.
If $\alpha = \pa{X}\beta$, we simulate $M_{\beta}$ on input $(\A,v,V[B/X])$ consecutively for each $B\subseteq A^{\arity(X)}$. 
If $\alpha = [\pth \varphi(\bar{x},\bar{X},\bar{y},\bar{Y})]$ where $\varphi$ is an $\so$ formula, we simulate the procedure that counts the number of paths for a graph of a given size. This procedure, on each iteration, nondeterministically chooses an assignment $(\bar{a},\bar{B})$ for $(\bar{x},\bar{X})$, and checks in polynomial space if $(\A,v,V)\models\varphi(\bar{a}',\bar{B}',\bar{a},\bar{B})$, where $(\bar{a}',\bar{B}')$ is the previously chosen value. If it holds, we continue, and otherwise we reject. 
This is repeated $n^{\length(\bar{x})}\cdot \prod_{X\in\bar{X}} 2^{A^{\arity(X)}}$ times, and it generates exactly $\sem{[\pth \varphi(\bar{x},\bar{X},\bar{y},\bar{Y})]}(\A,v,V)$ accepting branches. 
This ends the construction of the algorithm. 
Consider $f$ as the $\shpspace$ function associated to this procedure and we have that for each finite $\R$-structure $\A$: $f(\enc(\A)) = \sem{\alpha}(\A)$.

\vspace{1em}
For the second part 


\end{document}


