%!TEX root = main.tex
\newcommand{\pP}{\textit{P}}
\newcommand{\pN}{\textit{N}}
\newcommand{\pV}{\textit{V}}
\newcommand{\pT}{\textit{T}}
\newcommand{\pA}{\textit{A}}
\newcommand{\pNC}{\textit{NC}}
\newcommand{\pD}{\textit{D}}


The previous result shows that the classes $\QE{i}$ and $\QU{i}$ are more robust than the classes $\E{i}$ and $\U{i}$: they are closed under binary and sum quantifiers but the other not necessarily. 

Now, we study the complexity classes describe by this hierarchies. As the following result shows, $\eqso(\loge{0})$ defines only tractable counting functions and $\eqso(\loge{1})$ intractable counting functions but with an tractable decision problems. 
\begin{proposition} \label{prop:qe0-fp-qe1-totp-fptras}
All functions defined in $\eqso(\loge{0})$ and $\eqso(\loge{1})$ can be computed in $\fp$ and $\totp$, respectively. Furthermore, every function defined in $\eqso(\loge{1})$ has a FPTRAS.
\end{proposition}
Therefore, in terms of counting complexity, the $\eqso$-hierarchy behaves exactly the same as the $\#\fo$-hierarchy.

The next step is to study the closure properties of $\eqso$-hierarchy. 
An advantage of the $\eqso$-hierarchy is that, by its language syntax, all the classes are closed under addition and first and second order sum.
So, the first question is whether the multiplicative operators in $\qso$ can be defined in $\eqso(\LL)$. As the following result shows, if $\LL$ is closed under conjunction, then the binary product can be defined in  $\eqso(\LL)$.
\begin{theorem}\label{theo:binary-prod}
	If $\LL$ is closed under conjunction, then binary product can be defined in $\eqso(\LL)$.
\end{theorem}
The next question is whether the hierarchy is closed under subtraction. Formally, for any pair of functions $f,g$, we define $f - g$ as the function such that $(f - g)(\A) = f(\A)-g(\A)$ whenever $f(\A)>g(\A)$ and $0$ otherwise.
As the next result shows, all classes in the $\eqso$-hierarchy is not closed under subtraction unless ${\sc P} = {\sc NP}$
\begin{theorem} \label{sub-pnp}
If $\eqso(\loge{i})$ or $\eqso(\logu{i})$ is closed under subtraction for $i > 0$, then {\sc P} = {\sc NP}.
\end{theorem}
\cristian{Martin, el resultado que tienes en el apendice se generaliza trivialmente para todas las clases ya que todas contienen la clase $\eqso(\loge{0})$.}

By the previous result, we know that functions in the $\eqso$ hierarchy are unlikely to be closed under subtraction. Then, a natural restriction to this question is to ask whether these classes are closed under subtraction by one, namely, if $\CC$ is a class of functions and $f \in \CC$, is $f-1 \in \CC$ where $1$ is the constant function that outputs $1$ for every structure. 
We do not know $\E{1}$ is closed under subtraction by one. However, if we extend $\logex{1}$ with $\fo$ predicates we can show that this new fragment is closed under subtraction by one.
\begin{theorem} \label{sigmafo-minusone}
	$\eqso(\logex{1})$ is closed under substraction by one.
\end{theorem}


A positive literal is a formula of the form $X(\x)$, where $X$ is a second-order variable and $\x$ is a tuple of first-order variables, and a negative literal is a formula of the form $\exists \v \, \neg X(\u,\v)$, where $\u$ and $\v$ are tuples of first-order variables. Given a relational signature $\R$, a clause over $\R$ is a formula of the form:
$$
\forall \x \, (\varphi_1 \vee \cdots \vee \varphi_n),
$$
where each $\varphi_i$ ($1 \leq i \leq n$) is either a positive literal, a negative literal or an \fo-formula over $\R$.  A clause is said to be Horn if it contains at most one positive literal, and a formula is said to be Horn if it is a conjunction of Horn clauses over a relational signature $\R$. With this terminology, we define $\uhorn$ as the set of formulas $\psi$ such that $\psi$ is a conjunction of Horn clauses over a relational signature $\R$. 

\begin{proposition}\label{prop:uhorn-pe}
$\eqso(\uhorn) \subseteq \totp$
\end{proposition}

\begin{example} \label{ex-hornsat-esop1}
Let $\R = \{\pP(\cdot,\cdot), \pN(\cdot,\cdot), \pV(\cdot), \pNC(\cdot)\}$. This vocabulary is used as follows to encode a Horn formula. A fact $\pP(c,x)$ indicates that propositional variable $x$ is a disjunct in a clause $c$, while $\pN(c,x)$ indicates that $\neg x$ is a disjunct in $c$. Furthermore, $\pV(x)$ holds if  $x$ is a propositional variable, and $\pNC(c)$ holds if $c$ is a clause containing only negative literals, that is, $c$ is of the form $(\neg x_1 \vee \cdots \vee \neg x_n)$.

To encode $\chsat$, we define an \so-formula $\varphi(\pT)$ over $\R$, where $\pT$ is a unary predicate, such that for every Horn formula $\theta$ encoded by an $\R$-structure $\A$, the number of satisfying assignments of $\theta$ is equal to $\sem{\sa{\pT} \varphi(\pT)}(\A)$. In particular, $\pT(x)$ holds if and only if $x$ is a propositional variable that is assigned value 1.  More specifically, $\varphi(\pT)$ is defined as follows:
\begin{align*}
&\forall x \, (\pT(x) \to \pV(x)) \ \wedge\\
&\forall c \, (\pNC(c) \to \exists x \, (\pN(c,x) \wedge \neg \pT(x))) \ \wedge\\
&\forall c \forall x \, ([\pP(c,x) \wedge \forall y \, (\pN(c,y) \to \pT(y))] \to \pT(x)).
\end{align*}
Given that $\uhorn$ is designed with the goal in mind of capturing $\chsat$, we expect $\varphi(\pT)$ to be a formula in $\uhorn$. However, if we rewrite it as a conjunction of clauses we obtain the following:
\begin{align*}
&\forall x \, (\neg \pT(x) \vee \pV(x)) \ \wedge\\
&\forall c \, (\neg \pNC(c) \vee \exists x \, (\pN(c,x) \wedge \neg \pT(x)))\ \wedge\\
&\forall c \forall x \, (\neg \pP(c,x) \vee \exists y \, (\pN(c,y) \wedge \neg \pT(y)) \vee \pT(x)).
\end{align*}
The resulting formula $\varphi(\pT)$ is not in $\uhorn$, but it can be easily transformed into a formula in this class  by introducing an auxiliary binary predicate $\pA$ defined as follows:
\begin{align*}
\forall c \forall x \, (\neg \pA(c,x) \leftrightarrow [\pN(c,x) \wedge \neg \pT(x)]).
\end{align*}
In this way, we obtain the following formula $\psi(\pT,\pA)$ in $\uhorn$:
\begin{align*}
&\forall x \, (\neg \pT(x) \vee \pV(x)) \ \wedge\\
&\forall c \, (\neg \textit{NC}(c) \vee \exists x \, \neg \textit{A}(c,x)) \ \wedge\\
&\forall c \forall x \, (\neg \textit{P}(c,x) \vee \exists y \, \neg \textit{A}(c,y) \vee \textit{T}(x)) \ \wedge\\
&\forall c \forall x \, (\neg \textit{N}(c,x) \vee \textit{T}(x) \vee \neg \textit{A}(c,x)) \ \wedge\\
&\forall c \forall x \, (\textit{A}(c,x) \vee \textit{N}(c,x)) \ \wedge\\
&\forall c \forall x \, (\textit{A}(c,x) \vee \neg\textit{T}(x)).
\end{align*}
This formula effectively defines $\chsat$
as for every Horn formula $\theta$ encoded by an $\R$-structure $\A$, the number of satisfying assignments of $\theta$ is equal to $\sem{\sa{\pT} \sa{\pA} \psi(\pT,\pA)}(\A)$.  Therefore, we conclude that $\chsat \in \eqso(\uhorn)$. 
\end{example}
We extend the definition of $\uhorn$ to allow existential quantification. More precisely, a formula $\varphi$ is in $\ehorn$ if $\varphi$ is of the form $\exists \bar x \, \psi$ with $\psi$ a Horn formula. Interestingly, it hold that $\cdnf \in \eqso(\ehorn)$ and

\begin{proposition}\label{prop:ehorn-pe}
$\eqso(\ehorn) \subseteq \totp$.
\end{proposition}
A natural question at this point is whether in the definitions of $\uhorn$ and $\ehorn$, it is necessary to allow negative literals of the form $\exists \v \, \neg X(\u,\v)$. The following result shows that it is indeed the case:

\begin{proposition}\label{prop:hsat-not-sigma2}	
$\chsat \not\in \eqso(\loge{2})$.
\end{proposition}
We conclude this section by showing that a natural extension of $\chsat$ is $\eqso(\ehorn)$-complete under parsimonious reductions. We define the decision problem:
\begin{multline*}
\dhsat = \{\Phi \mid \Phi \text{ is a disjunction of}\\  \text{Horn formulas and $\Phi$ is satisfiable}\},
\end{multline*}
and the counting problem $\shdhsat$ as a function that counts all satisfying assignments of a formula $\Phi$ that is a disjunction of Horn formulas.

\begin{theorem} \label{sigma2hard}
	$\shdhsat$ is $\eqso(\ehorn)$-complete under parsimonious reductions. 
\end{theorem}