%!TEX root = main.tex

We proposed a framework based on Weighted Logics to develop a descriptive complexity theory for complexity classes of functions.
%In particular, we show how this framework can be used to capture fundamental counting complexity classes such as $\fp$, $\shp$ and $\fpspace$, among others. Moreover, we use it to define a hierarchy inside $\shp$, identifying counting complexity classes with good closure and approximation properties, and which admit natural complete problems. Finally, by adding recursion to the framework, we show how to capture lower counting complexity classes such as $\shl$.	
We consider the results of this paper as a first step in this direction.
%towards a descriptive complexity theory for complexity classes of functions. �
In this sense, there are several directions for future research, some of which are mentioned here. 
$\totp$ is an interesting counting complexity class as it naturally defines a class of functions in $\shp$ with easy decision counterparts. However, we do not have a logical characterization of this class.
%for which we are missing a logical characterization. 
In the same direction, we are missing characterizations of complexity classes such as $\spanl$, or characterizations of quantitative logics such as $\qso(\so)$.
\martin{corregido}
%We could not elaborate on the strength of unextended $\qso$. It could be a new class of double-exponential functions below $\fpspace$.
We would also like to define a larger syntactic subclass of $\shp$ where each function admits an FPRAS; notice that $\cpm$ is an important problem admitting an FPRAS\cite{JSV04} that is not included in the classes defined in Section \ref{sec-clo}. Moreover, by following the approach proposed in
\cite{I83}, we would like to include second-order free variables in the operator for counting paths introduced in Section \ref{sec:beyond}, so to have alternative ways to capture $\fpspace$ and even $\shp$. Finally, the least fixed point operator introduced in Section \ref{sec:beyond} clearly deserves further investigation.


