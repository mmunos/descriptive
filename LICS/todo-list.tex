%!TEX root = main.tex

\section*{\LARGE \textbf{TODO LIST}}

\bigskip

Reviewer 2

\begin{itemize}

\item[$\checkmark$] p1, c2, l12: "type of problem" not "problems"

\item[$\checkmark$] p2, c1, l23: "on top of" not "in"

\item[$\checkmark$] p2, c2, l9 and l12 and again at p10, c1, l2: "not necessarily" rather than "non-"

\item[$\checkmark$] Example III.1: "undirected graph" rather than "non-directed".

\item[$\checkmark$] p4, c2, l-23: "several results were proved" rather than "they were
proved..."

\item[$\checkmark$] p5, c1, l6: delete the word "previous".

\item I am puzzled by the definition of invariance under R-isomorphisms
that appears just before Definition IV.1. You have assumed that
structures are ordered, and the definition of the encoding enc is
based on the resulting lexicographical order. Under these
conditions, if A and B are isomorphic structures, then enc(A) =
enc(B), so any function would be invariant.

You say that the assumption is standard in descriptive complexity.
But there the issue of isomorphism invariance only arises when we
want to define properties of *unordered* structures. For such a
structure A, enc(A) may depend on the particular choice of order
and we want to consider properties that are invariant under this
choice.

In short, because you only consider ordered structures, I don't
think you need to define isomorphism-invariance and this assumption
can also be removed from Definition IV.1(2).
\martin{retire donde mencionaba la idea de invariante a isomorfismo y esta marcado donde hay que agregar el comentario de estructuras ordenadas o no ordenadas}

\item[$\checkmark$] p5, c1, l-23: "fixed" rather than "fix".

\item[$\checkmark$] p5, c1, l-9 to l-7: you haven't said what \cal{L} is? At least for
the third occurrence of this symbol, I think you actually mean FO.

\item[$\checkmark$] p5, c2, l5: "captures" rather than "capture".

\item p5, c2, l5 to l10: this explantion that \#L (when L is LFP or TC)
can express some \#P-complete problems seems out of place when you
have already mentioned (just before Prop. IV.2) that \#FO captures
all of \#P.
\martin{esta marcado donde hay que reescribir esto}

\item[$\checkmark$] p5, c2, l-18: "reconstruction" rather than "reconstructions".

\item Also, is the claim that sum-product alternation "is crucial" for
this something that you can prove? That is, is there a proof that
you cannot capture FP without the alternation?
\martin{esta marcado donde hay que reescribir esto}

\item p5, c2, l-14 to l-12: the sentence beginning with "Of course". Why
is this obvious? I can imagine that the powerful quantitative
mechanisms introduced could compensate for the lack of a powerful
logical core. So, again, this looks to me like a statement that
asks for proof.
\martin{esta marcado donde hay que reescribir esto}

\item p2, c2, l-22: this seems to say that PFP is a fragment of
second-order logic. But, surele we do not know this is the case.

\item Theorem IV.5. You defined QSO(L) for fragments L of second-order
logic, and PFP is not such a fragment, so at this point it is
undefined. Of course, I can see how to define it, but this should
be stated more carefully.
\martin{suger\'i cambiar 'fragments' por 'fragments and extensions' pero no fue muy acogida por el autor R.}

\item[$\checkmark$] p6, c1, l19: "is" not "in".

\item[$\checkmark$] p6, c1, l-13; and p6, c2, l-15; and p11, c2, l-15: "Similar than" sounds wrong. I would use "Just
as..."

\item[$\checkmark$] p6, c1, l-2: "QSO is" rather than "is QSO".

\item[$\checkmark$] p6, c2, l13: "as are many" instead of "as many"

\item[$\checkmark$] p6, c2, l-15: "distinguish" not "distinguished"

\item[$\checkmark$] p6, c2, l-6: "logics needed" rather than "needed logics"

\item[$\checkmark$] p7, c1, l27: Is \cal{A} the algorithm or the function?

\item[$\checkmark$] p7, c1, l-2: "functions" not "function"

\item[$\checkmark$] p7, c2, l22: "satisfies" rather than "satisfy"

\item[$\checkmark$] p9, c1, l13: "mention" not "mentioned"

\item[$\checkmark$] p9, c2, l-2 to l-1: perhaps just "the set of Horn formulae over a
signature R" rather than "the set of formulae psi such that psi is
a Horn formula over a signature R".

\item[$\checkmark$] p10, c1, l-12: delete "that" before "how"

\item[$\checkmark$] p10, c1, l-11: "capture" not "captures"

\item[$\checkmark$] p10, c2, l19: "of weighted logics" not "on"

\item[$\checkmark$] p11, c2, l12: "monotone" not "monotones"

\item[$\checkmark$] p11, c2, l29: "relational" not "relation"

\item[$\checkmark$] p12, c1, l18: "the widely believed assumption" that UL is properly
contained in NL? I don't think this assumption is widely believed
among complexity theorists. I think UL = NL is more widely
believed. See the references:
1. Klaus Reinhardt, Eric Allender: Making Nondeterminism
Unambiguous. SIAM J. Comput. 29(4): 1118-1131 (2000)
2. Eric Allender, Klaus Reinhardt, Shiyu Zhou: Isolation, Matching,
and Counting Uniform and Nonuniform Upper Bounds. 
J. Comput. Syst. Sci. 59(2): 164-181 (1999).
\martin{corregido en el ultimo parrafo antes de las conclusiones}
\end{itemize}

Reviewer 3

\begin{itemize}
	
\item[$\checkmark$] What is the strength of full QSO without any fixed-point operator? This is an apparent gap in the story you are telling.
\martin{esta mencionado en las conclusiones}
	
\item[$\checkmark$] After Corollary IV.8: "This is an interesting result..." Generally, I think that you should let the reader decide which results are interesting and which are not. I recommend to cut the "interesting" here.
	
\item[$\checkmark$] First paragraph of Section V: I find it weird that you cite [6] for the problem \#DNF. This problem has been studied long before; it was at least already known to be \#P-complete in [24], but I would guess Valiant already remarked in his early papers that the problem is \#P-complete (although I have not checked this).
\martin{cada vez que se menciona [6] por dnf lo cambie por [6,24]}
	
\item after Proposition V.6: Please make the definition of $\Sigma_1[FO]$ more readable. I had a hard time parsing it.
\martin{agregué una nota de cómo se puede hacer un poco más legible y además más corto}
	
\item[$\checkmark$] Typo: "this notion can be used to captures" -> capture
	
\item[$\checkmark$] After Theorem VI.4: "at the later level" -> latter
	
\item[$\checkmark$] I did not get the definition of min(x) in the appendix. Should the $<$ be $>$? Is this just supposed to define the lexicographically smallest of all tuples, so $a^m$ where a is the minimal element of the domain?
	
\end{itemize}

%%% Local Variables:
%%% mode: latex
%%% TeX-master: "main"
%%% End:
