%!TEX root = main.tex

\subsection{Transitive Logics}

It was shown in \cite{I86,I88} that first-order logic extended with a transitive closure operator captures $\nlog$. Inspired by this work, we extend the definition of $\qfo$ with an operator for counting the number of paths in a directed graph, and we show that it can be used to capture $\shl$. Besides, we show that the same idea can be used to extend $\qso$ allowing to capture harder complexity classes. 

Given a relation signature $\R$, the set of transitive $\qso$ formulas ($\tqso$-formulas) is defined by the following grammar:
\begin{multline}
	\label{eq-def-tqso}
	\alpha := \varphi \, \mid \, s \, \mid \, (\alpha \add \alpha) \, \mid\, (\alpha \mult \alpha) \, \mid \, 
	\sa{x} \alpha \, \mid\, \
	\pa{x} \alpha \, \mid \\ 
	\sa{X} \alpha \, \mid \, \pa{X} \alpha \, \mid \, [\pth \psi(\bar{x}, \bar{X},\bar{y}, \bar{Y})],
\end{multline}
where $\varphi$ is an $\so$-formula over $\R$, $\psi(\bar{x},\bar{X},\bar{y},\bar{Y})$ is an $\fo$-formula over $\R$, $\bar{x} = (x_1, \ldots, x_k)$, $\bar{y} = (y_1, \ldots, y_k)$ are tuples of pairwise distinct first-order variables, and $\bar{X} = (X_1, \ldots, X_\ell)$, $\bar{Y} = (Y_1, \ldots, Y_\ell)$ are tuples of pairwise distinct second-order variables such that $\arity(X_i) = \arity(Y_i) = m_i$ for every $i \in \{1, \ldots, \ell\}$. The semantics of $[\pth \psi(\bar{x},\bar{X},\bar{y}.\bar{Y})]$ is defined as follows. Given an $\R$-structure $\A$, define a (directed) graph $\cG_{\psi}(\A) = (N,E)$ such that $N = A^k \times \prod_{i=1}^\ell 2^{A^{m_i}}$, where $A$ is the domain of $\A$, and for every pair $(\bar b, \bar B)$, $(\bar c, \bar C)$ of elements of $N$, it holds that $((\bar b, \bar B), (\bar c, \bar C)) \in E$ if and only if $\A \models \psi(\bar b, \bar{B}, \bar c, \bar{C})$. Then given first-order and second-order assignments $v$, $V$ for $\A$, we have that $\sem{[\pth \psi(\bar{x},\bar{X}, \bar{y}, \bar{Y})]}(\A,v,V)$ is the number of paths in $\cG_\psi(\A)$ from $(v(\bar x), V(\bar X))$ to $(v(\bar y), V(\bar Y))$ whose length is at most $n = |N|$.

As for the case of $\qso$, the logic $\tqso(\LL)$ is obtained by restricting $\varphi$ in \eqref{eq-def-tqso} to be a formula in $\LL$. Moreover, the logic $\tqfo$ is obtained by disallowing in \eqref{eq-def-tqso} formulas $\sa{X} \alpha$ and $\pa{X} \alpha$, and by only allowing  first-order free-variables in the formula $\psi$ used in $[\pth \psi]$ in \eqref{eq-def-tqso}. With this notation, we have the following results:


\begin{theorem} \label{tqfo-shl}
	$\tqfo(\fo)$ captures $\shl$ over the class of ordered structures.
\end{theorem}

\begin{theorem} \label{tqso-fo-fpsace}
	$\tqso$ and $\tqso(\fo)$ captures $\fpspace$ over the class of ordered structures.
\end{theorem}

\marcelo{Puede que este equivocado, pero me parece que teniamos una forma de capturar $\shp$ usando el operator ${\bf path}$. Pero no logro recordar como se hacia esto, y me parece que lo que habiamos escrito antes en esta seccion estaba equivocado: ``$\tqso(\fo)$ captures $\shp$ over the class of ordered structures". Claro que yo puedo estar usando una definicion distinta de $\tqso(\fo)$.}


%We also define the set of transitive $\qso$ formulas ($\tqso$-formulas) using the following grammar:
%\begin{multline*}
%%	\label{eq-def-tqso}
%	\alpha := \varphi \ \mid \ s \ \mid \ (\alpha \add \alpha) \ \mid\ (\alpha \mult \alpha) \ \mid \\ \sa{x} \alpha \ \mid \ \pa{x} \alpha \ \mid \ \sa{X} \alpha \ \mid \ \pa{X} \alpha \ \mid \ [\pth \varphi]
%\end{multline*}
%
%
% 
%We define the operator {\bf path} as follows. Let $\A$ be an ordered structure. Given a formula $\psi(\bar{x},\bar{y})$, where $\vert \bar{x} \vert = \vert \bar{y} \vert = k$ let ${\cal G} = ({\cal V},\cal{E})$ be induced graph over the set of vertices ${\cal V} = A^k$, and for every $\bar{a},\bar{b}\in A^k$ it holds that ${\cal E}(\bar{a},\bar{b})$ if and only if $\A \models \psi(\bar{a},\bar{b})$. To formalize the semantics for this operator, let $n = \vert A^k \vert$.
%For a given first order assignment $v$ and a second order asssignment $V$, let $\bar{a} = v(\bar{x})$ and $\bar{b} = v(\bar{y})$, and $\sem{[\pth\, \psi(\bar{x},\bar{y})]}(\A,v,V)$ will take the value of the number of paths of size less or equal to $n$ from $\bar{a}$ to $\bar{b}$ in the graph ${\cal G}$. This operator lets us define the set of transitive $\qfo$ formulas over $\R$ ($\tqfo$-formulas) using the following grammar:
%\begin{multline*} 
%%	\label{eq-def-tqfo}
%	\alpha := \varphi \ \mid \ s \ \mid \ (\alpha \add \alpha) \ \mid\ (\alpha \mult \alpha) \\ \mid \ \sa{x} \alpha \ \mid \ \pa{x} \alpha \ \mid \ [\pth \varphi]
%\end{multline*}
%where $\varphi$ is an $\fo$-formula over $\R$, $s \in \bbN$ and $x \in \fv$.
%
%We also define the set of transitive $\qso$ formulas ($\tqso$-formulas) using the following grammar:
%\begin{multline*}
%%	\label{eq-def-tqso}
%	\alpha := \varphi \ \mid \ s \ \mid \ (\alpha \add \alpha) \ \mid\ (\alpha \mult \alpha) \ \mid \\ \sa{x} \alpha \ \mid \ \pa{x} \alpha \ \mid \ \sa{X} \alpha \ \mid \ \pa{X} \alpha \ \mid \ [\pth \varphi]
%\end{multline*}
%where $\varphi$ is an $\so$-formula over $\R$, $s \in \bbN$, $x \in \fv$ and $X \in \sv$.
%\begin{theorem} \label{so-rec}
%	Given a positive $\fo$ formula $\varphi(\bar{x},R)$ and a $\qfo$ formula $\alpha(\bar{x})$, there exists a $\qso$ formula $\beta(\bar{x})$ such that $\sem{[\alfp\varphi(\bar{x},R)\mid \alpha(\bar{x},R)](\bar{x})} = \sem{\beta(\bar{x})}$.
%\end{theorem}
%
%\begin{theorem} \label{tqfo-fo-cap}
%	$\tqfo(\fo)$ captures $\shl$ over the class of ordered structures.
%\end{theorem}
%
%\begin{theorem} \label{tqso-fo-cap}
%	$\tqso(\fo)$ captures $\shp$ over the class of ordered structures.
%\end{theorem}


\subsection{Recursive Logics}

We define an operator which extends least fixed point logic \cite{I86,vardi1982complexity} to allow counting. 
Fix a relational signature $\R$. Then the set of $\rqfo$ formulas over $\R$ is defined by the following grammar:
\begin{multline*}
	\alpha := \varphi \, \mid \, s \, \mid \, (\alpha \add \alpha) \, \mid\, (\alpha \mult \alpha) \, \mid \, 
	\sa{x} \alpha \, \mid\\ 
	\pa{x} \alpha \, \mid\,
	\clfp{\psi(x_1,\ldots,x_k,R)}{\beta(y_1, \ldots, y_\ell, R, \pi)},
\end{multline*}
where $R$ is second-order variable of arity $k$, $\psi(x_1, \ldots, x_k, R)$ is an $\fo$-formula over $(\R \cup \{R\})$ that is positive on $R$ and $\beta(y_1, \ldots, y_\ell, R, \pi)$ is a $\qfo$-formula over $\R$ including a fresh symbol $\pi$ that represents a function of arity $\ell$, that is, every atomic formula in $\beta(y_1, \ldots, y_\ell, R, \pi)$ is either an $\fo$-formula over $\R$, a constant $s \in \N$ or a term of the form $\pi(u_1, \ldots, u_\ell)$ with $u_1, \ldots, u_\ell$ non-necessarily distinct first-order variables. 
The free variables of the formula $\clfp{\psi(x_1,\ldots,x_k,R)}{\beta(y_1, \ldots, y_\ell, R, \pi)}$
are $y_1, \ldots, y_\ell$, in particular it does not have any second-order free variable.

We need to introduce some terminology to define the semantics of the formula $\clfp{\psi(x_1,\ldots,x_k,R)}{\beta(y_1, \ldots, y_\ell, R, \pi)}$
Given a $\qfo$-formula $\gamma$ with $\ell$ first-order free variables and no second-order free variable, define $\beta[\gamma/\pi]$ to be the $\qfo$-formula obtained from $\beta$ by replacing every occurrence of a term $\pi(u_1, \ldots, u_\ell)$ by $\gamma(u_1, \ldots, u_\ell)$.
Let $\A$ be an $\R$-structure with domain $A$. Then for every $a \in A$, let $\varphi_a(x)$ be an $\fo$-formula such that $\A \models \varphi_a(b)$ if and only if $b = a$. Formally, assuming that $a$ is the $p$-th element of $A$ according to the linear order $<^\A$, we have that:
%then $\varphi_a(x) = \forall y(x < y \vee x = y)$
%for every natural number $i \geq 1$, let $\varphi_i(x)$ be an $\fo$-formula such that $\varphi_i(a)$ holds in an $\R$-structure $\A$ if and only if $a$ is the $i$-th element in the linear order  $<^{\A}$,
%that is, $\varphi_1(x) = \forall y(x < y \vee x = y)$ and for every $p > 1$:
\begin{multline*}
\varphi_a(x) \ = \ \exists x_1 \cdots \exists x_{p-1}\bigg[\bigwedge_{i =1}^{p-2}x_i < x_{i+1} \wedge\,\\  
x_{p-1} < x  \wedge \forall y(y < x \to \bigvee_{i = 1}^{p-1} y = x_i)\bigg].
\end{multline*}
Finally, given a function $f : A^\ell \to \N$, define $\qfo$-formula $\delta_{f,\A}(z_1, \ldots, z_\ell)$ as follows:
\begin{multline*}
 \mathop{+}_{(a_1,\ldots,a_{\ell})\in A^{\ell}} \ \varphi_{a_1}(z_1) \cdot \ldots \cdot \varphi_{a_{\ell}}(z_{\ell})\cdot f(a_1,\ldots,a_{\ell}), 
\end{multline*}
where $z_1, \ldots, z_\ell$ is a sequence of pairwise distinct first-order variables.
We have that $\delta_{f,\A}(z_1, \ldots, z_\ell)$ encodes $f$ in the structure $\A$ since:
\begin{eqnarray*}
\sem{\delta_{f,\A}(z_1, \ldots, z_\ell)}(\A,v)  & = & f(v(z_1), \ldots, v(z_\ell)).
\end{eqnarray*}
%Recall that the least fixed point operator is defined by a formula $\psi(x_1,\ldots,x_k,R)$ that is positive on $R$, where $R$  is a predicate of arity $k$. For a structure $\A$ with domain $A$, the operator $T_{\varphi}:2^{A^k} \to 2^{A^k}$ is defined as $T_{\varphi}(X) = \{(a_1,\ldots,a_k)\mid (\A,X)\models \psi(a_1,\ldots,a_k,R) \}$, for each $X\subseteq A^k$. Let $T_0 = \emptyset$ and $T_{i+1} = T_{\varphi}(T_i)$ for each $i \in \nat$. Note that there exists $n\in \nat$ such that $T_{n+1} = T_n$ because $R$ is positive in $\psi(x_1,\ldots,x_k,R)$. Then the evaluation of $[\lfpop\psi(x_1,\ldots,x_k,R)]$ is defined by the fixed point $T_n$, that is, for every $(a_1,\ldots,a_k)\in A^k$, it holds that $\A\models[\lfpop\psi(x_1,\ldots,x_k,R)](a_1,\ldots,a_k)$ if and only if $(a_1,\ldots,a_k) \in T_n$.
We are now ready to define the semantics of the formula $\clfp{\psi(x_1,\ldots,x_k,R)}{\beta(y_1,\ldots,y_{\ell},R,\pi)}$,  whose free variables are $y_1$, $\ldots$, $y_\ell$. Assume that $\A$ is an $\R$-structure with domain $A$. Following the definition of the semantics of least fixed point logic \cite{I86,vardi1982complexity}, define an operator $T_{\varphi}:2^{A^k} \to 2^{A^k}$ such that:
$$
T_{\varphi}(X)  =  \{(a_1,\ldots,a_k) \in A^k \mid \A \models \psi(a_1,\ldots,a_k,X) \},
$$
for every $X \subseteq A^k$. This operator is used to defined 
a sequence $T_0 \subsetneq T_1 \subsetneq \cdots \subsetneq T_n \subseteq A^k$  such that $n \geq 0$, $T_0 = \emptyset$, $T_{i+1} = T_{\varphi}(T_i)$ for every $i \in \{0, \ldots, n-1\}$, and $T_n = T_\varphi(T_n)$. We know that such a sequence exists as formula $\psi(x_1,\ldots,x_k,R)$ is positive on $R$. Now, the sequence $T_0 \subsetneq T_1 \subsetneq \cdots \subsetneq T_n$ is used to defined  a sequence of functions $f_0,f_1,\ldots,f_n:A^{\ell}\to\nat$. Formally, we have that $f_0(a_1, \ldots, a_\ell) = 0$ for every $(a_1, \ldots, a_\ell) \in A^\ell$. Moreover, assuming that $(a_1, \ldots, a_\ell) \in A^\ell$, $v$ is a first-order assignment for $\A$ such that $v(y_i) = a_i$ for every $i \in \{1, \ldots, \ell\}$, and that $V$ is a second-order assignment for $\A$ such that $V(R) = T_{i+1}$, we have that:
\begin{eqnarray*}
f_{i+1}(a_1, \ldots, a_\ell) & = & \sem{\beta[\delta_{f_i,\A}/\pi]}(\A,v,V).
\end{eqnarray*}
%, for each $T_i \in \T$, let $V$ be a second-order assignment for $\A$ that assigns $T_i$ to $R$, let $\zeta_i: A^{\ell}\to\nat$ be such that for each $(a_1,\ldots,a_{\ell})\in A^{\ell}$ it holds $\zeta_i(a_1,\ldots,a_{\ell}) = \sem{\alpha\mid_{\pi(u_1,\ldots,u_{\ell})\to \beta_i(u_1,\ldots,u_{\ell})}}(\A,v,V)$, where $v$ is a first-order assignment for $\A$ that satisfies $a_i = v(x_i)$ for each free $x_i$ in $\alpha$.
Finally, the semantics of the formula $\clfp{\psi(x_1,\ldots,x_k,R)}{\beta(y_1,\ldots,y_{\ell},R,\pi)}$ is defined by means of function $f_n$:
%For a given first order assignment $v$ and a second order assignment $V$, let $a_i = v(x_i)$, the operator is then evaluated as:
\begin{align*}
&\llbracket [{\bf lfp} \, \psi(x_1,\ldots,x_k,R) \,|\\
&\hspace{10mm}\beta(y_1,\ldots,y_{\ell},R,\pi)]\rrbracket(\A,v) \ = \
f_n(v(y_1),\ldots,v(y_{\ell})).
\end{align*}

\begin{example}
%As an example, 
We would like to define a formula that, given a graph $G$ with $n$ nodes and a pair of nodes $b$, $c$ in $G$, counts the number of paths of length at most $n$ from $b$ to $c$ in $G$.
To this end, assume that $\R = \{ E(\cdot,\cdot) \}$, and define $\psi(x,R)$ as follows:
\begin{eqnarray*}
%\psi(x,R) = 
\forall y(x < y \vee x = y) \vee \exists z(R(z) \wedge \varphi_{succ}(z,x)),
\end{eqnarray*}
where $\varphi_{succ}(x,y)$ is a formula that is satisfied by pairs $(x,y)$ that are consecutive in the order $<$. That is, $\varphi_{succ}(x,y) = x < y \wedge \neg \exists z \, (x < z \wedge z < y)$. Moreover, define formula $\beta(x,y,R,\pi)$ as follows:
$$
%\alpha(x,y,R,\pi) = 
%(\neg \exists zR(z))\cdot(x = y) 
E(x,y) + \sa{z} [\pi(x,z)\cdot E(z,y)].
$$
Then we have that $\clfp{\psi(x,R)}{\beta(x,y,R,\pi)}$ defines our counting function. In fact, assume that $\A$ is an $\R$-structure with $n$ elements in its domain, $b,c$ are elements of $\A$ and $v$ is a first-order assignment over $\A$ such that $v(x) = b$ and $v(y) = c$. Then we have that $\sem{\clfp{\psi(x,R)}{\beta(x,y,R,\pi)}}(\A,v)$ is equal to the  number of paths in $\A$ from $b$ to $c$ of length at most $n$.
\end{example}

\marcelo{Deberiamos mostrar como el ejemplo anterior se generaliza para definir $[\pth \psi(\bar x, \bar y)]$ en terminos del operador ${\bf lfp}$.}

It is well known that least fixed point logic is contained in second-order logic \cite{L04}. In the following theorem we show that the same holds in our case.
\begin{theorem} \label{so-rec}
$\rqfo \subseteq \qso$
%	Given a positive $\fo$ formula $\varphi(\bar{x},R)$ and a $\qfo$ formula $\alpha(\bar{x})$, there exists a $\qso$ formula $\beta(\bar{x})$ such that $\sem{[\alfp\varphi(\bar{x},R)\mid \alpha(\bar{x},R)](\bar{x})} = \sem{\beta(\bar{x})}$.
\end{theorem}

Finally, the following is the main result of this section:
\begin{theorem} \label{rqfo-fo-cap}
	$\rqfo(\fo)$ captures $\fp$ over the class of ordered structures.
\end{theorem}

\marcelo{Notese que estoy permitiendo en la formula $\beta(y_1, \ldots, y_\ell,R,\pi)$ tener subformulas $\pa{x} \alpha$. Esta bien el resultado con esto? O tenemos que eliminar estas formulas?}

%Given a relation signature $\R$, the set of recursive $\qfo$ formulas ($\rqfo$-formulas) is defined by the following grammar:
%%This operator lets us define the set of recursive $\qfo$ formulas over $\R$ ($\rqfo$-formulas) using the following grammar:
%\begin{multline*}
%%	\label{eq-def-rqfo}
%	\alpha := \varphi \ \mid \ s \ \mid \ (\alpha \add \alpha) \ \mid \\ (\alpha \mult \alpha) \ \mid \ \sa{x} \alpha \ \mid \ \pa{x} \alpha \ \mid \ [\alfp \varphi \mid \alpha]
%\end{multline*}
%where $\varphi$ is an $\fo$-formula over $\R$, $s \in \bbN$ and $x \in \fv$.
%
%\marcelo{Vamos a permitir anidacion del operador $\alfp$? Esta gramatica lo permite.}
%
%\begin{theorem} \label{rqfo-fo-cap}
%	$\rqfo(\fo)$ captures $\fp$ over the class of ordered structures.
%\end{theorem}
%
