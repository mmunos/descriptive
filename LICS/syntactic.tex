%!TEX root = main.tex

In this section we study the fragment of $\eqso(\LL)$ when $\LL$ is a boolean logic contained in $\fo$. 

We say that a formula $\alpha$ in $\eqso(\LL)$ is in \emph{$\Sigma$-prenex normal form} if $\alpha$ is of the form:
\[
\alpha := \sa{\bar{X}} \sa{\bar{x}} \varphi(\bar{X}, \bar{x})
\]
where $\bar{X}$ and $\bar{x}$ is a sequence of $0$ or more second and first order variables, respectively.
In particular, note that a boolean formula is always in $\Sigma$-prenex normal form.
Then we say that a formula $\alpha$ is in \emph{sum normal form} if $\alpha$ is of the form $c + \Sigma_{i=1}^n \alpha_i$ where $c$ is a non-negative constant and each $\alpha_i$ is in $\Sigma$-prenex normal form.
Next we show that each formula in $\eqso(\LL)$ can be converted in \emph{sum normal form} but not always in prenex normal form.

\begin{theorem}
	Every formula in $\eqso(\LL)$ can be rewritten in \emph{sum normal form}. Furthermore, there exists a language $\LL$ and a formula $\alpha$ such that there does not exist a formula in $\Sigma$-prenex normal form equivalent to $\alpha$.
\end{theorem}

An interesting question here is whether the multiplicative operators in $\qso$ can be defined in $\eqso(\LL)$. As the following result shows, if $\LL$ is closed under conjunction, then the binary product can be defined in  $\eqso(\LL)$.

\begin{theorem}\label{theo:binary-prod}
	If $\LL$ is closed under conjunction, then binary product can be defined in $\eqso(\LL)$.
\end{theorem}

\subsection{Prenex-normal form classes}
Saluja et. al \cite{DBLP:journals/jcss/SalujaST95} define a family of counting classes $\#\cL$ for each fragment $\cL$ of $\fo$. For a formula $\varphi(x,X)$, the function $f_{\varphi(x,X)}$ is defined as
\[
f_{\varphi(x,X)}(\A) = \vert \{\langle e,P\rangle\mid \A\models\varphi(e,P)\}\vert.
\]
for each $\A\in\str$. A function $f:\Sigma^*\to\nat$ is in $\#\cL$ if there exists an $\cL$ formula $\varphi(x,X)$ such that $f = f_{\varphi(x,X)}$.

\begin{theorem} \label{saluja-eq}
	$\eqso(\cL) = \#\cL$ for each fragment $\L$ of $\fo$.
\end{theorem}

For every logic $\cL$, we define an $\cL$-extended quantifier-free (QF) formula as follows:
\begin{eqnarray*}
	\varphi &::=& \alpha, \alpha \text{ is an $\cL$-formula} \ \mid \\
	&& X_i(x_1,\dots,x_{a_i}), i\in\N \ \mid \ \\
	&& (\neg \varphi) \ \mid \ (\varphi \wedge \varphi) \ \mid \ (\varphi \vee \varphi).
\end{eqnarray*}

We define syntactically the fragments $\logex{i}$ and $\logux{i}$ according to the following grammar:
\begin{align*}
\logex{0} = \logux{0} &::= \varphi , \varphi \mbox{ is an $\fo$-extended QF formula,} \\
\logex{i+1} &::= \logux{i} \ \mid \ \exists x\, \logex{i+1}, \\
\logux{i+1} &::= \logex{i} \ \mid \ \forall x\, \logux{i+1}.
\end{align*}

\begin{theorem} \label{fp1}
	$\qfo(\logex{0}) \subseteq$ {\sc FP}.
\end{theorem}

The {\em decision problem} associated to a function $f$ is defined by the language $L_f = \{\A \in \str \mid f(\A) > 0\}$.

\begin{theorem} \label{decisionptime}
	The decision problem associated to a function in $\eqso(\logex{1})$ is in \textsc{P}.
\end{theorem}

For a given pair of functions $f,g$, we define $f \dotminus g$ as follows:
\begin{eqnarray*}
	(f \dotminus g)(\A) =
	\begin{cases}
		f(\A)-g(\A), & \text{if }f(\A)>g(\A) \\
		0, & \text{if }f(\A) \leq g(\A).
	\end{cases}
\end{eqnarray*}
for every $\L$-structure $\A \in \str$. A function class $\F$ is {\em closed under substraction} if for every pair of functions $f,g \in \F$, it holds that $f \dotminus g \in \F$.

\begin{theorem} \label{sub-pnp}
	If $\eqso(\loge{1})$ is closed under substraction, then {\sc P} = {\sc NP}.
\end{theorem}

\begin{theorem} \label{sigma1strict}
	$\eqso(\loge{1}) \subsetneq \eqso(\logex{1})$
\end{theorem}

For a given function $f$, we define $f \dotminus 1$ as follows:
\begin{eqnarray*}
	f \dotminus 1(\A) =
	\begin{cases}
		f(\A)-1, & \text{if }f(\A) > 0 \\
		0, & \text{if }f(\A) = 0.
	\end{cases}
\end{eqnarray*}
for every $\L$-structure $\A \in \str$. A function class $\F$ is {\em closed under substraction by one} if for every function $f \in \F$, it holds that $f \dotminus 1 \in \F$.

\begin{theorem} \label{sigmafo-minusone}
	$\eqso(\logex{1})$ is closed under substraction by one.
\end{theorem}

\begin{theorem} \label{dnf-pars}
	{\sc \#DNF} is hard for $\eqso(\loge{1})$ under parsimonious reductions. 
\end{theorem}

\begin{theorem} \label{nplusone-strict}
	$\U{1}$ with $n$ open first-order variables is properly contained in $\U{1}$ with $n+1$ open first-order variables for $n\in\N$.  
\end{theorem}

\subsection{Horn Counting Classes}
%!TEX root = main.tex
\newcommand{\pP}{\textit{P}}
\newcommand{\pN}{\textit{N}}
\newcommand{\pV}{\textit{V}}
\newcommand{\pT}{\textit{T}}
\newcommand{\pA}{\textit{A}}
\newcommand{\pNC}{\textit{NC}}
\newcommand{\pD}{\textit{D}}


A positive literal is a formula of the form $X(\x)$, where $X$ is a second-order variable and $\x$ is a tuple of first-order variables, and a negative literal is a formula of the form $\exists \v \, \neg X(\u,\v)$, where $\u$ and $\v$ are tuples of first-order variables. Given a relational signature $\R$, a clause over $\R$ is a formula of the form:
$$
\forall \x \, (\varphi_1 \vee \cdots \vee \varphi_n),
$$
where each $\varphi_i$ ($1 \leq i \leq n$) is either a positive literal, a negative literal or an \fo-formula over $\R$.  A clause is said to be Horn if it contains at most one positive literal, and a formula is said to be Horn if it is a conjunction of Horn clauses over a relational signature $\R$. With this terminology, we define $\uhorn$ as the set of formulas $\psi$ such that $\psi$ is a conjunction of Horn clauses over a relational signature $\R$. 

\begin{proposition}\label{prop:uhorn-pe}
$\eqso(\uhorn) \subseteq \totp$
\end{proposition}

\begin{example} \label{ex-hornsat-esop1}
Let $\R = \{\pP(\cdot,\cdot), \pN(\cdot,\cdot), \pV(\cdot), \pNC(\cdot)\}$. This vocabulary is used as follows to encode a Horn formula. A fact $\pP(c,x)$ indicates that propositional variable $x$ is a disjunct in a clause $c$, while $\pN(c,x)$ indicates that $\neg x$ is a disjunct in $c$. Furthermore, $\pV(x)$ holds if  $x$ is a propositional variable, and $\pNC(c)$ holds if $c$ is a clause containing only negative literals, that is, $c$ is of the form $(\neg x_1 \vee \cdots \vee \neg x_n)$.

To encode $\chsat$, we define an \so-formula $\varphi(\pT)$ over $\R$, where $\pT$ is a unary predicate, such that for every Horn formula $\theta$ encoded by an $\R$-structure $\A$, the number of satisfying assignments of $\theta$ is equal to $\sem{\sa{\pT} \varphi(\pT)}(\A)$. In particular, $\pT(x)$ holds if and only if $x$ is a propositional variable that is assigned value 1.  More specifically, $\varphi(\pT)$ is defined as follows:
\begin{align*}
&\forall x \, (\pT(x) \to \pV(x)) \ \wedge\\
&\forall c \, (\pNC(c) \to \exists x \, (\pN(c,x) \wedge \neg \pT(x))) \ \wedge\\
&\forall c \forall x \, ([\pP(c,x) \wedge \forall y \, (\pN(c,y) \to \pT(y))] \to \pT(x)).
\end{align*}
Given that $\uhorn$ is designed with the goal in mind of capturing $\chsat$, we expect $\varphi(\pT)$ to be a formula in $\uhorn$. However, if we rewrite it as a conjunction of clauses we obtain the following:
\begin{align*}
&\forall x \, (\neg \pT(x) \vee \pV(x)) \ \wedge\\
&\forall c \, (\neg \pNC(c) \vee \exists x \, (\pN(c,x) \wedge \neg \pT(x)))\ \wedge\\
&\forall c \forall x \, (\neg \pP(c,x) \vee \exists y \, (\pN(c,y) \wedge \neg \pT(y)) \vee \pT(x)).
\end{align*}
The resulting formula $\varphi(\pT)$ is not in $\uhorn$, but it can be easily transformed into a formula in this class  by introducing an auxiliary binary predicate $\pA$ defined as follows:
\begin{align*}
\forall c \forall x \, (\neg \pA(c,x) \leftrightarrow [\pN(c,x) \wedge \neg \pT(x)]).
\end{align*}
In this way, we obtain the following formula $\psi(\pT,\pA)$ in $\uhorn$:
\begin{align*}
&\forall x \, (\neg \pT(x) \vee \pV(x)) \ \wedge\\
&\forall c \, (\neg \textit{NC}(c) \vee \exists x \, \neg \textit{A}(c,x)) \ \wedge\\
&\forall c \forall x \, (\neg \textit{P}(c,x) \vee \exists y \, \neg \textit{A}(c,y) \vee \textit{T}(x)) \ \wedge\\
&\forall c \forall x \, (\neg \textit{N}(c,x) \vee \textit{T}(x) \vee \neg \textit{A}(c,x)) \ \wedge\\
&\forall c \forall x \, (\textit{A}(c,x) \vee \textit{N}(c,x)) \ \wedge\\
&\forall c \forall x \, (\textit{A}(c,x) \vee \neg\textit{T}(x)).
\end{align*}
This formula effectively defines $\chsat$
as for every Horn formula $\theta$ encoded by an $\R$-structure $\A$, the number of satisfying assignments of $\theta$ is equal to $\sem{\sa{\pT} \sa{\pA} \psi(\pT,\pA)}(\A)$.  Therefore, we conclude that $\chsat \in \eqso(\uhorn)$. 
\end{example}
We extend the definition of $\uhorn$ to allow existential quantification. More precisely, a formula $\varphi$ is in $\ehorn$ if $\varphi$ is of the form $\exists \bar x \, \psi$ with $\psi$ a Horn formula. Interestingly, it hold that $\cdnf \in \eqso(\ehorn)$ and

\begin{proposition}\label{prop:ehorn-pe}
$\eqso(\ehorn) \subseteq \totp$.
\end{proposition}
A natural question at this point is whether in the definitions of $\uhorn$ and $\ehorn$, it is necessary to allow negative literals of the form $\exists \v \, \neg X(\u,\v)$. The following result shows that it is indeed the case:

\begin{proposition}\label{prop:hsat-not-sigma2}	
$\chsat \not\in \eqso(\loge{2})$.
\end{proposition}
We conclude this section by showing that a natural extension of $\chsat$ is $\eqso(\ehorn)$-complete under parsimonious reductions. We define the decision problem:
\begin{multline*}
\dhsat = \{\Phi \mid \Phi \text{ is a disjunction of}\\  \text{Horn formulas and $\Phi$ is satisfiable}\},
\end{multline*}
and the counting problem $\shdhsat$ as a function that counts all satisfying assignments of a formula $\Phi$ that is a disjunction of Horn formulas.

\begin{theorem} \label{sigma2hard}
	$\shdhsat$ is $\eqso(\ehorn)$-complete under parsimonious reductions. 
\end{theorem}
