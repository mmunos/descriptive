%!TEX root = main.tex

\subsection{Saluja Classes}

We define syntactically the classes $\XE{i}$ and $\XU{i}$ according to the following grammar:
\begin{enumerate}
	\item $\XE{0}$:
	\[
	E_0 ::= \varphi , \varphi \mbox{ is an {\sc FO}-extended quantifier-free $\L$-formula }.
	\]
	\item $\XU{0}$:
	\[U_0 ::= E_0.\]
	\item $\XE{i+1}$:
	\[	E_{i+1} ::= U_i \ \mid \ \exists x \, E_{i+1}.\]
	\item $\XU{i+1}$:
	\[U_{i+1} ::= E_i \ \mid \ \forall x \, U_{i+1}.\]
\end{enumerate}

A function $f$ is in $\XE{i}$ (resp. $\XU{i}$) if there is an $\L$-formula $\varphi$ defined by the grammar $E_i$ (resp. $U_i$) such that $f = f_{\varphi}$.

\begin{theorem} \label{fp1}
	$\XE{0} \subseteq$ {\sc FP}.
\end{theorem}

The {\em decision problem} associated to a function $f$ is defined by the language $L_f = \{\A \in \str \mid f(\A) > 0\}$.

\begin{theorem} \label{decisionptime}
	The decision problem associated to a function in $\E1$ is in \textsc{P}.
\end{theorem}

For a given pair of functions $f,g$, we define $f \dotminus g$ as follows:
\begin{eqnarray*}
	(f \dotminus g)(\A) =
	\begin{cases}
		f(\A)-g(\A), & \text{if }f(\A)>g(\A) \\
		0, & \text{if }f(\A) \leq g(\A).
	\end{cases}
\end{eqnarray*}
for every $\L$-structure $\A \in \str$. A function class $\F$ is {\em closed under substraction} if for every pair of functions $f,g \in \F$, it holds that $f \dotminus g \in \F$.

\begin{theorem} \label{sub-pnp}
	If $\E1$ is closed under substraction, then {\sc P} = {\sc NP}.
\end{theorem}

\begin{theorem} \label{sigma1strict}
	$\#\Sigma_1 \subsetneq \E1$
\end{theorem}

For a given function $f$, we define $f \dotminus 1$ as follows:
\begin{eqnarray*}
	f \dotminus 1(\A) =
	\begin{cases}
		f(\A)-1, & \text{if }f(\A) > 0 \\
		0, & \text{if }f(\A) = 0.
	\end{cases}
\end{eqnarray*}
for every $\L$-structure $\A \in \str$. A function class $\F$ is {\em closed under substraction by one} if for every function $f \in \F$, it holds that $f \dotminus 1 \in \F$.

\begin{theorem} \label{sigmafo-minusone}
	$\XE{1}$ is closed under substraction by one.
\end{theorem}

\begin{theorem} \label{dnf-pars}
	{\sc \#DNF} is hard for $\E1$ under parsimonious reductions. 
\end{theorem}

\begin{theorem} \label{nplusone-strict}
	$\U{1}$ with $n$ open first-order variables is properly contained in $\U{1}$ with $n+1$ open first-order variables for $n\in\N$.  
\end{theorem}

\subsection{Horn Counting Classes}

We define the syntactic classes $\XE{i}$ and $\XU{i}$ as follows. First we define extended Horn clauses.
\begin{eqnarray*}
	PL &::=& X_i(\x),\, i\in\N,\\
	NL &::=& \neg X_i(\x),\, i\in\N \ \mid \ \exists x\, NL,\\
	NC &::=& NL \ \mid \ \alpha, \alpha \mbox{ is an {\sc FO}-formula over } \L \ \mid \ (NC \vee NC),\\
	HC &::=& NC \ \mid \ (NC \vee PL) \ \mid \ PL,
\end{eqnarray*}
where $PL$ represents a positive literal, $NL$ is an \textit{extended} negative literal, $NC$ is an extended Horn clause, and $HC$ is an extended Horn formula. Now we define the syntax of the classes inductively.\textbf{}
\begin{enumerate}
	\item $\HE{0}$:
	\begin{eqnarray*}
		E_0 &::=& HC \ \mid \ E_0 \wedge E_0.
	\end{eqnarray*}
	\item $\HU{0}$:
	\begin{eqnarray*}
		U_0 &::=& E_0.
	\end{eqnarray*}
	\item $\HE{i+1}$:
	\begin{eqnarray*}
		E_{i+1} &::=& U_i \ \mid \ \exists x \, E_{i+1}.
	\end{eqnarray*}
	\item $\HU{i+1}$:
	\begin{eqnarray*}
		U_{i+1} &::=& E_i \ \mid \ \forall x \, U_{i+1}.
	\end{eqnarray*}
\end{enumerate}
A function $f$ is in $\XE{i}$ (resp. $\XU{i}$) if there is an $\L$-formula $\varphi$ defined by the grammar $E_i$ (resp. $U_i$) such that $f = f_{\varphi}$.

The class $\pe$ is defined in \cite{DBLP:conf/mfcs/PagourtzisZ06} as $\pe = \{f\mid f\in\#P \mbox{ and its decision version } L_{f}\in P\}$.

\begin{theorem} \label{sigma2-pe}
	$\HE{2} \subseteq \pe$
\end{theorem}

We define the decision problem
\begin{multline*}
\shdhsat = \{\Phi \mid \Phi \text{ is a disjunction of Horn} \\ \text{formulas and $\Phi$ is satisfiable}\},
\end{multline*}
and the counting problem {\shdhsat } as a function that counts all satisfying assignments to a formula $\Phi$ which is a disjunction of Horn formulas.

\begin{theorem} \label{sigma2hard}
	$\shdhsat$ is hard for $\XE{2}$ under parsimonious reductions. 
\end{theorem}