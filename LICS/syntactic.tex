%!TEX root = main.tex

The class $\shp$ was introduced in \cite{Valiant79} to prove that computing the permanent of a matrix, as defined in Example \ref{exa-perm}, is a difficult problem. More specifically, it was shown in  \cite{Valiant79}  that this problem is $\shp$-complete. As a consequence of this result the problem of computing the number of perfect matchings in a bipartite graph was also shown to be $\shp$-complete. Since then  many counting problems have been proved to be $\shp$-complete \cite{V79b,PB83,P86,L86,BW91,HMRS98,BW05,DS12, PS13,PS14}. Among them, problems having easy decision counterparts play a fundamental role, as a counting problem with a hard decision version is expected to be hard. A first prominent example of such problems is counting the number of perfect matching in a bipartite graph, as it is well-known that the problem of verifying whether there exists a perfect matching in a bipartite graph can be solved in polynomial time. Other prominent examples of such problems include counting the number of: satisfying assignments of a 2-CNF propositional formula \cite{V79b}, satisfying assignments of a DNF propositional formula \cite{DHK05}, simple paths from a source node to a target node in a directed graph \cite{V79b}, extensions of a partial order to a linear order \cite{BW91} and Eulerian cycles in an undirected graph \cite{BW05}. 

Counting problems with easy decision versions play a fundamental role in the search of efficient approximations algorithms for functions in $\shp$. A fully-polynomial randomized approximation scheme (FPRAS) for a function $f : \Sigma^* \to \bbN$ is a randomized algorithm ${\cal A} : \Sigma^* \times (0,1) \to \bbN$ such that: (1) for every string $x \in \Sigma^*$ and real value $\varepsilon \in (0,1)$, the probability that $|f(x) - {\cal A}(x,\varepsilon)| \leq \varepsilon \cdot f(x)$ is at least $\frac{3}{4}$, and (2) the running time of ${\cal A}$ is polynomial in the size of $x$ and $1/\varepsilon$ \cite{KL83}. Notably, there exist $\shp$-complete functions that can be efficiently approximated as they admit FPRAS; for instance, there exist FPRAS for the problems of counting the number of satisfying assignments of a DNF propositional formula \cite{KL83} and the number of perfect matchings of a bipartite graph \cite{JSV04}. A key observation here is that if a $\shp$-complete function admits an FPRAS, then its associated decision problem is in the complexity class $\bpp$ (Bounded-Error Probabilistic Polynomial-Time). Hence, under standard complexity-theoretical assumptions we cannot hope for an FPRAS for a function in $\shp$ whose decision counterpart is $\np$-complete, and we have to concentrate on the class of counting problems with easy decision versions (in $\bpp$ or in a lower complexity class such as $\ptime$). 

The importance of the class counting problems with easy decision counterparts has motivated the search of robust definitions of classes of functions in $\shp$ with easy decision versions \cite{PagourtzisZ06}. In this section, we use the framework developed in this paper to address this problem. More specifically, we introduce in Section \ref{sec-hier-shp} a hierarchy of 


consider several fragments of $\eqso(\LL)$ where $\LL$ is a boolean logic contained in $\fo$, and we study 

% It should be noted that such class can be directly defined as the set functions f ? #P such that Lf ? P, which is denoted as #Pe in [50, 51]. However, such a definition does not lead to a well-behaved and robust function complexity class. In particular, for every function f ? #P, we have that f + 1 is trivially in #Pe, which is an undesirable property. This has led to the introduction of the more robust class TotP, which is defined as the class of functions f for which there exists a non-deterministic Turing machine M running in polynomial time such that, f(x) is the result of subtracting 1 to the number of (non-necessarily accepting) runs of M with input x [40]. In [51], it is proved that TotP ? #Pe and that TotP has a complete function problem under parsimonious reductions. However, no natural problem is known to be TotP-complete under this type of reductions [51].





In this section we study the fragment of $\eqso(\LL)$ when $\LL$ is a boolean logic contained in $\fo$. We show that by restricting $\LL$ we can find different subclasses below $\shp$ with interesting computational and closure properties. 

\cite{OH93,FH08}

From this point on, for each fragment $\FF$ of $\qso$, we will also use $\FF$ to refer to the class of functions defined by the formulas in $\FF$.

\subsection{A counting hierarchy below $\shp$}
\label{sec-hier-shp}
Saluja et. al \cite{DBLP:journals/jcss/SalujaST95} define a family of counting classes $\#\cL$ for each fragment $\cL$ of $\fo$. For a formula $\varphi(x,X)$, the function $f_{\varphi(x,X)}$ is defined as
\[
f_{\varphi(x,X)}(\A) = \vert \{\langle e,P\rangle\mid \A\models\varphi(e,P)\}\vert.
\]
for each $\A\in\str$. A function $f:\Sigma^*\to\nat$ is in $\#\cL$ if there exists an $\cL$ formula $\varphi(x,X)$ such that $f = f_{\varphi(x,X)}$.

\begin{theorem} \label{saluja-eq}
	$\eqso(\cL) = \#\cL$ for each fragment $\L$ of $\fo$.
\end{theorem}

For every logic $\cL$, we define an $\cL$-extended quantifier-free (QF) formula as follows:
\begin{eqnarray*}
	\varphi &::=& \alpha, \alpha \text{ is an $\cL$-formula} \ \mid \\
	&& X_i(x_1,\dots,x_{a_i}), i\in\N \ \mid \ \\
	&& (\neg \varphi) \ \mid \ (\varphi \wedge \varphi) \ \mid \ (\varphi \vee \varphi).
\end{eqnarray*}

We define syntactically the fragments $\logex{i}$ and $\logux{i}$ according to the following grammar:
\begin{align*}
\logex{0} = \logux{0} &::= \varphi , \varphi \mbox{ is an $\fo$-extended QF formula,} \\
\logex{i+1} &::= \logux{i} \ \mid \ \exists x\, \logex{i+1}, \\
\logux{i+1} &::= \logex{i} \ \mid \ \forall x\, \logux{i+1}.
\end{align*}

\begin{theorem} \label{fp1}
	$\qfo(\logex{0}) \subseteq$ {\sc FP}.
\end{theorem}

The {\em decision problem} associated to a function $f$ is defined by the language $L_f = \{\A \in \str \mid f(\A) > 0\}$.

\begin{theorem} \label{decisionptime}
	The decision problem associated to a function in $\eqso(\logex{1})$ is in \textsc{P}.
\end{theorem}

For a given pair of functions $f,g$, we define $f \dotminus g$ as follows:
\begin{eqnarray*}
	(f \dotminus g)(\A) =
	\begin{cases}
		f(\A)-g(\A), & \text{if }f(\A)>g(\A) \\
		0, & \text{if }f(\A) \leq g(\A).
	\end{cases}
\end{eqnarray*}
for every $\L$-structure $\A \in \str$. A function class $\F$ is {\em closed under substraction} if for every pair of functions $f,g \in \F$, it holds that $f \dotminus g \in \F$.

\begin{theorem} \label{sub-pnp}
	If $\eqso(\loge{1})$ is closed under substraction, then {\sc P} = {\sc NP}.
\end{theorem}

\begin{theorem} \label{sigma1strict}
	$\eqso(\loge{1}) \subsetneq \eqso(\logex{1})$
\end{theorem}

For a given function $f$, we define $f \dotminus 1$ as follows:
\begin{eqnarray*}
	f \dotminus 1(\A) =
	\begin{cases}
		f(\A)-1, & \text{if }f(\A) > 0 \\
		0, & \text{if }f(\A) = 0.
	\end{cases}
\end{eqnarray*}
for every $\L$-structure $\A \in \str$. A function class $\F$ is {\em closed under substraction by one} if for every function $f \in \F$, it holds that $f \dotminus 1 \in \F$.

\begin{theorem} \label{sigmafo-minusone}
	$\eqso(\logex{1})$ is closed under substraction by one.
\end{theorem}

\begin{theorem} \label{dnf-pars}
	{\sc \#DNF} is hard for $\eqso(\loge{1})$ under parsimonious reductions. 
\end{theorem}

\begin{theorem} \label{nplusone-strict}
	$\U{1}$ with $n$ open first-order variables is properly contained in $\U{1}$ with $n+1$ open first-order variables for $n\in\N$.  
\end{theorem}

\subsection{Counting hierarchy below $\shp$}


\subsection{Horn Counting Classes}
%!TEX root = main.tex
\newcommand{\pP}{\textit{P}}
\newcommand{\pN}{\textit{N}}
\newcommand{\pV}{\textit{V}}
\newcommand{\pT}{\textit{T}}
\newcommand{\pA}{\textit{A}}
\newcommand{\pNC}{\textit{NC}}
\newcommand{\pD}{\textit{D}}


A positive literal is a formula of the form $X(\x)$, where $X$ is a second-order variable and $\x$ is a tuple of first-order variables, and a negative literal is a formula of the form $\exists \v \, \neg X(\u,\v)$, where $\u$ and $\v$ are tuples of first-order variables. Given a relational signature $\R$, a clause over $\R$ is a formula of the form:
$$
\forall \x \, (\varphi_1 \vee \cdots \vee \varphi_n),
$$
where each $\varphi_i$ ($1 \leq i \leq n$) is either a positive literal, a negative literal or an \fo-formula over $\R$.  A clause is said to be Horn if it contains at most one positive literal, and a formula is said to be Horn if it is a conjunction of Horn clauses over a relational signature $\R$. With this terminology, we define $\uhorn$ as the set of formulas $\psi$ such that $\psi$ is a conjunction of Horn clauses over a relational signature $\R$. 

\begin{proposition}\label{prop:uhorn-pe}
$\eqso(\uhorn) \subseteq \totp$
\end{proposition}

\begin{example} \label{ex-hornsat-esop1}
Let $\R = \{\pP(\cdot,\cdot), \pN(\cdot,\cdot), \pV(\cdot), \pNC(\cdot)\}$. This vocabulary is used as follows to encode a Horn formula. A fact $\pP(c,x)$ indicates that propositional variable $x$ is a disjunct in a clause $c$, while $\pN(c,x)$ indicates that $\neg x$ is a disjunct in $c$. Furthermore, $\pV(x)$ holds if  $x$ is a propositional variable, and $\pNC(c)$ holds if $c$ is a clause containing only negative literals, that is, $c$ is of the form $(\neg x_1 \vee \cdots \vee \neg x_n)$.

To encode $\chsat$, we define an \so-formula $\varphi(\pT)$ over $\R$, where $\pT$ is a unary predicate, such that for every Horn formula $\theta$ encoded by an $\R$-structure $\A$, the number of satisfying assignments of $\theta$ is equal to $\sem{\sa{\pT} \varphi(\pT)}(\A)$. In particular, $\pT(x)$ holds if and only if $x$ is a propositional variable that is assigned value 1.  More specifically, $\varphi(\pT)$ is defined as follows:
\begin{align*}
&\forall x \, (\pT(x) \to \pV(x)) \ \wedge\\
&\forall c \, (\pNC(c) \to \exists x \, (\pN(c,x) \wedge \neg \pT(x))) \ \wedge\\
&\forall c \forall x \, ([\pP(c,x) \wedge \forall y \, (\pN(c,y) \to \pT(y))] \to \pT(x)).
\end{align*}
Given that $\uhorn$ is designed with the goal in mind of capturing $\chsat$, we expect $\varphi(\pT)$ to be a formula in $\uhorn$. However, if we rewrite it as a conjunction of clauses we obtain the following:
\begin{align*}
&\forall x \, (\neg \pT(x) \vee \pV(x)) \ \wedge\\
&\forall c \, (\neg \pNC(c) \vee \exists x \, (\pN(c,x) \wedge \neg \pT(x)))\ \wedge\\
&\forall c \forall x \, (\neg \pP(c,x) \vee \exists y \, (\pN(c,y) \wedge \neg \pT(y)) \vee \pT(x)).
\end{align*}
The resulting formula $\varphi(\pT)$ is not in $\uhorn$, but it can be easily transformed into a formula in this class  by introducing an auxiliary binary predicate $\pA$ defined as follows:
\begin{align*}
\forall c \forall x \, (\neg \pA(c,x) \leftrightarrow [\pN(c,x) \wedge \neg \pT(x)]).
\end{align*}
In this way, we obtain the following formula $\psi(\pT,\pA)$ in $\uhorn$:
\begin{align*}
&\forall x \, (\neg \pT(x) \vee \pV(x)) \ \wedge\\
&\forall c \, (\neg \textit{NC}(c) \vee \exists x \, \neg \textit{A}(c,x)) \ \wedge\\
&\forall c \forall x \, (\neg \textit{P}(c,x) \vee \exists y \, \neg \textit{A}(c,y) \vee \textit{T}(x)) \ \wedge\\
&\forall c \forall x \, (\neg \textit{N}(c,x) \vee \textit{T}(x) \vee \neg \textit{A}(c,x)) \ \wedge\\
&\forall c \forall x \, (\textit{A}(c,x) \vee \textit{N}(c,x)) \ \wedge\\
&\forall c \forall x \, (\textit{A}(c,x) \vee \neg\textit{T}(x)).
\end{align*}
This formula effectively defines $\chsat$
as for every Horn formula $\theta$ encoded by an $\R$-structure $\A$, the number of satisfying assignments of $\theta$ is equal to $\sem{\sa{\pT} \sa{\pA} \psi(\pT,\pA)}(\A)$.  Therefore, we conclude that $\chsat \in \eqso(\uhorn)$. 
\end{example}
We extend the definition of $\uhorn$ to allow existential quantification. More precisely, a formula $\varphi$ is in $\ehorn$ if $\varphi$ is of the form $\exists \bar x \, \psi$ with $\psi$ a Horn formula. Interestingly, it hold that $\cdnf \in \eqso(\ehorn)$ and

\begin{proposition}\label{prop:ehorn-pe}
$\eqso(\ehorn) \subseteq \totp$.
\end{proposition}
A natural question at this point is whether in the definitions of $\uhorn$ and $\ehorn$, it is necessary to allow negative literals of the form $\exists \v \, \neg X(\u,\v)$. The following result shows that it is indeed the case:

\begin{proposition}\label{prop:hsat-not-sigma2}	
$\chsat \not\in \eqso(\loge{2})$.
\end{proposition}
We conclude this section by showing that a natural extension of $\chsat$ is $\eqso(\ehorn)$-complete under parsimonious reductions. We define the decision problem:
\begin{multline*}
\dhsat = \{\Phi \mid \Phi \text{ is a disjunction of}\\  \text{Horn formulas and $\Phi$ is satisfiable}\},
\end{multline*}
and the counting problem $\shdhsat$ as a function that counts all satisfying assignments of a formula $\Phi$ that is a disjunction of Horn formulas.

\begin{theorem} \label{sigma2hard}
	$\shdhsat$ is $\eqso(\ehorn)$-complete under parsimonious reductions. 
\end{theorem}
