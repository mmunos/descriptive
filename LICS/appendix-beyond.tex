\subsection*{Proof of Theorem \ref{tqfo-shl}}

Let $\R$ be some relational signature.  First we address the first condition in the Definition \ref{def:cap}. Let $\alpha$ be a formula in $\tqfo(\fo)$. We will construct a nondeterministic logspace algorithm $M_{\alpha}$ that on input $\enc(\A)$, where a first-order assignment $v$ is being stored in memory, accepts in $\sem{\alpha}(\A)$ paths. Suppose the domain of $\A$ is $A = \{1,\ldots,n\}$. The algorithm needs $c\cdot\log_2(n)$ bits of memory to store $v$, where $c$ is the total number of first-order variables in $\alpha$. If $\alpha = \varphi$, we check if $(\A,v)\models\varphi$ in deterministic logarithmic space, and accept if and only if it does. If $\alpha = s$, we generate $s$ branches and accept in all of them. If $\alpha = (\alpha_1 + \alpha_2)$, we simulate $M_{\alpha_1}$ and $M_{\alpha_2}$ on separate branches. If $\alpha = (\alpha_1\cdot\alpha_2)$, we simulate $\alpha_1$ and if it accepts, instead of doing so, we simulate $\alpha_2$. If $\alpha = \sa{x}\beta$, for each $a\in A$ we generate a different branch where we simulate $M_{\beta}$ while storing $v[a/x]$. If $\alpha = \pa{x}\beta$, we simulate $M_{\beta}$ while storing $v[1/n]$, and on each accepting branch, instead of accepting we replace the assignment on $x$ to 2, to simulate $M_{\beta}$ while storing $v[2/x]$, and so on. If $\alpha = [\pth \varphi(\bar{x},\bar{y})]$ where $\varphi$ is an $\fo$ formula, we simulate the $\shl$ procedure that counts the number of paths for a graph of a given size. This procedure, on each iteration, nondeterministically chooses an assignment $\bar{a}$ for $\bar{x}$, continues if $(\A,v)\models\varphi(\bar{a}',\bar{a})$ where $\bar{a}'$ is the previously chosen value, and rejects otherwise. This is repeated $n^{\length(\bar{x})}$ times, and it generates exactly $\sem{[\pth \varphi(\bar{x},\bar{y})]}(\A,v)$ accepting branches. This ends the construction of the algorithm. Consider $f$ as the $\shl$ function associated to this procedure and we have that for each finite $\R$-structure $\A$: $f(\enc(\A)) = \sem{\alpha}(\A)$.

\vspace{1em}
For the second condition, let $f \in \shl$. We will address the case where $\R$ contains only one binary predicate $E$, and the rest of the cases can be deduced from this. Let $M$ be a non-deterministic logspace machine such that $f(\A) = \acc_M(\A)$ for each $\A \in \ostr[R]$. Suppose ${\cal Q} = \{q_1,\ldots,q_{\ell}\}$ is the set of states of $M$, where $q_1$ is the initial state, and $q_{\ell}$ is only final state of $M$. Let $n = \vert A \vert$ and let $w = \enc(\A) \in \{0,1\}^{n^2}$. We assume that $M$ with input $w$ uses space $s_M(w) < c\cdot\log(n)$ and furthermore, $s_M(w) < n-2$. We notate $M(w)$ as the graph of configurations of $M$ running on input $w$.

We represent configurations with a tuple of fixed size. The formula $\varphi(\bar{x},\bar{y})$ describes a procedure that given a configuration generates a possible next configuration. The formula $\varphi_I(\bar{x})$ describes that $\bar{x}$ is the initial configuration of $M(w)$. The formula $\varphi_F(\bar{x})$ describes that $\bar{x}$ is an accepting (final) configuration of $M(w)$. The formula we construct is:
$$
\alpha = \sa{\bar{x}}\sa{\bar{y}}([\pth \varphi(\bar{x},\bar{y})]\cdot \varphi_I(\bar{x})\cdot\varphi_F(\bar{y})).
$$

To illustrate our idea, we will show a simplified example. Consider a machine $M$ that works in exactly $\log_2(n)$ space and only allows 0 or 1 in the working tape. Consider an input $\A$ of size 16 (that is, $A = \{0,\ldots,9,A,\ldots,F\}$). Let some configuration $s$ have 0011 in the working tape, the head in the input tape is in position 26, and the head in the input tape is in position 2 (we consider 0-indexed positions). Also, $Q = \{q_1,\ldots,q_5\}$ and the current state is $q_3$.

As a first approach, we will use a 9-tuple $\bar{a} = (a_1,\ldots,a_9)$ to represent $s$. That is, $(a_1,a_2) = (1,A)$ represent the position of the head in the input tape (since 1A equal to 26 in base 16), $a_3 = 2$ represents the position of the head in the working tape, $a_4 = C$ (1100b in base 16) represents the content of the working tape, and $(a_5,\ldots,a_9) = (0,0,1,0,0)$ represents the current state. Then $\bar{a} = (1,A,2,C,0,0,1,0,0)$ will represent $s$.

The problem that arises from this representation, is that to describe a transition in $M$ we need to read an arbitrary character in the working tape. In the example, this translates to obtaining the $a_3$-th bit in $a_4$. Furthermore, to represent the following configuration, we need compute $a_4$ with the $a_3$-th bit flipped. This is generally not possible to describe with an $\fo$ formula. To deal with this issue, consider the following procedure. (In the example it would receive $x = a_4$ and $i = a_3$.)

\begin{algorithm}
	\caption{If the $i$-th bit in $x$ is 1 replace it by 0 and return the result}
	\label{switch1to0}
	\begin{algorithmic}
		\State $u \gets x,\; j \gets i$ \Comment{Get the $i$-th bit on $x$ and store it in $u$}
		\While{$j > 0$}
		\State $v \gets 0$
		\While{$u > 1$}
		\State $u \gets u-2,\; v \gets v+1$
		\EndWhile
		\State $u\gets v,\; j \gets j-1$
		\EndWhile
		\While{$u > 1$}
		\State $u \gets u-2$
		\EndWhile
		\State $\textbf{assert } u = 1$ \Comment{If $u \neq 1$ simply stop}	
		\State $y \gets 1$ \Comment{Compute $2^i$ and store it in $y$}
		\While{$i > 0$}
		\State $z \gets 0$
		\While{$y > 0$}
		\State $z \gets z+2,\; y \gets y-1$
		\EndWhile
		\State $i \gets i-1,\; y \gets z$
		\EndWhile
		\While{$y > 0$} \Comment{Substract $y$ from $x$}
		\State $x \gets x-1,\; y \gets y-1$
		\EndWhile
		\State \Return $x$.
	\end{algorithmic}
\end{algorithm}	
Each of the instructions can be expressed with $\fo$, so our strategy is to use the $\pth$ operator to simulate the algorithm and then we can describe a transition using the processed value of $a_4$. This procedure simulates a transition that writes 1 in the cell where it read a 0. We call this a $1 \to 0$ transition. At the end of the proof we provide in detail three more procedures that simulate a $0\to 0$ transition, a $0\to 1$ transition, and a $1\to 1$ transition. The rest of the proof only addresses the case where we are simulating a $1\to 0$ transition, and the rest of the cases can be described analogously.

We will now describe how to simulate both the procedure and the transition. A procedure tuple $\bar{p} = (a_1,\ldots,a_{3+c+\ell},b_1,b_2,c_1,c_2,c_3,d_1,\ldots,d_{5c+2})$ represents the current configuration of $M(w)$ in $a_1,\ldots,a_{2+c+\ell}$, the values that will be read and written in the working tape in $b_1,b_2$, the instruction pointer in $c_1,c_2,c_3$ and the values stored in memory in $d_1,\ldots,d_{10c+2}$. In detail:
\begin{enumerate}
	\item $a_1,a_2$ and $a_3$ represent the position of the head in the input tape and the working tape, respectively, $a_4,\ldots,a_{3+c}$ represent the content of the working tape and $a_{4+c},\ldots,a_{3+c+\ell}$ represent the current state in the current configuration that is being processed.
	\item $b_1$ and $b_2$ are equal to the value that is being read in the working tape and the value that will be written in the working tape respectively.
	\item $c_1,c_2,c_2$ represent the instruction pointer in the procedure. Only 8 different instructions are needed in the simulation.
	\item Each value in memory of $x,y,z,u,v$ need $c$ elements to represent them and $i,j$ need only one. We map $(d_1\ldots,d_{c}) \to x$, $(d_{c+1}\ldots,d_{2c}) \to y$,
	$(d_{2c+1}\ldots,d_{3c}) \to z$, $(d_{3c+1}\ldots,d_{4c}) \to u$,
	$(d_{4c+1}\ldots,d_{5c}) \to v$, $d_{5c+1} \to i$ and $d_{5c+2}\to j$.
\end{enumerate}
For each transition $\delta \in \Delta \subseteq Q \times \{0,1\} \times \{0,1\} \times Q \times \{-1,=,+1\} \times \{0,1\} \times \{-1,=,+1\}$ we define a formula $\varphi_{\delta}(\bar{x},\bar{s},\bar{w},\bar{u},\bar{y},\bar{t},\bar{z},\bar{v})$, where $\bar{x} = (x_1,\ldots,x_{3+c+\ell})$, $\bar{s} = (s_1,s_2)$, $\bar{w} = (w_1,w_2,w_3)$, $\bar{u} = (u_1,\ldots,u_{5c+2})$, $\bar{y} = (y_1,\ldots,y_{3+c+\ell})$, $\bar{t} = (t_1,t_2)$, $\bar{z} = (z_1,z_2,z_3)$ and $\bar{v} = (v_1,\ldots,v_{5c+2})$. The tuples $\bar{x}$ and $\bar{y}$ represent the current and next configuration of $M$ respectively, $\bar{s}$ and $\bar{y}$ indicate which algorithm is being followed, $\bar{w}$ and $\bar{z}$ represent the current and next instruction of the algorithm, $\bar{u}$ and $\bar{v}$ represent the current and next values in memory. We will describe the formula part by part. Suppose $\delta = (q_i,a,1,q_j,op_1,0,op_2)$, so we have to simulate Algorithm \ref{switch1to0}.

First we define some auxiliary formulas:
\begin{align*}
\varphi_0(x) &= \neg\exists y(y < x), \\
\varphi_{+1}(x,y) &= x < y \wedge \neg\exists z(x < z \wedge z < y), \\
\varphi_{-1}(x,y) &= \varphi_{+1}(y,x), \\
\varphi_1(x) &= \exists y(\varphi_0(y) \wedge \varphi_{+1}(y,x)), \\
\text{for $i,j \in\{0,1\}$, }	\varphi_{i,j}(x,y) &= \varphi_i(x) \wedge \varphi_j(y), \\
\text{for each $k \leq 7$, }\varphi^b_k(x_1,x_2,x_3) &= \varphi_{a_1}(x_1) \wedge \varphi_{a_2}(x_2) \wedge \varphi_{a_3}(x_3)\text{, where $a_1a_2a_3$ is the value of $k$ in binary}, \\
\mu_{<}(x_1,\ldots,x_c,y_1,\ldots,y_c) &= \bigvee_{i = 1}^c ((\bigwedge_{j = 1}^{i-1} x_j = y_j) \wedge x_i < y_i), \\
\mu_{+1}(\bar{x},\bar{y}) &= \mu_{<}(\bar{x},\bar{y}) \wedge \neg\exists \bar{z}(\mu_{<}(\bar{x},\bar{z}) \wedge \mu_{<}(\bar{z},\bar{y})), \\
\mu_{+2}(\bar{x},\bar{y}) &= \exists\bar{z}(\mu_{+1}(\bar{x},\bar{z}) \wedge \mu_{+1}(\bar{z},\bar{y})),\\
\mu_{-1}(\bar{x},\bar{y}) &= \varphi_{+1}(\bar{y},\bar{x}),\\
\mu_{-2}(\bar{x},\bar{y}) &= \varphi_{+2}(\bar{y},\bar{x}),\\
\mu_{0}(\bar{x}) &= \neg\exists\bar{y}(\mu_{<}(\bar{y},\bar{x})),\\
\mu_{1}(\bar{x}) &= \exists\bar{y}(\mu_{0}(\bar{y})\wedge\mu_{+1}(\bar{y},\bar{x})),\\
\varphi^2_{<}(x_1,x_2,y_1,y_2) &= x_1 < y_1 \vee (x_1 = y_1 \wedge x_2 < y_2),\\
\varphi^2_{+1}(x_1,x_2,y_1,y_2), \varphi^2_{-1}(x_1,x_2,y_1,y_2) &\text{ defined analogously to $\mu_{+1}$ and $\mu_{-1}$},\\
\text{for each } q_i\in Q, \varphi^q_i(x_1,\ldots,x_{\ell}) &= \bigwedge_{\substack{j = 1 \\ j \neq i}}^{\ell} \varphi_0(x_j) \wedge \varphi_1(x_i), \\
\varphi^E_0(x_1,x_2) &= \neg E(x_1,x_2),\\		\varphi^E_1(x_1,x_2) &= E(x_1,x_2),\\
\end{align*}

We start from instruction 0, which means that the procedure has not started yet and every value in the tuple is 0 except for the configuration values. It also initializes all the values in the tuple to 0 except for $x,u,i,j$.
\begin{multline*}
\varphi^{0,1}_{\delta}(\bar{x},\bar{s},\bar{w},\bar{u},\bar{y},\bar{t},\bar{z},\bar{v}) = \varphi_{0,0}(s_1,s_2)\wedge\varphi^b_0(\bar{w}) \wedge \varphi_{1,0}(t_1,t_2) \wedge \varphi^b_1(\bar{z})\, \wedge \\ 
\bigwedge_{i = 1}^c v_i = x_{3+i} \wedge \bigwedge_{i = c+1}^{2c} \varphi_0(v_i) \wedge \bigwedge_{i = 2c+1}^{3c} \varphi_0(v_i) \wedge \bigwedge_{i = 1}^c v_{3c+i} = x_{3+i} \wedge \bigwedge_{i = 4c+1}^{5c} \varphi_0(v_i) \wedge v_{5c+1} = x_3 \wedge v_{5c+2} = x_3.
\end{multline*}
Instruction 1 which checks whether the value of $j$ ($d_{5c+2}$ in the tuple) is more than 0 or not, and then proceeds to instruction 2 or 3 on each case.
\begin{multline*}
\varphi^{1,2}_{\delta}(\bar{x},\bar{s},\bar{w},\bar{u},\bar{y},\bar{t},\bar{z},\bar{v}) = 
\varphi_{1,0}(s_1,s_2) \wedge \varphi^b_1(\bar{w}) \wedge \neg \varphi_0(u_{5c+2}) \wedge \varphi_{1,0}(t_1,t_2) \wedge \varphi^b_2(\bar{z}) \wedge \bigwedge_{i = 1}^{5c+2} u_i = v_i, \\
\varphi^{1,3}_{\delta}(\bar{x},\bar{s},\bar{w},\bar{u},\bar{y},\bar{t},\bar{z},\bar{v}) = 
\varphi_{1,0}(s_1,s_2) \wedge \varphi^b_1(\bar{w}) \wedge \varphi_0(u_{5c+2}) \wedge \varphi_{1,0}(t_1,t_2) \wedge \varphi^b_3(\bar{z}) \wedge \bigwedge_{i = 1}^{5c+2} u_i = v_i.
\end{multline*}
Instruction 2 checks the value of $u$ ($d_{3c+1},\ldots,d_{4c}$ in the tuple). If it is $> 1$ then it substracts 2 from $u$ and adds 1 to $v$ ($d_{4c+1},\ldots,d_{5c}$ in the tuple), then repeats instruction 2. If it is equal to 0 or 1, then moves the value of $v$ to $u$, substracts 1 from $j$ and goes back to instruction 1.
\begin{multline*}
\varphi^{2,2}_{\delta}(\bar{x},\bar{s},\bar{w},\bar{u},\bar{y},\bar{t},\bar{z},\bar{v}) = 
\varphi_{1,0}(s_1,s_2) \wedge \varphi^b_2(\bar{w}) \wedge \neg \mu_0(u_{3c+1},\ldots,u_{4c}) \wedge \neg \mu_1(u_{3c+1},\ldots,u_{4c}) \wedge \varphi_{1,0}(t_1,t_2) \wedge 	\varphi^b_2(\bar{z})\, \wedge \\
\bigwedge_{i = 1}^c u_i = v_i \wedge
\bigwedge_{i = c+1}^{2c} u_i = v_i \wedge
\bigwedge_{i = 2c+1}^{3c} u_i = v_i \wedge
\mu_{-2}(u_{3c+1},\ldots,u_{4c},v_{3c+1},\ldots,v_{4c})\, \wedge \\
\mu_{+1}(u_{4c+1},\ldots,u_{5c},v_{4c+1},\ldots,v_{5c}) \wedge u_{5c+1} = v_{5c+1} \wedge u_{5c+2} = v_{5c+2}. \\
\varphi^{2,1}_{\delta}(\bar{x},\bar{s},\bar{w},\bar{u},\bar{y},\bar{t},\bar{z},\bar{v}) = \varphi_{1,0}(s_1,s_2) \wedge \varphi^b_2(\bar{w}) \wedge ( \mu_0(u_{3c+1},\ldots,u_{4c}) \vee \mu_1(u_{3c+1},\ldots,u_{4c})) \wedge \varphi_{1,0}(t_1,t_2) \wedge 	\varphi^b_1(\bar{z})\, \wedge \\
\bigwedge_{i = 1}^c u_i = v_i \wedge
\bigwedge_{i = c+1}^{2c} u_i = v_i \wedge
\bigwedge_{i = 2c+1}^{3c} u_i = v_i \wedge
\bigwedge_{i = 3c+1}^{4c} u_{c+i} = v_i \wedge
\mu_0(v_{4c+1},\ldots,v_{5c}) \wedge
u_{5c+1} = v_{5c+1} \wedge \varphi_{-1}(u_{5c+2},v_{5c+2}).	
\end{multline*}
Instruction 3 calculates the value of $u \mod 2$, that is, it repeats instruction 3 until the value of $u$ is equal to 0 or 1. On each iteration, it substracts 2 from $u$. Moreover, if the value of $u$ at the end of the iterations is not 1 then there is no step defined.
\begin{multline*}
\varphi^{3,3}_{\delta}(\bar{x},\bar{s},\bar{w},\bar{u},\bar{y},\bar{t},\bar{z},\bar{v}) = \varphi_{1,0}(s_1,s_2) \wedge \varphi^b_3(\bar{w}) \wedge \neg \mu_0(u_{3c+1},\ldots,u_{4c}) \wedge \neg \mu_1(u_{3c+1},\ldots,u_{4c}) \wedge \varphi_{1,0}(t_1,t_2) \wedge 	\varphi^b_3(\bar{z})\, \wedge \\
\bigwedge_{i = 1}^c u_i = v_i \wedge
\bigwedge_{i = c+1}^{2c} u_i = v_i \wedge
\bigwedge_{i = 2c+1}^{3c} u_i = v_i \wedge
\mu_{-2}(u_{3c+1},\ldots,u_{4c},v_{3c+1},\ldots,v_{4c}) \wedge
\bigwedge_{i = 4c+1}^{5c} u_i = v_i \wedge
u_{5c+1} = v_{5c+1} \wedge u_{5c+2} = v_{5c+2}, \\
\varphi^{3,3}_{\delta}(\bar{x},\bar{s},\bar{w},\bar{u},\bar{y},\bar{t},\bar{z},\bar{v}) = 
\varphi_{1,0}(s_1,s_2) \wedge \varphi^b_3(\bar{w}) \wedge \mu_1(u_{3c+1},\ldots,u_{4c}) \wedge \varphi_{1,0}(t_1,t_2) \wedge 	\varphi^b_4(\bar{z})\, \wedge \\
\bigwedge_{i = 1}^c u_i = v_i \wedge
\mu_1(v_{c+1},\ldots,v_{2c}) \wedge
\bigwedge_{i = 2c+1}^{3c} u_i = v_i \wedge
\bigwedge_{i = 3c+1}^{4c} u_i = v_i \wedge
\bigwedge_{i = 4c+1}^{5c} u_i = v_i \wedge
u_{5c+1} = v_{5c+1} \wedge u_{5c+2} = v_{5c+2}
\end{multline*}
Instruction 4 checks the value of $i$ ($d_{5c+1}$ in the tuple.) If it is not 0 then goes to instruction 5 and if is 0 then goes to instruction 6. Moreover it initializes the value of $z$ ($d_{2c+1},\ldots,d_{3c}$ in the tuple) to 0 (which was 0 all along.)
\begin{multline*}
\varphi^{4,5}_{\delta}(\bar{x},\bar{s},\bar{w},\bar{u},\bar{y},\bar{t},\bar{z},\bar{v}) = \varphi_{1,0}(s_1,s_2) \wedge \varphi^b_4(\bar{w}) \wedge \neg \varphi_0(u_{5c+1}) \wedge \varphi_{1,0}(t_1,t_2) \wedge 	\varphi^b_5(\bar{z}) \wedge \bigwedge_{i = 1}^{5c+2} u_i = v_i, \\
\varphi^{4,6}_{\delta}(\bar{x},\bar{s},\bar{w},\bar{u},\bar{y},\bar{t},\bar{z},\bar{v}) = \varphi_{1,0}(s_1,s_2) \wedge \varphi^b_4(\bar{w}) \wedge \varphi_0(u_{5c+1}) \wedge \varphi_{1,0}(t_1,t_2) \wedge 	\varphi^b_6(\bar{z}) \wedge \bigwedge_{i = 1}^{5c+2} u_i = v_i.
\end{multline*}
Instruction 5 checks the value of $y$ ($d_{c+1},\ldots,d_{2c}$ in the tuple.) If it is more than 0 then it adds 2 to $z$ and substracts 1 from $y$, then repeats instruction 2. If it is not, then copies the value of $z$ to $y$ and subtracts 1 from $i$ and returns to instruction 4.
\begin{multline*}
\varphi^{5,5}_{\delta}(\bar{x},\bar{s},\bar{w},\bar{u},\bar{y},\bar{t},\bar{z},\bar{v}) = \varphi_{1,0}(s_1,s_2) \wedge \varphi^b_5(\bar{w}) \wedge \neg \mu_0(u_{c+1},\ldots,u_{2c}) \wedge \varphi_{1,0}(t_1,t_2) \wedge \varphi^b_5(\bar{z}) \, \wedge \\
\bigwedge_{i = 1}^c u_i = v_i \wedge
\mu_{-1}(u_{c+1},\ldots,u_{2c},v_{c+1},\ldots,v_{2c}) \wedge
\mu_{+2}(u_{2c+1},\ldots,u_{3c},v_{2c+1},\ldots,v_{3c})\, \wedge \\
\bigwedge_{i = 3c+1}^{4c} u_i = v_i \wedge
\bigwedge_{i = 4c+1}^{5c} u_i = v_i \wedge
u_{5c+1} = v_{5c+1} \wedge u_{5c+2} = v_{5c+2}, \\
\varphi^{5,4}_{\delta}(\bar{x},\bar{s},\bar{w},\bar{u},\bar{y},\bar{t},\bar{z},\bar{v}) = \varphi_{1,0}(s_1,s_2) \wedge \varphi^b_5(\bar{w}) \wedge \mu_0(u_{c+1},\ldots,u_{2c}) \wedge \varphi_{1,0}(t_1,t_2) \wedge \varphi^b_4(\bar{z}) \, \wedge \\
\bigwedge_{i = 1}^c u_i = v_i \wedge
\bigwedge_{i = c+1}^{2c} u_{c+i} = v_i \wedge
\mu_0(v_{2c+1},\ldots,v_{3c}) \wedge
\bigwedge_{i = 3c+1}^{4c} u_i = v_i \wedge
\bigwedge_{i = 4c+1}^{5c} u_i = v_i \wedge
\varphi_{-1}(u_{5c+1},v_{5c+1}) \wedge u_{5c+2} = v_{5c+2}
\end{multline*}
Instruction 6 checks the value of $y$. If it is more than 0, then subtracts 1 from $x$ and $y$ and repeats instruction 6. If it is not, then goes to instruction 7.
\begin{multline*}
\varphi^{6,6}_{\delta}(\bar{x},\bar{s},\bar{w},\bar{u},\bar{y},\bar{t},\bar{z},\bar{v}) = \varphi_{1,0}(s_1,s_2) \wedge \varphi^b_6(\bar{w}) \wedge \neg \mu_0(u_{c+1},\ldots,u_{2c}) \wedge \varphi_{1,0}(t_1,t_2) \wedge \varphi^b_6(\bar{z}) \, \wedge \\
\mu_{-1}(u_1,\ldots,u_c,v_i,\ldots,v_c) \wedge \mu_{-1}(u_{c+1},\ldots,u_{2c},v_{c+1},\ldots,v_{2c}) \wedge \bigwedge_{i = 3c+1}^{5c+2} u_i = v_i]\,\vee \\
\varphi^{6,7}_{\delta}(\bar{x},\bar{s},\bar{w},\bar{u},\bar{y},\bar{t},\bar{z},\bar{v}) = \varphi_{1,0}(s_1,s_2) \wedge \varphi^b_6(\bar{w}) \wedge \mu_0(u_{c+1},\ldots,u_{2c}) \wedge \varphi_{1,0}(t_1,t_2) \wedge \varphi^b_7(\bar{z}) \wedge \bigwedge_{i = 1}^{5c+2} u_i = v_i.
\end{multline*}
Instruction 7 stores the value of $x$ after the corresponding bit has been switched. Then we can define $\varphi_{\delta}$ which also simulates the actual transition. If $u$ equals 1, then copy what is stored in $x$ to $a_4,\ldots,a_{3+c}$, go from state $q_i$ to state $q_j$, and move the heads to their corresponding positions.
\begin{multline*}
\varphi_{\delta}(\bar{x},\bar{s},\bar{w},\bar{u},\bar{y},\bar{t},\bar{z},\bar{v}) = 	[\bigwedge_{i = 1}^{3+c+\ell} x_i = y_i \wedge (\varphi^{0,1}(\bar{x},\bar{s},\bar{w},\bar{u},\bar{y},\bar{t},\bar{z},\bar{v}) \vee \varphi^{1,2}(\bar{x},\bar{s},\bar{w},\bar{u},\bar{y},\bar{t},\bar{z},\bar{v})\, \vee \\ \varphi^{1,3}(\bar{x},\bar{s},\bar{w},\bar{u},\bar{y},\bar{t},\bar{z},\bar{v}) \vee \varphi^{2,2}(\bar{x},\bar{s},\bar{w},\bar{u},\bar{y},\bar{t},\bar{z},\bar{v}) \vee \varphi^{2,1}(\bar{x},\bar{s},\bar{w},\bar{u},\bar{y},\bar{t},\bar{z},\bar{v}) \vee \varphi^{3,3}(\bar{x},\bar{s},\bar{w},\bar{u},\bar{y},\bar{t},\bar{z},\bar{v}) \vee \\ \varphi^{3,4}(\bar{x},\bar{s},\bar{w},\bar{u},\bar{y},\bar{t},\bar{z},\bar{v}) \vee \varphi^{4,5}(\bar{x},\bar{s},\bar{w},\bar{u},\bar{y},\bar{t},\bar{z},\bar{v}) \vee \varphi^{4,6}(\bar{x},\bar{s},\bar{w},\bar{u},\bar{y},\bar{t},\bar{z},\bar{v}) \vee \varphi^{5,5}(\bar{x},\bar{s},\bar{w},\bar{u},\bar{y},\bar{t},\bar{z},\bar{v}) \vee \\ \varphi^{5,6}(\bar{x},\bar{s},\bar{w},\bar{u},\bar{y},\bar{t},\bar{z},\bar{v}) \vee \varphi^{6,6}(\bar{x},\bar{s},\bar{w},\bar{u},\bar{y},\bar{t},\bar{z},\bar{v}) \vee \varphi^{6,7}(\bar{x},\bar{s},\bar{w},\bar{u},\bar{y},\bar{t},\bar{z},\bar{v}))] \, \vee \\
[\varphi^E_a(x_1,x_2) \wedge \varphi_{1,0}(s_1,s_2) \wedge \varphi^b_7(\bar{w}) \wedge \mu_1(u_{3c+1},\ldots,u_{4c}) \wedge \varphi_{0,0}(t_1,t_2) \wedge \varphi^b_0(\bar{z}) \wedge \bigwedge_{i = 1}^{5c+2} u_i = v_i \,\wedge \\
\varphi^2_{op_1}(x_1,x_2,y_1,y_2) \wedge \varphi_{op_2}(x_3,y_3) \wedge \bigwedge_{i = 1}^c u_i = x_{3+i} \wedge \varphi^q_i(x_{4+c},\ldots,x_{3+c+\ell}) \wedge \varphi^q_j(y_{4+c},\ldots,y_{3+c+\ell})].
\end{multline*}

Note that we also need to specify that the program we are following is Algorithm \ref{switch1to0} so we store $1,0$ in $b_1,b_2$ all along the procedure. We describe the three other algorithms that compute the switches from $0\to 0$, $0\to 1$ and $1\to 1$ (Algorithms \ref{switch0to0}, \ref{switch0to1} and \ref{switch1to1}.)
For the other three cases, where $\delta = (q_i,a,0,q_j,op_1,0,op_2)$, $\delta = (q_i,a,0,q_j,op_1,1,op_2)$ and $\delta = (q_i,a,1,q_j,op_1,1,op_2)$, $\varphi_{\delta}(\bar{x},\bar{s},\bar{w},\bar{u},\bar{y},\bar{t},\bar{z},\bar{v})$ is defined analogously. Then, $\varphi(\bar{x},\bar{s},\bar{w},\bar{u},\bar{y},\bar{t},\bar{z},\bar{v})$ is defined as:
$$
\varphi(\bar{x},\bar{s},\bar{w},\bar{u},\bar{y},\bar{t},\bar{z},\bar{v}) = \bigvee_{\delta \in \Delta} \varphi_{\delta}(\bar{x},\bar{s},\bar{w},\bar{u},\bar{y},\bar{t},\bar{z},\bar{v}).
$$
Finally we define $\varphi_I$ and $\varphi_F$:
\begin{align*}
\varphi_I(\bar{x},\bar{s},\bar{w},\bar{u}) &= \varphi_0(x_1) \wedge \varphi_0(x_2) \wedge \varphi_0(x_3) \wedge \bigwedge_{i = 4}^{3+c}\varphi_0(x_i) \wedge \varphi_1(x_{4+c}) \wedge \bigwedge_{i = 5+c}^{3+c+\ell} \varphi_0(x_i)\wedge \varphi_{0,0}(s_1,s_2)\ \wedge \varphi^b_0(\bar{w}) \wedge \bigwedge_{i = 1}^{5c+2}\varphi_0(u_i). \\
\varphi_F(\bar{x},\bar{s},\bar{w},\bar{u}) &= \varphi^q_{\ell}(x_{4+c},\ldots,x_{3+c+\ell}) \wedge \varphi_{0,0}(s_1,s_2)\ \wedge \varphi^b_0(\bar{w}) \wedge \bigwedge_{i = 1}^{5c+2}\varphi_0(u_i),
\end{align*}
and then $\sem{\alpha}(\A) = \sem{\sa{\bar{x}}\sa{\bar{y}}([\pth \varphi(\bar{x},\bar{y})]\cdot \varphi_I(\bar{x})\cdot\varphi_F(\bar{y}))}(\A) = \acc_M(\A)$.


\begin{algorithm}
	\caption{If the $i$-th bit in $x$ is 0 return $x$} \label{switch0to0}
	\begin{algorithmic}
		\State $u \gets x,\; j \gets i$ \Comment{Get the $i$-th bit on $x$ and store it in $u$}
		\While{$j > 0$}
		\State $v \gets 0$
		\While{$u > 1$}
		\State $u \gets u-2,\; v \gets v+1$
		\EndWhile
		\State $u\gets v,\; j \gets j-1$
		\EndWhile
		\While{$u > 1$}
		\State $u \gets u-2$
		\EndWhile
		\State $\textbf{assert } u = 0$ \Comment{If $u \neq 0$ simply stop}	
		\State \Return $x$.
	\end{algorithmic}
\end{algorithm}

\begin{algorithm}
	\caption{If the $i$-th bit in $x$ is 0 replace it by 1 and return the result}
	\label{switch0to1}
	\begin{algorithmic}
		\State $u \gets x,\; j \gets i$ \Comment{Get the $i$-th bit on $x$ and store it in $u$}
		\While{$j > 0$}
		\State $v \gets 0$
		\While{$u > 1$}
		\State $u \gets u-2,\; v \gets v+1$
		\EndWhile
		\State $u\gets v,\; j \gets j-1$
		\EndWhile
		\While{$u > 1$}
		\State $u \gets u-2$
		\EndWhile
		\State $\textbf{assert } u = 0$ \Comment{If $u \neq 0$ simply stop}	
		\State $y \gets 1$ \Comment{Compute $2^i$ and store it in $y$}
		\While{$i > 0$}
		\State $z \gets 0$
		\While{$y > 0$}
		\State $z \gets z+2,\; y \gets y-1$
		\EndWhile
		\State $i \gets i-1,\; y \gets z$
		\EndWhile
		\While{$y > 0$} \Comment{Add $y$ to $x$}
		\State $x \gets x+1,\; y \gets y-1$
		\EndWhile
		\State \Return $x$.
	\end{algorithmic}
\end{algorithm}	

\begin{algorithm}
	\caption{If the $i$-th bit in $x$ is 1 return $x$}
	\label{switch1to1}
	\begin{algorithmic}
		\State $u \gets x,\; j \gets i$ \Comment{Get the $i$-th bit on $x$ and store it in $u$}
		\While{$j > 0$}
		\State $v \gets 0$
		\While{$u > 1$}
		\State $u \gets u-2,\; v \gets v+1$
		\EndWhile
		\State $u\gets v,\; j \gets j-1$
		\EndWhile
		\While{$u > 1$}
		\State $u \gets u-2$
		\EndWhile
		\State $\textbf{assert } u = 1$ \Comment{If $u \neq 0$ simply stop}	
		\State \Return $x$.
	\end{algorithmic}
\end{algorithm}







\subsection*{}