%!TEX root = main.tex

\marcelo{Ahora solo consideramos la idea de capturar sobre estructuras ordenadas}

Assume that $\alpha$ is a sentence in $\qso$. Then slightly abusing notation, we also use $\alpha$ to denote a function defined over $\R$  such that for every $\A \in \ostr[\R] $:
\begin{eqnarray*}
	\alpha(\enc(\A)) & = & \sem{\alpha}(\A).
\end{eqnarray*}
\begin{definition}
	Let $\FF$ a fragment of $\qso$ and $\CC$ a function complexity class. Then $\FF$ {\em captures} $\CC$ if:
	\begin{itemize}
		\item for every $\alpha$ in $\FF$, there exists $f \in \CC$ such that for every 
		$\A \in \ostr[\R]$:  $\alpha(\enc(\A)) = f(\enc(\A))$; and
		
		\item for every $f \in \CC$, there exists $\alpha$ in $\FF$ such that for every 
		$\A \in \ostr[\R]$:  $f(\enc(\A)) = \alpha(\enc(\A))$.
	\end{itemize}
\end{definition}
%In the previous definition, if $\KK$ is the class of all finite structures, then we say that $\FF$ captures $\CC$ over the class of all structures, and if $\KK$is the class of all ordered finite structures, then we say that $\FF$ captures $\CC$ over the class of ordered structures.

\begin{theorem} \label{captfp}
	$\qfo(\lfp)$ captures $\fp$ over the class of ordered structures.
\end{theorem}

\begin{theorem} \label{qfo-pfp-cap}
	$\qfo(\pfp)$ captures $\nfpspace$ over the class of ordered structures.
\end{theorem}

\begin{theorem} \label{eqso-fo-cap}
	$\eqso(\fo)$ captures $\shp$ over the class of ordered structures.
\end{theorem}

\begin{theorem} \label{eqso-eso-cap}
	$\eqso(\eso)$ captures $\spp$ over the class of ordered structures.
\end{theorem}
