%!TEX root = main.tex

In this section, we show that by syntactically restricting $\qso$ one can capture different function complexity classes. 
In other words, by using $\qso$ we can extend the theory of descriptive complexity~\cite{immerman2012descriptive} from decision problems to counting problems. 
For this, we first formalize the notion of \emph{capturing} a function complexity class to then show how to capture classes like $\shp$, $\fp$, and $\fpspace$.

Fix a relational signature $\R = \{R_1, \ldots, R_k\}$ and assume that $\A$ is an ordered finite $\R$-structure with a domain $A$ containing $n$ elements. Recall that  $<$ is a linear order on $A$, say $a_1 < a_2 < \ldots < a_n$. For every $i \in \{1, \ldots, k\}$, define the encoding of $R_i^\A$, denoted by $\enc(R_i^\A)$, as the following binary string. Assume that $\ell = \arity(R_i)$ and consider an enumeration of the $\ell$-tuples over $A$ in the lexicographic order induced by $<$. 
%(that is, $(a_1, \ldots, a_1, a_1)$, $(a_1, \ldots, a_1, a_2)$, $\ldots$, $(a_n, \ldots, a_n, a_{n-1})$, $(a_n, \ldots, a_n, a_n)$). 
Then let $\enc(R_i^\A)$ be a binary string of length $n^\ell$ such that the $i$-th bit of $\enc(R_i^\A)$ is 1 if the $i$-th tuple in the previous enumeration belongs to $R_i^\A$, and 0 otherwise. Moreover, define the encoding of $\A$, denoted by $\enc(\A)$, as the following binary string~\cite{L04}:
\begin{eqnarray*}
	\enc(\A) & = & 0^n \, 1 \, \enc(R_1^\A) \, \cdots \, \enc(R_k^\A).
\end{eqnarray*}
%We define the class of all $\R$-functions, denoted by $\Func(\R)$, as the class of all functions $f: \ostr \rightarrow \bbN$.
%Given a function complexity class $\CC$ (i.e. $f: \Sigma^* \rightarrow \bbN$ for every $f \in \CC$), we say that a function $f \in \Func(\R)$ can be computed in $\CC$ if there exists $g \in \CC$ such that $f(\A) = g(\enc(\A))$ for every $\A \in \ostr$. 
%Note that the function $g$ outputs $f$ for encodings of structures and can behave arbitrarily otherwise.
We can now formalize the notion of capturing a function complexity class over $\R$-structures.
\begin{definition} \label{cap-def}
	Let $\FF$ be a fragment of $\qso$ and $\CC$ a function complexity class. Then we say that {\em  $\FF$ captures $\CC$ over $\R$-structures} if the  following conditions are true:
	\begin{enumerate}
		\item for every $\alpha \in \FF$, there exists $f \in \CC$ such that $\sem{\alpha}(\A) = f(\enc(\A))$ for every $\A \in \ostr[\R]$. 
		
		\item for every $f \in \CC$, there exists $\alpha \in \FF$ such that   $f(\enc(\A)) = \sem{\alpha}(\A)$ for every $\A \in \ostr[\R]$.
	\end{enumerate} 
	Finally, we say that $\FF$ captures $\CC$ if $\FF$ captures $\CC$ over $\R$-structures for every relational signature $\R$.
\end{definition}
%For the sake of simplification, we denote the first condition by $\FF \subseteq \CC$ and the second condition by $\CC \subseteq \FF$.
Note that in Definition~\ref{cap-def} the function $f \in \CC$ and the formula $\alpha \in \FF$ must coincide in all the strings that encode $\R$-structures. Since we want to capture $\CC$ over a fix set of structures (e.g. graphs, matrices), it is natural to just consider strings that encodes $\R$-structures. 
This definition is fairly standard in descriptive complexity \cite{L04,immerman2012descriptive} and all notions for capturing complexity classes restrict $f \in \CC$ similarly. 

Then, what functional complexity classes can be captured with fragments of $\qso$?
For answering this question, it is reasonable to start with $\shp$ who is one of the most well-known functional complexity class~\cite{arora2009computational}. 
Since $\shp$ has a strong similarity with $\np$, one could expect a ``Fagin-like'' Theorem~\cite{fagin1974generalized} for this class. 
Actually, in~\cite{SalujaST95} it was shown that the class $\sfo$ captures $\shp$.
In our setting, the class $\sfo$ is contained in $\eqso(\fo)$ which also captures $\shp$ as expected. 
\begin{proposition} \label{prop:capture-shP}
	$\eqso(\fo)$ captures $\shp$ over ordered structures.
\end{proposition}
Recall that (see Section~\ref{sec:previous}) that every function class $\# \LL$ is contained in $\eqso(\LL)$. Thus, by~\cite{SalujaST95} it directly follows that every $\shp$-function can be defined in $\eqso(\LL)$. The other direction is also direct and follows by showing that $\shp$ is closed under first- and second-order sum.
Moreover, these closure properties also hold if we change the boolean core $\fo$ to any polynomial-time logic like $\lfp$. This gives the following corollary.
\begin{corollary} \label{eqso-lfp-cap}
	$\eqso(\lfp)$ captures $\shp$ over ordered structures.
\end{corollary}
By following the same approach as~\cite{SalujaST95}, Compton et al.~\cite{ComptonG96} show that $\seso$ captures $\spp$. As one could expect, if we parametrize $\eqso$ with $\eso$, we can also capture~$\spp$.
\begin{proposition} \label{eqso-eso-cap}
	$\eqso(\eso)$ captures $\spp$ over ordered structures.
\end{proposition}
The previous result connecting fragments from $\# \LL$ with $\shp$ and $\spp$ is the further that we can go by using SO-formulas with free variables. 
So, what happen with other counting complexity classes like $\fp$ and $\fpspace$?
By trying to search for an $\# \LL$ class for capturing $\fp$ one can easily see that this is pointless.
Indeed, by using free SO-variables one will easily capture $\shp$-hard functions (see Section~\ref{sec:syntactic}) and by only using free FO-variables one cannot capture exponential functions definable in $\fp$ (e.g. $2^{n}$).
Therefore, it is impossible to capture classes like $\fp$ or $\fpspace$ if we follow previous approaches like~\cite{SalujaST95}. 

Instead, if we move from $\eqso$ we have other options for counting like first- and second-order products. Actually, if we combine $\qfo$ with $\lfp$ this is exactly what we need to capture $\fp$, the class of function that can be computed in~$\ptime$.
\begin{theorem} \label{theo:capture-fp}
	$\qfo(\lfp)$ captures $\fp$ over ordered structures.
\end{theorem}
To capture $\fp$ one first shows that any formula in $\qfo(\lfp)$ can be evaluated in $\ptime$. 
Indeed, $\lfp$ is a polynomial time logic~\cite{Immerman82,vardi1982complexity} and the sum and product quantifiers can be computed in $\fp$. 
For the other direction, one has to use $\lfp$ to simulate the run of the $\fp$-machine and use the quantitative quantifiers to reconstruct the binary number built in the tape. 
It is important to notice that the alternation between sum and products quantifiers is crucial for capturing $\fp$.
That is, without these operators is difficult to capture $\fp$.

Another natural function complexity class is $\fpspace$~\cite{Ladner89}, the functional version of $\pspace$. 
Of course, for capturing this class one needs a logical core powerful enough, like $\pfp$, for simulating the run of an $\pspace$-machine. 
On the other hand, an $\fpspace$ machine can generate outputs of exponentially many bits. 
For this reason, the second-order sum is not enough for the quantitative layer of a logic for $\fpspace$, that is, one needs to consider second-order products to produce big numbers. 
This leads us to the fragment $\qso(\pfp)$ which naturally captures $\fpspace$. 
\begin{theorem} \label{qso-pfp-cap}
	$\qso(\pfp)$ captures $\fpspace$ over ordered structures.
\end{theorem}
The proof of the previous theorem follows the same lines than the logical characterization of $\fp$: one shows that each function in $\qso(\pfp)$ can be computed in $\fpspace$ and, conversely, the output of an $\fpspace$ can be reconstructed by using $\pfp$ and second-order quantitative quantifiers.

From the proof of the previous theorem a natural question follows: what happen if we use first-order quantitative quantifiers and $\pfp$?
In~\cite{Ladner89}, Ladner also proposes the class $\nfpspace$ of all functions computed by $\pspace$-machines whose output is polynomially bounded. 
Interestingly, if we restrict to FO-quantitative quantifiers we can also capture this class.
\begin{corollary} \label{cor:capture-fpspace-poly}
	$\qfo(\pfp)$ captures $\nfpspace$ over ordered structures.
\end{corollary}

The results in this section validate $\qso$ as the logical framework for extending the theory of Descriptive Complexity to function computational complexity. In the next sections, we study subfragments of $\eqso$ and, moreover, show how to go beyond $\eqso$ to capture other computational classes.

