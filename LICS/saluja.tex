% We are interested in, for each $\fo$-fragment $\LL$, the biggest fragment of $\eqso$ that is contained in $\#\LL$.
%Let $\LL = \loge{0}$:
%\begin{theorem} \label{one-sigma-zero}
%	Positive constant functions are not expressible in $\E{0}$
%\end{theorem}
%\begin{corollary}
%	If a fragment of $\eqso(\loge{0})$ is contained in $\E{0}$, its grammar does not allow sole constants.
%\end{corollary}
%\begin{conjecture}
%	Sum is not expressible in $\E{0}$
%\end{conjecture}
%\begin{theorem} \label{mult-sigma-zero}
%	$\sqso(\loge{0})$ with binary product is contained in $\E{0}$.
%\end{theorem}
%\begin{theorem} \label{fo-prod-sigma-zero}
%	If an extension of $\sqso(\loge{0})$ is contained in $\E{0}$, its grammar does not allow first-order product.
%\end{theorem}


%For every logic $\LL$, we define an $\LL$-extended quantifier-free (QF) formula as follows:
%\begin{eqnarray*}
%	\varphi &::=& \alpha, \alpha \text{ is an $\LL$-formula} \ \mid \\
%	&& X_i(x_1,\dots,x_{a_i}), i\in\N \ \mid \ \\
%	&& (\neg \varphi) \ \mid \ (\varphi \wedge \varphi) \ \mid \ (\varphi \vee \varphi).
%\end{eqnarray*}
%
%We define syntactically the fragments $\logex{i}$ and $\logux{i}$ according to the following grammar:
%\begin{align*}
%\logex{0} = \logux{0} &::= \varphi , \varphi \mbox{ is an $\fo$-extended QF formula,} \\
%\logex{i+1} &::= \logux{i} \ \mid \ \exists x\, \logex{i+1}, \\
%\logux{i+1} &::= \logex{i} \ \mid \ \forall x\, \logux{i+1}.
%\end{align*}

We see that many of the results in Saluja et. al. \cite{DBLP:journals/jcss/SalujaST95} for $\#\LL$ still apply in $\eqso(\LL)$ for a given fragment $\LL$:

\begin{theorem} \label{eqso-sigma-zero-in-fp}
	For every $\eqso(\loge{0})$ formula $\alpha$ over a signature $\R$, the function $f$ over $\R$ defined as $f(\enc(\A)) = \sem{\alpha}(\A)$ is in $\fp$.
\end{theorem}

\begin{theorem} \label{eqso-sigma-one-in-eqso-pi-one}
	For every $\eqso(\loge{1})$ formula $\alpha$ over a signature $\R$ there exists a $\eqso(\logu{1})$ formula $\beta$ over $\R$ such that $\sem{\alpha}(\A) = \sem{\beta}(\A)$ for every $\A\in\ostr[\R]$.
\end{theorem}


The {\em decision problem} associated to a function $f$ is defined by the language $L_f = \{\A \in \str \mid f(\A) > 0\}$.

\begin{theorem} \label{decisionptime}
	The decision problem associated to a function in $\eqso(\logex{1})$ is in \textsc{P}.
\end{theorem}

For a given pair of functions $f,g$, we define $f \dotminus g$ as follows:
\begin{eqnarray*}
	(f \dotminus g)(\A) =
	\begin{cases}
		f(\A)-g(\A), & \text{if }f(\A)>g(\A) \\
		0, & \text{if }f(\A) \leq g(\A).
	\end{cases}
\end{eqnarray*}
for every $\L$-structure $\A \in \str$. A function class $\F$ is {\em closed under substraction} if for every pair of functions $f,g \in \F$, it holds that $f \dotminus g \in \F$.

\begin{theorem} \label{sub-pnp}
	If $\eqso(\loge{1})$ is closed under substraction, then {\sc P} = {\sc NP}.
\end{theorem}

\begin{theorem} \label{sigma1strict}
	$\eqso(\loge{1}) \subsetneq \eqso(\logex{1})$
\end{theorem}

For a given function $f$, we define $f \dotminus 1$ as follows:
\begin{eqnarray*}
	f \dotminus 1(\A) =
	\begin{cases}
		f(\A)-1, & \text{if }f(\A) > 0 \\
		0, & \text{if }f(\A) = 0.
	\end{cases}
\end{eqnarray*}
for every $\L$-structure $\A \in \str$. A function class $\F$ is {\em closed under substraction by one} if for every function $f \in \F$, it holds that $f \dotminus 1 \in \F$.

\begin{theorem} \label{sigmafo-minusone}
	$\eqso(\logex{1})$ is closed under substraction by one.
\end{theorem}

\begin{theorem} \label{dnf-pars}
	{\sc \#DNF} is hard for $\eqso(\loge{1})$ under parsimonious reductions. 
\end{theorem}

\begin{theorem} \label{nplusone-strict}
	$\U{1}$ with $n$ open first-order variables is properly contained in $\U{1}$ with $n+1$ open first-order variables for $n\in\N$.  
\end{theorem}