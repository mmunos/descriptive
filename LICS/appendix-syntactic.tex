\subsection*{Proof of Theorem \ref{theo-pnf-snf}}
Recall that a formula in $\eqso(\LL)$ is on the following grammar:
$$
\alpha = \varphi \ \mid \ s \ \mid \ (\alpha + \alpha) \ \mid \ \sa{x} \alpha \ \mid \ \sa{X} \alpha,
$$
where $\varphi$ is a formula in $\LL$ and $s\in\nat$. We will show a recursive function $\tau$ such that for every $\eqso(\LL)$ formula $\alpha$, $\tau(\alpha)$ is a formula in $\LL$-SNF that describes the same function as $\alpha$. If $\alpha = \varphi$ or $\alpha = s$, then we define $\tau(\alpha) = \alpha$ which clearly describes the same function, and is already in $\LL$-SNF. If $\alpha = (\alpha_1 + \alpha_2)$, then $\tau(\alpha) = (\tau(\alpha_1) + \tau(\alpha_2))$ which satisfies the condition. If $\alpha = \sa{x}\beta$, then let $\tau(\beta) = c + \sum_{i = 1}^{n}\beta_i$ where each $\beta_i$ is in $\LL$-PNF. First we replace $c$ with the formula $(\top \add \cdots \add \top)$ (c times) and we obtain a formula $\sum_{i = 1}^{m}\beta_i$ where each $\beta_i$ is in $\LL$-PNF. Since $\beta_i$ is in $\LL$-PNF, then $\sa{x}\beta_i$ also is, and then we define $\tau(\alpha) = \sum_{i = 1}^{m}\sa{x}\beta_i$ which satisfies the condition. If $\alpha = \sa{X}\beta$, then we proceed analogously as in the previous case. This covers all possible cases for $\alpha$ and we conclude the proof by taking $\tau(\alpha)$ as the desired rewrite of $\alpha$.

\subsection*{Proof of theorem \ref{theo-pi1-pnf}}
We divide the proof in three parts.

\vspace{1em}
First, we prove that the formula $\alpha_{0} = \sa{X(\cdot)}1 + 1$ is not equivalent to any formula in $\loge{0}$-PNF. Suppose that there exists some formula $\alpha = \sa{\bar{X}}\sa{\bar{x}}\varphi(\bar{X},\bar{x})$ in $\loge{0}$-PNF that is equivalent to the $\eqso(\loge{0})$ formula $\sa{X(\cdot)}1 + 1$. Note that this defines the function $2^{n}+1$, where $n$ is the size of the input structure. Then, note that if $\varphi(\bar{X},\bar{x})$ is not satisfiable, then the function is null. Note also that if $\length(\bar{X}) = \length(\bar{x}) = 0$, then the function defined by $\alpha$ is never greater than 1. Now note that $\sa{\bar{X}}\sa{\bar{x}}\varphi(\bar{X},\bar{x})$ is equivalent to a function in $\E{0}$. The authors in \cite{SalujaST95} proved that if $\length(\bar{X}) > 0$, then the function defined by the formula, for big enough structures, is always even. And since $\sa{\bar{x}}\varphi(\bar{x})$ defines a polynomially bounded function, we follow to a contradiction.

\vspace{1em}
Second, we prove that the formula $\alpha_{1} = 2$ is not equivalent to any formula in $\loge{1}$-PNF. Suppose that there exists some formula $\alpha = \sa{\bar{X}}\sa{\bar{x}}\exists\bar{y}\, \varphi(\bar{X},\bar{x},\bar{y})$ in $\loge{1}$-PNF that is equivalent to the $\eqso(\loge{1})$ formula $2$. Suppose that $\varphi$ is quantifier-free. Note first that if $\length(\bar{X}) = \length(\bar{x}) = 0$, then the function defined by $\alpha$ is never greater than 1. Consider an ordered structure $\A$. Since $\sem{\alpha}(\A) = 2$, there exist at least two assignations $(\bar{B}_1,\bar{b}_1,\bar{a}_1)$, $(\bar{B}_2,\bar{b}_2,\bar{a}_2)$ to $(\bar{X},\bar{x},\bar{y})$ such that for both, $\A\models\varphi(\bar{B}_i,\bar{b}_i,\bar{a}_i)$. Now consider the ordered structure $\A'$ that is obtained by duplicating $\A$. This is, each half of $\A'$ is isomorphic to $\A$. Note that $\A'\models\varphi(\bar{B}_i,\bar{b}_i,\bar{a}_i)$ for $i = 1,2$ and there exists a third assignation $(\bar{B}_1',\bar{b}_1',\bar{a}_1')$ that is identical to $(\bar{B}_1,\bar{b}_1,\bar{a}_1)$ but in the other half to the structure such that $\A'\models\varphi(\bar{B}_1',\bar{b}_1',\bar{a}_1')$. We have that $\sem{\alpha}(\A) \geq 3$, and we follow to a contradiction.


\vspace{3em}


Classes QE0, QE1, QU1, QE2, QU2 = QFO form a properly contained hierarchy:\vspace{1em}

We give this proof in various parts.

We show here that $\QE{0} \not\subseteq \E{1}$.. By contradiction, suppose that there is a $\QE{0}$ formula $\alpha$ over some signature $\R$ such that defines the following function. For every finite $\R$-structure with $n$ elements, and where every predicate in $\R$ is empty, $\alpha(\enc(\A)) = n - 1$. We use the following claim.
\begin{claim}
Let $\alpha = \sa{\bar{x}}\varphi(\bar{x})$	where $\varphi$ is quantifier free. Then the function defined by $\alpha$ is either null, greater or equal to $n$, or is in $\Omega(n^2)$.
\end{claim}
\begin{proof}
Suppose that the function defined by $\alpha$ is not $0$ and that $\varphi$ is in DNF. Furthermore, suppose $\bar{x} = (x_1,\ldots,x_{\length(\bar{x})})$. Then $\alpha = \sa{\bar{x}} \varphi_1(\bar{x}) \vee \cdots \vee \varphi_n(\bar{x})$. Since $\alpha$ is not null, then some $\varphi_i$ must be satisfiable. This is, the function defined by $\sa{\bar{x}}\varphi(\bar{x})$ is not null. We will prove by induction on $\length(\bar{x})$ that the function defined by $\sa{\bar{x}}\varphi(\bar{x})$ is either greater or equal to $n$, or in $\Omega(n^2)$. We address the case $\length(\bar{x})= 1$, then $\alpha = \sa{x}\bigwedge\psi(x)$. If any $\psi(x) = (x = x)$ or $\neg(x < x)$, then we can eliminate it and we obtain the same function. If any $\psi = (x < x)$ or $\neg(x=x)$, then the function becomes null. If $\psi(x) = R(x,\ldots,x)$ for some $R\in\R$ the function becomes null for the structures we are considering. If $\psi(x) = \neg R(x,\ldots,x)$, we can eliminate it and for the structures we are considering we obtain the same function. The only possible $\alpha$ left is $\alpha = \sa{x}\top$ which is equal to the function $n$. This covers all possible cases for $\length(\bar{x}) = 1$. Now suppose that it holds for $\length(\bar{x}) = k$ and suppose $\alpha = \sa{\bar{x}}\bigwedge\psi(\bar{x})$ for $\length(\bar{x}) = k+1$. If any $\psi(\bar{x}) = (x_i = x_j)$ where $i \neq j$, then $\alpha$ describes the same function as $\alpha$ where $x_j$ has been replaced by $x_i$. In this formula the tuple of first-order variables has $k$ elements so the function it describes if one of the mentioned in the hypothesis. If $i = j$, then we can eliminate it and obtain the same function. If any $\psi(\bar{x}) = R(\bar{v})$ or $\neg R(\bar{v})$ where $\bar{v}$ is a sub-tuple of $\bar{x}$ then we can either eliminate it or the function becomes null, following the same argument as in the case $\length(\bar{x}) = 1$. If any $\psi(\bar{x}) = \neg(x_i = x_j)$ or $(x_i < x_j)$ where $i = j$, then the function becomes null. If any $\psi(\bar{x}) = \neg(x_i < x_j)$ where $i = j$, we can eliminate it. The remaining formulas in $\bigwedge\psi(\bar{x})$ are either $\neg(x_i = x_j)$, $(x_i<x_j)$ or $\neg(x_i<x_j)$. If the formula violates transitivity in $<$ (for example, $x < y \wedge y < z \wedge z < x$), then the function $\alpha$ describes is null. Therefore, there is some order over $\bar{x}$ that satisfies $\bigwedge\psi(\bar{x})$. Consider the formula that describes this order (like $x_1 < x_3 \wedge x_3 < x_4 \wedge x_4 < x_2$). The function $\alpha$ describes is greater or equal to the one this formula describes, which is exactly $\binom{n}{\length(\bar{x})}$ which is in $\Omega(n^{\length(\bar{x})}) \subseteq \Omega(n^2)$ if $\length(\bar{x}) > 1$. This concludes the proof of the claim.
\end{proof}
We suppose that $\alpha$ is in SNF, this is, $\alpha = c + \sum_{i = 1}^n\alpha_i$. Let $\mathbb{1}$ be a $\R$-structure with one element. Note that if $c > 0$, then $\alpha(\enc(\mathbb{1})) \geq c > 0$ which is not possible since $\alpha(\enc(\mathbb{1})) = 0$. Since $\alpha$ is not null, consider some $\alpha_i$ that describes a non-null function. Let $\alpha_i = \sa{\bar{X}}\sa{\bar{x}}\varphi(\bar{X},\bar{x})$, where $\varphi$ is quantifier-free. Note that if $\length(\bar{X}) > 0$, then the function $\alpha$ describes is in $\Omega(2^n)$, as it was proven by the authors in \cite{SalujaST95}. We have that $\alpha_i = \sa{\bar{x}}\varphi(\bar{x})$, as we proved in the claim, describes either some function greater or equal to $n$, or in $\Omega(n^2)$, which leads to a contradiction. Lastly, note that the formula $\sa{x}\exists y(x < y)$ is in $\E{0}$ and describes the function $n-1$, which concludes the proof.

\vspace{1em}
We show here that $\E{1}\not\subseteq\QE{0}$. By contradiction, suppose that the constant 1 is expressible in $\E{1}$. Let $\alpha = \sa{\bar{X}}\sa{\bar{x}}\exists\bar{y}\,\varphi(\bar{X},\bar{x},\bar{y})$ describe this function for some quantifier-free formula $\varphi$.


\subsection*{Proof of Theorem \ref{theo:eqso-sum-normal-form}} %V.2
Every formula in $\eqso(\LL)$ can be rewritten in \emph{sum normal form}. Furthermore, there exists a formula $\alpha \in \QE{1}$ such that there does not exist a formula in $\Sigma$-prenex normal form in $\QE{1}$ equivalent to $\alpha$:\vspace{1em}

\subsection*{Proof of Theorem \ref{theo:prenex}}%V.3
If $\logu{1} \subseteq \LL$, then for every formula $\alpha \in \eqso(\LL)$ there exists a formula $\beta \in \eqso(\LL)$ equivalent to $\alpha$ in $\Sigma$-prenex normal form:\vspace{1em}

\subsection*{Proof of Proposition \ref{prop:sharp-Q-rel}}%V.4
The classes $\E{0}$ and $\E{1}$ are strictly contained in $\QE{0}$ and $\QE{1}$, respectively. Moreover, the classes $\U{1}$, $\E{2}$, and $\U{2}$ are equivalent with $\QU{1}$, $\QE{2}$, and $\QU{2}$, respectively:\vspace{1em}

\subsection*{Proof of Proposition \ref{prop:qe0-fp-qe1-totp-fptras}} %V.5
All functions defined in $\eqso(\loge{0})$ and $\eqso(\loge{1})$ can be computed in $\fp$ and $\totp$, respectively. Furthermore, every function defined in $\eqso(\loge{1})$ has a FPTRAS:\vspace{1em}

\subsection*{Proof of Theorem \ref{theo:binary-prod}}%V.6
If $\LL$ is closed under conjunction, then binary product can be defined in $\eqso(\LL)$:\vspace{1em}

\subsection*{Proof of Theorem \ref{sub-pnp}}%V.7
If $\eqso(\loge{i})$ or $\eqso(\logu{i})$ is closed under subtraction for $i > 0$, then {\sc P} = {\sc NP}:\vspace{1em}

Suppose that $\E{1}$ is closed under substraction, that is, for each pair of functions $f,g\in \E{1}$, there exists $h\in\E{1}$ such that $(f\dotminus g)(\A) = h(\A)$ for each $\A\in\str$.

Let $\A = \langle A, S_1^\A, S_2^\A, S_3^\A, S_4^\A, \leq^\A \rangle$ be an $\L-$structure that represents an instance of a 3DNF formula $\Phi$, where $A$ is the set of variables mentioned in $\Phi$, $S_i^\A$ is a ternary relation described as follows, for each $i\in\{1,2,3,4\}$:
\begin{eqnarray*}
	S_1^\A &=& \{(a_1,a_2,a_3)\mid (\neg a_1 \wedge \neg a_2 \wedge \neg a_3) \mbox{ appears as a disjunct in }\Phi\},\\
	S_2^\A &=& \{(a_1,a_2,a_3)\mid ( a_1 \wedge \neg a_2 \wedge \neg a_3) \mbox{ appears as a disjunct in }\Phi\},\\
	S_3^\A &=& \{(a_1,a_2,a_3)\mid ( a_1 \wedge  a_2 \wedge \neg a_3) \mbox{ appears as a disjunct in }\Phi\},\\
	S_4^\A &=& \{(a_1,a_2,a_3)\mid ( a_1 \wedge  a_2 \wedge  a_3) \mbox{ appears as a disjunct in }\Phi\}.
\end{eqnarray*}
Now we define $f_{\#3DNF} = f_{\psi(T)}$ where
\begin{multline*}
\psi(T) = \exists x \exists y \exists z\, [(S_1(x,y,z) \wedge \neg T(x) \wedge \neg T(y) \wedge \neg T(z)) \vee (S_2(x,y,z) \wedge T(x) \wedge \neg T(y) \wedge \neg T(z)) \, \vee \\ (S_3(x,y,z) \wedge T(x) \wedge T(y) \wedge \neg T(z)) \vee (S_4(x,y,z) \wedge T(x) \wedge T(y) \wedge T(z))].
\end{multline*}
Note that $f_{\#3DNF} \in \#\Sigma_1$. Let $f_{all} = f_{\exists x\:\varphi(x,X)}$, where
$$
\varphi(x,X) = (T(x) \vee \neg T(x)).
$$
Note that $f_{all}$ counts every possible truth assignment (satisfying or not) to a 3DNF formula. Given that $f_{\#3DNF}, f_{all} \in \E{1}$, we have by our initial assumption that $f_{all}-f_{\#3DNF} \in \E{1}$. Let $h\in\E{1}$ be such that $h = f_{all}-f_{\#3DNF}$. For each structure $\A$ that represents a 3DNF formula $\psi$, it holds that $h(\A) = f_{all}(\A)-f_{\#3DNF}(\A) = 0$ if and only if $\psi$ is a tautology, so the decision version $L_h$ of $f_{all}-f_{\#3DNF}$ is $\conp$-complete. However, as we showed previously in Theorem \ref{decisionptime}, since $h\in\E{1}$, we have that $L_h \in \ptime$. Then, $\conp \subseteq \ptime$, from which we conclude that $\ptime = \np$.


\subsection*{Proof of Theorem \ref{sigmafo-minusone}}%V.8

Let $\alpha$ be a $\eqso(\logex{1})$ formula over a signature $\R$. We will define a $\eqso(\logex{1})$ formula $\kappa(\alpha)$ such that for each finite structure $A$ over $\R$: $\sem{\kappa(\alpha)}(\A) = \sem{\alpha}(\A) \dotminus 1$. Let $\tau(\alpha) = \beta_1\add\cdots\add\beta_n$ be defined as in \ref{eqso-sigma-zero-in-fp}, that is, each $\beta_i$ is equal to $1$, $\varphi$, $\sa{\bar{x}}1$, $\sa{\bar{x}}\varphi$, $\sa{\bar{X}}1$, $\sa{\bar{X}}\varphi$, $\sa{\bar{X}}\sa{\bar{x}}1$ or $\sa{\bar{X}}\sa{\bar{x}}\varphi$, where $\varphi$ is a formula in $\logex{1}$ and $\bar{x}$ ($\bar{X}$) is a tuple of first-order (second-order) variables. We further assume that each $\beta_i$ is either $\sa{\bar{x}}\varphi$ or $\sa{\bar{X}}\sa{\bar{x}}\varphi$, since each $1$, $\varphi$ and $\sa{\bar{X}}\varphi$ can be replaced by the formulas $\sa{x}\min(x)$, $\sa{x}(\varphi\wedge\min(x))$ and $\sa{x}\sa{\bar{X}}(\varphi\wedge\min(x))$ respectively, while maintaining the value of $\sem{\alpha}(\A)$. 

The intuition behind our reasoning is separated in three points: (1) For each $\beta_i$ of the form $\sa{\bar{x}}\varphi$, the formula $\kappa(\beta_i)$ will count every tuple $\bar{x}$ that satisfies $\varphi$ except for the lexicographically smallest one. (2) For each $\beta_i$ of the form $\sa{\bar{X}}\sa{\bar{x}}\varphi$, the formula $\kappa(\beta_i)$ will isolate the smallest $\bar{X}$ that satisfies $\varphi$, and exclude the lexicographically smallest tuple $\bar{x}$ that satisfies $\varphi(\bar{X})$. And (3) if $\alpha = (\beta\add\sa{\bar{x}}\varphi)$ or $\alpha = (\beta\add\sa{\bar{X}}\sa{\bar{x}}\varphi)$, the formula $\kappa(\alpha)$ will exclude the lexicographically smallest tuple that satisfies $\varphi$ if and only if $\beta$ is equal to 0.

\vspace{1em}

For each $\alpha$ in $\eqso(\logex{1})$ such that $\alpha = \sa{\bar{x}}\exists\bar{y}\,\varphi(\bar{x},\bar{y})$ for some quantifier-free formula $\varphi$, we define $\kappa(\alpha) = \sa{\bar{x}}\exists\bar{y}[\varphi(\bar{x},\bar{y})\wedge\exists\bar{z}(\varphi(\bar{z},\bar{y})\wedge\bar{z} < \bar{x})]$, which is in $\eqso(\logex{1})$ and fulfils the desired condition.

\vspace{1em}

For each $\alpha$ in $\eqso(\logex{1})$ such that $\alpha = \sa{\bar{X}}\sa{\bar{x}}\exists\bar{y}\,\varphi(\bar{X},\bar{x},\bar{y})$ for some quantifier-free formula $\varphi$, we define $\kappa(\alpha)$ procedurally as follows: Let $\bar{x} = (x_1,\ldots,x_{\length(\bar{x})})$ and $\bar{X} = (X_1,\ldots,X_{\length(\bar{X})})$. We suppose that $\varphi$ is in a DNF form that leaves formulas that do not mention $\bar{X}$ as is. If it is not, we convert $\varphi$ to this form with a standard transformation algorithm. Let $\varphi(\bar{X},\bar{x},\bar{y}) = \bigvee_{i = 1}^m\varphi_i(\bar{X},\bar{x},\bar{y})$ where each $\varphi_i$ has the form:
$$
\varphi_i(\bar{X},\bar{x},\bar{y}) = (\text{conjunction of $X_j$'s}) \wedge (\text{conjunction of $\neg X_j$'s})  \wedge (\text{an $\fo$ formula that does not mention any $X_j$}).
$$
Define $\varphi_i^{+}$, $\varphi_i^{-}$ and $\varphi_i^{\fo}$ as the formulas mentioned above. 

In our procedure we assume that in $\varphi_i^{+}\wedge\varphi_i^{-}$ every first-order variable from $\bar{x},\bar{y}$ is mentioned at most once. If not, we add new first-order variables from $\fv$ to $\bar{y}$ so that no variable is repeated, replace them in $\varphi_i^{+}\wedge\varphi_i^{-}$ and add their respective equalities in $\varphi_i^{\fo}$ so that the new formula is equivalent. For example, if $\bar{x} = x$, $\bar{y} = y$ and $\varphi_i = X(x,y)\wedge \neg X(x,x) \wedge x < y$, then we redefine $\bar{y} = (y,v_1,v_2,v_3,v_4)$ and $\varphi_i := X(v_1,v_2) \wedge \neg X(v_3,v_4) \wedge x < y \wedge v_1 = x \wedge v_2 = y \wedge v_3 = x \wedge v_4 = x.$ 

Furthermore, we assume that $\varphi_i^{\fo}$ defines an ordered partition of the variables in the tuple $(\bar{x},\bar{y})$. For example, let $\bar{x} = (x_1,x_2,x_3,x_4)$. A possible ordered partition would be defined by the formula $\theta(\bar{x}) = x_2 < x_1 \wedge x_1 = x_4 \wedge x_4 < x_3$. On the other hand, the formula $\theta'(\bar{x}) = x_1 < x_2 \wedge x_1 < x_4 \wedge x_2 = x_3$ does not define an ordered partition since both $\{x_1\}<\{x_2,x_3\}<\{x_4\}$ and $\{x_1\} < \{x_2,x_3,x_4\}$ satisfy $\theta'$.
For a given $k$, we define $\cB_k$ as the number of possible ordered partitions for a set of size $k$. 
If $\varphi_i^{\fo}$ does not define an ordered partition, then for $1 \leq j \leq \cB_{\length(\bar{x},\bar{y})}$ 
let $\theta^j(\bar{x},\bar{y})$ be a formula that defines an ordered partition over $(\bar{x},\bar{y})$. The formula $\varphi_i(\bar{X},\bar{x},\bar{y})$ is converted into $\bigvee_{j = 1}^{\cB_{\length(\bar{x},\bar{y})}}\varphi_i(\bar{X},\bar{x},\bar{y}) \wedge \theta^j(\bar{x},\bar{y})$. Consider each of these disjuncts as a new $\varphi_i$.

After the previous assumptions, we do the following: for each $X_j\in \bar{X}$ we check every instance of $X_j(\bar{w})$ in $\varphi^{+}_i$ and every instance of $X_j(\bar{z})$ in $\varphi^{-}_i$, where $\bar{w}$ and $\bar{z}$ are subtuples of $(\bar{x},\bar{y})$. If the ordered partition in $\varphi^{\fo}_i$ satisfies $\bar{w} = \bar{z}$, the entire formula $\varphi_i$ is removed from $\varphi$.

Now let $\bar{v}$ be the tuple of every first-order variable mentioned in $\varphi_i^{+}$ and let $\bar{u}$ be such that $(\bar{x},\bar{y}) = (\bar{u},\bar{v})$. We define a formula $\mu_i$ that is satisfied only by the lexicographically smallest $\bar{v}$ that satisfies $\varphi_i^{\fo}$ for some $\bar{u}$:
$$
\mu_i(\bar{v}) = \exists\bar{u}\,\varphi_i^{\fo}(\bar{u},\bar{v})\wedge\forall\bar{u}'\forall\bar{v}'(\varphi_i^{\fo}(\bar{u}',\bar{v}')\to\bar{v}\leq\bar{v}').
$$
The following is a pivotal formula for this proof. Consider some ordered finite structure $\A$ over $\R$. Given our conditions for $\varphi_i$, if $\A\models\varphi_i$ and $\A\models\varphi^{\fo}_i(\bar{u},\bar{v})$ then for that $\bar{v}$ we can define a $\bar{X}$ that has exactly the tuples mentioned in $\bar{v}$ and nothing else. This is {\em the smallest $\bar{X}$ that satisfies $\varphi_i$ over $\bar{v}$}. And if $\bar{v}$ is the lexicographically smallest such that $\A\models\varphi_i(\bar{v})$, then it is the {\em the smallest $\bar{X}$ that satisfies $\varphi_i$}. The following formula is satisfied by every pair $(\bar{X},\bar{x})$, except for the one formed by the smallest $\bar{X}$ that satisfies $\varphi_i$ and the lexicographically smallest $\bar{x}$ that satisfies $\varphi_i(\bar{X})$. And if no pair satisfies $\varphi_i$, the following formula is satisfied by every pair.
$$
\psi_i(\bar{X},\bar{x}) = \exists\bar{v}(\mu_i(\bar{v})\wedge(\neg\varphi^{+}_i(\bar{X},\bar{v})\vee\bigvee_{X \in \bar{X}} \exists\bar{z}(X(\bar{z}) \wedge \bigwedge\limits_{\substack{\text{instances of }X(\bar{w}) \\ \text{in }\varphi^{+}_i(\bar{X},\bar{v})}}\bar{w}\neq\bar{z}))) \vee \exists\bar{x}'(\exists\bar{y}'\varphi_i(\bar{X},\bar{x}',\bar{y}')\wedge \bar{x}'<\bar{x})\vee\neg\exists\bar{v}\mu_i(\bar{v}) .
$$
\begin{claim}
	For a given ordered structure $\A$ such that $\A\models\exists\bar{X}\exists\bar{x}\exists\bar{y}\,\varphi_i(\bar{X},\bar{x},\bar{y})$, there is an assignment $(\bar{B},\bar{b})$ to $(\bar{X},\bar{x})$ that satisfies the following conditions (1) $\A\models\exists\bar{y}\,\varphi_i(\bar{B},\bar{b},\bar{y})$, (2) $\A\not\models\psi_i(\bar{B},\bar{b})$ and (3) this is the only assignment to $(\bar{X},\bar{x})$ that satisfies (1) and (2).
\end{claim}
\begin{proof}
	proof here.
\end{proof}
The following claim identifies a crucial condition that is verifiable with a $\fo$ formula. 
\begin{claim}
	Let $\varphi_i(\bar{X},\bar{x},\bar{y}) = \varphi^{\fo}_i(\bar{x},\bar{y}) \wedge \varphi^{-}_i(\bar{X},\bar{x},\bar{y}) \wedge \varphi^{+}_i(\bar{X},\bar{x},\bar{y})$ be a $\logex{0}$ formula that satisfies all the mentioned assumptions. For a given ordered structure $\A$ it holds that $\A\models\exists\bar{x}\exists\bar{y}\,\varphi^{\fo}_i(\bar{x},\bar{y})$ iff $\A\models\exists\bar{X}\exists\bar{x}\exists\bar{y}\,\varphi_i(\bar{X},\bar{x},\bar{y})$.
\end{claim}
\begin{proof}
	Let $\A$ be an ordered structure with domain $A$ and let $\bar{a}\in A^{\length(\bar{y})}$ and $\bar{b}\in A^{\length(\bar{x})}$ such that $\A\models\varphi^{\fo}_i(\bar{b},\bar{a})$. Define $\bar{B} = (B_1,\ldots,B_{\length(\bar{X})})$ as $B_j = \{\bar{c}\mid\bar{c}\text{ is a subtuple of $(\bar{b},\bar{a})$ and $X_j(\bar{c})$ is mentioned in $\varphi^{+}_i(\bar{X},\bar{b},\bar{a})$}\}$. Towards a contradiction, suppose that $\A\not\models\varphi_i(\bar{B},\bar{b},\bar{a})$. By the choice of $\bar{a}$ and $\bar{b}$, and construction of $\bar{B}$ it is clear that $\A\models\varphi^{\fo}_i(\bar{b},\bar{a})\wedge\varphi^{+}_i(\bar{B},\bar{b},\bar{a})$, so we have that $\A\not\models\varphi^{-}_i(\bar{B},\bar{b},\bar{a})$. Let $B_t(\bar{c})$ be such that $X_t(\bar{c})$ is mentioned in $\varphi^{-}_i(\bar{X},\bar{b},\bar{a})$ and $\A\not\models\neg B_t(\bar{c})$. This is, $\bar{c}\in B_t$. By the construction of $\bar{B}$ we have that $X_t(\bar{c})$ is mentioned in $\varphi^{+}_i(\bar{X},\bar{b},\bar{a})$. From our assumptions, we have that (1) every first-order variable is mentioned at most once in $\varphi^{-}_i(\bar{X},\bar{x},\bar{y}) \wedge \varphi^{+}_i(\bar{X},\bar{x},\bar{y})$, (2) $\varphi^{\fo}_i$ defines an ordered partition over $(\bar{x},\bar{y})$, and (3) there are no $X_j(\bar{w})$ in $\varphi^{+}_i$ and $X_j(\bar{z})$ in $\varphi^{-}_i$ such that the ordered partition defined by $\varphi^{\fo}_i$ satisfies $\bar{x} = \bar{z}$. Since $\bar{c}$ is a subtuple of $(\bar{b},\bar{a})$ that satisfies $\varphi^{\fo}_i$, and $X_t(\bar{w})$ mentioned in $\varphi^{+}_i$ and $X_t(\bar{z})$ mentioned in $\varphi^{-}_i$ were both assigned the value $\bar{c}$, then the ordered partition satisfies $\bar{w} = \bar{z}$, which follows to a contradiction.
\end{proof}
We define $\chi_i = \exists\bar{x}\exists\bar{y}\,\varphi^{\fo}_i(\bar{x},\bar{y})$. Now recall that $\varphi = \bigvee_{i = 1}^m\varphi_i(\bar{X},\bar{x},\bar{y})$, where each $\varphi_i$ satisfies all the previous assumptions. For each $\varphi_i$ we define:
$$
\varphi_i'(\bar{X},\bar{x},\bar{y}) = \varphi_i(\bar{X},\bar{x},\bar{y})\wedge\psi_1(\bar{X},\bar{x})\wedge(\chi_1\vee\psi_2(\bar{X},\bar{x}))\wedge(\chi_1\vee\chi_2\vee\psi_2(\bar{X},\bar{x}))\wedge\cdots\wedge(
\bigvee_{j = 1}^{j = i-1}\chi_j\vee\psi_i(\bar{X},\bar{x})),
$$
and lastly, $\kappa(\alpha)$ is defined as $\kappa(\alpha) = \sa{\bar{X}}\sa{\bar{x}}\exists\bar{y}\bigvee_{i = 1}^m\varphi_i'(\bar{X},\bar{x},\bar{y})$.

\vspace{1em}

For each $\alpha$ in $\eqso(\logex{1})$ such that $\alpha = (\beta + \sa{\bar{X}}\sa{\bar{x}}\exists\bar{y}\,\varphi(\bar{X},\bar{x},\bar{y}))$ for some algebraic formula $\beta$ and some quantifier-free formula $\varphi$, we define $\kappa(\alpha)$ as follows: First, perform the same transformations to $\varphi(\bar{X},\bar{x},\bar{y})$ as in the previous case. Let $\varphi = \bigvee_{i = 1}^m\varphi_i(\bar{X},\bar{x},\bar{y})$ where each $\varphi_i$ satisfies the previous assumptions. We also use the previously defined formulas $\chi_i$ and $\psi_i$. 

We recursively define a function $\lambda$ as follows. If $\alpha = \sa{\bar{x}}\exists\bar{y}\varphi(\bar{x},\bar{y})$, then $\lambda(\alpha) = \exists\bar{x}'\exists\bar{y}'\varphi(\bar{x}',\bar{y}')$. If $\alpha = \sa{\bar{X}}\sa{\bar{x}}\exists\bar{y}\,\varphi(\bar{X},\bar{x},\bar{y})$, then let $\varphi = \bigvee_{i = 1}^{m}\varphi_i$ and $\lambda(\alpha) = \chi_1\vee\cdots\vee\chi_m$ as each $\chi_i$ was previously defined. If $\alpha = (\beta + \gamma)$, then $\lambda(\alpha) = \lambda(\beta) \vee \lambda(\gamma)$. Note that for a given ordered structure $\A$, then $\A\models\lambda(\alpha)$ if and only if $\sem{\alpha}(\A) > 0$.

Now, for each $\varphi_i$ we define:
$$
\varphi_i'(\bar{X},\bar{x},\bar{y}) = \varphi_i(\bar{X},\bar{x},\bar{y})\wedge[\lambda(\beta)\vee[\psi_1(\bar{X},\bar{x})\wedge(\chi_1\vee\psi_2(\bar{X},\bar{x}))\wedge(\chi_1\vee\chi_2\vee\psi_2(\bar{X},\bar{x}))\wedge\cdots\wedge(
\bigvee_{j = 1}^{j = i-1}\chi_j\vee\psi_i(\bar{X},\bar{x}))]].
$$
And lastly $\kappa(\alpha)$ is defined as $\kappa(\alpha) = \kappa(\beta) + \sa{\bar{X}}\sa{\bar{x}}\exists\bar{y}\bigvee_{i = 1}^m\varphi_i'(\bar{X},\bar{x},\bar{y})$, which is in $\eqso(\logex{1})$ and satisfies the desired condition.

\subsection*{Proof of Proposition \ref{prop:uhorn-pe}} %V.9

\subsection*{Proof of Proposition \ref{prop:ehorn-pe}} %V.11

\subsection*{Proof of Proposition \ref{prop:hsat-not-sigma2}} %V.12
$\eqso(\uhorn) \subseteq \totp$:\vspace{1em}

\subsection*{Proof of Theorem \ref{sigma2hard}} %V.13

First we prove that $\shdhsat$ is in $\eqso(\ehorn)$. Recall that each instance of $\shdhsat$ is a disjunction of Horn formulas. Let $\R = \{\pP(\cdot,\cdot), \pN(\cdot,\cdot), \pV(\cdot), \pNC(\cdot), \pD(\cdot,\cdot)\}$. Each symbol in this vocabulary is used to indicate the same as in Example \ref{ex-hornsat-esop1}, with the addition of $\pD(d,c)$ which indicates that $c$ is a clause in the formula $d$. Recall that the formula
\begin{align*}
&\forall x \, (\neg \pT(x) \vee \pV(x)) \ \wedge\\
&\forall c \, (\neg \textit{NC}(c) \vee \exists x \, \neg \textit{A}(c,x)) \ \wedge\\
&\forall c \forall x \, (\neg \textit{P}(c,x) \vee \exists y \, \neg \textit{A}(c,y) \vee \textit{T}(x)) \ \wedge\\
&\forall c \forall x \, (\neg \textit{N}(c,x) \vee \textit{T}(x) \vee \neg \textit{A}(c,x)) \ \wedge\\
&\forall c \forall x \, (\textit{A}(c,x) \vee \textit{N}(c,x)) \ \wedge\\
&\forall c \forall x \, (\textit{A}(c,x) \vee \neg\textit{T}(x)).
\end{align*}
defines $\chsat$. We obtain the following formula $\psi(T,A)$ in $\ehorn$:
\begin{align*}
\exists d[&\forall x \, (\neg \pT(x) \vee \pV(x)) \ \wedge\\
&\forall c \, (\neg \pD(c,d)\vee \neg \textit{NC}(c) \vee \exists x \, \neg \textit{A}(c,x)) \ \wedge\\
&\forall c \forall x \, (\neg \pD(c,d)\vee\neg \textit{P}(c,x) \vee \exists y \, \neg \textit{A}(c,y) \vee \textit{T}(x)) \ \wedge\\
&\forall c \forall x \, (\neg \pD(c,d)\vee\neg \textit{N}(c,x) \vee \textit{T}(x) \vee \neg \textit{A}(c,x)) \ \wedge\\
&\forall c \forall x \, (\neg \pD(c,d)\vee\textit{A}(c,x) \vee \textit{N}(c,x)) \ \wedge\\
&\forall c \forall x \, (\neg \pD(c,d)\vee\textit{A}(c,x) \vee \neg\textit{T}(x))]
\end{align*}
effectively defines $\chsat$ as for every disjunction of Horn formulas $\theta = \theta_1\vee\cdots\vee\theta_m$ encoded by an $\R$-structure $\A$, the number of satisfying assignments of $\theta$ is equal to $\sem{\sa{\pT} \sa{\pA} \psi(\pT,\pA)}(\A)$.  Therefore, we conclude that $\shdhsat \in \eqso(\ehorn)$.

\vspace{1em}
We will now prove that $\shdhsat$ is hard for $\eqso$ over a signature $\R$ under parsimonious reductions. For each $\eqso(\ehorn)$ formula $\alpha$ over $\R$, we will define a polynomial-time procedure that computes a function $g_{\alpha}$. This function receives a finite $\R$-structure $\A$ and outputs an instance of $\shdhsat$ such that $\sem{\alpha}(\A) = \shdhsat(g_{\alpha}(\A))$. We suppose that $\alpha$ is in sum normal form and:
$$
\alpha = c + \sum_{i = 1}^{\text{\#clauses}} \sa{\bar{X}}\sa{\bar{x}}\exists\bar{y}\bigwedge_{j = 1}^{n}\forall\bar{z}\,\varphi^i_j(\bar{X},\bar{x},\bar{y},\bar{z}),
$$
where each $\varphi^i_j$ is a Horn clause.                                                                

Consider a finite $\R$-structure $\A$ with domain $A$. To simplify the proof, we extend our grammar to allow first-order constants. Consider each tuple $\bar{a}\in A^{\length(\bar{x})}$, each $\bar{b}\in A^{\length(\bar{y})}$ and each $\bar{c}\in A^{\length(\bar{z})}$ as a tuple of first-order constants. The following formula defines the same function as $\alpha$:
$$
c + \sum_{i = 1}^{\#clauses} \sum_{\bar{a}\in A^{\length(\bar{x})}} \sa{\bar{X}}\bigvee_{\bar{b}\in A^{\length(\bar{y})}}\bigwedge_{j = 1}^{n}\bigwedge_{\bar{c}\in A^{\length(\bar{z})}}\varphi^i_j(\bar{X},\bar{a},\bar{b},\bar{c}).
$$
Note that each $\fo$ formula over $(\bar{x},\bar{y},\bar{z})$ in each $\varphi^i_j$ has no free variables. Therefore, we can evaluate each of these in polynomial time and replace them by $\perp$ and $\top$ where it corresponds. Each $\varphi^i_j$ will be of the form $\perp \vee\, \chi^i_j(\bar{X})$ or $\top \vee \chi^i_j(\bar{X})$ where $\chi^i_j$ is a disjunction of $\neg X_{\ell}$'s and at most one $X_{\ell}$. The formulas of the form $\top \vee \chi^i_j(\bar{X})$ can be removed entirely, and the formulas of the form $\perp \vee\, \chi^i_j(\bar{X})$ can be replaced by $\chi^i_j(\bar{X})$. We obtain the formula
$$
c + \sum_{i = 1}^{m}\sa{\bar{X}}\bigvee_{j = 1}^{\#d}\bigwedge_{k = 1}^{\#c}\psi^{i}_{j,k}(\bar{X})
$$
where every $\psi^{i}_{j,k}(\bar{X})$ is a disjunction of $\neg X_{\ell}$'s and zero or one $X_{\ell}$.

Our idea for the rest of the proof is to define $g_{\alpha}$ for each $\alpha = \sa{\bar{X}}\bigvee_{j = 1}^{\#d}\bigwedge_{k = 1}^{\#c}\psi^{i}_{j,k}(\bar{X})$, for $\alpha = c$ and for $\alpha = \beta_1 + \cdots + \beta_m$ where each $\beta_i$ is in one of the two previous cases.

If $\alpha$ is equal to $\sa{\bar{X}}\bigvee_{j = 1}^{\#d}\bigwedge_{k = 1}^{\#c}\psi_{j,k}(\bar{X})$ where $\psi_{j,k}(\bar{X})$ is a disjunction of $\neg X_{\ell}$'s and zero or one $X_{\ell}$, then we obtain the {\bf propositional formula} $g_{\alpha}(\A) = \bigvee_{j = 1}^{\#d}\bigwedge_{k = 1}^{\#c}\psi_{j,k}(\bar{X})$ over the propositional alphabet $\{X(\bar{e}) \mid X \in \bar{X} \text{ and } \bar{e}\in A^{\arity(X)} \}$ which has exactly $\sem{\alpha}(\A)$ satisfying assignments and is precisely a disjunction of Horn formulas.

If $\alpha$ is equal to a constant $c$, then we define $g_{\alpha}(\A)$ as the following formula that has exactly $c$ satisfying assignments:
$$
g_{\alpha}(\A) = \bigvee_{i = 1}^{c}\neg t_1 \wedge \cdots \wedge \neg t_{i-1} \wedge t_i \wedge \neg t_{i+1} \wedge \cdots \wedge \neg p_c.
$$ 
If $\alpha = \beta_1 + \cdots + \beta_m$, let $g_{\beta_i}(\A) = \bigvee_{j = 1}^{\#d}\bigwedge_{k = 1}^{\#c}\theta^i_{j,k}$ for each $\beta_i$ where each $\theta^i_{j,k}$ is a Horn clause. Let $\Theta_i = g_{\beta_i}(\A)$. We rename the variables in each $\Theta_i$ so none of them are mentioned in any other $\Theta_j$. We add $m$ new variables $t_1,\ldots,t_m$ and we define:
\begin{align*}
g_{\alpha}(\A) = &\bigvee_{i = 1}^{\#d}(\bigwedge_{j = 1}^{\#c}\theta^1_{i,j} \wedge (\bigwedge\limits_{\substack{\text{each } t\\ \text{ mentioned in}\\ \Theta_2,\ldots,\Theta_{m}}}t) \wedge (t_1 \wedge \bigwedge_{\ell = 2}^{m} \neg t_{\ell})) \vee \\ 
&\bigvee_{i = 1}^{\#d}(\bigwedge_{j = 1}^{\#c}\theta^2_{i,j} \wedge (\bigwedge\limits_{\substack{\text{each $t$}\\ \text{ mentioned in}\\ \Theta_1,\Theta_3,\ldots,\Theta_{m}}}t) \wedge (t_2 \wedge \bigwedge\limits_{\substack{\ell = 1 \\ \ell \neq 2}}^{m} \neg t_{\ell})) \vee \cdots \vee\\ 
&\bigvee_{i = 1}^{\#d}(\bigwedge_{j = 1}^{\#c}\theta^m_{i,j} \wedge (\bigwedge\limits_{\substack{\text{each } t\\ \text{ mentioned in}\\ \Theta_2,\ldots,\Theta_{m-1}}}t) \wedge (t_m \wedge \bigwedge_{\ell = 1}^{m-1} \neg t_{\ell})).
\end{align*}
The formula is a disjunction of Horn formulas, and the number of satisfying assignments for this formula is exactly the sum of satisfying assignments for each $g_{\beta_i}(\A)$. This, at the same time, is equal to $\sem{\alpha}(\A)$. This covers all possible cases for $\alpha$, and the entire procedure takes polynomial time.