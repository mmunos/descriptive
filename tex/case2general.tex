We now continue with the general case, in which $\varphi(\x,\X)$ is an arbirtary extended quantifier-free $\L$-formula. By using a standard DNF transformation algorithm and considering FO-formulas over $\L$ as literals, we can find formulas $\gamma_i(\x,\X)$ with $i\in\{1,\ldots,\ell\}$ such that
\begin{eqnarray*}
\varphi(\x,\X) &\equiv& \gamma_1(\x,\X) \vee \gamma_2(\x,\X) \vee \dots  \vee \gamma_{\ell}(\x,\X),
\end{eqnarray*}
where, for every $i\in\{1,\ldots,\ell\}$, 
\begin{eqnarray*}
\gamma_i(\x,\X) &=& \left( \bigwedge_{j=1}^{n_i} X_{\lambda_i(j)}(\x_{i,j}) \right) \wedge \gamma^{-}_i(\X,\y_i)  \wedge \delta_i(\x).
\end{eqnarray*}
The function $\lambda_i$ is defined analogously to $\lambda$ of the first part of the proof. The tuple $\x_{i,j}$ has $a_{\lambda_i(j)}$ variables for $j\in\{1,\ldots,n_i\}$, $\y_i$ has $p_i$ variables and are such that $(\x_{i,1},\ldots,\x_{i,n_i},\y_i) = \x.$ The formulas $\gamma^{-}_i(\X,\y_i)$, $\delta_i(\x)$ are defined analogously to $\varphi^{-}(\X,\y)$ and $\beta(\x)\,$\footnote[2]{Note that $\delta_i(\x)$ can include subformulas of the form $(\u = \u)$.} respectively (see \eqref{f1}). Let $g$ be a function that counts the number of possible orders over partitions on a $d$-tuple of variables. Let the formulas $\theta^i(\x)$ for $i\in\{1,\ldots,g(d)\}$ represent each of these orders over $\x$. Note that for every formula $\eta(\x)$,
\begin{eqnarray*}
\exists\x\:\eta(\x) &\equiv& \exists\x(\eta(\x)\wedge\theta^1(\x)) \vee \cdots \vee \exists\x(\eta(\x)\wedge\theta^{g(d)}(\x)).
\end{eqnarray*}
We define the following formulas $\xi_i(\x,\X)$, for $i\in\{1,\ldots,m\}$ where $m = \ell \cdot g(d)$, as follows: 
\begin{eqnarray*}
\xi_1(\x,\X) &=& \gamma_1(\x,\X) \wedge \theta^1(\x), \\
& \vdots & \\
\xi_{g(d)}(\x,\X) &=& \gamma_1(\x,\X) \wedge \theta^{g(d)}(\x), \\
\xi_{g(d)+1}(\x,\X) &=& \gamma_2(\x,\X) \wedge \theta^1(\x), \\
& \vdots & \\
\xi_{2\cdot g(d)}(\x,\X) &=& \gamma_2(\x,\X) \wedge \theta^{g(d)}(\x), \\
& \vdots & \\
\xi_{(\ell-1) g(d)+1}(\x,\X) &=& \gamma_{\ell}(\x,\X) \wedge \theta^1(\x), \\
& \vdots & \\
\xi_{\ell \cdot g(d)}(\x,\X) &=& \gamma_{\ell}(\x,\X) \wedge \theta^{g(d)}(\x).
\end{eqnarray*}
Having every disjunct with a total order allows us to eliminate the ones that are unsatisfiable for every $\L$-structure $\A$, that is, each $i\in\{1,\ldots,m\}$ such that for every assignment $\s$ to $\x,$ $f_{\xi_i(\s,\X)}(\A) = 0,$ for every $\A\in\Truc.$ Let $k$ be the number of disjuncts that are left after eliminating the unsatisfiable disjuncts. We use an injective\footnote[3]{We need this function to be injective because this way we can assure each one of the $k$ satisfiable disjuncts to be represented} function $\rho:\{1,\ldots,k\}\to\{1,\ldots,m\}$ such that $\varphi_i(\x,\X) = \xi_{\rho(i)}(\x,\X)$ is satisfiable, for every $i\in\{1,\ldots,k\}.$

We can conclude that
\begin{eqnarray*}
\varphi(\x,\X) &\equiv& \varphi_1(\x,\X) \vee \varphi_2(\x,\X) \vee \dots  \vee \varphi_k(\x,\X),
\end{eqnarray*}
and for every $i\in\{1,\ldots,k\}$,
\begin{eqnarray*}
\varphi_i(\x,\X) &=& \left( \bigwedge_{j=1}^{n_i} X_{\lambda_i(j)}(\x_{i,j}) \right) \wedge \varphi^{-}_i(\X,\y_i) \wedge \theta_i(\x) \wedge \beta_i(\x),
\end{eqnarray*}
where $n_i$ is the number of times a non-negated variable in $\X$ is referred to, according to the function $\lambda_i:\{1,\ldots,n_i\}\to\{1,\ldots,r\}$, $\y_i$ is a $p_i$-tuple of variables in $\x$, $\varphi_i^{-}(\X,\y_i)$ is a conjunction of negated predicates in $\X$, $\theta_i(\x)$ defines a total order on a partition of $\x$, and $\beta_i(\x)$ is an FO-formula over $\L$ which mentions all variables in $\x$. Let $\alpha_i(\x) = \theta_i(\x)\wedge\beta_i(\x).$

%% CLAIM CLAIM CLAIM CLAIM CLAIM CLAIM CLAIM CLAIM CLAIM CLAIM CLAIM CLAIM CLAIM CLAIM CLAIM CLAIM CLAIM CLAIM
%% CLAIM CLAIM CLAIM CLAIM CLAIM CLAIM CLAIM CLAIM CLAIM CLAIM CLAIM CLAIM CLAIM CLAIM CLAIM CLAIM CLAIM CLAIM

\begin{claim} \label{alpha}
Let $\A\in\Truc$ and $i\in\{1,\ldots,k\}$. For every assignment $\s$ to $\x$ such that $\A\models\alpha_i(\s)$, it holds that $f_{\varphi_i(\s,\X)}(\A) > 0.$
\end{claim}
\begin{proof}
Let $\s = (\s_1,\ldots,\s_{n_i},\t)$ such that $\A\models\alpha_i(\s)$, and let $\P = (\P_1,\ldots,\P_r)$ where $P_i = \bigcup_{j\in[1,n_i]:\:\lambda(j)=i}\{s_i\}$. We will show that $\A\models\varphi_i(\s,\P)$. By contradiction, suppose that $\A\not\models\exists\x\,\varphi_i(\x,\P)$. That is, there is no assignment $\e$ to $\x$ such that $\A\models\varphi(\e,\P).$ Following the same proof as in case (a) of Lemma \ref{first}, we conclude that $\varphi_i(\x,\P)$ is inconsistent. However, as we mentioned previously, all of such disjuncts have been eliminated, which leads to a contradiction.
\end{proof}

Our plan now is to exclude the lexicographically smallest assignment $\P$ such that $\A\models\exists\x\:\varphi_1(\x,\P)$, and if there is no such $\P$, exclude the lexicographically smallest $\Q$ such that $\A\models\exists\x\:\varphi_2(\x,\Q)$, and so on. As we already know how to exclude that assignment in the first disjunct, we will now deal with the next disjuncts. Similarly to the first part of the proof, for each $i\in\{1,\ldots,k\}$ let $m_i = \sum_{j = 1}^{n_i} a_{\lambda_i(j)}$ and let
\begin{multline*}
\alpha^i_{\min}(\u_1,\dots,\u_{n_i}) = \exists\y\Big[ \alpha_i(\u_1,\dots,\u_{n_i},\y)\wedge \\ \forall\v_1\cdots\forall\v_{n_i}\forall\w\Big(\big(\alpha_i(\v_1,\dots,\v_{n_i},\w)\wedge\bigvee_{j=1}^{n_i}(\u_j\neq\v_j)\big)\to \varphi_{m_i,<}((\u_1,\dots,\u_{n_i}),(\v_1,\dots,\v_{n_i}))\Big)\Big],
\end{multline*}
where $\u_j$ and $\v_j,$ have $a_{\lambda_i(j)}$ variables for $j\in\{1,\ldots,n_i\}$, and $\w$ has $p_i$ variables. Also, let
\begin{multline*}
\psi_i(\X) = \forall\x\:\neg\alpha_i(\x) \vee \exists\u_1\cdots\exists\u_{n_i}\bigg[\alpha^i_{\min}(\u_1,\dots,\u_{n_i}) \\ \wedge \bigg(\bigg(\bigvee_{j = 1}^{n_i}\neg X_{\lambda_i(j)}(\u_i) \bigg) \vee \bigvee_{j=1}^r \exists \v\Big( X_j(\v) \wedge \bigwedge_{\ell\in[1,n_i]: \lambda_i(\ell) = j} \v \neq \u_\ell\Big) \bigg) \bigg].
\end{multline*}
Note that $\psi_i(\X)$ excludes only the lexicographically smallest tuple of predicates $\P$ such that $\A\models\exists\x\:\varphi_i(\P,\x),$ if there is at least one. In other words, every assignment $\P^\prime \neq \P$ is such that $\A\models\psi_i(\P^\prime).$ Our new formula $\varphi_i^\prime(\x,\X)$ is defined as follows:
\begin{multline}
\varphi_i^\prime(\x,\X) = \varphi_i(\x,\X) \wedge \psi_1(\X) \wedge (\exists\v\:\alpha_1(\v)\vee\psi_2(\X)) \wedge \cdots \wedge \\ (\exists\v\:\alpha_1(\v)\vee\cdots\vee\exists\v\:\alpha_{i-1}(\v)\vee\psi_i(\X)).
\end{multline}