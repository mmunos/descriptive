Let $f \in \E1$ be defined by an extended quantifier free $\L$-formula $\varphi(\x,\X,\z)$, where $\x = (x_1,\dots,x_d), \X = (X_1,\dots,X_r)$ and $\z = (z_1,\dots,z_p)$. That is,
\begin{eqnarray*}
f(\A) &=& \mid \{ \langle\P,\e\rangle \mid \A \models \exists \x \ \varphi(\x,\P,\e) \} \mid,
\end{eqnarray*}
for every $\A = \langle A, \S^{\A}, \leq^{\A} \rangle \in \Truc$, where $\P = (P_1,\ldots,P_r)$, $P_i \subseteq A^{a_i}$ for every $i \in \{1,\ldots,r\}$ and $\e \in A^p$. In order to prove that $f\dotminus 1\in\E1$, we define the formulas $\varphi_i(\x,\X,\z)$ for $i\in\{1,\ldots,k\}$ in the same way as case 2, where
\begin{eqnarray*}
\varphi(\x,\X,\z) \equiv \varphi_1(\x,\X,\z) \vee \cdots \vee \varphi_k(\x,\X,\z),
\end{eqnarray*}
and
\begin{eqnarray*}
\varphi_i(\x,\X,\z) &=& \left( \bigwedge_{j=1}^{n_i} X_{\lambda_i(j)}(\x_{i,j},\z_{i,j}) \right) \wedge \varphi^{-}_i(\X,\y_i,\w_i) \wedge \theta_i(\x,\z) \wedge \beta_i(\x,\z),
\end{eqnarray*}
where $n_i$ is the number of times a non-negated variable in $\X$ is referred to, according to the function $\lambda_i:\{1,\ldots,n_i\}\to\{1,\ldots,r\}$, the tuple $\x_{i,j}$ has $b_{\lambda_i(j)}$ variables and the tuple $\z_{i,j}$ has $c_{\lambda_i(j)}$ for $j\in\{1,\ldots,n_i\}$ (note that $b_{\lambda_i(j)} + c_{\lambda_i(j)} = a_{\lambda_i(j)}$), $\y_i$ has $p_i$ variables, and $\w_i$ has $q_i$ variables. Furthermore, we have that $(\x_{i,1},\ldots,\x_{i,n_i},\y) = \x$ and $(\z_{i,1},\ldots,\z_{i,n_i},\w_i) = \z.$ The formula $\theta(\x,\z)$ defines a total order, analogously to case 2. 
\icristian{
Este orden total considera 	$\x,\z$ como un todo? Sería bueno recalcar esto.
}
The formulas $\varphi^{-}_i(\X,\y_i,\w_i)$, $\beta_i(\x,\z)$, are also defined analogously.

In this case, we mix both the strategies in cases 1 and 2. That is, we are going to {\em isolate} the lexicographically smallest tuple of predicates that satisfies the first satisfiable disjunct, and then {\em exclude} the lexicographically smallest tuple that satisfies the isolated disjunct.
\icristian{
Esta intuición no se entiende nada. Mejorar al igual que las intuiciones anteriores.
}

Let $m_i = \sum_{j = 1}^{n_i} a_{\lambda_i(j)}$ and
\begin{multline*}
\alpha^i_{\min}(\x_1,\dots,\x_{n_i},\z_1,\dots,\z_{n_i}) = \exists\y\exists\w\Big[ \alpha_i(\x_1,\dots,\x_{n_i},\y,\z_1,\dots,\z_{n_i},\w)\wedge \\ \forall\u_1\cdots\forall\u_{n_i}\forall\s\forall\v_1\cdots\forall\v_{n_i}\forall\t\Big(\big(\alpha_i(\u_1,\dots,\u_{n_i},\s,\v_1,\dots,\v_{n_i},\t)\wedge\bigvee_{j=1}^{n_i}(\x_j\neq\u_j \vee \z_j\neq\v_j)\big)\to \\ \varphi_{m_i,<}((\x_1,\dots,\x_{n_i},\z_1,\dots,\z_{n_i}),(\u_1,\dots,\u_{n_i},\v_1,\dots,\v_{n_i}))\Big)\Big],
\end{multline*}
and let
\begin{multline*}
\psi_i(\X,\z) = \forall\x\forall\v\:\neg\alpha_i(\x,\v) \vee \exists\u_1\cdots\exists\u_{n_i}\exists\w_1\cdots\exists\w_{n_i}\bigg[\alpha^i_{\min}(\u_1,\dots,\u_{n_i},\w_1,\dots,\w_{n_i}) \\ \wedge \bigg(\bigg(\bigvee_{j = 1}^{n_i}\neg X_{\lambda_i(j)}(\u_i,\w_i) \bigg) \vee \bigvee_{j=1}^r \exists \s\exists \t\Big( X_j(\s,\t) \wedge \bigwedge_{\ell\in[1,n_i]: \lambda_i(\ell) = j} \s \neq \u_\ell \vee \t \neq \w_\ell\Big) \bigg) \bigg] \vee \\ \exists\w(\exists\u\:\varphi(\u,\X,\w) \wedge \varphi_{p,<}(\w,\z)).
\end{multline*}
Let $\P$ and $\e$ be assignments to $\X$ and $\z$. Note that $\psi_i(\X,\z)$ excludes $\P$ and $\e$ only if $\P$ is the {\em lexicographically smallest} tuple of predicates (same as case 2) such that $\A\models\exists\x\:\varphi_i(\x,\P,\d)$, for some assignment $\d$ to $\z$, and $\e$ is the lexicographically smallest assignment to $\z$ such that $\A\models\exists\x\:\varphi_i(\x,\P,\e).$ We define $\varphi_i^\prime(\x,\X,\z)$ in the same way as case 2:
\begin{multline*}
\varphi_i^\prime(\x,\X,\z) = \varphi_i(\x,\X,\z) \wedge \psi_1(\X,\z) \wedge (\exists\u\exists\v\:\alpha_1(\u,\v)\vee\psi_2(\X,\z)) \wedge \cdots \wedge \\ (\exists\u\exists\v\:\alpha_1(\u,\v)\vee\cdots\vee\exists\u\exists\v\:\alpha_{i-1}(\u,\v)\vee\psi_i(\X,\z)).
\end{multline*}
Finally, let $\varphi^\prime(\x,\X,\z) = \bigvee_{i = 1}^k \varphi_i^\prime(\x,\X,\z)$.

Let $q$ be the least $i\in\{1,\ldots,k\}$ such that $\A\models\exists\x\:\varphi_i(\x,\R,\d)$ for some assignment $R$ to $\X$ and some assignment $\d$ to $\z$. Let $\P$ be the lexicographically smallest tuple of predicates 
\icristian{
Acá no esta bien definido que significa ``lexicographically smallest tuple of predicates''. También deberias hablar de ``$(\P, \z)$ be the lexicographically smallest tuple of predicates'' incluyendo a $\z$.
}
such that $\A\models\exists\x\:\varphi_q(\x,\P,\d^\prime)$ for some assignment $\d^\prime$ to $\z$. Let $\e$ be the lexicographically smallest assignment to $\z$ such that $\A\models\exists\x\:\varphi_q(\x,\P,\e).$ This formula is such that (a) $\A\models\exists\x\:\varphi(\x,\P,\e)$, (b) $\A\not\models\exists\x\:\varphi^\prime(\x,\P,\e)$ and $\P$ and $\e$ are the only assignments that satisfy (a) and (b). The proof of this is analogous to case 2. Therefore, we conclude that $f_{\exists\x\,\varphi^\prime(\x,\X,\z)} = f_{\exists\x\,\varphi(\x,\X,\z)}\dotminus 1.$