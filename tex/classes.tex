\section{The Extended Horn Counting Hierarchy}

We define syntactically the classes $\EiE$ and $\EiU$ as follows. First we define extended Horn clauses.
\begin{eqnarray*}
PL &::=& X_i(\x),\, i\in\N,\\
NL &::=& \neg X_i(\x),\, i\in\N \ \mid \ \exists x\, NL,\\
NC &::=& NL \ \mid \ \alpha, \alpha \mbox{ is an {\sc FO}-formula over } \L \ \mid \ (NC \vee NC),\\
HC &::=& NC \ \mid \ (NC \vee PL),
\end{eqnarray*}
where $\x$ is a tuple with the corresponding number of variables. Now we define the syntax of the classes inductively.
\begin{enumerate}
	\item $\EzeroE$:
	\begin{eqnarray*}
	E_0 &::=& HC \ \mid \ E_0 \wedge E_0.
	\end{eqnarray*}
	\item $\EzeroU$:
	\begin{eqnarray*}
	U_0 &::=& E_0.
	\end{eqnarray*}
	\item $\EipE$:
	\begin{eqnarray*}
	E_{i+1} &::=& U_i \ \mid \ \exists x \, E_{i+1}.
	\end{eqnarray*}
	\item $\EipU$:
	\begin{eqnarray*}
		U_{i+1} &::=& E_i \ \mid \ \forall x \, U_{i+1}.
	\end{eqnarray*}
\end{enumerate}
A function $f$ is in $\EiE$ (resp. $\EiU$) if there is an $\L$-formula $\varphi$ defined by the grammar $E_i$ (resp. $U_i$) such that $f = f_{\varphi}$.

The class $\PE$ is defined in [?] as $\PE = \{f\mid f\in\#P \mbox{ and its decision version } L_{f}\in P\}$.

\begin{theo}
	$\EtwoE \subseteq \PE$
\end{theo} 
\begin{proof}
	Let $f = f_{\varphi(\X,\z)}$ such that $\varphi(\X,\z)$ is defined by the grammar $E_2$. First we notice that, as stated in \cite{DBLP:journals/jcss/SalujaST95}, for every $\L$-formula $\varphi$, $f_{\varphi}\in\#P$.\\
	
	We will now prove that there is a polynomial time algorithm that decides $L_{f}$. This is equivalent to decide, given $\A= \langle A, \S^\A, \leq^\A \rangle\in\Truc$, whether there exist assignments $\P,\z$ such that $\A\models\varphi(\P,\z)$. This is the same as $\A\models\exists\X\exists\x\,\varphi(\X,\x)$. Since $\varphi(\X,\x)$ is defined by the grammar $E_2$, there exists $\psi(\X,\x,\y,\u)$ such that $\varphi(\X,\x) = \exists\x\exists\y\forall\u\exists\v\,\psi(\X,\x,\y,\u,\v)$, where every variable in $\v$ appears on a negated second-order literal $\neg X_i$, with $i\in\N$.
	
	Given $\A\in\Truc$ we generate an equivalent $\L$-formula $\theta(\X,\x,\y)$ with a series of operations. First, we replace every instance of $\v$ by a disjunction of all $r$-tuples in $A^r$, where $r$ is the number of variables in $\v$. The result is still a Horn clause. Second, we replace every instance of $\u$ by a conjunction of all $s$-tulples in $A^s$, where $s$ is the number or variables in $\u$. The result is still a conjunction of Horn clauses.
	
	Now we notice that $\zeta = \exists\X\exists\x\exists\y\,\theta(\X,\x,\y)$ is an existential second-order FO-formula, for which $\A\models\zeta$ can be decided in polynomial time.
\end{proof}