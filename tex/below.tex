\section{Counting classes below \#P}

We define $\totp$ using the following function. For each non-deterministic Turing machine $M$ and $x\in\{0,1\}^*$,
\[
	\tot_M(x) = (\text{\#paths of $M$ on input $x$}) - 1.
\]
A function $f:\{0,1\}^*\to\N$ is in $\totp$ if there exists a polynomial-time non-deterministic Turing machine $M$ such that $f = \tot_M$.

We define $\spanl$ using the following function. For each non-deterministic Turing machine $M$ and $x\in\{0,1\}^*$,
\[
	\sp_M(x) = \vert\{w \mid \text{$w$ is an output of $M$ on input $w$}\}\vert.
\]
A function $f:\{0,1\}^*\to\N$ is in $\spanl$ if there exists a log-space non-deterministic Turing machine $M$ such that $f = \sp_M$.

\begin{theo}
	$\spanl \subseteq \totp$.
\end{theo}
\begin{proof}
	Let $f:\{0,1\}^*\to\N$ be a function in $\spanl$ and let $M$ be a non-deterministic log-space Turing machine such that $f = \sp_M$. We will provide a polynomial-time non-deterministic Turing machine $M'$ such that for each $x\in\{0,1\}^*$ it holds that $\sp_M(x) = \tot_{M'}(x)$. First we note that given a configuration $s$ of $M$ on input $x$ checking whether there exists a final state of $M$ which is reachable from $s$ can be done in polynomial time. This implies that checking whether $f(x) > 0$ can also be done in polynomial time. We present this machine $M'$ as a non-deterministic algorithm.
	
	\begin{algorithm}
		
	\begin{algorithmic}[]
		\Begin
		\If{$f(x) = 0$} stop
		\Else \text{ choose from}
		\State stop
		\State call $\textsc{GenTree}_{f}(x,\{s_0\},\epsilon)$
		\EndIf
		\EndBegin
		\Procedure{$\textsc{GenTree}_{f}(x,S,y)$}{}
		\State $S_0 = \emptyset$, $S_1 = \emptyset$
		\For{$s\in S$}
		\State do a search in the configuration graph of $M$ with input $x$ from $s$:
		\If{$s'$ is reached after writing 0 in the output tape} $S_0 \gets s'$
		\EndIf
		\If{$s'$ is reached after writing 1 in the output tape} $S_1 \gets s'$
		\EndIf
		\EndFor
		\If{a final state from $M$ is reachable from some $s\in S_0$ and some $s'\in S_1$} choose between:
		\State call $\textsc{GenTree}_{f}(x,S_0,y0)$
		\State call $\textsc{GenTree}_{f}(x,S_1,y1)$
		\ElsIf{a final state from $M$ is reachable from some $s\in S_i$ but not from any $s'\in S_{1-i}$}
		\State call $\textsc{GenTree}_{f}(x,S_i,yi)$
		\ElsIf{no final state from $M$ is reachable from any $s\in S_0 \cup S_1$}
		\Return $y$
		\EndIf
		\EndProcedure
	\end{algorithmic}
	
	\end{algorithm}
	
	We use $s_0$ for the initial state of $M$ with input $x$. The algorithm can be executed in a nondeterministic polynomial-time machine, so we only need to show that the number of computation paths is equal to $\sp_M(x)+1$.
	
	For each output $y$ consider some computation path $p$ of $M$ with input $x$ that outputs $y$. Note there is at least one computation path in $M'$ that follows $p$ and outputs $y$. Also, note that each distinct computation path of $M'$ outputs a different $y$, since in each branch the word differs in the last bit. Taking into account the dummy path in line 4, we conclude that the number of computation paths of $M'$ with input $x$ is equal to $\sp_M(x)+1$. 
	
	
	
\end{proof}

Given a deterministic machine $M$ we define the function $f_M\{0,1\}^*\to\N$ as:
\[
	f_M(x) = \text{the output of $M$ on input $w$}
\]
for each $w\in\{0,1\}^*$. A function $f:\{0,1\}^*\to\N$ is in FP if there exists a deterministic polynomial-time machine $M$ such that $f = f_M$.
\begin{theo} \label{diff}
	For each function $f\in\shp$ there exists functions $g\in \totp$ and $h\in\text{FP}$ such that $f = g - h$.
\end{theo}
\begin{proof}
	Let $f\in\shp$, and let $r:\{0,1\}^*\to\{0,1\}^*$ be such that for each $w\in\{0,1\}^*$ it holds that $f(w) = \shsat(r(w))$, and be computed in polynomial time. Let $\varphi_w = r(w)$.
	
	Let $M$ be a non-deterministic machine that on input $w$ does the following: Computes $\varphi_w$. Let $n$ be the number of variables on $\varphi_w$. Then it chooses $x\in\{0,1\}$. If $x = 0$ then stop. Else, choose a valuation $\sigma$ of size $n$. If $\sigma\models\varphi_w$ then stop. Else, choose $y\in\{0,1\}$ then stop. Note that $\tot_M(w) = \shsat(\varphi_w)+2^n+1$. Let $g:\{0,1\}^*\to\N$ defined as $g(x) = \tot_M(x) - 1$. Clearly $g\in\totp$.
	
	Let $M'$ be a deterministic machine that on input $w$ computes $\varphi_w$ and returns $2^n$. Let $h\in\{0,1\}^*\to\N$ defined as $f_M$. Since $M'$ runs in polynomial time, $h\in\text{FP}$.
	
	Finally, for each $w\in\{0,1\}^*$ we have that $g(w) - h(w) = \shsat(\varphi_w) + 2^n - 2^n = f(w)$.
\end{proof}