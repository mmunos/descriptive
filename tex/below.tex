\section{Counting classes below \#P}

We define $\totp$ using the following function. For each non-deterministic Turing machine $M$ and $x\in\{0,1\}^*$,
\[
	\tot_M(x) = (\text{\#paths of $M$ on input $x$}) - 1.
\]
A function $f:\{0,1\}^*\to\N$ is in $\totp$ if there exists a polynomial-time non-deterministic Turing machine $M$ such that $f = \tot_M$.

We define $\spanl$ using the following function. For each non-deterministic Turing machine $M$ and $x\in\{0,1\}^*$,
\[
	\sp_M(x) = \vert\{w \mid \text{$w$ is an output of $M$ on input $w$}\}\vert.
\]
A function $f:\{0,1\}^*\to\N$ is in $\spanl$ if there exists a log-space non-deterministic Turing machine $M$ such that $f = \sp_M$.

\begin{theo}
	$\spanl \subseteq \totp$.
\end{theo}
\begin{proof}
	Let $f:\{0,1\}^*\to\N$ be a function in $\spanl$ and let $M$ be a non-deterministic log-space Turing machine such that $f = \sp_M$. We will provide a polynomial-time non-deterministic Turing machine $M'$ such that for each $x\in\{0,1\}^*$ it holds that $\sp_M(x) = \tot_{M'}(x)$. First we note that given a configuration $s$ of $M$ on input $x$ checking whether there exists a final state of $M$ which is reachable from $s$ can be done in polynomial time. This implies that checking whether $f(x) > 0$ can also be done in polynomial time. We present this machine $M'$ as a non-deterministic algorithm.
	
	\begin{algorithm}	
	\caption{Compute $f(x)+1$ paths}
	\begin{algorithmic}[]
		\Begin
		\If{$f(x) = 0$} stop
		\Else \text{ choose from}
		\State stop
		\State call $\textsc{GenTree}_{f}(x,\{s_0\},\epsilon)$
		\EndIf
		\EndBegin
		\Procedure{$\textsc{GenTree}_{f}(x,S,y)$}{}
		\State $S_0 = \emptyset$, $S_1 = \emptyset$
		\For{$s\in S$}
		\State do a search in the configuration graph of $M$ with input $x$ from $s$:
		\If{$s'$ is reached after writing 0 in the output tape} $S_0 \gets s'$
		\EndIf
		\If{$s'$ is reached after writing 1 in the output tape} $S_1 \gets s'$
		\EndIf
		\EndFor
		\If{a final state from $M$ is reachable from some $s\in S_0$ and some $s'\in S_1$} choose between:
		\State call $\textsc{GenTree}_{f}(x,S_0,y0)$
		\State call $\textsc{GenTree}_{f}(x,S_1,y1)$
		\ElsIf{a final state from $M$ is reachable from some $s\in S_i$ but not from any $s'\in S_{1-i}$}
		\State call $\textsc{GenTree}_{f}(x,S_i,yi)$
		\ElsIf{no final state from $M$ is reachable from any $s\in S_0 \cup S_1$}
		\Return $y$
		\EndIf
		\EndProcedure
	\end{algorithmic}
	\end{algorithm}
	
	We use $s_0$ for the initial state of $M$ with input $x$. The algorithm can be executed in a nondeterministic polynomial-time machine, so we only need to show that the number of computation paths is equal to $\sp_M(x)-1$.
	
	For each output $y$ consider some computation path $p$ of $M$ with input $x$ that outputs $y$. Note there is at least one computation path in $M'$ that follows $p$ and outputs $y$. Also, note that each distinct computation path of $M'$ outputs a different $y$, since in each branch the word differs in the last bit. Taking into account the dummy path in line 4, we conclude that the number of computation paths of $M'$ with input $x$ is equal to $\sp_M(x)+1$. 
	
	
	
\end{proof}

Given a deterministic machine $M$ we define the function $f_M\{0,1\}^*\to\N$ as:
\[
	f_M(x) = \text{the output of $M$ on input $w$}
\]
for each $w\in\{0,1\}^*$. A function $f:\{0,1\}^*\to\N$ is in FP if there exists a deterministic polynomial-time machine $M$ such that $f = f_M$.
\begin{theo} \label{diff}
	For each function $f\in\shp$ there exists functions $g\in \totp$ and $h\in\text{FP}$ such that $f = g - h$.
\end{theo}
\begin{proof}
	Let $f\in\shp$, and let $r:\{0,1\}^*\to\{0,1\}^*$ be such that for each $w\in\{0,1\}^*$ it holds that $f(w) = \shsat(r(w))$, and be computed in polynomial time. Let $\varphi_w = r(w)$.
	
	Let $M$ be a non-deterministic machine that on input $w$ does the following: Computes $\varphi_w$. Let $n$ be the number of variables on $\varphi_w$. Then it chooses $x\in\{0,1\}$. If $x = 0$ then stop. Else, choose a valuation $\sigma$ of size $n$. If $\sigma\models\varphi_w$ then stop. Else, choose $y\in\{0,1\}$ then stop. Note that $\tot_M(w) = \shsat(\varphi_w)+2^n$. Let $g:\{0,1\}^*\to\N$ defined as $g(x) = \tot_M(x)$. Clearly $g\in\totp$.
	
	Let $M'$ be a deterministic machine that on input $w$ computes $\varphi_w$ and returns $2^n$. Let $h\in\{0,1\}^*\to\N$ defined as $f_M$. Since $M'$ runs in polynomial time, $h\in\text{FP}$.
	
	Finally, for each $w\in\{0,1\}^*$ we have that $g(w) - h(w) = \shsat(\varphi_w) + 2^n - 2^n = f(w)$.
\end{proof}

\begin{theo}
	$\XE{2} \subseteq \totp$
\end{theo}
\begin{proof}
	Let $f:\Truc\to\N$ be a function in $\XE{2}$. Let $h$ be the function described in Theorem \ref{sigma2hard}. Consider a non-deterministic $M$ machine that on input $\A$ first computes $\varphi = h(\A)$ and then simulates the $\totp$ machine that computes $\shdhsat$. Since $h$ is a parsimonious reduction from $f$ to $\shdhsat$, it holds that $f(\A) = \tot_M(\A)$.
\end{proof}

\begin{theo}
	$\totp$ is closed under subtraction of a polynomial on the size of the input.
\end{theo}
\begin{proof}
	Let $f:\Truc\to\N$ be a function in $\totp$ and let $p$ be a polynomial. We will show a function $g\in\totp$ such that for each input $w$ it holds that $g(w) = \max(f(w) - p(\vert w \vert),0)$. Let $M$ be the non-deterministic Turing Machine such that $f(w) = \tot_M(w)$. 
	
	We construct a non-deterministic Turing Machine $M'$ that first computes $c = p(\vert w \vert)$. Then it deterministically simulates $M$ until it finds the end of a path, following the first path of each branch. Each time it finds the end of a path, it subtracts one from $c$ and saves the string of non-deterministic choices that were made. The machine then backtracks until the previous branching path and continues repeating the process. When all choices in a branch have been saved, it erases every path in that branch and saves only the choices until that point. If it happens that there are no more choices to make, and $c > 0$ the machine simply stops. If $c$ reaches 0, $M'$ simulates $M$ with input $w$. Each time the machine is about to do a non-deterministic choice it checks if it corresponds to any of the saved paths. If it is the case, it does not follow the branch.
	
	We will now show that $g(w) = \tot_{M'}(w)$. Let $w\in\{0,1\}$. If $\tot_{M}(w) > p(\vert w \vert)$, then clearly $\tot_{M'}(w) = \tot_{M}(w) - p(\vert w \vert)$. If $\tot_{M}(w) \leq p(\vert w \vert)$, then it only computes the first simulation path and $\tot_{M'}(w) = 0$.
\end{proof}