\begin{theo}
If $\#\Sigma_1$ is closed under substraction, then {\sc P} = {\sc NP}.
\end{theo}
\begin{proof}
Suppose that $\#\Sigma_1$ is closed under substraction, that is, for each pair of functions $f,g\in\#\Sigma_1,$ there is an $h\in\#\Sigma_1$ such that $(f\dotminus g)(\A) = h(\A)$ for each $\A\in\Truc$.

Let $\A = \langle A, S_1^\A, S_2^\A, S_3^\A, S_4^\A, \leq^\A \rangle$ be an $\L-$structure that represents an instance of a 3DNF formula $\psi$, where $A$ is the set of variables mentioned in $\psi$, $S_i^\A$ is a ternary relation described as follows, for each $i\in\{1,2,3,4\}$:
\begin{eqnarray*}
S_1^\A &=& \{(a_1,a_2,a_3)\mid (\neg a_1 \wedge \neg a_2 \wedge \neg a_3) \mbox{ appears as a disjunct in }\psi\},\\
S_2^\A &=& \{(a_1,a_2,a_3)\mid ( a_1 \wedge \neg a_2 \wedge \neg a_3) \mbox{ appears as a disjunct in }\psi\},\\
S_3^\A &=& \{(a_1,a_2,a_3)\mid ( a_1 \wedge  a_2 \wedge \neg a_3) \mbox{ appears as a disjunct in }\psi\},\\
S_4^\A &=& \{(a_1,a_2,a_3)\mid ( a_1 \wedge  a_2 \wedge  a_3) \mbox{ appears as a disjunct in }\psi\}.
\end{eqnarray*}
Now let $f_{\#3DNF}$ be a function that counts the satisfying assignments to a 3DNF formula $\psi$. As shown in [1], $f_{\#3DNF} \in \#\Sigma_1$. Let $f_{all} = f_{\exists x\:\varphi(x,X)}$, where
$$
\varphi(x,X) = (X(x) \vee \neg X(x)).
$$
Note that $f_{all}$ counts every possible truth assignment (satisfying or not) to a 3DNF formula. As we supposed initially, let $h\in\#\Sigma_1$ such that $f_{all}-f_{\#3DNF} = h$. For each structure $\A$ that represents a 3DNF formula $\psi$, it holds that $h(\A) = f_{all}(\A)-f_{\#3DNF}(\A) = 0$ if and only if $\psi$ is a tautology, so the decision version $L_h$ of $f_{all}-f_{\#3DNF}$ is {\sc co-NP}-complete. However, as we showed previously in Theorem \ref{decisionptime}, $L_h \in \textsc{P}$. Then, $\textsc{co-NP} \subseteq \textsc{P}$, from which we conclude that $\textsc{P} = \textsc{NP}$.
\end{proof}