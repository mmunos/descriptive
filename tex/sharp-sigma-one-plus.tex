\documentclass[12pt]{article}
\usepackage[utf8]{inputenc}
\usepackage{amsmath}
\usepackage{amsthm} 
\usepackage{fullpage}
\usepackage{amsfonts}
\usepackage{amssymb}
\usepackage{natbib}
\usepackage{subfiles}
\usepackage{bm}

\def\dotminus{\mathbin{\ooalign{\hss\raise1ex\hbox{.}\hss\cr
  \mathsurround=0pt$-$}}}

\def\Eo{\#\Sigma_0[\mbox{{\sc FO}}]}
\def\E1{\#\Sigma_1[\mbox{{\sc FO}}]}

\def\Truc{\textsc{Struct}[\L]}

\def\A{{\frak A}}
\def\B{{\frak B}}
\def\C{{\cal C}}
\def\F{{\cal F}}
\def\L{{\cal L}}
\def\N{\mathbb{N}}
\def\P{\vec{P}}
\def\Q{\vec{Q}}
\def\R{\vec{R}}
\def\S{\vec{S}}
\def\X{\vec{X}}
\def\Y{\vec{Y}}
\def\Z{\vec{Z}}
%% a - arity of \X / arity of assignments \P to \X
%% b - arity of predicates in \S
%% c - arity of auxiliar predicates/variables
\def\c{\vec{c}} %% super auxiliar elements
\def\d{\vec{d}} %% counted elements
\def\e{\vec{e}} %% counted elements
%% f - counting function
%% g - other functions
%% h - other functions
%% i - index
%% j - index
%% k - emergency index / size of tuple
%% l - emergency index / size of tuple
\def\l{\vec{\ell}}
%% m - size of variable tuple
%% n - size of predicate tuple
%% o - not used
%% p - 
%% q - 
%% r - size of \X / \P
\def\s{\vec{s}}
%% t - size of \S
\def\t{\vec{t}}
\def\u{\vec{u}} %% auxiliary variables
\def\v{\vec{v}} %% auxiliary variables
\def\w{\vec{w}} %% auxiliary variables
\def\x{\vec{x}} %% quantified variables
\def\y{\vec{y}} %% auxiliary variables
\def\z{\vec{z}} %% open variables
\def\ep{\vec{o}}
\def\ga{\vec{p}}

\def\np{\mbox{{\sc NP}}}

\newtheorem{theo}{Theorem}
\newtheorem{lemma}{Lemma}[theo]
\newtheorem{claim}{Claim}[theo]
\newtheorem{coro}{Corollary}[theo]

\begin{document}


\begin{center}
{ \LARGE \bf
  Some properties of $\E1$
}
\end{center}

We define the vocabulary $\L = \{ S_1,\dots,S_t, \leq \}$, where $S_1,\dots,S_t$ have arity $b_1,\dots,b_t$. Let
\begin{eqnarray*}
\Truc &=& \{\A \mid \A \text{ is an } \L \text{-structure with a finite domain } A \text{ such that} \\
&& \leq \text{ is interpreted as a total order for } A \}.
\end{eqnarray*}
We also define a set of second order variables ${\cal X} = \{ X_i \mid i\in\N \}$ where $X_i$ has arity $a_i$, and for every $n \in \N$ there are infinite variables in ${\cal X}$ of arity $n$. A quantifier-free $\L$-formula is defined by the following grammar:
\begin{eqnarray*}
\varphi &::=& x = y \ \mid \ S_i(x_1,\dots,x_{b_i}), i \in \{1,\dots,t\} \ \mid \ x \leq y \ \mid \\
&& X_i(x_1,\dots,x_{a_i}), i\in\N \ \mid \\ 
&& (\neg \varphi) \ \mid \ (\varphi \wedge \varphi) \ \mid \ (\varphi \vee \varphi),
\end{eqnarray*}
where $x,y$ and $x_i$ are first order variables for every $i$. We now define an {\sc FO}-extended quantifier-free $\L$-formula as follows:
\begin{eqnarray*}
\varphi &::=& \alpha, \alpha \text{ is an FO-formula over } \L  \ \mid \\
&& X_i(x_1,\dots,x_{a_i}), i\in\N \ \mid \ \\
&& (\neg \varphi) \ \mid \ (\varphi \wedge \varphi) \ \mid \ (\varphi \vee \varphi).
\end{eqnarray*}
Let $\Y = (Y_1,\dots,Y_q)$ be a tuple of second-order variables of arity $c_1,\ldots,c_q$, and let $\y$ be a tuple of first order variables. For every $\L$-formula $\psi(\Y,\y)$, we define the function $f_{\psi(\Y,\y)}:\Truc \to \N$ as follows:
\begin{eqnarray*}
f_{\psi(\Y,\y)}(\A) &=& \mid \{ \langle \P, \e \rangle \mid \A \models \psi(\P,\e) \} \mid,
\end{eqnarray*}
for every $\A = \langle A, \S^{\A}, \leq^{\A} \rangle \in \Truc$, where $\P = (P_1,\ldots,P_q)$ is a tuple of predicates of arity $c_1,\ldots,c_q$, $P_i \subseteq A^{c_i}$ for every $i \in \{1,\ldots,q\}$, and $\e$ is a tuple of elements from $\A$.\\

A function $f:\Truc \to \N$ is in $\Eo$ if there exists an {\sc FO}-extended quantifier-free $\L$-formula $\varphi(\Y,\y)$ such that $f = f_{\varphi(\Y,\y)}.$

Similarly, a function $f:\Truc \to \N$ is in $\E1$ if there exists an {\sc FO}-extended quantifier-free $\L$-formula $\varphi(\x,\Y,\y)$ such that $f = f_{\exists \x \: \varphi(\x,\Y,\y)}.$\\

\begin{theo}
$\Eo \subseteq$ {\sc FP}.
\end{theo}
\begin{proof}
Let $f \in \Eo$, and let $\varphi(\X,\x)$ be and {\sc FO}-extended quantifier-free $\L$-formula such that:
\begin{eqnarray*}
f(\A) &=& \vert \{\langle \P,\e  \rangle \mid \A \models \varphi(\P,\e) \} \vert
\end{eqnarray*}
for each $\A = \langle A, \S^{\A}, \leq^{\A} \rangle \in \Truc$, where $\e \in A^m$ and $\P = (P_1,\ldots,P_q)$ is a tuple of predicates. We will now show that counting $f(\A)$ can be done in polynomial time.

For each {\sc FO}-formula $\beta(\x)$ in $\varphi(\X,\x)$, let $R_{\beta}$ be a predicate of arity $m$. Let $\A^\prime = \langle A, \S^{\A}, R_{\beta}^{\A^\prime}, \leq^{\A} \rangle \in \Truc$, where $R_{\beta}^{\A^\prime} = \{\d \mid \A \models \beta(\d)\}$. Note that each $\beta(\x)$ is fixed in $\varphi(\X,\x)$, and for each $\d \in A^m$, checking whether $\A \models \beta(\d)$ can be done in polynomial time. Therefore, generating $R_{\beta}^{\A^\prime}$ can also be done in polynomial time.

Let $\psi(\X,\x)$ be obtained by replacing each {\sc FO}-formula $\beta(\x)$ in $\varphi(\X,\x)$ by $R_{\beta}(\x)$. Also, let $g = f_{\psi(\X,\x)}$. Note that for each tuple of predicates $\P$ and each $\e \in A^m$, $\A \models \varphi(\P,\e)$ if and only if $\A^\prime \models \psi(\P,\e)$, and so, $g(\A^\prime) = f(\A)$. However, $\psi(\X,\x)$ is a quantifier-free $\L$-formula, and therefore, $g \in \#\Sigma_0$. Since it is shown in \cite{DBLP:journals/jcss/SalujaST95} that $\#\Sigma_0 \subseteq$ {\sc FP}, we conclude that $f(\A)$ can be evaluated in polynomial time.

\end{proof}

The {\em decision problem} associated to a function $f$ is defined by the language $L_f = \{\A \in \Truc \mid f(\A) > 0\}$.

%!TEX root = sharp-sigma-one-plus.tex
\begin{theo} \label{decisionptime}
The decision problem associated to a function in $\E1$ is in \textsc{P}.
\end{theo}
\begin{proof}
Let $f$ be a function in $\E1$. Then there is an extended quantifier-free $\L-$formula $\varphi(\x,\X,\z)$ such that
\begin{eqnarray*}
f(\A) &=& \mid \{ \langle\P,\e\rangle \mid \A \models \exists \x \, \varphi(\x,\P,\e) \} \mid,
\end{eqnarray*}
where $\A = \langle A, \S^{\A}, \leq^{\A} \rangle \in \Truc$, $A = \{a_1,\ldots,a_{\vert A \vert}\}$, $\z$ is an $m$-tuple of variables and $\x$ is a $k$-tuple of variables. As said in \cite{DBLP:journals/jcss/SalujaST95}, a function $h:\Truc \to \N$ is in $\#\Sigma_0$ if there exists a quantifier-free $\L$-formula $\theta(\Y,\y)$ such that $h = f_{\theta(\Y,\y)}$. Let $\y = (\x,\z)$ and let $\psi(\X,\y) = \varphi(\x,\X,\z)$. Moreover, let $g = f_{\psi(\X,\y)}$.
\begin{claim}
For each $\A \in \Truc$, $f(\A) > 0$ iff $g(\A) > 0$.
\end{claim}
\begin{proof}
($\Rightarrow$) Suppose $f(\A) > 0$. Let $\P$ and $\e$ be such that $\A \models \exists \x \, \varphi(\x,\P,\e)$. It follows that there is at least one $\d \in A^k$ such that $\A \models \varphi(\d,\P,\e) = \psi(\P,(\d,\e))$. Therefore, $g(\A) > 0$. ($\Leftarrow$) Suppose $g(\A) > 0$. Let $\Q$ and $\c = (\c_1,\c_2)$, where $\c_1$ and $\c_2$ have $k$ and $m$ elements respectively, be such that $\A \models \psi(\Q,\c) = \varphi(\c_1,\Q,\c_2)$. Therefore, $\A \models \exists \x \, \varphi(\x,\Q,\c_2)$, from which we conclude that $f(\A) > 0$.
\end{proof}
Note that $\psi(\X,\y)$ is an {\sc FO}-extended $\L$-formula, so $g \in\Eo$. As we showed, computing $g$ is enough to solve the decision version of $f$, but as we showed in Theorem \ref{fp1}, $g(\A)$ can be evaluated in polynomial time, from which we conclude that the decision version of $f$ is in {\sc P}.
\end{proof}

For a given pair of functions $f,g$, we define $f \dotminus g$ as follows:
\begin{eqnarray*}
(f \dotminus g)(\A) =
\begin{cases}
f(\A)-g(\A), & \text{if }f(\A)>g(\A) \\
0, & \text{if }f(\A) \leq g(\A).
\end{cases}
\end{eqnarray*}
for every $\L$-structure $\A \in \Truc$. A function class $\F$ is {\em closed under substraction} if for every pair of functions $f,g \in \F$, it holds that $f \dotminus g \in \F$.

\begin{theo}
$\#\Sigma_1$ is closed under substraction $\Rightarrow$ P = NP.
\end{theo}
\begin{proof}
Let $F_{\#3DNF}$ be a function that counts the satisfying assignments to a 3DNF formula. As shown in [1], $F_{\#3DNF} \in \#\Sigma_1$. Let $F_{2^n}$ be a $\#\Sigma_1$ function that counts every possible truth assignment (satisfying or not) to a 3DNF formula. Suppose that $F_{2^n}-F_{\#3DNF} \in \#\Sigma_1$. This function equals 0 only if the instanced formula is a tautology, so the decision version of it is co-NP-complete. However, as we showed previously in Theorem \ref{decisionptime}, it is also in P. Then, co-NP $\subseteq$ P.
\end{proof}

\begin{theo}
$\#\Sigma_1 \subsetneq \E1$
\end{theo}
\begin{proof}
We will show that the $\E1$ function $f$ defined by $\varphi(x_1) = (x_1 = x_1) \wedge \forall y \, S(y)$ is not in $\#\Sigma_1$. By contradiction, suppose that it is. Let $\A = \langle A =\{1\},S^\A = \{1\},\leq^\A=\{(1,1)\} \rangle$. Then, $f(\A) = 1$. Now let $\B = \langle B =\{1,2\},S^\B = \{1\},\leq^\B=\{(1,1),(1,2),(2,2)\} \rangle$. Note that $\A$ is an induced substructure of $\B$.

We have that for each function $g \in \#\Sigma_1$ and structures $\A_1,\A_2\in\Truc$, if $\A_1$ is an induced substructure of $\A_2$, then $g(\A_1) \leq g(\A_2)$ \cite{DBLP:journals/jcss/SalujaST95}. Therefore, $f(\B) \geq f(\A) = 1$. However, there is no assignment $s\in B$ to $x$ such that $\B\models\:\varphi(s)$, so $f(\B) = 0$, which leads to a contradiction.
\end{proof}

For a given function $f$, we define $f \dotminus 1$ as follows:
\begin{eqnarray*}
f \dotminus 1(\A) =
\begin{cases}
f(\A)-1, & \text{if }f(\A) > 0 \\
0, & \text{if }f(\A) = 0.
\end{cases}
\end{eqnarray*}
for every $\L$-structure $\A \in \Truc$. A function class $\F$ is {\em closed under substraction by one} if for every function $f \in \F$, it holds that $f \dotminus 1 \in \F$.

\begin{theo}
$\E1$ is closed under substraction by one.
\end{theo}
\begin{proof}
Consider an $\L$-formula of the form $\exists\x\:\varphi(\x,\X,\z)$ where $\z = (z_1,\dots,z_d)$ and $\X = (X_1,\dots,X_r)$. There are three possibilities regarding the size of the tuples of free variables $\z$ and $\X$: (1) $d>0$ and $r=0$ (2) $d=0$ and $r>0$ (3) $d,r>0$. This separates the proof in three cases:
\begin{enumerate}
\item Let $f \in \E1$ be defined by an extended quantifier free $\L$-formula $\varphi(\x,\z)$, where $\z = (z_1,\dots,z_d)$. That is,
\begin{eqnarray*}
	f(\A) &=& \mid \{ \langle\e\rangle \mid \A \models \exists \x \ \varphi(\x,\e) \} \mid,
\end{eqnarray*}
for every $\A = \langle A, \S^{\A}, \leq^{\A} \rangle \in \Truc$, where $\e \in A^d$. Our goal here is to eliminate the lexicographically smallest sequence of variables, which can be done easily. First, let $\y = (y_1,\dots,y_k)$, $\y\,^\prime = (y_1^\prime,\dots,y_k^\prime)$ and
\begin{eqnarray*}
	\varphi_{k,<}(\y\,^\prime,\y) &=& \bigvee_{i = 1}^k \left( \bigwedge_{j=1}^{i-1} y_j^\prime = y_j \wedge y_i^\prime < y_i \right).
\end{eqnarray*}
This formula is true if $\y\,^\prime$ is lexicographically smaller than $\y$. Now, let $f'$ be defined by
\begin{eqnarray*}
	\varphi^\prime(\x,\z) &=& \varphi(\x,\z) \wedge \exists \z\,^\prime (\varphi(\x,\z\,^\prime) \wedge \varphi_{d,<}(\z\,^\prime,\z ) ).
\end{eqnarray*}
If $f(\A)>0$, then $f'(\A)$ will count exactly one element less than $f(\A)$. Otherwise, if $f(\A)=0$, then $\A \not\models\exists \x\,\varphi(\x,\e)$ for every tuple $\e$ of elements in $A$, so $\A \not\models\exists \x\,\varphi^\prime(\x,\e)$ for every $\e$ and, therefore, $f'(\A)=0$. Hence, $f' = f\dotminus 1,$ from which we conclude that $f\dotminus 1\in\E1.$

\item Let $f \in \E1$ be defined by an extended quantifier free $\L$-formula $\varphi(\x,\X)$ where $\x = (x_1,\dots,x_d)$ and $\X = (X_1,\dots,X_r)$. That is,
\begin{eqnarray*}
	f(\A) &=& \mid \{ \langle\P\rangle \mid \A \models \exists \x \ \varphi(\x,\P) \} \mid \label{f1},
\end{eqnarray*}
for every $\A = \langle A, \S^{\A}, \leq^{\A} \rangle \in \Truc$, where $\P = (P_1,\ldots,P_r)$ and $P_i \subseteq A^{a_i}$ for every $i \in \{1,\ldots,r\}$. For the time being, suppose that
\begin{eqnarray}
\varphi(\x,\X) &=& \left( \bigwedge_{i=1}^n X_{\lambda(i)}(\x_i) \right) \wedge \varphi^{-}(\X,\y) \wedge \theta(\x) \wedge \beta(\x)
\end{eqnarray}
where $n$ is the number of times a non-negated variable in $\X$ is referred to, according to the function $\lambda:\{1,\ldots,n\}\to\{1,\ldots,r\}$, $\y$ is a $p$-tuple of variables in $\x$, $\varphi^{-}(\X,\y)$ is a conjunction of negated predicates in $\X$, $\theta(\x)$ defines a total order on a partition of $\x$, and $\beta(\x)$ is an FO-formula over $\L$ which mentions all variables in $\x$. Note that $\theta(\x)$ also mentions all variables in $\x$. We also assume that $(\x_1,\dots,\x_n,\y) = \x$. As an example, the following formula is of this form:
\begin{multline*}
\varphi(\x,\X) =  X_1(x_1,x_2) \wedge X_3(x_3) \wedge X_2(x_4,x_5) \wedge X_3(x_6) \wedge \neg X_1(x_7,x_8) \wedge \\ (x_1 < x_2 \wedge x_1 = x_3 \wedge x_1 = x_4 \wedge x_2 = x_8 \wedge x_2 = x_5 \wedge x_8 < x_6 \wedge x_6 = x_7 ) \wedge \\ \forall z\big( S_1(x_1,z,x_2) \wedge x_3 = x_3 \wedge x_4 = x_4 \wedge x_5 = x_5 \wedge x_6 = x_6 \wedge x_7 = x_7 \wedge x_8 = x_8 \big),
\end{multline*}
where $\x = (x_1,x_2,x_3,x_4,x_5,x_6,x_7,x_8)$ and $\X = (X_1,X_2,X_3)$. Here, $n = 4$, $\lambda(1) = 1$, $\lambda(2) = \lambda(4) = 3$ and $\lambda(3) = 2$, $\x_1 = (x_1,x_2)$, $\x_2 = (x_3)$, $\x_3 = (x_4,x_5)$, $\x_4 = (x_6)$ and $\y = (x_7,x_8)$. Moreover, $\varphi^{-}(\X,\y) = \neg X_1(x_7,x_8)$, $\theta(\x) = (x_1 < x_2 \wedge x_1 = x_3 \wedge x_1 = x_4 \wedge x_2 = x_8 \wedge x_2 = x_5 \wedge x_8 < x_6 \wedge x_6 = x_7 )$, which defines a total order on the partition of $\x$ $\{\{x_1,x_3,x_4\},\{x_2,x_5,x_8\},\{x_6,x_7\}\}$, and $\beta(\x) = \forall z\big( S_1(x_1,z,x_2) \wedge x_3 = x_3 \wedge x_4 = x_4 \wedge x_5 = x_5 \wedge x_6 = x_6 \wedge x_7 = x_7 \wedge x_8 = x_8 \big)$.

Similarly to the previous proof, we would like to eliminate the {\em lexicographically smallest}\footnote{We consider the lexicographically smallest tuple of predicates as the one in which its predicates contain the lexicographically smallest tuples and do not contain any more tuples than those} tuple of predicates that satisfies the formula \eqref{f1}. Let $\u_i$ be a $a_{\lambda(i)}$-tuple of variables for every $i \in \{1,\dots,n\}$, and let $m = \sum_{i = 1}^n a_{\lambda(i)}$ be the number of variables of $(\x_1,\dots,\x_n)$. We now define
\begin{multline*}
\alpha_{\min}(\u_1,\dots,\u_n) = \exists\y\Big[ \alpha(\u_1,\dots,\u_n,\y)\wedge \\ \forall\v_1\cdots\forall\v_n\forall\w\Big(\big(\alpha(\v_1,\dots,\v_n,\w)\wedge\bigvee_{i=1}^n(\u_i\neq\v_i)\big)\to \varphi_{m,<}((\u_1,\dots,\u_n),(\v_1,\dots,\v_n))\Big)\Big],
\end{multline*}
where $\alpha(\x) = \theta(\x) \wedge \beta(\x)$. Note that $\alpha_{\min}$ is satisfied only by the lexicographically \linebreak smallest assignment $(\d_1,\dots,\d_n)$ to $(\x_1,\dots,\x_n)$ such that $\A\models\theta(\d_1,\dots,\d_n,\l)$ and $\A\models\beta(\d_1,\dots,\d_n,\l)$ for some $\l \in A^p$. Our new formula is
\begin{multline}
\varphi^\prime(\x,\X) = \left( \bigwedge_{i=1}^n X_{\lambda(i)}(\x_i) \right) \wedge \varphi^{-}(\X,\y) \wedge \theta(\x) \wedge \beta(\x)\wedge \\ \exists\u_1\cdots\exists\u_n\bigg[\alpha_{\min}(\u_1,\dots,\u_n) \wedge \bigg(\bigg(\bigvee_{i = 1}^{n}\neg X_{\lambda(i)}(\u_i) \bigg) \vee \bigvee_{i=1}^r \exists \v\Big( X_i(\v) \wedge \bigwedge_{j\in[1,n]:\: \lambda(j) = i} \v \neq \u_j\Big) \bigg) \bigg] \label{f2}.
\end{multline}
We now show a result by which the main proof will follow.
\begin{lemma} \label{first}
	$f_{\exists \x \: \varphi^\prime(\x,\X)} = f_{\exists \x \: \varphi(\x,\X)} \dotminus 1$.
\end{lemma}
\begin{proof}
	Let $\A \in \Truc$. Consider two cases: assume first that $\A\models\exists\x\,\varphi(\x,\R)$ for some assignment $\R$ to $\X$. Let $\d = (\d_1,\dots,\d_n,\ep)$ be the lexicographically smallest assignment to $\x$ for which $\A\models\alpha(\d)$, where $\d_i$ is the respective assignment to $\x_i$, for every $i\in\{1,\dots,n\}$, and $\ep$ is an assignment for $\y$. Consider now the tuple $\P = (P_1,\dots,P_r)$ where $P_i = \bigcup_{j\in[1,n]:\:\lambda(j)=i}\{\d_j\}$. We will show that this assignment to $\X$ is such that (a) $\A\models\exists\x\,\varphi(\x,\P),$ (b) $\A\not\models\exists\x\,\varphi^\prime(\x,\P)$ and (c) $\P$ is the only assignment that satisfies (a) and (b).
	\begin{enumerate}
		\item[(a)] By contradiction, suppose that $\A\not\models\exists\x\,\varphi(\x,\P)$. That is, there is no assignment $\s$ to $\x$ such that $\A\models\varphi(\s,\P).$ Since $\d$ is such that $\A\models \bigwedge_{i=1}^n P_{\lambda(i)}(\d_i)$ and $\A\models \alpha(\d)$, it follows that $\A\not\models\varphi^{-}(\P,\ep)$ (since $\alpha(\d)=\theta(\d)\wedge\beta(\d)$). Therefore, there is an $i\in\{1,\ldots,n\}$ such that $\neg P_{\lambda(i)}(\d_i)$ appears in $\varphi^{-}(\P,\ep)$. Then, there is an $a_{\lambda(i)}$-tuple $\z$ in $\y$ such that $\neg X_{\lambda(i)}(\z)$ appears in $\varphi^{-}(\X,\x)$. We know that either (1) $\theta(\x)\models \z = \x_i$, (2) $\theta(\x)\models \varphi_{<,a_i}(\z,\x_i)$ or (3) $\theta(\x)\models \varphi_{<,a_i}(\x_i,\z)$. Considering that (2) and (3) are not possible given that both $\z$ and $\x_i$ are assigned the value $\d_i$ and $\A\models\theta(\d)$, we have that $\theta(\x)\models \z = \x_i$. But if this is the case, then $X_{\lambda(i)}(\x_i), \neg X_{\lambda(i)}(\z)$ and $\z = \x_i$ are all logical consequences of $\varphi(\x,\X)$, which means $\varphi(\x,\X)$ is inconsistent. That is, there's no structure $\A^\prime\in\Truc$ such that $\A^\prime\models\exists\x\:\varphi(\x,\R^\prime)$ for any assignment $\R^\prime$ to $\X$. In particular, $\A\not\models\exists\x\,\varphi(\x,\R^\prime)$ for every possible assignment $\R^\prime$ to $\X$, which contradicts the initial assumption that $\A\models\exists\x\,\varphi(\x,\R)$ for some assignment $\R$ to $\X$.
		\item[(b)] Note that if $\A \models \alpha_{\min}(\c_1,\ldots,\c_n)$, then necessarily $\c_i = \d_i$ for $i\in\{1,\ldots,n\}.$ However, by the construction of $\P$, we see that 
		$$\A\not\models\bigvee_{i = 1}^{n}\neg P_{\lambda(i)}(\d_i) \text{ and that } \A\not\models\bigvee_{i=1}^r \exists \v\Big( P_i(\v) \wedge \bigwedge_{j\in[1,n]:\: \lambda(j) = i} \v \neq \d_j\Big).$$ Then, $\A\not\models\exists\x\,\varphi^\prime(\x,\P)$.
		\item[(c)] By contradiction, let $\P^\prime \neq \P$ be such that $\A\models\exists\x\,\varphi(\x,\P^\prime)$ and $\A\not\models\exists\x\,\varphi^\prime(\x,\P^\prime)$. We consider two cases: first, suppose that $\P^\prime$ is missing a tuple of $\P$. Let $i\in\{1,\ldots,n\}$ such that in $\d_i$ is not in $P^\prime_i$. Then $\A\models \neg P^\prime_i(\d_i)$, and so,$$\A\models \bigg(\bigvee_{i = 1}^{n}\neg X_{\lambda(i)}(\d_i) \bigg), $$ from which we conclude that $\A\models\exists\x\,\varphi^\prime(\x,\P^\prime)$. Second, suppose there is some predicate $\P^\prime_i$ in $\P^\prime$ which has a tuple that $P_i$ does not have. If this is the case, then $$\A\models\bigvee_{i=1}^r \exists \v\Big( P_i(\v) \wedge \bigwedge_{j\in[1,n]:\: \lambda(j) = i} \v \neq \u_j\Big),$$ so $\A\models\exists\x\,\varphi^\prime(\x,\P^\prime)$. On both cases, we have a contradiction.
	\end{enumerate}
	With this, we conclude that for every $\A\in\Truc$ such that $f_{\exists\x\,\varphi(\x,\X)}(\A)>0,$ we have that $f_{\exists\x\,\varphi^\prime(\x,\X)}(\A) = f_{\exists\x\,\varphi(\x,\X)}(\A)-1.$
	
	Second, assume that there is no assignment $\R$ to $\X$ such that $\A\models\exists\x\,\varphi(\x,\R)$. Let $\P$ be an arbitrary assignment to $\X$. Since $\A\not\models\exists\x\,\varphi(\x,\P)$, we see that $\A\not\models\exists\x\,(\varphi(\x,\P)\wedge\psi(\x,\P))$ for any formula $\psi(\x,\P)$. It follows that there is no assignment $\R$ to $\X$ such that $\A\models\exists\x\,\varphi^\prime(\x,\R)$. And so, for every $\A\in\Truc$ such that $f_{\exists\x\,\varphi(\x,\X)}(\A)=0,$ it holds that $f_{\exists\x\,\varphi^\prime(\x,\X)}(\A) = 0.$ We conclude that $f_{\exists\x\,\varphi^\prime(\x,\X)} = f_{\exists\x\,\varphi(\x,\X)}\dotminus 1,$ which was to be shown.
\end{proof}
We now continue with the general case, in which $\varphi(\x,\X)$ is an arbirtary extended quantifier-free $\L$-formula. By using a standard DNF transformation algorithm and considering FO-formulas over $\L$ as literals, we can find formulas $\gamma_i(\x,\X)$ with $i\in\{1,\ldots,\ell\}$ such that
\begin{eqnarray*}
	\varphi(\x,\X) &\equiv& \gamma_1(\x,\X) \vee \gamma_2(\x,\X) \vee \dots  \vee \gamma_{\ell}(\x,\X),
\end{eqnarray*}
where, for every $i\in\{1,\ldots,\ell\}$, 
\begin{eqnarray*}
	\gamma_i(\x,\X) &=& \left( \bigwedge_{j=1}^{n_i} X_{\lambda_i(j)}(\x_{i,j}) \right) \wedge \gamma^{-}_i(\X,\y_i)  \wedge \delta_i(\x).
\end{eqnarray*}
The function $\lambda_i$ is defined analogously to $\lambda$ of the first part of the proof. The tuple $\x_{i,j}$ has $a_{\lambda_i(j)}$ variables for $j\in\{1,\ldots,n_i\}$, $\y_i$ has $p_i$ variables and are such that $(\x_{i,1},\ldots,\x_{i,n_i},\y_i) = \x.$ The formulas $\gamma^{-}_i(\X,\y_i)$, $\delta_i(\x)$ are defined analogously to $\varphi^{-}(\X,\y)$ and $\beta(\x)\,$\footnote[2]{Note that $\delta_i(\x)$ can include subformulas of the form $(\u = \u)$.} respectively (see \eqref{f1}). Let $g$ be a function that counts the number of possible orders over partitions on a $d$-tuple of variables. Let the formulas $\theta^i(\x)$ for $i\in\{1,\ldots,g(d)\}$ represent each of these orders over $\x$. Note that for every formula $\eta(\x)$,
\begin{eqnarray*}
	\exists\x\:\eta(\x) &\equiv& \exists\x(\eta(\x)\wedge\theta^1(\x)) \vee \cdots \vee \exists\x(\eta(\x)\wedge\theta^{g(d)}(\x)).
\end{eqnarray*}
We define the following formulas $\xi_i(\x,\X)$, for $i\in\{1,\ldots,m\}$ where $m = \ell \cdot g(d)$, as follows: 
\begin{eqnarray*}
	\xi_1(\x,\X) &=& \gamma_1(\x,\X) \wedge \theta^1(\x), \\
	& \vdots & \\
	\xi_{g(d)}(\x,\X) &=& \gamma_1(\x,\X) \wedge \theta^{g(d)}(\x), \\
	\xi_{g(d)+1}(\x,\X) &=& \gamma_2(\x,\X) \wedge \theta^1(\x), \\
	& \vdots & \\
	\xi_{2\cdot g(d)}(\x,\X) &=& \gamma_2(\x,\X) \wedge \theta^{g(d)}(\x), \\
	& \vdots & \\
	\xi_{(\ell-1) g(d)+1}(\x,\X) &=& \gamma_{\ell}(\x,\X) \wedge \theta^1(\x), \\
	& \vdots & \\
	\xi_{\ell \cdot g(d)}(\x,\X) &=& \gamma_{\ell}(\x,\X) \wedge \theta^{g(d)}(\x).
\end{eqnarray*}
Having every disjunct with a total order allows us to eliminate the ones that are unsatisfiable for every $\L$-structure $\A$, that is, each $i\in\{1,\ldots,m\}$ such that for every assignment $\s$ to $\x,$ $f_{\xi_i(\s,\X)}(\A) = 0,$ for every $\A\in\Truc.$ Let $k$ be the number of disjuncts that are left after eliminating the unsatisfiable disjuncts. We use an injective\footnote[3]{We need this function to be injective because this way we can assure each one of the $k$ satisfiable disjuncts to be represented} function $\rho:\{1,\ldots,k\}\to\{1,\ldots,m\}$ such that $\varphi_i(\x,\X) = \xi_{\rho(i)}(\x,\X)$ is satisfiable, for every $i\in\{1,\ldots,k\}.$

We can conclude that
\begin{eqnarray*}
	\varphi(\x,\X) &\equiv& \varphi_1(\x,\X) \vee \varphi_2(\x,\X) \vee \dots  \vee \varphi_k(\x,\X),
\end{eqnarray*}
and for every $i\in\{1,\ldots,k\}$,
\begin{eqnarray*}
	\varphi_i(\x,\X) &=& \left( \bigwedge_{j=1}^{n_i} X_{\lambda_i(j)}(\x_{i,j}) \right) \wedge \varphi^{-}_i(\X,\y_i) \wedge \theta_i(\x) \wedge \beta_i(\x),
\end{eqnarray*}
where $n_i$ is the number of times a non-negated variable in $\X$ is referred to, according to the function $\lambda_i:\{1,\ldots,n_i\}\to\{1,\ldots,r\}$, $\y_i$ is a $p_i$-tuple of variables in $\x$, $\varphi_i^{-}(\X,\y_i)$ is a conjunction of negated predicates in $\X$, $\theta_i(\x)$ defines a total order on a partition of $\x$, and $\beta_i(\x)$ is an FO-formula over $\L$ which mentions all variables in $\x$. Let $\alpha_i(\x) = \theta_i(\x)\wedge\beta_i(\x).$

%% CLAIM CLAIM CLAIM CLAIM CLAIM CLAIM CLAIM CLAIM CLAIM CLAIM CLAIM CLAIM CLAIM CLAIM CLAIM CLAIM CLAIM CLAIM
%% CLAIM CLAIM CLAIM CLAIM CLAIM CLAIM CLAIM CLAIM CLAIM CLAIM CLAIM CLAIM CLAIM CLAIM CLAIM CLAIM CLAIM CLAIM

\begin{claim} \label{alpha}
	Let $\A\in\Truc$ and $i\in\{1,\ldots,k\}$. For every assignment $\s$ to $\x$ such that $\A\models\alpha_i(\s)$, it holds that $f_{\varphi_i(\s,\X)}(\A) > 0.$
\end{claim}
\begin{proof}
	Let $\s = (\s_1,\ldots,\s_{n_i},\t)$ such that $\A\models\alpha_i(\s)$, and let $\P = (\P_1,\ldots,\P_r)$ where $P_i = \bigcup_{j\in[1,n_i]:\:\lambda(j)=i}\{s_i\}$. We will show that $\A\models\varphi_i(\s,\P)$. By contradiction, suppose that $\A\not\models\exists\x\,\varphi_i(\x,\P)$. That is, there is no assignment $\e$ to $\x$ such that $\A\models\varphi(\e,\P).$ Following the same proof as in case (a) of Lemma \ref{first}, we conclude that $\varphi_i(\x,\P)$ is inconsistent. However, as we mentioned previously, all of such disjuncts have been eliminated, which leads to a contradiction.
\end{proof}

Our plan now is to exclude the lexicographically smallest assignment $\P$ such that $\A\models\exists\x\:\varphi_1(\x,\P)$, and if there is no such $\P$, exclude the lexicographically smallest $\Q$ such that $\A\models\exists\x\:\varphi_2(\x,\Q)$, and so on. As we already know how to exclude that assignment in the first disjunct, we will now deal with the next disjuncts. Similarly to the first part of the proof, for each $i\in\{1,\ldots,k\}$ let $m_i = \sum_{j = 1}^{n_i} a_{\lambda_i(j)}$ and let
\begin{multline*}
\alpha^i_{\min}(\u_1,\dots,\u_{n_i}) = \exists\y\Big[ \alpha_i(\u_1,\dots,\u_{n_i},\y)\wedge \\ \forall\v_1\cdots\forall\v_{n_i}\forall\w\Big(\big(\alpha_i(\v_1,\dots,\v_{n_i},\w)\wedge\bigvee_{j=1}^{n_i}(\u_j\neq\v_j)\big)\to \varphi_{m_i,<}((\u_1,\dots,\u_{n_i}),(\v_1,\dots,\v_{n_i}))\Big)\Big],
\end{multline*}
where $\u_j$ and $\v_j,$ have $a_{\lambda_i(j)}$ variables for $j\in\{1,\ldots,n_i\}$, and $\w$ has $p_i$ variables. Also, let
\begin{multline*}
\psi_i(\X) = \forall\x\:\neg\alpha_i(\x) \vee \exists\u_1\cdots\exists\u_{n_i}\bigg[\alpha^i_{\min}(\u_1,\dots,\u_{n_i}) \\ \wedge \bigg(\bigg(\bigvee_{j = 1}^{n_i}\neg X_{\lambda_i(j)}(\u_i) \bigg) \vee \bigvee_{j=1}^r \exists \v\Big( X_j(\v) \wedge \bigwedge_{\ell\in[1,n_i]: \lambda_i(\ell) = j} \v \neq \u_\ell\Big) \bigg) \bigg].
\end{multline*}
Note that $\psi_i(\X)$ excludes only the lexicographically smallest tuple of predicates $\P$ such that $\A\models\exists\x\:\varphi_i(\P,\x),$ if there is at least one. In other words, every assignment $\P^\prime \neq \P$ is such that $\A\models\psi_i(\P^\prime).$ Our new formula $\varphi_i^\prime(\x,\X)$ is defined as follows:
\begin{multline}
\varphi_i^\prime(\x,\X) = \varphi_i(\x,\X) \wedge \psi_1(\X) \wedge (\exists\v\:\alpha_1(\v)\vee\psi_2(\X)) \wedge \cdots \wedge \\ (\exists\v\:\alpha_1(\v)\vee\cdots\vee\exists\v\:\alpha_{i-1}(\v)\vee\psi_i(\X)).
\end{multline}
Let $\varphi^\prime(\x,\X) = \varphi_1^\prime(\x,\X)\vee\cdots\vee\varphi_k^\prime(\x,\X).$ We will now show that $f_{\exists\x\varphi^\prime(\x,\X)} = f_{\exists\x\varphi(\x,\X)}\dotminus 1.$ Assume that $\A\in\Truc$. Suppose first that there is at least one assignment $\R$ to $\X$ such that $\A\models\exists\x\:\varphi(\x,\R).$ Let $q$ be the least $i\in\{1,\ldots,k\}$ such that there exists at least one assignment $\R^\prime$ to $\X$ for which $\A\models\exists\x\:\varphi_i(\x,\R^\prime).$ Let $\d = (\d_1,\dots,\d_{n_q},\ep)$ be the lexicographically smallest assignment to $\x$ for which $\A\models\alpha_q(\d)$, where $\d_i$ is the corresponding assignment to $\x_i$, for every $i\in\{1,\dots,n_q\}$, and $\ep$ is an assignment for $\y$. Consider now the tuple $\P = (P_1,\dots,P_r)$ where $P_i = \bigcup_{j:\lambda_q(j)=i,j\in[1,n_q]}\{\d_j\}$. As we did in Lemma \ref{first} we will show that $\P$ is such that (a) $\A\models\exists\x\:\varphi(\x,\P),$ (b) $\A\not\models\exists\x\:\varphi^\prime(\x,\P),$ and (c) $\P$ is the only assignment to $\X$ that satisfies both (a) and (b)
\begin{enumerate}
	\item[(a)] As we showed in part (a) of Lemma \ref{first}, if there is at least one assignment $\R$ to $\X$ such that $\A\models\exists\x\:\varphi_q(\x,\R)$, then $\A\models\exists\x\:\varphi_q(\x,\P)$ for this particular $\P$. However, as we showed in Claim \ref{alpha}, if there is an assignment $\s$ to $\x$ such that $\A\models\alpha_q(\s)$, then there {\em is} such an assignment to $\X$. The assignment $\d$ to $\x$ satisfies that $\A\models\alpha_q(\d)$, so it holds that $\A\models\exists\x\:\varphi_q(\x,\P)$. It immediately follows that $\A\models\exists\x\:\varphi(\x,\P).$
	
	\item[(b)] We will show that $\A\not\models\exists\x\:\varphi^\prime_i(\x,\P)$ for (1) $i\in\{1,\ldots,q-1\}$, and (2) $i\in\{q,\ldots,k\}$. (1) By the choice of $q$, it holds that $\A\not\models\exists\x\:\varphi^\prime_i(\x,\P)$ for every $i\in\{1,\ldots,q-1\}$ since there is no possible assignment to $\X$ for any of their sub-formulas $\varphi_i(\x,\X)$. (2) We can use the proof in Lemma \ref{first} to see that $\A\not\models\psi_q(\P).$ For each $i\in\{q,\ldots,k\}$, the sub-formula 
	$$ \zeta_q(\X) =  (\exists\v\:\alpha_1(\v)\vee\cdots\vee\exists\v\:\alpha_{q-1}(\v)\vee\psi_q(\X)) $$ 
	appears as a conjunct in $\varphi^\prime_i(\x,\X)$. However, by the choice of $q$, there is no $i\in\{1,\ldots,q-1\}$ such that $\A\models\exists\v\:\varphi_i(\v,\P)$, and also $\A\not\models\psi_q(\P)$. It follows that $\A\not\models\zeta_q(\P)$,  so $\A\not\models\exists\x\:\varphi^\prime_i(\x,\P)$. And so, we conclude that $\A\not\models\exists\x\:\varphi^\prime(\x,\P).$
	
	\item[(c)] Suppose there is an assignment $\P^\prime \neq \P$ to $\X$ that satisfies both (a) and (b). As we deduce from the part (c) of Lemma \ref{first}, $\P$ is the only assignment to $\X$ such that $\A\not\models\:\psi_q(\P)$, so necessarily $\A\models\:\psi_q(\P^\prime)$. Since $\P^\prime$ assigned to $\X$ satisfies (a), then $\A\models\exists\x\:\varphi(\x,\P^\prime).$ By the choice of $q,$ every $i\in\{1,\ldots,q-1\}$ is such that $\A\models\forall\x\:\neg\alpha_i(\x)$, so $\A\models\psi_i(\P^\prime)$ for each $i$. But as we mentioned, also $\A\models\:\psi_q(\P^\prime),$ which means that $\A\models\exists\x\:\varphi^\prime_q(\x,\P^\prime),$ and so, $\A\models\exists\x\:\varphi^\prime(\x,\P^\prime),$ which leads to a contradiction.
\end{enumerate}

With this, we conclude that for every $\A\in\Truc$ such that $f_{\exists\x\,\varphi(\x,\X)}(\A)>0$, we have that $f_{\exists\x\,\varphi^\prime(\x,\X)}(\A) = f_{\exists\x\,\varphi(\x,\X)}(\A)-1.$

Second, assume that there is no assignment $\R$ to $\X$ such that $\A\models\exists\x\,\varphi_i(\x,\R)$ for any $i\in\{1,\ldots,k\}$. Let $\P$ be an arbitrary assignment to $\X$. Since $\A\not\models\exists\x\,\varphi_i(\x,\P)$, for any $i,$ we see that $\A\not\models\exists\x\,(\varphi(\x,\P)\wedge\chi(\x,\P))$ for any formula $\chi(\x,\P)$. It follows that there is no assignment $\R$ to $\X$ such that $\A\models\exists\x\,\varphi^\prime_i(\x,\R),$ for any $i\in\{1,\ldots,k\}$. And so, for every $\A\in\Truc$ such that $f_{\exists\x\,\varphi(\x,\X)}(\A)=0,$ we have that $f_{\exists\x\,\varphi^\prime(\x,\X)}(\A) = 0.$ Hence, from the results in the previous paragraph, if $f^\prime = f_{\exists\x\,\varphi^\prime(\x,\X)}$, we have that $f^\prime = f\dotminus 1\in\E1$ 

\item Let $f \in \E1$ be defined by an extended quantifier free $\L$-formula $\varphi(\x,\X,\z)$, where $\x = (x_1,\dots,x_d), \X = (X_1,\dots,X_r)$ and $\z = (z_1,\dots,z_p)$. That is,
\begin{eqnarray*}
	f(\A) &=& \mid \{ \langle\P,\e\rangle \mid \A \models \exists \x \ \varphi(\x,\P,\e) \} \mid,
\end{eqnarray*}
for every $\A = \langle A, \S^{\A}, \leq^{\A} \rangle \in \Truc$, where $\P = (P_1,\ldots,P_r)$, $P_i \subseteq A^{a_i}$ for every $i \in \{1,\ldots,r\}$ and $\e \in A^p$. In order to prove that $f\dotminus 1\in\E1$, we define the formulas $\varphi_i(\x,\X,\z)$ for $i\in\{1,\ldots,k\}$ in the same way as case 2, where
\begin{eqnarray*}
	\varphi(\x,\X,\z) \equiv \varphi_1(\x,\X,\z) \vee \cdots \vee \varphi_k(\x,\X,\z),
\end{eqnarray*}
and
\begin{eqnarray*}
	\varphi_i(\x,\X,\z) &=& \left( \bigwedge_{j=1}^{n_i} X_{\lambda_i(j)}(\x_{i,j},\z_{i,j}) \right) \wedge \varphi^{-}_i(\X,\y_i,\w_i) \wedge \theta_i(\x,\z) \wedge \beta_i(\x,\z),
\end{eqnarray*}
where $n_i$ is the number of times a non-negated variable in $\X$ is referred to, according to the function $\lambda_i:\{1,\ldots,n_i\}\to\{1,\ldots,r\}$, the tuple $\x_{i,j}$ has $b_{\lambda_i(j)}$ variables and the tuple $\z_{i,j}$ has $c_{\lambda_i(j)}$ for $j\in\{1,\ldots,n_i\}$ (note that $b_{\ell} + c_{\ell} = a_{\ell}$ for $\ell\in\{1,\ldots,r\}$), $\y_i$ has $p_i$ variables, and $\w_i$ has $q_i$ variables. Furthermore, we have that $(\x_{i,1},\ldots,\x_{i,n_i},\y) = \x$ and $(\z_{i,1},\ldots,\z_{i,n_i},\w_i) = \z.$ The formula $\theta(\x,\z)$ defines a total order, analogously to case 2. The formulas $\varphi^{-}_i(\X,\y_i,\w_i)$, $\beta_i(\x,\z)$, are also defined analogously.

In this case, we mix both the strategies in cases 1 and 2. That is, we are going to {\em isolate} the lexicographically smallest tuple of predicates that satisfies the first satisfiable disjunct, and then {\em exclude} the lexicographically smallest tuple that satisfies the isolated disjunct.

Let $m_i = \sum_{j = 1}^{n_i} a_{\lambda_i(j)}$ and
\begin{multline*}
\alpha^i_{\min}(\x_1,\dots,\x_{n_i},\z_1,\dots,\z_{n_i}) = \exists\y\exists\w\Big[ \alpha_i(\x_1,\dots,\x_{n_i},\y,\z_1,\dots,\z_{n_i},\w)\wedge \\ \forall\u_1\cdots\forall\u_{n_i}\forall\s\forall\v_1\cdots\forall\v_{n_i}\forall\t\Big(\big(\alpha_i(\u_1,\dots,\u_{n_i},\s,\v_1,\dots,\v_{n_i},\t)\wedge\bigvee_{j=1}^{n_i}(\x_j\neq\u_j \vee \z_j\neq\v_j)\big)\to \\ \varphi_{m_i,<}((\x_1,\dots,\x_{n_i},\z_1,\dots,\z_{n_i}),(\u_1,\dots,\u_{n_i},\v_1,\dots,\v_{n_i}))\Big)\Big],
\end{multline*}
and let
\begin{multline*}
\psi_i(\X,\z) = \forall\x\forall\v\:\neg\alpha_i(\x,\v) \vee \exists\u_1\cdots\exists\u_{n_i}\exists\w_1\cdots\exists\w_{n_i}\bigg[\alpha^i_{\min}(\u_1,\dots,\u_{n_i},\w_1,\dots,\w_{n_i}) \\ \wedge \bigg(\bigg(\bigvee_{j = 1}^{n_i}\neg X_{\lambda_i(j)}(\u_i,\w_i) \bigg) \vee \bigvee_{j=1}^r \exists \s\exists \t\Big( X_j(\s,\t) \wedge \bigwedge_{\ell\in[1,n_i]: \lambda_i(\ell) = j} \s \neq \u_\ell \vee \t \neq \w_\ell\Big) \bigg) \bigg] \vee \\ \exists\w(\exists\u\:\varphi(\u,\X,\w) \wedge \varphi_{p,<}(\w,\z)).
\end{multline*}
Let $\P$ and $\e$ be assignments to $\X$ and $\z$. Note that $\psi_i(\X,\z)$ excludes $\P$ and $\e$ only if $\P$ is the {\em lexicographically smallest} tuple of predicates (Same as case 2) such that $\A\models\exists\x\:\varphi_i(\x,\P,\d)$, for some assignment $\d$ to $\z$, and $\e$ is the lexicographically smallest assignment to $\z$ such that $\A\models\exists\x\:\varphi_i(\x,\P,\e).$ We define $\varphi_i^\prime(\x,\X,\z)$ in the same way as case 2:
\begin{multline*}
\varphi_i^\prime(\x,\X,\z) = \varphi_i(\x,\X,\z) \wedge \psi_1(\X,\z) \wedge (\exists\u\exists\v\:\alpha_1(\u,\v)\vee\psi_2(\X,\z)) \wedge \cdots \wedge \\ (\exists\u\exists\v\:\alpha_1(\u,\v)\vee\cdots\vee\exists\u\exists\v\:\alpha_{i-1}(\u,\v)\vee\psi_i(\X,\z)).
\end{multline*}
Finally, let $\varphi^\prime(\x,\X,\z) = \bigvee_{i = 1}^k \varphi_i^\prime(\x,\X,\z)$.

Let $q$ be the least $i\in\{1,\ldots,k\}$ such that $\A\models\exists\x\:\varphi_i(\x,\R,\d)$ for some assignment $R$ to $\X$ and some assignment $\d$ to $\z$. Let $\P$ be the lexicographically smallest tuple of predicates such that $\A\models\exists\x\:\varphi_q(\x,\P,\d^\prime)$ for some assignment $\d^\prime$ to $\z$. Let $\e$ be the lexicographically smallest assignment to $\z$ such that $\A\models\exists\x\:\varphi_q(\x,\P,\e).$ This formula is such that (a) $\A\models\exists\x\:\varphi(\x,\P,\e)$, (b) $\A\not\models\exists\x\:\varphi^\prime(\x,\P,\e)$ and $\P$ and $\e$ are the only assignments that satisfy (a) and (b). The proof of this is analogous to case 2. Therefore, we conclude that $f_{\exists\x\,\varphi^\prime(\x,\X,\z)} = f_{\exists\x\,\varphi(\x,\X,\z)}\dotminus 1.$
\end{enumerate}
\end{proof}

\bibliographystyle{plain}
\bibliography{SalujaST95}

\end{document}