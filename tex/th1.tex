%!TEX root = sharp-sigma-one-plus.tex
\begin{theo} \label{decisionptime}
The decision problem associated to a function in $\E1$ is in \textsc{P}.
\end{theo}
\begin{proof}
Let $f$ be a function in $\E1$. Then there is an extended quantifier-free $\L-$formula $\varphi(\x,\X,\z)$ such that
\begin{eqnarray*}
f(\A) &=& \mid \{ \langle\P,\e\rangle \mid \A \models \exists \x \, \varphi(\x,\P,\e) \} \mid,
\end{eqnarray*}
where $\A = \langle A, \S^{\A}, \leq^{\A} \rangle \in \Truc$, $A = \{a_1,\ldots,a_{\vert A \vert}\}$, $\z$ is an $m$-tuple of variables and $\x$ is a $k$-tuple of variables. As said in \cite{DBLP:journals/jcss/SalujaST95}, a function $h:\Truc \to \N$ is in $\#\Sigma_0$ if there exists a quantifier-free $\L$-formula $\theta(\Y,\y)$ such that $h = f_{\theta(\Y,\y)}$. Let $\y = (\x,\z)$ and let $\psi(\X,\y) = \varphi(\x,\X,\z)$. Moreover, let $g = f_{\psi(\X,\y)}$.
\begin{claim}
For each $\A \in \Truc$, $f(\A) > 0$ iff $g(\A) > 0$.
\end{claim}
\begin{proof}
($\Rightarrow$) Suppose $f(\A) > 0$. Let $\P$ and $\e$ be such that $\A \models \exists \x \, \varphi(\x,\P,\e)$. It follows that there is at least one $\d \in A^k$ such that $\A \models \varphi(\d,\P,\e) = \psi(\P,(\d,\e))$. Therefore, $g(\A) > 0$. ($\Leftarrow$) Suppose $g(\A) > 0$. Let $\Q$ and $\c = (\c_1,\c_2)$, where $\c_1$ and $\c_2$ have $k$ and $m$ elements respectively, be such that $\A \models \psi(\Q,\c) = \varphi(\c_1,\Q,\c_2)$. Therefore, $\A \models \exists \x \, \varphi(\x,\Q,\c_2)$, from which we conclude that $f(\A) > 0$.
\end{proof}
Note that $\psi(\X,\y)$ is an {\sc FO}-extended $\L$-formula, so $g \in\Eo$. As we showed, computing $g$ is enough to solve the decision version of $f$, but as it is shown in Theorem \ref{fp1}, $g(\A)$ can be evaluated in polynomial time, from which we conclude that the decision version of $f$ is in {\sc P}.
\end{proof}