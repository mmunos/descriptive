\section{The Extended Horn Counting Hierarchy}

We define syntactically the classes $\EiE$ and $\EiU$ as follows. First we define extended Horn clauses.
\begin{eqnarray*}
PL &::=& X_i(\x),\, i\in\N,\\
NL &::=& \neg X_i(\x),\, i\in\N \ \mid \ \exists x\, NL,\\
NC &::=& NL \ \mid \ \alpha, \alpha \mbox{ is an {\sc FO}-formula over } \L \ \mid \ (NC \vee NC),\\
HC &::=& NC \ \mid \ (NC \vee PL),
\end{eqnarray*}
where $\x$ is a tuple with the corresponding number of variables. Now we define the syntax of the classes inductively.
\begin{enumerate}
	\item $\EzeroE$:
	\begin{eqnarray*}
	E_0 &::=& HC \ \mid \ E_0 \wedge E_0.
	\end{eqnarray*}
	\item $\EzeroU$:
	\begin{eqnarray*}
	U_0 &::=& E_0.
	\end{eqnarray*}
	\item $\EipE$:
	\begin{eqnarray*}
	E_{i+1} &::=& U_i \ \mid \ \exists x \, E_{i+1}.
	\end{eqnarray*}
	\item $\EipU$:
	\begin{eqnarray*}
		U_{i+1} &::=& E_i \ \mid \ \forall x \, U_{i+1}.
	\end{eqnarray*}
\end{enumerate}
A function $f$ is in $\EiE$ (resp. $\EiU$) if there is an $\L$-formula $\varphi$ defined by the grammar $E_i$ (resp. $U_i$) such that $f = f_{\varphi}$.

The class $\PE$ is defined in [?] as $\PE = \{f\mid f\in\#P \mbox{ and its decision version } L_{f}\in P\}$.

\begin{theo}
	$\EtwoE \subseteq \PE$
\end{theo} 
\begin{proof}
	Let $f = f_{\varphi(\X,\z)}$ such that $\varphi(\X,\z)$ is defined by the grammar $E_2$. First we notice that, as stated in \cite{DBLP:journals/jcss/SalujaST95}, for every $\L$-formula $\varphi$, $f_{\varphi}\in\#P$.\\
	
	We will now prove that there is a polynomial time algorithm that decides $L_{f}$. This is equivalent to decide, given $\A= \langle A, \S^\A, \leq^\A \rangle\in\Truc$, whether there exist assignments $\P,\z$ such that $\A\models\varphi(\P,\z)$. This is the same as $\A\models\exists\X\exists\x\,\varphi(\X,\x)$. Since $\varphi(\X,\x)$ is defined by the grammar $E_2$, there exists $\psi(\X,\x,\y,\u)$ such that $\varphi(\X,\x) = \exists\x\exists\y\forall\u\exists\v\,\psi(\X,\x,\y,\u,\v)$, where every variable in $\v$ appears on a negated second-order literal $\neg X_i$, with $i\in\N$.
	
	Given $\A\in\Truc$ we generate an equivalent $\L$-formula $\theta(\X,\x,\y)$ with a series of operations. First, we replace every instance of $\v$ by a disjunction of all $r$-tuples in $A^r$, where $r$ is the number of variables in $\v$. The result is still a Horn clause. Second, we replace every instance of $\u$ by a conjunction of all $s$-tulples in $A^s$, where $s$ is the number or variables in $\u$. The result is still a conjunction of Horn clauses.
	
	Now we notice that $\zeta = \exists\X\exists\x\exists\y\,\theta(\X,\x,\y)$ is an existential second-order FO-formula, for which $\A\models\zeta$ can be decided in polynomial time.
\end{proof}

We define the decision problem
\[
\textsc{Disj-Horn-Sat} = \{\Phi \mid \Phi \mbox{ is a disjunction of Horn formulas and $\Phi$ is satisfiable}\},
\]
and the counting problem {\sc \#Disj-Horn-Sat} as a function that counts all satisfying assignments to a formula $\Phi$ which is a disjunction of Horn formulas.

\begin{theo}
	$\textsc{Disj-Horn-Sat}$ is hard for $\EtwoE$ under parsimonious reductions. 
\end{theo} 
\begin{proof}
	Let $f$ be an arbitrary function in $\EtwoE$ and let $f_{\textsc{Disj-Horn-Sat}}$ be the function that characterizes {\sc \#Disj-Horn-Sat}. We will now show a function $h:\Truc\to L(P)$ such that for every $\A\in\Truc$, it holds that $f(\A) = f_{\textsc{Disj-Horn-Sat}}(h(\A))$, and which can be computed in polynomial time. Let $\psi$ be such that $f = f_{\psi}$. We separate the proof in three cases:\begin{enumerate}
		\item All free variables in $\psi$ are of first order. Let $\psi = \exists \x \forall \y\,\varphi(\x,\y,\z)$ where $\z = (z_1,\ldots,z_d)$. Given a structure $\A = \langle A, \S^{\A}, \leq^{\A} \rangle\in\Truc$, we generate a propositional formula $\Phi = h(\A)$. First we notice that this case is trivial since we can calculate this function in polynomial time, by checking for each $\e\in A^d$ if $\A\models\exists\x\forall \y\,\varphi(\x,\y,\e)$. Let $n$ be the number of such assignments to $\z$. Let $\p$ be an $n$-tuple of propositional variables where $\p = (p_1,\ldots,p_n)$. The Disj-Horn-Sat formula we return is
		\[
		\Phi = \bigvee_{i = 1}^n \neg p_1 \wedge \cdots \wedge \neg p_{i-1} \wedge p_i \wedge \neg p_{i+1} \wedge \cdots \wedge \neg p_n
		\]
		which has exactly $n$ possible assignments. Note that each $p_i$ and $\neg p_i$ is a Horn clause itself.
		
		\item There are only second-order free variables in $\psi$. Let $\psi = \exists \x\forall \y\,\varphi(\x,\y,\X)$, where $\x = (x_1,\ldots,x_c)$, $\y = (y_1,\ldots,y_b)$, $\X = (X_1,\ldots,X_r)$ and each $X_{\ell}$ has arity $c_{\ell}$. Given a structure $\A = \langle A, \S^{\A}, \leq^{\A} \rangle\in\Truc$, we generate a Disj-Horn formula $\Phi = h(\A)$ as follows:
		
		First, Let $\varphi(\x,\y,\X) = \bigwedge_{i = 1}^k \varphi_{i}(\x,\y,\X)$ where, without loss of generality, each $\varphi_{i}(\x,\y,\X)$ is of the form
		\[
		\varphi_{i}(\x,\y,\X) = X_s(\u_{s}) \vee \neg X_{t_1}(\u_{t_1}) \vee \cdots \vee \neg X_{t_n}(\u_{t_n}) \vee \alpha_i(\x,\y),
		\]
		where $\u_{s}$ and every $\u_{t_j}$ have the corresponding number of variables in $(\x,\y)$, and $\alpha(\x,\y)$ is a FO-formula. For the rest of the proof, $X_s(\u_s)$ can be absent and it does not affect the proof. Note that this formula is equivalent to
		\[
		\neg\alpha_i(\x,\y) \to (X_s(\u_{s}) \vee \neg X_{t_1}(\u_{t_1}) \vee \cdots \vee \neg X_{t_n}(\u_{t_n})),
		\]
		
		Then, for each $\a\in A^c$ let $\Gamma_i^{\a} = \{\b\in A^{b}\mid \A\models\neg\alpha(\a,\b)\}$. Note that this set can be generated in polynomial time. For each $\ell\in\{1,\ldots,r\}$, we generate a propositional variable $p^{\ell}_{\d}$ for each $\d\in A^{c_{\ell}}$. Let $\p$ be the tuple of all such variables. Also, let $\rho_{\ell}(\x,\y) = \u_{\ell}$, and $\rho_{\ell}(\d)$ be the tuple that results of reordering $\d$ using the reordering of the variables of $(\x,\y)$ in $\u_{\ell}$.
		
		For each $\a\in A^c$ we generate a Horn formula
		\[
		\Phi_{\a}^i = \bigwedge_{\b\in\Gamma_i} p^{s}_{\rho_s(\a,\b)} \vee \neg p^{t_1}_{\rho_{t_1}(\a,\b)} \vee \cdots \vee \neg p^{t_n}_{\rho_{t_n}(\a,\b)}.
		\]
		and finally we return $\Phi = \bigvee_{\a\in A^c} \bigwedge_{i = 1}^k \Phi_{\a}^i$.
		
		Note that for each assignment $\P$ to $\X$ such that $\A\models\exists\x\forall\y\,\varphi(\x,\y,\P)$ there is a corresponding assignment $\sigma$ to $\p$ where $\sigma(p^{\ell}_{\a}) = 1$ if and only if $\a\in P_{\ell}$. We can conclude that $f(\A) = f_{\textsc{Disj-Horn-Sat}}(\Phi)$.
		
		\item Let $\psi = \exists \x\forall\y\,\varphi(\x,\y,\X,\z)$, where $\x = (x_1,\ldots,x_c)$, $\y = (y_1,\ldots,y_b)$, $\z = (z_1,\ldots,z_d)$, $\X = (X_1,\ldots,X_r)$ and each $X_{\ell}$ has arity $c_{\ell}$. Given a structure $\A = \langle A, \S^{\A}, \leq^{\A} \rangle\in\Truc$ we generate a Disj-Horn formula $\Phi = h(\A)$ as follows:
		
		As in the previous case, let $\varphi(\x,\y,\X,\z) = \bigwedge_{i = 1}^k$ where $\varphi(\x,\y,\X,\z)$ is of the form:
		\[
		\varphi_{i}(\x,\y,\X,\z) = X_s(\u_{s}) \vee \neg X_{t_1}(\u_{t_1}) \vee \cdots \vee \neg X_{t_n}(\u_{t_n}) \vee \alpha_i(\x,\y,\z),
		\]
		where $\u_{s}$ and every $\u_{t_j}$ have the corresponding number of variables in $(\x,\y,\z)$, and $\alpha(\x,\y,\z)$ is a FO-formula. For the rest of the proof, $X_s(\u_s)$ can be absent and it does not affect the proof. Note that this formula is equivalent to
		\[
		\neg\alpha_i(\x,\y,\z) \to (X_s(\u_{s}) \vee \neg X_{t_1}(\u_{t_1}) \vee \cdots \vee \neg X_{t_n}(\u_{t_n})).
		\]
		
		For each $\e\in A^d$, we do the following. For each $\a\in A^c$ let $\Gamma^{\e}_{i,\a} = \{\b \in A^b \mid \A\models \neg\alpha_i(\a,\b,\e) \}$. Note that this set can be generated in polynomial time. For each $\ell\in\{1,\ldots,r\}$ we generate a propositional variable $p^{\e}_{\ell,\d}$ for each $\d\in A^{c_{\ell}}$. Let $\p$ be the tuple of all such variables. We also generate a set $T = \{t_1,\ldots,t_{L}\}$ where $L = \lceil \log\vert A^d\vert \rceil$, and an arbitrary injective function $\beta:A^d \to \{0,1\}^L$ where $\beta(\e) = (b^{\e}_1,\ldots,b^{\e}_L)$ for each $\e\in A^d$. Also, let $\rho_{\ell}(\x,\y,\z) = \u_{\ell}$, and $\rho_{\ell}(\d)$ be the tuple that results of reordering $\d$ using the order of the variables in $(\x,\y,\z)$ in $\u_{\ell}$. And so, the Horn formula we generate is
		\[
			\Phi^{\e}_{i,\a} = \bigwedge_{\a\in\Gamma^{\e}_i} p^{\e}_{s,\rho_{s}(\a,\b,\e)} \vee p^{\e}_{t_1,\rho_{t_1}(\a,\b,\e)} \vee \cdots \vee p^{\e}_{t_n,\rho_{t_n}(\a,\b,\e)}.
		\]
		And finally we return $\Phi = \bigvee_{\e\in A^d}\bigvee_{\a\in A^c} (\bigwedge_{i=1}^k \Phi_{\e}^{i,\a} \wedge \gamma_{\e} \wedge \theta_{\e} )$, where $\gamma_{\e}$ is defined as
		\[
		\gamma_{\e} = \bigwedge_{\e'\in A^d\setminus\{\e\}} \bigwedge_{\ell = 1}^r \bigwedge_{\b\,\in A^{c_{\ell}}} p_{\ell,\b}^{\e'},
		\]
		and $\theta_{\e}$ is defined as
		\[
		\theta_{\e} = \bigwedge_{i\in[1,L]:\,b^{\e}_i = 0} \neg t_i\  \wedge \bigwedge_{i\in[1,L]:\,b^{\e}_i = 1} t_i.
		\]
		
	\end{enumerate}
\end{proof}