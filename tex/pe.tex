\section{Appendix - Problems in $\PE$}

\subsection{\#Clique}
Counting the number of cliques in a graph is in $\HU{1}$. Let $\L = \{E,\leq\}$, and $\A = \langle A, E^{\A}, \leq^{\A} \rangle$ be an $\L$-structure that represents a graph $G = (A,E^{\A})$.

Let $C$ be a unary second-order variable, and
\[
\varphi(C) = \forall x \forall y((C(x)\wedge C(y))\to E(x,y)).
\]
Therefore, $f_{\varphi(C)}(\A)$ counts the number of cliques in $G$.

\subsection{Counting the number of matchings in a graph}
This function (also known as the Hosoya index of the graph) is in $\HU{1}$. Let $\L = \{E,\leq\}$, and $\A = \langle A, E^{\A}, \leq^{\A} \rangle$ be an $\L$-structure that represents a graph $G = (A,E^{\A})$.

Let $M$ be a unary second-order variable, and
\begin{align*}
\varphi(M) &= \forall x \forall y (M(x,y)\to E(x,y))\,\wedge \\
& \forall x \forall y (M(x,y)\to x < y)\,\wedge \\
& \forall x \forall y \forall z(M(x,y)\wedge M(z,y) \to x = z)\,\wedge \\
& \forall x \forall y \forall z(M(x,y)\wedge M(x,z) \to y = z)\,\wedge \\
& \forall x \forall y \forall z(\neg M(x,y)\vee \neg M(y,z)).
\end{align*}
Therefore, $f_{\varphi(M)}(\A)$ counts the number of matchings in $G$.
\subsection{\#2SAT}

\subsection{Counting the number of paths between two given nodes in a graph}

\subsection{Counting the number of Eulerian paths in a graph}

\subsection{Counting the number of perfect matchings in a bipartite graph}

\subsection{\shsat + $2^n$}

This function takes a propositional formula $\varphi$ as an input and outputs the number of satisfying assignments of $\varphi$ plus $2^n$, where $n$ is the number of variables in $\varphi$. Since the decision version of this function is trivial, it is in $\shp$. We also show using the machine described in Theorem \ref{diff} that this function is in $\totp$.