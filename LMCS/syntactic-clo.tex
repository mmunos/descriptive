%!TEX root = main.tex

We use the \emph{$\eqso(\fo)$-hierarchy} to define syntactic classes of functions with good algorithmic and closure properties.  
But before doing this, we introduce a more strict notion of counting problem with easy decision version.
Recall that a function $f : \Sigma^* \to \mathbb{N}$ has an easy decision counterpart if $L_f = \{ x \in \Sigma^* \mid f(x) > 0 \}$ is a language in $\ptime$. As the goal of this section is to define a syntactic class of functions in $\shp$ with easy decision versions and good closure properties, we do not directly consider the semantic condition $L_f  \in \ptime$, but instead we consider a more restricted syntactic condition. More precisely, a function $f : \Sigma^* \to \mathbb{N}$ is said to be in the complexity class $\totp$~\cite{PagourtzisZ06} if there exists a  polynomial-time NTM $M$ such that $f(x) = \tmt_M(x) - 1$ for every $x \in \Sigma^*$, where $\tmt_M(x)$ is the total number of runs of $M$ with input $x$. Notice that one is subtracted from $\tmt_M(x)$ to allow for $f(x) = 0$. Besides, notice that $\totp \subseteq \shp$ and that $f \in \totp$ implies that $L_f \in \ptime$. 

The complexity class $\totp$ contains many important counting problems with easy decision counterparts, such as $\cpm$, $\cdnf$, and $\chsat$ among others~\cite{PagourtzisZ06}. Besides, $\totp$ has good closure properties as it is closed under sum, multiplication and subtraction by one. However, some functions in $\totp$ do not admit FPRAS under standard complexity-theoretical assumptions\footnote{As an example consider the problem of counting the number of independent sets in a graph, and the widely believed assumption that $\np$ is not equal to the randomized complexity class $\rp$ (Randomized Polynomial-Time \cite{G77}). This counting problem is in $\totp$, and it is known that $\np = \rp$ if there exists an FPRAS for it \cite{DFJ02}.}, and no natural complete problems are known for this class \cite{PagourtzisZ06}. Hence, we use the $\eqso(\fo)$-hierarchy to find restrictions of $\totp$ with good approximation and closure properties.

It was proved in \cite{SalujaST95} that every function in $\E{1}$ admits an FPRAS. Besides, it can be proved that $\E{1} \subseteq \totp$. 
However, this class is not closed under sum, and then it is not robust under basic closure properties. 
\begin{prop}\label{prop-e1-nc}
There exist functions $f, g \in \E{1}$ such that $(f + g) \not\in \E{1}$.
\end{prop}
\proof
Towards a contradiction, assume that $\E{1}$ is closed under binary sum. 
Consider the formula $\alpha = \sa{x}(x = x) \in \E{1}$ over some signature $\R$. 
This defines the function $\sem{\alpha}(\A) = \size{\A}$. 
From our assumption, there exists some formula in $\E{1}$ equivalent to the formula $\alpha \add \alpha$, which describes the function $2\size{\A}$. 
Let $\sa{\bar{X}}\sa{\bar{x}}\exists\bar{y}\,\varphi(\bar{X},\bar{x},\bar{y})$ be this formula, where $\varphi$ is first-order and quantifier-free. 
For each $\R$-structure $\A$, we have the following inequality:
$$
\sem{\sa{\bar{X}}\sa{\bar{x}}\sa{\bar{y}}\,\varphi(\bar{X},\bar{x},\bar{y})}(\A)
\leq 
\sem{\sa{\bar{X}}\sa{\bar{x}}\exists\bar{y}\,\varphi(\bar{X},\bar{x},\bar{y})}(\A) \cdot  \size{\A}^{\length{\bar{y}}} \leq 2\size{\A}^{\length{\bar{y}}+1} 
$$
Note that the formula $\sa{\bar{X}}\sa{\bar{x}}\sa{\bar{y}}\,\varphi(\bar{X},\bar{x},\bar{y})$ defines a function in $\E{0}$. 
Therefore, as it was proven in \cite{SalujaST95}, if $\length{\bar{X}} > 0$ then the function is in $\Omega(2^{\size{\A}})$, which violates the inequality for large structures.

We now have that $\length{\bar{X}} = 0$.
Consider a structure $\mathfrak{1}$ with only one element. 
We have that $\sem{\sa{\bar{x}}\exists\bar{y}\,\varphi(\bar{x},\bar{y})}(\mathfrak{1}) = 2$, but since the structure has only one element, there is only one possible assignment to $\bar{x}$. 
And so, $\sem{\sa{\bar{x}}\exists\bar{y}\,\varphi(\bar{x},\bar{y})}(\mathfrak{1}) \leq 1$, which leads to a contradiction.

\qed
To overcome this limitation, one can consider the class $\QE{1}$, which is closed under sum by definition. In fact, the following proposition shows that the same good properties as for $\E{1}$ hold for $\QE{1}$, together with the fact that it is closed under sum and multiplication.
\begin{prop} \label{prop:qe0-fp-qe1-totp-fptras}
$\QE{1} \subseteq \totp$ and every function in $\QE{1}$ has an FPRAS. Moreover, $\QE{1}$ is closed under sum and multiplication.
\end{prop}
\proof
% !TeX root = main.tex

The authors in \cite{SalujaST95} proved that there exists a {\em product reduction} from every function in $\E{1}$ to a restricted version of $\cdnf$. 
That is, if $\alpha\in\E{1}$ over some signature $\R$, there exist polynomially computable functions $g\colon\ostr[\R]\to\ostr[\R_{\text{DNF}}]$ and $h\colon\nat\to\nat$ such that for every $\R$-structure $\A$, it holds that $\sem{\alpha}(\A) = \cdnf(\enc(g(\A)))\cdot h(\size{\A})$. We use this fact in the following arguments. 

To show that $\eqso(\loge{1})$ is contained  in \textsc{TotP}, let $\alpha$ be a $\eqso(\loge{1})$ formula and assume that it is in $\loge{1}$-SNF. 
That is, $\alpha = \sum_{i = 1}^n\alpha_i$ where each $\alpha_i$ is in $\loge{1}$-PNF. 
Consider the following nondeterministic procedure that on input $\enc(\A)$ generates $\sem{\alpha}(\A)$ branches. 
For each $\alpha_i = \varphi$, where $\varphi$ is a $\loge{1}$ formula, it checks if $\A\models\varphi$ in polynomial time and generates a new branch if that is the case. 
For each $\alpha_i = \sa{\bar{X}}\sa{\bar{x}}\varphi$, this formula is also in $\E{1}$. 
We use the reduction to $\cdnf$ provided in \cite{SalujaST95} and we obtain $g(\enc(\A))$, which is an instance to $\cdnf$. 
Since $\cdnf$ is also in $\totp$ \cite{PagourtzisZ06}, we simulate the corresponding nondeterministic procedure that generates exactly $\cdnf(\enc(g(\A)))$ branches. 
Since, $\fp\subseteq\totp$\cite{PagourtzisZ06}, each polynomially computable function is also in $\totp$, and then on each of these branches we simulate the corresponding nondeterministic procedure to generate $h(\size{\A})$ more. 
The number of branches for each $\alpha_i$ is $\sem{\alpha_i}(\A) = \cdnf(\enc(g(\A)))\cdot h(\size{\A})$, and the total number of branches is equal to $\sem{\alpha}(\A)$. 
We conclude that $\alpha\in\totp$.

To show that every function in $\eqso(\loge{1})$ has an FPRAS,  let $\alpha$ be a $\eqso(\loge{1})$ formula and assume that it is in $\loge{1}$-SNF. 
That is, $\alpha = \sum_{i = 1}^n\alpha_i$ where each $\alpha_i$ is in $\loge{1}$-PNF. 
Note that each $\alpha_i$ that is equal to some $\loge{1}$ formula $\varphi$ has an FPRAS given by the procedure that simply checks if $\A\models\varphi$ and returns 1 if it does and 0 otherwise. 
Also, each remaining $\alpha_i$ has an FPRAS since $\alpha_i\in \E{1}$ \cite{SalujaST95}. 
If two functions have an FPRAS, then their sum also has one given by the procedure that simulates them both and sums the results. 
We conclude that $\alpha$ has an FPRAS.

Finally, we show that $\eqso(\loge{1})$ is closed under sum and multiplication. 
Since $\eqso(\loge{1})$ is closed under sum by definition, we focus only on proving that it is closed under multiplication. 
We prove this for the more general case of $\eqso(\LL)$ with $\LL$ being a fragment of $\so$.

\begin{lem} \label{conj-mult}
	If $\LL$ is a fragment closed under conjunction, then $\eqso(\LL)$ is closed under binary multiplication.
\end{lem}
\proof
	Given two formulae $\alpha, \beta$ in $\eqso(\LL)$ we will construct a formula in the logic which is equivalent to $(\alpha\mult \beta)$. 
	From what was proven in Proposition \ref{theo-pnf-snf}, we may assume that $\alpha$ and $\beta$ are in $\LL$-SNF. 
	Let $\alpha = \sum_{i = 1}^n\sa{\bar{X}_i}\sa{\bar{x}_i}\varphi_i(\bar{X}_i,\bar{x}_i)$, and $\beta = \sum_{i = 1}^m\sa{\bar{Y}_i}\sa{\bar{y}_i}\psi_i(\bar{Y}_i,\bar{y}_i)$. Expanding the product in $(\alpha\mult \beta)$ and reorganizing results in the equivalent formula
	$$
	\sum_{i = 1}^n\sum_{j = 1}^m\sa{\bar{X}_i}\sa{\bar{Y}_j}\sa{\bar{x}_i}\sa{\bar{y}_j}(\varphi_i(\bar{X}_i,\bar{x}_i)\wedge\psi_i(\bar{Y}_j,\bar{y}_j)),
	$$
	which is in $\LL$-SNF, and therefore, in $\eqso(\LL)$.
\qed
Since $\loge{1}$ is closed under conjunction, then Lemma~\ref{conj-mult} can be applied to $\eqso(\loge{1})$. 
This concludes the proof.

\qed
Hence, it only remains to prove that $\QE{1}$ is closed under subtraction by one. Unfortunately, it is not clear whether this property holds; in fact, we conjecture that it is not the case. Thus, we need to find an extension of $\QE{1}$ that keeps all the previous properties and is closed under subtraction by one. It is important to notice that $\shp$ is believed not to be closed under subtraction by one by some complexity-theoretical assumption\footnote{A decision problem $L$ is in the randomized complexity class $\cspp$ if there exists a polynomial-time NTM $M$ such that for every $x \in L$ it holds that $\tma_M(x) - \tmr_M(x) = 2$, and for every $x \not\in L$ it holds that $\tma_M(x) = \tmr_M(x)$ \cite{OH93,FFK94}. It is believed that $\np \not\subseteq \cspp$.
However, if $\shp$ is closed under subtraction by one, then it holds that $\np \subseteq \cspp$ \cite{OH93}.}. So, the following proposition rules out any logic that extends $\Pi_1$ for a possible extension of~$\QE{1}$ with the desired closure property.
\begin{prop} \label{pi-minusone}
If $\Pi_1 \subseteq \LL \subseteq \fo$ and $\eqso(\LL)$ is closed under subtraction by one, then $\shp$ is closed under subtraction by~one. 
\end{prop}
\proof
Let $\LL$ be a fragment of $\fo$ that contains $\logu{1}$. Then we have that every function in $\U{1}$ is expressible in $\eqso(\LL)$. In particular, $\ctcnf \in \eqso(\LL)$. Suppose that $\eqso(\LL)$ is closed under subtraction by one. Then, the function $\ctcnf-1$, which counts the number of satisfying assignments of a 3-CNF formula minus one, is also in $\eqso(\LL)$. Note that $\eqso(\LL) \subseteq \eqso(\fo) = \shp$. We have that $\ctcnf$ is $\shp$-complete under parsimonious reductions\footnote{It can be easily verified that the standard reduction from SAT to 3-CNF (or 3-SAT) preserves the number of satisfying assignments}. Now, let $f$ be a function in $\shp$, and consider the nondeterministic polynomial-time procedure that on input $\enc(\A)$ computes the corresponding reduction to $\ctcnf$, name it $g(\enc(\A))$, and simulates the $\shp$ procedure for $\ctcnf-1$ on input $g(\enc(\A))$. We have that this is a $\shp$ procedure that computes $f-1$, from which we conclude that $\shp$ is closed under subtraction by one.
\qed
Therefore, the desired extension has to be achieved by allowing some local extensions to~$\Sigma_1$. More precisely, we define $\logex{1}$ as $\Sigma_1$ but allowing atomic formulae over a signature~$\R$ to be of the form either $u = v$ or $X(\bar u)$, where $X$ is a second-order variable, or $\varphi(\bar u)$, where $\varphi(\bar u)$ is a first-order formula over $\R$ (in particular, it does not mentioned any second-order variable). With this extension we obtain a class with the desired properties.
\begin{thm}\label{sigmafo-minusone}
The class $\eqso(\logex{1})$ is closed under sum, multiplication and subtraction by one. Moreover, $\eqso(\logex{1}) \subseteq \totp$ and every function in $\eqso(\logex{1})$ has an FPRAS.
\end{thm}
\proof
% !TeX root = main.tex

For the sake of readability, we divide the proof in three parts. The last part, i.e. subtraction by one, is probably the most technical proof of the paper. 

\medskip

\noindent {\bf Closed under sum and multiplication.} By the previous results, it is straightforward to prove that $\eqso(\logex{1})$ is closed under sum and multiplication. Indeed, $\eqso(\LL)$ is closed under sum by definition for every fragment $\LL$, and since $\logex{1}$ is closed under conjunction, from Lemma \ref{conj-mult} it follows that $\eqso(\logex{1})$ is closed under multiplication.
\medskip

\noindent {\bf Easy decision version and FPRAS.}  We show here that $\eqso(\logex{1}) \subseteq \totp$ and every function in $\eqso(\logex{1})$ has an FPRAS. We do this by showing a parsimonious reduction from any function in $\eqso(\logex{1})$ to some function in $\eqso(\loge{1})$, and using the result of Proposition~\ref{prop:qe0-fp-qe1-totp-fptras}. First, we define a function that converts any formula $\alpha$ in $\eqso(\logex{1})$ over a signature $\R$ into a formula $\lambda(\alpha)$ in $\eqso(\loge{1})$ over a signature $\R_{\alpha}$. Afterwards, we define a function $g_{\alpha}$ that receives an $\R$-structure $\A$ and outputs an $\R_{\alpha}$-structure $g_{\alpha}(\A)$.

Let $\alpha$ be in $\eqso(\logex{1})$. The signature $\R_{\alpha}$ is obtained by adding the symbol $R_{\psi}$ to $\R$ for every $\fo$ formula $\psi(\bar{z})$ in $\alpha$. Each symbol $R_{\psi}$ represents a predicate with arity $\length{\bar{z}}$. Then, $\lambda(\alpha)$ is defined as $\alpha$ where each $\fo$ formula $\psi(\bar{z})$ has been replaced by $R_{\psi}(\bar{z})$. We now define the function $g_{\alpha}$ with a polynomial time procedure. Let $\A$ be a $\R$-structure with domain $A$. Let $\A'$ be an $\R_{\alpha}$-structure obtained by copying $\A$ and leaving each $R_{\psi}^{\A'}$ empty. For each $\fo$-formula $\psi(\z)$ with $\length{\bar{z}}$ open first-order variables, we iterate over all tuples $\bar{a} \in A^{\length{\bar{z}}}$. If $\A\models\psi(\bar{a})$ (this can be done in $\ptime$), then the tuple $\bar{a}$ is added to $R_{\psi}^{\A'}$ . This concludes the construction of $\A'$. Note that the number of $\fo$ subformulas, arity and tuple size is fixed in $\alpha$, so computing this function takes polynomial time over the size of the structure. Moreover, the encoding of $\A'$ has polynomial size over the size of $\enc(\A)$. We define $g_{\alpha}(\A) = \A'$ and we have that for each $\R$-structure $\A$: $\sem{\alpha}(\A) = \sem{\lambda(\alpha)}(g_{\alpha}(\A))$. Therefore, we have a parsimonious reduction from $\alpha$ to the $\eqso(\loge{1})$ formula $\lambda(\alpha)$.

To show that $\alpha$ is in $\totp$, we can convert $\alpha$ and $\enc(\A)$ into $\lambda(\alpha)$ and $\enc(g_{\alpha}(\A))$, respectively, and run the procedure in Proposition~\ref{prop:qe0-fp-qe1-totp-fptras}. Similarly, to show that $\alpha$ has an FPRAS, we do the same as before and simulates the FPRAS for $\lambda(\alpha)$ in Proposition~\ref{prop:qe0-fp-qe1-totp-fptras}. These procedures also takes polynomial time and satisfies the required conditions.

\medskip


\noindent {\bf Closed under subtraction by one.} We prove here that $\eqso(\logex{1})$ is closed under subtraction by one. 
For this, given $\alpha \in \eqso(\logex{1})$ over a signature $\R$, we will define a $\eqso(\logex{1})$-formula $\kappa(\alpha)$ such that for each finite structure $A$ over $\R$: $\sem{\kappa(\alpha)}(\A) = \sem{\alpha}(\A) \dotminus 1$. 
Without loss of generality, we assume that $\alpha$ is in $\logex{1}$-SNF, that is, $\alpha = \sum_{i = 1}^{n}\sa{\bar{X}}\sa{\bar{x}}\varphi_i$ where each $\varphi_i$ is in $\logex{1}$. Moreover, we assume that $\length{\bar{x}} > 0$  since, if this is not the case, we can replace $\sa{\bar{X}}\varphi_i$ with the equivalent formula $\sa{\bar{X}} \sa{y} \varphi_i \wedge\first(y)$.

The proof will be separated in two parts. In the first part, we will assume that  $\alpha$ is in $\logex{1}$-PNF, namely, $\alpha = \sa{\bar{X}}\sa{\bar{x}}\varphi$ for some $\varphi$ in $\logex{1}$. We will show how to define a formula $\varphi'$ that satisfies the following condition: for each $\A$, if $(\A,V,v)\models \varphi(\bar{X},\bar{x})$ for some $V$ and $v$ over $\A$, then there exists exactly one assignment to $(\bar{X},\bar{x})$ that satisfies $\varphi$ and not $\varphi'$. 
From this, we clearly have that $\kappa(\alpha) = \sa{\bar{X}}\sa{\bar{x}}\varphi'$ will be the desired formula. In the second part, we suppose that $\alpha$ is of the form $\beta + \sa{\bar{X}}\sa{\bar{x}}\varphi$ with $\beta$ the sum of one or more formulas in $\logex{1}$-PNF. 
We define a formula $\varphi'$ that satisfies the following condition: if $(\A,V,v)\models \varphi(\bar{X},\bar{x})$ and $\sem{\beta}(\A) = 0$, then there exists exactly one assignment to $(\bar{X},\bar{x})$ that satisfies $\varphi$ and not $\varphi'$. From here, we can define $\kappa(\alpha)$ recursively as  $\kappa(\alpha) = \kappa(\beta) +  \sa{\bar{X}}\sa{\bar{x}}\varphi'$ and the property of subtraction by one will be~proven.

\medskip

\noindent {\em Part (1).}  Let $\alpha =  \sa{\bar{X}}\sa{\bar{x}} \varphi(\bar{X},\bar{x})$ where $\varphi$ is a $\logex{1}$-formula.
Note that, if $\alpha$ is of the form $\alpha = \sa{\bar{x}} \varphi(\bar{x})$ (i.e. $\length{\bar{X}} = 0$), we can define $\kappa(\alpha) = \sa{\bar{x}} [\varphi(\bar{x})\wedge \ex{\bar{z}}(\varphi(\bar{z})\wedge\bar{z} < \bar{x})]$, which is in $\eqso(\logex{1})$ and fulfills the desired condition. 
Therefore, for the rest of the proof we can assume that $\length{\bar{X}} > 0$ and $\length{\bar{x}} > 0$.

To simplify the analysis of $\varphi$, the first step is to rewrite $\varphi$ in a DNF formula. 
More precisely, we rewrite $\varphi$ into an equivalent formula of the form $\bigvee_{i = 1}^m \varphi_i$ for some $m\in \bbN$ where each $\varphi_i(\bar{X}, \bar{x}) = \ex{\bar{y}} \varphi_i'(\bar{X}, \bar{x}, \bar{y})$ and $\varphi_i'(\bar{X}, \bar{x}, \bar{y})$ is a conjunction of atomic formulas or negation of atomic formulas. Furthermore, we assume that each $\varphi_i'(\bar{X}, \bar{x}, \bar{y})$ has the~form:
$$
\varphi_i'(\bar{X}, \bar{x}, \bar{y}) =  \underbrace{\varphi_i^{\fo}(\bar{x},\bar{y})}_{\text{an $\fo$ formula}} \wedge 
\underbrace{\varphi_i^{+}(\bar{X},\bar{x},\bar{y})}_{\text{conjunction of $X_j$'s}} \wedge
\underbrace{\varphi_i^{-}(\bar{X},\bar{x},\bar{y})}_{\text{conjunction of $\neg X_j$'s}}.
$$
Note that atomic formulas, like $R(\bar{z})$ for $R \in \R$, will appear in the subformula $\varphi_i^{\fo}(\bar{x},\bar{y})$. 

Now, we define a series of rewritings to $\varphi$ that will make each formula $\varphi_i$ satisfy the following three conditions: (a) no variable from $\bar{x}$ is mentioned in $\varphi_i^{-}(\bar{X},\bar{x},\bar{y})\wedge\varphi_i^{+}(\bar{X},\bar{x},\bar{y})$, (b) $\varphi_i^{\fo}(\bar{x},\bar{y})$ defines an ordered partition over the variables in $(\bar{x},\bar{y})$ (see below for the formal definition of ordered partition) and (c) if $X_j(\bar{z})$ and $\neg X_j(\bar{w})$ are mentioned, then the ordered partition should not satisfy $\bar{z} = \bar{w}$.
We explain below how to rewrite $\varphi_i$ in order to satisfy each condition.
\begin{itemize} \itemsep1mm
	\item[(a)] In order to satisfy the first condition, consider some instance of a $X_j(\bar{w})$ in $\varphi_i$, where $\bar{w}$ is a subtuple of $(\bar{x},\bar{y})$. We add $\length{\bar{w}}$ new variables $z_1,\ldots,z_{\length{\bar{w}}}$ to the formula and let $\bar{z} = (z_1,\ldots,z_{\length{\bar{w}}})$. We redefine $\varphi_i^{+}(\bar{X},\bar{x},\bar{y})$ by replacing $X_j(\bar{w})$ with $X_j(\bar{z})$ (denoted by $\varphi_i^{+}(\bar{X},\bar{x},\bar{y})[X_j(\bar{w}) \leftarrow X_j(\bar{z})]$ and then the formula $\varphi_i$ is equivalently defined as:
	$$
	\varphi_i(\bar{X},\bar{x}) \ := \ \ex{\bar{y}} \ex{\bar{z}} (\bar{z} = \bar{w} \wedge \varphi_i^{\fo}(\bar{x},\bar{y})) \wedge \varphi_i^{+}(\bar{X},\bar{x},\bar{y})[X_j(\bar{w}) \leftarrow X_j(\bar{z})] \wedge
	\varphi_i^{-}(\bar{X},\bar{x},\bar{y}).
	$$
	We repeat this process for each instance of a $X_j(\bar{w})$ in $\varphi_i$, and we obtain a formula where none of the $X_j$'s acts over any variable in $\bar{x}$. We add the new first-order variables to $\bar{y}$ and we redefine $\varphi_i$ as:
	$$
	\varphi_i(\bar{X},\bar{x}) \ := \  \ex{\bar{y}} \varphi_i^{\fo}(\bar{x},\bar{y}) \wedge \varphi_i^{-}(\bar{X},\bar{y}) \wedge \varphi_i^{+}(\bar{X},\bar{y}).
	$$
	For example, if $\bar{x} = x$, $\bar{y} = y$ and $\varphi_i = \ex{\bar{y}}  x < y \wedge  X(x,y)\wedge \neg X(x,x)$, then we redefine $\bar{y} = (y,v_1,v_2,v_3,v_4)$ and:
	$$
	\varphi_i \ := \ \ex{\bar{y}} v_1 = x \wedge v_2 = y \wedge v_3 = x \wedge v_4 = x \wedge x < y  \wedge  X(v_1,v_2) \wedge \neg X(v_3,v_4).
	$$ 
	\item[(b)] An ordered partition on a set $S$ is defined by an equivalence relation $\sim$ over $S$, and a linear order over $S/\!\sim$. For example, let $\bar{x} = (x_1,x_2,x_3,x_4)$. A possible ordered partition would be defined by the formula $\theta(\bar{x}) = x_2 < x_1 \wedge x_1 = x_4 \wedge x_4 < x_3$. On the other hand, the formula $\theta'(\bar{x}) = x_1 < x_2 \wedge x_1 < x_4 \wedge x_2 = x_3$ does not define an ordered partition since both $\{x_1\}<\{x_2,x_3\}<\{x_4\}$ and $\{x_1\} < \{x_2,x_3,x_4\}$ satisfy $\theta'$.
	For a given $k$, let $\cB_k$ be the number of possible ordered partitions for a set of size $k$. For $1 \leq j \leq \cB_{\length{(\bar{x},\bar{y})}}$ 
	let $\theta^j(\bar{x},\bar{y})$ be the formula that defines the $j$-th ordered partition over $(\bar{x},\bar{y})$. 
	Thus, the formula $\varphi(\bar{X},\bar{x})$ is then redefined as:
	$$
	\varphi(\bar{X},\bar{x}) \ := \ \bigvee_{i = 1}^m \bigvee_{j = 1}^{\cB_{\length{(\bar{x},\bar{y})}}} \ex{\bar{y}} [\theta^j(\bar{x},\bar{y})\wedge \varphi_i^{\fo}(\bar{x},\bar{y}) \wedge \varphi_i^{-}(\bar{X},\bar{y}) \wedge \varphi_i^{+}(\bar{X},\bar{y})],
	$$
	Note that each $\theta^j(\bar{x},\bar{y})$ is an $\fo$-formula.
	Then, by redefining $\varphi_i^{\fo}(\bar{x},\bar{y})$  as $\theta^j(\bar{x},\bar{y})\wedge \varphi_i^{\fo}(\bar{x},\bar{y})$, we can suppose that each $\varphi_i^{\fo}(\bar{x},\bar{y})$ forces an ordered partition over the variables in $(\bar{x},\bar{y})$.
	\item[(c)] Finally, to rewrite the formula such that no $X_j(\bar{z})$ and $\neg X_j(\bar{w})$ are mentioned in $\varphi_i$ with $\bar{z}$ and $\bar{w}$ equivalent in the ordered partition (i.e. $\varphi_i$ is inconsistent), we do the following. If there exists an instance of $X_j(\bar{z})$ in $\varphi^{+}_i$, an instance of $\neg X_j(\bar{w})$ in $\varphi^{-}_i$ and the ordered partition in $\varphi^{\fo}_i$ satisfies $\bar{z} = \bar{w}$, then the entire formula $\varphi_i$ is removed from $\varphi$.
\end{itemize}
It is important to notice that the resulting $\varphi$ is equivalent to the initial one, and it is still a formula in $\eqso(\logex{1})$. From now on, we assume that each $\varphi_i(\bar{X},\bar{x}) = \ex{\bar{y}}  \varphi_i'(\bar{X},\bar{x}, \bar{y}) $ satisfies conditions (a), (b) and (c), and $\varphi_i'(\bar{X},\bar{x}, \bar{y})$ has the~form:
$$
\varphi_i'(\bar{X},\bar{x}, \bar{y}) \; = \; \varphi^{\fo}_i(\bar{x},\bar{y}) \wedge \varphi^{+}_i(\bar{X}, \bar{y}) \wedge \varphi^{-}_i(\bar{X},\bar{y}) 
$$
where $\varphi^{+}$ and $\varphi^{-}_i$ do not depend on  $\bar{x}$.
\begin{clm}\label{claim:minusone}
	For an ordered structure $\A$ and an FO assignment $v$ for $\A$, $(\A,v)\models\varphi^{\fo}_i(\bar{x},\bar{y})$ if, and only if, there exists an SO assignment $V$ for $\A$ such that $(\A,V,v)\models \varphi_i'(\bar{X},\bar{x}, \bar{y})$.
\end{clm}
\proof
	Let $\A$ be an ordered structure with domain $A$ and let $v$ be a first-order assignment for $\A$, such that $(\A,v)\models\varphi^{\fo}_i(\bar{x},\bar{y})$.
	Define $\bar{B} = (B_1,\ldots,B_{\length{\bar{X}}})$ as $B_j = \{v(\bar{w})\mid \text{ $X_j(\bar{w})$ is mentioned in $\varphi^{+}_i(\bar{X},\bar{y})$}\}$, and let $V$ be a second-order assignment for which $V(\bar{X}) = \bar{B}$.
	Towards a contradiction, suppose that $(\A,V,v)\not\models \varphi^{\fo}_i(\bar{x},\bar{y}) \wedge \varphi^{+}_i(\bar{X}, \bar{y}) \wedge \varphi^{-}_i(\bar{X},\bar{y})$.
	By the choice of $v$, and construction of $V$ it is clear that $(\A,V,v)\models\varphi^{\fo}_i(\bar{x},\bar{y})\wedge\varphi^{+}_i(\bar{X},\bar{y})$, so we necessarily have that $(\A,V,v)\not\models\varphi^{-}_i(\bar{X},\bar{y})$.
	Let $X_t$ be such that $\neg X_t(\bar{w})$ is mentioned in $\varphi^{-}_i(\bar{X},\bar{y})$ and $(\A,V, v)\not\models\neg X_t(\bar{w})$, namely, $v(\bar{w})\in B_t$. 
	However, by the construction of $B_t$, there exists a subtuple $\bar{z}$ of $\bar{y}$ such that $X_t(\bar{z})$ appears in $\varphi^{+}_i(\bar{X},\bar{y})$ and $v(\bar{z}) = v(\bar{w})$. Since $(\A,v)\models\varphi^{\fo}_i(\bar{x},\bar{y})$ and $v(\bar{z}) = v(\bar{w})$, then the ordered partition in $\varphi^{\fo}_i$ satisfies $\bar{z} = \bar{w}$. This violates condition (c) above since $\neg X_t(\bar{w})$ appears in $\varphi^{-}_i$ and $X_t(\bar{z})$ appears in $\varphi^{+}_i$, which leads to a contradiction. 
	
	For the other direction, if $(\A,V,v)\models \varphi_i'(\bar{X},\bar{x}, \bar{y})$ for a second order assignment $V$ for $\A$, then it is easy to check $(\A,v)\models\varphi^{\fo}_i(\bar{x},\bar{y})$ since $\varphi^{\fo}_i(\bar{x},\bar{y})$ is a subformula of $\varphi_i'$.
\qed

The previous claim and proof motivate the following definition.
For a structure $\A$ and a first-order assignment $v$ for $\A$, define $\bar{B}^v = (B^v_1,\ldots,B^v_{\length{\bar{X}}})$ where each $B^v_j = \{v(\bar{w}) \mid \text{$X_j(\bar{w})$ is mentioned in $\varphi^{+}_i(\bar{X},\bar{y})$}\}$.
One can easily check that for every assignments $(V, v)$ such that $(\A,V,v)\models \varphi_i'(\bar{X}, \bar{x}, \bar{y})$, it holds that $(\A,\bar{B}^v,v)\models \varphi_i'(\bar{X}, \bar{x}, \bar{y})$ and $\bar{B}^v \subseteq V(\bar{X})$, namely, $\bar{B}^v$ is a valid candidate for $\bar{X}$ and, furthermore, it is contained in all assignments of $\bar{X}$ when $v$ is fixed.
This motivates the main idea of Part (1): by choosing one particular $v$ the plan is to remove $\bar{B}^v$ as an assignment over $\bar{X}$ in $\varphi_i$. 
For this, we choose the minimal $v$ that satisfies $\varphi^{\fo}_i(\bar{x},\bar{y})$ which can be defined with the following formula:
\[
\text{$min$-}\varphi^{\fo}_i(\bar{x}, \bar{y}) \; = \;  \,\varphi_i^{\fo}(\bar{x},\bar{y})\wedge \fa{\bar{x}'} \fa{\bar{y}'}(\varphi_i^{\fo}(\bar{x}',\bar{y}')\to ( \bar{x}\leq\bar{x}' \wedge  \bar{y}\leq\bar{y}')).
\] 
If $\varphi^{\fo}_i$ is satisfiable, let $v$ be the only assignment such that  $(\A,v)\models \text{$min$-}\varphi^{\fo}_i(\bar{x}, \bar{y})$. 
Furthermore, let $V^*$ be the second order assignment and $v^*$ the first order assignment that satisfy $V^*(\bar{X}) = \bar{B}^{v}$ and $v^*(\bar{x}) = v(\bar{x})$.
By the previous discussion, $(\A,V^*,v^*)\models\varphi_i(\bar{X},\bar{x})$.

Now, we have all the ingredients in order to define $\kappa(\alpha)$. 
Intuitively, we want to exclude the assignment $(V^*, v^*)$ from the satisfying assignments of $\varphi_i(\bar{X},\bar{x})$.
Towards this goal, we can define a formula $\psi_i(\bar{X},\bar{x})$ such that, if $\varphi_i(\bar{X}, \bar{x})$ is satisfiable, $(\A, V, v) \models \psi_i(\bar{X},\bar{x})$ if, and only if either $V \neq V^*$ or $v \neq v^*$. 
This property can be defined as follows:
\begin{align}
\psi_i(\bar{X},\bar{x}) \; := \;  &\big( \, \ex{\bar{x}} \ex{\bar{y}} \varphi^{\fo}_i(\bar{x},\bar{y}) \,\big) \rightarrow  \label{eq:part1} \\
&\Big( \; 
\ex{\bar{y}} \text{$min$-}\varphi^{\fo}_i(\bar{x}, \bar{y}) \wedge \big( \; \varphi_i'(\bar{X},\bar{x}, \bar{y})  \rightarrow \!\! \bigvee_{X \in \bar{X}} \ex{\bar{z}} [ \, X(\bar{z}) \wedge   \!\!\!\!\!\!\!\!\!\! \bigwedge\limits_{X(\bar{w}) \, \in \, \varphi^{+}_i(\bar{X},\bar{y})} \!\!\!\!\!\!\!\!\!\!\! \bar{w}\neq\bar{z} \, ] \; \big) \;  \vee   \label{eq:part2} \\ 
&(\; \ex{\bar{x}'} \ex{\bar{y}} \varphi_i'(\bar{X},\bar{x}',\bar{y})\wedge \bar{x}' < \bar{x} \; ) \;\; \Big)  \label{eq:part3}
\end{align}
To understand the formula, first notice that the premise of the implication at (\ref{eq:part1}) is true if, and only if, $\varphi_i(\bar{X}, \bar{x})$ is satisfiable. 
Indeed, by Claim~\ref{claim:minusone} we know that if $\ex{\bar{x}} \ex{\bar{y}} \varphi^{\fo}_i(\bar{x},\bar{y})$ is true, then there exists assignments $V$ and $v$ such that $(\A,V,v) \models\varphi_i'(\bar{X},\bar{x},\bar{y})$.
The conclusion of the implication (divided into (\ref{eq:part2}) and (\ref{eq:part3})), take care that $V(\bar{X}) \neq V^*(\bar{X})$ or $v(\bar{x}) \neq v^*(\bar{x})$.
Here, the first disjunct (\ref{eq:part2}) checks that $V(\bar{X}) \neq V^*(\bar{X})$ by defining that if $\varphi_i'(\bar{X},\bar{x}, \bar{y})$ is satisfied then $V^*(\bar{X}) \subsetneq V(\bar{X})$. 
The second disjunct (\ref{eq:part3}) is satisfied when $v(\bar{x})$ is not the lexicographically smallest tuple that satisfies $\varphi_i$ (i.e. $v(\bar{x}) \neq v^*(\bar{x})$).
Finally, from the previous discussion one can easily check that $\psi_i(\bar{X},\bar{x})$ satisfies the desire property.

We are ready to define the formula  $\kappa(\alpha)$ as $\sa{\bar{X}}\sa{\bar{x}} \bigvee_{i = 1}^m\varphi_i^*(\bar{X},\bar{x})$
where each modified disjunct $\varphi_i^*(\bar{X},\bar{x})$ is constructed as follows. 
For the sake of simplicity, define the auxiliary formula $\chi_i = \neg \ex{\bar{x}} \ex{\bar{y}}\varphi^{\fo}_i(\bar{x},\bar{y})$. 
This formula basically checks if $\varphi_i$ is not satisfiable (recall Claim~\ref{claim:minusone}).
Define the first formula $\varphi_1^*$ as:
\[
\varphi^*_1(\bar{X},\bar{x}) \; := \; \varphi_1(\bar{X},\bar{x})\wedge\psi_1(\bar{X},\bar{x}).
\]
This formula accepts all the assignments that satisfy $\varphi_1$, except for the assignment $(V^*,v^*)$ of $\varphi_1$. The second formula $\varphi_2^*$ is defined as:
\[
\varphi^*_2(\bar{X},\bar{x}) \; := \; \varphi_2(\bar{X},\bar{x})\wedge\psi_1(\bar{X},\bar{x})\wedge(\chi_1\to\psi_2(\bar{X},\bar{x})).
\]
This models all the assignments that satisfy $\varphi_2$, except for the assignment $(V^*,v^*)$ of $\varphi_1$. Moreover, if $\varphi_1$ is not satisfiable, then $\psi_1(\bar{X},\bar{x})$ and $\chi_1$ will hold, and the formula $\psi_2(\bar{X},\bar{x})$ will forbid the assignment $(V^*,v^*)$ of $\varphi_2$. 
One can easily generalize this construction for each $\varphi_i$ as follows:
\begin{multline*}
\varphi_i^*(\bar{X},\bar{x}) \; := \; \varphi_i(\bar{X},\bar{x})\wedge\psi_1(\bar{X},\bar{x})\,\wedge \\ (\chi_1\to\psi_2(\bar{X},\bar{x}))\wedge((\chi_1\wedge\chi_2)\to\psi_3(\bar{X},\bar{x}))\wedge\cdots\wedge(
\bigwedge_{j = 1}^{j = i-1}\chi_j\to\psi_i(\bar{X},\bar{x})).
\end{multline*}
From the construction of $\kappa(\alpha)$, one can easily check that  $\sem{\kappa(\alpha)}(\A) = \sem{\alpha}(\A)-1$ for each $\A$.

\medskip

\noindent {\em Part (2).} Let $\alpha = \beta + \sa{\bar{X}}\sa{\bar{x}} \varphi(\bar{X},\bar{x})$ for some $\eqso(\logex{1})$ formula $\beta$. We define $\kappa(\alpha)$ as follows.
First, rewrite $\varphi(\bar{X},\bar{x})$ as in Part (1). Let $\varphi = \bigvee_{i = 1}^m\varphi_i(\bar{X},\bar{x})$ where each $\varphi_i$ satisfies conditions (a), (b) and (c) defined above. Also, consider the previously defined formulas $\chi_i$ and $\psi_i$, for each $i \leq m$. 
We construct a function $\lambda$ that receives a formula $\beta \in \eqso(\logex{1})$ and produces a logic formula $\lambda(\beta)$ that satisfies $\A\models\lambda(\beta)$ if, and only if, $\sem{\beta}(\A) = 0$. If $\beta = \sa{\bar{x}} \varphi(\bar{x})$, then $\lambda(\beta) = \neg \ex{\bar{x}'} \varphi(\bar{x}')$. If $\beta = \sa{\bar{X}}\sa{\bar{x}}\varphi(\bar{X},\bar{x})$, then 
%let $\varphi = \bigvee_{i = 1}^{m}\varphi_i$ where each $\varphi_i$ satisfies conditions (a), (b) and (c) of the previous section, and 
define $\lambda(\beta) = \chi_1\wedge \cdots\wedge\chi_m$. If $\beta = (\beta_1 + \beta_2)$, then $\lambda(\beta) = \lambda(\beta_1) \wedge \lambda(\beta_2)$.
Now, following the same ideas as in Part (1) we define a formula $\varphi_i^*(\bar{X},\bar{x})$ that removes the minimal $(V^*, v^*)$ of $\varphi_i$ whenever $\beta$ cannot be satisfied (i.e. $\lambda(\beta)$ is true). Formally, we define $\varphi_i^*$ as follows:
\begin{multline*}
\varphi_i^*(\bar{X},\bar{x}) \ := \ \varphi_i(\bar{X},\bar{x})\wedge\Big(\lambda(\beta)\to\Big(\psi_1(\bar{X},\bar{x})\,\wedge \\
(\chi_1\to\psi_2(\bar{X},\bar{x}))\wedge((\chi_1\wedge\chi_2)\to\psi_3(\bar{X},\bar{x}))\wedge\cdots\wedge(
\bigwedge_{j = 1}^{j = i-1}\chi_j\to\psi_i(\bar{X},\bar{x}))\Big)\Big).
\end{multline*}
Finally, $\kappa(\alpha)$ is defined as $\kappa(\alpha) = \kappa(\beta) + \sa{\bar{X}}\sa{\bar{x}} \bigvee_{i = 1}^m\varphi_i^*(\bar{X},\bar{x})$, which is in $\eqso(\logex{1})$ and satisfies the desired conditions. This concludes the proof.
\qed
The proof that $\eqso(\logex{1})$ is closed under subtraction by one is the most involved of the paper. We think the main technique used in this proof, which is based on considering some witnesses of logarithmic size, is of independent interest.