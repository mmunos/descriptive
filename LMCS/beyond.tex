% !TeX root = main.tex

We have used weighted logics to give a framework for descriptive complexity of counting complexity classes. Here, we go beyond weighted logics and give the first steps on defining recursion at the quantitative level.
This goal is not trivial not only because we want to add recursion over functions, but also because 
it is not clear what could be the right notion of ``fixed point''. 
To this end, we show first how to extend $\qso$ with function symbols that are later
used 
to define a natural generalization for functions of 
the notion of least fixed point of LFP.
 As a proof of concept, we show that this notion can be used to capture $\fp$.
Moreover, we use this concept
to define an operator for counting paths in a graph, a natural generalization of the transitive closure operator~\cite{immerman1999descriptive}, and show that this gives rise to a logic that captures~$\shl$. 

We start by defining an extension of $\qso$ with function symbols. Assume that $\fs$ is an infinite set of function symbols, where each $h \in \fs$ has an associated arity denoted by $\arity(h)$. Then the set of $\fqso$ formulae over a signature $\R$ is defined by the following grammar:
\begin{multline}
\label{eq-fqso}
	\alpha \ :=  \ \varphi \ \mid \  s \  \mid \  h(x_1, \ldots, x_\ell) \  \mid \
	(\alpha \add \alpha) \  \mid\  (\alpha \mult \alpha) \  \mid \\
	\sa{x} \alpha \  \mid \
	\pa{x} \alpha \  \mid \
	\sa{X} \alpha \  \mid \
	\pa{X} \alpha,
\end{multline}
where $h \in \fs$, $\arity(h) = \ell$ and $x_1, \ldots, x_\ell$ is a sequence of (not necessarily distinct) first-order variables. Given an $\R$-structure $\A$ with domain $A$, we say that $F$ is a \emph{function assignment} for $\A$ if for every $h \in \fs$ with $\arity(h) = \ell$, we have that $F(h) :  A^\ell \to \N$. The notion of function assignment is used to extend the semantics of $\qso$ to the case of a quantitative formula of the form $h(x_1, \ldots, x_\ell)$. More precisely, given first-order and second-order assignments $v$ and $V$ for $\A$, respectively, 
we have that:
\begin{align*}
\sem{h(x_1, \ldots, x_\ell)}(\A,v,V,F) &= F(h)(v(x_1),\ldots, v(x_\ell)).
\end{align*}
As for the case of $\qfo$, we define $\fqfo$ disallowing quantifiers $\Sigma X$ and $\Pi X$ in \eqref{eq-fqso}.

It is worth noting that function symbols in $\fqso$ represent functions from tuples to natural numbers, so they are different from the classical notion of function symbol in $\fo$~\cite{L04}. 
Furthermore, a function symbol can be seen as an ``oracle'' that is instantiated by the function assignment. 
To the best of our knowledge, this is the first article to propose this extension on weighted logics, which we think should be further investigated. 

We define an extension of LFP \cite{I86,vardi1982complexity} to allow counting. 
More precisely, the set of $\rqfo(\fo)$ formulae over a signature $\R$, where $\rqfo$ stands for recursive $\qfo$, is defined as an extension of $\qfo(\fo)$ that includes the formula $\clfp{\beta(\x, h)}$, where (1) $\x = (x_1, \ldots, x_\ell)$ is a sequence of $\ell$ distinct first-order variables, (2) $\beta(\x, h)$ is an $\fqfo(\fo)$-formula over $\R$ whose only function symbol is $h$, and (3) $\arity(h) = \ell$. The free variables of the formula $\clfp{\beta(\x,h)}$ are $x_1, \ldots, x_\ell$; in particular, $h$ is not considered to be free.

Fix an $\R$-structure with domain $A$ and a quantitative formula $\clfp{\beta(\x,h)}$ with $\arity(h) = \ell$, and assume that $\F$ is the set of functions $f :A^\ell \to \N$. To define the semantics of $\clfp{\beta(\x,h)}$, we first show how $\beta(\x,h)$ can be interpreted as an operator $T_{\beta}$ on $\F$. More precisely, for every $f \in \F$ and tuple $\a = (a_1, \ldots, a_{\ell}) \in A^\ell$, the function $T_{\beta}(f)$ satisfies that:
\begin{align*}
T_{\beta}(f)(\a) &= \sem{\beta(\x, h)}(\A,v,F),
\end{align*}
where $v$ is a first-order assignment  for $\A$ such that $v(x_i) = a_i$ for every $i \in \{1, \ldots, \ell\}$, and $F$ is a function assignment for $\A$ such that $F(h) = f$. 

As for the case of LFP, it would be natural to consider the point-wise partial order $\leq$ on $\F$ defined as $f \leq g$ if, and only if, $f(\bar{a}) \leq g(\bar{a})$ for every $\bar{a} \in A^{\ell}$, and let the semantics of $\clfp{\beta(\x,h)}$ be the least fixed point of the operator $T_\beta$. However, $(\F, \leq)$ is not a complete lattice, so we do not have a Knaster-Tarski Theorem ensuring that such a fixed point exists. Instead, we generalize the semantics of LFP as follows. In the definition of the semantics of LFP, an operator $T$ on relations is considered, and the semantics is defined in terms of the least fixed point of $T$, that is, a relation $R$ such that~\cite{I86,vardi1982complexity}: 
(a) $T(R) = R$, and (b) $R \subseteq S$ for every $S$ such that $T(S) = S$. 
We can view $T$ as an operator on functions if we consider the characteristic function of a relation. Given a relation $R \subseteq A^\ell$, let $\chi_R$ be its characteristic function, that is $\chi_R(\bar a) = 1$ if $\bar a \in R$, and $\chi_R(\bar a) = 0$ otherwise. Then define an operator $T^\star$ on characteristic functions as $T^\star(\chi_R) = \chi_{T(R)}$. Moreover, we can rewrite the conditions defining a least fixed point of $T$ in terms of the operator $T^\star$ if we consider the notion of support of a function. Given a function $f \in \F$, define the support of $f$, denoted by $\support(f)$, as $\{ \bar a \in A^\ell \mid f(\bar a) > 0 \}$. Then given that $\support(\chi_R) = R$, we have that the conditions (a) and (b) are equivalent to the following conditions on $T^\star$:
(a) $\support(T^\star(\chi_R)) = \support(\chi_R)$, and  (b) $\support(\chi_R) \subseteq \support(\chi_S)$ for every $S$ such that  $\support(T^\star(\chi_{S})) = \support(\chi_S)$.
To define a notion of fixed point for $T_\beta$ we simply generalize these conditions. More precisely, a function $f \in \F$ is a {\em s-fixed point} of $T_{\beta}$ if $\support(T_\beta(f)) = \support(f)$, and $f$ is a {\em least s-fixed point} of $T_{\beta}$ if $f$ is a s-fixed point of $T_\beta$ and for every s-fixed point $g$ of $T_\beta$ it holds that $\support(f) \subseteq \support(g)$. The existence of such fixed point is ensured by the following lemma:
\begin{lem}\label{lem-support}
Let $h \in \fs$ such that $\arity(h) = \ell$, and $\beta$ be an $\fqfo(\fo)$-formula over a signature $\R$ such that if a function symbol occurs in $\beta$, then this function symbol is $h$. Moreover, let $\A$ be an $\R$-structure with domain $A$, $f,g : A^\ell \to \mathbb{N}$ and $F,G$ be function assignments such that $F(h) = f$ and $F(h) = g$. If $\support(f) \subseteq \support(g)$, then for every first-order and second-order assignments $v$ and $V$, respectively, it holds that:
\begin{center}
if $\sem{\beta}(\A,v,V,F) > 0$, then $\sem{\beta}(\A,v,V,G) > 0$.
\end{center}
\end{lem}
\proof
{\bf For }$\boldsymbol{\fqfo(\fo).}$ Let $\R$ be a signature. We prove the statement for $\fqfo(\fo)$ over $\R$.

We prove inductively that for each formula $\beta(\bar{x},h)$ in $\fqfo(\fo)$ over $\R$, for a given pair of functions $f,g$ such that $\supp(f)\subseteq\supp(g)$, it holds that $\supp(T_{\beta}(f))\subseteq\supp(T_{\beta}(g))$. Let $\length{\bar{x}} = \ell$.

We separate the proof in each case determined by the $\fqfo$ grammar. For each of the following cases.
\begin{itemize}
\item[1.] $\beta$ is either equal to a constant $s$ or an $\fo$ formula $\varphi$. Then $h$ does not appear. Then, for each structure $\A$, each first-order assignment $v$ and functional assignments $F,G$ over $\A$, we have that $\sem{\beta(\bar{x},h)}(\A,v,F) = \sem{\beta(\bar{x},h)}(\A,v,G)$. As a result, $\supp(T_{\beta}(f)) = \supp(T_{\beta}(g))$ for every pair of functions $f,g$.
\item[2.] $\beta$ is equal to $h(\bar{y})$ for some subtuple $\bar{y}$ of $\bar{x}$. Then $T_{\beta}(f) = f$ and $T_{\beta}(g) = g$ and the condition holds trivially.
\end{itemize}
Suppose that the statement holds for each formula smaller than $\beta$.
\begin{itemize}
\item[3.] $\beta = (\beta_1 + \beta_2)$. It is easy to see that for each $\bar{a} \in A^{\ell}$ and function $f:A^{\ell}\to\nat$: $T_{\beta}(f)(\bar{a}) = T_{\beta_1}(f)(\bar{a}) + T_{\beta_2}(f)(\bar{a})$. Suppose $\supp(f)\subseteq\supp(g)$ and let $\bar{a} \in \supp(T_{\beta}(f))$, or in other words, $T_{\beta}(f)(\bar{a}) > 0$. Then, for some $\beta_i$ it holds that $T_{\beta_i}(f)(\bar{a}) > 0$. From the supposition we have that $T_{\beta_i}(g)(\bar{a}) > 0$ from which the statement follows.
\item[4.] $\beta = (\beta_1 \mult \beta_2)$. It is easy to see that for each $\bar{a}$ in $A^{\ell}$ and function $f:A^{\ell}\to\nat$: $T_{\beta}(f)(\bar{a}) = T_{\beta_1}(f)(\bar{a}) \mult T_{\beta_2}(f)(\bar{a})$. Suppose $\supp(f)\subseteq\supp(g)$ and let $\bar{a}$ be such that $T_{\beta}(f)(\bar{a}) > 0$. Then $T_{\beta_i}(f)(\bar{a}) > 0$ for both $\beta_i$. From the supposition we have that $T_{\beta_i}(g)(\bar{a}) > 0$ for both $\beta_i$ and the statement holds.
\item[5.] $\beta = \sa{y}\delta(y,\bar{x},h)$. Here we extend the grammar slightly to allow constants, and we use the notation $\delta[a/y]$ to denote the formula obtained by replacing each instance of $y$ by the constant $a$. It can be seen that $T_{\beta}(f)(\bar{a}) = \sum_{a \in A} T_{\delta[a/y]}(f)(\bar{a})$. Suppose $\supp(f)\subseteq\supp(g)$ and let $\bar{a}$ be such that $T_{\beta}(f)(\bar{a}) > 0$. Then for some $a\in A$ we have $T_{\delta[y/a]}(f)(\bar{a}) > 0$. The statement now follows as in the case 3.
\item[6.] $\beta = \pa{y}\delta(y,\bar{x},h)$. 
It can be seen that $T_{\beta}(f)(\bar{a}) = \prod_{a \in A} T_{\delta[a/y]}(f)(\bar{a})$. 
%If $T_{\beta}(f)(\bar{a}) > 0$, then $T_{\delta[y/a]}(f)(\bar{a}) > 0$ for each $a\in A$. As in the $\mult$ case, the statement follows directly.
The statement follows using the same argument from cases 4 and 5.
\end{itemize}
This covers all possible cases for $\beta$ and we finish the proof of the statement for $\fqfo(\fo)$.

\vspace{1em}
{\bf For} $\boldsymbol{\rqfo(\fo).}$ The only additional case is where $\beta = \clfp{\delta(\bar{y},h')}$ for some subtuple $\bar{y}$ of $\bar{x}$. We have that $\beta$ does not mention $h$, and so, the statement follows directly as we showed in the previous part of the proof.
\qed

In the particular case of an $\rqfo(\fo)$-formula $\clfp{\beta(\x, h)}$, Lemma  \ref{lem-support} tell us that if $f,g \in \F$ and $\support(f) \subseteq \support(g)$, then $\support(T_\beta(f)) \subseteq \support(T_\beta(g))$. Hence, as for the case of LFP, this lemma gives us a simple way to compute a least s-fixed point of $T_\beta$. Let $f_0 \in \F$ be a function such that $f_0(\bar a) = 0$ for every $\bar a \in A^\ell$ (i.e. $f_0$ is the only function with empty support), and let function $f_{i+1}$ be defined as $T_\beta(f_i)$ for every $i \in \N$. Then there exists $j \geq 0$ such that $\support(f_j) = \support(T_\beta(f_j))$. Let $k$ be the smallest natural number such that $\support(f_{k}) = \support(T_\beta(f_k))$. We have that $f_k$ is a least s-fixed point of $T_\beta$, which is used to define the semantics of $\clfp{\beta(\x, h)}$. More specifically, for an arbitrary first-order assignment $v$ for $\A$:
\begin{align*}
\sem{\clfp{\beta(\x, h)}}(\A,v) &= f_{k}(v(\x)).
\end{align*}

\begin{exa} \label{ex:count-path}
We would like to define an $\rqfo(\fo)$-formula that, given a directed acyclic graph $G$ with $n$ nodes and a pair of nodes $b$, $c$ in $G$, counts the number of paths of length less than $n$ from $b$ to $c$ in $G$. To this end, assume that graphs are encoded using the signature $\R = \{ E(\cdot,\cdot)\}$, and then define formula $\alpha(x, y, f)$ as follows:
$$ 
E(x,y) + \sa{z} f(x,z)\cdot E(z,y).
$$
We have that $\clfp{\alpha(x,y,f)}$ defines our counting function. In fact, assume that $\A$ is an $\R$-structure with $n$ elements in its domain encoding an acyclic directed graph. Moreover, assume that $b,c$ are elements of $\A$ and $v$ is a first-order assignment over $\A$ such that $v(x) = b$ and $v(y) = c$. Then we have that $\sem{\clfp{\alpha(x,y,f)}}(\A,v)$ is equal to the  number of paths in $\A$ from $b$ to $c$ of length at most $n$.

Assume now that we need to extend our previous counting function to the case of arbitrary directed graphs. To this end, suppose that $\varphi_{\text{\rm first}}(x)$ and $\varphi_{\text{succ}}(x,y)$ are $\fo$-formulae defining the first element of $<$ and the successor relation associated to $<$, respectively.\footnote{Recall that in this paper we consider ordered $\R$-structures.} Moreover, define formula $\beta(x, y, t, g)$ as follows:
\begin{align*}
(E(x,y) + \sa{z} g(x,z,t)\cdot E(z,y)) \cdot \varphi_{\text{\rm first}}(t) \ +
\sa{t'} \varphi_{\text{succ}}(t',t) \cdot \left(\sa{x'} \sa{y'} g(x',y',t') \right)
\end{align*}
Then our extended counting function is defined by:
$$
\sa{t} (\varphi_{\text{\rm first}}(t) \cdot \clfp{\beta(x,y,t,g)}).
$$ 
In fact, the number of paths of length at most $n$ from a node $x$ to a node $y$ is recursively computed by using the formula $(E(x,y) + \sa{z} g(x,z,t)\cdot E(z,y)) \cdot \varphi_{\text{\rm first}}(t)$, which stores this value in $g(x,y,t)$ with $t$ the first element in the domain.  The other formula $\sa{t'} \varphi_{\text{succ}}(t',t) \cdot \left(\sa{x'} \sa{y'} g(x',y',t') \right)$ is just an auxiliary artifact that is used as a counter to allow reaching a fixed point in the support of $g$ in $n$ steps. Notice that the use of the filter $\varphi_{\text{succ}}(t',t)$ prevents this formula for incrementing the value of $g(x,y,t)$ when $t$ is the first element in the domain. \qed
\end{exa}
In contrast to $\lfp$, to reach a fixed point we do not need to impose any positive restriction on the formula $\beta(\x,h)$.
In fact, since $\beta$ is constructed from monotone operations (sum and product) over the natural numbers, the resulting operator $T_{\beta}$ is monotone as well.

Now that a least fixed point operator over functions is defined, the next step is to understand its expressive power.
In the following theorem, we show that this operator can be used to capture $\fp$.
\begin{thm} \label{rqfo-fo-cap}
	$\rqfo(\fo)$ captures $\fp$ over ordered structures.
\end{thm}
\proof
Given the definition of the semantics of $\rqfo(\fo)$, it is clear that a fixed formula $\clfp{\beta(\x, h)}$ can be evaluated in polynomial time, from which it is possible to conclude that each fixed formula in $\rqfo(\fo)$ can be evaluated in polynomial time. Thus, to prove that $\rqfo(\fo)$ captures $\fp$, we only need to prove the second condition in Definition \ref{def:cap}.

\newcommand{\ttB}{\mathtt{B}}
\newcommand{\successor}{\text{succ}}

%For the second condition, 
Let $f$ be a function in $\fp$. We address the case when $f$ is defined for the encodings of the structures of a relational signature $\R = \{ E(\cdot, \cdot) \}$, as the proof for an arbitrary signature is analogous.
%$\R$ contains only one binary predicate $E$, and the remaining cases can be deduced from this. 
 Let $M$ be a deterministic polynomial-time TM with a working tape and an output tape, such that the output of $M$ on input $\enc(\A)$ is $f(\enc(\A))$ for each $\R$-structure $\A$. We assume that $M = (Q,\{0,1\},q_0,\delta)$, 
 %without final states, 
 where $Q = \{q_0,\ldots,q_{\ell}\}$, and $\delta:Q\times\{0,1,\ttB, \vdash\}\to Q\times\{0,1,\ttB, \vdash\}\times \{\leftarrow,\rightarrow\}\times\{0,1,\emptyset\}$ is a partial function. In particular, the tapes of $M$ are infinite to the right so the symbol $\vdash$ is used to indicate the first position in each tape, and $M$ does not have any final states, as it produces an output for each input. Moreover, the only allowed operations in the output tape are: 
 %The machine has an output tape and the only allowed operations in that tape on each step are 
 (1) writing 0 and moving the head one cell to the right, (2) writing 1 and moving the head one cell to the right, or (3) doing nothing. These operations are represented by the set $\{0,1,\emptyset\}$. Finally, assume that $M$, on input $\enc(\A)$ with domain $A = \{1,\dots,n\}$, executes exactly $n^k$ steps for some $k \geq 1$.

We construct a formula $\alpha$ in an extension of the grammar of $\rqfo(\fo)$ such that $\sem{\alpha}(\A) = f(\enc(\A))$ for each $\R$-structure $\A$. This extension allows defining the operator ${\bf lsfp}$ for multiple functions, analogously to the notion of simultaneous LFP~\cite{L04}.
%generalization of fixed-point operator with multiple predicates. 
Let $\bar{x} = (x_1,\ldots,x_k)$ and $\bar{t} = (t_1,\ldots,t_k)$. Then $\alpha$ is defined as:
\begin{align*}
\alpha = \sa{\bar{t}}\clfp{out(\bar{t}): \,&\alpha_{T_0}(\bar{t},\bar{x},T_0,T_1,T_{\ttB},T_{\vdash},h,\hat{h},s_{q_0},\ldots,s_{q_{\ell}},out),\\
	&\alpha_{T_1}(\bar{t},\bar{x},T_0,T_1,T_{\ttB},T_{\vdash},h,\hat{h},s_{q_0},\ldots,s_{q_{\ell}},out),\\
	&\alpha_{T_{\ttB}}(\bar{t},\bar{x},T_0,T_1,T_{\ttB},T_{\vdash},h,\hat{h},s_{q_0},\ldots,s_{q_{\ell}},out),\\
	&\alpha_{T_{\vdash}}(\bar{t},\bar{x},T_0,T_1,T_{\ttB},T_{\vdash},h,\hat{h},s_{q_0},\ldots,s_{q_{\ell}},out),\\
	&\alpha_{h}(\bar{t},\bar{x},T_0,T_1,T_{\ttB},T_{\vdash},h,\hat{h},s_{q_0},\ldots,s_{q_{\ell}},out),\\
	&\alpha_{\hat{h}}(\bar{t},\bar{x},T_0,T_1,T_{\ttB},T_{\vdash},h,\hat{h},s_{q_0},\ldots,s_{q_{\ell}},out),\\
	&\alpha_{s_{q_0}}(\bar{t},T_0,T_1,T_{\ttB},T_{\vdash},h,\hat{h},s_{q_0},\ldots,s_{q_{\ell}},out),\\
	&\vdots \\
	&\alpha_{s_{q_{\ell}}}(\bar{t},T_0,T_1,T_{\ttB},T_{\vdash},h,\hat{h},s_{q_0},\ldots,s_{q_{\ell}},out),\\
	&\alpha_{out}(\bar{t},T_0,T_1,T_{\ttB},T_{\vdash},h,\hat{h},s_{q_0},\ldots,s_{q_{\ell}},out)}\mult \last(\bar{t}).
\end{align*}
Function $T_0$ is used to indicate whether the content of a cell of the working tape is 0 at some point of time, that is, $T_0(\bar{t},\bar{x}) > 0$ if the cell at position $\bar{x}$ of the working tape contains the symbol 0 at step $\bar{t}$, and $T_0(\bar{t},\bar{x}) = 0$ otherwise. Functions $T_1$, $T_{\ttB}$ and $T_{\vdash}$ are defined analogously. Function $h$ is used to indicate whether the head of the working tape is in some position at some point of time, that is, $h(\bar{t},\bar{x}) > 0$ if the head of the working tape is at position $\bar{x}$ at step $\bar{t}$, and $h(\bar{t},\bar{x}) = 0$ otherwise. 
Function $\hat{h}$ is used to indicate whether the head of the working tape is {\bf not} in some position at some point of time, that is, $\hat{h}(\bar{t},\bar{x}) > 0$ if the head of the working tape is {\bf not} at position $\bar{x}$ at step $\bar{t}$, and $h(\bar{t},\bar{x}) = 0$ otherwise. For each $i \in \{0, \ldots, \ell\}$, function 
$s_{q_i}$ is used to indicate whether the TM $M$ is in state $q_i$ at some point of time, that is, $s_{q_i}(\bar{t}) > 0$ if the TM $M$ is in state $q_i$ at step $\bar{t}$, and $s_{q_i}(\bar{t},\bar{x}) = 0$ otherwise. Finally, function $out$ stores the output of the TM $M$; in particular, $out(\bar t)$ is the value returned by $M$ when $\bar t$ is the last step (that is, when $\last(\bar t)$ holds).

Formulas $\alpha_{T_0}$, $\alpha_{T_1}$, $\alpha_{T_{\ttB}}$ and $\alpha_{T_{\vdash}}$ are defined as follows, assuming that $\bar y = (y_1, \ldots, y_k)$:
\begin{align*}
\alpha_{T_0}(\bar{t},\bar{x},T_0,T_1,T_{\ttB},T_{\vdash},&h,\hat{h},s_{q_0},\ldots,s_{q_{\ell}},out) = \\ 
&(\first(\bar{t}) \wedge \exists\bar{y}(\first(y_1,\ldots,y_{k-2})\wedge\neg E(y_{k-1},y_k) \wedge \successor(\bar{y},\bar{x}) ))+ \\
&\sa{\bar{t}'}(\successor(\bar{t}',\bar{t}) \mult \hat{h}(\bar{t}',\bar{x}) \mult T_0(\bar{t}',\bar{x})) + \\
&\bigplus_{\delta(q,a) = (q',0,op,v)}\sa{\bar{t}'}(\successor(\bar{t}',\bar{t}) \mult h(\bar{t}',\bar{x}) \mult T_a(\bar{t}',\bar{x}) \mult s_{q}(\bar{t}')),\\
\alpha_{T_1}(\bar{t},\bar{x},T_0,T_1,T_{\ttB},T_{\vdash},&h,\hat{h},s_{q_0},\ldots,s_{q_{\ell}},out) = \\
&(\first(\bar{t}) \wedge \exists\bar{y}(\first(y_1,\ldots,y_{k-2})\wedge E(y_{k-1},y_k) \wedge \successor(\bar{y},\bar{x}) ))+ \\
&\sa{\bar{t}'}(\successor(\bar{t}',\bar{t}) \mult \hat{h}(\bar{t}',\bar{x}) \mult T_1(\bar{t}',\bar{x})) + \\
&\bigplus_{\delta(q,a) = (q',1,op,v)}\sa{\bar{t}'}(\successor(\bar{t}',\bar{t}) \mult h(\bar{t}',\bar{x}) \mult T_a(\bar{t}',\bar{x}) \mult s_{q}(\bar{t}')),\\
\alpha_{T_{\ttB}}(\bar{t},\bar{x},T_0,T_1,T_{\ttB},T_{\vdash},&h,\hat{h},s_{q_0},\ldots,s_{q_{\ell}},out) = \\
&(\first(\bar{t})\wedge\exists\bar{y}\exists\bar{y}'(\first(y_1,\ldots,y_{k-2})\wedge\last(y_{k-1},y_k)\wedge\successor(\bar{y},\bar{y}')\wedge\bar{y}' < \bar{x})) + \\
&\sa{\bar{t}'}(\successor(\bar{t}',\bar{t}) \mult \hat{h}(\bar{t}',\bar{x}) \mult T_{\ttB}(\bar{t}',\bar{x})) + \\
&\bigplus_{\delta(q,a) = (q',{\ttB},op,v)}\sa{\bar{t}'}(\successor(\bar{t}',\bar{t}) \mult h(\bar{t}',\bar{x}) \mult T_a(\bar{t}',\bar{x}) \mult s_{q}(\bar{t}')),\\
\alpha_{T_{\vdash}}(\bar{t},\bar{x},T_0,T_1,T_{\ttB},T_{\vdash},&h,\hat{h},s_{q_0},\ldots,s_{q_{\ell}},out) = \\
&(\first(\bar{t}) \wedge \first(\bar{x})) +
\sa{\bar{t}'}(\successor(\bar{t}',\bar{t}) \mult \hat{h}(\bar{t}',\bar{x}) \mult T_{\vdash}(\bar{t}',\bar{x})) + \\
&\bigplus_{\delta(q,a) = (q',\vdash,op,v)}\sa{\bar{t}'}(\successor(\bar{t}',\bar{t}) \mult h(\bar{t}',\bar{x}) \mult T_a(\bar{t}',\bar{x}) \mult s_{q}(\bar{t}')).
\end{align*}
Formulas $\alpha_{h}$ and $\alpha_{\hat{h}}$ are defined as:
\begin{align*}
\alpha_{h}(&\bar{t},\bar{x},T_0,T_1,T_{\ttB},T_{\vdash},h,\hat{h},s_{q_0},\ldots,s_{q_{\ell}},out) = \\
& (\first(\bar{t}) \wedge \successor(\bar{t},\bar{x})) + \\
&\bigplus_{\delta(q,a) = (q',b,\leftarrow,v)}\sa{\bar{t}'}\sa{\bar{x}'}(\successor(\bar{t}',\bar{t}) \mult \successor(\bar{x},\bar{x}')\mult T_a(\bar{t}',\bar{x}') \mult h(\bar{t}',\bar{x}') \mult s_{q}(\bar{t}')) + \\
&\bigplus_{\delta(q,a) = (q',b,\rightarrow,v)}\sa{\bar{t}'}\sa{\bar{x}'}(\successor(\bar{t}',\bar{t}) \mult \successor(\bar{x}',\bar{x})\mult T_a(\bar{t}',\bar{x}') \mult h(\bar{t}',\bar{x}') \mult s_{q}(\bar{t}')),\\
\alpha_{\hat{h}}(&\bar{t},\bar{x},T_0,T_1,T_{\ttB},T_{\vdash},h,\hat{h},s_{q_0},\ldots,s_{q_{\ell}},out) = \\
& (\first(\bar{t}) \wedge \neg\successor(\bar{t},\bar{x})) + \\
&\bigplus_{\delta(q,a) = (q',b,\leftarrow,v)}\sa{\bar{t}'}\sa{\bar{x}'}\sa{\bar{x}''}(\successor(\bar{t}',\bar{t}) \mult T_a(\bar{t}',\bar{x}')  \mult h(\bar{t}',\bar{x}') \mult s_{q}(\bar{t}') \mult \successor(\bar{x}'',\bar{x}') \mult (\bar{x} \neq \bar{x}'')) + \\
&\bigplus_{\delta(q,a) = (q',b,\rightarrow,v)}\sa{\bar{t}'}\sa{\bar{x}'}\sa{\bar{x}''}(\successor(\bar{t}',\bar{t}) \mult T_a(\bar{t}',\bar{x}')  \mult h(\bar{t}',\bar{x}') \mult s_{q}(\bar{t}') \mult \successor(\bar{x}',\bar{x}'') \mult (\bar{x} \neq \bar{x}'')).
\end{align*}
Formula $\alpha_{q_0}$ is defined as:
\begin{align*}
\alpha_{q_0}(\bar{t},T_0,T_1,T_{\ttB},T_{\vdash},&h,\hat{h},s_{q_0},\ldots,s_{q_{\ell}},out) = \first(\bar{t}) + \\
&\bigplus_{\delta(q,a) = (q_0,b,op,v)}\sa{\bar{t}'}\sa{\bar{x}'}(\successor(\bar{t}',\bar{t}) \mult T_a(\bar{t}',\bar{x}') \mult h(\bar{t}',\bar{x}') \mult s_{q}(\bar{t}')).
\end{align*}
Moreover, for every $i \in \{1, \ldots, \ell\}$, formula $\alpha_{q_i}$ is defined as:
\begin{align*}
\alpha_{q_i}(\bar{t},T_0,T_1,T_{\ttB},T_{\vdash},&h,\hat{h},s_{q_0},\ldots,s_{q_{\ell}},out) = \\ &\bigplus_{\delta(q,a) = (q_i,b,op,v)}\sa{\bar{t}'}\sa{\bar{x}'}(\successor(\bar{t}',\bar{t}) \mult T_a(\bar{t}',\bar{x}') \mult h(\bar{t}',\bar{x}') \mult s_{q}(\bar{t}')).
\end{align*}
Finally, formula $\alpha_{out}$ is defined as:
\begin{align*}
\alpha_{out}(\bar{t},&T_0,T_1,T_{\ttB},T_{\vdash},h,\hat{h},s_{q_0},\ldots,s_{q_{\ell}},out) =\\
&\bigplus_{\delta(q,a) = (q',b,op,0)}\sa{\bar{t}'}\sa{\bar{x}'}(\successor(\bar{t}',\bar{t}) \mult h(\bar{t}',\bar{x}') \mult T_a(\bar{t}',\bar{x}') \mult s_{q}(\bar{t}') \mult 2\mult out(\bar{t}')) + \\
&\bigplus_{\delta(q,a) = (q',b,op,1)}\sa{\bar{t}'}\sa{\bar{x}'}(\successor(\bar{t}',\bar{t}) \mult h(\bar{t}',\bar{x}') \mult T_a(\bar{t}',\bar{x}') \mult s_{q}(\bar{t}') \mult (2\mult out(\bar{t}')+1)) + \\ 
&\bigplus_{\delta(q,a) = (q',b,op,\emptyset)}\sa{\bar{t}'}\sa{\bar{x}'}(\successor(\bar{t}',\bar{t}) \mult h(\bar{t}',\bar{x}') \mult T_a(\bar{t}',\bar{x}') \mult s_{q}(\bar{t}') \mult out(\bar{t}')).
\end{align*}
Clearly, at each iteration of the LSFP operator, the tuple $\bar{t}$ represents the step the machine is currently in. From the construction of the formula, and since the machine is deterministic, it can be seen that in each function $g\in\{T_0,T_1,T_{\ttB},T_{\vdash},h,\hat{h}\}$, at the $\bar{a}$-th iteration of the LSFP operator, it holds that $g(\bar{a},\bar{b}) \leq 1$ for each $\bar{b}\in A^k$, that $g(\bar{a}+1,\bar{b}) = 0$ for each $\bar{b}\in A^k$. Also, at the $\bar{a}$-th iteration, $g(\bar{a}) \leq 1$ and $g(\bar{a}+1) = 0$ for each $g\in\{s_{q_1},\ldots,s_{q_{\ell}}\}$. From this, we have that at each iteration $\bar{a}$ of the operator, $out(\bar{a})$ is equal to either $2\cdot out(\bar{a}-1)$, $2\cdot out(\bar{a}-1) + 1$, or $out(\bar{a}-1)$, which represents precisely the value in the output tape at each step of $M$ running on input $\enc(\A)$. From this argument, it can be seen that $\sem{\alpha}(\A) = f(\enc(\A))$.

\medskip

To conclude the proof, we show that for each formula $\alpha$ in the previously defined extension of $\rqfo(\fo)$, there exists an equivalent formula confirming to the grammar of $\rqfo(\fo)$ defined in Section \ref{sec:beyond}. It suffices to consider a formula $\alpha$ of the form 
$$
\alpha(\bar{x}_1) = \clfp{f_1(\bar{x}_1): \alpha_1(\bar{x}_1,f_1,\ldots,f_n),\alpha_2(\bar{x}_2,f_1,\ldots,f_n),\ldots,\alpha_n(\bar{x}_n,f_1,\ldots,f_n)},
$$
and show an equivalent formula defined by a LSFP operator which uses one formula less in its definition.

We construct the equivalent formula as follows. We use a new function symbol $f$ with arity $\length{\bar{x}_1} + \length{\bar{x}_2}$. For every $i \in \{1,\ldots,n\}$, let $\alpha_i'$ be the formula obtained by performing the following replacements in $\alpha_i$:
\begin{align*}
f_1(\bar{y}_1) &\text{ is replaced by } \sa{\bar{y}_2}f(\bar{y}_1,\bar{y}_2)\mult[\first(\bar{y}_1)\mult\last(\bar{y}_2)  \add (\neg\first(\bar{y}_1))\mult\first(\bar{y}_2)], \\
f_2(\bar{y}_2) &\text{ is replaced by } \sa{\bar{y}_1}f(\bar{y}_1,\bar{y}_2)\mult[\first(\bar{y}_1)\mult\first(\bar{y}_2)  \add \last(\bar{y}_1)\mult(\neg\first(\bar{y}_2))].
\end{align*}
Moreover, let $\beta$ be a formula defined as:
\begin{align*}
\beta(\bar{x}_1,\bar{x}_2) = \,&\alpha_1'(\bar{x}_1)\mult(\first(\bar{x}_1)\mult\last(\bar{x}_2)  \add (\neg\first(\bar{x}_1))\mult\first(\bar{x}_2)) \add \\ &\alpha_2'(\bar{x}_2)\mult(\first(\bar{x}_1)\mult\first(\bar{x}_2)  \add \last(\bar{x}_1)\mult(\neg\first(\bar{x}_2))).
\end{align*}
It can be seen that all values of $f_1$, besides the first one, are stored in the first assignment of $\bar{x}_2$, while the first value of $f_1$ is stored in the last assignment of $\bar{x}_2$. Moreover, all values of $f_2$, besides the first one, are stored in the last assignment of $\bar{x}_1$, while the first value of $f_2$ is stored in the first assignment of $\bar{x}_1$. 
%This schema is depicted in Table \ref{table-clfp}. 
We use formula $\beta$ to define the following formula:
\begin{align*}
\alpha'(\bar{x}_1) = \,&\sa{\bar{x}_2}\clfp{f(\bar{x}_1,\bar{x}_2): \beta(\bar{x}_1,\bar{x}_2,f,f_3,\ldots,f_n),\\
&\alpha'_3(\bar{x}_3,f,f_3,\ldots,f_n),\ldots,\alpha'_n(\bar{x}_n,f,f_3,\ldots,f_n)}\mult(\first(\bar{x}_1)\mult\last(\bar{x}_2)  \add (\neg\first(\bar{x}_1))\mult\first(\bar{x}_2))
\end{align*}

It is not difficult to see that $\alpha'(\bar{x}_1)$ is equivalent to $\alpha(\bar{x}_1)$, which concludes the proof.
\qed
Our last goal in this section is to use the new characterization of $\fp$ to explore classes below it.
It was shown in \cite{I86,I88} that $\fo$ extended with a transitive closure operator captures $\nlog$. 
Inspired by this work, we show that a restricted version of $\rqfo$ can be used to capture $\shl$, the counting version of $\nlog$. 
Specifically, we use $\rqfo$ to define an operator for counting the number of paths in a directed graph, which is what is needed to capture~$\shl$.

Given a relational signature $\R$, the set of transitive $\qfo$ formulae ($\tqfo$-formulae) is defined as an extension of $\qfo$ with the formula $[\pth \psi(\bar{x},\bar{y})]$, where $\psi(\x, \y)$ is an $\fo$-formula over $\R$, and $\bar{x} = (x_1, \ldots, x_k)$, $\bar{y} = (y_1, \ldots, y_k)$ are tuples of pairwise distinct first-order variables. The semantics of $[\pth \psi(\bar{x},\bar{y})]$ can easily be defined in terms of $\rqfo(\fo)$ as follows. 
Given an $\R$-structure $\A$ with domain $A$, define a (directed) graph $\cG_{\psi}(\A) = (N,E)$ such that $N = A^k$ and for every pair $\bar b, \bar c \in N$, it holds that $(\bar b, \bar c) \in E$ if, and only if, $\A \models \psi(\bar b, \bar c)$.
Similar than for Example~\ref{ex:count-path}, we can count the paths of length at most $|A^k|$ in $ \cG_{\psi}(\A)$ with the formula $\beta_{\psi(\bar{x},\bar{y})}(\x, \y, \t, g)$:
\begin{align*}
(\psi(\bar{x},\bar{y}) + \sa{\z} g(\x,\z,\t)\cdot \psi(\z,\y)) \cdot \varphi_{\text{\rm first-lex}}(\t) \ +
\sa{\t'} \varphi_{\text{succ-lex}}(\t',\t) \cdot \left(\sa{\x'} \sa{\y'} g(\x',\y',\t') \right),
\end{align*}
where $\varphi_{\text{\rm first-lex}}$ and $\varphi_{\text{succ-lex}}$ are $\fo$-formulae defining the first and successor predicates over tuples in $A^k$, following the lexicographic order induced by~$<$.
Then the semantics of the path operator can be defined by using the following definition of $[\pth \psi(\bar{x},\bar{y})]$ in $\rqfo$:
\begin{eqnarray*}
[\pth \psi(\bar{x}, \bar{y})] & := & \sa{\t} (\varphi_{\text{\rm first}}(\t) \cdot \clfp{\beta_{\psi(\bar{x},\bar{y})}(\x,\y,\t,g)}).
\end{eqnarray*}
In other words, $\sem{[\pth \psi(\bar{x}, \bar{y})]}(\A,v)$ counts the number of paths from $v(\bar x)$ to $v(\bar y)$ in the graph $\cG_{\psi}(\A)$ whose length is at most~$|A^k|$.
As mentioned before, the operator for counting paths is exactly what we need to capture $\shl$.
\begin{thm} \label{tqfo-shl}
	$\tqfo(\fo)$ captures $\shl$ over ordered structures.
\end{thm}
\proof
%!TEX root = main.tex

%&$\boldsymbol{\tqfo(\fo)}$ {\bf can be computed in} $\boldsymbol{\shl.}$
First, we show that every formula in $\tqfo(\fo)$ defines a function that is in $\shl$.
Let $\R$ be a relational signature and $\alpha$ a formula over $\R$ in $\tqfo(\fo)$. Next we construct a logarithmic-space nondeterministic Turing Machine $M_{\alpha}$ that on input $(\enc(\A),v)$, where $\A$ is an $\R$-structure and $v$ is a first-order assignment for $\A$, has $\sem{\alpha}(\A,v)$ accepting runs (so that we can conclude that the function defined by $\alpha$ is in $\shl$). Suppose that the domain of $\A$ is $A = \{1,\ldots,n\}$. The TM $M_{\alpha}$ needs $\ell \mult\log_2(n)$ bits of memory to store the first-order variables occurring in $\alpha$, where $\ell$ is the number of variables occurring in this formula (which is the same as the number of variables in the domain of $v$). If $\alpha = \varphi$, where $\varphi$ is an $\fo$-formula, then we check if $(\A,v)\models\varphi$ in deterministic logarithmic space, and accept if and only if this condition holds. If $\alpha = s$, where $s$ is a fixed natural number, then we generate $s$ possible runs and accept in all of them. If $\alpha = (\alpha_1 + \alpha_2)$, we simulate $M_{\alpha_1}$ and $M_{\alpha_2}$ on separate branches. If $\alpha = (\alpha_1\mult\alpha_2)$, we simulate $M_{\alpha_1}$ and if it accepts, then instead of accepting we simulate $M_{\alpha_2}$. If $\alpha = \sa{x}\beta$, for each $a\in A$ we generate a different run where we simulate $M_{\beta}$ with input $v[a/x]$. If $\alpha = \pa{x}\beta$, we simulate $M_{\beta}$ with input $v[1/n]$, and on each accepting run, instead of accepting we replace the assignment of $x$ to 2, to simulate $M_{\beta}$ with input $v[2/x]$, and so on. If $\alpha = [\pth \varphi(\bar{x},\bar{y})]$, where $\varphi$ is an $\fo$-formula, then we simulate the $\shl$ procedure that counts the number of paths of a given length from a source to a target node in an input graph (where the length is at most the number of nodes in the graph).
%This procedure starts by setting $\bar{a} = v(\bar{x})$. On each iteration, it nondeterministically chooses an assignment $\bar{b}$ for $\bar{x}$, and continues if $(\A,v)\models\varphi(\bar{a},\bar{b})$ where $\bar{a}$ is the previously chosen value for $\bar{x}$, and it rejects otherwise. If at any point we obtain that the current value for $\bar x$ is  $\bar{a}$ and $\bar a = v(\bar{y})$, we generate an accepting branch, and continue simulating the procedure in the current branch. We simulate $n^{\length{\bar{x}}}$ iterations of the procedure, and this generates exactly $\sem{[\pth \varphi(\bar{x},\bar{y})]}(\A,v)$ accepting branches. This ends the construction of the algorithm. Consider $f$ as the $\shl$ function associated to this procedure and we have that for each finite $\R$-structure $\A$: $f(\enc(\A)) = \sem{\alpha}(\A)$.

%\vspace{1em}
%$\boldsymbol{\shl}$ {\bf can be modelled in }$\boldsymbol{\tqfo(\fo).}$ 
Second, we show that every function in $\shl$ can be encoded by a formula in $\tqfo(\fo)$.
Let $f$ be a function in $\shl$ and $M$ a logarithmic-space nondeterministic  Turing Machine such that $\tma_M(\enc(\A)) = f(\enc(\A))$. We assume that $M$ has only one accepting state, and that no transition is defined for this state. Moreover, we assume that there exists only one accepting configuration. We make use of transitive closure logic ($\tc$) to simplify our proof~\cite{G07}. We have that $\tc$ captures $\nlog$\cite{I83}, so that there exists a formula $\varphi$ in $\tc$ such that $\A\models\varphi$ if and only if $M$ accepts $\enc(\A)$. This formula can be expressed as:
$$
\varphi = \exists\bar{u}\exists\bar{z}(\first(\bar{u})\wedge \psi_{\bf acc}(\bar{z})\wedge[{\bf tc}_{\bar{x},\bar{y}}\,\psi_{\bf next}(\bar{x},\bar{y})](\bar{u},\bar{z})),
$$
where $\psi_{\bf acc}(\bar{z})$ is an $\fo$-formula that indicates that $\bar{z}$ is an accepting configuration, and $\psi_{\bf next}(\bar{x},\bar{y})$ is an $\fo$-formula that indicates that $\bar{y}$ is a successor configuration of $\bar{x}$~\cite{G07}. Here, there is a one-to-one correspondence between configurations of $M$ and assignments to $\bar{z}$. As a consequence, given a structure $\A$ and a first-order assignment $v$ for $\A$, where $v(\bar{x})$ is the starting configuration and $v(\bar{y})$ is the sole accepting configuration, the value of $\sem{[\pth\psi_{\bf next}(\bar{x},\bar{y})]}(\A,v)$ is equal to $\tma_M(\enc(\A))$.
Therefore, the $\tqfo(\fo)$-formula
$
\alpha = \sa{\bar{u}}\sa{\bar{z}}(\first(\bar{u})\mult\psi_{\bf acc}(\bar{z})\mult[\pth \psi_{\bf next}(\bar{u},\bar{z})])
$
satisfies that $\sem{\alpha}(\A) = f(\enc(\A))$. This concludes the proof of the theorem.
\qed
%This last result perfectly illustrates the benefits of our logical framework for the development of descriptive complexity for counting complexity classes.  
%The distinction in the language between the Boolean and the quantitative level allows us to define operators at the later level that cannot be defined at the former. 
%As a example showing how fundamental this separation is, consider the issue of extending $\qfo(\fo)$ at the Boolean level in order to capture $\shl$. The natural alternative to do this is to use $\fo$ extended with a transitive closure operator, which is denoted by $\tc$. But then the problem is that for every language $L \in \nlog$, it holds that its characteristic function $\chi_L$ is in $\qfo(\tc)$, where $\chi_L(x) = 1$ if $x \in L$, and $\chi_L(x) = 0$ otherwise. Thus, if we assume that $\qfo(\tc)$ captures $\shl$ (over ordered structures), then we have that $\chi_L \in \nlog$ for every $L \in \nlog$. This implies that $\nlog = \ulog$,\footnote{A decision language $L$ is in $\ulog$ is there exists a logarithmic-space NTM $M$ accepting $L$ and satisfying that $\tma_M(x) = 1$ for every $x \in L$.} 
%and thus contradicts the widely believed assumption that $\ulog \subsetneq \nlog$.
This last result 
perfectly 
illustrates the benefits of our logical framework for the development of descriptive complexity for counting %functions.
complexity classes.  
The distinction in the language between the Boolean and the quantitative level allows us to define operators at the latter level that cannot be defined at the former. 
As an example showing how fundamental this separation is, consider the issue of extending $\qfo(\fo)$ at the Boolean level in order to capture $\shl$. The natural alternative to do this is to use $\fo$ extended with a transitive closure operator, which is denoted by $\tc$. But then the problem is that for every language $L \in \nlog$, it holds that its characteristic function $\chi_L$ is in $\qfo(\tc)$, where $\chi_L(x) = 1$ if $x \in L$, and $\chi_L(x) = 0$ otherwise. Thus, if we assume that $\qfo(\tc)$ captures $\shl$ (over ordered structures), then we have that $\chi_L \in \shl$ for every $L \in \nlog$. This would imply that $\nlog = \ulog$,\footnote{A decision language $L$ is in $\ulog$ is there exists a logarithmic-space NTM $M$ accepting $L$ and satisfying that $\tma_M(x) = 1$ for every $x \in L$.} solving an outstanding open problem \cite{Reinhardt97}.