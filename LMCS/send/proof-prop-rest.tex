We give this proof in three parts.
First, we show that $\E{1} \not\subseteq \QE{0}$. 
Towards a contradiction, let $\R = \{ < \}$ and suppose that there is a $\QE{0}$ formula $\alpha$ over $\R$ which is equivalent to the $\E{1}$ formula $\sa{x} \ex{y} (x < y)$. 
This is, for every finite $\R$-structure $\A$, $\sem{\alpha}(\A) = \size{\A} - 1$.

Suppose that $\alpha$ is in SNF, namely, $\alpha = \sum_{i = 1}^k \alpha_i$ for some fixed $k$. 
Since $\alpha$ is not null, consider some $\alpha_i$ that describes a non-null function. 
Let $\alpha_i = \sa{\bar{X}}\sa{\bar{x}}\varphi(\bar{X},\bar{x})$ where $\varphi$ is quantifier-free. 
Note that if $\length{\bar{X}} > 0$, then the function $\sem{\alpha}$ is in $\Omega(2^{\size{\A}})$, as it was proven in~\cite{SalujaST95}. 
Therefore, we have that $\alpha_i = \sa{\bar{x}}\varphi(\bar{x})$. 
We conclude our proof with the following claim.
\begin{clm}
	Let $\alpha = \sa{\bar{x}}\varphi(\bar{x})$	where $\varphi$ is quantifier free. 
	Then the function $\sem{\alpha}$ is either null, greater or equal to $n$, or is in $\Omega(n^2)$, where $n$ is the size of the input structure.
\end{clm}
\proof
	Note that each atomic sub-formula in $\varphi(\bar{x})$ is either $(x = y)$, $(x < y)$, $\top$ or a negation thereof, where $x,y\in\bar{x}$. 
	Suppose $\sem{\alpha}$ is not null and consider some $\R$-structure $\A$ such that $\sem{\alpha}(\A) > 0$. 
	Let $\bar{a}$ be an assignment to $\bar{x}$ such that $\A\models\varphi(\bar{a})$. 
	It can be seen that each assignment $\bar{a}'$ that has the same order over its variables\footnote{For example, if $\bar{a} = (a_1,a_2,a_3,a_4)$, such order may be $a_1 > a_3 = a_4 > a_2$, which has tree partitions.}	as $\bar{a}$ also satisfies $\A\models\varphi(\bar{a}')$. 
	If this order has $k$ partitions then $\binom{\size{\A}}{k}$ assignments for $\bar{x}$ satisfy this order, and therefore, $\sem{\alpha} \geq \binom{\size{\A}}{k}$. 
	If $k = 1$, then $\binom{\size{\A}}{k} = \size{\A}$, and if $k \geq 2$, then $\binom{\size{\A}}{k} \in \Omega(\size{\A}^2)$, which proves the claim.
\qed

Now we show that $\QE{0} \not\subseteq \E{1}$. 
In Theorem \ref{theo-pi1-pnf} we proved that there is no formula in $\loge{1}$-PNF equivalent to the formula $\alpha = 2$. 
Every formula in $\E{1}$ can be expressed in $\loge{1}$-PNF, which implies that $2 \in \QE{0}$ and $2 \not\in \E{1}$.

Finally, we prove that $\eqso(\loge{1})\subsetneq\eqso(\logu{1})$. 
For inclusion, let $\alpha$ be a formula in $\eqso(\loge{1})$. 
Suppose that it is in $\loge{1}$-SNF, namely, $\alpha = c + \sum_{i = 1}^{n}\alpha_i$. 
Let $\alpha_i = \sa{\bar{X}}\sa{\bar{x}}\ex{\bar{y}}\varphi_i(\bar{X},\bar{x},\bar{y})$, where $\varphi_i$ is quantifier-free for each $\alpha_i$. 
We use the same construction used in \cite{SalujaST95}, and we obtain that the formula $\ex{\bar{y}}\varphi_i(\bar{X},\bar{x},\bar{y})$ is equivalent to $\sa{\bar{y}}\,[\varphi_i(\bar{X},\bar{x},\bar{y}) \wedge \fa{\bar{y}'}(\varphi_i(\bar{X},\bar{x},\bar{y}')\to\bar{y}\leq\bar{y}')]$ for every assignment to $(\bar{X},\bar{x})$. 
Doing this replacement for each $\alpha_i$ renders an equivalent formula in $\eqso(\logu{1})$.

To prove that the inclusion is proper, consider the $\eqso(\logu{1})$ formula $\sa{x} \fa{y}(y = x)$. 
This formula defines the following function over each ordered structure $\A$:
$$
\sem{\alpha}(\A) = 
\begin{cases}
1 &\A \text{ has one element}\\
0 &\text{ otherwise}.
\end{cases}
$$
Suppose that there exists an equivalent formula $\alpha$ in $\eqso(\loge{1})$. 
Also, suppose that it is in $\L$-PNF, so $\alpha = \sum_{i = 1}^n\sa{\bar{X}}\sa{\bar{x}}\ex{\bar{y}}\varphi_i(\bar{X},\bar{x},\bar{y})$. 
Consider a structure $\A'$ with one element. 
We have that for some $i$, there exists an assignment $(\bar{B},\bar{b},\bar{a})$ for $(\bar{X},\bar{x},\bar{y})$ such that $\A' \models\varphi_i(\bar{B},\bar{b},\bar{a})$. 
Consider now the structure $\A''$ that is obtained by duplicating $\A'$, as we did for Theorem \ref{theo-pi1-pnf}. 
Note that $\A''\models\varphi_i(\bar{B},\bar{b},\bar{a})$, which implies that $\sem{\alpha}(\A' \uplus \A'') > 1$, which leads to a contradiction.