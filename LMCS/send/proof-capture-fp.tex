In this and the following proofs, we will reuse the symbol $<$ to denote the lexicographic order over same-sized tuples. 
Formally, for $\bar{x} = (x_1,\ldots,x_m)$ and $\bar{y} = (y_1,\ldots,y_m)$ we denote by $\bar{x} < \bar{y}$ the formula:
$$
\bigvee_{i = 1}^m \, \bigwedge_{j = 1}^{i-1} \, (x_j = y_j \wedge x_i < y_i).
$$
Similarly, we use $\bar{x} = \bar{y}$ to denote equality between tuples and $\bar{x} \leq \bar{y}$  to denote $\bar{x} < \bar{y} \vee \bar{x} = \bar{y}$.
%Furthermore, If $\bar{x} = (x_1,\ldots,x_m)$ ($\bar{X} = (X_1,\ldots,X_m)$) is a tuple of first-order (second-order) variables, we denote $\sa{\bar{x}}\alpha$ for $\sa{x_1}\cdots\sa{x_m}\alpha$ and $\sa{\bar{X}}\alpha$ for $\sa{X_1}\cdots\sa{X_m}\alpha$ for each $\qso$ formula $\alpha$.
We will also use some syntactic sugar in $\qso$ to simplify formulas. 
Specifically, we will use the {\em conditional count} symbol $(\varphi \mapsto \alpha)$ defined as $ (\varphi\cdot\alpha) + \neg\varphi$ for any Boolean formula $\varphi$ and any quantitative formula $\alpha$. 
Note that for each $\A \in \ostr[\R]$, and each first-order (second-order) assignment $v$ ($V$) over $\A$:
\[
\sem{(\varphi \mapsto \alpha)}(\A,v,V) = 
\begin{cases}
\sem{\alpha}(\A,v,V) &\text{if } (\A,v,V)\models\varphi,\\
1 &\text{otherwise}.
\end{cases}
\]
Furthermore, we use $\size{\A}$ to denote the size of an $\R$-structure $\A$.
Now we prove Theorem~\ref{theo:capture-fp}. 
For condition (1), recall that checking whether $\A\models\varphi$ for any $\lfp$ formula $\varphi$ can be done in deterministic polynomial time on the size of $\A$\cite{I83}. 
%Also, the constant function $s$ can be trivially simulated in $\fp$. 
Furthermore, it is easy to check that $\fp$ is closed under polynomial sum and multiplication. 
We conclude then that any formula in $\qfo(\lfp)$ can be computed in $\fp$.	
For condition~(2), let $\R$ be a signature, $f\in \fp$
and $\ell\in\nat$ such that $\log_2\left(f(\enc(\A)) \right) \leq \size{\A}^\ell$ for every $\A\in\ostr[\R]$ (i.e. $\size{\A}^\ell$ is an upper bound for the output size of $f$ over $\A$).
Consider the language:
\[
L = \{(\A,\bar{a})\mid \bar{a} \in A^l \text{ and the } \bar{a}\text{-th bit of }f(\enc(\A))\text{ is 1}\}.
\]
where $\bar{a}$ encodes a number by following the lexicographic order over $A^l$.
%Consider a procedure that receives $\enc(\A)$ and an assignment $\bar{a}$ to $\bar{x}$. Let $m$ be the position of $\bar{a}$ in the lexicographic order of the tuples in $A^{\ell}$. The procedure then computes the $m$-th bit of $f(\enc(\A))$, from least to most significant. 
Clearly, the language $L$ is in $\ptime$ and by \cite{I83} there exists a formula $\Phi(\bar{x})$ in $\lfp$ such that $\A\models\Phi(\bar{a})$ if, and only if, $(\A,\bar{a})\in L$. 
We use then the following formula to encode $f$:
$$
\alpha = \sa{\bar{x}} \Phi(\bar{x})\cdot \pa{\bar{y}}(\bar{y} < \bar{x}) \mapsto 2)$$
Note that the subformula $\pa{\bar{y}}(\bar{y} < \bar{x}) \mapsto 2$ takes the value $2^m$ if there exist $m$ tuples in $A^{\ell}$ that are smaller than $\bar{x}$. Adding these values for each $\bar{a}\in A^{\ell}$ gives exactly $f(\enc(\A))$. 
In other words, $\Phi(\bar{x})$ simulates the behavior of the $\fp$-machine and the formula $\alpha$ reconstructs the binary output bit by bit.
Then $\alpha$ is in $\qfo(\lfp)$ and $\sem{\alpha}(\A) = f(\enc(\A))$.