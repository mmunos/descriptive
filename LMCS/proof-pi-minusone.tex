Let $\LL$ be a fragment of $\fo$ that contains $\logu{1}$. Then we have that every function in $\U{1}$ is expressible in $\eqso(\LL)$. In particular, $\ctcnf \in \eqso(\LL)$. Suppose that $\eqso(\LL)$ is closed under subtraction by one. Then, the function $\ctcnf-1$, which counts the number of satisfying assignments of a 3-CNF formula minus one, is also in $\eqso(\LL)$. Note that $\eqso(\LL) \subseteq \eqso(\fo) = \shp$. We have that $\ctcnf$ is $\shp$-complete under parsimonious reductions\footnote{It can be easily verified that the standard reduction from SAT to 3-CNF (or 3-SAT) preserves the number of satisfying assignments}. Now, let $f$ be a function in $\shp$, and consider the nondeterministic polynomial-time procedure that on input $\enc(\A)$ computes the corresponding reduction to $\ctcnf$, name it $g(\enc(\A))$, and simulates the $\shp$ procedure for $\ctcnf-1$ on input $g(\enc(\A))$. We have that this is a $\shp$ procedure that computes $f-1$, from which we conclude that $\shp$ is closed under subtraction by one.