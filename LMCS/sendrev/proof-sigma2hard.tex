%!TEX root = main.tex

First we prove that $\shdhsat$ is in $\eqso(\ehorn)$. Recall that each instance of $\shdhsat$ is a disjunction of Horn formulae. Let $\R$ be a relational signature such that $\R = \{\pP(\cdot,\cdot), \pN(\cdot,\cdot), \pV(\cdot), \pNC(\cdot), \pD(\cdot,\cdot)\}$. Each symbol in this vocabulary is used to indicate the same as in Example \ref{ex-hornsat-esop1}, with the addition of $\pD(d,c)$ which indicates that $c$ is a clause in the formula $d$. Define $\psi$ as in Example \ref{ex-hornsat-esop1} such that $\sa{\pT} \sa{\pA} \psi(\pT,\pA)$
defines $\chsat$. In order to encode $\shdhsat$, we extend $\psi(\pT,\pA)$ by adding the information of $\pD(d,c)$ as follows:
\begin{align*}
\psi'(T,A) \ := \ \ex{d} \big[ \ & \fa{x} (\neg \pT(x) \vee \pV(x)) \ \wedge\\
& \fa{c} (\neg \pD(c,d)\vee \neg \textit{NC}(c) \vee  \ex{x} \neg \textit{A}(c,x)) \ \wedge\\
& \fa{c} \fa{x} (\neg \pD(c,d)\vee\neg \textit{P}(c,x) \vee \ex{y} \neg \textit{A}(c,y) \vee \textit{T}(x)) \ \wedge\\
& \fa{c} \fa{x} (\neg \pD(c,d)\vee\neg \textit{N}(c,x) \vee \textit{T}(x) \vee \neg \textit{A}(c,x)) \ \wedge\\
& \fa{c} \fa{x}  (\neg \pD(c,d)\vee\textit{A}(c,x) \vee \textit{N}(c,x)) \ \wedge\\
& \fa{c} \fa{x} (\neg \pD(c,d)\vee\textit{A}(c,x) \vee \neg\textit{T}(x)) \ \big].
\end{align*}
One can check that $\psi'(T,A)$ effectively defines $\shdhsat$ as for every disjunction of Horn formulae $\theta = \theta_1\vee\cdots\vee\theta_m$ encoded by an $\R$-structure $\A$, the number of satisfying assignments of $\theta$ is equal to $\sem{\sa{\pT} \sa{\pA} \psi'(\pT,\pA)}(\A)$.  Therefore, we conclude that $\shdhsat \in \eqso(\ehorn)$.

Next, we prove that $\shdhsat$ is hard for $\eqso(\ehorn)$ over any signature~$\R$ under parsimonious reductions. For each $\eqso(\ehorn)$ formula $\alpha$ over $\R$, we will define a polynomial-time function $g_{\alpha}$ that receives an $\R$-structure $\A$ and outputs an instance of $\shdhsat$ such that $\sem{\alpha}(\A) = \shdhsat(g_{\alpha}(\A))$. By Proposition~\ref{theo-pnf-snf}, we can assume that $\alpha$ is of the form:
$$
\alpha \ = \ \sum_{i = 1}^{m} \sa{\bar{X}_{i}}\sa{\bar{x}} \ex{\bar{y}} \bigwedge_{j = 1}^{n} \fa{\bar{z}} \varphi^i_j(\bar{X}_{i},\bar{x},\bar{y},\bar{z}),
$$
where each $\varphi^i_j$ is a Horn clause, and each $\bar{X}_{i}$ is a sequence of second-order variables.
Consider $\bar{X}$ as the union of all $\bar{X}_{i}$. We replace each of the $m$ summands in $\alpha$ with
$$
\sa{\bar{X}}\sa{\bar{x}} \ex{\bar{y}} \bigwedge_{j = 1}^{n} \fa{\bar{z}} \varphi^i_j(\bar{X}_{i},\bar{x},\bar{y},\bar{z})\wedge\bigwedge_{X\not\in\bar{X}_{i}}\!\!\fa{\bar{u}}X(\bar{u}),
$$
whose sum is equivalent to $\alpha$.
Now, consider a finite $\R$-structure $\A$ with domain $A$. 
The next transformation of $\alpha$ and $\A$ towards a disjunction of Horn-formulae is to expand each first-order quantifier (i.e. $\Sigma{\bar{x}}$,  $\exists\bar{y}$, and $\forall\bar{z}$) as we replace their variables with first-order constants. Specifically, we obtain the following formula which defines the same function as $\alpha$ and is of polynomial size with respect to $\A$:
$$
\alpha_{\A} \ = \ \sum_{i = 1}^{m} \sum_{\bar{a}\,\in A^{\length{\bar{x}}}} \sa{\bar{X}}\bigvee_{\bar{b}\,\in A^{\length{\bar{y}}}}\bigwedge_{j = 1}^n\bigwedge_{\bar{c}\,\in A^{\length{\bar{z}}}}\varphi^i_j(\bar{X}_i,\bar{a},\bar{b},\bar{c})\wedge \bigwedge_{X \not\in \bar{X}_i}\bigwedge_{\bar{e}\,\in A^{\arity(X)}}\!\!\!\!\!\!\! X(\bar{e}).
$$
\martin{reescribi esta formula}
Note that each first-order subformula in $\varphi^i_j(\bar{X}_i,\bar{a},\bar{b},\bar{c})$ has no free variables and, therefore, we can evaluate each of them in polynomial time and easily rewrite $\alpha_{\A}$ to an equivalent formula that does not have any first-order subformula. In other words, in polynomial time we can replace $\varphi^i_j$ with a disjunction of $\neg X_{\ell}$ and at most one $X_{\ell}$, evaluated on constants.
After simplifying, grouping and reordering the previous formula, we can obtain an equivalent formula $\alpha_{\A}'$ of the form:
$$
\alpha_{\A}' \ := \ \sum_{i = 1}^{m'}\sa{\bar{X}}\bigvee_{j = 1}^{n_1'}\bigwedge_{k = 1}^{n_2'}\psi^{i}_{j,k}(\bar{X})
$$
where every $\psi^{i}_{j,k}(\bar{X})$ is a disjunction of $\neg X_{\ell}$ and at most one $X_{\ell}$.
The reader can easily check that $\sem{\alpha}(\A) = \sem{\alpha_{\A}'}(\A)$. 

The idea for the rest of the proof is to show how to obtain $g_{\alpha}(\A)$, i.e. an instance of $\shdhsat$, from $\alpha_{\A}'$.
First, if $\alpha_{\A}'$ is equal to $\sa{\bar{X}}\bigvee_{j = 1}^{n_1'}\bigwedge_{k = 1}^{n_2'}\psi_{j,k}(\bar{X})$, then we can define $g_{\alpha}(\A)$ equal to the propositional formula $\bigvee_{j = 1}^{n_1'}\bigwedge_{k = 1}^{n_2'}\psi_{j,k}(\bar{X})$ over the propositional alphabet $\{X(\bar{e}) \mid X \in \bar{X} \text{ and } \bar{e}\in A^{\arity(X)} \}$ which has exactly $\sem{\alpha}(\A)$ satisfying assignments and is precisely a disjunction of Horn formulae.
Otherwise, if $m' > 1$ we can use $m'$ fresh new variables $t_1,\ldots,t_{m'}$ and define:
$$
g_{\alpha}(\A) \ := \ \bigvee_{i=1}^{m'} \, \bigvee_{j = 1}^{n_1'} \, \bigwedge_{k = 1}^{n_2'} \, \psi^i_{j,k}(\bar{X}) \wedge t_i \wedge \bigwedge_{\ell \neq i} \neg t_{\ell}
$$ 
over the propositional alphabet $\{X(\bar{e}) \mid X \in \bar{X} \text{ and } \bar{e}\in A^{\arity(X)} \} \cup \{t_1,\ldots,t_{m'}\}$.
Variables $t_1,\ldots,t_{m'}$ are used to have disjoint sets of propositional assignments for the different disjuncts of the outermost disjunction, which correspond to the summands in the original formula.
One can easily check that $g_{\alpha}(\A)$ is a disjunction of Horn formulae, and the number of satisfying assignments is exactly $\sem{\alpha}(\A)$. This covers all possible cases for $\alpha$, and the entire procedure takes polynomial time.
