%!TEX root = main.tex

The class $\shp$ was introduced in \cite{Valiant79} to prove that computing the permanent of a matrix, as defined in Example \ref{exa-perm}, is a $\shp$-complete problem.
As a consequence of this result many counting problems have been proved to be $\shp$-complete~\cite{V79b,arora2009computational}.
%\cite{V79b,PB83,P86,L86,BW91,HMRS98,DG00,BW05,DS12, PS13,PS14}.
Among them, problems having easy decision counterparts play a fundamental role, as a counting problem with a hard decision version is expected to be hard. Formally, the decision problem associated to a function $f\colon\Sigma^* \to \mathbb{N}$ is defined as $L_f = \{ x \in \Sigma^* \mid f(x) > 0 \}$, and $f$ is said to have an easy decision version if $L_f \in \ptime$. 
Many prominent examples satisfy this property, like computing the number of: perfect matchings of a bipartite graph ($\cpm$)~\cite{Valiant79}, satisfying assignments of a DNF ($\cdnf$)~\cite{DHK05,KL83} or Horn ($\chsat$)~\cite{V79b} propositional formula, among others.

Counting problems with easy decision versions play a fundamental role in the search for efficient approximation algorithms for functions in $\shp$. 
A fully-polynomial randomized approximation scheme (FPRAS) for a function $f\colon \Sigma^* \to \bbN$ is a randomized algorithm $\mathcal{A} \colon \Sigma^* \times (0,1) \to \bbN$ such that: (1) for every string $x \in \Sigma^*$ and real value $\varepsilon \in (0,1)$, the probability that $|f(x) - \mathcal{A}(x,\varepsilon)| \leq \varepsilon \cdot f(x)$ is at least $\frac{3}{4}$, and (2) the running time of $\mathcal{A}$ is polynomial in the size of $x$ and $1/\varepsilon$ \cite{KL83}. 
Notably, there exist $\shp$-complete functions that can be efficiently approximated as they admit FPRAS; for instance, there exist FPRAS for $\cdnf$~\cite{KL83} and $\cpm$~\cite{JSV04}. 
A key observation here is that if a function $f$ admits an FPRAS, then $L_f$ is in the randomized complexity class $\bpp$~\cite{G77}.
Hence, under the widely believed assumption that  $\np \not\subseteq \bpp$, we cannot hope for an FPRAS for a function in $\shp$ whose decision counterpart is $\np$-complete, and we have to concentrate on the class of counting problems with easy decision versions. That is, with decision versions in P.

The importance of the class of counting problems with easy decision counterparts has motivated the search for robust classes of functions in $\shp$ with this property \cite{PagourtzisZ06}. But the key question here is what should be considered a {\em robust} class. 
A first desirable condition has to do with the closure properties satisfied by the class, which is a common theme when studying function complexity classes \cite{OH93,FH08}. Analogously to the cases of $\ptime$ and $\np$, which are closed under intersection and union, we expect our class to be closed under multiplication and sum. For a more elaborated closure property, assume that $\textit{sat\_one}$ is a function that returns one plus the number of satisfying assignments of a propositional formula. Clearly $\textit{sat\_one}$ is a $\shp$-complete function whose decision counterpart $L_{\textit{sat\_one}}$ is trivial. But should $\textit{sat\_one}$ be part of a robust class of counting functions with easy decision versions? The key insight here is that if a function in $\shp$ has an easy decision counterpart $L$, then as $L \in \np$ we expect  to have a polynomial-time algorithm that verifies whether $x \in L$ by constructing witnesses for~$x$. 
%As an example of this considered all the counting functions mentioned in the previous paragraphs and, in particular, the decision counterpart of $\cdnf$ that can be solved by explicitly constructing in polynomial-time a satisfying assignment for an input propositional formula. 
Moreover, if such an algorithm for constructing witnesses exists, then we also expect to be able to manipulate such witnesses and in some cases to remove them. In other words, we expect a robust class $\CC$ of counting functions with easy decision versions to be closed under subtraction by one, that is, if $g \in \CC$, then the function $g \dotdiv 1$ should also be in $\CC$, where $(g \dotdiv 1)(x)$ is defined as $g(x) - 1$ if $g(x) \geq 1$, and as~$0$ otherwise. Notice that, unless $\ptime = \np$, no such class can contain the function $\textit{sat\_one}$ because $\textit{sat\_one} \dotdiv 1$ counts the number of satisfying assignments of a propositional formula.

A second desirable condition of robustness is the existence of natural complete problems~\cite{P94}. Special attention has to be paid here to the notion of reduction used for completeness. Notice that under the notion of Cook reduction, originally used in \cite{Valiant79}, the problems $\cdnf$ and $\csat$ are $\shp$-complete. However, $\cdnf$ has an easy decision counterpart and admits an FPRAS, while $\csat$ does not satisfy these conditions unless $\ptime = \np$. Hence a more strict notion of reduction has to be considered; in particular, the notion of parsimonious reduction (to be defined later) satisfies that if a function $f$ is parsimoniously reducible to a function $g$, then $L_g \in \ptime$ implies that $L_f \in \ptime$ and the existence of an FPRAS for $g$ implies the existence of a FPRAS for $f$. 

In this section, we use the framework developed in this paper to address the problem of defining a robust class of functions with easy decision versions. More specifically, we use the framework to introduce in Section \ref{sec-hier-shp} a syntactic hierarchy of counting complexity classes contained in $\shp$. Then this hierarchy is used in Section \ref{sec-clo} to define a class of functions with easy decision versions and good closure properties, and in Section \ref{sec-horn} to define a class of functions with easy decision versions and 
natural complete problems.


\subsection{The $\eqso(\fo)$-hierarchy inside $\shp$}
\label{sec-hier-shp}
%!TEX root = syntactic.tex

Inspired by the connection between $\shp$ and $\sfo$, a hierarchy of subclases of $\sfo$ was introduced in~\cite{SalujaST95} 
%studied subclasses of  $\sfo$ syntactically 
by restricting the alternation of quantifiers in Boolean formulas.
%, defining what we call 
Specifically, the \emph{$\sfo$-hierarchy} consists of the 
%they define 
the classes $\E{i}$ and $\U{i}$ for every $i \geq 0$, where $\E{i}$ (resp., $\U{i}$) is defined as $\sfo$ but restricting the formulas used to be in $\loge{i}$ (resp., $\logu{i}$).
%is there exists a for are defined where functions are defined by $\fo$-formulas in $\loge{i}$ and $\logu{i}$, respectively, 
%for every $i \geq 0$. 
By definition ,we have that $\U{0} = \E{0}$. Moreover, it is shown in~\cite{SalujaST95} that:
% this function classes defined a finite hierarchy of the form:
\[
\E{0} \; \subsetneq \; \E{1} \; \subsetneq \; \U{1} \; \subsetneq \; \E{2} \; \subsetneq \; \U{2} \; = \; \#\fo 
\]
In light of the framework introduced in this paper, natural extensions of these classes are obtained by considering 
%in light of the $\eqso$ logic. Specifically, we consider 
%the classes 
$\eqso(\loge{i})$ and $\eqso(\logu{i})$ for every $i \geq 0$, which form the \emph{$\eqso$-hierarchy}.
%, where the boolean logic is restricted to $\loge{i}$ and $\logu{i}$, respectively. 
%We denote each class by $\QE{i}$ and $\QU{i}$ for short. 
Clearly, we have that $\E{i} \subseteq \QE{i}$ and $\U{i} \subseteq \QU{i}$. Indeed, each formula $\varphi(\bar{X}, \bar{x})$ in $\E{i}$ is equivalent to the formula $\sa{\bar X} \sa{\bar x} \varphi(\bar{X}, \bar{x})$ in $\QE{i}$, and likewise for $\U{i}$ and $\QU{i}$.
But what is the exact relationship between these two hierarchies?
%Then it is left to know whether these containments are strict.
%between the sharp and quantitative classes is strict or not. 
%For this, 
To answer this question, we start by introducing two normal forms for $\eqso(\LL)$ that helps us to characterize the expressive power of this quantitative logic.
%study first whether a formula $\alpha$ in $\eqso(\LL)$ can be transformed into a 
%\emph{prenex 
%normal form where sum quantifiers are restricted to be at the beginning of the formula. 
%Formally, we say that 
A formula $\alpha$ in $\eqso(\LL)$ is in \emph{$\LL$-prenex normal form ($\LL$-PNF)} 
%(or just prenex normal form) 
if $\alpha$ is of the form
%\[
%\alpha := 
$\sa{\bar{X}} \sa{\bar{x}} \varphi(\bar{X}, \bar{x})$,
%\]
where $\bar{X}$ and $\bar{x}$ are sequences of zero or more second-order and first-order variables, respectively, and $\varphi(\bar{X}, \bar{x})$ is a formula in $\LL$. Notice that 
%a sentence in $\LL$ is in $\LL$-PNF, while 
a formula $\varphi(\bar{X}, \bar{x})$ in $\#\LL$ is equivalent to the formula $\sa{\bar X} \sa{\bar x} \varphi(\bar{X}, \bar{x})$ in $\LL$-PNF. 
%In particular, 
%%note that a 
%each Boolean formula is in 
%%$\Sigma$-
%prenex normal form.
Moreover, a formula $\alpha$ is in \emph{$\LL$-sum normal form ($\LL$-SNF)} if $\alpha$ is of the form $c + \Sigma_{i=1}^n \alpha_i$, where $c$ is a non-negative constant and each $\alpha_i$ is in 
%$\Sigma$-
$\LL$-PNF.
%prenex normal form.
%The following results shows that 
%%Next we show that 
%each formula in $\eqso(\LL)$ can be converted in sum normal form.
% but not always in prenex normal form.
\begin{theorem}\label{theo-pnf-snf}
Every formula in $\eqso(\LL)$ can be rewritten in $\LL$-SNF.
%sum normal form.
%such that there does not exist a formula in $\Sigma$-prenex normal form in $\QE{1}$ equivalent to $\alpha$.
\end{theorem}
%\martin{Creo que el resultado de que existe una formula en $\QE{1}$ que no se puede pasar a prenex está mas relacionado con el teorema siguiente que con este.}
%By the previous result, we know that there exists logics $\LL$ where no $\Sigma$-prenex normal form exists. 
%Then
Therefore, to unveil the relationship between the $\sfo$-hierarchy and the $\eqso$-hierarchy, we need to understand the boundary between PNF and SNF. We do this in the following theorem. 
%An interesting question at this point is when a formula $\alpha \in \eqso(\LL)$ can be converted in 
%%$\Sigma$-
%prenex normal form. 
%%Of course, this would depend on the expressibility of $\LL$.
%Interestingly, if $\LL$ contains $\logu{1}$, then it can be shown that $\alpha$ can be converted in
%%we can always convert any formula in $\Sigma$-
%%prenex 
%this normal form. 
\begin{theorem}\label{theo-pi1-pnf}
There exists a formula $\alpha$ in $\QE{1}$ that is not equivalent to any formula in $\Sigma_1$-PNF. 
%prenex normal form. 
On the other hand, if $\logu{1} \subseteq \LL$, then 
	%for 
	every formula in
	%$\alpha \in \eqso(\LL)$ 
	$\eqso(\LL)$ can be rewritten in $\LL$-PNF. 
	%there exists a formula $\beta \in \eqso(\LL)$ equivalent to $\alpha$ in $\Sigma$-
%	prenex normal form.
\end{theorem}

\begin{figure*}

\begin{center}
\begin{tabular}{ccc}
$\E{0}$ & $\subsetneq$ & 
\end{tabular}
\end{center}

\caption{The relationship between the $\sfo$-hierarchy and the $\eqso$-hierarchy.\label{fir-sfo-eqso}}
\end{figure*}


As our first result, we show that in terms of containment the $\eqso$-hierarchy behaves as the $\sfo$-hierarchy:
%$\eqso$ also defined a finite hierarchy similar than in~\cite{SalujaST95} that we called the $\eqso$-hierarchy.
\begin{proposition}
\begin{multline*}
\; \QE{0} \; \subsetneq \; \QE{1} \; \subsetneq \; \QU{1} \; \subsetneq \\ \QE{2} \; \subsetneq \; \QU{2} \; = \; \eqso(\fo)
\end{multline*}
\end{proposition}
As our second result, we establish precise connections between 
%A natural question at this point is what is 
%the connection between 
$\E{i}$ and $\U{i}$ and their corresponding classes $\QE{i}$ and $\QU{i}$. 






Theorems \ref{theo-pnf-snf} and \ref{theo-pi1-pnf} are instrumental in answering our question of what is the relationship between the $\sfo$-hierarchy and the $\eqso$-hierarchy. 
%The previous results gives the connection between the hierarchy in \cite{SalujaST95} and the $\eqso$-hierarchy. 
Indeed, if $\LL$ contains $\logu{1}$, then we have that $\sh{\LL}$ is equal to $\eqso(\LL)$ since each formula in $\eqso(\LL)$ can be converted in prefix normal form and, therefore, it is equivalent to a formula in $\sh{\LL}$. 
The following proposition summarizes these results, also including the cases of $\E{0}$ and $\E{1}$.
%Unfortunately, this is not the case for $\loge{0}$ and $\loge{1}$ as the following result shows.
\begin{proposition}
	The classes $\E{0}$ and $\E{1}$ are strictly contained in $\QE{0}$ and $\QE{1}$, respectively. Moreover, the classes $\U{1}$, $\E{2}$, and $\U{2}$ are equivalent with $\QU{1}$, $\QE{2}$, and $\QU{2}$, respectively.
\end{proposition}
The previous result shows that the classes $\QE{i}$ and $\QU{i}$ are more robust than the classes $\E{i}$ and $\U{i}$: they are closed under binary and sum quantifiers but the other not necessarily. 

Now, we study the complexity classes describe by this hierarchies. As the following result shows, $\eqso(\loge{0})$ defines only tractable counting functions and $\eqso(\loge{1})$ intractable counting functions but with an tractable decision problems. 
\begin{proposition} \label{prop:qe0-fp-qe1-totp-fptras}
All functions defined in $\eqso(\loge{0})$ and $\eqso(\loge{1})$ can be computed in $\fp$ and $\totp$, respectively. Furthermore, every function defined in $\eqso(\loge{1})$ has a FPTRAS.
\end{proposition}
Therefore, in terms of counting complexity, the $\eqso$-hierarchy behaves exactly the same as the $\#\fo$-hierarchy.

The next step is to study the closure properties of $\eqso$-hierarchy. 
An advantage of the $\eqso$-hierarchy is that, by its language syntax, all the classes are closed under addition and first and second order sum.
So, the first question is whether the multiplicative operators in $\qso$ can be defined in $\eqso(\LL)$. As the following result shows, if $\LL$ is closed under conjunction, then the binary product can be defined in  $\eqso(\LL)$.
\begin{theorem}\label{theo:binary-prod}
	If $\LL$ is closed under conjunction, then binary product can be defined in $\eqso(\LL)$.
\end{theorem}
The next question is whether the hierarchy is closed under subtraction. Formally, for any pair of functions $f,g$, we define $f - g$ as the function such that $(f - g)(\A) = f(\A)-g(\A)$ whenever $f(\A)>g(\A)$ and $0$ otherwise.
As the next result shows, all classes in the $\eqso$-hierarchy is not closed under subtraction unless ${\sc P} = {\sc NP}$
\begin{theorem} \label{sub-pnp}
If $\eqso(\loge{i})$ or $\eqso(\logu{i})$ is closed under subtraction for $i > 0$, then {\sc P} = {\sc NP}.
\end{theorem}
\cristian{Martin, el resultado que tienes en el apendice se generaliza trivialmente para todas las clases ya que todas contienen la clase $\eqso(\loge{0})$.}

By the previous result, we know that functions in the $\eqso$ hierarchy are unlikely to be closed under subtraction. Then, a natural restriction to this question is to ask whether these classes are closed under subtraction by one, namely, if $\CC$ is a class of functions and $f \in \CC$, is $f-1 \in \CC$ where $1$ is the constant function that outputs $1$ for every structure. 
We do not know $\E{1}$ is closed under subtraction by one. However, if we extend $\logex{1}$ with $\fo$ predicates we can show that this new fragment is closed under subtraction by one.
\begin{theorem} \label{sigmafo-minusone}
	$\eqso(\logex{1})$ is closed under substraction by one.
\end{theorem}


 



% We are interested in, for each $\fo$-fragment $\LL$, the biggest fragment of $\eqso$ that is contained in $\#\LL$.
%Let $\LL = \loge{0}$:
%\begin{theorem} \label{one-sigma-zero}
%	Positive constant functions are not expressible in $\E{0}$
%\end{theorem}
%\begin{corollary}
%	If a fragment of $\eqso(\loge{0})$ is contained in $\E{0}$, its grammar does not allow sole constants.
%\end{corollary}
%\begin{conjecture}
%	Sum is not expressible in $\E{0}$
%\end{conjecture}
%\begin{theorem} \label{mult-sigma-zero}
%	$\sqso(\loge{0})$ with binary product is contained in $\E{0}$.
%\end{theorem}
%\begin{theorem} \label{fo-prod-sigma-zero}
%	If an extension of $\sqso(\loge{0})$ is contained in $\E{0}$, its grammar does not allow first-order product.
%\end{theorem}


%For every logic $\LL$, we define an $\LL$-extended quantifier-free (QF) formula as follows:
%\begin{eqnarray*}
%	\varphi &::=& \alpha, \alpha \text{ is an $\LL$-formula} \ \mid \\
%	&& X_i(x_1,\dots,x_{a_i}), i\in\N \ \mid \ \\
%	&& (\neg \varphi) \ \mid \ (\varphi \wedge \varphi) \ \mid \ (\varphi \vee \varphi).
%\end{eqnarray*}
%
%We define syntactically the fragments $\logex{i}$ and $\logux{i}$ according to the following grammar:
%\begin{align*}
%\logex{0} = \logux{0} &::= \varphi , \varphi \mbox{ is an $\fo$-extended QF formula,} \\
%\logex{i+1} &::= \logux{i} \ \mid \ \exists x\, \logex{i+1}, \\
%\logux{i+1} &::= \logex{i} \ \mid \ \forall x\, \logux{i+1}.
%\end{align*}

%We see that many of the results in Saluja et. al. \cite{SalujaST95} for $\#\LL$ still apply in $\eqso(\LL)$ for a given fragment $\LL$:
%
%\begin{theorem} \label{eqso-sigma-zero-in-fp}
%	For every $\eqso(\loge{0})$ formula $\alpha$ over a signature $\R$, the function $f$ over $\R$ defined as $f(\enc(\A)) = \sem{\alpha}(\A)$ is in $\fp$.
%\end{theorem}
%
%\begin{theorem} \label{eqso-sigma-one-in-eqso-pi-one}
%	For every $\eqso(\loge{1})$ formula $\alpha$ over a signature $\R$ there exists a $\eqso(\logu{1})$ formula $\beta$ over $\R$ such that $\sem{\alpha}(\A) = \sem{\beta}(\A)$ for every $\A\in\ostr[\R]$.
%\end{theorem}
%
%
%The {\em decision problem} associated to a function $f$ is defined by the language $L_f = \{\A \in \str \mid f(\A) > 0\}$.
%
%\begin{theorem} \label{decisionptime}
%	The decision problem associated to a function in $\eqso(\logex{1})$ is in \textsc{P}.
%\end{theorem}

%For a given pair of functions $f,g$, we define $f \dotminus g$ as follows:
%\begin{eqnarray*}
%	(f \dotminus g)(\A) =
%	\begin{cases}
%		f(\A)-g(\A), & \text{if }f(\A)>g(\A) \\
%		0, & \text{if }f(\A) \leq g(\A).
%	\end{cases}
%\end{eqnarray*}
%for every $\L$-structure $\A \in \str$. A function class $\F$ is {\em closed under substraction} if for every pair of functions $f,g \in \F$, it holds that $f \dotminus g \in \F$.
%
%\begin{theorem} \label{sub-pnp}
%	If $\eqso(\loge{1})$ is closed under substraction, then {\sc P} = {\sc NP}.
%\end{theorem}
%
%\begin{theorem} \label{sigma1strict}
%	$\eqso(\loge{1}) \subsetneq \eqso(\logex{1})$
%\end{theorem}
%
%For a given function $f$, we define $f \dotminus 1$ as follows:
%\begin{eqnarray*}
%	f \dotminus 1(\A) =
%	\begin{cases}
%		f(\A)-1, & \text{if }f(\A) > 0 \\
%		0, & \text{if }f(\A) = 0.
%	\end{cases}
%\end{eqnarray*}
%for every $\L$-structure $\A \in \str$. A function class $\F$ is {\em closed under substraction by one} if for every function $f \in \F$, it holds that $f \dotminus 1 \in \F$.
%
%\begin{theorem} \label{sigmafo-minusone}
%	$\eqso(\logex{1})$ is closed under substraction by one.
%\end{theorem}
%
%\begin{theorem} \label{dnf-pars}
%	{\sc \#DNF} is hard for $\eqso(\loge{1})$ under parsimonious reductions. 
%\end{theorem}
%
%\begin{theorem} \label{nplusone-strict}
%	$\U{1}$ with $n$ open first-order variables is properly contained in $\U{1}$ with $n+1$ open first-order variables for $n\in\N$.  
%\end{theorem}

\subsection{Defining a class of functions with easy decision versions and good closure properties}
\label{sec-clo}
%!TEX root = main.tex

We use the \emph{$\eqso(\fo)$-hierarchy} to define syntactic classes of functions with good algorithmic and closure properties.  
But before 
%starting our search
doing this, we introduce a more strict notion of counting problem with easy decision version.
Recall that a function $f : \Sigma^* \to \mathbb{N}$ has an easy decision counterpart if $L_f = \{ x \in \Sigma^* \mid f(x) > 0 \}$ is a language in $\ptime$. As the goal of this section is to define a syntactic class of functions in $\shp$ with easy decision versions and good closure properties, we do not directly consider the semantic condition $L_f  \in \ptime$, but instead we consider a more restricted 
%class of functions that has a simple 
syntactic 
%definition. 
condition. More precisely, a function $f : \Sigma^* \to \mathbb{N}$ is said to be in the complexity class $\totp$~\cite{PagourtzisZ06} if there exists a  polynomial-time NTM $M$ such that $f(x) = \tmt_M(x) - 1$ for every $x \in \Sigma^*$, where $\tmt_M(x)$ is the total number of runs of $M$ with input $x$. Notice that one is subtracted from $\tmt_M(x)$ to allow for $f(x) = 0$. Besides, notice that $\totp \subseteq \shp$ and that $f \in \totp$ implies that $L_f \in \ptime$. 

The complexity class $\totp$ contains many important counting problems with easy decision counterparts, such as $\cpm$, $\cdnf$, and $\chsat$ among others~\cite{PagourtzisZ06}. Besides, $\totp$ has good closure properties as it is closed under sum, multiplication and subtraction by one. However, some functions in $\totp$ do not admit FPRAS under standard complexity-theoretical assumptions,\footnote{As an example consider the problem of counting the number of independent sets in a graph, and the widely believed assumption that $\np$ is not equal to the randomized complexity class $\rp$ (Randomized Polynomial-Time \cite{G77}). This counting problem is in $\totp$, and it is known that $\np = \rp$ if there exists an FPRAS for it \cite{DFJ02}.} and no natural complete problems are known for this class \cite{PagourtzisZ06}. Hence, we use the $\eqso(\fo)$-hierarchy to find restrictions of $\totp$ with good approximation and closure properties.

%The previous result shows that the classes $\QE{i}$ and $\QU{i}$ are more robust than the classes $\E{i}$ and $\U{i}$: they are closed under binary and sum quantifiers but the other not necessarily. 

It was proved in \cite{SalujaST95} that every function in $\E{1}$ admits an FPRAS. Besides, it can be proved that $\E{1} \subseteq \totp$. 
However, 
%we have to discard 
this class 
%since it i
is not closed under sum, and then it is not robust under basic closure properties. 
\begin{proposition}\label{prop-e1-nc}
There exist functions $f, g \in \E{1}$ such that $(f + g) \not\in \E{1}$.
\end{proposition}
To overcome this limitation, one can consider the class $\QE{1}$, which is closed under sum by definition. In fact, the following proposition shows that the same good properties as for $\E{1}$ hold for $\QE{1}$, together with the fact that 
%plus being 
it is closed under sum and multiplication.
\begin{proposition} \label{prop:qe0-fp-qe1-totp-fptras}
$\QE{1} \subseteq \totp$ and every function in $\QE{1}$ has an FPRAS. Moreover, $\QE{1}$ is closed under sum and multiplication.
\end{proposition}
%Moreover, from the following theorem one can conclude that $\QE{1}$ is closed under multiplication.
%\begin{theorem}\label{theo:binary-prod}
%	If $\LL$ is closed under conjunction, then $\eqso(\LL)$ is closed under multiplication.
%\end{theorem}
Hence, it only remains to prove that $\QE{1}$ is closed under subtraction by one. Unfortunately, it is not clear whether this property holds; in fact, we conjecture that it is not the case. Thus, we need to find an extension of $\QE{1}$ that keeps all the previous properties and is closed under subtraction by one. It is important to notice that $\shp$ is believed not to be closed under subtraction by one by some complexity-theoretical assumption.\footnote{A decision problem $L$ is in the randomized complexity class $\cspp$ if there exists a polynomial-time NTM $M$ such that for every $x \in L$ it holds that $\tma_M(x) - \tmr_M(x) = 2$, and for every $x \not\in L$ it holds that $\tma_M(x) = \tmr_M(x)$ \cite{OH93,FFK94}. It is believed that $\np \not\subseteq \cspp$, as there is no evidence that a TM with the property just described exists for the propositional satisfiability problem. However, if $\shp$ is closed under subtraction by one, then it holds that $\np \subseteq \cspp$ \cite{OH93}.} So, the following proposition rules out any logic that extends $\Pi_1$ for a possible extension of~$\QE{1}$ with the desired closure property.
\begin{proposition} \label{pi-minusone}
If $\Pi_1 \subseteq \LL \subseteq \fo$ and $\eqso(\LL)$ is closed under subtraction by one, then $\shp$ is closed under subtraction by~one. 
\end{proposition}
Therefore, the desired extension has to be achieved by allowing some local extensions to $\Sigma_1$. More precisely, we define $\logex{1}$ as $\Sigma_1$ but allowing atomic formulae over a signature $\R$ to be of the form either $u = v$ or $X(\bar u)$, where $X$ is a second-order variable, or $\varphi(\bar u)$, where $\varphi(\bar u)$ is a first-order formula over $\R$ (in particular, it does not mentioned any second-order variable). With this extension we obtain a class with the desired properties.
%find the promised class.
\begin{theorem}\label{sigmafo-minusone}
The class $\eqso(\logex{1})$ is closed under sum, multiplication and subtraction by one. Moreover, $\eqso(\logex{1}) \subseteq \totp$ and every function in $\eqso(\logex{1})$ has an FPRAS.
\end{theorem}

%A natural question at this point is whether the property of being closed under subtraction by one could be generalized to proper subtraction. More precisely, a function complexity class $\CC$ is said to be closed under subtraction if for every $f,g \in \CC$, it holds that $(f \dotdiv g) \in \CC$, where $(f \dotdiv g)(x)$ is defined as $f(x) - g(x)$ if $f(x) \geq g(x)$, and as $0$ otherwise. The following theorem shows that such generalization does not work for any of the function complexity classes considered in this section.
%\begin{theorem} \label{sub-pnp}
%If any of the classes $\E{1}$, $\QE{1}$ or $\eqso(\logex{1})$ is closed under subtraction, then $\ptime = \np$.
%\end{theorem}

%
%The next question is whether the hierarchy is closed under subtraction. Formally, for any pair of functions $f,g$, we define $f - g$ as the function such that $(f - g)(\A) = f(\A)-g(\A)$ whenever $f(\A)>g(\A)$ and $0$ otherwise.
%As the next result shows, all classes in the $\eqso$-hierarchy is not closed under subtraction unless ${\sc P} = {\sc NP}$
%\begin{theorem} \label{sub-pnp}
%If $\eqso(\loge{i})$ or $\eqso(\logu{i})$ is closed under subtraction for $i > 0$, then {\sc P} = {\sc NP}.
%\end{theorem}
%\cristian{Martin, el resultado que tienes en el apendice se generaliza trivialmente para todas las clases ya que todas contienen la clase $\eqso(\loge{0})$.}


%Now, we study the complexity classes describe by this hierarchies. As the following result shows, $\eqso(\loge{0})$ defines only tractable counting functions and $\eqso(\loge{1})$ intractable counting functions but with an tractable decision problems. 
%\begin{proposition} \label{prop:qe0-fp-qe1-totp-fptras}
%All functions defined in $\eqso(\loge{0})$ and $\eqso(\loge{1})$ can be computed in $\fp$ and $\totp$, respectively. Furthermore, every function defined in $\eqso(\loge{1})$ has a FPTRAS.
%\end{proposition}
%Therefore, in terms of counting complexity, the $\eqso$-hierarchy behaves exactly the same as the $\#\fo$-hierarchy.

%The next step is to study the closure properties of $\eqso$-hierarchy. 
%An advantage of the $\eqso$-hierarchy is that, by its language syntax, all the classes are closed under addition and first and second order sum.
%So, the first question is whether the multiplicative operators in $\qso$ can be defined in $\eqso(\LL)$. As the following result shows, if $\LL$ is closed under conjunction, then the binary product can be defined in  $\eqso(\LL)$.
%\begin{theorem}\label{theo:binary-prod}
%	If $\LL$ is closed under conjunction, then binary product can be defined in $\eqso(\LL)$.
%\end{theorem}


%By the previous result, we know that functions in the $\eqso$ hierarchy are unlikely to be closed under subtraction. Then, a natural restriction to this question is to ask whether these classes are closed under subtraction by one, namely, if $\CC$ is a class of functions and $f \in \CC$, is $f-1 \in \CC$ where $1$ is the constant function that outputs $1$ for every structure. 
%We do not know $\E{1}$ is closed under subtraction by one. However, if we extend $\logex{1}$ with $\fo$ predicates we can show that this new fragment is closed under subtraction by one.
%\begin{theorem} \label{sigmafo-minusone}
%	$\eqso(\logex{1})$ is closed under substraction by one.
%\end{theorem}



\subsection{Defining a class of functions with easy decision versions and natural complete problems}
\label{sec-horn}
%!TEX root = main.tex
\newcommand{\pP}{\textit{P}}
\newcommand{\pN}{\textit{N}}
\newcommand{\pV}{\textit{V}}
\newcommand{\pT}{\textit{T}}
\newcommand{\pA}{\textit{A}}
\newcommand{\pNC}{\textit{NC}}
\newcommand{\pD}{\textit{D}}




The goal of this section is to define a class of functions in $\shp$ with easy decision counterparts and natural complete problems. To this end, we consider the notion of parsimonious reduction. Formally, a function $f\colon\Sigma^* \to \N$ is parsimoniously reducible to a function $g\colon\Sigma^* \to \N$ if there exists a function $h\colon\Sigma^* \to \Sigma^*$ such that $h$ is computable in polynomial time and $f(x) = g(h(x))$ for every $x \in \Sigma^*$. As mentioned at the beginning of this section, if $f$ can be parsimoniously reduced to $g$, then $L_g \in \ptime$ implies that $L_f \in \ptime$ and the existence of an FPRAS for $g$ implies the existence of an FPRAS for $f$. 

In the previous section, we showed that the class $\eqso(\logex{1})$ has good closure and approximation properties. Unfortunately, it is not clear whether it admits a {\em natural} complete problem under parsimonious reductions, where {\em natural} means any of the counting problems defined in this section or any other well-known counting problem (not one specifically designed to be complete for the class). On the other hand, $\totp$ admits a natural complete problem under parsimonious reductions, which is the problem of counting the number of inputs accepted by a monotone circuit~\cite{BCPPZ17}. However, the notion of monotone circuit used in \cite{BCPPZ17} does not correspond with the usual notion of monotone circuit~\cite{GS90}, that is, circuits with AND and OR gates but without negation. In this sense, we still lack a class of functions in $\shp$ with easy decision counterparts and a complete problem that is well known and has been widely studied. In this section, we follow a different approach to find such a class,
%Hence, in this section we follow a different approach to find a class of functions in $\shp$ with easy decision counterparts and natural complete problems, 
which is inspired by the approach followed in \cite{G92} that uses a restriction of second-order logic to Horn clauses for capturing $\ptime$ (over ordered structures). The following example shows how our approach works.
%\marcelo{Cambie este parrafo para mencionar un resultado nuevo de Pagourtzis et al. donde muestran un problema completo para Totp, que es un poco mas natural}

\begin{exa} \label{ex-hornsat-esop1}
Let $\R = \{\pP(\cdot,\cdot), \pN(\cdot,\cdot), \pV(\cdot), \pNC(\cdot),<\}$. This vocabulary is used as follows to encode a Horn formula. A fact $\pP(c,x)$ indicates that propositional variable $x$ is a disjunct in a clause $c$, while $\pN(c,x)$ indicates that $\neg x$ is a disjunct in $c$. Furthermore, $\pV(x)$ holds if  $x$ is a propositional variable, and $\pNC(c)$ holds if $c$ is a clause containing only negative literals, that is, $c$ is of the form $(\neg x_1 \vee \cdots \vee \neg x_n)$.

To define $\chsat$, we consider an \so-formula $\varphi(\pT)$ over $\R$, where $\pT$ is a unary predicate, such that for every Horn formula $\theta$ encoded by an $\R$-structure $\A$, the number of satisfying assignments of $\theta$ is equal to $\sem{\sa{\pT} \varphi(\pT)}(\A)$. In particular, $\pT(x)$ holds if and only if $x$ is a propositional variable that is assigned value true.  More specifically, 
%$\varphi(\pT)$ is defined as follows:
\begin{align*}
\varphi(\pT) \; :=\;\;  & \fa{x} (\pT(x) \to \pV(x)) \ \wedge\\
& \fa{c}  (\pNC(c) \to \ex{x} (\pN(c,x) \wedge \neg \pT(x))) \ \wedge\\
& \fa{c} \fa{x} ([\pP(c,x) \wedge \fa{y} (\pN(c,y) \to \pT(y))] \to \pT(x)).
\end{align*}
%Given that $\uhorn$ is designed with the goal in mind of capturing $\chsat$, we expect $\varphi(\pT)$ to be a formula in $\uhorn$. However, if we rewrite it as a conjunction of clauses we obtain the following:
We can rewrite $\varphi(\pT)$ in the following way:
\begin{align*}
& \fa{x}  (\neg \pT(x) \vee \pV(x)) \ \wedge\\
& \fa{c}  (\neg \pNC(c) \vee \ex{x} (\pN(c,x) \wedge \neg \pT(x)))\ \wedge\\
& \fa{c} \fa{x}  (\neg \pP(c,x) \vee \ex{y} (\pN(c,y) \wedge \neg \pT(y)) \vee \pT(x)).
\end{align*}
%The resulting formula $\varphi(\pT)$ is not in $\uhorn$, but it can be easily transformed into a formula in this class  by introducing an auxiliary binary predicate $\pA$ defined as follows:
Moreover, by introducing an auxiliary predicate $\pA$ defined as 
\begin{align*}
\fa{c} \fa{x}  (\neg \pA(c,x) \leftrightarrow [\pN(c,x) \wedge \neg \pT(x)]),
\end{align*}
we can translate $\varphi(\pT)$ into the following equivalent formula:
\begin{align*}
\psi(\pT,\pA) \; := \;\;  & \fa{x} (\neg \pT(x) \vee \pV(x)) \ \wedge\\
& \fa{c} (\neg \textit{NC}(c) \vee \ex{x} \neg \textit{A}(c,x)) \ \wedge\\
& \fa{c} \fa{x}  (\neg \textit{P}(c,x) \vee \ex{y} \neg \textit{A}(c,y) \vee \textit{T}(x)) \ \wedge\\
& \fa{c} \fa{x} (\neg \textit{N}(c,x) \vee \textit{T}(x) \vee \neg \textit{A}(c,x)) \ \wedge \\
& \fa{c} \fa{x} (\textit{A}(c,x) \vee \textit{N}(c,x)) \ \wedge\\
& \fa{c} \fa{x} (\textit{A}(c,x) \vee \neg\textit{T}(x)).
\end{align*}
More precisely, we have that:
\begin{align*}
\sem{\sa{\pT} \varphi(\pT)}(\A) &= \sem{\sa{\pT} \sa{\pA} \psi(\pT,\pA)}(\A),
\end{align*}
 for every $\R$-structure $\A$ encoding a Horn formula. Therefore, the formula $\psi(\pT,\pA)$ also defines $\chsat$. More importantly, $\psi(\pT,\pA)$ resembles a conjunction of Horn clauses except for the use of negative literals of the form $\ex{v} \neg \textit{A}(u,v)$. \qed
\end{exa}
The previous example suggests that to define $\chsat$, we can use Horn formulae defined as follows. 
A positive literal is a formula of the form $X(\x)$, where $X$ is a second-order variable and $\x$ is a tuple of first-order variables, and a negative literal is a formula of the form $\ex{\v} \neg X(\u,\v)$, where $\u$ and $\v$ are tuples of first-order variables. Given a signature $\R$, a clause over $\R$ is a formula of the form $\fa{\x} (\varphi_1 \vee \cdots \vee \varphi_n)$, 
where each $\varphi_i$ ($1 \leq i \leq n$) is either a positive literal, a negative literal or an \fo-formula over $\R$.  A clause is said to be Horn if it contains at most one positive literal, and a formula is said to be Horn if it is a conjunction of Horn clauses. With this terminology, we define $\uhorn$ as the set of formulae $\psi$ such that $\psi$ is a Horn formula over a signature $\R$. 

As we have seen, we have that $\chsat \in \eqso(\uhorn)$. Moreover, one can show that $\eqso(\uhorn)$ forms a class of functions with easy decision counterparts, namely, $\eqso(\uhorn) \subseteq \totp$.
Thus, $\eqso(\uhorn)$ is a new alternative in our search for a class of functions in $\shp$ with easy decision counterparts and natural complete problems. Moreover, an even larger class for our search can be generated by extending the definition of $\uhorn$ with outermost existential quantification. 
Formally, a formula $\varphi$ is in $\ehorn$ if $\varphi$ is of the form $\ex{\bar x} \psi$ with $\psi$ a Horn formula. 

\begin{prop}\label{prop:ehorn-pe}
$\eqso(\ehorn) \subseteq \totp$.
\end{prop}
In this section, we identify a complete problem for $\eqso(\ehorn)$ under parsimonious reductions. Hence, to prove that $\eqso(\ehorn) \subseteq \totp$, it is enough to prove that such a problem is in $\totp$, as $\totp$ is closed under parsimonious reductions. We give this proof at the end of this section, after the complete problem has been identified.
%For the sake of simplicity, we postpone the proof of the previous proposition to after the proof of Theorem~\ref{sigma2hard}.

Interestingly, we have that both $\chsat$ and $\cdnf$ belong to $\eqso(\ehorn)$. 
An imperative question at this point is whether in the definitions of $\uhorn$ and $\ehorn$, it is necessary to allow negative literals of the form $\ex{\v} \neg X(\u,\v)$. Actually, this forces our Horn classes to be included in $\eqso(\logu{2})$ and not necessarily in $\eqso(\loge{2})$. The following result shows that this is indeed the case.

\begin{prop}\label{prop:hsat-not-sigma2}	
$\chsat \not\in \eqso(\loge{2})$.
\end{prop}
\proof
We use a similar proof to the one provided by the authors in \cite{SalujaST95} to separate the classes $\E{2}$ and $\U{2}$. Suppose that the statement is false, this is, $\chsat \in \eqso(\loge{2})$. We consider the signature $\R$ that we used as the encoding for a Horn formula (Example \ref{ex-hornsat-esop1}) and that the formula $\alpha \in \eqso(\loge{2})$ follows the encoding in the same way. From what we proved in Theorem \ref{theo-pnf-snf}, we have that every formula in $\eqso(\loge{2})$ can be rewritten in $\loge{2}$-PNF, so we assume that $\alpha$ is in this form. Let $\alpha = \sa{\bar{X}}\sa{\bar{x}}\exists\bar{y}\,\forall\bar{z}\,\varphi(\bar{X},\bar{x},\bar{y},\bar{z})$. Consider the following Horn formula $\Phi$:
$$
\Phi = p \wedge \bigwedge_{i = 1}^n (t_i \wedge p \to q) \wedge \neg q,
$$
where $n = \length{\bar{x}} + \length{\bar{y}} + 1$. Let $\A$ be the encoding of this formula. In the encoding, each variable appears as an element in the domain of $\A$. This formula is satisfiable, so $\sem{\alpha}(\A) \geq 1$. Let $(\bar{B},\bar{b},\bar{a})$ be an assignment to $(\bar{X},\bar{x},\bar{y})$ such that $\A\models\forall\bar{z}\,\varphi(\bar{B},\bar{b},\bar{a},\bar{z})$. Let $t_{\ell}$ be such that it does not appear in $\bar{b}$ or $\bar{a}$. Consider the induced substructure $\A'$ that is obtained by removing $t_{\ell}$ from $\A$ and $\bar{B}'$ as the subset of $\bar{B}$ obtained by deleting each appearance of $t_{\ell}$ in $\bar{B}$. We have that $\A'\models\forall\bar{z}\,\varphi(\bar{B},\bar{b},\bar{a},\bar{z})$. This is because each subformula of the form $\exists y \neg B_i$ is still true, and universal formulas are monotone over induced substructures. It follows that $\sem{\alpha}(\A') \geq 1$ which is not possible since $\A'$ encodes the formula
$$
\Phi' = p \wedge \bigwedge_{i = 1}^{\ell-1} (t_i \wedge p \to q) \wedge (p\to q) \wedge \bigwedge_{i = \ell+1}^{n} (t_i \wedge p \to q) \wedge \neg q,
$$
which is unsatisfiable. We arrive to a contradiction and we conclude that $\chsat$ is not in $\eqso(\ehorn)$.
\qed
%We conclude this section by showing 
Next we show that $\eqso(\ehorn)$ is the class we were looking for, as not only every function in $\eqso(\ehorn)$ has an easy decision counterpart, but also $\eqso(\ehorn)$ admits a natural complete problem under parsimonious reductions. More precisely, define 
$\shdhsat$ as the problem of counting the satisfying assignments of a formula $\Phi$ that is a disjunction of Horn formulae. Then we have that:

\begin{thm} \label{sigma2hard}
	$\shdhsat$ is $\eqso(\ehorn)$-complete under parsimonious reductions. 
\end{thm}
\proof
First we prove that $\shdhsat$ is in $\eqso(\ehorn)$. Recall that each instance of $\shdhsat$ is a disjunction of Horn formulas. Let $\R = \{\pP(\cdot,\cdot), \pN(\cdot,\cdot), \pV(\cdot), \pNC(\cdot), \pD(\cdot,\cdot)\}$. Each symbol in this vocabulary is used to indicate the same as in Example \ref{ex-hornsat-esop1}, with the addition of $\pD(d,c)$ which indicates that $c$ is a clause in the formula $d$. Recall that the formula
\begin{align*}
&\forall x \, (\neg \pT(x) \vee \pV(x)) \ \wedge\\
&\forall c \, (\neg \textit{NC}(c) \vee \exists x \, \neg \textit{A}(c,x)) \ \wedge\\
&\forall c \forall x \, (\neg \textit{P}(c,x) \vee \exists y \, \neg \textit{A}(c,y) \vee \textit{T}(x)) \ \wedge\\
&\forall c \forall x \, (\neg \textit{N}(c,x) \vee \textit{T}(x) \vee \neg \textit{A}(c,x)) \ \wedge\\
&\forall c \forall x \, (\textit{A}(c,x) \vee \textit{N}(c,x)) \ \wedge\\
&\forall c \forall x \, (\textit{A}(c,x) \vee \neg\textit{T}(x)).
\end{align*}
defines $\chsat$. We obtain the following formula $\psi(T,A)$ in $\ehorn$:
\begin{align*}
\exists d[&\forall x \, (\neg \pT(x) \vee \pV(x)) \ \wedge\\
&\forall c \, (\neg \pD(c,d)\vee \neg \textit{NC}(c) \vee \exists x \, \neg \textit{A}(c,x)) \ \wedge\\
&\forall c \forall x \, (\neg \pD(c,d)\vee\neg \textit{P}(c,x) \vee \exists y \, \neg \textit{A}(c,y) \vee \textit{T}(x)) \ \wedge\\
&\forall c \forall x \, (\neg \pD(c,d)\vee\neg \textit{N}(c,x) \vee \textit{T}(x) \vee \neg \textit{A}(c,x)) \ \wedge\\
&\forall c \forall x \, (\neg \pD(c,d)\vee\textit{A}(c,x) \vee \textit{N}(c,x)) \ \wedge\\
&\forall c \forall x \, (\neg \pD(c,d)\vee\textit{A}(c,x) \vee \neg\textit{T}(x))]
\end{align*}
effectively defines $\chsat$ as for every disjunction of Horn formulas $\theta = \theta_1\vee\cdots\vee\theta_m$ encoded by an $\R$-structure $\A$, the number of satisfying assignments of $\theta$ is equal to $\sem{\sa{\pT} \sa{\pA} \psi(\pT,\pA)}(\A)$.  Therefore, we conclude that $\shdhsat \in \eqso(\ehorn)$.

\vspace{1em}
We will now prove that $\shdhsat$ is hard for $\eqso$ over a signature $\R$ under parsimonious reductions. For each $\eqso(\ehorn)$ formula $\alpha$ over $\R$, we will define a polynomial-time procedure that computes a function $g_{\alpha}$. This function receives a finite $\R$-structure $\A$ and outputs an instance of $\shdhsat$ such that $\sem{\alpha}(\A) = \shdhsat(g_{\alpha}(\A))$. We suppose that $\alpha$ is in sum normal form and:
$$
\alpha = \sum_{i = 1}^{\text{\#clauses}} \sa{\bar{X}}\sa{\bar{x}}\exists\bar{y}\bigwedge_{j = 1}^{n}\forall\bar{z}\,\varphi^i_j(\bar{X},\bar{x},\bar{y},\bar{z}),
$$
where each $\varphi^i_j$ is a Horn clause.                                                                

Consider a finite $\R$-structure $\A$ with domain $A$. To simplify the proof, we extend our grammar to allow first-order constants. Consider each tuple $\bar{a}\in A^{\length{\bar{x}}}$, each $\bar{b}\in A^{\length{\bar{y}}}$ and each $\bar{c}\in A^{\length{\bar{z}}}$ as a tuple of first-order constants. The following formula defines the same function as $\alpha$:
$$
\sum_{i = 1}^{\#clauses} \sum_{\bar{a}\in A^{\length{\bar{x}}}} \sa{\bar{X}}\bigvee_{\bar{b}\in A^{\length{\bar{y}}}}\bigwedge_{j = 1}^{n}\bigwedge_{\bar{c}\in A^{\length{\bar{z}}}}\varphi^i_j(\bar{X},\bar{a},\bar{b},\bar{c}).
$$
Note that each $\fo$ formula over $(\bar{x},\bar{y},\bar{z})$ in each $\varphi^i_j$ has no free variables. Therefore, we can evaluate each of these in polynomial time and replace them by $\perp$ and $\top$ where it corresponds. Each $\varphi^i_j$ will be of the form $\perp \vee\, \chi^i_j(\bar{X})$ or $\top \vee \chi^i_j(\bar{X})$ where $\chi^i_j$ is a disjunction of $\neg X_{\ell}$'s and at most one $X_{\ell}$. The formulas of the form $\top \vee \chi^i_j(\bar{X})$ can be removed entirely, and the formulas of the form $\perp \vee\, \chi^i_j(\bar{X})$ can be replaced by $\chi^i_j(\bar{X})$. We obtain the formula
$$
\sum_{i = 1}^{m}\sa{\bar{X}}\bigvee_{j = 1}^{\#d}\bigwedge_{k = 1}^{\#c}\psi^{i}_{j,k}(\bar{X})
$$
where every $\psi^{i}_{j,k}(\bar{X})$ is a disjunction of $\neg X_{\ell}$'s and zero or one $X_{\ell}$.

Our idea for the rest of the proof is to define $g_{\alpha}$ for each $\alpha = \sa{\bar{X}}\bigvee_{j = 1}^{\#d}\bigwedge_{k = 1}^{\#c}\psi^{i}_{j,k}(\bar{X})$, for $\alpha = c$ and for $\alpha = \beta_1 + \cdots + \beta_m$ where each $\beta_i$ is in one of the two previous cases.

If $\alpha$ is equal to $\sa{\bar{X}}\bigvee_{j = 1}^{\#d}\bigwedge_{k = 1}^{\#c}\psi_{j,k}(\bar{X})$ where $\psi_{j,k}(\bar{X})$ is a disjunction of $\neg X_{\ell}$'s and zero or one $X_{\ell}$, then we obtain the {\bf propositional formula} $g_{\alpha}(\A) = \bigvee_{j = 1}^{\#d}\bigwedge_{k = 1}^{\#c}\psi_{j,k}(\bar{X})$ over the propositional alphabet $\{X(\bar{e}) \mid X \in \bar{X} \text{ and } \bar{e}\in A^{\arity(X)} \}$ which has exactly $\sem{\alpha}(\A)$ satisfying assignments and is precisely a disjunction of Horn formulas.

If $\alpha$ is equal to a constant $c$, then we define $g_{\alpha}(\A)$ as the following formula that has exactly $c$ satisfying assignments:
$$
g_{\alpha}(\A) = \bigvee_{i = 1}^{c}\neg t_1 \wedge \cdots \wedge \neg t_{i-1} \wedge t_i \wedge \neg t_{i+1} \wedge \cdots \wedge \neg p_c.
$$ 
If $\alpha = \beta_1 + \cdots + \beta_m$, let $g_{\beta_i}(\A) = \bigvee_{j = 1}^{\#d}\bigwedge_{k = 1}^{\#c}\theta^i_{j,k}$ for each $\beta_i$ where each $\theta^i_{j,k}$ is a Horn clause. Let $\Theta_i = g_{\beta_i}(\A)$. We rename the variables in each $\Theta_i$ so none of them are mentioned in any other $\Theta_j$. We add $m$ new variables $t_1,\ldots,t_m$ and we define:
\begin{align*}
g_{\alpha}(\A) = &\bigvee_{i = 1}^{\#d}(\bigwedge_{j = 1}^{\#c}\theta^1_{i,j} \wedge (\bigwedge\limits_{\substack{\text{each } t\\ \text{ mentioned in}\\ \Theta_2,\ldots,\Theta_{m}}}t) \wedge (t_1 \wedge \bigwedge_{\ell = 2}^{m} \neg t_{\ell})) \vee \\ 
&\bigvee_{i = 1}^{\#d}(\bigwedge_{j = 1}^{\#c}\theta^2_{i,j} \wedge (\bigwedge\limits_{\substack{\text{each $t$}\\ \text{ mentioned in}\\ \Theta_1,\Theta_3,\ldots,\Theta_{m}}}t) \wedge (t_2 \wedge \bigwedge\limits_{\substack{\ell = 1 \\ \ell \neq 2}}^{m} \neg t_{\ell})) \vee \cdots \vee\\ 
&\bigvee_{i = 1}^{\#d}(\bigwedge_{j = 1}^{\#c}\theta^m_{i,j} \wedge (\bigwedge\limits_{\substack{\text{each } t\\ \text{ mentioned in}\\ \Theta_2,\ldots,\Theta_{m-1}}}t) \wedge (t_m \wedge \bigwedge_{\ell = 1}^{m-1} \neg t_{\ell})).
\end{align*}
The formula is a disjunction of Horn formulas, and the number of satisfying assignments for this formula is exactly the sum of satisfying assignments for each $g_{\beta_i}(\A)$. This, at the same time, is equal to $\sem{\alpha}(\A)$. This covers all possible cases for $\alpha$, and the entire procedure takes polynomial time.
\qed
Now that we have a complete problem for $\eqso(\ehorn)$, we can provide a simple proof of Proposition~\ref{prop:ehorn-pe}.

\medskip

\noindent{\emph{Proof of Proposition~\ref{prop:ehorn-pe}.}}
%Let $\shdhsat$ be the problem of counting the number of satisfying assignments of a disjunction of Horn formulas. 
In~\cite{PagourtzisZ06}, Pagourtzis and Zachos gave a $\totp$ procedure that computes the number of satisfying assignments of a DNF formula. This procedure can be easily extended to receive Horn formulas, and furthermore, a disjunction of Horn formulas. Hence $\shdhsat$ is in $\totp$. As we saw in Theorem~\ref{sigma2hard}, $\shdhsat$ is complete for $\eqso(\ehorn)$ under parsimonious reductions. Let $\alpha$ be a formula in $\eqso(\ehorn)$ and let $g_{\alpha}$ be the reduction to $\shdhsat$. Then the $\totp$ procedure consist in computing $g_{\alpha}(\enc(\A))$ on input $\enc(\A)$ and then simulating the $\totp$ procedure for $\shdhsat$ on input $g_{\alpha}(\enc(\A))$. Therefore, we conclude that $\alpha$ defines a function in~$\totp$.
\qed

Finally, it is important to mention that from the previous proof one can easily derive that $\eqso(\ehorn) \equiv \#(\ehorn)$. Therefore, the framework in~\cite{SalujaST95} is enough for defining the class of problems in $\eqso(\ehorn)$.


