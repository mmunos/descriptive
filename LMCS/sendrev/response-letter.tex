\documentclass[a4paper]{article}

\usepackage{fullpage}
\usepackage{cite}
\usepackage{amsmath,amsfonts,amssymb,mathabx}
\usepackage{stmaryrd}
\usepackage[textwidth=2cm,textsize=small]{todonotes}
\usepackage{enumitem}
\usepackage{mathrsfs}

\usetikzlibrary{chains,fit,shapes}
\usetikzlibrary{arrows,positioning} 

\newcommand{\op}[1]{\operatorname{#1}}

% domains
\newcommand{\bbB}{\mathbb{B}}
\newcommand{\bbD}{\mathbb{D}}
\newcommand{\bbI}{\mathbb{I}}
\newcommand{\bbJ}{\mathbb{J}}
\newcommand{\bbN}{\mathbb{N}}
\newcommand{\bbNp}{\mathbb{N}_{>0}}
\newcommand{\bbNinf}{\mathbb{N}\cup\{\infty\}}
\newcommand{\bbNpinf}{\mathbb{N}_{>0}\cup\{\infty\}}
\newcommand{\bbZ}{\mathbb{Z}}
\newcommand{\bbP}{\mathbb{P}}
\newcommand{\bbQ}{\mathbb{Q}}
\newcommand{\bbQp}{\mathbb{Q}_{>0}}
\newcommand{\bbR}{\mathbb{R}}
\newcommand{\bbRp}{\mathbb{R}_{>0}}
\newcommand{\bbC}{\mathbb{C}}
\newcommand{\bbS}{\mathbb{S}}

% classes
\newcommand{\cA}{\mathcal{A}}
\newcommand{\cB}{\mathcal{B}}
%\newcommand{\cC}{\mathcal{C}}
\newcommand{\cD}{\mathcal{D}}
\newcommand{\cE}{\mathcal{E}}
\newcommand{\cF}{\mathcal{F}}
\newcommand{\cG}{\mathcal{G}}
\newcommand{\cH}{\mathcal{H}}
\newcommand{\cI}{\mathcal{I}}
\newcommand{\cJ}{\mathcal{J}}
\newcommand{\cK}{\mathcal{K}}
\newcommand{\cL}{\mathcal{L}}
\newcommand{\cM}{\mathcal{M}}
\newcommand{\cN}{\mathcal{N}}
\newcommand{\cO}{\mathcal{O}}
\newcommand{\cP}{\mathcal{P}}
\newcommand{\cQ}{\mathcal{Q}}
\newcommand{\cR}{\mathcal{R}}
\newcommand{\cS}{\mathcal{S}}
\newcommand{\cT}{\mathcal{T}}
\newcommand{\cU}{\mathcal{U}}
\newcommand{\cV}{\mathcal{V}}
\newcommand{\cX}{\mathcal{X}}
\newcommand{\cY}{\mathcal{Y}}
\newcommand{\cZ}{\mathcal{Z}}
\newcommand{\cW}{\mathcal{W}}

% languages
%%\newcommand{\sA}{\mathscr{A}}
%\newcommand{\sB}{\mathscr{B}}
%\newcommand{\sC}{\mathscr{C}}
%\newcommand{\sD}{\mathscr{D}}
%\newcommand{\sE}{\mathscr{E}}
%\newcommand{\sF}{\mathscr{F}}
%\newcommand{\sG}{\mathscr{G}}
%\newcommand{\sH}{\mathscr{H}}
%\newcommand{\sI}{\mathscr{I}}
%\newcommand{\sJ}{\mathscr{J}}
%\newcommand{\sK}{\mathscr{K}}
%\newcommand{\sL}{\mathscr{L}}
%\newcommand{\sM}{\mathscr{M}}
%\newcommand{\sN}{\mathscr{N}}
%\newcommand{\sO}{\mathscr{O}}
%\newcommand{\sP}{\mathscr{P}}
%\newcommand{\sQ}{\mathscr{Q}}
%\newcommand{\sR}{\mathscr{R}}
%\newcommand{\sS}{\mathscr{S}}
%\newcommand{\sT}{\mathscr{T}}
%\newcommand{\sU}{\mathscr{U}}
%\newcommand{\sV}{\mathscr{V}}
%\newcommand{\sX}{\mathscr{X}}
%\newcommand{\sY}{\mathscr{Y}}
%\newcommand{\sZ}{\mathscr{Z}}
%\newcommand{\sW}{\mathscr{W}}

% structures
\newcommand{\fA}{\mathfrak{A}}
\newcommand{\fB}{\mathfrak{B}}
\newcommand{\fC}{\mathfrak{C}}
\newcommand{\fD}{\mathfrak{D}}
\newcommand{\fE}{\mathfrak{E}}
\newcommand{\fF}{\mathfrak{F}}
\newcommand{\fG}{\mathfrak{G}}
\newcommand{\fH}{\mathfrak{H}}
\newcommand{\fI}{\mathfrak{I}}
\newcommand{\fJ}{\mathfrak{J}}
\newcommand{\fK}{\mathfrak{K}}
\newcommand{\fL}{\mathfrak{L}}
\newcommand{\fM}{\mathfrak{M}}
\newcommand{\fN}{\mathfrak{N}}
\newcommand{\fO}{\mathfrak{O}}
\newcommand{\fP}{\mathfrak{P}}
\newcommand{\fQ}{\mathfrak{Q}}
\newcommand{\fR}{\mathfrak{R}}
\newcommand{\fS}{\mathfrak{S}}
\newcommand{\fT}{\mathfrak{T}}
\newcommand{\fU}{\mathfrak{U}}
\newcommand{\fV}{\mathfrak{V}}
\newcommand{\fX}{\mathfrak{X}}
\newcommand{\fY}{\mathfrak{Y}}
\newcommand{\fZ}{\mathfrak{Z}}
\newcommand{\fW}{\mathfrak{W}}

% objects
\newcommand{\bA}{\mathbf{A}}
\newcommand{\bB}{\mathbf{B}}
\newcommand{\bC}{\mathbf{C}}
\newcommand{\bD}{\mathbf{D}}
\newcommand{\bE}{\mathbf{E}}
\newcommand{\bF}{\mathbf{F}}
\newcommand{\bG}{\mathbf{G}}
\newcommand{\bH}{\mathbf{H}}
\newcommand{\bI}{\mathbf{I}}
\newcommand{\bJ}{\mathbf{J}}
\newcommand{\bK}{\mathbf{K}}
\newcommand{\bL}{\mathbf{L}}
\newcommand{\bM}{\mathbf{M}}
\newcommand{\bN}{\mathbf{N}}
\newcommand{\bO}{\mathbf{O}}
\newcommand{\bP}{\mathbf{P}}
\newcommand{\bQ}{\mathbf{Q}}
\newcommand{\bR}{\mathbf{R}}
\newcommand{\bS}{\mathbf{S}}
\newcommand{\bT}{\mathbf{T}}
\newcommand{\bU}{\mathbf{U}}
\newcommand{\bV}{\mathbf{V}}
\newcommand{\bX}{\mathbf{X}}
\newcommand{\bY}{\mathbf{Y}}
\newcommand{\bZ}{\mathbf{Z}}
\newcommand{\bW}{\mathbf{W}}

% other objects
\newcommand{\tA}{\mathtt{A}}
\newcommand{\tB}{\mathtt{B}}
\newcommand{\tC}{\mathtt{C}}
\newcommand{\tD}{\mathtt{D}}
\newcommand{\tE}{\mathtt{E}}
\newcommand{\tF}{\mathtt{F}}
\newcommand{\tG}{\mathtt{G}}
\newcommand{\tH}{\mathtt{H}}
\newcommand{\tI}{\mathtt{I}}
\newcommand{\tJ}{\mathtt{J}}
\newcommand{\tK}{\mathtt{K}}
\newcommand{\tL}{\mathtt{L}}
\newcommand{\tM}{\mathtt{M}}
\newcommand{\tN}{\mathtt{N}}
\newcommand{\tO}{\mathtt{O}}
\newcommand{\tP}{\mathtt{P}}
\newcommand{\tQ}{\mathtt{Q}}
\newcommand{\tR}{\mathtt{R}}
\newcommand{\tS}{\mathtt{S}}
\newcommand{\tT}{\mathtt{T}}
\newcommand{\tU}{\mathtt{U}}
\newcommand{\tV}{\mathtt{V}}
\newcommand{\tX}{\mathtt{X}}
\newcommand{\tY}{\mathtt{Y}}
\newcommand{\tZ}{\mathtt{Z}}
\newcommand{\tW}{\mathtt{W}}


\newcommand\loge[1]{\Sigma_{#1}} %Existential logic
\newcommand\logu[1]{\Pi_{#1}} %Universal logic
\newcommand\logex[1]{\Sigma_{#1}\textsc{[FO]}} %Existential extended logic
\newcommand\logux[1]{\Pi_{#1}\textsc{[FO]}} %Universal extended logic
\newcommand\logeh[1]{\Sigma_{#1}\textsc{[FO]-Horn}} %Existential extended Horn logic
\newcommand\loguh[1]{\Pi_{#1}\textsc{[FO]-Horn}} %Universal extended Horn logic
\newcommand\ehorn{\Sigma_2\textsc{-Horn}} 
\newcommand\uhorn{\Pi_1\textsc{-Horn}} 
\newcommand\E[1]{\#\Sigma_{#1}} %Existential
\newcommand\U[1]{\#\Pi_{#1}} %Universal
\newcommand\sfo{\#\fo} %#FO
\newcommand\seso{\text{\sc \#($\eso$)}} %#FO
\newcommand\sh[1]{\##1} %Universal
\newcommand\QE[1]{\eqso(\Sigma_{#1})} %Existential
\newcommand\QU[1]{\eqso(\Pi_{#1})} %Universal
\newcommand\XE[1]{\#\Sigma_{#1}\textsc{[FO]}} %Extended Existential
\newcommand\XU[1]{\#\Pi_{#1}\textsc{[FO]}} %Extended Universal
\newcommand\HE[1]{\#\Sigma_{#1}\textsc{[FO]-Horn}} %Horn Existential
\newcommand\HU[1]{\#\Pi_{#1}\textsc{[FO]-Horn}} %Horn Universal

\def\dhsat{\textsc{DisjHornSAT}}
\def\shdhsat{\textsc{\#DisjHornSAT}}
\def\cpm{\textsc{\#PerfectMatching}}
\def\chsat{\textsc{\#HornSAT}}
\def\cdnf{\textsc{\#DNF}}
\def\ctdnf{\textsc{\#3-DNF}}
\def\ccnf{\textsc{\#CNF}}
\def\ctcnf{\textsc{\#3-CNF}}
\def\ctwcnf{\textsc{\#2-CNF}}
\def\csp{\textsc{\#SimplePath}}
\def\csat{\textsc{\#SAT}}

\def\dotminus{\mathbin{\ooalign{\hss\raise1ex\hbox{.}\hss\cr
			\mathsurround=0pt$-$}}}

%\def\A{{\frak A}}
\def\B{{\frak B}}
\def\C{\mathcal{C}}
\def\F{\mathcal{F}}
\def\L{\mathcal{L}}
\def\cG{\mathcal{G}}
\def\N{\mathbb{N}}
\def\P{\bar{P}}
\def\Q{\bar{Q}}
%\def\R{\bar{R}}
\def\S{\bar{S}}
\def\X{\bar{X}}
\def\Y{\bar{Y}}
\def\Z{\bar{Z}}
%% a - arity of \X / arity of assignments \P to \X
\def\a{\bar{a}}
%% b - arity of predicates in \S
\def\b{\bar{b}}
%% c - arity of auxiliar predicates/variables
\def\c{\bar{c}} %% super auxiliar elements
\def\d{\bar{d}} %% counted elements
\def\e{\bar{e}} %% counted elements
%% f - counting function
%% g - other functions
%% h - other functions
%% i - index
%% j - index
%% k - emergency index / size of tuple
%% l - emergency index / size of tuple
\def\l{\bar{\ell}}
%% m - size of variable tuple
%% n - size of predicate tuple
%% o - not used
\def\p{\bar{p}}
%% q - 
%% r - size of \X / \P
\def\s{\bar{s}}
%% t - size of \S
\def\t{\bar{t}}
\def\u{\bar{u}} %% auxiliary variables
\def\v{\bar{v}} %% auxiliary variables
\def\w{\bar{w}} %% auxiliary variables
\def\x{\bar{x}} %% quantified variables
\def\y{\bar{y}} %% auxiliary variables
\def\z{\bar{z}} %% open variables
\def\ep{\bar{o}}
\def\ga{\bar{p}}

% formulas
\newcommand{\sat}{\vDash}
\newcommand{\nsat}{\nvDash}
\newcommand{\et}{\;\wedge\;}
\newcommand{\vel}{\;\vee\;}
\newcommand{\then}{\;\rightarrow\;}
\newcommand{\Then}{\;\Rightarrow\;}
\renewcommand{\iff}{\;\leftrightarrow\;}
\newcommand{\Iff}{\;\Leftrightarrow\;}
\newcommand{\fa}[1]{\forall{#1}.\:}
\newcommand{\ex}[1]{\exists{#1}.\:}
\newcommand{\exinfinite}[1]{\exists^{\omega}{\:#1}.\:}
\newcommand{\exfinite}[1]{\exists^{<\omega}{\:#1}.\:}
\newcommand{\nex}[1]{\nexists{\:#1}.\:}


\newcommand{\fo}{{\rm FO}}
\newcommand{\so}{{\rm SO}}
\newcommand{\lfp}{{\rm LFP}}
\newcommand{\lfpop}{{\bf lfp} \,\, }
\newcommand{\alfp}{{\bf alfp} \,\, }
\newcommand{\clfp}[1]{[{\bf lsfp} \, #1]}
\newcommand{\fqfo}{{\rm FQFO}}
\newcommand{\fqso}{{\rm FQSO}}
\newcommand{\pth}{{\bf path} \,\, }
\newcommand{\tc}{{\rm TC}}
\newcommand{\dtc}{{\rm DTC}}
\newcommand{\pfp}{{\rm PFP}}
\newcommand{\eso}{\exists\so}
\newcommand{\first}{\operatorname{first}}
\newcommand{\last}{\operatorname{last}}
\newcommand{\succesor}{\operatorname{succ}}
\newcommand{\partition}{\operatorname{partition}}

\newcommand{\R}{\mathbf{R}}
\newcommand{\T}{\mathcal{T}}
\newcommand{\A}{\mathfrak{A}}
\newcommand{\G}{\mathbf{G}}
\newcommand{\all}{\text{\sc All}}
\newcommand{\allo}{\text{\sc AllOrd}}
\newcommand{\qso}{{\rm QSO}}
\newcommand{\qsoz}{\qso_{\bbZ}}
\newcommand{\optqso}{{\rm OptQSO}}
\newcommand{\maxqso}{{\rm MaxQSO}}
\newcommand{\minqso}{{\rm MinQSO}}
\newcommand{\rqfo}{{\rm RQFO}}
\newcommand{\tqfo}{{\rm TQFO}}
\newcommand{\tqso}{{\rm TQSO}}
\newcommand{\tqsos}{{\rm TQSO}_{\rm succ}}
\newcommand{\qfo}{{\rm QFO}}
\newcommand{\qfoz}{\qfo_{\bbZ}}
\newcommand{\eqfo}{\Sigma\qfo}
\newcommand{\eqso}{\Sigma\qso}
\newcommand{\eqsoz}{\eqso_{\bbZ}}
\newcommand{\sqso}{\text{\rm Saluja}\qso}
\newcommand{\fv}{\mathbf{FV}}
\newcommand{\sv}{\mathbf{SV}}
\newcommand{\fs}{\mathbf{FS}}
\newcommand{\arity}{{\rm arity}}
\newcommand{\length}[1]{\vert #1 \vert}
\newcommand{\size}[1]{\vert #1 \vert}
\newcommand{\shp}{\text{\sc \#P}}
\newcommand{\ptime}{\text{\sc P}}
\newcommand{\np}{\text{\sc NP}}
\newcommand{\bpp}{\text{\sc BPP}}
\newcommand{\cspp}{\text{\sc SPP}}
\newcommand{\pp}{\text{\sc PP}}
\newcommand{\rp}{\text{\sc RP}}
\newcommand{\pspace}{\text{\sc PSPACE}}
\newcommand{\nlog}{\text{\sc NL}}
\newcommand{\ulog}{\text{\sc UL}}
\newcommand{\conp}{\text{\sc NP}}
\newcommand{\pe}{\text{\sc \#PE}}
\newcommand{\shl}{\text{\sc \#L}}
\newcommand{\spp}{\text{\sc SpanP}}
\newcommand{\spanl}{\text{\sc SpanL}}
\newcommand{\gp}{\text{\sc GapP}}
\newcommand{\optp}{\text{\sc OptP}}
\newcommand{\maxp}{\text{\sc MaxP}}
\newcommand{\minp}{\text{\sc MinP}}
\newcommand{\maxpb}{\text{\sc MaxPB}}
\newcommand{\minpb}{\text{\sc MinPB}}
\newcommand{\fp}{\text{\sc FP}}
\newcommand{\totp}{\text{\sc TotP}}
\newcommand{\shpspace}{\text{\sc \#PSPACE}}
\newcommand{\fpspace}{\text{\sc FPSPACE}}
\newcommand{\nfpspace}{\text{\sc FPSPACE(poly)}}
\newcommand{\acc}{\textbf{acc}}

\newcommand{\CC}{\mathscr{C}}
\newcommand{\KK}{\mathscr{K}}
\newcommand{\FF}{\mathscr{F}}
\newcommand{\GG}{\mathscr{G}}
\newcommand{\LL}{\mathscr{L}}
\newcommand{\QQ}{\mathscr{Q}}
\newcommand{\enc}{{\rm enc}}
%\newcommand{\str}{\text{\sc Struct}}
\newcommand{\ostr}{\text{\sc OrdStruct}}
\newcommand{\Func}{\text{\sc Func}}
\newcommand{\res}[2]{#1|_{#2}}

%semiring
\newcommand{\nat}{\mathbb{N}}
\newcommand{\natinf}{\mathbb{N}_\infty}
\newcommand{\trop}{\mathbb{N}_{\min,+}}
\newcommand{\integ}{\mathbb{Z}}
\newcommand{\bln}{\mathbb{B}}
\newcommand{\pwset}[1]{2^{#1}}
\newcommand{\true}{\operatorname{true}}
\newcommand{\false}{\operatorname{false}}

\newcommand{\SR}{\bbS}
\newcommand{\add}{+}
\newcommand{\bigadd}{\sum}
\newcommand{\mult}{\cdot}
\newcommand{\bigmult}{\prod}
\newcommand{\adds}{\oplus}
\newcommand{\bigadds}{\bigoplus}
\newcommand{\mults}{\odot}
\newcommand{\bigmults}{\bigodot}
\newcommand{\zero}{\mathbb{0}}
\newcommand{\one}{\mathbb{1}}

%quantitative logic
\newcommand{\QL}{\operatorname{QL}}
\newcommand{\QMSO}{\operatorname{QMSO}}
\newcommand{\Op}{\operatorname{O}}
\newcommand{\sem}[1]{{\llbracket{}{#1}\rrbracket}}
\newcommand{\pa}[1]{\Pi{#1}.\,}
%\newcommand{\pa}[1]{\Pi{#1}}
\newcommand{\pas}{\Pi}
\newcommand{\paq}[1]{\Pi{#1}}
\newcommand{\sa}[1]{\Sigma{#1}.\,}
%\renewcommand{\sa}[1]{\Sigma{#1}}
\newcommand{\sas}{\Sigma}
\newcommand{\saq}[1]{\Sigma{#1}}
\newcommand{\fpa}[1]{\overrightarrow{\prod}{#1}.\:}
\newcommand{\lmid}{\;\mid\;}

\newcommand{\maxa}[1]{\operatorname{Max}{#1}.\,}
\newcommand{\mina}[1]{\operatorname{Min}{#1}.\,}
\newcommand{\clique}{\operatorname{clique}}

%tikz definition
\tikzset{
	defaultstyle/.style={>=stealth,semithick, auto,font=\small,
		initial text= {},
		initial distance= {3.5mm},
		accepting distance= {3.5mm}},
	accepting/.style=accepting by arrow,
	nstate/.style={circle, semithick,inner sep=1pt, minimum size=4mm}}

\tikzset{
	rect/.style={
		rectangle,
		rounded corners,
		draw=black, 
		thick,
		text centered},
	rectw/.style={
		rectangle,
		rounded corners,
		draw=white, 
		thick,
		text centered},
	sq/.style={
		rectangle,
		draw=black, 
		thick,
		text centered},
	sqw/.style={
		rectangle,
		draw=white, 
		thick,
		text centered},
	arrout/.style={
		->,
		-latex,
		thick,
	},
	arrin/.style={
		<-,
		latex-,
		thick,
		El         },
	arrd/.style={
		<->,
		>=latex,
		thick,
	},
	arrw/.style={
		thick,
	}
}


%Turing machine
\newcommand{\tmo}{\text{\rm \#output}}
\newcommand{\tma}{\text{\rm \#accept}}
\newcommand{\tmr}{\text{\rm \#reject}}
\newcommand{\tmg}{\text{\rm gap}}
\newcommand{\tmt}{\text{\rm \#total}}

\newcommand{\support}{\text{\rm supp}}
\newcommand{\supp}{\text{\rm supp}}
\newcommand{\val}{\text{\rm val}}

% commands
\newcommand{\cristian}[1]{\todo[inline, color=blue!10]{{\bf Cristian:} #1}}
\newcommand{\marcelo}[1]{\todo[inline, color=red!20]{{\bf Marcelo:} #1}}
\newcommand{\martin}[1]{\todo[inline, color=green!20]{{\bf Martin:} #1}}



\def\eg{{\em e.g.}}
\def\cf{{\em cf.}}

\newcommand{\red}[1]{{\bf\textcolor{red}{#1}}}

\begin{document}

\title{Response letter to the reviews of the paper: Descriptive Complexity for Counting Complexity Classes}

\maketitle

We would like to thank the anonymous reviewers for their useful comments.
Below are our answers to comments or suggestions made about our paper.
We highlight the reviewer comment in red and then write our answer.

\section*{Reviewer 1}

\subsection*{Technical}

\begin{itemize}
	\setlength\itemsep{0.5em}
	\item \red{p.2, first paragraph: ``However, other function classes have emerged from the need to understand	the complexity of some computation problems for which little can be said if their decision counterparts are considered.'' Why can only little be said about the decision counterparts of the examples mentionend[sic] afterwards?}
	
	This was meant to say that the complexity of the decision counterparts of the problems mentioned afterwards did not provide a full understanding of the complexity of the problem in its computation version. This sentence has been reformulated to explain that idea.
	
%	\item p.7, first sentence: ``fragments or extensions [...] obtained by restricting'': extensions are not obtained by restricting something
%	\item p.10, before Thm. 4.5: ``by considering second-order product'': We need both second-order product and sum (at least both are used in Thm. 4.5)
	\item \red{p.10, last line: ``$B$-th bit'': mention in what way $B$ is interpreted as a number}
	
	This proof was rewitten to explain this notion better.
	
%	\item[$\checkmark$] p.12: ``To formalize this set of optimization problems'': class instead of set
%	\item p.14: ``we have to concentrate on the class of counting problems with easy decision versions'': here, easy decision version means ``in \bpp''. Before, it was defined as ``in P''. Should be consistent.
%	\item p.14: ``as expected, $\Sigma \bar{X}$ is the respective nesting of $\Sigma X.$’s'': why is this explained for second order 	quantifiers but not for first-order quantifiers?
%	\item p.14: $\sum_{i=1}^n\alpha_i$: explain that this is notation for $\alpha_1+\cdots+\alpha_n$
%	\item p.17: ``Since $\alpha$ is not null'' $\to$ ``Since $\sem{\alpha}$ is not the identically zero function'' (or similar)
	\item \red{p.17/18: In the proof of Prop 5.3, the vocabulary only contains the numerical predicate $<$. When encoding structures as binary words, numerical predicates are implicit (and not encoded), though. What is the intended meaning here?}
	
	For ordered structures the vocabulary should always include $<$. The encoding function in the paper was revised to make this more clear.
	
%	\item p.18, after proof of Claim 5.4: ``which implies that 2 $\in \eqso(\Sigma_0)$ and $2 \not\in \E{1}$'': $2 \in \eqso(\Sigma_0)$ does not follow from the argument before as stated in this sentence, but is trivial
%	\item[$\checkmark$] p.18: ``Also, suppose that it is in $\L$-PNF'': $\L$ should be $\LL$ , PNF should be SNF
%	\item[$\checkmark$] p.19, end of proof of Prop. 5.5: ``which violates the inequality'' $\to$ ``which violates the inequality 	for large structures''
%	\item[$\checkmark$] p.20: ``we will show a formula in the grammar'' $\to$ ``... in the logic''
%	\item p.20: ``Since $\Sigma_1$ is closed under conjunction, then Lemma 5.7 holds for ...'' $\to$ ``Since $\Sigma_1$ is closed under cnojunction, Lemma 5.7 can be applied to ...'' (or similar)
%	\item[$\checkmark$] p.22: ``and leaving each $R_{\psi}^{\A}$ empty'' $\to$ $R_{\psi}^{\A'}$
%	\item p.22: ``To show that $\alpha$ is in \totp'' $\to$ ``To show that the function defined by $\alpha$ ...'' (or similar)
%	\item[$\checkmark$] p.22: ``Let $\alpha$ = ... where $\varphi$ is an FO-formula'' $\to$ ``... where $\varphi$ is an $\Sigma_1$[FO]-formula''
	\item p.23, formulae: Maybe put (large) parantheses around the relativization of quantifiers, for	example the part of formula $\varphi_i$ after the quantifiers
	\item[$\checkmark$] p.24, formula 5.6: subscript of large conjunction: $\varphi_i^{+}(\bar{X},\bar{v})$ $\to$ $\varphi_i^{+}(\bar{X},\bar{y})$
	\item[$\checkmark$] p.25: ``We construct a function $\lambda$ that receives a quantitative formula $\beta$ ...'' $\to$ ``... that receives
	a formula $\beta \in \eqso(\logex{1})$ ...''
	\item p.25, last sentence: ``which is based on considering some witnesses of logarithmic size'' could
	be explained a little more / made more clear
	\item p.26: ``... auxiliary predicate $A$ defined as'' $\to$ ``... auxiliary predicate $A$ which will be defined
	as ... in the formula'' (or similar)
	\item p.28: ``since universal formulas are monotone over induced substructures'': ``monotone over
	induced substructures'' should be reformulated
	\item p.28: ``Next we prove that ... is hard for ... over a signature $\R$ ...'' sounds like it only works
	for one specific vocabulary. Maybe simply ``over any signature $\R$''
	\item p.28, formula $\alpha_{\A}$: second big conjunction should have $\bar{z}'$ in the subscript instead of $\bar{z}$ (because
	the quantifier $\forall u$ is handled by this as well if I understand correctly)
	\item p.29: ``In other words, in poly time we can replace $\varphi_j^i$ by a disjunction of ....'': Some of these
	formulae might evaluate to $\perp$ or $\top$ altogether and can be removed (potentially together with
	the surrounding conjunction)
	\item p.29: ``Intuitively $t_1,\ldots,t_{m'}$ are used to count from 1 to $m'$ each subformula ...'': Should be
	reformulated, maybe rather write that they are used to have disjoint sets of propositional
	assignments for the different disjuncts of the outermost disjunction (corresponding to the
	summands in the original formula)
	\item p.30, proof of Prop. 5.12: For $\dhsat \in \totp$, [35] is cited. It’s stated that a \totp-procedure
	for DNF-SAT is given there and it can be adapted to \dhsat. While the
	cited paper contains the result for DNF-SAT, the procedure is not explicitly given as far as
	I can see, so I can't check whether it is adaptable. On the other hand, membership of any
	function counting the number of satisfying assignments for a class of propositional formulae
	with poly-time decision version in \totp should hold by a similar argument as membership in
	DelayP. Maybe change the citation or give a short proof sketch.
	\item[$\checkmark$] p.31, definition of $\leq$ on $\F$: ``$f(i) \leq g(i)$ for every $i \in \{1,\ldots,\ell\}$'': Inputs for $f$ and $g$ should be
	elements of $A^{\ell}$, not $\{1,\ldots,\ell\}$
	\item p.32/33, example 6.2: I don't see why the formula given for dag's does not work for arbitrary
	graphs: In arbitrary graphs, the fixpoint is still reached after at most $n-1$ steps, since the
	maximal length of any shortest path in any graph is $n-1$. On the other hand, assuming that
	I am mistaken and the support does not become stable within $n-1$ steps: Using the formula
	given for arbitrary graphs, the support of the constructed function restricted to those tuples
	that have first in their last component is the same as the support of the constructed function
	using the formula for dags in each step of the recursion. Hence, the formula should not be
	able to fix such a problem.
	\item p.33: ``executes exactly $n^k$ steps for a fixed $k \geq 1$'' → ``executes exactly $n^k$
	steps on large
	inputs for a fixed $k \geq 1$'' (at least it does not work for $n = 1$) (small inputs can be handled
	seperately)
	\item p.35 and p.36: While $\sum$ was used for finite sums in \qso-formulae throughout the paper, now
	a large +-sign is used. This might be confusing
	\item p.36: $\alpha_{q_i}$ used instead of $\alpha_{s_{q_{i}}}$
	\item p.36: ``Clearly, at each iteration of the fixed point operator, the tuple $\bar{t}$ represents the step the
	machine is currently in.'': This is quite vague. Maybe something like ``Clearly, tuple $\bar{t}$ encodes
	the number of steps the machine has done and in each iteration of the fixed point operator,
	one timestep of the machine is executed''
	\item p.36: ``at the $\bar{a}$-th iteration of the fixed point operator, it holds that $f(\bar{a}, \bar{b}) \leq 1$ and $f(\bar{a}', \bar{b}) = 0$ for every $\bar{b} \in A^k$. 
	In the same way, it can be seen that for each function $g \in \ldots$ it holds that $g(\bar{a}) \leq 1$ and $g(\bar{a}')= 0$. 
	I think that more properties of these functions should be mentioned
	in order to argue that the defined function is the function computed by the machine:
	\begin{itemize}
		\item[-] For each $\bar{b}$, only one of the functions $T_0$,$T_1$,$T_B$,$T_{\vdash}$ outputs 1 on input ($\bar{a}, \bar{b})$
		\item[-] For each $\bar{b}$, only one of the functions $h,\hat{h}$ outputs 1 on input $(\bar{a}, \bar{b})$
		\item[-] There is exactly one $\bar{b}$ with $h(\bar{a}, \bar{b}) = 1$
		\item[–] There is exatcly one $i$ such that $s_{q_i}(\bar{a}) = 1$
	\end{itemize}
	(It could also generally be argued that the functions are 0/1-functions encoding the intended
	predicates and are 0 when the timestep is the successor of the current timestep)
	\item p.36: ``is defined as an extension of QFO with the formula ...'' $\to$	 ``is defined as the extension
	of QFO by the additional operator ...'' (or similar)
\end{itemize}
\vspace{1em}
\subsection*{Grammatical/Language}
\begin{itemize}
	\setlength\itemsep{0.5em}
	\item generally: both formulae and formulas are used as plural of formula---should be consistent
	\item p.2, in the middle: we propose a restriction: Should be something like ``we propose to restrict
	WL to the natural numbers as fixed semiring, calling this restriction ...''
	\item[$\checkmark$] p.3, first paragraph of 2.1: ``omit the word finite or ordered'' $\to$ ``omit the words finite and
	ordered''
	\item p5, description of syntax of QSO: ``This division between Boolean and quantitative level
	is essential for understanding the difference between the logic and the quantitative part.''
	reformulate this?
	\item[$\checkmark$] p.5, next sentence: ``this will allow us later'' $\to$ ``this will later allow us''
	\item[$\checkmark$] p.6, second to last sentence: ``similarity of $\alpha_3$ with'': ``similarity to'' is much more common
	than ``similarity with''
	\item[$\checkmark$] p.7, next sentence: ''in this direction'' $\to$ ``in this regard''
	\item[$\checkmark$] p.7, last sentence before 3.1: ''connection of QSO with'' $\to$ ``connection of QSO to''
	\item[$\checkmark$] p.7, middle: ``Another difference of WL with our approach'' $\to$ ``Another difference between
	WL and our approach''
	\item p.9, first sentence: ``then condition (2) holds'': omit the word then or reformulate the sentence
	\item[$\checkmark$] p.9, first paragraph: ``the constant function $s$ can be trivially simulated'' $\to$ ``any constant
	function $s$ can be trivially simulated''
	\item[$\checkmark$] p.9, proof of Prop. 4.3: ``To prove the condition (2)'' $\to$ ``To prove condition (2)''
	\item throughout the paper: ``similar than'' $\to$ ``similar to'' / ``similar than for X, we do Y'' $\to$
	``similarly as for''. Examples:
	\begin{itemize}
		\item[-] p.10, proof of Thm. 4.5: ``the proof is similar than in Theorem 4.4'' $\to$ ``the proof is
		similar to the proof of Theorem 4.4''	
		\item[-] p.10, second sentence: ``similar than the previous proof'' $\to$ ``similar as in the previous
		proof''
		\item[-] p.11, last sentence before example 4.7: ``Similar than for QSO'' $\to$ ``Similarly as for QSO''
	\end{itemize}
	\item p.11, end of proof of Thm. 4.5: ``Using an analogous argument'': state what the argument is
	analogous to (the argument in Thm. 4.4)
	\item[$\checkmark$] p.12, after Corollary 4.8: ``shows how robust and versatile is QSO'' $\to$ ``shows how robust and
	versatile QSO is''
	\item p.12, after Corollary 4.8: ``counting complexity classes [...] even beyond $\nat$'' - make more precise
	what ``beyond $\nat$'' refers to
	\item[$\checkmark$] p.12, ``given CNF propositional formula, that can be made true'': no comma
	\item p.12: ``in [43] they defined'' $\to$ something like ``$<$authors$>$ [43] defined''
	\item[$\checkmark$] p.12: ``we extend as follows QSO'' $\to$ ``we extend QSO by max and min quantifiers as follows''
	\item[$\checkmark$] p.12: ``MinQSO is defined analogously changing max with min'' $\to$ ``replacing max by min''
	\item[$\checkmark$] p.12: ``the same holds with MinQSO(FO)'' $\to$ ``the same holds for ...''
	\item[$\checkmark$] p.13: ``in contrast with'' $\to$ ``in contrast to''
	\item[$\checkmark$] p.13: ``follow the same approach but considering'' $\to$ ``but consider''
	\item[$\checkmark$] p.13: ``in the search of efficient approximation algorithms'' $\to$ ``in the search for''
	\item[$\checkmark$] p.14: ``the search of robust classes'' $\to$ ''the search for robust classes''
	\item p.14: ''the first desirable condition has to do with the closure properties satisfied by the class'':
	Reformulate this sentence
	\item[$\checkmark$] p.14: ``As in the cases of P and NP that are closed'' $\to$ ``Analogously to the cases of P and
	NP, which are closed under intersection and union, ...''
	\item[$\checkmark$] p.16: ``such that for both, $A\mod\varphi(\bar{B}_i,\bar{b}_i,\bar{a}_i)$'' $\to$ ''such that for $i \in \{1,2\}$: ... ''
	\item[$\checkmark$] p.16: ``Similarly to Theorem 5.1, we show ..'' $\to$ ``Similarly as in the proof of Theorem 5.1, we
	show''
	\item p.19: ``However, this class is not closed under sum, and then it is not robust under basic
	closure properties'': Should be reformulated
	\item[$\checkmark$] p.19: ``Let ... be this formula, where $\varphi$ is in first-order and quantifier-free'' $\to$ ``... where $\varphi$ is
	first-order and quantifier-free''
	\item[$\checkmark$] several occurrences: ``This is, ...'' $\to$ ``That is, ...'', for example two times on p.20
	\item[$\checkmark$] p.20: ``we focus only in proving'' $\to$ ``we focus only on proving''
	\item[$\checkmark$] p.20: ``We prove this for a more general case for $\eqso(\LL)$ where $\LL$ is a fragment of FO.''
	$\to$ ``We prove this for the more general case of $\eqso(\LL)$ with $\LL$ being a fragment of SO.''
	(or similar)
	
	\item[$\checkmark$] p.20: ``we will show a formula'' $\to$ ``give'' or ``construct'' instead of ``show''
	\item[$\checkmark$] p.20/21: ``$\shp$ is believed not to be closed under subtraction by one by some complexitytheoretic
	assumption'' $\to$ ``... due to some complexity-theoretic assumption'' (or similar)
	\item[$\checkmark$] p.21, next sentence: ``for a possible extension'' $\to$ ``as a possible extension''
	\item p.21: ``the desired extension has to be achieved by ... . More precisely, we define...'': The second
	sentence doesn’t make the statement of the first one more precise, but rather explains the
	actual definiton (which is chosen in accordance with the sentence before)
	\item[$\checkmark$] p.21: ``a parsimonious reduction from a function'' $\to$ ``a parsimonious reduction from any
	function''
	\item[$\checkmark$] p.21: ``we define a function that converts a formula $\alpha$ in...'' $\to$ ``that converts a given formula
	$\alpha$...''
	\item[$\checkmark$] p.22: ``we iterate for every tuple'' $\to$ ``we iterate over all tuples''
	\item[$\checkmark$] p.22: ``Then we will show how to define a formula $\varphi'$
	...'' $\to$ ``We will show how to define ...''
	\item[$\checkmark$] p.22: ''no variable from $\bar{x}$ are mentioned'' $\to$ ``... is mentioned''
	\item[$\checkmark$] p.24: ``if, and only if, if X, then Y'': This construction makes it unneccesarily hard to parse
	the sentence
	\item[$\checkmark$] p.24: ``Then, the conclusion of the implication...'' $\to$ ``The conclusion of the implication...''
	\item[$\checkmark$] p.25: ``For the sake of simplification...'' $\to$ ``For the sake of simplicity...''
	\item[$\checkmark$] p.26: ``In the previous section, we show ...'' $\to$ ``In the previous section we showed ...''
	\item[$\checkmark$] p.27: ``we postpone the proof the previous proposition after ...'' $\to$ ``we postpone the proof of
	the previous proposition to after ...''
	\item p.29: ``replace by $\perp$ and $\top$ \underline{wherever it corresponds}'': Should be reformulated
	\item[$\checkmark$] p.30: ``consist in computing $g_{\alpha}(\enc(\A))$ for each input ...'' $\to$ ``... on input ...''
	\item[$\checkmark$] p.30, same sentence: ``and then simulate'' $\to$ ``and then simulating''
	\item[$\checkmark$] p.33: ``In contrast with LFP'' $\to$ ``In contrast to LFP''
	\item[$\checkmark$] p.33: ``These operations are represented by the set $\{0,1,\emptyset\}$, respectively.'' $\to$ ``... by 0, 1 and
	$\emptyset$, respectively''
	\item[$\checkmark$] p.34: ``Notice that the values of each function $f$ in $\alpha$'' $\to$ ``Notice that the values of each
	function in $\alpha$''
	\item[$\checkmark$] p.34: ``that is why we talk about ...'' $\to$ ``which is why we talk about''
	\item[$\checkmark$] p.34: ``Besides, notice that notation ...'' $\to$ ``Besides, notice that the notation ...''
	\item[$\checkmark$] p.34: ``Function $h$ is used to indicate whether the head of the working tape is in some position
	at some point of time'' $\to$ ``... whether the head of the working tape is in a specific position
	at a certain point of time'' (or similar)
	\item[$\checkmark$] p.36: ``Given a relation signature $\R$'' $\to$ ``... relational signature ...''
	\item[$\checkmark$] p.37: ``As it was mentioned before'' $\to$ ``As mentioned before''
	\item[$\checkmark$] p.37: ``Next we construct a logarithmic-space ....'' $\to$ ``We construct a logarithmic-space ...''
	6
	\item[?] p.38: ``In this sense, there are several directions for future research'' $\to$ ``Consequently, there
	are ...'' (or similar)
	\item[$\checkmark$] p.38: ``In the same direction, ...'' $\to$ ``Similarly, ...'' (or similar)
	\item[$\checkmark$] p.38: ``so to have'' $\to$ ``in order to have'' (or similar)
\end{itemize}
\vspace{1em}
\subsection*{Typos}
\begin{itemize}
	\setlength\itemsep{0.5em}
	\item p.1, footnote: ``clases'' should be ``classes''
	\item throughout the paper: plural used with verb ending on s. Examples:
	\begin{itemize}
		\item[-] p.2, middle: ``fragments of QSO captures''
		\item[-] p.6, example 3.1: ``the sum quantifiers in $\alpha_1$ aggregates''
	\end{itemize}
	\item[$\checkmark$] p4, beginning of 2.2: ``some of \underline{the} are recalled here''
	\item[$\checkmark$] p6, example 3.2, first paragraph: ``otherwise $M(i.j)$'', dot used instead of comma
	\item[$\checkmark$] throughout the paper: a instead of an (espacially: a FO formula/assignment, a SO formula/assignment).
	Example:
	\begin{itemize}
		\item[-] p.9, third sentence: ``Given a FO formula'' $\to$ ``Given an FO formula''
		\item[-] p.38: ``As a example'' $\to$ ``As an example''
	\end{itemize}
	\item throughout the paper: singular used with verb not ending on s. Examples:
	\begin{itemize}
		\item[-] p.9, after proof of Prop. 4.3: ``use of a fragment [...] that capture''
		\item[-] p.11, end of proof of Thm. 4.5: ``and $\alpha$ reconstruct''
	\end{itemize}
	\item[$\checkmark$] p.9, middle: ``A second attempt could be based then'' $\to$ ``A second attempt could then be
	based''
	\item[$\checkmark$] p.11, 4.1: ``output of a function \underline{in} not a natural number''
	\item[$\checkmark$] p.12: ``we have to dinstinguished''
	\item[$\checkmark$] p.14, beginning of 5.1: ``subclases''
	\item[$\checkmark$] p.21: ``it does not mentioned'' $\to$ ``it does not mention''
	\item[$\checkmark$] p.22: ``Without lost of generality'' $\to$ ``Without loss of generality''
	\item[$\checkmark$] p.24: ``The previous claim and proof motivates the following definitions.'' $\to$ ``The previous
	claim and proof motivate the following definition.''
	\item p.30: ``to define a natural generalization for functions of the notion of least fixed point of
	LFP'' $\to$ ``to define a natural generalization of LFP to functions'' (or similar)
	\item[$\checkmark$] p.30: ``this notion can be used to captures FP'' $\to$ ``... to capture FP''
	\item[$\checkmark$] p.31: ``we simply generalized these conditions'' $\to$ ``we simply generalize these conditions''
	\item[$\checkmark$] p.32: ``which is used to defined'' $\to$ ``which is used to define''
	\item[$\checkmark$] p.33: ``since $\beta$ is constructed from monotones operations'' $\to$ ``... monotone ...''
	\item[$\checkmark$] p.35: ``Formulas $\alpha_h$ is defined ...'' $\to$ ``Formula $\alpha_h$ is defined ...''
	7
	\item[$\checkmark$] p.37: ``we simulate $M_\beta$ on input $v[1/n]$'' $\to$ ``... with input $v[1/x]$''
\end{itemize}
\vspace{1em}
{\bf Other}
\begin{itemize}
	\setlength\itemsep{0.5em}
	\item[$\checkmark$] running title is ``COUNTING''
	\item on definition of functions, \verb|\colon| should be used instead of :
	\item In [1], a similar hierarchy to the one by Saluja et al. was introduced and studied. This work
	should be cited.
\end{itemize}

\bigskip

\section*{Reviewer 2}

\begin{itemize}
	\setlength\itemsep{0.5em}
	\item In several proofs, you rely on disjoint union of solutions and cartesian product of solutions. Indeed, its gives a polynomial time counting reduction. Can you make connections with (+,*) circuits ? It would be similar to what is done in A Circuit-Based Approach to Efficient Enumeration (Antoine Amarilli, Pierre Bourhis, Louis Jachiet, Stefan Mengel) where simple circuits (dDNNF, which are basically disjoint unions and cartesian products) are used to capture the solutions of MSO queries.
	
	\item In general I would be interested to know the relationship of your couting classes with their enumeration counterpart. 
	For instance there is a paper adapting [39] to enumeration: Enumeration Complexity of Logical Query Problems with Second-order Variables (Arnaud Durand and Yann Strozecki). In particular they consider class with some Horn structure to capture easy enumeration problems, it may have some connection with the class $\sigma$ QSO($\Pi_1$ Horn) you introduce.
	
	\item For the most involved proofs, you could describe informally your proof strategy in a few lines to help the reader. You may also add some example to illustrate the different fragment of QSO (you already do it sometimes and it is very helpful).
	
	\item[$\checkmark$] page 4:
	-some of the $\to$ some of them 
	\item[$\checkmark$] page 10: 
	-$2^{|A|^l}$-bits $\to$ $2^{|A|^l}$ bits
	\item[$\checkmark$] page 21:
	-it does not mentioned $\to$ it does not mention
	\item page 22: 
	-you make a long proof about some closure under subtraction by one.
	You should put the result in a theorem envirpnment to make what you are
	proving more evident.
	\item[$\checkmark$] -replace .... for $\to$ replace ... by;  there are several instances of
	that
	\item[$\checkmark$] -$\phi$ is an FO-formula $\to$ a $\Sigma_1[FO]$ formula
	\item -you explain that you rewrite your formula to obtain conditions a,b and c, 
	you should give an idea how these conditions will be used
	\item[$\checkmark$] page 23:
	a SO, a FO  an SO, an FO
	\item page 24: 
	if, if $\phi$$\to$ $\phi$
	\item page 25: 
	-you explain that the proof relies on witnesses of logarithmic size.
	It was not clear to me while reading the proof. Can you explain more ?
	\item page 28:
	-the complete problem for $\Sigma$ QSO($\Sigma_2$ Horn) would be complete also for $\#\Sigma-2$ Horn. You should mention here that you do not need sum here
	and that the framework of [39] is enough for this result.
	\item[?] page 30: ``this notion can be used to captures FP'' $\to$ ``... to capture FP''
	
	\item[$\checkmark$] page 32: 
	- paths of length at most n $\to$ in an acylic graph, all paths are of size less than n
	\item page 33:
	-there is a proof symbol \verb|\qed| but it does not end a well defined proof environment
	
	\item page 38: 
	the last sentence on fixed point operator is to vague. Either expand it to 
	say what king of question on fixed point operator you want to investigate or remove it	
\end{itemize}

	
\end{document}
