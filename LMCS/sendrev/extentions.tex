% !TeX root = main.tex

There exist complexity classes that do not fit in our framework because either the output of a function is not a natural number (e.g. a negative number) or the class is not defined purely in terms of arithmetical operations (e.g. min and max).
To remedy this problem, we show here how $\qso$ can be easily extended  to capture such classes that go beyond sum and product over natural numbers. 

It is well-known that, under some reasonable complexity-theoretical assumptions, $\shp$ is not closed under subtraction, not even under subtraction by one~\cite{OH93}.
To overcome this limitation, $\gp$ was introduced in~\cite{FFK94} as the class of functions $f$ for which there exists a polynomial-time NTM $M$ such that $f(x) = \tma_M(x) - \tmr_M(x)$, where  $\tmr_M(x)$ is the number of rejecting runs of $M$ with input $x$.
That is, $\gp$ is the closure of $\shp$ functions under subtraction, and its functions can obviously take negative values.
Given that our logical framework was built on top of the natural numbers, we need to extend $\qso$ in order to capture $\gp$. 
The most elegant way to do this is by allowing constants coming from $\bbZ$ instead of just $\bbN$. 
Formally, we define the logic $\qsoz$ whose syntax is the same as in \eqref{syntax} and whose semantics is the same as in Table~\ref{tab-semantics} except that the atomic formula $s$ (i.e. a constant) comes from $\bbZ$.  
Similar than for $\qso$, we define the fragment $\eqsoz$ as the extension of $\eqso$ with constants in $\bbZ$.
\begin{exa}
	Recall the setting of Example~\ref{ex:cliques} and suppose now that we want to compute the number of cliques in a graph that are not triangles. This can be easily done in $\qsoz$ with the formula:
	$
	\alpha_5 :=	\alpha_2 + (-1) \cdot \alpha_1. 
	$ \qed
\end{exa}
Adding negative constants is a mild extension to allow subtraction in the logic. 
It follows from our characterization of $\shp$ that this is exactly what we need to capture  $\gp$.
\begin{cor} \label{prop:capture-gapp}
	$\eqsoz(\fo)$ captures $\gp$ over ordered structures.
\end{cor}
This is an interesting result that shows how robust and versatile $\qso$ is for capturing different counting complexity classes even beyond $\bbN$.

A different class of functions comes from considering the optimization version of a decision problem. For example, one can define MAX-SAT as the problem of determining the maximum number of clauses, of a given CNF propositional formula that can be made true by an assignment. Here, MAX-SAT is defined in terms of a maximization problem which in its essence differs from the functions in $\shp$. 
To formalize this class of optimization problems, Krentel defined $\optp$~\cite{krentel1988complexity} as the class of functions computable by taking the maximum or minimum of the output values over all runs of a polynomial-time NTM machine with output tape (i.e. each run produces a binary string which is interpreted as a number). 
For instance, MAX-SAT is in $\optp$ as many other optimization versions of $\np$-problems.
Given that in~\cite{krentel1988complexity} Krentel did not make the distinction between $\max$ and $\min$, in~\cite{vollmer1995complexity} they defined the classes $\maxp$ and $\minp$ as the max and min version of the problems in $\optp$ (i.e. $\optp = \maxp \cup \minp$).

In order to capture classes of optimization functions, we extend $\qso$ with $\max$ and $\min$ quantifiers as follows (called $\optqso$). 
Given a signature $\R$, the set of $\optqso$-formulae over $\R$ is given by extending the syntax in (\ref{syntax}) with the following operators:
\begin{align*}
\max\{\alpha,\alpha\} \ \mid\ \min\{\alpha,\alpha\} \ \mid \maxa{x} \alpha \ \mid \ \mina{x} \alpha \ \mid \ \maxa{X} \alpha \ \mid \ \mina{X} \alpha 
\end{align*}
where $x \in \fv$ and $X \in \sv$. The semantics of the $\qso$-operators in $\optqso$ are defined as usual. Furthermore, the semantics of the max and min quantifiers are defined as the natural extension of the sum quantifiers in $\qso$ (see Table~\ref{tab-semantics}) by maximizing or minimizing, respectively, instead of computing a sum or a product. 
\begin{exa}\label{ex:optqso}
	Recall again the setting of Example~\ref{ex:cliques} and suppose now that we want to compute the size of the largest clique in a graph. This can be done in $\optqso$ as follows:
	\[
\alpha_6 := \maxa{X} \left( \, \clique(X) \cdot \sa{z} X(z)  \, \right)
	\]
	Notice that formula $\sa{z} X(z)$ is used to compute the number of nodes in a set $X$.  \qed
\end{exa}
Similar than for $\maxp$ and $\minp$, we have to distinguish between the $\max$ and $\min$ fragments of $\optqso$. For this, we define the fragment $\maxqso$ of all $\optqso$ formulae constructed from $\qfo$ operators and $\max$-formulae $\max\{\alpha,\alpha\}$, $\maxa{x} \alpha$ and  $\maxa{X} \alpha$.
The class $\minqso$ is defined analogously replacing $\max$ by $\min$. Notice that in $\maxqso$ and $\minqso$, second-order sum and product are not allowed. For instance, formula $\alpha_6$ in Example \ref{ex:optqso} is in $\maxqso$.
As one could expect, $\maxqso$ and $\minqso$ are the needed logics to capture $\maxp$ and $\minp$.
\begin{thm} \label{theo:capture-optp}
	$\maxqso(\fo)$ and $\minqso(\fo)$ capture $\maxp$ and $\minp$, respectively, over ordered structures.
\end{thm}
\proof
It is straightforward to prove that $\maxp$ can compute any $\fo$-formula, is closed under first-order sum and product, and second-order maximization. 
Therefore, condition (1) in Definition~\ref{def:cap} follows similarly as in the previous characterizations. Furthermore, one can easily see that the same holds for $\minqso(\fo)$.
The proof for the other direction is similar to the one described in \cite{kolaitis1994logical} extended with the ideas of Theorem~\ref{theo:capture-fp}. Let $f\in \maxp$ be a function defined over some signature $\R$ and
$\ell\in\nat$ such that $\lceil\log_2 f(\enc(\A)) \rceil \leq |\A|^\ell$ for each $\A\in\ostr[\R]$.
For $U \subseteq A^{\ell}$, we can interpret the encoding of $U$ ($\enc(U)$) as the binary encoding of a number with $|\A|^\ell$-bits. We denote this value by $\val(\enc(U))$.
Then, given $\A\in\ostr[\R]$ and $U \subseteq A^{\ell}$, consider the problem of checking whether $f(\enc(\A)) \geq \val(\enc(U))$. 
Clearly, this is an $\np$-problem and, by Fagin's theorem, there exists a formula of the form $\ex{\bar{X}} \Phi(\bar{X}, Y)$ with $\Phi(\bar{X}, Y)$ in $\fo$ and $\arity(Y) = \ell$ such that $f(\enc(\A)) \geq \val(\enc(U))$ if, and only if, $(\A,v,V) \models \ex{\bar{X}} \Phi(\bar{X}, Y)$ with $V(Y) = U$. 
Then we can describe $f$ by the following $\maxqso$ formula:
$$
\alpha = \maxa{\bar{X}} \maxa{Y} \ \Phi(\bar{X}, Y) \cdot \big( \sa{\bar{x}} Y(\bar{x}) \cdot \pa{\bar{y}}(\bar{x} < \bar{y} \mapsto 2) \big).
$$
Note that, in contrast to previous proofs, we use $\bar{x} < \bar{y}$ instead of $\bar{y} < \bar{x}$ because the most significant bit in $\enc(U)$ correspond to the smallest tuple in $U$.  
It is easy to check that $\Phi(\bar{X}, Y)$ simulates the NP-machine and, if $\Phi(\bar{X}, Y)$ holds, the formula to the right  reconstructs the binary output from the relation in $Y$.
Then, $\alpha$ is in $\maxqso(\fo)$ over $\R$ and $\sem{\alpha}(\A) = f(\enc(\A))$. 

For the case of $\minqso(\fo)$ and a function $f \in \minp$, one has to follow the same approach but consider the $\np$-problem of checking whether $f(\enc(\A)) \leq \val(\enc(U))$. Then, the formula for describing $f$ is the following:
$$
\alpha = \mina{\bar{X}} \mina{Y} \ \sa{\bar{x}} \big( (\Phi(\bar{X}, Y) \rightarrow Y(\bar{x})) \cdot \pa{\bar{y}}(\bar{x} < \bar{y} \mapsto 2)  \big).
$$
In this case, if the formula $\Phi(\bar{X}, Y)$ is false, then the output produced by the subformula inside the $\min$-quantifiers will be the biggest possible value (i.e. $2^{{|\A|}^\ell}$).
On the other hand, if $\Phi(\bar{X}, Y)$ holds, the subformula will produce $\val(\enc(U))$. 
In a similar way as in $\max$, we conclude that $\alpha$ is in $\minqso(\fo)$ and $\sem{\alpha}(\A) = f(\enc(\A))$.

\qed
It is important to mention that a similar result, following the framework of~\cite{SalujaST95}, was proved in~\cite{kolaitis1994logical} for the class $\maxpb$ (resp., $\minpb$) of problems in $\maxp$ (resp., $\minp$) whose output value is polynomially bounded.
Interestingly, Theorem \ref{theo:capture-optp} is stronger since our logic has the freedom to use sum and product quantifiers, instead of using a max-and-count problem over Boolean formulae. 
Finally, it is easy to prove that our framework can also capture $\maxpb$ and $\minpb$ by disallowing the product $\Pi x$ in $\maxqso(\fo)$ and $\minqso(\fo)$, respectively.
