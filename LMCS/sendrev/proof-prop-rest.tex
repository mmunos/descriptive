%!TEX root = main.tex

We divide this proof into three parts.
First, we show that $\E{1} \not\subseteq \QE{0}$. 
For this inclusion to be true, it is required to hold for an arbitrary ordered relational signature $\R$, so it suffices to show that it  is not true for at least one such a signature.
Let $\R$ be the ordered signature that contains only the relation name $<$.
Suppose that there is a $\QE{0}$ formula $\alpha$ over $\R$ which is equivalent to the $\E{1}$-formula $\sa{x} \ex{y} (x < y)$. 
That is, for every finite $\R$-structure $\A$, $\sem{\alpha}(\A) = \size{\A} - 1$.
%\martin{reformul\'e por qu\'e elijo ese R}
%\marcelo{OK}

Suppose that $\alpha$ is in SNF, namely, $\alpha = \sum_{i = 1}^k \alpha_i$ for some fixed $k$. 
Since $\sem{\alpha}$ is not the identically zero function, consider some $\alpha_i$ that describes a non-null function. 
Let $\alpha_i = \sa{\bar{X}}\sa{\bar{x}}\varphi(\bar{X},\bar{x})$ where $\varphi$ is quantifier-free. 
Notice that if $\length{\bar{X}} > 0$, then the function $\sem{\alpha}$ is in $\Omega(2^{\size{\A}})$, as it was proven in~\cite{SalujaST95}. 
Therefore, we have that $\alpha_i = \sa{\bar{x}}\varphi(\bar{x})$. 
We conclude our proof with the following claim, from which we conclude that $\alpha = \sum_{i = 1}^k \alpha_i$ cannot be equivalent to the $\E{1}$-formula $\sa{x} \ex{y} (x < y)$. 
\begin{clm}
	Let $\beta = \sa{\bar{x}}\varphi(\bar{x})$	where $\varphi$ is quantifier free. 
	Then the function $\sem{\beta}$ is either null, greater or equal to $n$, or is in $\Omega(n^2)$, where $n$ is the size of the input structure.
\end{clm}
\proof
Assume that $\bar x = (x_1, \ldots, x_m)$, and notice that each atomic sub-formula in $\varphi(\bar{x})$ is either $(x_i = x_j)$, $(x_i < x_j)$, $\top$ or a negation thereof, where $i,j \in \{1, \ldots, m\}$. 
	Suppose $\sem{\beta}$ is not null and consider some $\R$-structure $\A$ such that $\sem{\beta}(\A) > 0$. Hence, there exists an assignment $\bar a = (a_1, \ldots, a_m)$
%	Let $\bar{a}$ be an assignment 
	for $\bar{x}$ such that $\A\models\varphi(\bar{a})$.
	Given this assignment, define an equivalence relation $\sim$ on $\{x_1, \ldots, x_m\}$ as follows: $x_i \sim x_j$ if and only if $a_i = a_j$, and assume that $\sim$ partitions $\{x_1, \ldots, x_m\}$ into $\ell$ equivalence classes, where $\ell \geq 1$.
%	It can be seen that each assignment $\bar{a}'$ that has the same ordering over its variables\footnote{For example, if $\bar{a} = (a_1,a_2,a_3,a_4)$, such ordering may be $a_1 > a_3 = a_4 > a_2$, which has tree partitions.}	as $\bar{a}$ also satisfies $\A\models\varphi(\bar{a}')$. 
%	If this ordering has $k$ partitions then $\binom{\size{\A}}{k}$ assignments for $\bar{x}$ will satisfy it, and therefore, $\sem{\beta} \geq \binom{\size{\A}}{k}$. 
Then we have that there exist at least $\binom{\size{\A}}{\ell}$ assignments $\bar b$ for $\bar{x}$ such that $\A\models\varphi(\bar{b})$. Thus, given that if $\ell = 1$, then $\binom{\size{\A}}{\ell} = \size{\A}$, and if $\ell \geq 2$, then $\binom{\size{\A}}{\ell} \in \Omega(\size{\A}^2)$, we conclude that the claim holds.
\qed
%\martin{cambie order por ordering, creo que se entiende mejor}
%\marcelo{me parece que las nociones de ordering y particion no quedaban claras, por esto defini de manera precisa la nocion de particion y cambie la demostracion}

Now we show that $\QE{0} \not\subseteq \E{1}$. 
In Theorem \ref{theo-pi1-pnf} we proved that there is no formula in $\loge{1}$-PNF equivalent to the formula $\alpha = 2$. 
Every formula in $\E{1}$ can be expressed in $\loge{1}$-PNF, which implies that $2 \not\in \E{1}$. Therefore, given that $2 \in \QE{0}$ by the definition of this logic, we conclude that $\QE{0} \not\subseteq \E{1}$.

Finally, we prove that $\eqso(\loge{1})\subsetneq\eqso(\logu{1})$. 
For inclusion, let $\alpha$ be a formula in $\eqso(\loge{1})$. 
Suppose that it is in $\loge{1}$-SNF, namely, $\alpha = c + \sum_{i = 1}^{n}\alpha_i$. 
Let $\alpha_i = \sa{\bar{X}}\sa{\bar{x}}\ex{\bar{y}}\varphi_i(\bar{X},\bar{x},\bar{y})$, where $\varphi_i$ is quantifier-free for each $\alpha_i$. 
We use the same construction used in \cite{SalujaST95}, and we obtain that the formula $\ex{\bar{y}}\varphi_i(\bar{X},\bar{x},\bar{y})$ is equivalent to $\sa{\bar{y}}\,[\varphi_i(\bar{X},\bar{x},\bar{y}) \wedge \fa{\bar{y}'}(\varphi_i(\bar{X},\bar{x},\bar{y}')\to\bar{y}\leq\bar{y}')]$ for every assignment to $(\bar{X},\bar{x})$. 
We do this replacement for each $\alpha_i$, and we obtain an equivalent formula to $\alpha$ in $\eqso(\logu{1})$.

To prove that the inclusion is proper, consider the $\eqso(\logu{1})$ formula $\sa{x} \fa{y}(y = x)$. 
This formula defines the following function over each ordered structure $\A$:
$$
\sem{\alpha}(\A) = 
\begin{cases}
1 &\A \text{ has one element}\\
0 &\text{ otherwise}.
\end{cases}
$$
Suppose that there exists an equivalent formula $\alpha$ in $\eqso(\loge{1})$. 
Also, suppose that it is in $\LL$-SNF, so $\alpha = \sum_{i = 1}^n\sa{\bar{X}}\sa{\bar{x}}\ex{\bar{y}}\varphi_i(\bar{X},\bar{x},\bar{y})$. 
Consider a structure $\A'$ with one element. 
We have that for some $i$, there exists an assignment $(\bar{B},\bar{b},\bar{a})$ for $(\bar{X},\bar{x},\bar{y})$ such that $\A' \models\varphi_i(\bar{B},\bar{b},\bar{a})$. 
Consider now the structure $\A''$ that is obtained by duplicating $\A'$, as we did for Theorem \ref{theo-pi1-pnf}. 
Note that $\A''\models\varphi_i(\bar{B},\bar{b},\bar{a})$, which implies that $\sem{\alpha}(\A' \uplus \A'') > 1$, which leads to a contradiction.
