%!TEX root = main.tex

For the first condition of Definition~\ref{def:cap}, notice that each $\pfp$ formula can be evaluated in deterministic polynomial space, the constant function $s$ can be trivially simulated in $\fpspace$, and $\fpspace$ is closed under exponential sum and multiplication. This suffices to show that the condition holds.
For the second condition, the proof is similar to the proof of Theorem~\ref{theo:capture-fp}. Let $f\in \fpspace$ defined over some $\R$ and $\ell\in\nat$ such that $\log_2\left( f(\enc(\A)) \right) \leq 2^{{|\A|}^\ell}$ for every $\A\in\ostr[\R]$  (i.e. $2^{{|\A|}^\ell}$ is an upper bound for the size of the output). Let $X$ be a second-order variable of arity $\ell$. Consider the linear order induced by $<$ over predicates of arity $\ell$ which can be defined by the following formula:
$$
\varphi_{<}(X,Y) = \ex{\bar{u}}\big[\neg X(\bar{u})\wedge Y(\bar{u})\wedge \fa{\bar{v}}\big(
\bar{u}<\bar{v}\to(X(\bar{u})\iff Y(\bar{v}))\big)\big].
$$
Namely, we use predicates to encode a number that will have most $2^{{|\A|}^\ell}$ bits. We define this encoding through the function $\tau\colon 2^{A^\ell}\to\nat$, such that $\tau(B)$ is equal to the number of predicates in $2^{A^\ell}$ that are smaller than $B$ with respect to the induced order. For example, we have that $\tau(\emptyset) = 0$ and $\tau(A^{\ell}) = 2^{{|\A|}^\ell}-1$. Furthermore, we can use a relation~$X$ to index a position in the binary output of $f(\enc(\A))$ as follows.
%Consider a polynomial space machine over the $\R$ that receives as input an $\R$-structure $\A$ and a number $p$ encoded by a relation $X$. Then the machine accepts if, and only if, the $p$-th bit of $f(\enc(\A))$ is $1$. 
Define the language:
\[
L = \{(\A,B)\mid B \subseteq A^{\ell}\text{ and the $\tau(B)$-th bit of $f(\enc(\A))$ is 1}\}.
\]
Since $L$ is in $\pspace$, it can be specified in $\pfp$ \cite{AbiteboulV89} by a formula $\Phi(X)$ such that $\A\models\Phi(B)$ if and only if $(\A,B)\in L$. Then, similarly as for the previous proof we define:
$$
\alpha := \sa{X} \Phi(X)\mult  \pa{Y}(\varphi_{<}(Y,X)\mapsto 2).
$$ 
where $\pa{Y}(\varphi_{<}(Y,X)\mapsto 2)$ takes the value $2^{\tau(X)}$ and $\alpha$ reconstructs the output of $f(\enc(\A))$. Using an argument analogous to the previous proof, we conclude that $\alpha\in\qso(\pfp)$ and $\sem{\alpha}(\A) = f(\enc(\A))$.
%\martin{Reescrib\'i varias l\'ineas de esta demostraci\'on}
