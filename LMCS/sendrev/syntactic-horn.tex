%!TEX root = main.tex
\newcommand{\pP}{\textit{P}}
\newcommand{\pN}{\textit{N}}
\newcommand{\pV}{\textit{V}}
\newcommand{\pT}{\textit{T}}
\newcommand{\pA}{\textit{A}}
\newcommand{\pNC}{\textit{NC}}
\newcommand{\pD}{\textit{D}}




The goal of this section is to define a class of functions in $\shp$ with easy decision counterparts and natural complete problems. To this end, we consider the notion of parsimonious reduction. Formally, a function $f\colon\Sigma^* \to \N$ is parsimoniously reducible to a function $g\colon\Sigma^* \to \N$ if there exists a function $h\colon\Sigma^* \to \Sigma^*$ such that $h$ is computable in polynomial time and $f(x) = g(h(x))$ for every $x \in \Sigma^*$. As mentioned at the beginning of this section, if $f$ can be parsimoniously reduced to $g$, then $L_g \in \ptime$ implies that $L_f \in \ptime$ and the existence of an FPRAS for $g$ implies the existence of an FPRAS for $f$. 

In the previous section, we showed that the class $\eqso(\logex{1})$ has good closure and approximation properties. Unfortunately, it is not clear whether it admits a {\em natural} complete problem under parsimonious reductions, where {\em natural} means any of the counting problems defined in this section or any other well-known counting problem (not one specifically designed to be complete for the class). On the other hand, $\totp$ admits a natural complete problem under parsimonious reductions, which is the problem of counting the number of inputs accepted by a monotone circuit~\cite{BCPPZ17}. However, the notion of monotone circuit used in \cite{BCPPZ17} does not correspond with the usual notion of monotone circuit~\cite{GS90}, that is, circuits with AND and OR gates but without negation. In this sense, we still lack a class of functions in $\shp$ with easy decision counterparts and a complete problem that is well known and has been widely studied. In this section, we follow a different approach to find such a class,
%Hence, in this section we follow a different approach to find a class of functions in $\shp$ with easy decision counterparts and natural complete problems, 
which is inspired by the approach followed in \cite{G92} that uses a restriction of second-order logic to Horn clauses for capturing $\ptime$ (over ordered structures). The following example shows how our approach works.
\marcelo{Cambie este parrafo para mencionar un resultado nuevo de Pagourtzis et al. donde muestran un problema completo para Totp, que es un poco mas natural}
\begin{exa} \label{ex-hornsat-esop1}
Let $\R = \{\pP(\cdot,\cdot), \pN(\cdot,\cdot), \pV(\cdot), \pNC(\cdot),<\}$. This vocabulary is used as follows to encode a Horn formula. A fact $\pP(c,x)$ indicates that propositional variable $x$ is a disjunct in a clause $c$, while $\pN(c,x)$ indicates that $\neg x$ is a disjunct in $c$. Furthermore, $\pV(x)$ holds if  $x$ is a propositional variable, and $\pNC(c)$ holds if $c$ is a clause containing only negative literals, that is, $c$ is of the form $(\neg x_1 \vee \cdots \vee \neg x_n)$.

To define $\chsat$, we consider an \so-formula $\varphi(\pT)$ over $\R$, where $\pT$ is a unary predicate, such that for every Horn formula $\theta$ encoded by an $\R$-structure $\A$, the number of satisfying assignments of $\theta$ is equal to $\sem{\sa{\pT} \varphi(\pT)}(\A)$. In particular, $\pT(x)$ holds if and only if $x$ is a propositional variable that is assigned value true.  More specifically, 
%$\varphi(\pT)$ is defined as follows:
\begin{align*}
\varphi(\pT) \; :=\;\;  & \fa{x} (\pT(x) \to \pV(x)) \ \wedge\\
& \fa{c}  (\pNC(c) \to \ex{x} (\pN(c,x) \wedge \neg \pT(x))) \ \wedge\\
& \fa{c} \fa{x} ([\pP(c,x) \wedge \fa{y} (\pN(c,y) \to \pT(y))] \to \pT(x)).
\end{align*}
%Given that $\uhorn$ is designed with the goal in mind of capturing $\chsat$, we expect $\varphi(\pT)$ to be a formula in $\uhorn$. However, if we rewrite it as a conjunction of clauses we obtain the following:
We can rewrite $\varphi(\pT)$ in the following way:
\begin{align*}
& \fa{x}  (\neg \pT(x) \vee \pV(x)) \ \wedge\\
& \fa{c}  (\neg \pNC(c) \vee \ex{x} (\pN(c,x) \wedge \neg \pT(x)))\ \wedge\\
& \fa{c} \fa{x}  (\neg \pP(c,x) \vee \ex{y} (\pN(c,y) \wedge \neg \pT(y)) \vee \pT(x)).
\end{align*}
%The resulting formula $\varphi(\pT)$ is not in $\uhorn$, but it can be easily transformed into a formula in this class  by introducing an auxiliary binary predicate $\pA$ defined as follows:
Moreover, by introducing an auxiliary predicate $\pA$ defined as 
\begin{align*}
\fa{c} \fa{x}  (\neg \pA(c,x) \leftrightarrow [\pN(c,x) \wedge \neg \pT(x)]),
\end{align*}
we can translate $\varphi(\pT)$ into the following equivalent formula:
\begin{align*}
\psi(\pT,\pA) \; := \;\;  & \fa{x} (\neg \pT(x) \vee \pV(x)) \ \wedge\\
& \fa{c} (\neg \textit{NC}(c) \vee \ex{x} \neg \textit{A}(c,x)) \ \wedge\\
& \fa{c} \fa{x}  (\neg \textit{P}(c,x) \vee \ex{y} \neg \textit{A}(c,y) \vee \textit{T}(x)) \ \wedge\\
& \fa{c} \fa{x} (\neg \textit{N}(c,x) \vee \textit{T}(x) \vee \neg \textit{A}(c,x)) \ \wedge \\
& \fa{c} \fa{x} (\textit{A}(c,x) \vee \textit{N}(c,x)) \ \wedge\\
& \fa{c} \fa{x} (\textit{A}(c,x) \vee \neg\textit{T}(x)).
\end{align*}
More precisely, we have that:
\begin{align*}
\sem{\sa{\pT} \varphi(\pT)}(\A) &= \sem{\sa{\pT} \sa{\pA} \psi(\pT,\pA)}(\A),
\end{align*}
 for every $\R$-structure $\A$ encoding a Horn formula. Therefore, the formula $\psi(\pT,\pA)$ also defines $\chsat$. More importantly, $\psi(\pT,\pA)$ resembles a conjunction of Horn clauses except for the use of negative literals of the form $\ex{v} \neg \textit{A}(u,v)$. \qed
\end{exa}
The previous example suggests that to define $\chsat$, we can use Horn formulae defined as follows. 
A positive literal is a formula of the form $X(\x)$, where $X$ is a second-order variable and $\x$ is a tuple of first-order variables, and a negative literal is a formula of the form $\ex{\v} \neg X(\u,\v)$, where $\u$ and $\v$ are tuples of first-order variables. Given a signature $\R$, a clause over $\R$ is a formula of the form $\fa{\x} (\varphi_1 \vee \cdots \vee \varphi_n)$, 
where each $\varphi_i$ ($1 \leq i \leq n$) is either a positive literal, a negative literal or an \fo-formula over $\R$.  A clause is said to be Horn if it contains at most one positive literal, and a formula is said to be Horn if it is a conjunction of Horn clauses. With this terminology, we define $\uhorn$ as the set of formulae $\psi$ such that $\psi$ is a Horn formula over a signature $\R$. 

As we have seen, we have that $\chsat \in \eqso(\uhorn)$. Moreover, one can show that $\eqso(\uhorn)$ forms a class of functions with easy decision counterparts, namely, $\eqso(\uhorn) \subseteq \totp$.
Thus, $\eqso(\uhorn)$ is a new alternative in our search for a class of functions in $\shp$ with easy decision counterparts and natural complete problems. Moreover, an even larger class for our search can be generated by extending the definition of $\uhorn$ with outermost existential quantification. 
Formally, a formula $\varphi$ is in $\ehorn$ if $\varphi$ is of the form $\ex{\bar x} \psi$ with $\psi$ a Horn formula. 

\begin{prop}\label{prop:ehorn-pe}
$\eqso(\ehorn) \subseteq \totp$.
\end{prop}
In this section, we identify a complete problem for $\eqso(\ehorn)$ under parsimonious reductions. Hence, to prove that $\eqso(\ehorn) \subseteq \totp$, it is enough to prove that such a problem is in $\totp$, as $\totp$ is closed under parsimonious reductions. We give this proof at the end of this section, after the complete problem has been identified.
%For the sake of simplicity, we postpone the proof of the previous proposition to after the proof of Theorem~\ref{sigma2hard}.

Interestingly, we have that both $\chsat$ and $\cdnf$ belong to $\eqso(\ehorn)$. 
An imperative question at this point is whether in the definitions of $\uhorn$ and $\ehorn$, it is necessary to allow negative literals of the form $\ex{\v} \neg X(\u,\v)$. Actually, this forces our Horn classes to be included in $\eqso(\logu{2})$ and not necessarily in $\eqso(\loge{2})$. The following result shows that this is indeed the case.

\begin{prop}\label{prop:hsat-not-sigma2}	
$\chsat \not\in \eqso(\loge{2})$.
\end{prop}
\proof
We use a similar proof to the one provided by the authors in \cite{SalujaST95} to separate the classes $\E{2}$ and $\U{2}$. Suppose that the statement is false, this is, $\chsat \in \eqso(\loge{2})$. We consider the signature $\R$ that we used as the encoding for a Horn formula (Example \ref{ex-hornsat-esop1}) and that the formula $\alpha \in \eqso(\loge{2})$ follows the encoding in the same way. From what we proved in Theorem \ref{theo-pnf-snf}, we have that every formula in $\eqso(\loge{2})$ can be rewritten in $\loge{2}$-PNF, so we assume that $\alpha$ is in this form. Let $\alpha = \sa{\bar{X}}\sa{\bar{x}}\exists\bar{y}\,\forall\bar{z}\,\varphi(\bar{X},\bar{x},\bar{y},\bar{z})$. Consider the following Horn formula $\Phi$:
$$
\Phi = p \wedge \bigwedge_{i = 1}^n (t_i \wedge p \to q) \wedge \neg q,
$$
where $n = \length{\bar{x}} + \length{\bar{y}} + 1$. Let $\A$ be the encoding of this formula. In the encoding, each variable appears as an element in the domain of $\A$. This formula is satisfiable, so $\sem{\alpha}(\A) \geq 1$. Let $(\bar{B},\bar{b},\bar{a})$ be an assignment to $(\bar{X},\bar{x},\bar{y})$ such that $\A\models\forall\bar{z}\,\varphi(\bar{B},\bar{b},\bar{a},\bar{z})$. Let $t_{\ell}$ be such that it does not appear in $\bar{b}$ or $\bar{a}$. Consider the induced substructure $\A'$ that is obtained by removing $t_{\ell}$ from $\A$ and $\bar{B}'$ as the subset of $\bar{B}$ obtained by deleting each appearance of $t_{\ell}$ in $\bar{B}$. We have that $\A'\models\forall\bar{z}\,\varphi(\bar{B},\bar{b},\bar{a},\bar{z})$. This is because each subformula of the form $\exists y \neg B_i$ is still true, and universal formulas are monotone over induced substructures. It follows that $\sem{\alpha}(\A') \geq 1$ which is not possible since $\A'$ encodes the formula
$$
\Phi' = p \wedge \bigwedge_{i = 1}^{\ell-1} (t_i \wedge p \to q) \wedge (p\to q) \wedge \bigwedge_{i = \ell+1}^{n} (t_i \wedge p \to q) \wedge \neg q,
$$
which is unsatisfiable. We arrive to a contradiction and we conclude that $\chsat$ is not in $\eqso(\ehorn)$.
\qed
%We conclude this section by showing 
Next we show that $\eqso(\ehorn)$ is the class we were looking for, as not only every function in $\eqso(\ehorn)$ has an easy decision counterpart, but also $\eqso(\ehorn)$ admits a natural complete problem under parsimonious reductions. More precisely, define 
$\shdhsat$ as the problem of counting the satisfying assignments of a formula $\Phi$ that is a disjunction of Horn formulae. Then we have that:

\begin{thm} \label{sigma2hard}
	$\shdhsat$ is $\eqso(\ehorn)$-complete under parsimonious reductions. 
\end{thm}
\proof
First we prove that $\shdhsat$ is in $\eqso(\ehorn)$. Recall that each instance of $\shdhsat$ is a disjunction of Horn formulas. Let $\R = \{\pP(\cdot,\cdot), \pN(\cdot,\cdot), \pV(\cdot), \pNC(\cdot), \pD(\cdot,\cdot)\}$. Each symbol in this vocabulary is used to indicate the same as in Example \ref{ex-hornsat-esop1}, with the addition of $\pD(d,c)$ which indicates that $c$ is a clause in the formula $d$. Recall that the formula
\begin{align*}
&\forall x \, (\neg \pT(x) \vee \pV(x)) \ \wedge\\
&\forall c \, (\neg \textit{NC}(c) \vee \exists x \, \neg \textit{A}(c,x)) \ \wedge\\
&\forall c \forall x \, (\neg \textit{P}(c,x) \vee \exists y \, \neg \textit{A}(c,y) \vee \textit{T}(x)) \ \wedge\\
&\forall c \forall x \, (\neg \textit{N}(c,x) \vee \textit{T}(x) \vee \neg \textit{A}(c,x)) \ \wedge\\
&\forall c \forall x \, (\textit{A}(c,x) \vee \textit{N}(c,x)) \ \wedge\\
&\forall c \forall x \, (\textit{A}(c,x) \vee \neg\textit{T}(x)).
\end{align*}
defines $\chsat$. We obtain the following formula $\psi(T,A)$ in $\ehorn$:
\begin{align*}
\exists d[&\forall x \, (\neg \pT(x) \vee \pV(x)) \ \wedge\\
&\forall c \, (\neg \pD(c,d)\vee \neg \textit{NC}(c) \vee \exists x \, \neg \textit{A}(c,x)) \ \wedge\\
&\forall c \forall x \, (\neg \pD(c,d)\vee\neg \textit{P}(c,x) \vee \exists y \, \neg \textit{A}(c,y) \vee \textit{T}(x)) \ \wedge\\
&\forall c \forall x \, (\neg \pD(c,d)\vee\neg \textit{N}(c,x) \vee \textit{T}(x) \vee \neg \textit{A}(c,x)) \ \wedge\\
&\forall c \forall x \, (\neg \pD(c,d)\vee\textit{A}(c,x) \vee \textit{N}(c,x)) \ \wedge\\
&\forall c \forall x \, (\neg \pD(c,d)\vee\textit{A}(c,x) \vee \neg\textit{T}(x))]
\end{align*}
effectively defines $\chsat$ as for every disjunction of Horn formulas $\theta = \theta_1\vee\cdots\vee\theta_m$ encoded by an $\R$-structure $\A$, the number of satisfying assignments of $\theta$ is equal to $\sem{\sa{\pT} \sa{\pA} \psi(\pT,\pA)}(\A)$.  Therefore, we conclude that $\shdhsat \in \eqso(\ehorn)$.

\vspace{1em}
We will now prove that $\shdhsat$ is hard for $\eqso$ over a signature $\R$ under parsimonious reductions. For each $\eqso(\ehorn)$ formula $\alpha$ over $\R$, we will define a polynomial-time procedure that computes a function $g_{\alpha}$. This function receives a finite $\R$-structure $\A$ and outputs an instance of $\shdhsat$ such that $\sem{\alpha}(\A) = \shdhsat(g_{\alpha}(\A))$. We suppose that $\alpha$ is in sum normal form and:
$$
\alpha = \sum_{i = 1}^{\text{\#clauses}} \sa{\bar{X}}\sa{\bar{x}}\exists\bar{y}\bigwedge_{j = 1}^{n}\forall\bar{z}\,\varphi^i_j(\bar{X},\bar{x},\bar{y},\bar{z}),
$$
where each $\varphi^i_j$ is a Horn clause.                                                                

Consider a finite $\R$-structure $\A$ with domain $A$. To simplify the proof, we extend our grammar to allow first-order constants. Consider each tuple $\bar{a}\in A^{\length{\bar{x}}}$, each $\bar{b}\in A^{\length{\bar{y}}}$ and each $\bar{c}\in A^{\length{\bar{z}}}$ as a tuple of first-order constants. The following formula defines the same function as $\alpha$:
$$
\sum_{i = 1}^{\#clauses} \sum_{\bar{a}\in A^{\length{\bar{x}}}} \sa{\bar{X}}\bigvee_{\bar{b}\in A^{\length{\bar{y}}}}\bigwedge_{j = 1}^{n}\bigwedge_{\bar{c}\in A^{\length{\bar{z}}}}\varphi^i_j(\bar{X},\bar{a},\bar{b},\bar{c}).
$$
Note that each $\fo$ formula over $(\bar{x},\bar{y},\bar{z})$ in each $\varphi^i_j$ has no free variables. Therefore, we can evaluate each of these in polynomial time and replace them by $\perp$ and $\top$ where it corresponds. Each $\varphi^i_j$ will be of the form $\perp \vee\, \chi^i_j(\bar{X})$ or $\top \vee \chi^i_j(\bar{X})$ where $\chi^i_j$ is a disjunction of $\neg X_{\ell}$'s and at most one $X_{\ell}$. The formulas of the form $\top \vee \chi^i_j(\bar{X})$ can be removed entirely, and the formulas of the form $\perp \vee\, \chi^i_j(\bar{X})$ can be replaced by $\chi^i_j(\bar{X})$. We obtain the formula
$$
\sum_{i = 1}^{m}\sa{\bar{X}}\bigvee_{j = 1}^{\#d}\bigwedge_{k = 1}^{\#c}\psi^{i}_{j,k}(\bar{X})
$$
where every $\psi^{i}_{j,k}(\bar{X})$ is a disjunction of $\neg X_{\ell}$'s and zero or one $X_{\ell}$.

Our idea for the rest of the proof is to define $g_{\alpha}$ for each $\alpha = \sa{\bar{X}}\bigvee_{j = 1}^{\#d}\bigwedge_{k = 1}^{\#c}\psi^{i}_{j,k}(\bar{X})$, for $\alpha = c$ and for $\alpha = \beta_1 + \cdots + \beta_m$ where each $\beta_i$ is in one of the two previous cases.

If $\alpha$ is equal to $\sa{\bar{X}}\bigvee_{j = 1}^{\#d}\bigwedge_{k = 1}^{\#c}\psi_{j,k}(\bar{X})$ where $\psi_{j,k}(\bar{X})$ is a disjunction of $\neg X_{\ell}$'s and zero or one $X_{\ell}$, then we obtain the {\bf propositional formula} $g_{\alpha}(\A) = \bigvee_{j = 1}^{\#d}\bigwedge_{k = 1}^{\#c}\psi_{j,k}(\bar{X})$ over the propositional alphabet $\{X(\bar{e}) \mid X \in \bar{X} \text{ and } \bar{e}\in A^{\arity(X)} \}$ which has exactly $\sem{\alpha}(\A)$ satisfying assignments and is precisely a disjunction of Horn formulas.

If $\alpha$ is equal to a constant $c$, then we define $g_{\alpha}(\A)$ as the following formula that has exactly $c$ satisfying assignments:
$$
g_{\alpha}(\A) = \bigvee_{i = 1}^{c}\neg t_1 \wedge \cdots \wedge \neg t_{i-1} \wedge t_i \wedge \neg t_{i+1} \wedge \cdots \wedge \neg p_c.
$$ 
If $\alpha = \beta_1 + \cdots + \beta_m$, let $g_{\beta_i}(\A) = \bigvee_{j = 1}^{\#d}\bigwedge_{k = 1}^{\#c}\theta^i_{j,k}$ for each $\beta_i$ where each $\theta^i_{j,k}$ is a Horn clause. Let $\Theta_i = g_{\beta_i}(\A)$. We rename the variables in each $\Theta_i$ so none of them are mentioned in any other $\Theta_j$. We add $m$ new variables $t_1,\ldots,t_m$ and we define:
\begin{align*}
g_{\alpha}(\A) = &\bigvee_{i = 1}^{\#d}(\bigwedge_{j = 1}^{\#c}\theta^1_{i,j} \wedge (\bigwedge\limits_{\substack{\text{each } t\\ \text{ mentioned in}\\ \Theta_2,\ldots,\Theta_{m}}}t) \wedge (t_1 \wedge \bigwedge_{\ell = 2}^{m} \neg t_{\ell})) \vee \\ 
&\bigvee_{i = 1}^{\#d}(\bigwedge_{j = 1}^{\#c}\theta^2_{i,j} \wedge (\bigwedge\limits_{\substack{\text{each $t$}\\ \text{ mentioned in}\\ \Theta_1,\Theta_3,\ldots,\Theta_{m}}}t) \wedge (t_2 \wedge \bigwedge\limits_{\substack{\ell = 1 \\ \ell \neq 2}}^{m} \neg t_{\ell})) \vee \cdots \vee\\ 
&\bigvee_{i = 1}^{\#d}(\bigwedge_{j = 1}^{\#c}\theta^m_{i,j} \wedge (\bigwedge\limits_{\substack{\text{each } t\\ \text{ mentioned in}\\ \Theta_2,\ldots,\Theta_{m-1}}}t) \wedge (t_m \wedge \bigwedge_{\ell = 1}^{m-1} \neg t_{\ell})).
\end{align*}
The formula is a disjunction of Horn formulas, and the number of satisfying assignments for this formula is exactly the sum of satisfying assignments for each $g_{\beta_i}(\A)$. This, at the same time, is equal to $\sem{\alpha}(\A)$. This covers all possible cases for $\alpha$, and the entire procedure takes polynomial time.
\qed
Now that we have a complete problem for $\eqso(\ehorn)$, we can provide a simple proof of Proposition~\ref{prop:ehorn-pe}.

\medskip

\noindent{\emph{Proof of Proposition~\ref{prop:ehorn-pe}.}}
%Let $\shdhsat$ be the problem of counting the number of satisfying assignments of a disjunction of Horn formulas. 
In~\cite{PagourtzisZ06}, Pagourtzis and Zachos gave a $\totp$ procedure that computes the number of satisfying assignments of a DNF formula. This procedure can be easily extended to receive Horn formulas, and furthermore, a disjunction of Horn formulas. Hence $\shdhsat$ is in $\totp$. As we saw in Theorem~\ref{sigma2hard}, $\shdhsat$ is complete for $\eqso(\ehorn)$ under parsimonious reductions. Let $\alpha$ be a formula in $\eqso(\ehorn)$ and let $g_{\alpha}$ be the reduction to $\shdhsat$. Then the $\totp$ procedure consist in computing $g_{\alpha}(\enc(\A))$ on input $\enc(\A)$ and then simulating the $\totp$ procedure for $\shdhsat$ on input $g_{\alpha}(\enc(\A))$. Therefore, we conclude that $\alpha$ defines a function in~$\totp$.
\qed

Finally, it is important to mention that from the previous proof one can easily derive that $\eqso(\ehorn) \equiv \#(\ehorn)$. Therefore, the framework in~\cite{SalujaST95} is enough for defining the class of problems in $\eqso(\ehorn)$.