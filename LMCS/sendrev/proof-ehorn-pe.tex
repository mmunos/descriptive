%Let $\shdhsat$ be the problem of counting the number of satisfying assignments of a disjunction of Horn formulas. 
%In~\cite{PagourtzisZ06}, Pagourtzis and Zachos gave a $\totp$ procedure that computes the number of satisfying assignments of a DNF formula. This procedure can be easily extended to receive Horn formulas, and furthermore, a disjunction of Horn formulas. Hence $\shdhsat$ is in $\totp$.
First we will prove that $\shdhsat$ is in $\totp$. For this consider a non-deterministic procedure that receives a $\dhsat$ formula $\Phi$ as input and does the following. First check if $\Phi$ is satisfiable. If it is not, stop, and if it is, create a dummy branch that simply stops. Back in the main branch pick a variable $x$ in $\Phi$ and create two formulas $\Phi_0$ and $\Phi_1$ where $x$ has been replaced by $\perp$ or $\top$ respectively. If only one of these is satisfiable, continue on this branch with the respective $\Phi_i$, and if both are, create a new branch for $\Phi_1$ and continue on this branch with $\Phi_0$. On each branch repeat the same instructions until no variables are left to replace. Since $\dhsat$ is in $\ptime$ each formula check can be done in $\ptime$, so the procedure takes at most $h(\vert\Phi\vert)\vert\Phi\vert$ steps, for some polynomial function $h$, and it produces exactly $\shdhsat(\Phi)+1$ branches. Hence $\shdhsat$ is in $\totp$. For the next part of the proof, notice that in Theorem~\ref{sigma2hard} we proved that $\shdhsat$ is complete for $\eqso(\ehorn)$ under parsimonious reductions. Let $\alpha$ be a formula in $\eqso(\ehorn)$ and let $g_{\alpha}$ be the reduction to $\shdhsat$. Then the $\totp$ procedure consist in computing $g_{\alpha}(\enc(\A))$ on input $\enc(\A)$ and then simulating the $\totp$ procedure for $\shdhsat$ on input $g_{\alpha}(\enc(\A))$. Therefore, we conclude that $\alpha$ defines a function in~$\totp$.
\martin{agregu\'e una demostraci\'on de que $\shdhsat$ esta en $\totp$}
