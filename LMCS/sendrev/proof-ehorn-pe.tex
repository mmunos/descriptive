%!TEX root = main.tex
%
%Let $\shdhsat$ be the problem of counting the number of satisfying assignments of a disjunction of Horn formulas. 
%In~\cite{PagourtzisZ06}, Pagourtzis and Zachos gave a $\totp$ procedure that computes the number of satisfying assignments of a DNF formula. This procedure can be easily extended to receive Horn formulas, and furthermore, a disjunction of Horn formulas. Hence $\shdhsat$ is in $\totp$.
%
As we mentioned before, $\totp$ is closed under parsimonious reductions, so we only need to show that 
%First, we prove that 
$\shdhsat$ is in $\totp$. For this consider a non-deterministic procedure that receives a $\dhsat$ formula $\Phi$ as input and does the following. First it checks whether $\Phi$ is satisfiable. If it is not, then it stops; otherwise, it creates a dummy branch that simply stops, and continues in the main branch. More precisely, it picks in the main branch a propositional variable $x$ in $\Phi$ and creates two formulae $\Phi_0$ and $\Phi_1$, where $x$ has been replaced by $\perp$ and $\top$, respectively. If only one of these is satisfiable, it continues on this branch with the respective $\Phi_i$, and if both are satisfiable, then it creates a new branch for $\Phi_1$ and continues on this branch with $\Phi_0$. On each branch, it repeats the same instructions until no variables are left to replace. Since $\dhsat$ is in $\ptime$, all the aforementioned checks can be done in polynomial time, so that the procedure takes at most $h(n)$ steps in each branch, where $n$ is the size of $\Phi$ and $h$ is some fixed polynomial. Moreover, the algorithm produces exactly $\shdhsat(\Phi)+1$ branches, from which we conclude that $\shdhsat$ is in $\totp$. 
%For the next part of the proof, notice that in Theorem~\ref{sigma2hard} we establish that $\shdhsat$ is complete for $\eqso(\ehorn)$ under parsimonious reductions. Let $\alpha$ be a formula in $\eqso(\ehorn)$ and let $g_{\alpha}$ be the reduction to $\shdhsat$. Then the $\totp$ procedure consist in computing $g_{\alpha}(\enc(\A))$ on input $\enc(\A)$ and then simulating the $\totp$ procedure for $\shdhsat$ on input $g_{\alpha}(\enc(\A))$. Therefore, we conclude that $\alpha$ defines a function in~$\totp$.
\martin{agregu\'e una demostraci\'on de que $\shdhsat$ esta en $\totp$}
\marcelo{simplifique la demostracion, por favor revisar}
