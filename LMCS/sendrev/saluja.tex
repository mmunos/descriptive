%!TEX root = main.tex

Inspired by the connection between $\shp$ and $\sfo$, a hierarchy of subclasses of $\sfo$ was introduced in~\cite{SalujaST95} 
by restricting the alternation of quantifiers in Boolean formulae.
Specifically, the \emph{$\sfo$-hierarchy} consists of the 
the classes $\E{i}$ and $\U{i}$ for every $i \geq 0$, where $\E{i}$ (resp., $\U{i}$) is defined as $\sfo$ but restricting the formulae used to be in $\loge{i}$ (resp., $\logu{i}$).
By definition, we have that $\U{0} = \E{0}$. Moreover, it is shown in~\cite{SalujaST95} that:
\[
\E{0} \; \subsetneq \; \E{1} \; \subsetneq \; \U{1} \; \subsetneq \; \E{2} \; \subsetneq \; \U{2} \; = \; \sfo 
\]
In light of the framework introduced in this paper, natural extensions of these classes are obtained by considering 
$\eqso(\loge{i})$ and $\eqso(\logu{i})$ for every $i \geq 0$, which form the \emph{$\eqso(\fo)$-hierarchy}.
Clearly, we have that $\E{i} \subseteq \QE{i}$ and $\U{i} \subseteq \QU{i}$. Indeed, each formula $\varphi(\bar{X}, \bar{x})$ in $\E{i}$ is equivalent to the formula $\sa{\bar X} \sa{\bar x} \varphi(\bar{X}, \bar{x})$ in $\QE{i}$, and likewise for $\U{i}$ and $\QU{i}$.
But what is the exact relationship between these two hierarchies?
To answer this question, we first introduce two normal forms for $\eqso(\LL)$ that helps us to characterize the expressive power of this quantitative logic.
A formula $\alpha$ in $\eqso(\LL)$ is in \emph{$\LL$-prenex normal form} ($\LL$-PNF) 
if $\alpha$ is of the form
$\sa{\bar{X}} \sa{\bar{x}} \varphi(\bar{X}, \bar{x})$,
where $\bar{X}$ and $\bar{x}$ are sequences of zero or more second-order and first-order variables, respectively, (as expected, $\sa{\bar{X}}\!$ and $\sa{\bar{x}}\!$ are the respective nestings of $\sa{X}$'s and $\sa{x}$'s) and $\varphi(\bar{X}, \bar{x})$ is a formula in $\LL$. Notice that 
a formula $\varphi(\bar{X}, \bar{x})$ in $\sh{\LL}$ is equivalent to the formula $\sa{\bar X} \sa{\bar x} \varphi(\bar{X}, \bar{x})$ in $\LL$-PNF. 
Moreover, a formula $\alpha$ in $\eqso(\LL)$ is in \emph{$\LL$-sum normal form} ($\LL$-SNF) if $\alpha$ is of the form $\sum_{i=1}^n \alpha_i$, where this is a shorthand notation for $\alpha_1+\cdots+\alpha_n$, and each $\alpha_i$ is in $\LL$-PNF. 
\begin{prop}\label{theo-pnf-snf}
Every formula in $\eqso(\LL)$ can be rewritten in $\LL$-SNF.
\end{prop}
\proof
Recall that a formula in $\eqso(\LL)$ is written on the following grammar:
\[
\alpha = \varphi \ \mid \ s \ \mid \ (\alpha + \alpha) \ \mid \ \sa{x} \alpha \ \mid \ \sa{X} \alpha,
\]
where $\varphi$ is a formula in $\LL$ and $s\in\nat$. We will construct a recursive function $\tau$ such that for every $\eqso(\LL)$ formula $\alpha$, it outputs an equivalent formula $\tau(\alpha)$ which is in $\LL$-SNF. If $\alpha = \varphi$, let $\tau(\alpha) = \alpha$. If $\alpha = s$, let $\tau(\alpha) = (\top \add \cdots \add \top)$ ($s$ times). We assume that for every sub-formula $\beta$ in $\alpha$, $\tau(\beta)$ is an equivalent formula in $\LL$-SNF. If $\alpha = (\alpha_1 + \alpha_2)$, let $\tau(\alpha) = (\tau(\alpha_1) + \tau(\alpha_2))$. If $\alpha = \sa{x}\beta$, then $\tau(\beta) = \sum_{i = 1}^{k}\beta_i$ for some $k$ where each $\beta_i$ is in $\LL$-PNF. We define $\tau(\alpha) = \sum_{i = 1}^{k}\sa{x}\beta_i$. If $\alpha = \sa{X}\beta$, then we proceed analogously as in the previous case. This covers all possible cases for $\alpha$ and we conclude the proof by taking $\tau(\alpha)$ as the desired rewrite of $\alpha$.
\qed
If a formula is in $\LL$-PNF then clearly the formula is in $\LL$-SNF.
Unfortunately, for some $\LL$ there exist formulae in $\eqso(\LL)$  that cannot be rewritten in $\LL$-PNF.
Therefore, to unveil the relationship between the $\sfo$-hierarchy and the $\eqso(\fo)$-hierarchy, we need to understand the boundary between PNF and SNF. We do this in the following theorem. 
\begin{thm}\label{theo-pi1-pnf}
For $i = 0,1$, there exists a formula $\alpha_i$ in $\QE{i}$ that is not equivalent to any formula in $\Sigma_i$-PNF. 
On the other hand, if $\logu{1} \subseteq \LL$ and $\LL$ is closed under conjunction and disjunction, then every formula in $\eqso(\LL)$ can be rewritten in $\LL$-PNF. 
\end{thm}
\proof
%!TEX root = main.tex

We divide the proof in three parts.

\vspace{1em}
First, we prove that the formula $\alpha_{0} = \left( \sa{X} 1 \right) + 1$ with $\arity(X) = 1$ (i.e. the function $2^{n}+1$, where $n$ is the size of the input structure) is not equivalent to any formula in $\loge{0}$-PNF. Suppose that there exists some formula $\alpha = \sa{\bar{X}}\sa{\bar{x}}\varphi(\bar{X},\bar{x})$ in $\loge{0}$-PNF that is equivalent to the $\eqso(\loge{0})$ formula $\alpha_0$.
In \cite{SalujaST95}, it was proved that if $\length{\bar{X}} > 0$, then the function defined by $\alpha$, for big enough structures, is always even which is not possible in our case.
On the other hand, if $\alpha$ is of the form $\sa{\bar{x}}\varphi(\bar{x})$, then $\alpha$ defines a polynomially bounded function which also leads to a contradiction.

\vspace{1em}
Second, we prove that the formula $\alpha_{1} = 2$ (i.e. the constant $2$) is not equivalent to any formula in $\loge{1}$-PNF. Suppose that there exists some formula $\alpha = \sa{\bar{X}}\sa{\bar{x}}\exists\bar{y}\, \varphi(\bar{X},\bar{x},\bar{y})$ in $\loge{1}$-PNF that is equivalent to the $\eqso(\loge{1})$ formula $2$. 
First, if $\length{\bar{X}} = \length{\bar{x}} = 0$, then the function defined by $\alpha$ is never greater than 1. 
Therefore, suppose that $\length{\bar{X}} > 0$ or $\length{\bar{x}} > 0$, and consider any ordered structure $\A$. 
Since $\sem{\alpha}(\A) = 2$, there exist at least two assignments $(\bar{B}_1,\bar{b}_1,\bar{a}_1)$, $(\bar{B}_2,\bar{b}_2,\bar{a}_2)$ to $(\bar{X},\bar{x},\bar{y})$ such that for both, $\A\models\varphi(\bar{B}_i,\bar{b}_i,\bar{a}_i)$. Now consider the ordered structure $\A'$ that is obtained by duplicating $\A$. This is, each half of $\A'$ is isomorphic to $\A$. Note that $\A'\models\varphi(\bar{B}_i,\bar{b}_i,\bar{a}_i)$ for $i = 1,2$ and there exists a third assignments $(\bar{B}_1',\bar{b}_1',\bar{a}_1')$ that is isomorphic to $(\bar{B}_1,\bar{b}_1,\bar{a}_1)$ but in the other half to the structure such that $\A'\models\varphi(\bar{B}_1',\bar{b}_1',\bar{a}_1')$. We have that $\sem{\alpha}(\A) \geq 3$ which is a contradiction.

\vspace{1em}
We now show that if $\LL$ contains $\logu{1}$ and is closed under conjunction and disjunction, then for every formula $\alpha$ in $\eqso(\LL)$ there is an equivalent formula $\beta$ in $\LL$-PNF. As in Theorem \ref{theo-pnf-snf}, we show a recursive function $\tau$ that produces such formula. As we showed, there exists an equivalent formula in $\LL$-SNF, so we assume that $\alpha$ is in that form. Let $\alpha = \sum_{i = 1}^n \alpha_i$ where each $\alpha_i$ is in $\LL$-SNF. 
Without lost of generality, we assume that each $\alpha_i = \sa{\bar{X}}\sa{\bar{x}}\varphi_i(\bar{X},\bar{x})$ with $\length{\bar{X}} > 0$ and $\length{\bar{x}} > 0$. If not, we replace each $\alpha_i$ for the equivalent formula $\sa{\bar{X}} \sa{Y}\sa{\bar{x}}\sa{y}(\varphi_i(\bar{X},\bar{x})\wedge\forall z\,Y(z) \wedge \forall z(y \leq z))$.

Now we begin describing the function $\tau$. If $\alpha = \sa{\bar{X}}\sa{\bar{x}}\varphi(\bar{X},\bar{x})$, then the formula is already in $\LL$-PNF so we define $\tau(\alpha) = \alpha$. If $\alpha = \alpha_1 + \alpha_2$, then we assume that $\tau(\alpha_1) = \sa{\bar{X}}\sa{\bar{x}}\varphi(\bar{X},\bar{x})$ and $\tau(\alpha_2) = \sa{\bar{Y}}\sa{\bar{y}}\psi(\bar{Y},\bar{y})$. The construction that we will provide for this function works by identifying a ``first'' assignment for both $(\bar{X},\bar{x})$ and $(\bar{Y},\bar{y})$ and a ``last'' assignment for both $(\bar{X},\bar{x})$ and $(\bar{Y},\bar{y})$. These are identified by the following formulas:
\begin{align*}
\gamma_{\text{first}}(\bar{X},\bar{x}) &= \bigwedge_{i = 1}^{\length{\bar{X}}} \forall\bar{z}\neg X_i(\bar{z}) \wedge \forall\bar{z}(\bar{x}\leq\bar{z}), \\
\gamma_{\text{last}}(\bar{X},\bar{x}) &= \bigwedge_{i = 1}^{\length{\bar{X}}} \forall\bar{z} X_i(\bar{z}) \wedge \forall\bar{z}(\bar{z}\leq\bar{x}).
\end{align*}
Similarly, we define the formulas $\gamma_{\text{first}}(\bar{Y},\bar{y})$ and $\gamma_{\text{last}}(\bar{Y},\bar{y})$ (for the sake of simplicity we reuse the names $\gamma_{\text{first}}$ and $\gamma_{\text{last}}$).
In other words, the ``first'' assignment is the one where every second-order predicate is empty and the first-order assignment is the lexicographically smallest, and the ``last'' assignment is the one where every second-order predicate is full and the first-order assignment is the lexicographically greatest. We also need to identify assignments that are not first and are not last. We do this by negating the two formulas above and grouping together the first-order variables:
\begin{align*}
\gamma_{\text{not first}}(\bar{X},\bar{x}) &= \exists\bar{z}(\bar{z}_0 < \bar{x} \vee \bigvee_{i = 1}^{\length{\bar{X}}}X(\bar{z}_i)), \\
\gamma_{\text{not last}}(\bar{X},\bar{x}) &= \exists\bar{z}(\bar{x} < \bar{z}_0 \vee \bigvee_{i = 1}^{\length{\bar{X}}}\neg X(\bar{z}_i)),
\end{align*}
where $\bar{z} = (\bar{z}_0,\bar{z}_1,\ldots,\bar{z}_{\length{\bar{X}}})$. The following formula is equivalent to $\alpha$:
\begin{align}
\sa{\bar{X}}\sa{\bar{x}}\sa{\bar{Y}}\sa{\bar{y}}[&(\varphi(\bar{X},\bar{x})\wedge\gamma_{\text{not first}}(\bar{X},\bar{x})\wedge\gamma_{\text{first}}(\bar{Y},\bar{y}))\vee \label{eq:partition1} \\
&(\varphi(\bar{X},\bar{x})\wedge\gamma_{\text{first}}(\bar{X},\bar{x})\wedge\gamma_{\text{last}}(\bar{Y},\bar{y}))\vee \label{eq:partition2}\\
&(\psi(\bar{Y},\bar{y})\wedge\gamma_{\text{first}}(\bar{X},\bar{x})\wedge\gamma_{\text{not last}}(\bar{Y},\bar{y}))\vee \label{eq:partition3}\\
&(\psi(\bar{Y},\bar{y})\wedge\gamma_{\text{last}}(\bar{X},\bar{x})\wedge\gamma_{\text{last}}(\bar{Y},\bar{y}))]. \label{eq:partition4}
\end{align}
To show that the formula is indeed equivalent to $\alpha$, note that the formulas in lines (\ref{eq:partition1}) and (\ref{eq:partition2}) form a partition over the assignments of $(\bar{X},\bar{x})$, while fixing an assignment for $(\bar{Y},\bar{y})$, and the formulas in lines (\ref{eq:partition3}) and (\ref{eq:partition4}) form a partition over the assignments of $(\bar{Y},\bar{y})$, while fixing an assignment for $(\bar{X},\bar{x})$. Altogether, the four lines define pairwise disjoint assignments for $(\bar{X},\bar{x}),(\bar{Y},\bar{y})$. With this, it is straightforward to show that the above formula is equivalent to $\alpha$. However, the formula is not yet in the correct form since it has existential quantifiers in the subformulas $\gamma_{\text{not first}}$ and $\gamma_{\text{not last}}$. To solve this, first let take a close look to the complete formula:
\begin{align*}
\sa{\bar{X}}\sa{\bar{x}}\sa{\bar{Y}}\sa{\bar{y}}[&(\varphi(\bar{X},\bar{x})\wedge\exists\bar{v}(\bar{v}_0 < \bar{x} \vee \bigvee_{i = 1}^{\length{\bar{X}}}X(\bar{v}_i))\wedge\bigwedge_{i = 1}^{\length{\bar{Y}}} \forall\bar{z}\neg Y_i(\bar{z}) \wedge \forall\bar{z}(\bar{y}\leq\bar{z}))\vee\\
&(\varphi(\bar{X},\bar{x})\wedge\bigwedge_{i = 1}^{\length{\bar{X}}} \forall\bar{z}\neg X_i(\bar{z}) \wedge \forall\bar{z}(\bar{x}\leq\bar{z})\wedge\bigwedge_{i = 1}^{\length{\bar{Y}}} \forall\bar{z} Y_i(\bar{z}) \wedge \forall\bar{z}(\bar{z}\leq\bar{y}))\vee\\
&(\psi(\bar{Y},\bar{y})\wedge\bigwedge_{i = 1}^{\length{\bar{X}}} \forall\bar{z}\neg X_i(\bar{z}) \wedge \forall\bar{z}(\bar{x}\leq\bar{z})\wedge\exists\bar{w}(\bar{y} < \bar{w}_0 \vee \bigvee_{i = 1}^{\length{\bar{Y}}}\neg Y(\bar{w}_i))\vee\\
&(\psi(\bar{Y},\bar{y})\wedge\bigwedge_{i = 1}^{\length{\bar{X}}} \forall\bar{z} X_i(\bar{z}) \wedge \forall\bar{z}(\bar{z}\leq\bar{x})\wedge\bigwedge_{i = 1}^{\length{\bar{Y}}} \forall\bar{z} Y_i(\bar{z}) \wedge \forall\bar{z}(\bar{z}\leq\bar{y}))].
\end{align*}
To construct an equivalent formula that is in the correct form, we define $\bar{u} = (\bar{v},\bar{w})$ and we replace the first-order quantifiers by a first-sum and count the first assignment to $\bar{v}$ and $\bar{w}$ that satisfies the formula. A similar construction was used in \cite{SalujaST95}. Then the final formula equivalent to $\alpha$ is the following:
\begin{align*}
\sa{\bar{X}}&\sa{\bar{Y}}\sa{\bar{x}}\sa{\bar{y}}\sa{\bar{u}}[ \\
&(\varphi(\bar{X},\bar{x})\wedge(\bar{v}_0 < \bar{x} \vee \bigvee_{i = 1}^{\length{\bar{X}}}X(\bar{v}_i))\wedge \forall\bar{u}'((\bar{v}_0' < \bar{x} \vee \bigvee_{i = 1}^{\length{\bar{X}}}X(\bar{v}_i'))\to\bar{u}\leq\bar{u}') \wedge \\
&\bigwedge_{i = 1}^{\length{\bar{Y}}} \forall\bar{z}\neg Y_i(\bar{z}) \wedge \forall\bar{z}(\bar{y}\leq\bar{z}))\vee\\
&
(\varphi(\bar{X},\bar{x})\wedge\bigwedge_{i = 1}^{\length{\bar{X}}} \forall\bar{z}\neg X_i(\bar{z}) \wedge \forall\bar{z}(\bar{x}\leq\bar{z})\wedge\bigwedge_{i = 1}^{\length{\bar{Y}}} \forall\bar{z} Y_i(\bar{z}) \wedge \forall\bar{z}(\bar{z}\leq\bar{y})\wedge\forall\bar{u}'(\bar{u}\leq\bar{u}'))\vee
\\
&(\psi(\bar{Y},\bar{y})\wedge\bigwedge_{i = 1}^{\length{\bar{X}}} \forall\bar{z}\neg X_i(\bar{z}) \wedge \forall\bar{z}(\bar{x}\leq\bar{z})\wedge \\
&(\bar{y} < \bar{w}_0 \vee \bigvee_{i = 1}^{\length{\bar{Y}}}\neg Y(\bar{w}_i))\wedge\forall\bar{u}'(\bar{y} < \bar{w}_0' \vee \bigvee_{i = 1}^{\length{\bar{Y}}}\neg Y(\bar{w}_i'))\to\bar{u}\leq\bar{u}')\vee \\
&(\psi(\bar{Y},\bar{y})\wedge\bigwedge_{i = 1}^{\length{\bar{X}}} \forall\bar{z} X_i(\bar{z}) \wedge \forall\bar{z}(\bar{z}\leq\bar{x})\wedge\bigwedge_{i = 1}^{\length{\bar{Y}}} \forall\bar{z} Y_i(\bar{z}) \wedge \forall\bar{z}(\bar{z}\leq\bar{y})\wedge\forall\bar{u}'(\bar{u}\leq\bar{u}'))].
\end{align*}
Finally, consider a $\eqso(\LL)$ formula $\alpha$ in $\LL$-SNF. If $\alpha = \sum_{i = 1}^n\alpha_i$, then by induction we can consider $\alpha = \alpha_1 + (\sum_{i = 2}^n\alpha_i)$ and use $\tau(\alpha_1 + \tau(\sum_{i = 2}^n\alpha_i))$ as the rewrite of $\alpha$, which satisfies the condition in the hypothesis.
\qed

\begin{figure*}
%\begin{center}
%\begin{tikzpicture}
%\node[rectw] (n1) {$\E{0}$};
%\node[rectw, right=0.5cm of n1] (n2) {};
%\node[rectw, above=0.3cm of n2] (n3) {$\E{1}$}
%	edge[draw=white] node {\rotatebox{45}{$\subsetneq$}} (n1);
%\node[rectw, below=0.3cm of n2] (n4) {$\QE{0}$}
%        edge[draw=white] node {\rotatebox{315}{$\subsetneq$}} (n1);
%\node[rectw, right=0.3cm of n2] (n5) {$\QE{1}$}
%        edge[draw=white] node {\rotatebox{315}{$\subsetneq$}} (n3)
%         edge[draw=white] node {\rotatebox{45}{$\subsetneq$}} (n4);
%\node[rectw, right=0.3cm of n5] (n6) {$\U{1}$}       
%        edge[draw=white] node {$\subsetneq$} (n5);
%\node[rectw, right=0.3cm of n6] (n7) {$\QU{1}$}       
%        edge[draw=white] node {$=$} (n6);
%\node[rectw, right=0.3cm of n7] (n8) {$\E{2}$}       
%        edge[draw=white] node {$\subsetneq$} (n7);
%\node[rectw, right=0.3cm of n8] (n9) {$\QE{2}$}       
%        edge[draw=white] node {$=$} (n8);        
%\node[rectw, right=0.3cm of n9] (n10) {$\U{2}$}       
%        edge[draw=white] node {$\subsetneq$} (n9);
%\node[rectw, right=0.3cm of n10] (n11) {$\QU{2}$}       
%        edge[draw=white] node {$=$} (n10); 
%\node[rectw, right=0.3cm of n11] (n12) {$\sfo$}       
%        edge[draw=white] node {$=$} (n11); 
%\end{tikzpicture}
%\end{center}
\begin{center}
	\begin{tikzpicture}
	\node[rectw] (n1) {$\E{0}$};
	\node[rectw, right=0.5cm of n1] (n2) {};
	\node[rectw, above=0.3cm of n2] (n3) {$\E{1}$}
		edge[draw=white] node {\rotatebox{45}{$\subsetneq$}} (n1);
	\node[rectw, below=0.5cm of n2] (n4) {$\QE{0}$}
		edge[draw=white] node {\rotatebox{315}{$\subsetneq$}} (n1);
	\node[rectw, right=0.5cm of n2] (n5) {$\QE{1}$}
		edge[draw=white] node {\rotatebox{315}{$\subsetneq$}} (n3)
		edge[draw=white] node {\rotatebox{45}{$\subsetneq$}} (n4);
	\node[rectw, right=0.8cm of n5] (n6) {$\QU{1}$}       
		edge[draw=white] node {$\subsetneq$} (n5);
	\node[rectw, below=0.3cm of n6] (n7) {$\U{1}$}       
		edge[draw=white] node {\rotatebox{90}{$=$}} (n6);
	\node[rectw, right=1.0cm of n6] (n8) {$\QE{2}$}       
		edge[draw=white] node {$\subsetneq$} (n6);
	\node[rectw, below=0.3cm of n8] (n9) {$\E{2}$}       
		edge[draw=white] node {\rotatebox{90}{$=$}} (n8);        
	\node[rectw, right=1.0cm of n8] (n10) {$\QU{2}$}       
		edge[draw=white] node {$\subsetneq$} (n8);
	\node[rectw, below=0.3cm of n10] (n11) {$\U{2}$}       
		edge[draw=white] node {\rotatebox{90}{$=$}} (n10); 
	\node[rectw, right=0.5cm of n10] (n12) {$\sfo$}       
		edge[draw=white] node {$=$} (n10); 
	\end{tikzpicture}
\end{center}
%\vspace{1cm}
%\begin{center}
%	\begin{tikzpicture}
%	\node[rectw] (n1) {$\E{0}$};
%	\node[rectw, right=1.2cm of n1] (n2) {};
%	\node[rectw, above=0.2cm of n2] (n3) {$\E{1}$}
%	edge[draw=white] node {\rotatebox{45}{$\subsetneq$}} (n1);
%	\node[rectw, below=0.1cm of n2] (n4) {$\QE{0}$}
%	edge[draw=white] node {\rotatebox{315}{$\subsetneq$}} (n1);
%	\node[rectw, right=1.2cm of n2] (n5) {$\QE{1}$}
%	edge[draw=white] node {\rotatebox{315}{$\subsetneq$}} (n3)
%	edge[draw=white] node {\rotatebox{45}{$\subsetneq$}} (n4);
%	\node[rectw, right=1.3cm of n5] (n6) {\rotatebox{90}{$=$}}       
%	edge[draw=white] node {$\subsetneq$} (n5);
%	\node[rectw, above=0cm of n6] (n13) {$\U{1}$};
%	\node[rectw, below=0cm of n6] (n7) {$\QU{1}$};      
%	\node[rectw, right=1.8cm of n6] (n8) {\rotatebox{90}{$=$}}       
%	edge[draw=white] node {$\subsetneq$} (n6);
%	\node[rectw, above=0cm of n8] (n14) {$\E{2}$};
%	\node[rectw, below=0cm of n8] (n9) {$\QE{2}$};       
%	\node[rectw, right=1.8cm of n8] (n10) {\rotatebox{90}{$=$}}       
%	edge[draw=white] node {$\subsetneq$} (n8);
%	\node[rectw, above=0cm of n10] (n15) {$\U{2}$};
%	\node[rectw, below=0cm of n10] (n11) {$\QU{2}$};      
%	edge[draw=white] node {\rotatebox{90}{$=$}} (n10); 
%	\node[rectw, right=0.5cm of n10] (n12) {$\sfo$}       
%	edge[draw=white] node {$=$} (n10); 
%	\end{tikzpicture}
%\end{center}
\caption{The relationship between the $\sfo$-hierarchy and the $\eqso(\fo)$-hierarchy, where $\E{1}$ and $\QE{0}$ are incomparable. \label{fig-sfo-eqso}}
\vspace{-0.1cm}
\end{figure*}

As a consequence of Proposition~\ref{theo-pnf-snf} and Theorem~\ref{theo-pi1-pnf}, we obtain that $\E{i} \subsetneq \QE{i}$ for $i = 0,1$, and that $\sh{\LL} = \eqso(\LL)$ for $\LL$ equal to  $\Pi_1$, $\Sigma_2$ or $\Pi_2$. The following proposition completes our picture of the relationship between the $\sfo$-hierarchy and the $\eqso(\fo)$-hierarchy.
\begin{prop}\label{prop-rest}
The following properties hold:
\begin{itemize}
\item $\QE{0}$ and $\E{1}$ are incomparable, that is, $\E{1} \not\subseteq \QE{0}$ and $\QE{0} \not\subseteq \E{1}$,
\item $\QE{1} \subsetneq \QU{1}$.
\end{itemize}
\end{prop}
\proof
We give this proof in three parts.

\vspace{1em}
First, we show that $\QE{0} \not\subseteq \E{1}$. By contradiction, suppose that there is a $\QE{0}$ formula $\alpha$ over some signature $\R$ such that defines the following function. For every finite $\R$-structure with $n$ elements, and where every predicate in $\R$ is empty, $\alpha(\enc(\A)) = n - 1$. We use the following claim.
\begin{claim}
	Let $\alpha = \sa{\bar{x}}\varphi(\bar{x})$	where $\varphi$ is quantifier free. Then the function defined by $\alpha$ is either null, greater or equal to $n$, or is in $\Omega(n^2)$.
\end{claim}
\begin{proof}
	Suppose that the function defined by $\alpha$ is not $0$ and that $\varphi$ is in DNF. Furthermore, suppose $\bar{x} = (x_1,\ldots,x_{\length{\bar{x}}})$.
	Then $\alpha = \sa{\bar{x}} \varphi_1(\bar{x}) \vee \cdots \vee \varphi_n(\bar{x})$. Since $\alpha$ is not null, then some $\varphi_i$ must be satisfiable. This is, the function defined by $\sa{\bar{x}}\varphi(\bar{x})$ is not null.
	We will prove by induction on $\length{\bar{x}}$ that the function defined by $\sa{\bar{x}}\varphi(\bar{x})$ is either greater or equal to $n$, or in $\Omega(n^2)$.
	We address the case $\length{\bar{x}}= 1$, then $\alpha = \sa{x}\bigwedge\psi(x)$. If any $\psi(x) = (x = x)$ or $\neg(x < x)$, then we can eliminate it and we obtain the same function. If any $\psi = (x < x)$ or $\neg(x=x)$, then the function becomes null. If $\psi(x) = R(x,\ldots,x)$ for some $R\in\R$ the function becomes null for the structures we are considering. If $\psi(x) = \neg R(x,\ldots,x)$, we can eliminate it and for the structures we are considering we obtain the same function. The only possible $\alpha$ left is $\alpha = \sa{x}\top$ which is equal to the function $n$. This covers all possible cases for $\length{\bar{x}} = 1$. Now suppose that it holds for $\length{\bar{x}} = k$ and suppose $\alpha = \sa{\bar{x}}\bigwedge\psi(\bar{x})$ for $\length{\bar{x}} = k+1$. If any $\psi(\bar{x}) = (x_i = x_j)$ where $i \neq j$, then $\alpha$ describes the same function as $\alpha$ where $x_j$ has been replaced by $x_i$. In this formula the tuple of first-order variables has $k$ elements so the function it describes if one of the mentioned in the hypothesis. If $i = j$, then we can eliminate it and obtain the same function. If any $\psi(\bar{x}) = R(\bar{v})$ or $\neg R(\bar{v})$ where $\bar{v}$ is a sub-tuple of $\bar{x}$ then we can either eliminate it or the function becomes null, following the same argument as in the case $\length{\bar{x}} = 1$. If any $\psi(\bar{x}) = \neg(x_i = x_j)$ or $(x_i < x_j)$ where $i = j$, then the function becomes null. If any $\psi(\bar{x}) = \neg(x_i < x_j)$ where $i = j$, we can eliminate it. The remaining formulas in $\bigwedge\psi(\bar{x})$ are either $\neg(x_i = x_j)$, $(x_i<x_j)$ or $\neg(x_i<x_j)$. If the formula violates transitivity in $<$ (for example, $x < y \wedge y < z \wedge z < x$), then the function $\alpha$ describes is null. Therefore, there is some order over $\bar{x}$ that satisfies $\bigwedge\psi(\bar{x})$. Consider the formula that describes this order (like $x_1 < x_3 \wedge x_3 < x_4 \wedge x_4 < x_2$). The function $\alpha$ describes is greater or equal to the one this formula describes, which is exactly $\binom{n}{\length{\bar{x}}}$ which is in $\Omega(n^{\length{\bar{x}}}) \subseteq \Omega(n^2)$ if $\length{\bar{x}} > 1$. This concludes the proof of the claim.
\end{proof}
We suppose that $\alpha$ is in SNF, this is, $\alpha = \sum_{i = 1}^n\alpha_i$. Since $\alpha$ is not null, consider some $\alpha_i$ that describes a non-null function. Let $\alpha_i = \sa{\bar{X}}\sa{\bar{x}}\varphi(\bar{X},\bar{x})$, where $\varphi$ is quantifier-free. Note that if $\length{\bar{X}} > 0$, then the function $\alpha$ describes is in $\Omega(2^n)$, as it was proven by the authors in \cite{SalujaST95}. We have that $\alpha_i = \sa{\bar{x}}\varphi(\bar{x})$, as we proved in the claim, describes either some function greater or equal to $n$, or in $\Omega(n^2)$, which leads to a contradiction. Lastly, note that the formula $\sa{x}\exists y(x < y)$ is in $\E{0}$ and describes the function $n-1$, which concludes the proof.

\vspace{1em}
Now we show that $\E{1}\not\subseteq\QE{0}$. In Theorem \ref{theo-pi1-pnf} we proved that there is no formula in $\loge{1}$-PNF equivalent to the formula $\alpha = 2$. Every formula in $\E{1}$ can be expressed in $\loge{1}$-PNF, which implies that $2 \in \QE{0}$ and $2 \not\in \E{1}$.

\vspace{1em}
Lastly, we prove that $\eqso(\loge{1})\subsetneq\eqso(\logu{1})$. For inclusion, let $\alpha$ be a formula in $\eqso(\loge{1})$. Suppose that it is in $\loge{1}$-SNF. This is, $\alpha = c + \sum_{i = 1}^{n}\alpha_i$. Let $\alpha_i = \sa{\bar{X}}\sa{\bar{x}}\exists\bar{y}\,\varphi_i(\bar{X},\bar{x},\bar{y})$, where $\varphi_i$ is quantifier-free, for each $\alpha_i$. We use the same construction used in \cite{SalujaST95}, and we obtain that the formula $\exists\bar{y}\,\varphi_i(\bar{X},\bar{x},\bar{y})$ is equivalent to $\sa{\bar{y}}\,\varphi_i(\bar{X},\bar{x},\bar{y}) \wedge \forall\bar{y}'(\varphi_i(\bar{X},\bar{x},\bar{y}')\to\bar{y}\leq\bar{y}')$ for every assignment to $(\bar{X},\bar{x})$. We do this replacement for each $\alpha_i$ and we obtain an equivalent formula in $\eqso(\logu{1})$.

To prove that the inclusion is proper, consider the $\eqso(\logu{1})$ formula $\sa{x}\forall y(y = x)$. This formula defines the following function that takes an ordered structure $\A$ as input:
$$
\sem{\alpha}(\A) = 
\begin{cases}
1 &\A \text{ has one element}\\
0 &\text{ otherwise}.
\end{cases}
$$
Suppose that there exists an equivalent formula $\alpha$ in $\eqso(\loge{1})$. Also, suppose that it is in $\L$-PNF, so $\alpha = c + \sum_{i = 1}^n\sa{\bar{X}}\sa{\bar{x}}\exists\bar{y}\varphi_i(\bar{X},\bar{x},\bar{y})$. Since $\alpha$ takes the value 0 for some structures, $c$ must be 0. Consider a structure $\mathfrak{1}$ with one element. We have that for some $i$, there exists an assignment $(\bar{B},\bar{b},\bar{a})$ for $(\bar{X},\bar{x},\bar{y})$ such that $\mathfrak{1}\models\varphi_i(\bar{B},\bar{b},\bar{a})$. Consider now the structure $\mathfrak{2}$ that is obtained by duplicating $\mathfrak{1}$, as we did for Theorem \ref{theo-pi1-pnf}. Note that $\mathfrak{2}\models\varphi_i(\bar{B},\bar{b},\bar{a})$, which implies that $\sem{\alpha}(\mathfrak{2}) \geq 1$, which leads to a contradiction.
\qed
The relationship between the two hierarchies is summarized in Figure \ref{fig-sfo-eqso}.
Our hierarchy and the one proposed in~\cite{SalujaST95} only differ in~$\Sigma_0$ and~$\Sigma_1$. 
Interestingly, we show next that this difference is crucial for finding classes of functions with easy decision versions and good closure properties.
