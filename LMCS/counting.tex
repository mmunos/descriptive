%%!TEX root = main.tex

In this section, we show that by syntactically restricting $\qso$ one can capture different counting complexity classes. 
In other words, by using $\qso$ we can extend the theory of descriptive complexity~\cite{immerman1999descriptive} from decision problems to counting problems. 
For this, we first formalize the notion of \emph{capturing} a complexity class of functions.
%, and then show how to capture classes like $\shp$, $\fp$, and $\fpspace$.

Fix a signature $\R = \{R_1, \ldots, R_k\}$ and assume that $\A$ is an ordered $\R$-structure with a domain $A = \{a_1, \ldots, a_n\}$ and $a_1 < a_2 < \ldots < a_n$. For every $i \in \{1, \ldots, k\}$, define the encoding of $R_i^\A$, denoted by $\enc(R_i^\A)$, as the following binary string. Assume that $\ell = \arity(R_i)$ and consider an enumeration of the $\ell$-tuples over $A$ in the lexicographic order induced by $<$. 
%(that is, $(a_1, \ldots, a_1, a_1)$, $(a_1, \ldots, a_1, a_2)$, $\ldots$, $(a_n, \ldots, a_n, a_{n-1})$, $(a_n, \ldots, a_n, a_n)$). 
Then let $\enc(R_i^\A)$ be a binary string of length $n^\ell$ such that the $i$-th bit of $\enc(R_i^\A)$ is 1 if the $i$-th tuple in the previous enumeration belongs to $R_i^\A$, and 0 otherwise. Moreover, define the encoding of $\A$, denoted by $\enc(\A)$, as the string~\cite{L04}:
%following binary string~\cite{L04}:
%\begin{eqnarray*}
	%\enc(\A) & = & 0^n \, 1 \, \enc(R_1^\A) \, \cdots \, \enc(R_k^\A).
%\end{eqnarray*}
$$
0^n \, 1 \, \enc(R_1^\A) \, \cdots \, \enc(R_k^\A).
$$
%We define the class of all $\R$-functions, denoted by $\Func(\R)$, as the class of all functions $f: \ostr \rightarrow \bbN$.
%Given a function complexity class $\CC$ (i.e. $f: \Sigma^* \rightarrow \bbN$ for every $f \in \CC$), we say that a function $f \in \Func(\R)$ can be computed in $\CC$ if there exists $g \in \CC$ such that $f(\A) = g(\enc(\A))$ for every $\A \in \ostr$. 
%Note that the function $g$ outputs $f$ for encodings of structures and can behave arbitrarily otherwise.
We can now formalize the notion of capturing a counting complexity class.
\begin{defi} \label{def:cap}
	Let $\FF$ be a fragment of $\qso$ and $\CC$ a counting complexity class. Then {\em  $\FF$ captures $\CC$ over ordered $\R$-structures} if the  following conditions hold:
	\begin{enumerate}
		\item for every $\alpha \in \FF$, there exists $f \in \CC$ such that $\sem{\alpha}(\A) = f(\enc(\A))$ for every $\A \in \ostr[\R]$. 
		
		\item for every $f \in \CC$, there exists $\alpha \in \FF$ such that   $f(\enc(\A)) = \sem{\alpha}(\A)$ for every $\A \in \ostr[\R]$.
	\end{enumerate} 
	Moreover, {\em $\FF$ captures $\CC$ over ordered structures} if $\FF$ captures~$\CC$ over ordered $\R$-structures for every signature~$\R$. \qed
\end{defi}
%For the sake of simplification, we denote the first condition by $\FF \subseteq \CC$ and the second condition by $\CC \subseteq \FF$.
In Definition~\ref{def:cap}, function $f \in \CC$ and formula $\alpha \in \FF$ must coincide in all the strings that encode ordered $\R$-structures. Notice that this restriction is natural as we want to capture %Since we want to capture 
$\CC$ over a fixed set of structures (e.g. graphs, matrices).
%, it is natural to just consider strings that encodes $\R$-structures. 
Moreover, this restriction is fairly standard in descriptive complexity \cite{immerman1999descriptive,L04}, and it has also been used in the previous work on capturing complexity classes of functions \cite{SalujaST95,ComptonG96}.
%all notions for capturing complexity classes restrict $f \in \CC$ similarly. 

What counting complexity classes can be captured with fragments of $\qso$?
For answering this question, it is reasonable to start with $\shp$, a well-known and widely-studied counting complexity class~\cite{arora2009computational}. 
Since $\shp$ has a strong similarity with $\np$, one could expect a ``Fagin-like'' Theorem~\cite{F75} for this class. 
Actually, in~\cite{SalujaST95} it was shown that the class $\sfo$ captures $\shp$.
In our setting, the class $\sfo$ is contained in $\eqso(\fo)$, which also captures $\shp$ as expected.
 
\begin{prop} \label{prop:capture-shP}
	$\eqso(\fo)$ captures $\shp$ over ordered structures.
\end{prop}
\proof
Saluja et al.~\cite{SalujaST95} proved that $\shp = \sfo$. Hence, given that every function in $\sfo$ can also be defined $\eqso(\fo)$ (see Section~\ref{sec:previous}), we have that every function in $\shp$ can be defined in $\eqso(\fo)$.

For the other direction, let $\alpha\in\eqso(\fo)$ over some signature $\R$, defined by the grammar in \ref{syntax}. Notice that given a $\fo$ formula $\varphi$, checking whether $\A\models\varphi$ can be done in deterministic polynomial time on the size of $\A$. Also, the constant function $s$ for some $s\in\nat$ can be trivially simulated in $\shp$. These facts, together with the closure under exponential sum and polynomial product of $\shp$\cite{F97}, suffice to show that the function represented $\alpha$ is also in $\shp$.

%We construct recursively a $\shp$-machine $M_{\alpha}$ for each $\eqso(\fo)$ formula $\alpha$ over a signature $\R$. This machine, on input $(\A,v,V)$ accepts in $\sem{\alpha}(\A,v,V)$ of its non-deterministic paths for each $(\A,v,V) \in \ostr[\R]^*$. Suppose $\A$ has domain $A$. If $\alpha$ is a $\fo$-formula $\varphi$, then the machine checks if $(\A,v,V)\models\varphi$ deterministically in polynomial time, and accepts if and only if it holds true. If $\alpha$ is a constant $s$, it produces $s$ branches and accepts in all of them. If $\alpha = (\beta \add \gamma)$, then it chooses between 0 or 1, if it is 0 (1), it simulates $M_{\beta}$ ($M_{\gamma}$) on input $(\A,v,V)$. 
%If $\alpha = \sa{x}\beta$, it chooses $a\in A$ non-deterministically and simulates $M_{\beta}$ on input $(\A,v[a/x],V)$.
%If $\alpha = \sa{X}\beta$, it chooses $B\in A^{arity(X)}$ and simulates $M_{\beta}$ on input $(\A,v,V[B/X])$. This covers all possible cases for $\alpha$. Let $\alpha$ be a formula in $\eqso(\fo)$ over a signature $\R$ and let $f$ be a function over $\R$ such that $f(\enc(\A))$ is equal to the accepting paths of $M_{\alpha}$ on input $(\A,v,V)$ for some $(\A,v,V) \in \ostr[\R]^*$. We have that $f$ is a $\shp$-function over $\R$ and $f(\enc(\A)) = \sem{\alpha}(\A)$ for every $\A\in\ostr[\R]$. 
\qed

By following the same approach as~\cite{SalujaST95}, Compton and Gr\"adel~\cite{ComptonG96} show that $\seso$ captures $\spp$, where $\eso$ is the existential fragment of $\so$. As one could expect, if we parametrize $\eqso$ with $\eso$, we can also capture~$\spp$.
\begin{prop} \label{prop:capture-spanP}
	$\eqso(\eso)$ captures $\spp$ over ordered structures.
\end{prop}
\proof
To prove the condition (2), we use the fact that $\spp = \#(\eso)$. The condition holds using the same argument as in Proposition~\ref{prop:capture-shP}.
For condition (1), notice that given an $\eso$ formula $\varphi$, checking whether $\A\models\varphi$ can be done in non-deterministic polynomial time on the size of $\A$\cite{fagin1974generalized}. 
Therefore, a $\spp$ machine for $\varphi$ will simulate the non-deterministic polynomial time machine and produce the same string as output in each accepting non-deterministic run. Furthermore, the constant function $s$ for some $s\in\nat$ can be trivially simulated in $\spp$ and, thus, condition (1) holds analogously to Proposition \ref{prop:capture-shP} since $\spp$ is also closed under exponential sum and polynomial product~\cite{OH93}.
%Similar than the previous proof, we construct recursively a $\spp$ machine $M_{\alpha}$ for each $\eqso(\eso)$ formula $\alpha$ over a signature $\R$. This machine, on input $(\A,v,V)$, non-deterministically produces $\sem{\alpha}(\A,v,V)$ distinct accepting outputs for each $\A \in \ostr[\R]$. Suppose $\A$ has domain $A$. 
%If $\alpha$ is a $\eso$-formula $\varphi$ it checks if $(\A,v,V)\models\varphi$ non-deterministically in polynomial time \cite{F75}, and accepts if and only if the condition holds true. 
%If $\alpha$ is a constant $s$, then the machine produces $s$ branches and accepts in all of them. 
%If $\alpha = (\beta \add \gamma)$, then it chooses between 0 or 1, if it is 0 (1), it simulates $M_{\beta}$ ($M_{\gamma}$) on input $(\A,v,V)$.  
%If $\alpha = \sa{x}\beta$, it chooses $a\in A$ non-deterministically and simulates $M_{\beta}$ on input $(\A,v[a/x],V)$. 
%If $\alpha = \sa{X} \beta$, it chooses $B\in A^{\arity(X)}$ and simulates $M_{\beta}$ on input $(\A,v,V[B/X])$. 
%This covers all possible cases for $\alpha$.
%Additionally, the machine produces a different output on each path. This can be done by printing the trace of all the non-deterministic choices.
%However, when the machine starts checking whether $(\A,v,V)\models\varphi$ for some $\eso$ formula $\varphi$, it stops printing in the output tape. This way the machine produces exactly one output from that point onwards.
%Let $\alpha$ be a formula in $\eqso(\eso)$ over a signature $\R$ and let $f$ be a function over $\R$ such that $f(\enc(\A))$ is equal to the number of accepting outputs of $M_{\alpha}$ on input $(\A,v,V)$ for some $\A \in \ostr[\R]$. 
%We have that $f$ is a $\spp$ function over $\R$ and that $f(\enc(\A)) = \sem{\alpha}(\A)$ for every $\A\in\ostr[\R]$.
\qed
Can we capture $\fp$ by using $\# \LL$ for some fragment $\LL$ of $\so$? A first attempt could be based on the use of a fragment $\LL$ of $\so$ that capture either $\ptime$ or $\nlog$~\cite{G92}. Such an approach fails as $\# \LL$ can encode $\shp$-complete problems in both cases; in the first case, one can encode the problem of counting the number of satisfying assignments of a Horn  propositional formula, while in the second case one can encode the problem of counting the number of satisfying assignments of a 2-CNF propositional formula. A second attempt could be based then on considering a fragment $\LL$ of $\fo$. 
But even if we consider the existential fragment $\Sigma_1$ of $\fo$ the approach fails, as $\# \Sigma_1$ can encode $\shp$-complete problems like counting the number of satisfying assignments of a 3-DNF propositional formula\cite{SalujaST95}. One last attempt could be based on disallowing the use of second-order free variables in $\sfo$. But in this case one 
cannot capture exponential functions definable in $\fp$ such as~$2^n$.
Thus, it is not clear how to capture $\fp$ 
by following the approach proposed in~\cite{SalujaST95}. 
On the other hand, if we consider our framework and move out from $\eqso$, we have other alternatives for counting like first- and second-order products. In fact, the combination of $\qfo$ with $\lfp$ is exactly what we need to capture $\fp$.
\begin{thm} \label{theo:capture-fp}
	$\qfo(\lfp)$ captures $\fp$ over ordered structures.
\end{thm}
\proof
For the first condition, let $\alpha\in\qfo(\lfp)$ over some signature $\R$. Let $f$ be a function over $\R$ defined by the following procedure. Let $\enc(\A)$ be an input, where $\A$ is an ordered structure over $\R$ with domain $A = \{1,\ldots,n\}$. In the procedure we slightly extend the grammar of $\qfo(\lfp)$ to include constants. We replace each first order sum and first order product in $\alpha$ by an expansion using the elements in $A$. This is, $\sa{x} \beta(x)$ is replaced by $(\beta(1)+\cdots+\beta(n))$ and $\pa{x}\beta(x)$ is replaced by $(\beta(1)\cdot\,\cdots\,\cdot\beta(n))$. Then each sub-formula $\varphi\in\lfp$ in $\alpha$ is evaluated in polynomial time and replaced by 1 if $\A\models\varphi$ and by 0 otherwise. The resulting formula is an arithmetic expression of polynomial size (recall that $\alpha$ is fixed) which is evaluated and lastly given as output. Note that $f\in\fp$ and $f(\enc(\A)) = \sem{\alpha}(\A)$.
	
For the second condition, let $f\in \fp$ defined over some signature $\R$.
Let $\ell\in\nat$ be such that for each $\A\in\ostr[\R]$, $\lceil\log_2 f(\enc(\A)) \rceil \leq n^\ell$ (i.e. $n^\ell$ is an upper bound for the output size), where $\A$ has a domain of size $n$.
Let $\bar{x} = (x_1,\ldots,x_{\ell})$.
Consider a procedure that receives $\enc(\A)$ and an assignment $\bar{a}$ to $\bar{x}$. Let $m$ be the position of $\bar{a}$ in the lexicographic order of the tuples in $A^{\ell}$. The procedure then computes the $m$-th bit of $f(\enc(\A))$, from least to most significant. Since this procedure works in polynomial time, it can be described by an $\lfp$ formula $\Phi(\bar{x})$. Then we use
$$
\alpha = \sa{\bar{x}} \Phi(\bar{x})\cdot\varphi(\bar{x}),
$$
where $\varphi(\bar{x}) := \pa{\bar{y}}(\bar{y} < \bar{x} \mapsto 2).$ Note that if $\bar{a} \in A^{\ell}$ is the $m$-th tuple in the given order (starting from 0), then $\sem{\varphi(\bar{a})}(\A) = 2^{m}$. Adding these values for each $\bar{a}\in A^{\ell}$ gives exactly $f(\enc(\A))$. 
In other words, $\Phi(\bar{x})$ simulates the behavior of the $\fp$-machine and the formula $\alpha$ reconstruct the binary output.
Then, $\alpha$ is in $\qfo(\lfp)$ over $\R$ and $\sem{\alpha}(\A) = f(\enc(\A))$.
\qed
%To prove this theorem, 
%capture $\fp$ 
%one first shows that every formula in $\qfo(\lfp)$ can be evaluated in polynomial time. 
%Indeed, $\lfp$ is a polynomial-time logic~\cite{I86,vardi1982complexity}, and the sum and product quantifiers can also be computed in polynomial time. 
%For the other direction, one has to use $\qfo(\lfp)$ to simulate the run of a polynomial time TM $M$ computing a function, in particular using the quantitative quantifiers to reconstruct the natural number returned by $M$ in the output tape. 
%It is important to notice that the alternation between sum and product quantifiers is crucial for this reconstructions and, thus, crucial for capturing $\fp$.

At this point it is natural to ask whether one can extend the previous idea to capture $\fpspace$~\cite{Ladner89}, the class of functions computable in polynomial space. 
Of course, for capturing this class one needs a logical core powerful enough, like $\pfp$, for simulating the run of a polynomial-space TM.
Moreover, 
one also needs more powerful quantitative quantifiers as functions like $2^{2^n}$ can be computed in polynomial space,
so $\eqso$ is not enough for the quantitative layer of a logic for $\fpspace$.
In fact, by considering second-order product we obtain the fragment $\qso(\pfp)$ that captures $\fpspace$. 
\begin{thm} \label{theo:capture-fpspace}
	$\qso(\pfp)$ captures $\fpspace$ over ordered structures.
\end{thm}
\proof
For the first condition of Definition~\ref{def:cap}, notice that each $\pfp$ formula can be evaluated in deterministic polynomial space, the constant function $s$ can be trivially simulated in $\fpspace$, and $\fpspace$ is closed under exponential sum and multiplication. This suffices to show that the condition holds.
For the second condition, the proof is similar than in Theorem~\ref{theo:capture-fp}. Let $f\in \fpspace$ defined over some $\R$ and $\ell\in\nat$ such that $\log_2\left( f(\enc(\A)) \right) \leq 2^{{|\A|}^\ell}$ for every $\A\in\ostr[\R]$  (i.e. $2^{{|\A|}^\ell}$ is an upper bound for the output size). Let $X$ be a second-order variable of arity $\ell$. Consider the linear order induced by $<$ over predicates of arity $\ell$ which can be defined by the following formula:
$$
\varphi_{<}(X,Y) = \ex{\bar{u}}\big[\neg X(\bar{u})\wedge Y(\bar{u})\wedge \fa{\bar{v}}\big(
\bar{u}<\bar{v}\to(X(\bar{u})\iff Y(\bar{v}))\big)\big].
$$
Namely, we use relations to encode numbers with at most $2^{{|\A|}^\ell}$ bits where the empty relation represents $0$ and the total-relation represents $2^{2^{{|\A|}^\ell}}-1$.
Furthermore, we can use a relation~$X$ to index a position in the binary output of $f(\enc(\A))$ as follows.
%Consider a polynomial space machine over the $\R$ that receives as input an $\R$-structure $\A$ and a number $p$ encoded by a relation $X$. Then the machine accepts if, and only if, the $p$-th bit of $f(\enc(\A))$ is $1$. 
Define the language:
\[
L = \{(\A,B)\mid B \subseteq A^{\ell}\text{ and the $B$-th bit of $f(\enc(\A))$ is 1}\}.
\]
Since $L$ is in $\pspace$, it can be specified in $\pfp$ \cite{AbiteboulV89} by a formula $\Phi(X)$ where the free variable $X$ encodes relation $B$ in $L$. Then, similar than the previous proof we define:
$$
\alpha := \sa{X} \Phi(X)\mult  \pa{Y}(\varphi_{<}(Y,X)\mapsto 2).
$$ 
where $\pa{Y}(\varphi_{<}(Y,X)\mapsto 2)$ takes the value $2^m$ if there exist $m$ predicates that are smaller than~$X$ and $\alpha$ reconstruct the output of $f(\enc(\A))$ by simulating $f$ with $\Phi(X)$. Using an analogous argument, we conclude that $\alpha\in\qso(\pfp)$ and $\sem{\alpha}(\A) = f(\enc(\A))$.

\qed
%The proof of the previous theorem follows the same line as for the logical characterization of $\fp$: one shows that each function in $\qso(\pfp)$ can be computed in $\fpspace$ and, conversely, the output of a polynomial-space TM can be reconstructed by using $\pfp$ and quantitative quantifiers.

From the proof of the previous theorem a natural question follows: what happens if we use first-order quantitative quantifiers and $\pfp$?
In~\cite{Ladner89}, Ladner also introduced the class $\nfpspace$ of all functions computed by polynomial-space TMs 
with output length bounded by a polynomial.
Interestingly, if we restrict to FO-quantitative quantifiers we can also capture this class.
\begin{cor} \label{cor:capture-fpspace-poly}
	$\qfo(\pfp)$ captures $\nfpspace$ over ordered structures.
\end{cor}
\proof
For the first condition, let $\alpha\in\qfo(\pfp)$ over some signature $\R$. Let $f$ be a function over $\R$ defined by the following procedure. Let $\enc(\A)$ be an input, where $\A$ is an ordered structure over $\R$ with domain $A = \{1,\ldots,n\}$. In the procedure we slightly extend the grammar of $\qfo(\pfp)$ to include constants. We replace each first order sum and first order product in $\alpha$ by an expansion using the elements in $A$. This is, $\sa{x} \beta(x)$ is replaced by $(\beta(1)+\cdots+\beta(n))$ and $\pa{x}\beta(x)$ is replaced by $(\beta(1)\cdot\,\cdots\,\cdot\beta(n))$. Then each sub-formula $\varphi\in\pfp$ in $\alpha$ is evaluated in polynomial space and replaced by 1 if $\A\models\varphi$ and by 0 otherwise. The resulting formula is an arithmetic expression of polynomial size  (recall that $\alpha$ is fixed) which is evaluated and lastly given as output. Note that $f\in\nfpspace$ and $f(\enc(\A)) = \sem{\alpha}(\A)$.

For the second condition, let $f\in \fpspace$ defined over some signature $\R$.
Let $\ell\in\nat$ be such that for each $\A\in\ostr[\R]$, $\lceil\log_2 f(\enc(\A)) \rceil \leq n^\ell$, where $\A$ has a domain of size $n$.
Let $\bar{x} = (x_1,\ldots,x_{\ell})$.
Consider a procedure that receives $\enc(\A)$ and an assignment $\bar{a}$ to $\bar{x}$. Let $m$ be the position of $\bar{a}$ in the lexicographic order of the tuples in $A^{\ell}$. The procedure then computes the $m$-th bit of $f(\enc(\A))$, from least to most significant. Since this procedure works in polynomial space, it can be described by an $\pfp$ formula $\Phi(\bar{x})$. Then we use
$$
\alpha = \sa{\bar{x}} \Phi(\bar{x})\cdot\varphi(\bar{x}),
$$
where $\varphi(\bar{x}) := \pa{\bar{y}}(\bar{y} < \bar{x} \mapsto 2).$ Note that if $\bar{a} \in A^{\ell}$ is the $m$-th tuple in the given order (starting from 0), then $\sem{\varphi(\bar{a})}(\A) = 2^{m}$. Adding these values for each $\bar{a}\in A^{\ell}$ gives exactly $f(\enc(\A))$. Then, $\alpha$ is in $\qfo(\pfp)$ over $\R$ and $\sem{\alpha}(\A) = f(\enc(\A))$.
\qed

The results of this section validate $\qso$ as an appropriate logical framework for extending the theory of descriptive complexity to counting complexity classes. In the following sections, we provide more arguments for this claim, by considering some fragments of $\eqso$ and, moreover, by showing how to go beyond $\eqso$ to capture other classes.