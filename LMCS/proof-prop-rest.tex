We give this proof in three parts.
First, we show that $\E{1} \not\subseteq \QE{0}$. By contradiction, let $\R$ be the empty signature and suppose that there is a $\QE{0}$ formula $\alpha$ over $\R$ such that defines the following function: for every finite $\R$-structure with $n$ elements, $\alpha(\enc(\A)) = n - 1$. We need the following claim.
\begin{clm}
	Let $\alpha = \sa{\bar{x}}\varphi(\bar{x})$	where $\varphi$ is quantifier free. Then the function defined by $\alpha$ is either null, greater or equal to $n$, or is in $\Omega(n^2)$.
\end{clm}
\proof
	Suppose that the function defined by $\alpha$ is not $0$ and that $\varphi$ is in DNF, namely, $\alpha = \sa{\bar{x}} \varphi_1(\bar{x}) \vee \cdots \vee \varphi_n(\bar{x})$. Furthermore, suppose $\bar{x} = (x_1,\ldots,x_{\length{\bar{x}}})$. Since $\alpha$ is not null, then some $\varphi_i$ must be satisfiable and suppose that $\varphi_i(\bar{x}) = \bigwedge\psi(\bar{x})$ where each $\psi(\bar{x})$ is a literal (i.e. an atomic formula or its negation).
	We will prove by induction on $\length{\bar{x}}$ that the function defined by $\sa{\bar{x}} \bigwedge\psi(\bar{x})$ is either greater or equal to $n$, or in $\Omega(n^2)$.
	For $\length{\bar{x}}= 1$ and $\alpha = \sa{x}\bigwedge\psi(x)$, if any $\psi(x) = (x = x)$ or $\neg(x < x)$, then we can eliminate it and we obtain the same function. If any $\psi = (x < x)$ or $\neg(x=x)$, then the function becomes null. The only possible $\alpha$ left is $\alpha = \sa{x}\top$ which is equal to the function $n$. This covers all possible cases for $\length{\bar{x}} = 1$. 
	Now suppose that it holds for $\length{\bar{x}} = k$ and suppose $\alpha = \sa{\bar{x}}\bigwedge\psi(\bar{x})$ for $\length{\bar{x}} = k+1$. If any $\psi(\bar{x}) = (x_i = x_j)$ where $i \neq j$, then $\alpha$ describes the same function as $\alpha$ where $x_j$ has been replaced by $x_i$. In this formula the tuple of first-order variables has now $k$ elements so, by induction, the function satisfies our conclusions. Furthermore, if any $\psi(\bar{x}) = (x_i = x_j)$ or $\neg (x_i < x_j)$ where $i = j$, then we can eliminate $\psi(\bar{x})$ and obtain the same function, and if any $\psi(\bar{x}) = \neg(x_i = x_j)$ or $(x_i < x_j)$ where $i = j$, then the function becomes null. 
	The remaining formulas in $\bigwedge\psi(\bar{x})$ are either $\neg(x_i = x_j)$, $(x_i<x_j)$ or $\neg(x_i<x_j)$ with $i \neq j$. If the formula violates transitivity in $<$ (e.g. $x < y \wedge y < z \wedge z < x$), then $\alpha$ defines the null function. Therefore, there must be some order over $\bar{x}$ that satisfies $\bigwedge\psi(\bar{x})$ and this order can be chosen in  $\binom{n}{\length{\bar{x}}}$ different ways over the domain of $\A$. This implies that $\sem{\alpha} \in \Omega(n^{\length{\bar{x}}}) \subseteq \Omega(n^2)$ if $\length{\bar{x}} > 1$ and concludes the proof of the claim.
\qed
Coming back on proving that $\alpha$ does not define the $(n-1)$-function, suppose that $\alpha$ is in SNF, namely, $\alpha = \sum_{i = 1}^k \alpha_i$ for some fix $k$. Since $\alpha$ is not null, consider some $\alpha_i$ that describes a non-null function. Let $\alpha_i = \sa{\bar{X}}\sa{\bar{x}}\varphi(\bar{X},\bar{x})$ where $\varphi$ is quantifier-free. Note that if $\length{\bar{X}} > 0$, then the function $\sem{\alpha}$ is in $\Omega(2^n)$, as it was proven in~\cite{SalujaST95}. Therefore, we have that $\alpha_i = \sa{\bar{x}}\varphi(\bar{x})$ and, as we proved in the above claim, it describes either some function greater or equal to $n$, or in $\Omega(n^2)$, which leads to a contradiction. 
To conclude, note that the formula $\sa{x} \ex{y} (x < y)$ is in $\E{1}$ and describes the function $n-1$.

Now we show that $\QE{0} \not\subseteq \E{1}$. In Theorem \ref{theo-pi1-pnf} we proved that there is no formula in $\loge{1}$-PNF equivalent to the formula $\alpha = 2$. Every formula in $\E{1}$ can be expressed in $\loge{1}$-PNF, which implies that $2 \in \QE{0}$ and $2 \not\in \E{1}$.

Finally, we prove that $\eqso(\loge{1})\subsetneq\eqso(\logu{1})$. For inclusion, let $\alpha$ be a formula in $\eqso(\loge{1})$. Suppose that it is in $\loge{1}$-SNF, namely, $\alpha = c + \sum_{i = 1}^{n}\alpha_i$. Let $\alpha_i = \sa{\bar{X}}\sa{\bar{x}}\ex{\bar{y}}\varphi_i(\bar{X},\bar{x},\bar{y})$, where $\varphi_i$ is quantifier-free for each $\alpha_i$. We use the same construction used in \cite{SalujaST95}, and we obtain that the formula $\ex{\bar{y}}\varphi_i(\bar{X},\bar{x},\bar{y})$ is equivalent to $\sa{\bar{y}}\,\varphi_i(\bar{X},\bar{x},\bar{y}) \wedge \fa{\bar{y}'}(\varphi_i(\bar{X},\bar{x},\bar{y}')\to\bar{y}\leq\bar{y}')$ for every assignment to $(\bar{X},\bar{x})$. We do this replacement for each $\alpha_i$ and we obtain an equivalent formula in $\eqso(\logu{1})$.

To prove that the inclusion is proper, consider the $\eqso(\logu{1})$ formula $\sa{x} \fa{y}(y = x)$. This formula defines the following function over each ordered structure $\A$:
$$
\sem{\alpha}(\A) = 
\begin{cases}
1 &\A \text{ has one element}\\
0 &\text{ otherwise}.
\end{cases}
$$
Suppose that there exists an equivalent formula $\alpha$ in $\eqso(\loge{1})$. Also, suppose that it is in $\L$-PNF, so $\alpha = c + \sum_{i = 1}^n\sa{\bar{X}}\sa{\bar{x}}\ex{\bar{y}}\varphi_i(\bar{X},\bar{x},\bar{y})$. Since $\alpha$ takes the value 0 for some structures, $c$ must be 0. Consider a structure $\A'$ with one element. We have that for some $i$, there exists an assignment $(\bar{B},\bar{b},\bar{a})$ for $(\bar{X},\bar{x},\bar{y})$ such that $\A' \models\varphi_i(\bar{B},\bar{b},\bar{a})$. Consider now the structure $\A''$ that is obtained by duplicating $\A'$, as we did for Theorem \ref{theo-pi1-pnf}. Note that $\A''\models\varphi_i(\bar{B},\bar{b},\bar{a})$, which implies that $\sem{\alpha}(\A' \uplus \A'') > 1$, which leads to a contradiction.