We give this proof in three parts.

\vspace{1em}
First, we show that $\QE{0} \not\subseteq \E{1}$. By contradiction, suppose that there is a $\QE{0}$ formula $\alpha$ over some signature $\R$ such that defines the following function. For every finite $\R$-structure with $n$ elements, and where every predicate in $\R$ is empty, $\alpha(\enc(\A)) = n - 1$. We use the following claim.
\begin{claim}
	Let $\alpha = \sa{\bar{x}}\varphi(\bar{x})$	where $\varphi$ is quantifier free. Then the function defined by $\alpha$ is either null, greater or equal to $n$, or is in $\Omega(n^2)$.
\end{claim}
\begin{proof}
	Suppose that the function defined by $\alpha$ is not $0$ and that $\varphi$ is in DNF. Furthermore, suppose $\bar{x} = (x_1,\ldots,x_{\length{\bar{x}}})$.
	Then $\alpha = \sa{\bar{x}} \varphi_1(\bar{x}) \vee \cdots \vee \varphi_n(\bar{x})$. Since $\alpha$ is not null, then some $\varphi_i$ must be satisfiable. This is, the function defined by $\sa{\bar{x}}\varphi(\bar{x})$ is not null.
	We will prove by induction on $\length{\bar{x}}$ that the function defined by $\sa{\bar{x}}\varphi(\bar{x})$ is either greater or equal to $n$, or in $\Omega(n^2)$.
	We address the case $\length{\bar{x}}= 1$, then $\alpha = \sa{x}\bigwedge\psi(x)$. If any $\psi(x) = (x = x)$ or $\neg(x < x)$, then we can eliminate it and we obtain the same function. If any $\psi = (x < x)$ or $\neg(x=x)$, then the function becomes null. If $\psi(x) = R(x,\ldots,x)$ for some $R\in\R$ the function becomes null for the structures we are considering. If $\psi(x) = \neg R(x,\ldots,x)$, we can eliminate it and for the structures we are considering we obtain the same function. The only possible $\alpha$ left is $\alpha = \sa{x}\top$ which is equal to the function $n$. This covers all possible cases for $\length{\bar{x}} = 1$. Now suppose that it holds for $\length{\bar{x}} = k$ and suppose $\alpha = \sa{\bar{x}}\bigwedge\psi(\bar{x})$ for $\length{\bar{x}} = k+1$. If any $\psi(\bar{x}) = (x_i = x_j)$ where $i \neq j$, then $\alpha$ describes the same function as $\alpha$ where $x_j$ has been replaced by $x_i$. In this formula the tuple of first-order variables has $k$ elements so the function it describes if one of the mentioned in the hypothesis. If $i = j$, then we can eliminate it and obtain the same function. If any $\psi(\bar{x}) = R(\bar{v})$ or $\neg R(\bar{v})$ where $\bar{v}$ is a sub-tuple of $\bar{x}$ then we can either eliminate it or the function becomes null, following the same argument as in the case $\length{\bar{x}} = 1$. If any $\psi(\bar{x}) = \neg(x_i = x_j)$ or $(x_i < x_j)$ where $i = j$, then the function becomes null. If any $\psi(\bar{x}) = \neg(x_i < x_j)$ where $i = j$, we can eliminate it. The remaining formulas in $\bigwedge\psi(\bar{x})$ are either $\neg(x_i = x_j)$, $(x_i<x_j)$ or $\neg(x_i<x_j)$. If the formula violates transitivity in $<$ (for example, $x < y \wedge y < z \wedge z < x$), then the function $\alpha$ describes is null. Therefore, there is some order over $\bar{x}$ that satisfies $\bigwedge\psi(\bar{x})$. Consider the formula that describes this order (like $x_1 < x_3 \wedge x_3 < x_4 \wedge x_4 < x_2$). The function $\alpha$ describes is greater or equal to the one this formula describes, which is exactly $\binom{n}{\length{\bar{x}}}$ which is in $\Omega(n^{\length{\bar{x}}}) \subseteq \Omega(n^2)$ if $\length{\bar{x}} > 1$. This concludes the proof of the claim.
\end{proof}
We suppose that $\alpha$ is in SNF, this is, $\alpha = \sum_{i = 1}^n\alpha_i$. Since $\alpha$ is not null, consider some $\alpha_i$ that describes a non-null function. Let $\alpha_i = \sa{\bar{X}}\sa{\bar{x}}\varphi(\bar{X},\bar{x})$, where $\varphi$ is quantifier-free. Note that if $\length{\bar{X}} > 0$, then the function $\alpha$ describes is in $\Omega(2^n)$, as it was proven by the authors in \cite{SalujaST95}. We have that $\alpha_i = \sa{\bar{x}}\varphi(\bar{x})$, as we proved in the claim, describes either some function greater or equal to $n$, or in $\Omega(n^2)$, which leads to a contradiction. Lastly, note that the formula $\sa{x}\exists y(x < y)$ is in $\E{0}$ and describes the function $n-1$, which concludes the proof.

\vspace{1em}
Now we show that $\E{1}\not\subseteq\QE{0}$. In Theorem \ref{theo-pi1-pnf} we proved that there is no formula in $\loge{1}$-PNF equivalent to the formula $\alpha = 2$. Every formula in $\E{1}$ can be expressed in $\loge{1}$-PNF, which implies that $2 \in \QE{0}$ and $2 \not\in \E{1}$.

\vspace{1em}
Lastly, we prove that $\eqso(\loge{1})\subsetneq\eqso(\logu{1})$. For inclusion, let $\alpha$ be a formula in $\eqso(\loge{1})$. Suppose that it is in $\loge{1}$-SNF. This is, $\alpha = c + \sum_{i = 1}^{n}\alpha_i$. Let $\alpha_i = \sa{\bar{X}}\sa{\bar{x}}\exists\bar{y}\,\varphi_i(\bar{X},\bar{x},\bar{y})$, where $\varphi_i$ is quantifier-free, for each $\alpha_i$. We use the same construction used in \cite{SalujaST95}, and we obtain that the formula $\exists\bar{y}\,\varphi_i(\bar{X},\bar{x},\bar{y})$ is equivalent to $\sa{\bar{y}}\,\varphi_i(\bar{X},\bar{x},\bar{y}) \wedge \forall\bar{y}'(\varphi_i(\bar{X},\bar{x},\bar{y}')\to\bar{y}\leq\bar{y}')$ for every assignment to $(\bar{X},\bar{x})$. We do this replacement for each $\alpha_i$ and we obtain an equivalent formula in $\eqso(\logu{1})$.

To prove that the inclusion is proper, consider the $\eqso(\logu{1})$ formula $\sa{x}\forall y(y = x)$. This formula defines the following function that takes an ordered structure $\A$ as input:
$$
\sem{\alpha}(\A) = 
\begin{cases}
1 &\A \text{ has one element}\\
0 &\text{ otherwise}.
\end{cases}
$$
Suppose that there exists an equivalent formula $\alpha$ in $\eqso(\loge{1})$. Also, suppose that it is in $\L$-PNF, so $\alpha = c + \sum_{i = 1}^n\sa{\bar{X}}\sa{\bar{x}}\exists\bar{y}\varphi_i(\bar{X},\bar{x},\bar{y})$. Since $\alpha$ takes the value 0 for some structures, $c$ must be 0. Consider a structure $\mathfrak{1}$ with one element. We have that for some $i$, there exists an assignment $(\bar{B},\bar{b},\bar{a})$ for $(\bar{X},\bar{x},\bar{y})$ such that $\mathfrak{1}\models\varphi_i(\bar{B},\bar{b},\bar{a})$. Consider now the structure $\mathfrak{2}$ that is obtained by duplicating $\mathfrak{1}$, as we did for Theorem \ref{theo-pi1-pnf}. Note that $\mathfrak{2}\models\varphi_i(\bar{B},\bar{b},\bar{a})$, which implies that $\sem{\alpha}(\mathfrak{2}) \geq 1$, which leads to a contradiction.