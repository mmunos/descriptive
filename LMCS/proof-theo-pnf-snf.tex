Recall that a formula in $\eqso(\LL)$ is written on the following grammar:
\[
\alpha = \varphi \ \mid \ s \ \mid \ (\alpha + \alpha) \ \mid \ \sa{x} \alpha \ \mid \ \sa{X} \alpha,
\]
where $\varphi$ is a formula in $\LL$ and $s\in\nat$. We will construct a recursive function $\tau$ such that for every $\eqso(\LL)$ formula $\alpha$, it outputs an equivalent formula $\tau(\alpha)$ which is in $\LL$-SNF. If $\alpha = \varphi$, let $\tau(\alpha) = \alpha$. If $\alpha = s$, let $\tau(\alpha) = (\top \add \cdots \add \top)$ ($s$ times). We assume that for every sub-formula $\beta$ in $\alpha$, $\tau(\beta)$ is an equivalent formula in $\LL$-SNF. If $\alpha = (\alpha_1 + \alpha_2)$, let $\tau(\alpha) = (\tau(\alpha_1) + \tau(\alpha_2))$. If $\alpha = \sa{x}\beta$, then $\tau(\beta) = \sum_{i = 1}^{k}\beta_i$ for some $k$ where each $\beta_i$ is in $\LL$-PNF. We define $\tau(\alpha) = \sum_{i = 1}^{k}\sa{x}\beta_i$. If $\alpha = \sa{X}\beta$, then we proceed analogously as in the previous case. This covers all possible cases for $\alpha$ and we conclude the proof by taking $\tau(\alpha)$ as the desired rewrite of $\alpha$.