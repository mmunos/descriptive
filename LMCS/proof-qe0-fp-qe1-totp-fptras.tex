The authors in \cite{SalujaST95} proved that there exists a {\em product reduction} from every function in $\E{1}$ to a restricted version of $\cdnf$. This is, if $\alpha\in\E{1}$ over some signature $\R$, there exist polynomially computable functions $g:\ostr[\R]\to\ostr[\R_{\text{DNF}}]$ and $h:\nat\to\nat$ such that for every finite $\R$-structure $\A$ with domain $A$, it holds that $\sem{\alpha}(\A) = \cdnf(\enc(g(\A)))\cdot h(\vert A \vert)$. We use this fact in the next arguments. 

To show that $\eqso(\loge{1})$ is contained  in \textsc{TotP}, let $\alpha$ be a $\eqso(\loge{1})$ formula and assume that it is in $\loge{1}$-SNF. This is, $\alpha = \sum_{i = 1}^n\alpha_i$ where each $\alpha_i$ is in $\loge{1}$-PNF. Consider the following nondeterministic procedure that on input $\enc(\A)$ generates $\sem{\alpha}(\A)$ branches. For each $\alpha_i = \varphi$, where $\varphi$ is a $\loge{1}$ formula, it checks if $\A\models\varphi$ in polynomial time and generates a new branch if that is the case. For each $\alpha_i = \sa{\bar{X}}\sa{\bar{x}}\varphi$, this formula is also in $\E{1}$. We use the reduction to $\cdnf$ provided in \cite{SalujaST95} and we obtain $g(\enc(\A))$, which is an instance to $\cdnf$. Since $\cdnf$ is also in $\totp$ \cite{PagourtzisZ06}, we simulate the corresponding nondeterministic procedure that generates exactly $\cdnf(\enc(g(\A)))$ branches. Since, $\fp\subseteq\totp$\cite{PagourtzisZ06}, each polynomially computable function is also in $\totp$, and then on each of these branches we simulate the corresponding nondeterministic procedure to generate $h(\vert A \vert)$ more. The number of branches for each $\alpha_i$ is $\sem{\alpha_i}(\A) = \cdnf(\enc(g(\A)))\cdot h(\vert A \vert)$, and the total number of branches is equal to $\sem{\alpha}(\A)$. We conclude that $\alpha\in\totp$.

To show that every function in $\eqso(\loge{1})$ has an FPRAS,  let $\alpha$ be a $\eqso(\loge{1})$ formula and assume that it is in $\loge{1}$-SNF. This is, $\alpha = \sum_{i = 1}^n\alpha_i$ where each $\alpha_i$ is in $\loge{1}$-PNF. Note that each $\alpha_i$ that is equal to some $\loge{1}$ formula $\varphi$ has an FPRAS given by the procedure that simply checks if $\A\models\varphi$ and returns 1 if it does and 0 otherwise. Also, each remaining $\alpha_i$ has an FPRAS since $\alpha_i\in \E{1}$ \cite{SalujaST95}. If two functions have an FPRAS, then their sum also has one given by the procedure that simulates them both and sums the results. We conclude that $\alpha$ has an FPRAS.

Finally, we show that $\eqso(\loge{1})$ is closed under sum and multiplication. Since $\eqso(\loge{1})$ is closed under sum by definition, we focus only in proving that is closed under multiplication. We prove this for a more general case for $\eqso(\LL)$ where $\LL$ is a fragment of $\so$.

\begin{lem} \label{conj-mult}
	If $\LL$ is a fragment closed under conjunction, then $\eqso(\LL)$ is closed under binary multiplication.
\end{lem}
\proof
	We define a recursive function $\tau$ that receives a formula $\alpha$ over the grammar of $\eqso(\LL)$ extended by binary product, and outputs an equivalent formula $\tau(\alpha)$ over the unextended grammar of $\eqso(\LL)$. In fact, the formula $\tau(\alpha)$ is in $\LL$-SNF. First we replace each constant $s$ with $(\top \add \cdots \add \top)$ ($s$ times). If $\alpha = \varphi$, then we define $\tau(\alpha) = \alpha$. We assume that for every $\beta$ that has less algebraic operators than $\alpha$, $\tau(\beta)$ is in $\LL$-SNF. If $\alpha = (\alpha_1 + \alpha_2)$ then we define $\tau(\alpha) = \tau(\alpha_1) + \tau(\alpha_2)$. If $\alpha = \sa{x}\beta$ or $\alpha = \sa{X}\beta$, then we define $\tau(\alpha)$ as the formula in $\LL$-SNF that is equivalent to $\sa{x}\tau(\beta)$ and to $\tau(\alpha) = \sa{X}\tau(\beta)$, respectively. If $\alpha = (\alpha_1 \cdot \alpha_2)$, we assume that each $\alpha_i$ is in $\LL$-SNF. We identify three cases. (1) Some $\alpha_i$ is equal to $\sum_{j = 1}^n\beta_j$ for $n > 1$. Suppose wlog. that it is $\alpha_1$. We then define $\tau(\alpha) = \sum_{j = 1}^n\tau(\beta_j\cdot\alpha_2)$. In the following cases, $\alpha_1$ and $\alpha_2$ are in $\LL$-SNF. (2) If some $\alpha_i$ is equal to $\sa{X}\beta$ or $\sa{x}\beta$, we define $\tau(\alpha)$ as the $\LL$-SNF formula that is equivalent to $\sa{x}\tau(\beta\cdot\alpha_2)$ and $\sa{X}\tau(\beta\cdot\alpha_2)$, respectively. The remaining case is (3) $\alpha_1 = \varphi_1$ and $\alpha_2 = \varphi_2$ where each $\varphi$ is an $\LL$ formula. Then we define $\tau(\alpha) = \varphi_1 \wedge \varphi_2$. This covers all possible cases for $\alpha$. For every pair of formulas $\alpha,\beta$ in $\eqso(\LL)$, we have that their multiplication $(\alpha\cdot\beta)$ is a formula in the grammar $\eqso(\LL)$ extended by binary product, and so, there exists an equivalent formula $\tau(\alpha\cdot\beta)$ which is in $\eqso(\LL)$.
\qed
Since $\loge{1}$ is closed under conjunction, then Lemma~\ref{conj-mult} also holds for $\eqso(\loge{1})$. This concludes the proof.