%!TEX root = main.tex

Inspired by the connection between $\shp$ and $\sfo$, a hierarchy of subclases of $\sfo$ was introduced in~\cite{SalujaST95} 
by restricting the alternation of quantifiers in Boolean formulae.
Specifically, the \emph{$\sfo$-hierarchy} consists of the 
the classes $\E{i}$ and $\U{i}$ for every $i \geq 0$, where $\E{i}$ (resp., $\U{i}$) is defined as $\sfo$ but restricting the formulae used to be in $\loge{i}$ (resp., $\logu{i}$).
By definition, we have that $\U{0} = \E{0}$. Moreover, it is shown in~\cite{SalujaST95} that:
\[
\E{0} \; \subsetneq \; \E{1} \; \subsetneq \; \U{1} \; \subsetneq \; \E{2} \; \subsetneq \; \U{2} \; = \; \sfo 
\]
In light of the framework introduced in this paper, natural extensions of these classes are obtained by considering 
$\eqso(\loge{i})$ and $\eqso(\logu{i})$ for every $i \geq 0$, which form the \emph{$\eqso(\fo)$-hierarchy}.
Clearly, we have that $\E{i} \subseteq \QE{i}$ and $\U{i} \subseteq \QU{i}$. Indeed, each formula $\varphi(\bar{X}, \bar{x})$ in $\E{i}$ is equivalent to the formula $\sa{\bar X} \sa{\bar x} \varphi(\bar{X}, \bar{x})$ in $\QE{i}$, and likewise for $\U{i}$ and $\QU{i}$.
But what is the exact relationship between these two hierarchies?
To answer this question, we first introduce two normal forms for $\eqso(\LL)$ that helps us to characterize the expressive power of this quantitative logic.
A formula $\alpha$ in $\eqso(\LL)$ is in \emph{$\LL$-prenex normal form ($\LL$-PNF)} 
if $\alpha$ is of the form
$\sa{\bar{X}} \sa{\bar{x}} \varphi(\bar{X}, \bar{x})$,
where $\bar{X}$ and $\bar{x}$ are sequences of zero or more second-order and first-order variables, respectively, and $\varphi(\bar{X}, \bar{x})$ is a formula in $\LL$. Notice that 
a formula $\varphi(\bar{X}, \bar{x})$ in $\sh{\LL}$ is equivalent to the formula $\sa{\bar X} \sa{\bar x} \varphi(\bar{X}, \bar{x})$ in $\LL$-PNF. 
Moreover, a formula $\alpha$ in $\eqso(\LL)$ is in \emph{$\LL$-sum normal form ($\LL$-SNF)} if $\alpha$ is of the form $\Sigma_{i=1}^n \alpha_i$ where each $\alpha_i$ is in $\LL$-PNF. 
\begin{prop}\label{theo-pnf-snf}
Every formula in $\eqso(\LL)$ can be rewritten in $\LL$-SNF.
\end{prop}
\proof
Recall that a formula in $\eqso(\LL)$ is defined by the following grammar:
\[
\alpha = \varphi \ \mid \ s \ \mid \ (\alpha + \alpha) \ \mid \ \sa{x} \alpha \ \mid \ \sa{X} \alpha
\]
where $\varphi$ is a formula in $\LL$ and $s\in\nat$. 
To find an equivalent formula in $\LL$-SNF for every $\alpha \in \eqso(\LL)$, we give a recursive function $\tau$ such that $\tau(\alpha)$ is in $\LL$-SNF and $\tau(\alpha) \equiv \alpha$. 
Specifically, if $\alpha = \varphi$, define $\tau(\alpha) = \alpha$ and if $\alpha = s$, define $\tau(\alpha) = (\top \add \overset{\text{$s$-times}}{\ldots} \add \top)$.  If $\alpha = (\alpha_1 + \alpha_2)$, define $\tau(\alpha) = (\tau(\alpha_1) + \tau(\alpha_2))$ and if $\alpha = \sa{x}\beta$, define $\tau(\alpha) = \sum_{i = 1}^{k}\sa{x}\beta_i$ where $\tau(\beta) = \sum_{i = 1}^{k}\beta_i$ for some $k$ and each $\beta_i$ is in $\LL$-PNF. Finally, if $\alpha = \sa{X}\beta$, then we proceed analogously as in the previous case. This covers all possible cases for $\alpha$ and we conclude the proof by taking $\tau(\alpha)$ as the desired rewrite of $\alpha$.
\qed
If a formula is in $\LL$-PNF then clearly the formula is in $\LL$-SNF.
Unfortunately, for some $\LL$ there exist formulae in $\eqso(\LL)$  that cannot be rewritten in $\LL$-PNF.
Therefore, to unveil the relationship between the $\sfo$-hierarchy and the $\eqso(\fo)$-hierarchy, we need to understand the boundary between PNF and SNF. We do this in the following theorem. 
\begin{thm}\label{theo-pi1-pnf}
For $i = 0,1$, there exists a formula $\alpha_i$ in $\QE{i}$ that is not equivalent to any formula in $\Sigma_i$-PNF. 
On the other hand, if $\logu{1} \subseteq \LL$ and $\LL$ is closed under conjunction and disjunction, then every formula in $\eqso(\LL)$ can be rewritten in $\LL$-PNF. 
\end{thm}
\proof
% !TeX root = main.tex

From now on, for every first-order tuple $\bar{x}$ or second-order tuple $\bar{X}$ we write $\length{\bar{x}}$ or $\length{\bar{X}}$ as the number of variables in $\bar{x}$ or $\bar{X}$ respectively. 
We divide the proof in three parts.

First, we prove that the formula $\alpha_{0} = \left( \sa{X} 1 \right) + 1$ with $\arity(X) = 1$ (i.e. the function $2^{\vert\A\vert}+1$) is not equivalent to any formula in $\loge{0}$-PNF. Suppose that there exists some formula $\alpha = \sa{\bar{X}}\sa{\bar{x}}\varphi(\bar{X},\bar{x})$ in $\loge{0}$-PNF that is equivalent to $\alpha_0$.
In \cite{SalujaST95}, it was proved that if $\length{\bar{X}} > 0$, the function defined by $\alpha$ is always even for big enough structures, which is not possible in our case.
On the other hand, if $\alpha$ is of the form $\sa{\bar{x}}\varphi(\bar{x})$, then $\alpha$ defines a polynomially bounded function which leads to a contradiction.

Second, we prove that the formula $\alpha_{1} = 2$ (i.e. $\sem{\alpha_{1}}$ is the constant function $2$) is not equivalent to any formula in $\loge{1}$-PNF. 
Suppose that there exists some formula $\alpha = \sa{\bar{X}}\sa{\bar{x}}\exists\bar{y}\, \varphi(\bar{X},\bar{x},\bar{y})$ in $\loge{1}$-PNF that is equivalent to $\alpha_1$. 
First, if $\length{\bar{X}} = \length{\bar{x}} = 0$, then the function defined by $\alpha$ is never greater than 1. 
Therefore, suppose that $\length{\bar{X}} > 0$ or $\length{\bar{x}} > 0$, and consider some ordered structure $\A$. 
Since $\sem{\alpha}(\A) = 2$, there exist at least two assignments $(\bar{B}_1,\bar{b}_1,\bar{a}_1)$, $(\bar{B}_2,\bar{b}_2,\bar{a}_2)$ to $(\bar{X},\bar{x},\bar{y})$ such that for $i\in\{1,2\}$: $\A\models\varphi(\bar{B}_i,\bar{b}_i,\bar{a}_i)$. 
Now consider the ordered structure $\A'$ that is obtained by taking the disjoint union of $\A$ twice. 
Indeed, each half of $\A'$ is isomorphic to $\A$. 
Note that $\A'\models\varphi(\bar{B}_i,\bar{b}_i,\bar{a}_i)$ for $i = 1,2$ and there exists a third assignment $(\bar{B}_1',\bar{b}_1',\bar{a}_1')$ that is isomorphic to $(\bar{B}_1,\bar{b}_1,\bar{a}_1)$, in the other half of the structure, such that $\A'\models\varphi(\bar{B}_1',\bar{b}_1',\bar{a}_1')$. 
As a result, we have that $\sem{\alpha}(\A') \geq 3$ which leads to a contradiction.

For the last part of the proof, we show that if $\LL$ contains $\logu{1}$ and is closed under conjunction and disjunction, then for every formula $\alpha$ in $\eqso(\LL)$ there exists an equivalent formula in $\LL$-PNF. 
Similarly as in the proof of Theorem~\ref{theo-pnf-snf}, we show a recursive function $\tau$ that produces such a formula. 
Assume that $\alpha = \sum_{i = 1}^n \alpha_i$ is in $\LL$-SNF where each $\alpha_i$ is in $\LL$-PNF. 
Without loss of generality, we assume that each $\alpha_i = \sa{\bar{X}}\sa{\bar{x}}\varphi_i(\bar{X},\bar{x})$ with $\length{\bar{X}} > 0$ and $\length{\bar{x}} > 0$. 
If that is not the case, we can replace each $\alpha_i$ by the equivalent formula
$$
\sa{\bar{X}} \sa{Y}\sa{\bar{x}}\sa{y}(\varphi_i(\bar{X},\bar{x})\wedge \fa{z} Y(z) \wedge \fa{z} z \leq y).
$$
Now we begin describing the function $\tau$. 
If $\alpha = \sa{\bar{X}}\sa{\bar{x}}\varphi(\bar{X},\bar{x})$, then the formula is already in $\LL$-PNF so we define $\tau(\alpha) = \alpha$. 
If $\alpha = \alpha_1 + \alpha_2$, then we assume that $\tau(\alpha_1) = \sa{\bar{X}}\sa{\bar{x}}\varphi(\bar{X},\bar{x})$ and $\tau(\alpha_2) = \sa{\bar{Y}}\sa{\bar{y}}\psi(\bar{Y},\bar{y})$. 
Our construction works by identifying a ``first'' assignment for both $(\bar{X},\bar{x})$ and $(\bar{Y},\bar{y})$ and a ``last'' assignment for both $(\bar{X},\bar{x})$ and $(\bar{Y},\bar{y})$ using the following formulas:
\begin{align*}
\gamma_{\text{first}}(\bar{X},\bar{x}) & \; = \;  \bigwedge_{i = 1}^{\length{\bar{X}}} \fa{\bar{z}}\neg X_i(\bar{z}) \wedge \fa{\bar{z}}(\bar{x}\leq\bar{z}), \\
\gamma_{\text{last}}(\bar{X},\bar{x}) & \; = \;  \bigwedge_{i = 1}^{\length{\bar{X}}} \fa{\bar{z}} X_i(\bar{z}) \wedge \fa{\bar{z}}(\bar{z}\leq\bar{x}).
\end{align*}
Similarly, we can define the formulas $\gamma_{\text{first}}(\bar{Y},\bar{y})$ and $\gamma_{\text{last}}(\bar{Y},\bar{y})$.
In other words, the ``first'' assignment is the one where every second-order predicate is empty and the first-order assignment is the lexicographically smallest, and the ``last'' assignment is the one where every second-order predicate is full and the first-order assignment is the lexicographically greatest. 
We also need to identify the assignments that are not first and the ones that are not last. 
We do this by negating the two formulas above and grouping together the first-order variables:
\begin{align*}
\gamma_{\text{not-first}}(\bar{X},\bar{x}) & \; = \; \ex{\bar{z}}(\bar{z}_0 < \bar{x} \vee \bigvee_{i = 1}^{\length{\bar{X}}}X(\bar{z}_i)), \\
\gamma_{\text{not-last}}(\bar{X},\bar{x}) & \; = \; \ex{\bar{z}}(\bar{x} < \bar{z}_0 \vee \bigvee_{i = 1}^{\length{\bar{X}}}\neg X(\bar{z}_i)),
\end{align*}
where $\bar{z} = (\bar{z}_0,\bar{z}_1,\ldots,\bar{z}_{\length{\bar{X}}})$. Then the following formula is equivalent to $\alpha$:
\begin{align}
\sa{\bar{X}}\sa{\bar{x}}\sa{\bar{Y}}\sa{\bar{y}}[&(\varphi(\bar{X},\bar{x})\wedge\gamma_{\text{not-first}}(\bar{X},\bar{x})\wedge\gamma_{\text{first}}(\bar{Y},\bar{y}))\vee \label{eq:partition1} \\
&(\varphi(\bar{X},\bar{x})\wedge\gamma_{\text{first}}(\bar{X},\bar{x})\wedge\gamma_{\text{last}}(\bar{Y},\bar{y}))\vee \label{eq:partition2}\\
&(\psi(\bar{Y},\bar{y})\wedge\gamma_{\text{first}}(\bar{X},\bar{x})\wedge\gamma_{\text{not-last}}(\bar{Y},\bar{y}))\vee \label{eq:partition3}\\
&(\psi(\bar{Y},\bar{y})\wedge\gamma_{\text{last}}(\bar{X},\bar{x})\wedge\gamma_{\text{last}}(\bar{Y},\bar{y}))]. \label{eq:partition4}
\end{align}

To show that the formula is indeed equivalent to $\alpha$, note that the formulas in lines (\ref{eq:partition1}) and (\ref{eq:partition2}) form a partition over the assignments of $(\bar{X},\bar{x})$, while fixing an assignment for $(\bar{Y},\bar{y})$, and the formulas in lines (\ref{eq:partition3}) and (\ref{eq:partition4}) form a partition over the assignments of $(\bar{Y},\bar{y})$, while fixing an assignment for $(\bar{X},\bar{x})$. 
Altogether the four lines define pairwise disjoint assignments for $(\bar{X},\bar{x}),(\bar{Y},\bar{y})$. 
With this, it is straightforward to show that the above formula is equivalent to $\alpha$. 
However, the formula is not yet in the correct form since it has existential quantifiers in the sub-formulas $\gamma_{\text{not-first}}$ and $\gamma_{\text{not-last}}$. 
To solve this, we can replace each existential quantifier by a first order sum that counts just the first assignment that satisfies the inner formula and this can be defined in $\logu{1}$. 
A similar construction was used in \cite{SalujaST95}. 

Finally, consider a $\eqso(\LL)$ formula $\alpha$ in $\LL$-SNF. 
If $\alpha = \sum_{i = 1}^n\alpha_i$, then by induction we consider $\alpha = \alpha_1 + (\sum_{i = 2}^n\alpha_i)$ and use $\tau(\alpha_1 + \tau(\sum_{i = 2}^n\alpha_i))$ as the rewrite of $\alpha$, which satisfies the hypothesis.
%To show that the formula is indeed equivalent to $\alpha$, note that the formulas in lines (\ref{eq:partition1}) and (\ref{eq:partition2}) form a partition over the assignments of $(\bar{X},\bar{x})$, while fixing an assignment for $(\bar{Y},\bar{y})$, and the formulas in lines (\ref{eq:partition3}) and (\ref{eq:partition4}) form a partition over the assignments of $(\bar{Y},\bar{y})$, while fixing an assignment for $(\bar{X},\bar{x})$. Altogether, the four lines define pairwise disjoint assignments for $(\bar{X},\bar{x}),(\bar{Y},\bar{y})$. With this, it is straightforward to show that the above formula is equivalent to $\alpha$. However, the formula is not yet in the correct form since it has existential quantifiers in the sub-formulas $\gamma_{\text{not-first}}$ and $\gamma_{\text{not-last}}$. To solve this, first let us take a close look to the complete formula:
%\begin{align*}
%\sa{\bar{X}}\sa{\bar{x}}\sa{\bar{Y}}\sa{\bar{y}}[&(\varphi(\bar{X},\bar{x})\wedge\exists\bar{v}(\bar{v}_0 < \bar{x} \vee \bigvee_{i = 1}^{\length{\bar{X}}}X(\bar{v}_i))\wedge\bigwedge_{i = 1}^{\length{\bar{Y}}} \forall\bar{z}\neg Y_i(\bar{z}) \wedge \forall\bar{z}(\bar{y}\leq\bar{z}))\vee\\
%&(\varphi(\bar{X},\bar{x})\wedge\bigwedge_{i = 1}^{\length{\bar{X}}} \forall\bar{z}\neg X_i(\bar{z}) \wedge \forall\bar{z}(\bar{x}\leq\bar{z})\wedge\bigwedge_{i = 1}^{\length{\bar{Y}}} \forall\bar{z} Y_i(\bar{z}) \wedge \forall\bar{z}(\bar{z}\leq\bar{y}))\vee\\
%&(\psi(\bar{Y},\bar{y})\wedge\bigwedge_{i = 1}^{\length{\bar{X}}} \forall\bar{z}\neg X_i(\bar{z}) \wedge \forall\bar{z}(\bar{x}\leq\bar{z})\wedge\exists\bar{w}(\bar{y} < \bar{w}_0 \vee \bigvee_{i = 1}^{\length{\bar{Y}}}\neg Y(\bar{w}_i))\vee\\
%&(\psi(\bar{Y},\bar{y})\wedge\bigwedge_{i = 1}^{\length{\bar{X}}} \forall\bar{z} X_i(\bar{z}) \wedge \forall\bar{z}(\bar{z}\leq\bar{x})\wedge\bigwedge_{i = 1}^{\length{\bar{Y}}} \forall\bar{z} Y_i(\bar{z}) \wedge \forall\bar{z}(\bar{z}\leq\bar{y}))].
%\end{align*}
%To construct an equivalent formula that is in the correct form, we define $\bar{u} = (\bar{v},\bar{w})$ and we replace the first-order quantifiers by a first-sum and count the first assignment to $\bar{v}$ and $\bar{w}$ that satisfies the formula. A similar construction was used in \cite{SalujaST95}. Then the final formula equivalent to $\alpha$ is the following:
%\begin{align*}
%\sa{\bar{X}}&\sa{\bar{Y}}\sa{\bar{x}}\sa{\bar{y}}\sa{\bar{u}}\big[ \\
%&[\varphi(\bar{X},\bar{x})\wedge(\bar{v}_0 < \bar{x} \vee \bigvee_{i = 1}^{\length{\bar{X}}}X(\bar{v}_i))\wedge \forall\bar{u}'((\bar{v}_0' < \bar{x} \vee \bigvee_{i = 1}^{\length{\bar{X}}}X(\bar{v}_i'))\to\bar{u}\leq\bar{u}') \wedge \\
%&\bigwedge_{i = 1}^{\length{\bar{Y}}} \forall\bar{z}\neg Y_i(\bar{z}) \wedge \forall\bar{z}(\bar{y}\leq\bar{z})]\vee\\
%&
%[\varphi(\bar{X},\bar{x})\wedge\bigwedge_{i = 1}^{\length{\bar{X}}} \forall\bar{z}\neg X_i(\bar{z}) \wedge \forall\bar{z}(\bar{x}\leq\bar{z})\wedge\bigwedge_{i = 1}^{\length{\bar{Y}}} \forall\bar{z} Y_i(\bar{z}) \wedge \forall\bar{z}(\bar{z}\leq\bar{y})\wedge\forall\bar{u}'(\bar{u}\leq\bar{u}')]\vee
%\\
%&[\psi(\bar{Y},\bar{y})\wedge\bigwedge_{i = 1}^{\length{\bar{X}}} \forall\bar{z}\neg X_i(\bar{z}) \wedge \forall\bar{z}(\bar{x}\leq\bar{z})\wedge \\
%&(\bar{y} < \bar{w}_0 \vee \bigvee_{i = 1}^{\length{\bar{Y}}}\neg Y(\bar{w}_i))\wedge\forall\bar{u}'(\bar{y} < \bar{w}_0' \vee \bigvee_{i = 1}^{\length{\bar{Y}}}\neg Y(\bar{w}_i'))\to\bar{u}\leq\bar{u}']\vee \\
%&[\psi(\bar{Y},\bar{y})\wedge\bigwedge_{i = 1}^{\length{\bar{X}}} \forall\bar{z} X_i(\bar{z}) \wedge \forall\bar{z}(\bar{z}\leq\bar{x})\wedge\bigwedge_{i = 1}^{\length{\bar{Y}}} \forall\bar{z} Y_i(\bar{z}) \wedge \forall\bar{z}(\bar{z}\leq\bar{y})\wedge\forall\bar{u}'(\bar{u}\leq\bar{u}')]\big].
%\end{align*}
%
%Finally, consider a $\eqso(\LL)$ formula $\alpha$ in $\LL$-SNF. If $\alpha = \sum_{i = 1}^n\alpha_i$, then by induction we can consider $\alpha = \alpha_1 + (\sum_{i = 2}^n\alpha_i)$ and use $\tau(\alpha_1 + \tau(\sum_{i = 2}^n\alpha_i))$ as the rewrite of $\alpha$, which satisfies the hypothesis.
\qed

\begin{figure*}
%\begin{center}
%\begin{tikzpicture}
%\node[rectw] (n1) {$\E{0}$};
%\node[rectw, right=0.5cm of n1] (n2) {};
%\node[rectw, above=0.3cm of n2] (n3) {$\E{1}$}
%	edge[draw=white] node {\rotatebox{45}{$\subsetneq$}} (n1);
%\node[rectw, below=0.3cm of n2] (n4) {$\QE{0}$}
%        edge[draw=white] node {\rotatebox{315}{$\subsetneq$}} (n1);
%\node[rectw, right=0.3cm of n2] (n5) {$\QE{1}$}
%        edge[draw=white] node {\rotatebox{315}{$\subsetneq$}} (n3)
%         edge[draw=white] node {\rotatebox{45}{$\subsetneq$}} (n4);
%\node[rectw, right=0.3cm of n5] (n6) {$\U{1}$}       
%        edge[draw=white] node {$\subsetneq$} (n5);
%\node[rectw, right=0.3cm of n6] (n7) {$\QU{1}$}       
%        edge[draw=white] node {$=$} (n6);
%\node[rectw, right=0.3cm of n7] (n8) {$\E{2}$}       
%        edge[draw=white] node {$\subsetneq$} (n7);
%\node[rectw, right=0.3cm of n8] (n9) {$\QE{2}$}       
%        edge[draw=white] node {$=$} (n8);        
%\node[rectw, right=0.3cm of n9] (n10) {$\U{2}$}       
%        edge[draw=white] node {$\subsetneq$} (n9);
%\node[rectw, right=0.3cm of n10] (n11) {$\QU{2}$}       
%        edge[draw=white] node {$=$} (n10); 
%\node[rectw, right=0.3cm of n11] (n12) {$\sfo$}       
%        edge[draw=white] node {$=$} (n11); 
%\end{tikzpicture}
%\end{center}
\begin{center}
	\begin{tikzpicture}
	\node[rectw] (n1) {$\E{0}$};
	\node[rectw, right=0.5cm of n1] (n2) {};
	\node[rectw, above=0.3cm of n2] (n3) {$\E{1}$}
		edge[draw=white] node {\rotatebox{45}{$\subsetneq$}} (n1);
	\node[rectw, below=0.5cm of n2] (n4) {$\QE{0}$}
		edge[draw=white] node {\rotatebox{315}{$\subsetneq$}} (n1);
	\node[rectw, right=0.5cm of n2] (n5) {$\QE{1}$}
		edge[draw=white] node {\rotatebox{315}{$\subsetneq$}} (n3)
		edge[draw=white] node {\rotatebox{45}{$\subsetneq$}} (n4);
	\node[rectw, right=0.8cm of n5] (n6) {$\QU{1}$}       
		edge[draw=white] node {$\subsetneq$} (n5);
	\node[rectw, below=0.3cm of n6] (n7) {$\U{1}$}       
		edge[draw=white] node {\rotatebox{90}{$=$}} (n6);
	\node[rectw, right=1.0cm of n6] (n8) {$\QE{2}$}       
		edge[draw=white] node {$\subsetneq$} (n6);
	\node[rectw, below=0.3cm of n8] (n9) {$\E{2}$}       
		edge[draw=white] node {\rotatebox{90}{$=$}} (n8);        
	\node[rectw, right=1.0cm of n8] (n10) {$\QU{2}$}       
		edge[draw=white] node {$\subsetneq$} (n8);
	\node[rectw, below=0.3cm of n10] (n11) {$\U{2}$}       
		edge[draw=white] node {\rotatebox{90}{$=$}} (n10); 
	\node[rectw, right=0.5cm of n10] (n12) {$\sfo$}       
		edge[draw=white] node {$=$} (n10); 
	\end{tikzpicture}
\end{center}
%\vspace{1cm}
%\begin{center}
%	\begin{tikzpicture}
%	\node[rectw] (n1) {$\E{0}$};
%	\node[rectw, right=1.2cm of n1] (n2) {};
%	\node[rectw, above=0.2cm of n2] (n3) {$\E{1}$}
%	edge[draw=white] node {\rotatebox{45}{$\subsetneq$}} (n1);
%	\node[rectw, below=0.1cm of n2] (n4) {$\QE{0}$}
%	edge[draw=white] node {\rotatebox{315}{$\subsetneq$}} (n1);
%	\node[rectw, right=1.2cm of n2] (n5) {$\QE{1}$}
%	edge[draw=white] node {\rotatebox{315}{$\subsetneq$}} (n3)
%	edge[draw=white] node {\rotatebox{45}{$\subsetneq$}} (n4);
%	\node[rectw, right=1.3cm of n5] (n6) {\rotatebox{90}{$=$}}       
%	edge[draw=white] node {$\subsetneq$} (n5);
%	\node[rectw, above=0cm of n6] (n13) {$\U{1}$};
%	\node[rectw, below=0cm of n6] (n7) {$\QU{1}$};      
%	\node[rectw, right=1.8cm of n6] (n8) {\rotatebox{90}{$=$}}       
%	edge[draw=white] node {$\subsetneq$} (n6);
%	\node[rectw, above=0cm of n8] (n14) {$\E{2}$};
%	\node[rectw, below=0cm of n8] (n9) {$\QE{2}$};       
%	\node[rectw, right=1.8cm of n8] (n10) {\rotatebox{90}{$=$}}       
%	edge[draw=white] node {$\subsetneq$} (n8);
%	\node[rectw, above=0cm of n10] (n15) {$\U{2}$};
%	\node[rectw, below=0cm of n10] (n11) {$\QU{2}$};      
%	edge[draw=white] node {\rotatebox{90}{$=$}} (n10); 
%	\node[rectw, right=0.5cm of n10] (n12) {$\sfo$}       
%	edge[draw=white] node {$=$} (n10); 
%	\end{tikzpicture}
%\end{center}
\caption{The relationship between the $\sfo$-hierarchy and the $\eqso(\fo)$-hierarchy, where $\E{1}$ and $\QE{0}$ are incomparable. \label{fig-sfo-eqso}}
\vspace{-0.1cm}
\end{figure*}

As a consequence of Proposition~\ref{theo-pnf-snf} and Theorem~\ref{theo-pi1-pnf}, we obtain that $\E{i} \subsetneq \QE{i}$ for $i = 0,1$, and that $\sh{\LL} = \eqso(\LL)$ for $\LL$ equal to  $\Pi_1$, $\Sigma_2$ or $\Pi_2$. The following proposition completes our picture of the relationship between the $\sfo$-hierarchy and the $\eqso(\fo)$-hierarchy.
\begin{prop}\label{prop-rest}
The following properties hold:
\begin{itemize}
\item $\QE{0}$ and $\E{1}$ are incomparable, that is, $\E{1} \not\subseteq \QE{0}$ and $\QE{0} \not\subseteq \E{1}$,
\item $\QE{1} \subsetneq \QU{1}$.
\end{itemize}
\end{prop}
\proof
%!TEX root = main.tex

We give this proof in three parts.
First, we show that $\E{1} \not\subseteq \QE{0}$. 
For this inclusion to be true, it is required to hold for an arbitrary ordered relational signature $\R$, so it suffices to show it is not for at least one.
Let $\R$ be the minimal ordered signature that contains only the relation name $<$.
Suppose that there is a $\QE{0}$ formula $\alpha$ over $\R$ which is equivalent to the $\E{1}$ formula $\sa{x} \ex{y} (x < y)$. 
That is, for every finite $\R$-structure $\A$, $\sem{\alpha}(\A) = \size{\A} - 1$.
\martin{reformul\'e por qu\'e elijo ese R}
Suppose that $\alpha$ is in SNF, namely, $\alpha = \sum_{i = 1}^k \alpha_i$ for some fixed $k$. 
Since $\sem{\alpha}$ is not the identically zero function, consider some $\alpha_i$ that describes a non-null function. 
Let $\alpha_i = \sa{\bar{X}}\sa{\bar{x}}\varphi(\bar{X},\bar{x})$ where $\varphi$ is quantifier-free. 
Note that if $\length{\bar{X}} > 0$, then the function $\sem{\alpha}$ is in $\Omega(2^{\size{\A}})$, as it was proven in~\cite{SalujaST95}. 
Therefore, we have that $\alpha_i = \sa{\bar{x}}\varphi(\bar{x})$. 
We conclude our proof with the following claim.
\begin{clm}
	Let $\alpha = \sa{\bar{x}}\varphi(\bar{x})$	where $\varphi$ is quantifier free. 
	Then the function $\sem{\alpha}$ is either null, greater or equal to $n$, or is in $\Omega(n^2)$, where $n$ is the size of the input structure.
\end{clm}
\proof
	Note that each atomic sub-formula in $\varphi(\bar{x})$ is either $(x = y)$, $(x < y)$, $\top$ or a negation thereof, where $x,y\in\bar{x}$. 
	Suppose $\sem{\alpha}$ is not null and consider some $\R$-structure $\A$ such that $\sem{\alpha}(\A) > 0$. 
	Let $\bar{a}$ be an assignment to $\bar{x}$ such that $\A\models\varphi(\bar{a})$. 
	It can be seen that each assignment $\bar{a}'$ that has the same ordering over its variables\footnote{For example, if $\bar{a} = (a_1,a_2,a_3,a_4)$, such ordering may be $a_1 > a_3 = a_4 > a_2$, which has tree partitions.}	as $\bar{a}$ also satisfies $\A\models\varphi(\bar{a}')$. 
	If this ordering has $k$ partitions then $\binom{\size{\A}}{k}$ assignments for $\bar{x}$ will satisfy it, and therefore, $\sem{\alpha} \geq \binom{\size{\A}}{k}$. 
	If $k = 1$, then $\binom{\size{\A}}{k} = \size{\A}$, and if $k \geq 2$, then $\binom{\size{\A}}{k} \in \Omega(\size{\A}^2)$, which proves the claim.
\qed
\martin{cambie order por ordering, creo que se entiende mejor}
Now we show that $\QE{0} \not\subseteq \E{1}$. 
In Theorem \ref{theo-pi1-pnf} we proved that there is no formula in $\loge{1}$-PNF equivalent to the formula $\alpha = 2$. 
Every formula in $\E{1}$ can be expressed in $\loge{1}$-PNF, which implies that $2 \not\in \E{1}$. Therefore, given that $2 \in \QE{0}$ by the definition of this logic, we conclude that $\QE{0} \not\subseteq \E{1}$.

Finally, we prove that $\eqso(\loge{1})\subsetneq\eqso(\logu{1})$. 
For inclusion, let $\alpha$ be a formula in $\eqso(\loge{1})$. 
Suppose that it is in $\loge{1}$-SNF, namely, $\alpha = c + \sum_{i = 1}^{n}\alpha_i$. 
Let $\alpha_i = \sa{\bar{X}}\sa{\bar{x}}\ex{\bar{y}}\varphi_i(\bar{X},\bar{x},\bar{y})$, where $\varphi_i$ is quantifier-free for each $\alpha_i$. 
We use the same construction used in \cite{SalujaST95}, and we obtain that the formula $\ex{\bar{y}}\varphi_i(\bar{X},\bar{x},\bar{y})$ is equivalent to $\sa{\bar{y}}\,[\varphi_i(\bar{X},\bar{x},\bar{y}) \wedge \fa{\bar{y}'}(\varphi_i(\bar{X},\bar{x},\bar{y}')\to\bar{y}\leq\bar{y}')]$ for every assignment to $(\bar{X},\bar{x})$. 
We do this replacement for each $\alpha_i$, and we obtain an equivalent formula to $\alpha$ in $\eqso(\logu{1})$.

To prove that the inclusion is proper, consider the $\eqso(\logu{1})$ formula $\sa{x} \fa{y}(y = x)$. 
This formula defines the following function over each ordered structure $\A$:
$$
\sem{\alpha}(\A) = 
\begin{cases}
1 &\A \text{ has one element}\\
0 &\text{ otherwise}.
\end{cases}
$$
Suppose that there exists an equivalent formula $\alpha$ in $\eqso(\loge{1})$. 
Also, suppose that it is in $\LL$-SNF, so $\alpha = \sum_{i = 1}^n\sa{\bar{X}}\sa{\bar{x}}\ex{\bar{y}}\varphi_i(\bar{X},\bar{x},\bar{y})$. 
Consider a structure $\A'$ with one element. 
We have that for some $i$, there exists an assignment $(\bar{B},\bar{b},\bar{a})$ for $(\bar{X},\bar{x},\bar{y})$ such that $\A' \models\varphi_i(\bar{B},\bar{b},\bar{a})$. 
Consider now the structure $\A''$ that is obtained by duplicating $\A'$, as we did for Theorem \ref{theo-pi1-pnf}. 
Note that $\A''\models\varphi_i(\bar{B},\bar{b},\bar{a})$, which implies that $\sem{\alpha}(\A' \uplus \A'') > 1$, which leads to a contradiction.

\qed
The relationship between the two hierarchies is summarized in Figure \ref{fig-sfo-eqso}.
Our hierarchy and the one proposed in~\cite{SalujaST95} only differ in~$\Sigma_0$ and~$\Sigma_1$. 
Interestingly, we show next that this difference is crucial for finding classes of functions with easy decision versions and good closure properties.