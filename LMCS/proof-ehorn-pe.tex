%Let $\shdhsat$ be the problem of counting the number of satisfying assignments of a disjunction of Horn formulas. 
In~\cite{PagourtzisZ06}, Pagourtzis and Zachos gave a $\totp$ procedure that computes the number of satisfying assignments of a DNF formula. This procedure can be easily extended to receive Horn formulas, and furthermore, a disjunction of Horn formulas. Hence $\shdhsat$ is in $\totp$. As we saw in Theorem~\ref{sigma2hard}, $\shdhsat$ is complete for $\eqso(\ehorn)$ under parsimonious reductions. Let $\alpha$ be a formula in $\eqso(\ehorn)$ and let $g_{\alpha}$ be the reduction to $\shdhsat$. Then the $\totp$ procedure consist in computing $g_{\alpha}(\enc(\A))$ for each input $\enc(\A)$ and then simulate the $\totp$ procedure for $\shdhsat$ on input $g_{\alpha}(\enc(\A))$. Therefore, we conclude that $\alpha$ defines a function in~$\totp$.