To show how to evaluate a $\qso(\pfp)$-formula, we construct recursively a $\shpspace$-machine $M_{\alpha}$ for each $\qso(\pfp)$ formula $\alpha$ over a signature $\R$. This machine runs in non-deterministic polynomial space and, on input $(\A,v,V)$, accepts in $\sem{\alpha}(\A,v,V)$ of its non-deterministic paths for each $(\A,v,V) \in \ostr[\R]^*$. Suppose $\A$ has domain $A$. If $\alpha$ is a $\pfp$-formula $\varphi$, then the machine checks if $(\A,v,V)\models\varphi$ deterministically in polynomial space~\cite{L04}, and accepts if and only if it holds true. If $\alpha$ is a constant $s$, it produces $s$ branches and accepts in all of them. If $\alpha = (\beta \add \gamma)$, then it chooses between 0 or 1, if it is 0 (1), it simulates $M_{\beta}$ ($M_{\gamma}$) on input $(\A,v,V)$. If $\alpha = (\beta \mult \gamma)$, it simulates $M_{\beta}$ on input $(\A,v,V)$ and on each accepting path, it continues simulating $M_{\gamma}$ on input $(\A,v,V)$.
If $\alpha = \sa{x}\beta$, it chooses $a\in A$ non-deterministically and simulates $M_{\beta}$ on input $(\A,v[a/x],V)$. If $\alpha = \pa{x}\beta$, it simulates $M_{\beta}$ on input $(\A,v[a/x],V)$ consecutively for each $a\in A$, continuing to the next $A$-value if the run of $M_{\beta}$ accepts. If $\alpha = \sa{X}\beta$, it chooses $B\in A^{arity(X)}$ and simulates $M_{\beta}$ on input $(\A,v,V[B/X])$. If $\alpha = \pa{X}\beta$, it simulates $M_{\beta}$ on input $(\A,v,V[B/X])$ consecutively for each $B\in A^{arity(X)}$, again continuing to the next $A$-value if the run of $M_{\beta}$ accepts. This covers all possible cases for $\alpha$, and each of these steps can be computed in polynomial space. Let $\alpha$ be a formula in $\qso(\pfp)$ over a signature $\R$ and let $f$ be a function over $\R$ such that $f(\enc(\A))$ is equal to the accepting paths of $M_{\alpha}$ on input $(\A,v,V)$ for some $(\A,v,V) \in \ostr[\R]^*$. We have that $f$ is a $\shpspace$ function over $\R$, which implies that $f$ is also a $\fpspace$ function over $\R$, by the fact that $\shpspace = \fpspace$ \cite{Ladner89}, and that $f(\enc(\A)) = \sem{\alpha}(\A)$ for every $\A\in\ostr[\R]$.

For the second condition, let $f\in \fpspace$ defined over some $\R$. Let $\ell\in\nat$ be such that for each $\A\in\ostr[\R]$, $\lceil\log_2 f(\enc(\A)) \rceil \leq 2^{n^\ell}$ (i.e. $2^{n^\ell}$ is an upper bound for the output size), where $\A$ has a domain of size $n$. Let $X$ be a second-order variable of arity $\ell$. Consider a linear order over predicates of arity $\ell$ given by the formula 
$$
\varphi_{<}(X,Y) = \exists\bar{u}\big[\neg X(\bar{u})\wedge Y(\bar{u})\wedge \forall\bar{v}\big(
\bar{u}<\bar{v}\to(X(\bar{u})\iff Y(\bar{v}))\big)\big].
$$
Namely, we use relations to encode numbers with at most $2^{n^\ell}$-bits where the empty relation represents $0$ and the total-relation represents $2^{2^{n^\ell}}-1$.
Furthermore, each relation $X$ indexes a position in the binary output of $f(\enc(\A))$ as follows.
Consider a polynomial space machine over the $\R$ that receives as input an $\R$-structure $\A$ and a number $p$ encoded by a relation $X$. Then the machine accepts if, and only if, the $p$-th bit of $f(\enc(\A))$ is $1$. 
Since this procedure works in polynomial space, it can be described in $\pfp$ \cite{AbiteboulV89} by a formula $\Phi(X)$ where the free variable $X$ encodes the number $p$. Then, similar than the previous proof we define:
$$
\alpha := \sa{X}(\Phi(X)\mult\varphi(X)),
$$ 
where $\varphi(X) = \pa{Y}(\varphi_{<}(Y,X)\mapsto 2)$. Note that for each $B\subseteq A^{\ell}$ such that $B$ is the $m$-th element in the order $\varphi_{<}(X,Y)$, it holds that $\sem{\varphi(B)}(\A) = 2^m$. Therefore, $\alpha\in\qso(\pfp)$ and $\sem{\alpha}(\A) = f(\enc(\A))$.