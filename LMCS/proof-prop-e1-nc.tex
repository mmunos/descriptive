Towards a contradiction, assume that the statement is false, namely, $\E{1}$ is closed under binary sum. Consider the formula $\sa{x}(x = x) \in \E{1}$ over some signature $\R$. Clearly, for every finite $\R$-structure $\A$ with $n$ elements it holds that $\alpha(\enc(\A)) = n$. From our assumption, there exists some formula in $\E{1}$ equivalent to the formula $\alpha \add \alpha$, which describes the function $2n$. Let $\sa{\bar{X}}\sa{\bar{x}}\exists\bar{y}\,\varphi(\bar{X},\bar{x},\bar{y})$ be this formula, where $\varphi$ is a quantifier-free first-order formula. Note that the function defined by this formula is equal or greater than the one defined by $\sa{\bar{X}}\sa{\bar{x}}\sa{\bar{y}}\,\varphi(\bar{X},\bar{x},\bar{y})$ divided by a polynomial factor. More specifically, for each ordered structure $\A$ with $n$ elements, we have the following inequality:
$$
\sem{\sa{\bar{X}}\sa{\bar{x}}\sa{\bar{y}}\,\varphi(\bar{X},\bar{x},\bar{y})}(\A)
\leq 
\sem{\sa{\bar{X}}\sa{\bar{x}}\exists\bar{y}\,\varphi(\bar{X},\bar{x},\bar{y})}(\A) \cdot  n^{\length{\bar{y}}} \leq 2n^{\length{\bar{y}}+1} 
$$
Note that the formula $\sa{\bar{X}}\sa{\bar{x}}\sa{\bar{y}}\,\varphi(\bar{X},\bar{x},\bar{y})$ defines a function in $\E{0}$ and it was shown by the authors in \cite{SalujaST95} that every function in $\E{0}$ grows exponentially over the size of the structure for large enough structures, when $\length{\bar{X}} > 0$. This contradict the above inequality and, therefore, for $\sa{\bar{X}}\sa{\bar{x}}\exists\bar{y}\,\varphi(\bar{X},\bar{x},\bar{y})$ we have that $\length{\bar{X}} = 0$.

Now, if the formula for defining $\alpha + \alpha$ is of the form $\sa{\bar{x}}\exists\bar{y}\,\varphi(\bar{x},\bar{y})$ consider a structure $\mathfrak{1}$ with only one element $a$. We have that $\sem{\sa{\bar{x}}\exists\bar{y}\,\varphi(\bar{x},\bar{y})}(\mathfrak{1}) = 2$, but the only possible assignment to $\bar{x}$ is the tuple $(a,\ldots,a)$ so $\sem{\sa{\bar{x}}\exists\bar{y}\,\varphi(\bar{x},\bar{y})}(\mathfrak{1}) \leq 1$, which leads to a contradiction.
