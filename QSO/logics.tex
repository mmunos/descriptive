Consider $\bbN$ as the set of non-negative integers. Then given a relational signature $\R$, the set of quantitative second-order logic formulas (or just $\qso$-formulas) over $\R$ is given by the following grammar:
\begin{eqnarray}
\label{eq-def-qso}
  \alpha & := & \varphi \ \mid \ s \ \mid \ (\alpha \add \alpha) \ \mid\ (\alpha \mult \alpha) \ \mid \ \sa{x} \alpha \ \mid \ \pa{x} \alpha \ \mid \ \sa{X} \alpha \ \mid \ \pa{X} \alpha 
\end{eqnarray}
where $\varphi$ is an $\so$-formula over $\R$, $s \in \bbN$, $x \in \fv$ and $X \in \sv$.

Let $\R$ be a relational signature, $\A$ be a finite $\R$-structure with domain $A$, $v$ a first-order assignment for $\A$ and $V$ a second-order assignment for $\A$. Then the \emph{evaluation} of a $\qso$-formula $\alpha$ over $(\A, v, V)$ is defined as a function $\sem{\alpha}$ that on input $(\A, v, V)$ returns a non-negative integer in $\bbN$. Formally, the function $\sem{\alpha}$ is recursively defined as follows:
$$
\renewcommand{\arraystretch}{1.7}
\begin{array}{rcl} 
\sem{\varphi}(\A, v, V) & = & 
\begin{cases}
1 & \mbox{if } (\A, v, V) \models \varphi \\
0 & \mbox{otherwise}
\end{cases}\\
\sem{s}(\A, v, V) & = & s \\
\sem{\alpha_1 \add \alpha_2}(\A, v, V) & = & \sem{\alpha_1}(\A, v, V) + \sem{\alpha_2}(\A, v, V)\\
\sem{\alpha_1 \mult \alpha_2}(\A, v, V) & = & \sem{\alpha_1}(\A, v, V) \cdot \sem{\alpha_2}(\A, v, V)\\ 
\sem{\sa{x} \alpha}(\A, v, V) & = & \displaystyle \sum_{a \in A} \sem{\alpha}(\A,v[a/x],V)\\
\sem{\pa{x} \alpha}(\A, v, V) & = & \displaystyle \prod_{a \in A} \sem{\alpha}(\A,v[a/x],V)\\
\sem{\sa{X} \alpha}(\A, v, V) & = & \displaystyle \sum_{B \subseteq A^{\arity(X)}} \sem{\alpha}(\A, v, V[B/X])\\
\sem{\pa{X} \alpha}(\A, v, V) & = & \displaystyle \prod_{B \subseteq A^{\arity(X)}} \sem{\alpha}(\A, v, V[B/X])
\end{array}
$$
A formula in $\qso$ can mention the usual quantifiers in $\so$ (that is, $\exists x$ and $\exists X$) and the quantifiers that make use of addition and multiplication (that is, $\Sigma x$, $\Pi x$, $\Sigma X$ and $\Pi X$), which are called {\em quantitative quantifiers} . A $\qso$-formula $\alpha$ is said to be a \emph{sentence} if it does not have any free variable, that is, every variable in $\alpha$ is under the scope of a usual quantifier or a quantitative quantifier. It is important to notice that if $\alpha$ is a $\qso$-sentence over a relational signature $\R$, then for every finite $\R$-structure $\A$, first-order assignments $v_1$, $v_2$ for $\A$ and second-order assignments $V_1$, $V_2$ for $\A$, it holds that:
\begin{eqnarray*}
\sem{\alpha}(\A, v_1, V_1) & = & \sem{\alpha}(\A, v_2, V_2).
\end{eqnarray*}
Thus, in such a case we use the term $\sem{\alpha}(\A)$ to denote $\sem{\alpha}(\A, v, V)$, for some arbitrary first-order assignment $v$ for $\A$ and some arbitrary second-order assignment $V$ for $\A$. 

In this paper, we consider several fragments of $\qso$, which are obtained by restricting the syntax of the formula $\varphi$ in \eqref{eq-def-qso} or the use of the quantitative quantifiers. Let $\qfo$ be the fragment of $\qso$ obtained by only allowing the quantitative quantifiers $\Sigma x$, $\Pi x$ in \eqref{eq-def-qso}. Let $\eqso$ be a fragment of $\qso$ defined as $\qfo$ but also allowing the quantitative quantifier $\Sigma X$. Moreover, assuming that $\LL$ is a fragment of $\so$, let $\qso(\LL)$ be the fragment of $\qso$ obtained by restricting $\varphi$ in \eqref{eq-def-qso} to be a formula in $\LL$, and let $\eqso(\LL)$ be the fragment of $\eqso$ obtained by imposing the same restriction. In particular, in this paper we consider the following fragments $\LL$: $\fo$, which is obtained by disallowing the use of the second-order quantifier ($\exists X$ or $\forall X$) in the formula $\varphi$ in \eqref{eq-def-qso}, and $\eso$, which is obtained by allowing $\varphi$ in \eqref{eq-def-qso} to be a formula of the form $\exists X_1 \cdots \exists X_k \, \psi$, where $\psi$ is an $\fo$-formula. 


\begin{theorem}
	Let $\alpha$ be a $\qso$-formula in some fragment of $\qso$. There is a $\qso$ formula $\alpha'$ in the same fragment such that $\alpha'$ does not have any subformula of the form $(\beta_1 \cdot \beta_2)$ and $\sem{\alpha} = \sem{\alpha'}$.
\end{theorem}
\begin{proof}
	Let $\{+, \cdot, \Sigma, \Pi\}$ be the set of {\it algebraic operators} in $\qso$. We will prove by induction over the number of algebraic operators in a $\qso$-formula $\alpha$ that there exists some $\qso$-formula $\alpha'$ that respects the stated condition. Let $\text{op}(\alpha)$ be the number of algebraic operators in $\alpha$. The case $\text{op}(\alpha) = 0$ is trivial. Let $n\in\nat$ be such that for each $\qso$-formula $\beta$ such that $\text{op}(\beta) < n$ there is some $\beta'$ that respects the condition. Let $\alpha$ be a $\qso$-formula such that $\text{op}(\alpha) = n$. Note that if $\alpha$ is of the form $\alpha = (\beta_1 + \beta_2)$, $\alpha = \Sigma x \beta$, $\alpha = \Pi x \beta$, $\alpha = \Sigma X \beta$ or $\alpha = \Pi X \beta$ the proof follows directly. Therefore we assume that $\alpha = (\beta_1 \cdot \beta_2)$. We separate the proof in two main cases:
	\begin{enumerate}
		\item At least one of $\beta_1, \beta_2$ is not of the form $\Pi x \gamma$, $\Pi X \gamma$, or $\varphi$, where $\varphi$ is an $\so$-formula. Without loss of generality, let $\beta_1$ be such one.
		\begin{enumerate}
			\item $\beta_1 = s \in \nat$. Then we use $\alpha' = \beta_2 + \cdots + \beta_2$, where $\beta_2$ is repeated $s$ times, and if $s = 0$, take $\alpha' = \perp$. Since each of the formulas paired by $+$ has less than $n$ algebraic operators, the statement holds.
			\item $\beta_1 = (\gamma_1 + \gamma_2)$. Then, we use $\alpha' = (\gamma_1\cdot\beta_2 + \gamma_2\cdot\beta_2)$ and each of the formulas paired by $+$ have less than $n$ algebraic operators so the statement holds.
			\item $\beta_1 = \Sigma x \gamma$. Then, we use $\alpha' = \Sigma x (\gamma\cdot\beta_2)$ and as in the previous case the statement holds.
			\item $\beta_1 = \Sigma X \gamma$, which we treat exactly as the previous one.
		\end{enumerate}
		\item Both of $\beta_1$ and $\beta_2$ are of the form $\Pi x \gamma$, $\Pi X \gamma$, or $\varphi$, where $\varphi$ is an $\so$-formula. This case also separates in further cases. We define $(\varphi \mapsto \alpha) := (\neg\varphi + (\varphi\cdot\alpha))$ for which $\sem{(\varphi \mapsto \alpha)}(\A,v,V)$ equals $\sem{\alpha}(\A,v,V)$ when $(\A,v,V)\models\varphi$ and 1 otherwise.
		\begin{enumerate}
			\item $\beta_1 = \Sigma x \gamma_1(x)$ and $\beta_2 = \Sigma y \gamma_2(y)$. Then we use $\alpha' = \Sigma x(\gamma_1(x)\cdot\gamma_2(x))$, and the statement holds.
			\item $\beta_1 = \Sigma X \gamma_1(X)$ and $\beta_2 = \Sigma Y \gamma_2(Y)$. [something about arities].
			\item $\beta_1 = \Sigma X \gamma_1(X)$ and $\beta_2 = \Sigma x \gamma_2(x)$ (w.l.o.g.). [use singleton predicates].
			\item $\beta_1 = \Sigma x \gamma$ and $\beta_2 = \varphi$, which is an $\so$-formula (w.l.o.g.). Then, we use $\alpha' = \Sigma x (\gamma \cdot \varphi)$, and the statement holds.
			\item $\beta_1 = \Sigma X \gamma$ and $\beta_2 = \varphi$, which is an $\so$-formula (w.l.o.g.). We treat this case exactly as the previous one.
			\item $\beta_1 = \varphi_1$ and $\beta_2 = \varphi_2$ both $\so$-formulas. Then, we use $\alpha' = (\varphi_1 \wedge \varphi_2)$ and the statement holds.
		\end{enumerate}
	\end{enumerate}
\end{proof}

We define an operator which extends the Least Fixed Point logic to counting. Recall that a fixed point operator is defined by a formula $\varphi(x_1,\ldots,x_k,R)$ which is positive on $R$ where $R$ is a predicate or arity $k$. The operator $T_{\varphi}:2^{A^k} \to 2^{A^k}$ is defined as $T_{\varphi}(X) = \{(a_1,\ldots,a_k)\mid (\A,X)\models \varphi(a_1,\ldots,a_k,R) \}$, for each $X\subseteq A^k$ and $X^*\subseteq A^k$ is a fixed point of $T_{\varphi}$ if $T_{\varphi}(X^*) = X^*$. Then, the formula $[\textbf{lfp}\,\varphi(x_1,\ldots,x_k,R)](y_1,\ldots,y_k)$ is evaluated at the least fixed point of $T_{\varphi}$.

We extend QSO with the operator $[\textbf{alfp}_{\varphi(y_1,\ldots,y_k,R)}\alpha(x_1,\ldots,x_{\ell},R,\pi)](x_1,\ldots,x_{\ell})$. It is defined by a FO-formula $\varphi$ and a QFO-formula $\alpha$ which mentions a placeholder function $\pi$ with domain $\fv^{\ell}$. 

To formally characterize its semantics we define the partial function $\zeta_{\varphi,\alpha}:2^{A^k} \to (A^{\ell}\times\nat)$ for each $\varphi$ and $\alpha$ as follows. Let $\A$ be a structure. For $R = \emptyset$, it holds that $\zeta_{\varphi,\alpha}[\emptyset] = \sem{\alpha\mid_{\pi(\bar{u})\to 0}}$. For each $R \subseteq A^k$ such that $T_{\varphi}(R) \neq R$, let $\beta$ be a virtual $\qfo$ formula such that $\sem{\beta}(\A,v) = \zeta_{\varphi,\alpha}[R](v)$ for each first order assignment $v$, and it holds that $\zeta_{\varphi,\alpha}[T_{\varphi}(R)] = \sem{\alpha\mid_{\pi\to\beta}}$.

For a given first order asignment $v$, the operator is evaluated as:
$$
[\textbf{alfp}_{\varphi(y_1,\ldots,y_k,R)}\alpha(x_1,\ldots,x_{\ell},R,\pi)](\A,v) = \zeta_{\varphi,\alpha}[X^*](v),
$$
where $X^*$ is the least fixed point of $\varphi$.

As an example, we will show a formula that counts the number of paths of size $n$ of a structure $\A$ with a binary relation $E$. First we define $\varphi(x,R)$:
$$
\varphi(x,R) = \forall y(x < y \vee x = y) \vee \exists z(R(z) \wedge \varphi_{succ}(z,x)),x
$$
where $\varphi_{succ}(x,y)$ is a formula that is satisfied by pairs $(x,y)$ that are consecutive in the order $<$. That is, $\varphi_{succ}(x,y) = x < y \wedge \forall z((x < z \wedge z < y) \to (x = z \vee z = y) )$. Now we define $\alpha(R)$ as:
$$
\alpha(x,y,R) = \neg \exists z(R(z)) + \Sigma z[E(x,z)\cdot \pi(z,y)].
$$
Then, the formula $[\alfp_{\varphi(x,R)} \alpha(x,y,R)](x,y)$ when evaluated on $(\A,a,b)$ will count the number of paths of size $n$ from $a$ to $b$.

We also define the less powerful operator $\pth$ as follows. Given a positive formula $\varphi(x,R)$ and a formula $\psi(x,y)$ we define $[\pth_{\varphi(x,R)} \psi(x,y)](x,y)$ as a formula that when evaluated on $(\A, a, b)$ induces a graph over $\A$ with the binary relation determined by $\psi(x,y)$ and counts the number of paths from $a$ to $b$ of a size determined by the $\varphi(x,R)$. Formally, we define:
$$
\sem{[\pth_{\varphi(x,R)} \psi(x,y)](x,y)} = \sem{[\alfp_{\varphi(x,R)} \alpha_{\psi}(x,y,R)](x,y)},
$$
where $\alpha_{\psi}(x,y,R) = \neg \exists z(R(z)) + \Sigma z[\psi(x,z)\cdot \pi(z,y)]$.


\begin{theorem}
	Given a positive $\fo$ formula $\varphi(\bar{x},R)$ and a $\qfo$ formula $\alpha(\bar{x})$, there exists a $\qso$ formula $\beta(\bar{x})$ such that $\sem{[\alfp_{\varphi(\bar{x},R)} \alpha_{\psi}(\bar{x},R)](\bar{x})} = \sem{\beta(\bar{x})}$.
\end{theorem}
\begin{proof}
	
\end{proof}
