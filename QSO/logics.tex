Fix a semiring $\SR = (S, \adds, \mults, \zero, \one)$. Then given a relational signature $\R$, the set of $\SR$-quantitative second-order logic formulas (or just $\qso(\SR)$-formulas) over $\R$ is given by the following grammar:
\begin{eqnarray}
\label{eq-def-qso}
  \theta & := & \varphi \ \mid \ s \ \mid \ (\theta \add \theta) \ \mid\ (\theta \mult \theta) \ \mid \ \sa{x} \theta \ \mid \ \pa{x} \theta \ \mid \ \sa{X} \theta \ \mid \ \pa{X} \theta 
\end{eqnarray}
where $\varphi$ is an $\so$-formula over $\R$, $s \in S$, $x \in \fv$ and $X \in \sv$.

Let $\R$ be a relational signature, $\A$ be a finite $\R$-structure with domain $A$, $v$ a first-order variable assignment for $\A$ and $V$ a second-order variable assignment for $\A$. Then the \emph{evaluation} of a $\qso(\SR)$-formula $\theta$ over $(\A, v, V)$ is defined as a function $\sem{\theta}$ that on input $(\A, v, V)$ returns a value in the semiring $\SR$. Formally, the function $\sem{\theta}$ is recursively defined as follows:
\begin{eqnarray*}
\sem{\varphi}(\A, v, V) & = & 
\begin{cases}
\one & \mbox{if } (\A, v, V) \models \varphi \\
\zero & \mbox{otherwise}
\end{cases}\\
\sem{s}(\A, v, V) & = & s \\
\sem{\theta_1 \add \theta_2}(\A, v, V) & = & \sem{\theta_1}(\A, v, V) \adds \sem{\theta_2}(\A, v, V)\\
\sem{\theta_1 \mult \theta_2}(\A, v, V) & = & \sem{\theta_1}(\A, v, V) \mults \sem{\theta_2}(\A, v, V)\\ 
\sem{\sa{x} \theta}(\A, v, V) & = & \bigadds_{a \in A} \sem{\theta}(\A,v[a/x],V)\\
\sem{\pa{x} \theta}(\A, v, V) & = & \bigmults_{a \in A} \sem{\theta}(\A,v[a/x],V)\\
\sem{\sa{X} \theta}(\A, v, V) & = & \bigadds_{B \subseteq A^{\arity(X)}} \sem{\theta}(\A, v, V[B/X])\\
\sem{\pa{X} \theta}(\A, v, V) & = & \bigmults_{B \subseteq A^{\arity(X)}} \sem{\theta}(\A, v, V[B/X])
\end{eqnarray*}
A formula in $\qso(\SR)$ can mention the usual quantifiers in $\so$ (that is, $\exists x$ and $\exists X$) and the quantifiers that make use of the two operations in the semiring (that is, $\Sigma x$, $\Pi x$, $\Sigma X$ and $\Pi X$), which are called {\em semiring quantifiers} . A $\qso(\SR)$-formula $\theta$ is said to be a \emph{sentence} if it does not have any free variable, that is, every variable in $\theta$ is under the scope of a usual quantifier or a semiring quantifier. It is important to notice that if $\theta$ is a $\qso(\SR)$-sentence over a relational signature $\R$, then for every finite $\R$-structure $\A$, first-order variable assignments $v_1$, $v_2$ for $\A$ and second-order variable assignments $V_1$, $V_2$ for $\A$, it holds that:
\begin{eqnarray*}
\sem{\theta}(\A, v_1, V_1) & = & \sem{\theta}(\A, v_2, V_2).
\end{eqnarray*}
Thus, in such a case we use the term $\sem{\theta}(\A)$ to denote $\sem{\theta}(\A, v, V)$, for some arbitrary first-order variable assignment $v$ for $\A$ and some arbitrary second-order variable assignment $V$ for $\A$. 

In this paper, we consider several fragments of $\qso(\SR)$, which are obtained by restricting the syntax of the formula $\varphi$ in \eqref{eq-def-qso} or the use of the semiring quantifiers. Let $\eqso(\SR)$ be the fragment of $\qso(\SR)$ that is obtained by only allowing the semiring quantifiers $\Sigma x$, $\Pi x$ and $\Sigma X$ in \eqref{eq-def-qso}. Moreover, assuming that $\LL$ is a fragment of $\so$, let $\qso(\SR, \LL)$ be the fragment of $\qso(\SR)$ obtained by restricting $\varphi$ in \eqref{eq-def-qso} to be a formula in $\LL$, and let $\eqso(\SR, \LL)$ be the fragment of $\eqso(\SR)$ obtained by imposing the same restriction. In particular, in this paper we consider the following fragments $\LL$: $\fo$, which is obtained by disallowing the use of the second-order quantifier ($\exists X$) in the formula $\varphi$ in \eqref{eq-def-qso}, and $\eso$, which is obtained by allowing $\varphi$ in \eqref{eq-def-qso} to be a formula of the form $\exists X_1 \cdots \exists X_k \, \psi$, where $\psi$ is an $\fo$-formula. 




