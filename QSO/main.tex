\documentclass[a4paper]{article}

\usepackage{fullpage}
\usepackage{amsmath}
\usepackage{amssymb}
\usepackage{amsfonts}
\usepackage{MnSymbol}
\usepackage{bbold}
\usepackage{mathrsfs}
\usepackage{amsthm}
\usepackage{todonotes}

%\usepackage[utf8]{inputenc}
%\usepackage{latexsym}
%\usepackage{stmaryrd}
%\usepackage{bbold}
%\usepackage[T1]{fontenc}
%\usepackage{longtable}
%\usepackage[newitem,newenum,flushleft,neverdecrease]{paralist}
%\usepackage{subfig}

\usepackage{tikz}
\usetikzlibrary{automata}
\usetikzlibrary{arrows}
\usetikzlibrary{trees}
\usetikzlibrary{shapes}
\usetikzlibrary{fit}
\usetikzlibrary{calc}
\usetikzlibrary{positioning}

% packages
\usepackage[english]{babel}
\usepackage{amsmath,amssymb,amsfonts,mathrsfs,latexsym,stmaryrd}
\usepackage{array}
\usepackage{url}
\usepackage{comment}
\usepackage[newitem,newenum,flushleft,neverdecrease]{paralist}
\usepackage{epsfig}
\usepackage{ifpdf}
%\usepackage[usenames,dvipsnames]{color}
%\usepackage[colorlinks=true,linkcolor=Blue,citecolor=Blue,urlcolor=Magenta,pdfpagelabels,plainpages=false]{hyperref}
%\usepackage[all]{hypcap}

% fix for metapost
\ifpdf
  \DeclareGraphicsRule{*}{mps}{*}{}
\else
  \DeclareGraphicsRule{*}{eps}{*}{}
\fi

\newtheorem{theorem}{Theorem}[section]
\newtheorem{example}[theorem]{Example}
\newtheorem{definition}[theorem]{Definition}
\newtheorem{lemma}[theorem]{Lemma}
\newtheorem{claim}[theorem]{Claim}
\newtheorem{property}[theorem]{Property}
\newtheorem{proposition}[theorem]{Proposition}
\newtheorem{invariant}[theorem]{Invariant}

\newenvironment{reptheorem}[1]{\begin{trivlist} \item[ {\textbf{Theorem #1}.}] \it}{\end{trivlist}}
\newenvironment{replemma}[1]{\begin{trivlist} \item[ {\textbf{Lemma #1}.}] \it}{\end{trivlist}}
\newenvironment{repproposition}[1]{\begin{trivlist} \item[{\textbf{Proposition #1}.}] \it}{\end{trivlist}}

\newenvironment{runexample}{\begin{example}}{\end{example}}
\newenvironment{runexamplec}[1]{\begin{trivlist} \item[\hskip 15pt {{\sc Example #1 (continued)}.}] \it}{\end{trivlist}}

% environments
\def\labelitemi{\textbullet}
\setlength{\pltopsep}{\smallskipamount}
\setlength{\plitemsep}{\smallskipamount}
\newenvironment{dotlist}{\begin{compactitem}}{\end{compactitem}}
\newenvironment{numlist}{\begin{compactenum}}{\end{compactenum}}
\newenvironment{romlist}{\begin{compactenum}[i)]}{\end{compactenum}}
\newenvironment{condlist}{\begin{compactenum}[{(C}1{)}]}{\end{compactenum}}
\newenvironment{proplist}{\begin{compactenum}[{(P}1{)}]}{\end{compactenum}}
\newenvironment{axiomlist}{\begin{compactenum}[{(A}1{)}]}{\end{compactenum}}
\newcommand{\refeq}[1]{\hyperref[#1]{(\ref*{#1})}}
\newcommand{\refcond}[1]{\hyperref[#1]{C\ref*{#1}}}
\newcommand{\refprop}[1]{\hyperref[#1]{P\ref*{#1}}}
\newcommand{\refaxiom}[1]{\hyperref[#1]{A\ref*{#1}}}
\newcommand{\margincomment}[1]{\marginpar{\small\textit{#1}}}

\providecommand{\qed}{\hfill $\Box$}
\renewcommand{\qed}{\hfill $\Box$}
\newcommand{\eproof}{\qed}

% domains
\newcommand{\bbB}{\mathbb{B}}
\newcommand{\bbD}{\mathbb{D}}
\newcommand{\bbI}{\mathbb{I}}
\newcommand{\bbJ}{\mathbb{J}}
\newcommand{\bbN}{\mathbb{N}}
\newcommand{\bbNp}{\mathbb{N}_{>0}}
\newcommand{\bbNinf}{\mathbb{N}\cup\{\infty\}}
\newcommand{\bbNpinf}{\mathbb{N}_{>0}\cup\{\infty\}}
\newcommand{\bbZ}{\mathbb{Z}}
\newcommand{\bbP}{\mathbb{P}}
\newcommand{\bbQ}{\mathbb{Q}}
\newcommand{\bbQp}{\mathbb{Q}_{>0}}
\newcommand{\bbR}{\mathbb{R}}
\newcommand{\bbRp}{\mathbb{R}_{>0}}
\newcommand{\bbC}{\mathbb{C}}
\newcommand{\bbS}{\mathbb{S}}

% classes

\newcommand{\cA}{\mathcal{A}}
\newcommand{\cB}{\mathcal{B}}
\newcommand{\cC}{\mathcal{C}}
\newcommand{\cD}{\mathcal{D}}
\newcommand{\cE}{\mathcal{E}}
\newcommand{\cF}{\mathcal{F}}
\newcommand{\cG}{\mathcal{G}}
\newcommand{\cH}{\mathcal{H}}
\newcommand{\cI}{\mathcal{I}}
\newcommand{\cJ}{\mathcal{J}}
\newcommand{\cK}{\mathcal{K}}
\newcommand{\cL}{\mathcal{L}}
\newcommand{\cM}{\mathcal{M}}
\newcommand{\cN}{\mathcal{N}}
\newcommand{\cO}{\mathcal{O}}
\newcommand{\cP}{\mathcal{P}}
\newcommand{\cQ}{\mathcal{Q}}
\newcommand{\cR}{\mathcal{R}}
\newcommand{\cS}{\mathcal{S}}
\newcommand{\cT}{\mathcal{T}}
\newcommand{\cU}{\mathcal{U}}
\newcommand{\cV}{\mathcal{V}}
\newcommand{\cX}{\mathcal{X}}
\newcommand{\cY}{\mathcal{Y}}
\newcommand{\cZ}{\mathcal{Z}}
\newcommand{\cW}{\mathcal{W}}

% languages
\newcommand{\sA}{\mathscr{A}}
\newcommand{\sB}{\mathscr{B}}
\newcommand{\sC}{\mathscr{C}}
\newcommand{\sD}{\mathscr{D}}
\newcommand{\sE}{\mathscr{E}}
\newcommand{\sF}{\mathscr{F}}
\newcommand{\sG}{\mathscr{G}}
\newcommand{\sH}{\mathscr{H}}
\newcommand{\sI}{\mathscr{I}}
\newcommand{\sJ}{\mathscr{J}}
\newcommand{\sK}{\mathscr{K}}
\newcommand{\sL}{\mathscr{L}}
\newcommand{\sM}{\mathscr{M}}
\newcommand{\sN}{\mathscr{N}}
\newcommand{\sO}{\mathscr{O}}
\newcommand{\sP}{\mathscr{P}}
\newcommand{\sQ}{\mathscr{Q}}
\newcommand{\sR}{\mathscr{R}}
\newcommand{\sS}{\mathscr{S}}
\newcommand{\sT}{\mathscr{T}}
\newcommand{\sU}{\mathscr{U}}
\newcommand{\sV}{\mathscr{V}}
\newcommand{\sX}{\mathscr{X}}
\newcommand{\sY}{\mathscr{Y}}
\newcommand{\sZ}{\mathscr{Z}}
\newcommand{\sW}{\mathscr{W}}

% structures
\newcommand{\fA}{\mathfrak{A}}
\newcommand{\fB}{\mathfrak{B}}
\newcommand{\fC}{\mathfrak{C}}
\newcommand{\fD}{\mathfrak{D}}
\newcommand{\fE}{\mathfrak{E}}
\newcommand{\fF}{\mathfrak{F}}
\newcommand{\fG}{\mathfrak{G}}
\newcommand{\fH}{\mathfrak{H}}
\newcommand{\fI}{\mathfrak{I}}
\newcommand{\fJ}{\mathfrak{J}}
\newcommand{\fK}{\mathfrak{K}}
\newcommand{\fL}{\mathfrak{L}}
\newcommand{\fM}{\mathfrak{M}}
\newcommand{\fN}{\mathfrak{N}}
\newcommand{\fO}{\mathfrak{O}}
\newcommand{\fP}{\mathfrak{P}}
\newcommand{\fQ}{\mathfrak{Q}}
\newcommand{\fR}{\mathfrak{R}}
\newcommand{\fS}{\mathfrak{S}}
\newcommand{\fT}{\mathfrak{T}}
\newcommand{\fU}{\mathfrak{U}}
\newcommand{\fV}{\mathfrak{V}}
\newcommand{\fX}{\mathfrak{X}}
\newcommand{\fY}{\mathfrak{Y}}
\newcommand{\fZ}{\mathfrak{Z}}
\newcommand{\fW}{\mathfrak{W}}

% objects
\newcommand{\bA}{\mathbf{A}}
\newcommand{\bB}{\mathbf{B}}
\newcommand{\bC}{\mathbf{C}}
\newcommand{\bD}{\mathbf{D}}
\newcommand{\bE}{\mathbf{E}}
\newcommand{\bF}{\mathbf{F}}
\newcommand{\bG}{\mathbf{G}}
\newcommand{\bH}{\mathbf{H}}
\newcommand{\bI}{\mathbf{I}}
\newcommand{\bJ}{\mathbf{J}}
\newcommand{\bK}{\mathbf{K}}
\newcommand{\bL}{\mathbf{L}}
\newcommand{\bM}{\mathbf{M}}
\newcommand{\bN}{\mathbf{N}}
\newcommand{\bO}{\mathbf{O}}
\newcommand{\bP}{\mathbf{P}}
\newcommand{\bQ}{\mathbf{Q}}
\newcommand{\bR}{\mathbf{R}}
\newcommand{\bS}{\mathbf{S}}
\newcommand{\bT}{\mathbf{T}}
\newcommand{\bU}{\mathbf{U}}
\newcommand{\bV}{\mathbf{V}}
\newcommand{\bX}{\mathbf{X}}
\newcommand{\bY}{\mathbf{Y}}
\newcommand{\bZ}{\mathbf{Z}}
\newcommand{\bW}{\mathbf{W}}

% other objects
\newcommand{\tA}{\mathtt{A}}
\newcommand{\tB}{\mathtt{B}}
\newcommand{\tC}{\mathtt{C}}
\newcommand{\tD}{\mathtt{D}}
\newcommand{\tE}{\mathtt{E}}
\newcommand{\tF}{\mathtt{F}}
\newcommand{\tG}{\mathtt{G}}
\newcommand{\tH}{\mathtt{H}}
\newcommand{\tI}{\mathtt{I}}
\newcommand{\tJ}{\mathtt{J}}
\newcommand{\tK}{\mathtt{K}}
\newcommand{\tL}{\mathtt{L}}
\newcommand{\tM}{\mathtt{M}}
\newcommand{\tN}{\mathtt{N}}
\newcommand{\tO}{\mathtt{O}}
\newcommand{\tP}{\mathtt{P}}
\newcommand{\tQ}{\mathtt{Q}}
\newcommand{\tR}{\mathtt{R}}
\newcommand{\tS}{\mathtt{S}}
\newcommand{\tT}{\mathtt{T}}
\newcommand{\tU}{\mathtt{U}}
\newcommand{\tV}{\mathtt{V}}
\newcommand{\tX}{\mathtt{X}}
\newcommand{\tY}{\mathtt{Y}}
\newcommand{\tZ}{\mathtt{Z}}
\newcommand{\tW}{\mathtt{W}}

% shorthands
\newcommand{\s}[1]{\vspace{#1mm}}
\newcommand{\proofskip}{\smallskip\noindent}

\providecommand{\mit}{\mathit}
\renewcommand{\mit}{\mathit}
\newcommand{\mt}{\operatorname}
\newcommand{\mtt}{\mathtt}
\newcommand{\msl}{\mathsl}
\newcommand{\mbf}{\mathbf}
\newcommand{\und}{\underline}

% other objects
\newcommand{\emptystr}{\varepsilon}
\newcommand{\fil}{\blacksquare}
\newcommand{\gap}{\square}
\newcommand{\sep}{\raisebox{0.9pt}{\ensuremath{\mspace{1mu}\wr\mspace{1mu}}}}
\newcommand{\filsep}{\raisebox{1pt}{\text{$\blacktriangleleft$}}}

% vectors
\makeatletter
\def\shortrightarrowfill@{\arrowfill@\relbar\relbar\shortrightarrow}
\newcommand{\ort}{\mathpalette{\overarrow@\shortrightarrowfill@}}
\makeatother
\makeatletter
\def\shortleftarrowfill@{\arrowfill@\relbar\relbar\shortleftarrow}
\newcommand{\olft}{\mathpalette{\overarrow@\shortleftarrowfill@}}
\makeatother
\makeatletter
\def\shortleftrightarrowfill@{\arrowfill@\relbar\relbar\leftrightarrow}
\newcommand{\olftrt}{\mathpalette{\overarrow@\shortleftrightarrowfill@}}
\makeatother
\newcommand{\til}[1]{\widetilde{#1}}

% formulas
\newcommand{\sat}{\vDash}
\newcommand{\nsat}{\nvDash}
\newcommand{\et}{\;\wedge\;}
\newcommand{\vel}{\;\vee\;}
\newcommand{\then}{\;\rightarrow\;}
\newcommand{\Then}{\;\Rightarrow\;}
\renewcommand{\iff}{\;\leftrightarrow\;}
\newcommand{\Iff}{\;\Leftrightarrow\;}
\newcommand{\fa}[1]{\forall{#1}.\:}
\newcommand{\ex}[1]{\exists{#1}.\:}
\newcommand{\exinfinite}[1]{\exists^{\omega}{\:#1}.\:}
\newcommand{\exfinite}[1]{\exists^{<\omega}{\:#1}.\:}
\newcommand{\nex}[1]{\nexists{\:#1}.\:}

% sets, tuples, ...
\newcommand{\set}[1]{{\{ #1 \}}}
\newcommand{\bigset}[1]{{\bigl\{ #1 \bigr\}}}
\newcommand{\biggset}[1]{{\left\{ #1 \right\}}}
\newcommand{\settc}[2]{{\{ #1 \,:\, #2 \}}}
\newcommand{\bigsettc}[2]{{\bigl\{ #1 \,:\,#2 \bigr\}}}
\newcommand{\biggsettc}[2]{{\left\{ #1 \,:\,#2 \right\}}}
\newcommand{\ang}[1]{{\langle #1 \rangle}}
\newcommand{\bigang}[1]{{\bigl\langle #1 \bigr\rangle}}
\newcommand{\biggang}[1]{{\left\langle #1 \right\rangle}}
\newcommand{\intr}[1]{{\llbracket #1 \rrbracket}}
\newcommand{\bigintr}[1]{{\bigl\llbracket #1 \bigr\rrbracket}}
\newcommand{\biggintr}[1]{{\left\llbracket #1 \right\rrbracket}}
\newcommand{\len}[1]{{\lvert #1 \rvert}}
\newcommand{\biglen}[1]{{\bigl\lvert #1 \bigr\rvert}}
\newcommand{\bigglen}[1]{{\left\lvert #1 \right\rvert}}
\newcommand{\dlen}[1]{{\lVert #1 \rVert}}
\newcommand{\bigdlen}[1]{{\bigl\lVert #1 \bigr\rVert}}
\newcommand{\biggdlen}[1]{{\left\lVert #1 \right\rVert}}
\newcommand{\occ}[2]{\len{#2}_{#1}}
\newcommand{\bigocc}[2]{\biglen{#2}_{#1}}
\newcommand{\biggocc}[2]{\bigglen{#2}_{#1}}
\newcommand{\prj}[2]{\mspace{2mu}\downarrow_{#1}\mspace{-5mu}{#2}}

% relations
\newcommand{\trans}[2][]{\raisebox{-1pt}[10pt][0pt]{$\overset{#2}{\underset{^{#1}}{\raisebox{0pt}[3pt][0pt]{$\relbar\mspace{-8mu}\rightarrow$}}}$}}
\newcommand{\ftrans}[2][]{\raisebox{-1pt}[10pt][0pt]{$\overset{#2}{\underset{^{#1}}{\raisebox{0pt}[3pt][0pt]{$\relbar\mspace{-8mu}\circledcirc\mspace{-7.5mu}\rightarrow$}}}$}}
\newcommand{\noftrans}[2][]{\raisebox{-1pt}[10pt][0pt]{$\overset{#2}{\underset{^{#1}}{\raisebox{0pt}[3pt][0pt]{$\relbar\mspace{-8mu}\otimes\mspace{-7.5mu}\rightarrow$}}}$}}
\newcommand{\leads}[1]{\hookrightarrow^{#1}}
\newcommand{\leadsconst}[2]{\hookrightarrow^{#1}_{^{(#2)}}}
\newcommand{\groupsinto}{\trianglelefteq}
\newcommand{\finerthan}{\preceq}
\newcommand{\subgran}{\sqsubseteq}
\newcommand{\border}{\dashv}
\newcommand{\maxborder}{\dashv\!\!\dashv}
\newcommand{\notborder}{\not\dashv}
\newcommand{\notmaxborder}{\not\maxborder}
\newcommand{\dsim}{\approx}
\newcommand{\quotient}[1]{/\mspace{-4mu}#1}
\newcommand{\entails}{\rightvdash}

% operators
\DeclareMathOperator{\lcm}{lcm}
\newcommand{\dom}{{\cD\mit{om}}}
\newcommand{\img}{{\cI\mspace{-2mu}\mit{mg}}}
\newcommand{\fr}{{\cF\mspace{-2mu}\mit{r}}}
\newcommand{\bch}{{\cB\mspace{-2mu}\mit{ch}}}
\newcommand{\infocc}{{\cI\mspace{-2mu}\mit{nf}}}
\newcommand{\unf}{\cU\mspace{-2mu}\mit{nf}}
\newcommand{\nop}{\mathtt{nop}}
\renewcommand{\b}[2]{\bigl(\begin{smallmatrix} #1 \\ #2 \end{smallmatrix}\bigr)}
\newcommand{\bb}[2]{\text{\small $\bigl(\begin{smallmatrix} #1 \\ #2 \end{smallmatrix}\bigr)$}}
\newcommand{\bbb}[2]{\text{\scriptsize $\bigl(\begin{smallmatrix} #1 \\ #2 \end{smallmatrix}\bigr)$}}
\renewcommand{\t}[3]{\biggl(\begin{smallmatrix} #1 \\ #2 \\ #3 \end{smallmatrix}\biggr)}
\renewcommand{\tt}[3]{\text{\small $\biggl(\begin{smallmatrix} #1 \\ #2 \\ #3 \end{smallmatrix}\biggr)$}}
\newcommand{\ttt}[3]{\text{\scriptsize $\biggl(\begin{smallmatrix} #1 \\ #2 \\ #3 \end{smallmatrix}\biggr)$}}

% complexity
\newcommand{\lowerbound}[1]{\Omega(#1)}
\newcommand{\biglowerbound}[1]{\Omega\bigl(#1\bigr)}
\newcommand{\bigglowerbound}[1]{\Omega\left(#1\right)}
\newcommand{\upperbound}[1]{\cO(#1)}
\newcommand{\bigupperbound}[1]{\cO\bigl(#1\bigr)}
\newcommand{\biggupperbound}[1]{\cO\left(#1\right)}
\newcommand{\comparable}[1]{\Theta(#1)}
\newcommand{\bigcomparable}[1]{\Theta\bigl(#1\bigr)}
\newcommand{\biggcomparable}[1]{\Theta\left(#1\right)}
%%% Local Variables: 
%%% mode: latex
%%% TeX-master: "main"
%%% End: 

% packages
\usepackage[english]{babel}
\usepackage{amsmath,amsfonts,amssymb,mathrsfs,latexsym,stmaryrd}
\usepackage{amsthm}
\usepackage{algorithm,algpseudocode}
\usepackage{array}
\usepackage{url}
\usepackage{comment}
\usepackage[newitem,newenum,flushleft,neverdecrease]{paralist}
\usepackage{epsfig}
\usepackage{ifpdf}
%\usepackage[usenames,dvipsnames]{color}
%\usepackage[colorlinks=true,linkcolor=Blue,citecolor=Blue,urlcolor=Magenta,pdfpagelabels,plainpages=false]{hyperref}
%\usepackage[all]{hypcap}

% fix for metapost
\ifpdf
\DeclareGraphicsRule{*}{mps}{*}{}
\else
\DeclareGraphicsRule{*}{eps}{*}{}
\fi

\newtheorem{theorem}{Theorem}[section]
\newtheorem{example}[theorem]{Example}
\newtheorem{conjecture}[theorem]{Conjecture}
\newtheorem{definition}[theorem]{Definition}
\newtheorem{lemma}[theorem]{Lemma}
\newtheorem{claim}[theorem]{Claim}
\newtheorem{corollary}[theorem]{Corollary}
\newtheorem{property}[theorem]{Property}
\newtheorem{proposition}[theorem]{Proposition}
\newtheorem{invariant}[theorem]{Invariant}

\newenvironment{reptheorem}[1]{\begin{trivlist} \item[ {\textbf{Theorem #1}.}] \it}{\end{trivlist}}
\newenvironment{replemma}[1]{\begin{trivlist} \item[ {\textbf{Lemma #1}.}] \it}{\end{trivlist}}
\newenvironment{repproposition}[1]{\begin{trivlist} \item[{\textbf{Proposition #1}.}] \it}{\end{trivlist}}

\newenvironment{runexample}{\begin{example}}{\end{example}}
\newenvironment{runexamplec}[1]{\begin{trivlist} \item[\hskip 15pt {{\sc Example #1 (continued)}.}] \it}{\end{trivlist}}

% environments
\def\labelitemi{\textbullet}
\setlength{\pltopsep}{\smallskipamount}
\setlength{\plitemsep}{\smallskipamount}
\newenvironment{dotlist}{\begin{compactitem}}{\end{compactitem}}
\newenvironment{numlist}{\begin{compactenum}}{\end{compactenum}}
\newenvironment{romlist}{\begin{compactenum}[i)]}{\end{compactenum}}
\newenvironment{condlist}{\begin{compactenum}[{(C}1{)}]}{\end{compactenum}}
\newenvironment{proplist}{\begin{compactenum}[{(P}1{)}]}{\end{compactenum}}
\newenvironment{axiomlist}{\begin{compactenum}[{(A}1{)}]}{\end{compactenum}}
\newcommand{\refeq}[1]{\hyperref[#1]{(\ref*{#1})}}
\newcommand{\refcond}[1]{\hyperref[#1]{C\ref*{#1}}}
\newcommand{\refprop}[1]{\hyperref[#1]{P\ref*{#1}}}
\newcommand{\refaxiom}[1]{\hyperref[#1]{A\ref*{#1}}}
\newcommand{\margincomment}[1]{\marginpar{\small\textit{#1}}}
\newcommand{\op}[1]{\operatorname{#1}}

\providecommand{\qed}{\hfill $\Box$}
\renewcommand{\qed}{\hfill $\Box$}
\newcommand{\eproof}{\qed}

% domains
\newcommand{\bbB}{\mathbb{B}}
\newcommand{\bbD}{\mathbb{D}}
\newcommand{\bbI}{\mathbb{I}}
\newcommand{\bbJ}{\mathbb{J}}
\newcommand{\bbN}{\mathbb{N}}
\newcommand{\bbNp}{\mathbb{N}_{>0}}
\newcommand{\bbNinf}{\mathbb{N}\cup\{\infty\}}
\newcommand{\bbNpinf}{\mathbb{N}_{>0}\cup\{\infty\}}
\newcommand{\bbZ}{\mathbb{Z}}
\newcommand{\bbP}{\mathbb{P}}
\newcommand{\bbQ}{\mathbb{Q}}
\newcommand{\bbQp}{\mathbb{Q}_{>0}}
\newcommand{\bbR}{\mathbb{R}}
\newcommand{\bbRp}{\mathbb{R}_{>0}}
\newcommand{\bbC}{\mathbb{C}}
\newcommand{\bbS}{\mathbb{S}}

% classes
\newcommand{\cA}{\mathcal{A}}
\newcommand{\cB}{\mathcal{B}}
\newcommand{\cC}{\mathcal{C}}
\newcommand{\cD}{\mathcal{D}}
\newcommand{\cE}{\mathcal{E}}
\newcommand{\cF}{\mathcal{F}}
\newcommand{\cG}{\mathcal{G}}
\newcommand{\cH}{\mathcal{H}}
\newcommand{\cI}{\mathcal{I}}
\newcommand{\cJ}{\mathcal{J}}
\newcommand{\cK}{\mathcal{K}}
\newcommand{\cL}{\mathcal{L}}
\newcommand{\cM}{\mathcal{M}}
\newcommand{\cN}{\mathcal{N}}
\newcommand{\cO}{\mathcal{O}}
\newcommand{\cP}{\mathcal{P}}
\newcommand{\cQ}{\mathcal{Q}}
\newcommand{\cR}{\mathcal{R}}
\newcommand{\cS}{\mathcal{S}}
\newcommand{\cT}{\mathcal{T}}
\newcommand{\cU}{\mathcal{U}}
\newcommand{\cV}{\mathcal{V}}
\newcommand{\cX}{\mathcal{X}}
\newcommand{\cY}{\mathcal{Y}}
\newcommand{\cZ}{\mathcal{Z}}
\newcommand{\cW}{\mathcal{W}}

% languages
\newcommand{\sA}{\mathscr{A}}
\newcommand{\sB}{\mathscr{B}}
\newcommand{\sC}{\mathscr{C}}
\newcommand{\sD}{\mathscr{D}}
\newcommand{\sE}{\mathscr{E}}
\newcommand{\sF}{\mathscr{F}}
\newcommand{\sG}{\mathscr{G}}
\newcommand{\sH}{\mathscr{H}}
\newcommand{\sI}{\mathscr{I}}
\newcommand{\sJ}{\mathscr{J}}
\newcommand{\sK}{\mathscr{K}}
\newcommand{\sL}{\mathscr{L}}
\newcommand{\sM}{\mathscr{M}}
\newcommand{\sN}{\mathscr{N}}
\newcommand{\sO}{\mathscr{O}}
\newcommand{\sP}{\mathscr{P}}
\newcommand{\sQ}{\mathscr{Q}}
\newcommand{\sR}{\mathscr{R}}
\newcommand{\sS}{\mathscr{S}}
\newcommand{\sT}{\mathscr{T}}
\newcommand{\sU}{\mathscr{U}}
\newcommand{\sV}{\mathscr{V}}
\newcommand{\sX}{\mathscr{X}}
\newcommand{\sY}{\mathscr{Y}}
\newcommand{\sZ}{\mathscr{Z}}
\newcommand{\sW}{\mathscr{W}}

% structures
\newcommand{\fA}{\mathfrak{A}}
\newcommand{\fB}{\mathfrak{B}}
\newcommand{\fC}{\mathfrak{C}}
\newcommand{\fD}{\mathfrak{D}}
\newcommand{\fE}{\mathfrak{E}}
\newcommand{\fF}{\mathfrak{F}}
\newcommand{\fG}{\mathfrak{G}}
\newcommand{\fH}{\mathfrak{H}}
\newcommand{\fI}{\mathfrak{I}}
\newcommand{\fJ}{\mathfrak{J}}
\newcommand{\fK}{\mathfrak{K}}
\newcommand{\fL}{\mathfrak{L}}
\newcommand{\fM}{\mathfrak{M}}
\newcommand{\fN}{\mathfrak{N}}
\newcommand{\fO}{\mathfrak{O}}
\newcommand{\fP}{\mathfrak{P}}
\newcommand{\fQ}{\mathfrak{Q}}
\newcommand{\fR}{\mathfrak{R}}
\newcommand{\fS}{\mathfrak{S}}
\newcommand{\fT}{\mathfrak{T}}
\newcommand{\fU}{\mathfrak{U}}
\newcommand{\fV}{\mathfrak{V}}
\newcommand{\fX}{\mathfrak{X}}
\newcommand{\fY}{\mathfrak{Y}}
\newcommand{\fZ}{\mathfrak{Z}}
\newcommand{\fW}{\mathfrak{W}}

% objects
\newcommand{\bA}{\mathbf{A}}
\newcommand{\bB}{\mathbf{B}}
\newcommand{\bC}{\mathbf{C}}
\newcommand{\bD}{\mathbf{D}}
\newcommand{\bE}{\mathbf{E}}
\newcommand{\bF}{\mathbf{F}}
\newcommand{\bG}{\mathbf{G}}
\newcommand{\bH}{\mathbf{H}}
\newcommand{\bI}{\mathbf{I}}
\newcommand{\bJ}{\mathbf{J}}
\newcommand{\bK}{\mathbf{K}}
\newcommand{\bL}{\mathbf{L}}
\newcommand{\bM}{\mathbf{M}}
\newcommand{\bN}{\mathbf{N}}
\newcommand{\bO}{\mathbf{O}}
\newcommand{\bP}{\mathbf{P}}
\newcommand{\bQ}{\mathbf{Q}}
\newcommand{\bR}{\mathbf{R}}
\newcommand{\bS}{\mathbf{S}}
\newcommand{\bT}{\mathbf{T}}
\newcommand{\bU}{\mathbf{U}}
\newcommand{\bV}{\mathbf{V}}
\newcommand{\bX}{\mathbf{X}}
\newcommand{\bY}{\mathbf{Y}}
\newcommand{\bZ}{\mathbf{Z}}
\newcommand{\bW}{\mathbf{W}}

% other objects
\newcommand{\tA}{\mathtt{A}}
\newcommand{\tB}{\mathtt{B}}
\newcommand{\tC}{\mathtt{C}}
\newcommand{\tD}{\mathtt{D}}
\newcommand{\tE}{\mathtt{E}}
\newcommand{\tF}{\mathtt{F}}
\newcommand{\tG}{\mathtt{G}}
\newcommand{\tH}{\mathtt{H}}
\newcommand{\tI}{\mathtt{I}}
\newcommand{\tJ}{\mathtt{J}}
\newcommand{\tK}{\mathtt{K}}
\newcommand{\tL}{\mathtt{L}}
\newcommand{\tM}{\mathtt{M}}
\newcommand{\tN}{\mathtt{N}}
\newcommand{\tO}{\mathtt{O}}
\newcommand{\tP}{\mathtt{P}}
\newcommand{\tQ}{\mathtt{Q}}
\newcommand{\tR}{\mathtt{R}}
\newcommand{\tS}{\mathtt{S}}
\newcommand{\tT}{\mathtt{T}}
\newcommand{\tU}{\mathtt{U}}
\newcommand{\tV}{\mathtt{V}}
\newcommand{\tX}{\mathtt{X}}
\newcommand{\tY}{\mathtt{Y}}
\newcommand{\tZ}{\mathtt{Z}}
\newcommand{\tW}{\mathtt{W}}

% shorthands
\newcommand{\s}[1]{\vspace{#1mm}}
\newcommand{\proofskip}{\smallskip\noindent}

\providecommand{\mit}{\mathit}
\renewcommand{\mit}{\mathit}
\newcommand{\mt}{\operatorname}
\newcommand{\mtt}{\mathtt}
\newcommand{\msl}{\mathsl}
\newcommand{\mbf}{\mathbf}
\newcommand{\und}{\underline}

% other objects
\newcommand{\emptystr}{\varepsilon}
\newcommand{\fil}{\blacksquare}
\newcommand{\gap}{\square}
\newcommand{\sep}{\raisebox{0.9pt}{\ensuremath{\mspace{1mu}\wr\mspace{1mu}}}}
\newcommand{\filsep}{\raisebox{1pt}{\text{$\blacktriangleleft$}}}

% vectors
\makeatletter
\def\shortrightarrowfill@{\arrowfill@\relbar\relbar\shortrightarrow}
\newcommand{\ort}{\mathpalette{\overarrow@\shortrightarrowfill@}}
\makeatother
\makeatletter
\def\shortleftarrowfill@{\arrowfill@\relbar\relbar\shortleftarrow}
\newcommand{\olft}{\mathpalette{\overarrow@\shortleftarrowfill@}}
\makeatother
\makeatletter
\def\shortleftrightarrowfill@{\arrowfill@\relbar\relbar\leftrightarrow}
\newcommand{\olftrt}{\mathpalette{\overarrow@\shortleftrightarrowfill@}}
\makeatother
\newcommand{\til}[1]{\widetilde{#1}}

% formulas
\newcommand{\sat}{\vDash}
\newcommand{\nsat}{\nvDash}
\newcommand{\et}{\;\wedge\;}
\newcommand{\vel}{\;\vee\;}
\newcommand{\then}{\;\rightarrow\;}
\newcommand{\Then}{\;\Rightarrow\;}
\renewcommand{\iff}{\;\leftrightarrow\;}
\newcommand{\Iff}{\;\Leftrightarrow\;}
\newcommand{\fa}[1]{\forall{#1}.\:}
\newcommand{\ex}[1]{\exists{#1}.\:}
\newcommand{\exinfinite}[1]{\exists^{\omega}{\:#1}.\:}
\newcommand{\exfinite}[1]{\exists^{<\omega}{\:#1}.\:}
\newcommand{\nex}[1]{\nexists{\:#1}.\:}

% sets, tuples, ...
\newcommand{\set}[1]{{\{ #1 \}}}
\newcommand{\bigset}[1]{{\bigl\{ #1 \bigr\}}}
\newcommand{\biggset}[1]{{\left\{ #1 \right\}}}
\newcommand{\settc}[2]{{\{ #1 \,:\, #2 \}}}
\newcommand{\bigsettc}[2]{{\bigl\{ #1 \,:\,#2 \bigr\}}}
\newcommand{\biggsettc}[2]{{\left\{ #1 \,:\,#2 \right\}}}
\newcommand{\ang}[1]{{\langle #1 \rangle}}
\newcommand{\bigang}[1]{{\bigl\langle #1 \bigr\rangle}}
\newcommand{\biggang}[1]{{\left\langle #1 \right\rangle}}
\newcommand{\intr}[1]{{\llbracket #1 \rrbracket}}
\newcommand{\bigintr}[1]{{\bigl\llbracket #1 \bigr\rrbracket}}
\newcommand{\biggintr}[1]{{\left\llbracket #1 \right\rrbracket}}
\newcommand{\len}[1]{{\lvert #1 \rvert}}
\newcommand{\biglen}[1]{{\bigl\lvert #1 \bigr\rvert}}
\newcommand{\bigglen}[1]{{\left\lvert #1 \right\rvert}}
\newcommand{\dlen}[1]{{\lVert #1 \rVert}}
\newcommand{\bigdlen}[1]{{\bigl\lVert #1 \bigr\rVert}}
\newcommand{\biggdlen}[1]{{\left\lVert #1 \right\rVert}}
\newcommand{\occ}[2]{\len{#2}_{#1}}
\newcommand{\bigocc}[2]{\biglen{#2}_{#1}}
\newcommand{\biggocc}[2]{\bigglen{#2}_{#1}}
\newcommand{\prj}[2]{\mspace{2mu}\downarrow_{#1}\mspace{-5mu}{#2}}

% relations
\newcommand{\trans}[2][]{\raisebox{-1pt}[10pt][0pt]{$\overset{#2}{\underset{^{#1}}{\raisebox{0pt}[3pt][0pt]{$\relbar\mspace{-8mu}\rightarrow$}}}$}}
\newcommand{\ftrans}[2][]{\raisebox{-1pt}[10pt][0pt]{$\overset{#2}{\underset{^{#1}}{\raisebox{0pt}[3pt][0pt]{$\relbar\mspace{-8mu}\circledcirc\mspace{-7.5mu}\rightarrow$}}}$}}
\newcommand{\noftrans}[2][]{\raisebox{-1pt}[10pt][0pt]{$\overset{#2}{\underset{^{#1}}{\raisebox{0pt}[3pt][0pt]{$\relbar\mspace{-8mu}\otimes\mspace{-7.5mu}\rightarrow$}}}$}}
\newcommand{\leads}[1]{\hookrightarrow^{#1}}
\newcommand{\leadsconst}[2]{\hookrightarrow^{#1}_{^{(#2)}}}
\newcommand{\groupsinto}{\trianglelefteq}
\newcommand{\finerthan}{\preceq}
\newcommand{\subgran}{\sqsubseteq}
\newcommand{\border}{\dashv}
\newcommand{\maxborder}{\dashv\!\!\dashv}
\newcommand{\notborder}{\not\dashv}
\newcommand{\notmaxborder}{\not\maxborder}
\newcommand{\dsim}{\approx}
\newcommand{\quotient}[1]{/\mspace{-4mu}#1}
\newcommand{\entails}{\rightvdash}

% operators
\DeclareMathOperator{\lcm}{lcm}
\newcommand{\dom}{{\cD\mit{om}}}
\newcommand{\img}{{\cI\mspace{-2mu}\mit{mg}}}
\newcommand{\fr}{{\cF\mspace{-2mu}\mit{r}}}
\newcommand{\bch}{{\cB\mspace{-2mu}\mit{ch}}}
\newcommand{\infocc}{{\cI\mspace{-2mu}\mit{nf}}}
\newcommand{\unf}{\cU\mspace{-2mu}\mit{nf}}
\newcommand{\nop}{\mathtt{nop}}
\renewcommand{\b}[2]{\bigl(\begin{smallmatrix} #1 \\ #2 \end{smallmatrix}\bigr)}
\newcommand{\bb}[2]{\text{\small $\bigl(\begin{smallmatrix} #1 \\ #2 \end{smallmatrix}\bigr)$}}
\newcommand{\bbb}[2]{\text{\scriptsize $\bigl(\begin{smallmatrix} #1 \\ #2 \end{smallmatrix}\bigr)$}}
\renewcommand{\t}[3]{\biggl(\begin{smallmatrix} #1 \\ #2 \\ #3 \end{smallmatrix}\biggr)}
\renewcommand{\tt}[3]{\text{\small $\biggl(\begin{smallmatrix} #1 \\ #2 \\ #3 \end{smallmatrix}\biggr)$}}
\newcommand{\ttt}[3]{\text{\scriptsize $\biggl(\begin{smallmatrix} #1 \\ #2 \\ #3 \end{smallmatrix}\biggr)$}}

% complexity
\newcommand{\lowerbound}[1]{\Omega(#1)}
\newcommand{\biglowerbound}[1]{\Omega\bigl(#1\bigr)}
\newcommand{\bigglowerbound}[1]{\Omega\left(#1\right)}
\newcommand{\upperbound}[1]{\cO(#1)}
\newcommand{\bigupperbound}[1]{\cO\bigl(#1\bigr)}
\newcommand{\biggupperbound}[1]{\cO\left(#1\right)}
\newcommand{\comparable}[1]{\Theta(#1)}
\newcommand{\bigcomparable}[1]{\Theta\bigl(#1\bigr)}
\newcommand{\biggcomparable}[1]{\Theta\left(#1\right)}
%%% Local Variables: 
%%% mode: latex
%%% TeX-master: "main"
%%% End: 


\def\dotminus{\mathbin{\ooalign{\hss\raise1ex\hbox{.}\hss\cr
			\mathsurround=0pt$-$}}}

\newcommand\loge[1]{\Sigma_{#1}} %Existential logic
\newcommand\logu[1]{\Pi_{#1}} %Universal logic
\newcommand\logex[1]{\Sigma_{#1}\textsc{[FO]}} %Existential extended logic
\newcommand\logux[1]{\Pi_{#1}\textsc{[FO]}} %Universal extended logic
\newcommand\logeh[1]{\Sigma_{#1}\textsc{[FO]-Horn}} %Existential extended Horn logic
\newcommand\loguh[1]{\Pi_{#1}\textsc{[FO]-Horn}} %Universal extended Horn logic
\newcommand\ehorn{\Sigma_2\textsc{-Horn}} 
\newcommand\uhorn{\Pi_1\textsc{-Horn}} 
\newcommand\E[1]{\#\Sigma_{#1}} %Existential
\newcommand\U[1]{\#\Pi_{#1}} %Universal
\newcommand\sfo{\#\fo} %#FO
\newcommand\seso{\text{\sc \#($\eso$)}} %#FO
%\newcommand\sh[1]{\text{\sc \#} #1} %Universal
\newcommand\sh[1]{\##1} %Universal
\newcommand\QE[1]{\eqso(\Sigma_{#1})} %Existential
\newcommand\QU[1]{\eqso(\Pi_{#1})} %Universal
\newcommand\XE[1]{\#\Sigma_{#1}\textsc{[FO]}} %Extended Existential
\newcommand\XU[1]{\#\Pi_{#1}\textsc{[FO]}} %Extended Universal
\newcommand\HE[1]{\#\Sigma_{#1}\textsc{[FO]-Horn}} %Horn Existential
\newcommand\HU[1]{\#\Pi_{#1}\textsc{[FO]-Horn}} %Horn Universal

\def\dhsat{\textsc{DisjHornSAT}}
\def\shdhsat{\textsc{\#DisjHornSAT}}
\def\cpm{\textsc{\#PerfectMatching}}
\def\chsat{\textsc{\#HornSAT}}
\def\cdnf{\textsc{\#DNF}}
\def\ctdnf{\textsc{\#3-DNF}}
\def\ccnf{\textsc{\#CNF}}
\def\ctcnf{\textsc{\#3-CNF}}
\def\ctwcnf{\textsc{\#2-CNF}}
\def\csp{\textsc{\#SimplePath}}
\def\csat{\textsc{\#SAT}}

%\def\A{{\frak A}}
\def\B{{\frak B}}
\def\C{{\cal C}}
\def\F{{\cal F}}
\def\L{{\cal L}}
\def\cG{{\cal G}}
\def\N{\mathbb{N}}
\def\P{\bar{P}}
\def\Q{\bar{Q}}
%\def\R{\bar{R}}
\def\S{\bar{S}}
\def\X{\bar{X}}
\def\Y{\bar{Y}}
\def\Z{\bar{Z}}
%% a - arity of \X / arity of assignments \P to \X
\def\a{\bar{a}}
%% b - arity of predicates in \S
\def\b{\bar{b}}
%% c - arity of auxiliar predicates/variables
\def\c{\bar{c}} %% super auxiliar elements
\def\d{\bar{d}} %% counted elements
\def\e{\bar{e}} %% counted elements
%% f - counting function
%% g - other functions
%% h - other functions
%% i - index
%% j - index
%% k - emergency index / size of tuple
%% l - emergency index / size of tuple
\def\l{\bar{\ell}}
%% m - size of variable tuple
%% n - size of predicate tuple
%% o - not used
\def\p{\bar{p}}
%% q - 
%% r - size of \X / \P
\def\s{\bar{s}}
%% t - size of \S
\def\t{\bar{t}}
\def\u{\bar{u}} %% auxiliary variables
\def\v{\bar{v}} %% auxiliary variables
\def\w{\bar{w}} %% auxiliary variables
\def\x{\bar{x}} %% quantified variables
\def\y{\bar{y}} %% auxiliary variables
\def\z{\bar{z}} %% open variables
\def\ep{\bar{o}}
\def\ga{\bar{p}}





% commands

\newcommand{\cristian}[1]{\todo[inline, color=blue!10]{{\bf Cristian:} #1}}
\newcommand{\marcelo}[1]{\todo[inline, color=red!20]{{\bf Marcelo:} #1}}
\newcommand{\martin}[1]{\todo[inline, color=green!20]{{\bf Martin:} #1}}


%logic
\newcommand{\fo}{{\rm FO}}
\newcommand{\so}{{\rm SO}}
\newcommand{\lfp}{{\rm LFP}}
\newcommand{\lfpop}{{\bf lfp} \,\, }
\newcommand{\alfp}{{\bf alfp} \,\, }
\newcommand{\clfp}[1]{[{\bf lsfp} \, #1]}
\newcommand{\fqfo}{{\rm FQFO}}
\newcommand{\fqso}{{\rm FQSO}}
\newcommand{\pth}{{\bf path} \,\, }
\newcommand{\tc}{{\rm TC}}
\newcommand{\dtc}{{\rm DTC}}
\newcommand{\pfp}{{\rm PFP}}
\newcommand{\eso}{\exists\so}
\newcommand{\first}{\operatorname{first}}
\newcommand{\last}{\operatorname{last}}
\newcommand{\succesor}{\operatorname{succ}}
\newcommand{\partition}{\operatorname{partition}}

\newcommand{\R}{\mathbf{R}}
\newcommand{\T}{\mathcal{T}}
\newcommand{\A}{\mathfrak{A}}
\newcommand{\G}{\mathbf{G}}
\newcommand{\all}{\text{\sc All}}
\newcommand{\allo}{\text{\sc AllOrd}}
\newcommand{\qso}{{\rm QSO}}
\newcommand{\qsoz}{\qso_{\bbZ}}
\newcommand{\rqfo}{{\rm RQFO}}
\newcommand{\tqfo}{{\rm TQFO}}
\newcommand{\tqso}{{\rm TQSO}}
\newcommand{\tqsos}{{\rm TQSO}_{\rm succ}}
\newcommand{\qfo}{{\rm QFO}}
\newcommand{\qfoz}{\qfo_{\bbZ}}
\newcommand{\eqfo}{\Sigma\qfo}
\newcommand{\eqso}{\Sigma\qso}
\newcommand{\eqsoz}{\eqso_{\bbZ}}
\newcommand{\sqso}{\text{\rm Saluja}\qso}
\newcommand{\fv}{\mathbf{FV}}
\newcommand{\sv}{\mathbf{SV}}
\newcommand{\fs}{\mathbf{FS}}
\newcommand{\arity}{{\rm arity}}
\newcommand{\length}{\ell}
\newcommand{\shp}{\text{\sc \#P}}
\newcommand{\ptime}{\text{\sc P}}
\newcommand{\np}{\text{\sc NP}}
\newcommand{\bpp}{\text{\sc BPP}}
\newcommand{\cspp}{\text{\sc SPP}}
\newcommand{\pp}{\text{\sc PP}}
\newcommand{\rp}{\text{\sc RP}}
\newcommand{\pspace}{\text{\sc PSPACE}}
\newcommand{\nlog}{\text{\sc NL}}
\newcommand{\conp}{\text{\sc NP}}
\newcommand{\pe}{\text{\sc \#PE}}
\newcommand{\shl}{\text{\sc \#L}}
\newcommand{\spp}{\text{\sc span-P}}
\newcommand{\gp}{\text{\sc gap-P}}
\newcommand{\optp}{\text{\sc opt-P}}
\newcommand{\fp}{\text{\sc FP}}
\newcommand{\totp}{\text{\sc TotP}}
\newcommand{\shpspace}{\text{\sc \#PSPACE}}
\newcommand{\fpspace}{\text{\sc FPSPACE}}
\newcommand{\nfpspace}{\text{\sc PSPACE(poly)}}
\newcommand{\acc}{\textbf{acc}}

\newcommand{\CC}{\mathscr{C}}
\newcommand{\KK}{\mathscr{K}}
\newcommand{\FF}{\mathscr{F}}
\newcommand{\GG}{\mathscr{G}}
\newcommand{\LL}{\mathscr{L}}
\newcommand{\QQ}{\mathscr{Q}}
\newcommand{\enc}{{\rm enc}}
\newcommand{\str}{\text{\sc Struct}}
\newcommand{\ostr}{\text{\sc OrdStruct}}
\newcommand{\Func}{\text{\sc Func}}
\newcommand{\res}[2]{#1|_{#2}}

%semiring
\newcommand{\nat}{\mathbb{N}}
\newcommand{\natinf}{\mathbb{N}_\infty}
\newcommand{\trop}{\mathbb{N}_{\min,+}}
\newcommand{\integ}{\mathbb{Z}}
\newcommand{\bln}{\mathbb{B}}
\newcommand{\pwset}[1]{2^{#1}}
\newcommand{\true}{\operatorname{true}}
\newcommand{\false}{\operatorname{false}}

\newcommand{\SR}{\bbS}
\newcommand{\add}{+}
\newcommand{\bigadd}{\sum}
\newcommand{\mult}{\cdot}
\newcommand{\bigmult}{\prod}
\newcommand{\adds}{\oplus}
\newcommand{\bigadds}{\bigoplus}
\newcommand{\mults}{\odot}
\newcommand{\bigmults}{\bigodot}
\newcommand{\zero}{\mathbb{0}}
\newcommand{\one}{\mathbb{1}}

%quantitative logic
\newcommand{\QL}{\operatorname{QL}}
\newcommand{\QMSO}{\operatorname{QMSO}}
\newcommand{\Op}{\operatorname{O}}
\newcommand{\sem}[1]{{\llbracket{}{#1}\rrbracket}}
\newcommand{\pa}[1]{\Pi{#1}.\,}
\newcommand{\pas}{\Pi}
\newcommand{\paq}[1]{\Pi{#1}}
\newcommand{\sa}[1]{\Sigma{#1}.\,}
\newcommand{\sas}{\Sigma}
\newcommand{\saq}[1]{\Sigma{#1}}
\newcommand{\fpa}[1]{\overrightarrow{\prod}{#1}.\:}
\newcommand{\lmid}{\;\mid\;}

% equations and quotes skip
\abovedisplayskip=6pt 
\belowdisplayskip=6pt
\newenvironment{myquote}{\begin{quote}\vspace{-0.75mm}}{\end{quote}\vspace{-0.75mm}}

%tikz definition
\tikzset{
	defaultstyle/.style={>=stealth,semithick, auto,font=\small,
		initial text= {},
		initial distance= {3.5mm},
		accepting distance= {3.5mm}},
	accepting/.style=accepting by arrow,
	nstate/.style={circle, semithick,inner sep=1pt, minimum size=4mm}}

%Turing machine
\newcommand{\tma}{\text{\rm accept}}
\newcommand{\tmr}{\text{\rm reject}}
\newcommand{\tmg}{\text{\rm gap}}
\newcommand{\tmt}{\text{\rm total}}

\newcommand{\support}{\text{\rm support}}


\begin{document}

\title{Quantitative Descriptive Complexity}
\author{Marcelo Arenas \and Martin Mu\~noz \and Cristian Riveros}

\maketitle

\section{Preliminaries}
\label{sec:prelim}

%!TEX root = main.tex

%In this section, 
%We introduce here the main terminology used in the paper.

\subsection{Second-order logic, LFP and PFP}
A relational signature $\R$ (or just signature) is a finite set $\{R_1, \ldots, R_k\}$, where each $R_i$ ($1 \leq i \leq k$) is a relation name with an associated arity greater than 0, which is denoted by $\arity(R_i)$. A finite structure over $\R$ (or just finite $\R$-structure) is a tuple $\A = \langle A, R_1^\A, \ldots, R_k^\A \rangle$ such that $A$ is a finite set and $R_i^\A \subseteq A^{\arity(R_i)}$ for every $i \in \{1, \ldots, k\}$. Further, an $\R$-structure $\A$ is said to be ordered if $<$ is a binary predicate name in $\R$ and $<^\A$ is a linear order on $A$.
We denote by $\ostr[\R]$ the class of all finite ordered $\R$-structures. 
In this paper we only consider finite ordered structures, so we will usually omit the words finite and ordered when referring to them.

From now on, assume given disjoint infinite sets $\fv$ and $\sv$ of first-order variables and second-order variables, respectively. Notice that every variable in $\sv$ has an associated arity, which is denoted by $\arity(X)$. Then given a  signature $\R$, the set of second-order logic formulae ($\so$-formulae) over $\R$ is given by the following grammar:
\begin{align*}\ 
	\varphi \ &:= \ x = y \ \mid \ R(\bar u) \ \mid \ \top  \ \mid\  
	X(\bar v)  \ \mid
	\neg \varphi \ \mid\ 
	(\varphi \vee \varphi) \ \mid\ 
	\ex{x} \varphi \ \mid\ 
	\ex{X} \varphi
 \end{align*}
where $x,y \in \fv$, $R \in \R$, $\bar u$ is a tuple of (not necessarily distinct) variables from $\fv$ whose length is $\arity(R)$, $\top$ is a reserved symbol to represent a tautology, $X \in \sv$, $\bar v$ is a tuple of (not necessarily distinct) variables from $\fv$ whose length is $\arity(X)$, and $x \in \fv$. 

%\marcelo{Tenemos una definicion muy detallada de $\fo$ y $\so$, no es necesario para esta conferencia. Por otro lado seria bueno mencionar las definiciones de $\Sigma_i$ y $\Pi_i$, aunque son estandar es bueno decir que estamos considerando restricciones de $\fo$.}

%\martin{Decir que $\LL$ incluye tautología y decir que Saluja no lo menciona}

We assume that the reader is familiar with the semantics of $\so$, so we only introduce here some notation that will be used in this paper. 
%To define the semantics of $\so$, we need to introduce some terminology. 
Given a signature $\R$ and an $\R$-structure $\A$ with domain $A$, a first-order assignment $v$ for $\A$ is a total function from $\fv$ to $A$, while a second-order assignment $V$ for $\A$ is a total function with domain $\sv$ that maps each $X \in \sv$ to a subset of $A^{\arity(X)}$. Moreover, given a first-order assignment $v$ for $\A$, $x \in \fv$ and $a \in A$, we denote by $v[a/x]$ a first-order assignment such that $v[a/x](x) = a$ and $v[a/x](y) = v(y)$ for every $y \in \fv$ distinct from $x$. Similarly, given a second-order assignment $V$ for $\A$, $X \in \sv$ and $B  \subseteq A^{\arity(X)}$, we denote by $V[B/X]$ a second-order assignment such that $V[B/X](X) = B$ and $V[B/X](Y) = V(Y)$ for every $Y \in \sv$ distinct from $X$. We use notation $(\A, v, V) \models \varphi$ to indicate that structure $\A$ satisfies $\varphi$ under $v$ and $V$.
% In particular, we have that $(\A, v, V) \models \top$.
%\martin{En vez de decir "for every $y\in \fv$ distinct from $x$" podría ser "for every $y\in\fv\setminus\{x\}$", y lo mismo con $Y\in\sv$.}

In this paper, we consider several fragments or extensions of $\so$ like first-order logic~($\fo$), least fixed point logic (LFP) and partial fixed point logic (PFP) \cite{L04}. Moreover, for every $i \in \N$, we consider the fragment $\Sigma_i$ (resp., $\Pi_i$) of $\fo$, which is the set of $\fo$-formulae of the form 
$\exists \bar x_1 \forall \bar x_2 \cdots \exists \bar x_{i-1} \forall \bar x_{i} \, \psi$ (resp., 
$\forall \bar x_1 \exists \bar x_2 \cdots \forall \bar x_{i-1} \exists \bar x_{i} \, \psi$) if $i$ is even, and of the form
$\exists \bar x_1 \forall \bar x_2 \cdots \forall \bar x_{i-1} \exists \bar x_{i} \, \psi$ (resp., 
$\forall \bar x_1 \exists \bar x_2 \cdots \exists \bar x_{i-1} \forall \bar x_{i} \, \psi$) if $i$ is odd, where $\psi$ is a quantifier-free formula. Finally, we say that a fragment $\L_1$ is contained in a fragment $\L_2$, denoted by $\L_1 \subseteq \L_2$, if for every formula $\varphi$ in $\L_1$, there exists a formula $\psi$ in $\L_2$ such that $\varphi$ is logically equivalent to $\psi$.  Besides, we say that $\L_1$ is properly contained in $\L_2$, denoted by $\L_1 \subsetneq \L_2$, if $\L_1 \subseteq \L_2$ and $\L_2 \not\subseteq \L_1$.


%Assume that $\varphi$ is an $\so$-formula over a signature $\R$. Then given an ordered finite $\R$-structure $\A$ with domain $A$, a first-order assignment $v$ for $\A$ and a second-order assignment $V$ for $\A$, we say that $(\A, v, V)$ satisfies $\varphi$, denoted by $(\A, v, V) \models \varphi$, if: (1) $\varphi$ is the formula $R(x_1, \ldots, x_\ell)$ and $(v(x_1), \ldots, v(x_\ell)) \in R^\A$; (2) $\varphi$ is the formula $X(x_1, \ldots, x_m)$ and $(v(x_1), \ldots, v(x_m)) \in V(X)$; (3) $\varphi$ is the formula $\neg \psi$ and $(\A, v, V)$ does not satisfy $\psi$; (4) $\varphi$ is the formula $(\varphi_1 \vee \varphi_2)$, and $(\A, v, V) \models \varphi_1$ or $(\A, v, V) \models \varphi_2$; (5) $\varphi$ is the formula $\exists x \, \psi$ and there exists $a \in A$ such that $(\A, v[a/x], V) \models \psi$; or (6) $\varphi$ is the formula $\exists X \, \psi$ and there exists $B \subseteq A^{\arity(X)}$ such that $(\A, v, V[B/X]) \models \psi$.  As usual, we consider the propositional operators $\wedge$, $\rightarrow$, and $\leftrightarrow$ that can be obtained from $\vee$ and $\neg$. 
%Moreover, we use the abbreviations $x \not \leq y$ and $x \notin X$ for the negation of the atoms $\leq$ and $\in$. 
%Finally, we consider standard abbreviations of formulae that can be defined in $\so$-logic (actually, $\fo$-logic) like $\first(x) := \fa{x} y \leq x$ and $\last(x) := \fa{x} x \leq y$ to denote the first and last element of the linear order $\leq$, respectively, and $\succesor(x,y) := x \leq y \wedge y \not\leq x \wedge \fa{z} ( z \leq x \vee y \leq z)$ to denote the successor relation.

%\cristian{Agregar que la igualdad $=$ esta en nuestra lógica.}

%\cristian{Agregar acá que usamos las abreviaciones de $<$ o $\subset$.}

%\marcelo{Agregar las definiciones de LFP y PFP}

%\cristian{Explicar acá que significa el containment de lógicas booleans, esto es, la notación $\U{1} \subseteq \LL$.}

\subsection{Counting complexity classes}

We consider several counting complexity classes in this paper, some of them are recalled here~(see \cite{F97,hemaspaandra2013complexity}).
%We define the following function complexity classes: 
$\fp$ is the class of functions $f : \Sigma^* \to \N$ computable in polynomial time, while $\fpspace$ is the class of functions $f : \Sigma^* \to \N$ computable in polynomial space. 
%Moreover, $\nfpspace$ is the class of functions computable in polynomial space with output length bounded by a polynomial. 
Given a nondeterministic Turing Machine (NTM) $M$, let $\tma_M(x)$ be the number of accepting runs of $M$ with input $x$. Then $\shp$ is the class of functions $f$ for which there exists a polynomial-time NTM $M$ such that $f(x) = \tma_M(x)$ for every input $x$, while $\shl$ is the class of functions $f$ for which there exists a logarithmic-space NTM $M$ such that $f(x) = \tma_M(x)$ for every input $x$.  Given an NTM $M$ with output tape, let $\tmo_M(x)$ be the number of distinct outputs of $M$ with input $x$ (notice that  $M$ produces an output if it halts in an accepting state). Then $\spp$ is the  class of functions $f$ for which there exists a polynomial-time NTM $M$ such that $f(x) = \tmo_M(x)$ for every input $x$. Notice that $\shp \subseteq \spp$, and this inclusion is believed to be strict. 

%as the class of functions that on an input are equal to the number of different outputs of an $\np$ transducer on that input. . $\shpspace$ is defined analogously to $\shp$ but in polynomial space. $\nfpspace$ is the class of single-exponentially bounded functions computable in polynomial space. 

%\marcelo{La notacion $\tma_M(x)$ se usa en el paper, hay que definirla aqui. Donde vamos a usar la clase $\gp$? Para definir esta clase tambien necesitamos la notacion $\tmr_M(x)$, podriamos definirla aqui. }

%If $\KK$ is a class of finite structures and $f$ is a function from a signature $\R$ to $\bbD$, we denote by $\res{f}{\KK}$ the restriction of $f$ to $\KK$, that is, a function such that: the domain of $\res{f}{\KK}$ is $\{ \enc(\A) \mid \A \in \str[\R] \cap \KK\}$, and 
%$\res{f}{\KK}(\enc(\A)) = f(\enc(\A))$ for every $\A \in \str[\R] \cap \KK$.


\section{Quantitative second order logic}
\label{sec:logics}

Consider $\bbN$ as the set of non-negative integers. Then given a relational signature $\R$, the set of quantitative second-order logic formulas (or just $\qso$-formulas) over $\R$ is given by the following grammar:
\begin{eqnarray}
\label{eq-def-qso}
  \alpha & := & \varphi \ \mid \ s \ \mid \ (\alpha \add \alpha) \ \mid\ (\alpha \mult \alpha) \ \mid \ \sa{x} \alpha \ \mid \ \pa{x} \alpha \ \mid \ \sa{X} \alpha \ \mid \ \pa{X} \alpha 
\end{eqnarray}
where $\varphi$ is an $\so$-formula over $\R$, $s \in \bbN$, $x \in \fv$ and $X \in \sv$.

Let $\R$ be a relational signature, $\A$ be a finite $\R$-structure with domain $A$, $v$ a first-order assignment for $\A$ and $V$ a second-order assignment for $\A$. Then the \emph{evaluation} of a $\qso$-formula $\alpha$ over $(\A, v, V)$ is defined as a function $\sem{\alpha}$ that on input $(\A, v, V)$ returns a non-negative integer in $\bbN$. Formally, the function $\sem{\alpha}$ is recursively defined as follows:
$$
\renewcommand{\arraystretch}{1.7}
\begin{array}{rcl} 
\sem{\varphi}(\A, v, V) & = & 
\begin{cases}
1 & \mbox{if } (\A, v, V) \models \varphi \\
0 & \mbox{otherwise}
\end{cases}\\
\sem{s}(\A, v, V) & = & s \\
\sem{\alpha_1 \add \alpha_2}(\A, v, V) & = & \sem{\alpha_1}(\A, v, V) + \sem{\alpha_2}(\A, v, V)\\
\sem{\alpha_1 \mult \alpha_2}(\A, v, V) & = & \sem{\alpha_1}(\A, v, V) \cdot \sem{\alpha_2}(\A, v, V)\\ 
\sem{\sa{x} \alpha}(\A, v, V) & = & \displaystyle \sum_{a \in A} \sem{\alpha}(\A,v[a/x],V)\\
\sem{\pa{x} \alpha}(\A, v, V) & = & \displaystyle \prod_{a \in A} \sem{\alpha}(\A,v[a/x],V)\\
\sem{\sa{X} \alpha}(\A, v, V) & = & \displaystyle \sum_{B \subseteq A^{\arity(X)}} \sem{\alpha}(\A, v, V[B/X])\\
\sem{\pa{X} \alpha}(\A, v, V) & = & \displaystyle \prod_{B \subseteq A^{\arity(X)}} \sem{\alpha}(\A, v, V[B/X])
\end{array}
$$
A formula in $\qso$ can mention the usual quantifiers in $\so$ (that is, $\exists x$ and $\exists X$) and the quantifiers that make use of addition and multiplication (that is, $\Sigma x$, $\Pi x$, $\Sigma X$ and $\Pi X$), which are called {\em quantitative quantifiers} . A $\qso$-formula $\alpha$ is said to be a \emph{sentence} if it does not have any free variable, that is, every variable in $\alpha$ is under the scope of a usual quantifier or a quantitative quantifier. It is important to notice that if $\alpha$ is a $\qso$-sentence over a relational signature $\R$, then for every finite $\R$-structure $\A$, first-order assignments $v_1$, $v_2$ for $\A$ and second-order assignments $V_1$, $V_2$ for $\A$, it holds that:
\begin{eqnarray*}
\sem{\alpha}(\A, v_1, V_1) & = & \sem{\alpha}(\A, v_2, V_2).
\end{eqnarray*}
Thus, in such a case we use the term $\sem{\alpha}(\A)$ to denote $\sem{\alpha}(\A, v, V)$, for some arbitrary first-order assignment $v$ for $\A$ and some arbitrary second-order assignment $V$ for $\A$. 

In this paper, we consider several fragments of $\qso$, which are obtained by restricting the syntax of the formula $\varphi$ in \eqref{eq-def-qso} or the use of the quantitative quantifiers. Let $\qfo$ be the fragment of $\qso$ obtained by only allowing the quantitative quantifiers $\Sigma x$, $\Pi x$ in \eqref{eq-def-qso}. Let $\eqso$ be a fragment of $\qso$ defined as $\qfo$ but also allowing the quantitative quantifier $\Sigma X$. Moreover, assuming that $\LL$ is a fragment of $\so$, let $\qso(\LL)$ be the fragment of $\qso$ obtained by restricting $\varphi$ in \eqref{eq-def-qso} to be a formula in $\LL$, and let $\eqso(\LL)$ be the fragment of $\eqso$ obtained by imposing the same restriction. In particular, in this paper we consider the following fragments $\LL$: $\fo$, which is obtained by disallowing the use of the second-order quantifier ($\exists X$ or $\forall X$) in the formula $\varphi$ in \eqref{eq-def-qso}, and $\eso$, which is obtained by allowing $\varphi$ in \eqref{eq-def-qso} to be a formula of the form $\exists X_1 \cdots \exists X_k \, \psi$, where $\psi$ is an $\fo$-formula. 


\begin{theorem}
	Let $\alpha$ be a $\qso$-formula in some fragment of $\qso$. There is a $\qso$ formula $\alpha'$ in the same fragment such that $\alpha'$ does not have any subformula of the form $(\beta_1 \cdot \beta_2)$ and $\sem{\alpha} = \sem{\alpha'}$.
\end{theorem}
\begin{proof}
	Let $\{+, \cdot, \Sigma, \Pi\}$ be the set of {\it algebraic operators} in $\qso$. We will prove by induction over the number of algebraic operators in a $\qso$-formula $\alpha$ that there exists some $\qso$-formula $\alpha'$ that respects the stated condition. Let $\text{op}(\alpha)$ be the number of algebraic operators in $\alpha$. The case $\text{op}(\alpha) = 0$ is trivial. Let $n\in\nat$ be such that for each $\qso$-formula $\beta$ such that $\text{op}(\beta) < n$ there is some $\beta'$ that respects the condition. Let $\alpha$ be a $\qso$-formula such that $\text{op}(\alpha) = n$. Note that if $\alpha$ is of the form $\alpha = (\beta_1 + \beta_2)$, $\alpha = \Sigma x \beta$, $\alpha = \Pi x \beta$, $\alpha = \Sigma X \beta$ or $\alpha = \Pi X \beta$ the proof follows directly. Therefore we assume that $\alpha = (\beta_1 \cdot \beta_2)$. We separate the proof in two main cases:
	\begin{enumerate}
		\item At least one of $\beta_1, \beta_2$ is not of the form $\Pi x \gamma$, $\Pi X \gamma$, or $\varphi$, where $\varphi$ is an $\so$-formula. Without loss of generality, let $\beta_1$ be such one.
		\begin{enumerate}
			\item $\beta_1 = s \in \nat$. Then we use $\alpha' = \beta_2 + \cdots + \beta_2$, where $\beta_2$ is repeated $s$ times, and if $s = 0$, take $\alpha' = \perp$. Since each of the formulas paired by $+$ has less than $n$ algebraic operators, the statement holds.
			\item $\beta_1 = (\gamma_1 + \gamma_2)$. Then, we use $\alpha' = (\gamma_1\cdot\beta_2 + \gamma_2\cdot\beta_2)$ and each of the formulas paired by $+$ have less than $n$ algebraic operators so the statement holds.
			\item $\beta_1 = \Sigma x \gamma$. Then, we use $\alpha' = \Sigma x (\gamma\cdot\beta_2)$ and as in the previous case the statement holds.
			\item $\beta_1 = \Sigma X \gamma$, which we treat exactly as the previous one.
		\end{enumerate}
		\item Both of $\beta_1$ and $\beta_2$ are of the form $\Pi x \gamma$, $\Pi X \gamma$, or $\varphi$, where $\varphi$ is an $\so$-formula. This case also separates in further cases. We define $(\varphi \mapsto \alpha) := (\neg\varphi + (\varphi\cdot\alpha))$ for which $\sem{(\varphi \mapsto \alpha)}(\A,v,V)$ equals $\sem{\alpha}(\A,v,V)$ when $(\A,v,V)\models\varphi$ and 1 otherwise.
		\begin{enumerate}
			\item $\beta_1 = \Sigma x \gamma_1(x)$ and $\beta_2 = \Sigma y \gamma_2(y)$. Then we use $\alpha' = \Sigma x(\gamma_1(x)\cdot\gamma_2(x))$, and the statement holds.
			\item $\beta_1 = \Sigma X \gamma_1(X)$ and $\beta_2 = \Sigma Y \gamma_2(Y)$. [something about arities].
			\item $\beta_1 = \Sigma X \gamma_1(X)$ and $\beta_2 = \Sigma x \gamma_2(x)$ (w.l.o.g.). [use singleton predicates].
			\item $\beta_1 = \Sigma x \gamma$ and $\beta_2 = \varphi$, which is an $\so$-formula (w.l.o.g.). Then, we use $\alpha' = \Sigma x (\gamma \cdot \varphi)$, and the statement holds.
			\item $\beta_1 = \Sigma X \gamma$ and $\beta_2 = \varphi$, which is an $\so$-formula (w.l.o.g.). We treat this case exactly as the previous one.
			\item $\beta_1 = \varphi_1$ and $\beta_2 = \varphi_2$ both $\so$-formulas. Then, we use $\alpha' = (\varphi_1 \wedge \varphi_2)$ and the statement holds.
		\end{enumerate}
	\end{enumerate}
\end{proof}

We define an operator which extends the Least Fixed Point logic to counting. Recall that a fixed point operator is defined by a formula $\varphi(x_1,\ldots,x_k,R)$ which is positive on $R$ where $R$ is a predicate of arity $k$. For a structure $\A$ with domain $A$, the operator $T_{\varphi}:2^{A^k} \to 2^{A^k}$ is defined as $T_{\varphi}(X) = \{(a_1,\ldots,a_k)\mid (\A,X)\models \varphi(a_1,\ldots,a_k,R) \}$, for each $X\subseteq A^k$. Let $T_0 = \emptyset$ and $T_{i+1} = T_{\varphi}(T_i)$ for each $i \in \nat$. Note that there exists $n\in \nat$ such that $T_{n+1} = T_n$. Semantically, $[\lfpop\,\varphi(x_1,\ldots,x_k,R)]$ is defined such that for each $(a_1,\ldots,a_k)\in A^k$, it holds that $\A\models[\lfpop\,\varphi(x_1,\ldots,x_k,R)](a_1,\ldots,a_k)$ if and only if $\A\models T_n(a_1,\ldots,a_k)$.

We extend $\qfo$ with the operator $[\textbf{alfp}\,\varphi(y_1,\ldots,y_k,R)\mid\alpha(x_1,\ldots,x_{\ell},R,\pi)](x_1,\ldots,x_{\ell})$. It is defined by a FO-formula $\varphi$ and a QFO-formula $\alpha$ which mentions a placeholder function $\pi$ with domain $\fv^{\ell}$. Let $\{T_i\}_{i\in\nat}$ and $n\in\nat$ be defined from $\varphi$ as it was for the $\lfp$ operator.

For each $\qfo$ formulas $\alpha, \beta, \gamma$, where $\beta$ and $\gamma$ have $\ell$ and $m$ first-order open variables respectively, $u_1,\ldots,u_{\ell} \in \fv$, and $v_1,\ldots,v_m \in \{u_1,\ldots,u_{\ell}\}$, we define $\alpha\mid_{\beta(u_1,\ldots,u_{\ell})\to\gamma(v_1,\ldots,v_{m})}$ as $\alpha$ where every instance of the subformula $\beta(u_1,\ldots,u_{\ell})$ is replaced by $\gamma(v_1,\ldots,v_{m})$. Moreover, for each $a\in A$, suppose that $a$ is the $p$-th element in the order $<^{\A}$. Then if $a$ is the first element, we define $\varphi_a(x) = \forall y(x < y \vee x = y)$, and if it is not we define:
$$
\varphi_a(x) = \exists x_1 \cdots \exists x_{p-1}[\bigwedge_{1\leq i,j < p}x_i\neq x_j \wedge \bigwedge_{i = 1}^p x_i < x \wedge \forall y(y < x \to \bigvee_{i = 1}^p y = x_i)].
$$

To formally characterize the semantics of $[\textbf{alfp}\,\varphi(y_1,\ldots,y_k,R)\mid\alpha(x_1,\ldots,x_{\ell},R,\pi)](x_1,\ldots,x_{\ell})$. %. 
We name $\T = \{T_0,\ldots,T_n\}$ and we define the operator $Z:\T \to \{f\mid f:A^{\ell}\to\nat\}$ as follows. For each $T_i \in \T$, we define a $\qfo$ formula $\beta_i$ as follows. If $i = 0$, then $\beta_i(u_1,\ldots,u_{\ell}) = 0$. If $i \geq 1$, then let $f = Z[T_{i-1}]$ and
$$
\beta_i(u_1,\ldots,u_{\ell}) = \bigplus_{(a_1,\ldots,a_{\ell})\in A^{\ell}} \varphi_{a_1}(u_1)\cdot\varphi_{a_2}(u_2)\,\cdots\,\varphi_{a_{\ell}}(u_{\ell})\cdot f(a_1,\ldots,a_{\ell}).
$$
Then, for each $T_i \in \T$, let $V$ be a second-order assignment for $\A$ that assigns $T_i$ to $R$, let $f: A^{\ell}\to\nat$ be such that for each $(a_1,\ldots,a_{\ell})\in A^{\ell}$ it holds $f(a_1,\ldots,a_{\ell}) = \sem{\alpha\mid_{\pi(u_1,\ldots,u_{\ell})\to \beta_i(u_1,\ldots,u_{\ell})}}(\A,v,V)$, where $v$ is a first-order assignment for $\A$ that satisfies $a_i = v(x_i)$ for each open $x_i$ in $\alpha$. We define $Z[T_i] = f$.

For a given first order asignment $v$, let $a_i = v(x_i)$, let $f = Z[T_n]$, and the operator is evaluated as:
$$
[\textbf{alfp}\,\varphi(y_1,\ldots,y_k,R)\mid\alpha(x_1,\ldots,x_{\ell},R,\pi)](\A,v) = f(a_1,\ldots,a_{\ell}).
$$
As an example, we will show a formula that counts the number of paths of size $n$ of a structure $\A$ with a binary relation $E$. First we define $\varphi(x,R)$:
$$
\varphi(x,R) = \forall y(x < y \vee x = y) \vee \exists z(R(z) \wedge \varphi_{succ}(z,x)),x
$$
where $\varphi_{succ}(x,y)$ is a formula that is satisfied by pairs $(x,y)$ that are consecutive in the order $<$. That is, $\varphi_{succ}(x,y) = x < y \wedge \forall z((x < z \wedge z < y) \to (x = z \vee z = y) )$. Now we define $\alpha(x,y,R,\pi)$ as:
$$
\alpha(x,y,R,\pi) = \neg \exists z(R(z)) + \Sigma z[E(x,z)\cdot \pi(z,y)].
$$
Then, the formula $[\alfp_{\varphi(x,R)} \alpha(x,y,R,\pi)](x,y)$ when evaluated on $(\A,a,b)$ will count the number of paths of size $n$ from $a$ to $b$.

We also define the less powerful operator $\pth$ as follows. Given a positive formula $\varphi(x,R)$ and a formula $\psi(x,y)$ we define $[\pth_{\varphi(x,R)} \psi(x,y)](x,y)$ as a formula that when evaluated on $(\A, a, b)$ induces a graph over $\A$ with the binary relation determined by $\psi(x,y)$ and counts the number of paths from $a$ to $b$ of a size determined by the $\varphi(x,R)$. Formally, we define:
$$
\sem{[\pth_{\varphi(x,R)} \psi(x,y)](x,y)} = \sem{[\alfp_{\varphi(x,R)} \alpha_{\psi}(x,y,R,\pi)](x,y)},
$$
where $\alpha_{\psi}(x,y,R,\pi) = \neg \exists z(R(z)) + \Sigma z[\psi(x,z)\cdot \pi(z,y)]$.


\begin{theorem}
	Given a positive $\fo$ formula $\varphi(\bar{x},R)$ and a $\qfo$ formula $\alpha(\bar{x})$, there exists a $\qso$ formula $\beta(\bar{x})$ such that $\sem{[\alfp_{\varphi(\bar{x},R)} \alpha_{\psi}(\bar{x},R)](\bar{x})} = \sem{\beta(\bar{x})}$.
\end{theorem}
\begin{proof}
	
\end{proof}


\section{Capturing function complexity classes}
\label{sec:cap-comp}


Assume that $\alpha$ is a sentence in $\qso$. Then slightly abusing notation, we also use $\alpha$ to denote a function from $\R$ to $\bbN$ such that for every $\A \in \str[\R] $:
\begin{eqnarray*}
\alpha(\enc(\A)) & = & \sem{\alpha}(\A).
\end{eqnarray*}
\begin{definition}
Let $\FF$ a fragment of $\qso$, $\CC$ a function complexity class over $\bbN$ and $\KK$ a class of finite structures. Then $\FF$ {\em captures} $\CC$ over $\KK$ if:
\begin{itemize}
\item for every $\alpha$ in $\FF$, there exists $f \in \CC$ such that $\res{\alpha}{\KK} = \res{f}{\KK}$; and

\item for every $f \in \CC$, there exists $\alpha$ in $\FF$ such that $\res{f}{\KK} = \res{\alpha}{\KK}$.
\end{itemize}
\end{definition}
In the previous definition, if $\KK$ is the class of all finite structures, then we say that $\FF$ captures $\CC$ over the class of all structures, and if $\KK$is the class of all ordered finite structures, then we say that $\FF$ captures $\CC$ over the class of ordered structures. 

\cristian{Aca hay que definir esto sobre estructuras finitas directamente. No veo el sentido de tener estructuras infinitas.}

\begin{theorem} \label{captfp}
	$\qfo(\lfp)$ captures $\fp$ over the class of ordered structures.
\end{theorem}
\begin{proof}
	For the first condition, let $\alpha\in\qfo(\lfp)$, $\A\in\ostr$, $v$ and $V$ a first and second order assignment for $\A$, respectively. To evaluate $\sem{\alpha}(\A,v,V)$, we replace each first order sum and first order product by their corresponding expansion. This is, $\Sigma x \beta(x)$ is replaced by $(\beta(a_1)+\cdots+\beta(a_n))$, where $A = \{a_1,\ldots,a_n\}$, and $\Pi x \beta(x)$ is replaced by $(\beta(a_1)\cdot\,\cdots\,\cdot\beta(a_n))$. Then we replace each logic sub-formula $\varphi$ in $\alpha$ by their evaluated value, 0 or 1. Since each of this formulas is in $\lfp$, this can be done in polynomial time. The resulting formula is an arithmetic expression, which can be evaluated recursively in polynomial time.
	
	For the second condition, let $f\in \fp$. Let $k\in\nat$ be such that $\vert f(\A) \vert = \vert A \vert^k$ for each $\A\in\ostr[\R]$. Let $\Phi(x_0,\ldots,x_{k-1})$ be a $\lfp$ formula such that $(\A,a_0,\ldots,a_{k-1})\models\Phi$ if and only if in the string $y = f(\A)$ there is 1 in the $(\vert A \vert^{k-1}m_{k-1} + \cdots + \vert A \vert^2 m_2 + \vert A \vert m_1 + m_0)$-th position from right to left, and where $a_i$ is the (0-indexed) $m_i$-th element of $A$. We use
	$$
	\alpha = \Sigma x_0 \cdots \Sigma x_{k-1} \Phi(x_0,\ldots,x_{k-1})\cdot\varphi_{k-1}(x_{k-1})\cdot\,\cdots\,\cdot\varphi_0(x_0).
	$$
	where $\varphi_i(x) = \Pi y[(y < x)\mapsto\Pi z_1\cdots\Pi z_i\,2]$ for $i > 0$ and $\varphi_0(x) = \Pi y[(y < x)\mapsto 2]$. Note that if $a$ is the  $m$-th element in $A$, then $\sem{\varphi_i(x)}(\A,a) = 2^{\vert A \vert^i m}$. Therefore, for each $(a_0,\ldots,a_{k-1})\in A^k$, we have that $\sem{\Phi(x_0,\ldots,x_{k-1})\cdot\varphi_{k-1}(x_{k-1})\,\cdots\,\varphi_0(x_0)}(\A,a_0,\ldots,a_{k-1}) = 2^{(\vert A \vert^{k-1}m_{k-1} + \cdots + \vert A \vert^2 m_2 + \vert A \vert m_1 + m_0)}$ if $(\A,a_0,\ldots,a_{k-1})\models\Phi$ and 0 otherwise, and adding these values gives $f(\A)$. 	
\end{proof}

\begin{theorem}
	$\qfo(\pfp)$ captures $\nfpspace$ over the class of ordered structures.
\end{theorem}
\begin{proof}
	For the first condition, we use the exact same procedure as in Theorem \ref{captfp}, noting that evaluating each of the $\pfp$ formulas can be done in polynomial space, and that the result is also polynomial on the size of the input.
	
	For the second condition, note that for each function $f\in \nfpspace$, $\vert f(\A) \vert$ is polynomial on $\vert A \vert$. Thus, there is a $\pfp$ formula $\Phi$ that models the string $f(\A)$ by the bit. The rest of the proof is analogous to Theorem \ref{captfp}.
\end{proof}

\begin{theorem}
	$\qfo(\tc)$ captures $\shl$ over the class of ordered structures.
\end{theorem}
\begin{proof}
	For the first condition, let $\alpha\in\qfo(\tc)$, let $\A\in\ostr$ $v$ and $V$ first and second order assignments of $\A$ respectively. We evaluate $\sem{\alpha}(\A,v,V)$ recursively. A logic formula $\varphi$ is evaluated...?
	
	For the second condition, note that for each function $f\in \shl$, $\vert f(\A) \vert$ is polynomial on $\vert A \vert$. Thus, there is a $\tc$ formula $\Phi$ that models the string $f(\A)$ by the bit. The rest of the proof is analogous to Theorem \ref{captfp}.
\end{proof}

\begin{theorem}
	$\rqfo(\fo)$ captures $\fp$ over the class of ordered structures.
\end{theorem}

\begin{theorem}
	$\tqfo(\fo)$ captures $\shl$ over the class of ordered structures.
\end{theorem}
\begin{proof}
	For the second condition, let $f \in \shl$. We will address the case where $\R$ contains only one binary predicate $E$, and the rest of the cases can be deduced from this. Let $M$ be a non-deterministic logspace machine such that $f(\A) = \acc_M(\A)$ for each $\A \in \ostr[R]$. Suppose ${\cal Q} = \{q_1,\ldots,q_{\ell}\}$ is the set of states of $M$, where $q_1$ is the initial state, and $q_{\ell}$ is only final state of $M$. Let $n = \vert A \vert$ and let $w = \enc(\A) \in \{0,1\}^{n^2}$. We assume that $M$ with input $w$ uses space $s_M(w) < c\cdot\log(n)$ and furthermore, $s_M(w) < n-2$. We notate $M(w)$ as the graph of configurations of $M$ running on input $w$.
	
	We represent configurations with a tuple of fixed size $k$. The formula $\varphi(\bar{x},\bar{y})$ describes a procedure that given a configuration generates a possible next configuration. The formula $\varphi_I(\bar{x})$ models that $\bar{x}$ is the initial configuration of $M(w)$. The formula $\varphi_F(\bar{x})$ models that $\bar{x}$ is an accepting (final) configuration of $M(w)$. The formula we construct is:
	$$
	\alpha = \sa{\bar{x}}\sa{\bar{y}}([\pth \varphi(\bar{u},\bar{v})](\bar{x},\bar{y})\cdot \varphi_I(\bar{x})\cdot\varphi_F(\bar{y})).
	$$
	
	To illustrate our idea, we will show a greatly simplified example. Consider a machine $M$ that works in exactly $\log_2(n)$ space and only allows 0 or 1 in the working tape. Consider an input $\A$ of size 16 (that is, $A = \{0,\ldots,9,A,\ldots,F\}$). Let some configuration $s$ have 0011 in the working tape, the head in the input tape is in position 26, and the head in the input tape is in position 2 (we consider 0-indexed positions). Also, $Q = \{q_1,\ldots,q_5\}$ and the current state is $q_3$.
	
	As a first approach, we will use a 9-tuple $\bar{a} = (a_1,\ldots,a_9)$ to represent $s$. That is, $(a_1,a_2) = (1,A)$ represent the position of the head in the input tape, $a_3 = 2$ represents the position of the head in the working tape, $a_4 = C$ (1100b in base 16) represents the content of the working tape, and $(a_5,\ldots,a_9) = (0,0,1,0,0)$ represents the current state. Then $\bar{a} = (1,A,2,C,0,0,1,0,0)$ will represent $s$.
	
	The problem that arises from this representation, is that to describe a transition in $M$ we need to read an arbitrary character in the working tape, and therefore we have to obtain the $a_3$-th bit in $a_4$. Furthermore, to get the following configuration, we need compute $a_4$ with the $a_3$-th bit flipped. This is generally not possible to describe with an $FO$ formula. Consider the following procedure that receives $x = a_4$ and $i = a_3$.
	
	\begin{algorithm} \label{switch1to0}
	\caption{If the $i$-th bit in $x$ is 1 replace it by 0 and return the result}
	\begin{algorithmic}
		\State $u \gets x,\; j \gets i$ \Comment{Get the $i$-th bit on $x$ and store it in $u$}
		\While{$j > 0$}
		\State $v \gets 0$
		\While{$u > 1$}
		\State $u \gets u-2,\; v \gets v+1$
		\EndWhile
		\State $u\gets v,\; j \gets j-1$
		\EndWhile
		\While{$u > 1$}
		\State $u \gets u-2$
		\EndWhile
		\State $\textbf{assert } u = 1$ \Comment{If $u \neq 1$ simply stop}	
		\State $y \gets 1$ \Comment{Compute $2^i$ and store it in $y$}
		\While{$i > 0$}
		\State $z \gets 0$
		\While{$y > 0$}
		\State $z \gets z+2,\; y \gets y-1$
		\EndWhile
		\State $i \gets i-1,\; y \gets z$
		\EndWhile
		\While{$y > 0$} \Comment{Substract $y$ from $x$}
		\State $x \gets x-1,\; y \gets y-1$
		\EndWhile
		\State \Return $x$.
	\end{algorithmic}
	\end{algorithm}	
	Each of the instructions can be expressed with $\fo$, so our strategy is to use the $\pth$ operator to simulate the algorithm and then we can describe a transition using the processed value of $a_4$. Note that we need to describe three more procedures to simulate the transitions for $0\to 0$, $0 \to 1$ and $1\to 1$.
	
	We will now describe how to simulate both the procedure and the transition. A procedure tuple $\bar{p} = (a_1,\ldots,a_{3+c+\ell},b_1,b_2,c_1,c_2,c_3,d_1,\ldots,d_{5c+2})$ represents the current configuration of $M(w)$ in $a_1,\ldots,a_{2+c+\ell}$, the values that will be read and written in the working tape in $b_1,b_2$, the instruction pointer in $c_1,c_2,c_3$ and the values stored in memory in $d_1,\ldots,d_{10c+2}$. In detail:
	\begin{enumerate}
		\item $a_1,a_2$ and $a_3$ represent the position of the head in the input tape and the working tape, respectively, $a_4,\ldots,a_{3+c}$ represent the content of the working tape and $a_{4+c},\ldots,a_{3+c+\ell}$ represent the current state in the current configuration that is being processed.
		\item $b_1$ and $b_2$ are equal to the value that is being read in the working tape and the value that will be written in the working tape respectively.
		\item $c_1,c_2,c_2$ represent the instruction pointer in the procedure. Only 8 different instructions are needed in the simulation.
		\item Each value in memory of $x,y,z,u,v$ need $c$ elements to represent them and $i,j$ need only one. We map $(d_1\ldots,d_{c}) \to x$, $(d_{c+1}\ldots,d_{2c}) \to y$,
		$(d_{2c+1}\ldots,d_{3c}) \to z$, $(d_{3c+1}\ldots,d_{4c}) \to u$,
		$(d_{4c+1}\ldots,d_{5c}) \to v$, $d_{5c+1} \to i$ and $d_{5c+2}\to j$.
	\end{enumerate}
	For each transition $\delta \subseteq Q \times \{0,1\} \times \{0,1\} \times Q \times \{-1,=,+1\} \times \{0,1\} \times \{-1,=,+1\}$ we define a formula $\varphi_{\delta}(\bar{x},\bar{s},\bar{w},\bar{u},\bar{y},\bar{t},\bar{z},\bar{v})$, where $\bar{x} = (x_1,\ldots,x_{3+c+\ell})$, $\bar{s} = (s_1,s_2)$, $\bar{w} = (w_1,w_2,w_3)$, $\bar{u} = (u_1,\ldots,u_{5c+2})$, $\bar{y} = (y_1,\ldots,y_{3+c+\ell})$, $\bar{t} = (t_1,t_2)$, $\bar{z} = (z_1,z_2,z_3)$ and $\bar{v} = (v_1,\ldots,v_{5c+2})$ that describes the procedure to compute the values in the configuration and the transition itself. We will describe the formula part by part. Suppose $\delta = (q_i,a,1,q_j,op_1,0,op_2)$, so we have to simulate Algorithm \ref{switch1to0}.
	
	We start from instruction 0, which means that the procedure has not started yet and every value in the tuple is 0 except for the configuration values. It also initializes all the values in the tuple to 0 except for $x,u,i,j$.
	\begin{multline*}
	\varphi^{0,1}_{\delta}(\bar{x},\bar{s},\bar{w},\bar{u},\bar{y},\bar{t},\bar{z},\bar{v}) = \varphi_{0,0}(s_1,s_2)\wedge\varphi^b_0(\bar{w}) \wedge \varphi_{1,0}(t_1,t_2) \wedge \varphi^b_1(\bar{z})\, \wedge \\ 
	\bigwedge_{i = 1}^c v_i = x_{3+i} \wedge \bigwedge_{i = c+1}^{2c} \varphi_0(v_i) \wedge \bigwedge_{i = 2c+1}^{3c} \varphi_0(v_i) \wedge \bigwedge_{i = 1}^c v_{3c+i} = x_{3+i} \wedge \bigwedge_{i = 4c+1}^{5c} \varphi_0(v_i) \wedge v_{5c+1} = x_3 \wedge v_{5c+2} = x_3.
	\end{multline*}
	Instruction 1 which checks whether the value of $j$ ($d_{5c+2}$ in the tuple) is more than 0 or not, and then proceeds to instruction 2 or 3 on each case.
	\begin{multline*}
	\varphi^{1,2}_{\delta}(\bar{x},\bar{s},\bar{w},\bar{u},\bar{y},\bar{t},\bar{z},\bar{v}) = 
	\varphi_{1,0}(s_1,s_2) \wedge \varphi^b_1(\bar{w}) \wedge \neg \varphi_0(u_{5c+2}) \wedge \varphi_{1,0}(t_1,t_2) \wedge \varphi^b_2(\bar{z}) \wedge \bigwedge_{i = 1}^{5c+2} u_i = v_i, \\
	\varphi^{1,3}_{\delta}(\bar{x},\bar{s},\bar{w},\bar{u},\bar{y},\bar{t},\bar{z},\bar{v}) = 
	\varphi_{1,0}(s_1,s_2) \wedge \varphi^b_1(\bar{w}) \wedge \varphi_0(u_{5c+2}) \wedge \varphi_{1,0}(t_1,t_2) \wedge \varphi^b_3(\bar{z}) \wedge \bigwedge_{i = 1}^{5c+2} u_i = v_i.
	\end{multline*}
	Instruction 2 checks the value of $u$ ($d_{3c+1},\ldots,d_{4c}$ in the tuple). If it is $> 1$ then it substracts 2 from $u$ and adds 1 to $v$ ($d_{4c+1},\ldots,d_{5c}$ in the tuple), then repeats instruction 2. If it is equal to 0 or 1, then moves the value of $v$ to $u$, substracts 1 from $j$ and goes back to instruction 1.
	\begin{multline*}
	\varphi^{2,2}_{\delta}(\bar{x},\bar{s},\bar{w},\bar{u},\bar{y},\bar{t},\bar{z},\bar{v}) = 
	\varphi_{1,0}(s_1,s_2) \wedge \varphi^b_2(\bar{w}) \wedge \neg \varphi^m_0(u_{3c+1},\ldots,u_{4c}) \wedge \neg \varphi^m_1(u_{3c+1},\ldots,u_{4c}) \wedge \varphi_{1,0}(t_1,t_2) \wedge 	\varphi^b_2(\bar{z})\, \wedge \\
	\bigwedge_{i = 1}^c u_i = v_i \wedge
	\bigwedge_{i = c+1}^{2c} u_i = v_i \wedge
	\bigwedge_{i = 2c+1}^{3c} u_i = v_i \wedge
	\varphi^m_{-2}(u_{3c+1},\ldots,u_{4c},v_{3c+1},\ldots,v_{4c})\, \wedge \\
	\varphi^m_{+1}(u_{4c+1},\ldots,u_{5c},v_{4c+1},\ldots,v_{5c}) \wedge u_{5c+1} = v_{5c+1} \wedge u_{5c+2} = v_{5c+2}. \\
	\varphi^{2,1}_{\delta}(\bar{x},\bar{s},\bar{w},\bar{u},\bar{y},\bar{t},\bar{z},\bar{v}) = \varphi_{1,0}(s_1,s_2) \wedge \varphi^b_2(\bar{w}) \wedge ( \varphi^m_0(u_{3c+1},\ldots,u_{4c}) \vee \varphi^m_1(u_{3c+1},\ldots,u_{4c})) \wedge \varphi_{1,0}(t_1,t_2) \wedge 	\varphi^b_1(\bar{z})\, \wedge \\
	\bigwedge_{i = 1}^c u_i = v_i \wedge
	\bigwedge_{i = c+1}^{2c} u_i = v_i \wedge
	\bigwedge_{i = 2c+1}^{3c} u_i = v_i \wedge
	\bigwedge_{i = 3c+1}^{4c} u_{c+i} = v_i \wedge
	\varphi^m_0(v_{4c+1},\ldots,v_{5c}) \wedge
	u_{5c+1} = v_{5c+1} \wedge \varphi_{-1}(u_{5c+2},v_{5c+2}).	
	\end{multline*}
	Instruction 3 calculates the value of $u \mod 2$, that is, it repeats instruction 3 until the value of $u$ is equal to 0 or 1. On each iteration, it substracts 2 from $u$. Moreover, if the value of $u$ at the end of the iterations is not 1 then there is no step defined.
	\begin{multline*}
	\varphi^{3,3}_{\delta}(\bar{x},\bar{s},\bar{w},\bar{u},\bar{y},\bar{t},\bar{z},\bar{v}) = \varphi_{1,0}(s_1,s_2) \wedge \varphi^b_3(\bar{w}) \wedge \neg \varphi^m_0(u_{3c+1},\ldots,u_{4c}) \wedge \neg \varphi^m_1(u_{3c+1},\ldots,u_{4c}) \wedge \varphi_{1,0}(t_1,t_2) \wedge 	\varphi^b_3(\bar{z})\, \wedge \\
	\bigwedge_{i = 1}^c u_i = v_i \wedge
	\bigwedge_{i = c+1}^{2c} u_i = v_i \wedge
	\bigwedge_{i = 2c+1}^{3c} u_i = v_i \wedge
	\varphi^m_{-2}(u_{3c+1},\ldots,u_{4c},v_{3c+1},\ldots,v_{4c}) \wedge
	\bigwedge_{i = 4c+1}^{5c} u_i = v_i \wedge
	u_{5c+1} = v_{5c+1} \wedge u_{5c+2} = v_{5c+2}, \\
	\varphi^{3,3}_{\delta}(\bar{x},\bar{s},\bar{w},\bar{u},\bar{y},\bar{t},\bar{z},\bar{v}) = 
	\varphi_{1,0}(s_1,s_2) \wedge \varphi^b_3(\bar{w}) \wedge \varphi^m_1(u_{3c+1},\ldots,u_{4c}) \wedge \varphi_{1,0}(t_1,t_2) \wedge 	\varphi^b_4(\bar{z})\, \wedge \\
	\bigwedge_{i = 1}^c u_i = v_i \wedge
	\varphi^m_1(v_{c+1},\ldots,v_{2c}) \wedge
	\bigwedge_{i = 2c+1}^{3c} u_i = v_i \wedge
	\bigwedge_{i = 3c+1}^{4c} u_i = v_i \wedge
	\bigwedge_{i = 4c+1}^{5c} u_i = v_i \wedge
	u_{5c+1} = v_{5c+1} \wedge u_{5c+2} = v_{5c+2}
	\end{multline*}
	Instruction 4 checks the value of $i$ ($d_{5c+1}$ in the tuple.) If it is not 0 then goes to instruction 5 and if is 0 then goes to instruction 6. Moreover it initializes the value of $z$ ($d_{2c+1},\ldots,d_{3c}$ in the tuple) to 0 (which was 0 all along.)
	\begin{multline*}
	\varphi^{4,5}_{\delta}(\bar{x},\bar{s},\bar{w},\bar{u},\bar{y},\bar{t},\bar{z},\bar{v}) = \varphi_{1,0}(s_1,s_2) \wedge \varphi^b_4(\bar{w}) \wedge \neg \varphi_0(u_{5c+1}) \wedge \varphi_{1,0}(t_1,t_2) \wedge 	\varphi^b_5(\bar{z}) \wedge \bigwedge_{i = 1}^{5c+2} u_i = v_i, \\
	\varphi^{4,6}_{\delta}(\bar{x},\bar{s},\bar{w},\bar{u},\bar{y},\bar{t},\bar{z},\bar{v}) = \varphi_{1,0}(s_1,s_2) \wedge \varphi^b_4(\bar{w}) \wedge \varphi_0(u_{5c+1}) \wedge \varphi_{1,0}(t_1,t_2) \wedge 	\varphi^b_6(\bar{z}) \wedge \bigwedge_{i = 1}^{5c+2} u_i = v_i.
	\end{multline*}
	Instruction 5 checks the value of $y$ ($d_{c+1},\ldots,d_{2c}$ in the tuple.) If it is more than 0 then it adds 2 to $z$ and substracts 1 from $y$, then repeats instruction 2. If it is not, then copies the value of $z$ to $y$ and subtracts 1 from $i$ and returns to instruction 4.
	\begin{multline*}
	\varphi^{5,5}_{\delta}(\bar{x},\bar{s},\bar{w},\bar{u},\bar{y},\bar{t},\bar{z},\bar{v}) = \varphi_{1,0}(s_1,s_2) \wedge \varphi^b_5(\bar{w}) \wedge \neg \varphi^m_0(u_{c+1},\ldots,u_{2c}) \wedge \varphi_{1,0}(t_1,t_2) \wedge \varphi^b_5(\bar{z}) \, \wedge \\
	\bigwedge_{i = 1}^c u_i = v_i \wedge
	\varphi^m_{-1}(u_{c+1},\ldots,u_{2c},v_{c+1},\ldots,v_{2c}) \wedge
	\varphi^m_{+2}(u_{2c+1},\ldots,u_{3c},v_{2c+1},\ldots,v_{3c})\, \wedge \\
	\bigwedge_{i = 3c+1}^{4c} u_i = v_i \wedge
	\bigwedge_{i = 4c+1}^{5c} u_i = v_i \wedge
	u_{5c+1} = v_{5c+1} \wedge u_{5c+2} = v_{5c+2}, \\
	\varphi^{5,4}_{\delta}(\bar{x},\bar{s},\bar{w},\bar{u},\bar{y},\bar{t},\bar{z},\bar{v}) = \varphi_{1,0}(s_1,s_2) \wedge \varphi^b_5(\bar{w}) \wedge \varphi^m_0(u_{c+1},\ldots,u_{2c}) \wedge \varphi_{1,0}(t_1,t_2) \wedge \varphi^b_4(\bar{z}) \, \wedge \\
	\bigwedge_{i = 1}^c u_i = v_i \wedge
	\bigwedge_{i = c+1}^{2c} u_{c+i} = v_i \wedge
	\varphi^m_0(v_{2c+1},\ldots,v_{3c}) \wedge
	\bigwedge_{i = 3c+1}^{4c} u_i = v_i \wedge
	\bigwedge_{i = 4c+1}^{5c} u_i = v_i \wedge
	\varphi_{-1}(u_{5c+1},v_{5c+1}) \wedge u_{5c+2} = v_{5c+2}
	\end{multline*}
	Instruction 6 checks the value of $y$. If it is more than 0, then subtracts 1 from $x$ and $y$ and repeats instruction 6. If it is not, then goes to instruction 7.
	\begin{multline*}
	\varphi^{6,6}_{\delta}(\bar{x},\bar{s},\bar{w},\bar{u},\bar{y},\bar{t},\bar{z},\bar{v}) = \varphi_{1,0}(s_1,s_2) \wedge \varphi^b_6(\bar{w}) \wedge \neg \varphi^m_0(u_{c+1},\ldots,u_{2c}) \wedge \varphi_{1,0}(t_1,t_2) \wedge \varphi^b_6(\bar{z}) \, \wedge \\
	\varphi^m_{-1}(u_1,\ldots,u_c,v_i,\ldots,v_c) \wedge \varphi^m_{-1}(u_{c+1},\ldots,u_{2c},v_{c+1},\ldots,v_{2c}) \wedge \bigwedge_{i = 3c+1}^{5c+2} u_i = v_i]\,\vee \\
	\varphi^{6,7}_{\delta}(\bar{x},\bar{s},\bar{w},\bar{u},\bar{y},\bar{t},\bar{z},\bar{v}) = \varphi_{1,0}(s_1,s_2) \wedge \varphi^b_6(\bar{w}) \wedge \varphi^m_0(u_{c+1},\ldots,u_{2c}) \wedge \varphi_{1,0}(t_1,t_2) \wedge \varphi^b_7(\bar{z}) \wedge \bigwedge_{i = 1}^{5c+2} u_i = v_i.
	\end{multline*}
	Instruction 7 stores the value of $x$ after the corresponding bit has been switched. Then we can define $\varphi_{\delta}$ which also simulates the actual transition. If $u$ equals 1, then copy what is stored in $x$ to $a_4,\ldots,a_{3+c}$, go from state $q_i$ to state $q_j$, and move the heads to their corresponding positions.
	\begin{multline*}
	\varphi_{\delta}(\bar{x},\bar{s},\bar{w},\bar{u},\bar{y},\bar{t},\bar{z},\bar{v}) = 	[\bigwedge_{i = 1}^{3+c+\ell} x_i = y_i \wedge (\varphi^{0,1}(\bar{x},\bar{s},\bar{w},\bar{u},\bar{y},\bar{t},\bar{z},\bar{v}) \vee \varphi^{1,2}(\bar{x},\bar{s},\bar{w},\bar{u},\bar{y},\bar{t},\bar{z},\bar{v})\, \vee \\ \varphi^{1,3}(\bar{x},\bar{s},\bar{w},\bar{u},\bar{y},\bar{t},\bar{z},\bar{v}) \vee \varphi^{2,2}(\bar{x},\bar{s},\bar{w},\bar{u},\bar{y},\bar{t},\bar{z},\bar{v}) \vee \varphi^{2,1}(\bar{x},\bar{s},\bar{w},\bar{u},\bar{y},\bar{t},\bar{z},\bar{v}) \vee \varphi^{3,3}(\bar{x},\bar{s},\bar{w},\bar{u},\bar{y},\bar{t},\bar{z},\bar{v}) \vee \\ \varphi^{3,4}(\bar{x},\bar{s},\bar{w},\bar{u},\bar{y},\bar{t},\bar{z},\bar{v}) \vee \varphi^{4,5}(\bar{x},\bar{s},\bar{w},\bar{u},\bar{y},\bar{t},\bar{z},\bar{v}) \vee \varphi^{4,6}(\bar{x},\bar{s},\bar{w},\bar{u},\bar{y},\bar{t},\bar{z},\bar{v}) \vee \varphi^{5,5}(\bar{x},\bar{s},\bar{w},\bar{u},\bar{y},\bar{t},\bar{z},\bar{v}) \vee \\ \varphi^{5,6}(\bar{x},\bar{s},\bar{w},\bar{u},\bar{y},\bar{t},\bar{z},\bar{v}) \vee \varphi^{6,6}(\bar{x},\bar{s},\bar{w},\bar{u},\bar{y},\bar{t},\bar{z},\bar{v}) \vee \varphi^{6,7}(\bar{x},\bar{s},\bar{w},\bar{u},\bar{y},\bar{t},\bar{z},\bar{v}))] \, \vee \\
	[\varphi_{1,0}(s_1,s_2) \wedge \varphi^b_7(\bar{w}) \wedge \varphi^m_1(u_{3c+1},\ldots,u_{4c}) \wedge \varphi_{0,0}(t_1,t_2) \wedge \varphi^b_0(\bar{z}) \wedge \bigwedge_{i = 1}^{5c+2} u_i = v_i \,\wedge \\
	\varphi^2_{op_1}(x_1,x_2,y_1,y_2) \wedge \varphi_{op_2}(x_3,y_3) \wedge \bigwedge_{i = 1}^c u_i = x_{3+i} \wedge \varphi^q_i(x_{4+c},\ldots,x_{3+c+\ell}) \wedge \varphi^q_j(y_{4+c},\ldots,y_{3+c+\ell})].
	\end{multline*}
	Note that we also need to specify that the program we are following is Algorithm \ref{switch1to0} so we store $1,0$ in $b_1,b_2$ all along the procedure. We now describe the three other algorithms that compute the switches from $0\to 0$, $0\to 1$ and $1\to 1$.
\end{proof}

\begin{theorem}
	$\tqso(\fo)$ captures $\shp$ over the class of ordered structures.
\end{theorem}

\begin{theorem}
	($\qfo(\lfp)$ or $\qfo(\fo)$) with algebraic least fixed point operator captures $\totp$ over the class of ordered structures.
\end{theorem}
	

\begin{theorem}
$\eqso(\fo)$ captures $\shp$ over the class of ordered structures.
\end{theorem}

\begin{proposition}
$\eqso(\fo)$ does not capture $\shp$ over the class of all structures.
\end{proposition}

%\begin{theorem}
%$\eqso(\integ,\fo)$ captures $\gp$ over the class of ordered structures.
%\end{theorem}


\begin{theorem}
$\eqso(\eso)$ captures $\spp$ over the class of ordered structures.
\end{theorem}


\bibliographystyle{abbrv}
\bibliography{biblio}


\end{document}
