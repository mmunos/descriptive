
\subsection{Logics}

A relational signature $\R$ is a finite set $\{R_1, \ldots, R_k\}$, where each $R_i$ ($1 \leq i \leq k$) is a relation name with an associate arity greater than 0, which is denoted by $\arity(R_i)$. A finite structure over $\R$ (or just finite $\R$-structure) is a tuple $\A = \langle A, R_1^\A, \ldots, R_k^\A \rangle$ such that $A$ is a finite set and $R_i^\A \subseteq A^{\arity(R_i)}$ for every $i \in \{1, \ldots, k\}$. A finite $\R$-structure $\A$ is said to be ordered if $<$ is a binary predicate name in $\R$ and $<^\A$ is a linear order on $A$. Let $\str[\R]$ be the class of all finite $\R$-structures and $\ostr[\R]$ be the class of all ordered finite $\R$-structures. 

From now on, assume given disjoint infinite sets $\fv$ and $\sv$ of first-order variables and second-order variables, respectively. Notice that every variable in $\sv$ has an associated arity, which is denoted by $\arity(X)$. Then given a relational signature $\R$, the set of second-order logic formulas ($\so$-formulas) over $\R$ is given by the following grammar:
\begin{eqnarray*}\ 
\varphi &:=& R(\bar u) \ \mid\  
X(\bar v)  \ \mid\ 
\neg \varphi \ \mid\ 
(\varphi \vee \varphi) \ \mid\ 
\exists x \, \varphi \ \mid\ 
\exists X \, \varphi
\end{eqnarray*}
where $R \in \R$, $\bar u$ is a tuple of (non-necessarily distinct) variables from $\fv$ whose length is $\arity(R)$, $X \in \sv$, $\bar v$ is a tuple of (non-necessarily distinct) variables from $\fv$ whose length is $\arity(X)$, and $x \in \fv$. 

To define the semantics of $\so$, we need to introduce some terminology. Given a relational signature $\R$ and a finite $\R$-structure $\A$ with domain $A$, a first-order variable-assignment $v$ for $\A$ is a total function from $\fv$ to $A$, while a second-order variable-assignment $V$ for $\A$ is a total function with domain $\sv$ that maps each $X \in \sv$ to a subset of $A^{\arity(X)}$. Moreover, given a first-order variable-assignment $v$ for $\A$, $x \in \fv$ and $a \in A$, we denote by $v[a/x]$ a first-order variable-assignment such that $v[a/x](x) = a$ and $v[a/x](y) = v(y)$ for every $y \in \fv$ distinct from $x$. Similarly, given a second-order variable-assignment $V$ for $\A$, $X \in \sv$ and $B  \subseteq A^{\arity(X)}$, we denote by $V[B/X]$ a second-order variable-assignment such that $V[B/X](X) = B$ and $V[B/X](Y) = V(Y)$ for every $Y \in \sv$ distinct from $X$. 

Assume that $\varphi$ is an $\so$-formula over a signature $\R$. Then given a finite $\R$-structure $\A$ with domain $A$, a first-order variable-assignment $v$ for $\A$ and a second-order variable-assignment $V$ for $\A$, we say that $(\A, v, V)$ satisfies $\varphi$, denoted by $(\A, v, V) \models \varphi$, if: (1) $\varphi$ is the formula $R(x_1, \ldots, x_\ell)$ and $(v(x_1), \ldots, v(x_\ell)) \in R^\A$; (2) $\varphi$ is the formula $X(x_1, \ldots, x_m)$ and $(v(x_1), \ldots, v(x_m)) \in V(X)$; (3) $\varphi$ is the formula $\neg \psi$ and it not the case that $(\A, v, V) \models \psi$; (4) $\varphi$ is the formula $(\varphi_1 \vee \varphi_2)$, and $(\A, v, V) \models \varphi_1$ or $(\A, v, V) \models \varphi_2$; (5) $\varphi$ is the formula $\exists x \, \psi$ and there exists $a \in A$ such that $(\A, v[a/x], V) \models \psi$; or (6) $\varphi$ is the formula $\exists X \, \psi$ and there exists $B \subseteq A^{\arity(X)}$ such that $(\A, v, V[B/X]) \models \psi$.  As usual, we consider the propositional operators $\wedge$, $\rightarrow$, and $\leftrightarrow$ that can be obtained from $\vee$ and $\neg$. 
%Moreover, we use the abbreviations $x \not \leq y$ and $x \notin X$ for the negation of the atoms $\leq$ and $\in$. 
%Finally, we consider standard abbreviations of formulas that can be defined in $\so$-logic (actually, $\fo$-logic) like $\first(x) := \fa{x} y \leq x$ and $\last(x) := \fa{x} x \leq y$ to denote the first and last element of the linear order $\leq$, respectively, and $\succesor(x,y) := x \leq y \wedge y \not\leq x \wedge \fa{z} ( z \leq x \vee y \leq z)$ to denote the successor relation.

\subsection{Semirings}

A semiring is a structure $\SR = (S, \adds, \mults, \zero, \one)$ where $(S, \adds, \zero)$ is a commutative monoid, $(S, \mults, \one)$ is a monoid, multiplication ($\mults$) distributes over addition ($\adds$), and $\zero \mults s = s \mults \zero = \zero$ for every $s \in S$. If the multiplication is commutative, we say that $\SR$ is commutative. In this paper, we always assume that $\SR$ is  commutative.
For the sake of simplification, we usually denote the set of elements $S$ with the name of the semiring $\SR$.
% and we say sentences like ``for all $s \in \SR$''.

In general, our definitions are given with respect to a generic semiring.
However, in some cases we consider some particular semiring like, for example, the semiring of \emph{natural numbers} $\nat = (\mathbb{N}, +, \cdot, 0, 1)$, the semiring of \emph{integers} $\integ = (\mathbb{Z}, +, \cdot, 0, 1)$, the \emph{boolean} semiring $\bln = (\{\false, \true\}, \vee, \wedge, \false, \true)$, or the \emph{tropical} semiring $\trop = (\nat \cup \{\infty\}, \min, +, \infty, 0)$.
For the last semiring, the $\min$-operator is defined by the linear order $0 < 1 < 2 < \cdots < \infty$ and the $+$-operator is just the addition over natural number with the exception that $v + \infty = \infty + v = \infty$ for all $v \in \nat \cup \{\infty\}$.

\subsection{Function complexity classes}
Assume that $\R = \{R_1, \ldots, R_k\}$ is a relational signature and $\A$ is a finite $\R$-structure with a domain $A$ containing $n$ elements, and assume that  $<$ is a linear order on $A$, say $a_1 < a_2 < \ldots < a_n$. For every $i \in \{1, \ldots, k\}$, define the encoding of $R_i^\A$, denoted by $\enc(R_i^\A)$, as the following binary string. Assume that $\ell = \arity(R_i)$ and consider an enumeration of the $\ell$-tuples over $A$ in the lexicographic order induced by $<$ (that is, $(a_1, \ldots, a_1, a_1)$, $(a_1, \ldots, a_1, a_2)$, $\ldots$, $(a_n, \ldots, a_n, a_{n-1})$, $(a_n, \ldots, a_n, a_n)$). Then let $\enc(R_i^\A)$ be a binary string of length $n^\ell$ such that the $i$-th bit of $\enc(R_i^\A)$ is 1 if the $i$-th tuple in the previous enumeration belongs to $R_i^\A$, and 0 otherwise. Moreover, define the encoding of $\A$, denoted by $\enc(\A)$, as the following binary string~\cite{L04}:
\begin{eqnarray*}
\enc(\A) & = & 0^n \, 1 \, \enc(R_1^\A) \, \cdots \, \enc(R_k^\A).
\end{eqnarray*}
Given a semiring $\SR = (S, \adds, \mults, \zero, \one)$ and a relational signature $\R$, a function $f$ is said to be from $\R$ to $\SR$ if $f$ is a total function from $\{\enc(\A) \mid \A \in \str[\R]\}$ to $S$. Moreover, a set $\CC$ of functions is said to be a {\em function complexity class over $\SR$} if every $f \in \CC$ is a function from some relational signature $\R$ to $\SR$.

If $\KK$ is a class of finite structures and $f$ is a function from a relational signature $\R$ to a semiring $\SR$, we denote by $\res{f}{\KK}$ the restriction of $f$ to $\KK$, that is, a function such that: the domain of $\res{f}{\KK}$ is $\{ \enc(\A) \mid \A \in \str[\R] \cap \KK\}$, and 
$\res{f}{\KK}(\enc(\A)) = f(\enc(\A))$ for every $\A \in \str[\R] \cap \KK$. 

