
Assume that $\alpha$ is a sentence in $\qso$. Then slightly abusing notation, we also use $\alpha$ to denote a function from $\R$ to $\bbN$ such that for every $\A \in \str[\R] $:
\begin{eqnarray*}
\alpha(\enc(\A)) & = & \sem{\alpha}(\A).
\end{eqnarray*}
\begin{definition}
Let $\FF$ a fragment of $\qso$, $\CC$ a function complexity class over $\bbN$ and $\KK$ a class of finite structures. Then $\FF$ {\em captures} $\CC$ over $\KK$ if:
\begin{itemize}
\item for every $\alpha$ in $\FF$, there exists $f \in \CC$ such that $\res{\alpha}{\KK} = \res{f}{\KK}$; and

\item for every $f \in \CC$, there exists $\alpha$ in $\FF$ such that $\res{f}{\KK} = \res{\alpha}{\KK}$.
\end{itemize}
\end{definition}
In the previous definition, if $\KK$ is the class of all finite structures, then we say that $\FF$ captures $\CC$ over the class of all structures, and if $\KK$is the class of all ordered finite structures, then we say that $\FF$ captures $\CC$ over the class of ordered structures. 

\cristian{Aca hay que definir esto sobre estructuras finitas directamente. No veo el sentido de tener estructuras infinitas.}

\begin{theorem}
$\eqso(\fo)$ captures $\shp$ over the class of ordered structures.
\end{theorem}

\begin{proposition}
$\eqso(\fo)$ does not capture $\shp$ over the class of all structures.
\end{proposition}

%\begin{theorem}
%$\eqso(\integ,\fo)$ captures $\gp$ over the class of ordered structures.
%\end{theorem}


\begin{theorem}
$\eqso(\eso)$ captures $\spp$ over the class of ordered structures.
\end{theorem}
