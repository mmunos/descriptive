
Assume that $\alpha$ is a sentence in $\qso$. Then slightly abusing notation, we also use $\alpha$ to denote a function from $\R$ to $\bbN$ such that for every $\A \in \str[\R] $:
\begin{eqnarray*}
\alpha(\enc(\A)) & = & \sem{\alpha}(\A).
\end{eqnarray*}
\begin{definition}
Let $\FF$ a fragment of $\qso$, $\CC$ a function complexity class over $\bbN$ and $\KK$ a class of finite structures. Then $\FF$ {\em captures} $\CC$ over $\KK$ if:
\begin{itemize}
\item for every $\alpha$ in $\FF$, there exists $f \in \CC$ such that $\res{\alpha}{\KK} = \res{f}{\KK}$; and

\item for every $f \in \CC$, there exists $\alpha$ in $\FF$ such that $\res{f}{\KK} = \res{\alpha}{\KK}$.
\end{itemize}
\end{definition}
In the previous definition, if $\KK$ is the class of all finite structures, then we say that $\FF$ captures $\CC$ over the class of all structures, and if $\KK$is the class of all ordered finite structures, then we say that $\FF$ captures $\CC$ over the class of ordered structures. 

\cristian{Aca hay que definir esto sobre estructuras finitas directamente. No veo el sentido de tener estructuras infinitas.}

\begin{theorem} \label{captfp}
	$\qfo(\lfp)$ captures $\fp$ over the class of ordered structures.
\end{theorem}
\begin{proof}
	For the first condition, let $\alpha\in\qfo(\lfp)$, $\A\in\ostr$, $v$ and $V$ a first and second order assignment for $\A$, respectively. To evaluate $\sem{\alpha}(\A,v,V)$, we replace each first order sum and first order product by their corresponding expansion. This is, $\Sigma x \beta(x)$ is replaced by $(\beta(a_1)+\cdots+\beta(a_n))$, where $A = \{a_1,\ldots,a_n\}$, and $\Pi x \beta(x)$ is replaced by $(\beta(a_1)\cdot\,\cdots\,\cdot\beta(a_n))$. Then we replace each logic sub-formula $\varphi$ in $\alpha$ by their evaluated value, 0 or 1. Since each of this formulas is in $\lfp$, this can be done in polynomial time. The resulting formula is an arithmetic expression, which can be evaluated recursively in polynomial time.
	
	For the second condition, let $f\in \fp$. Let $k\in\nat$ be such that $\vert f(\A) \vert = \vert A \vert^k$ for each $\A\in\ostr[\R]$. Let $\Phi(x_0,\ldots,x_{k-1})$ be a $\lfp$ formula such that $(\A,a_0,\ldots,a_{k-1})\models\Phi$ if and only if in the string $y = f(\A)$ there is 1 in the $(\vert A \vert^{k-1}m_{k-1} + \cdots + \vert A \vert^2 m_2 + \vert A \vert m_1 + m_0)$-th position from right to left, and where $a_i$ is the (0-indexed) $m_i$-th element of $A$. We use
	$$
	\alpha = \Sigma x_0 \cdots \Sigma x_{k-1} \Phi(x_0,\ldots,x_{k-1})\cdot\varphi_{k-1}(x_{k-1})\cdot\,\cdots\,\cdot\varphi_0(x_0).
	$$
	where $\varphi_i(x) = \Pi y[(y < x)\mapsto\Pi z_1\cdots\Pi z_i\,2]$ for $i > 0$ and $\varphi_0(x) = \Pi y[(y < x)\mapsto 2]$. Note that if $a$ is the  $m$-th element in $A$, then $\sem{\varphi_i(x)}(\A,a) = 2^{\vert A \vert^i m}$. Therefore, for each $(a_0,\ldots,a_{k-1})\in A^k$, we have that $\sem{\Phi(x_0,\ldots,x_{k-1})\cdot\varphi_{k-1}(x_{k-1})\,\cdots\,\varphi_0(x_0)}(\A,a_0,\ldots,a_{k-1}) = 2^{(\vert A \vert^{k-1}m_{k-1} + \cdots + \vert A \vert^2 m_2 + \vert A \vert m_1 + m_0)}$ if $(\A,a_0,\ldots,a_{k-1})\models\Phi$ and 0 otherwise, and adding these values gives $f(\A)$. 	
\end{proof}

\begin{theorem}
	$\qfo(\pfp)$ captures $\nfpspace$ over the class of ordered structures.
\end{theorem}
\begin{proof}
	For the first condition, we use the exact same procedure as in Theorem \ref{captfp}, noting that evaluating each of the $\pfp$ formulas can be done in polynomial space, and that the result is also polynomial on the size of the input.
	
	For the second condition, note that for each function $f\in \nfpspace$, $\vert f(\A) \vert$ is polynomial on $\vert A \vert$. Thus, there is a $\pfp$ formula $\Phi$ that models the string $f(\A)$ by the bit. The rest of the proof is analogous to Theorem \ref{captfp}.
\end{proof}

\begin{theorem}
	$\qfo(\tc)$ captures $\shl$ over the class of ordered structures.
\end{theorem}
\begin{proof}
	For the first condition, let $\alpha\in\qfo(\tc)$, let $\A\in\ostr$ $v$ and $V$ first and second order assignments of $\A$ respectively. We evaluate $\sem{\alpha}(\A,v,V)$ recursively. A logic formula $\varphi$ is evaluated...?
	
	For the second condition, note that for each function $f\in \shl$, $\vert f(\A) \vert$ is polynomial on $\vert A \vert$. Thus, there is a $\tc$ formula $\Phi$ that models the string $f(\A)$ by the bit. The rest of the proof is analogous to Theorem \ref{captfp}.
\end{proof}

\begin{theorem}
	$\rqfo(\fo)$ captures $\fp$ over the class of ordered structures.
\end{theorem}

\begin{theorem}
	$\tqfo(\fo)$ captures $\shl$ over the class of ordered structures.
\end{theorem}
\begin{proof}
	For the second condition, let $f \in \shl$. Let $M$ be a non-deterministic logspace machine such that $f(w) = \acc_M(w)$ Let ${\cal Q} = \{q_1,\ldots,q_{\ell}\}$ be the set of states of $M$, where $q_1$ is the initial state, and let $q_{\ell}$ be the only final state of $M$. Let $\Sigma = \{0,1\}$ wlog. and let $w\in\Sigma^n$. Let $w = w_1\cdots w_n$, with $w_i \in \{0,1\}$ and let $\A$ be the ordered structure that represents this word with $A = \{1,\ldots,n\}$ and an unary predicate $P(\cdot)$ where $i\in P$ if and only if $w_i = 1$. We notate $M(w)$ as the graph of configurations of $M$ running on input $w$. Suppose that $M$ has a reading tape and a working tape, and the latter uses space $c\cdot\log(n)$.
	
	We represent configurations with a tuple of fixed size $k$. The formula $\varphi(\bar{x},\bar{y})$ describes consecutive configurations. The formula $\varphi_I(\bar{x})$ models that $\bar{x}$ is the initial configuration of $M(w)$. The formula $\varphi_F(\bar{x})$ models that $\bar{x}$ is a accepting (final) configuration of $M(w)$. The formula we construct is:
	$$
	\alpha = \sa{\bar{x}}\sa{\bar{y}}([\pth \varphi(\bar{u},\bar{v})](\bar{x},\bar{y})\cdot \varphi_I(\bar{x})\cdot\varphi_F(\bar{y})).
	$$
	Note that there is a total of $m = (n+2)\cdot (c\log(n)+2)\cdot 3^{c\log n} \cdot \ell$ distinct configurations in $M(w)$, so we will represent them in the following way. The first two elements of the tuple represent the position of the head in the reading tape. The next $e$ elements represent the position of the head in the working tape. We choose an appropriate $e$ such that $n^e \geq c\log(n)$ for each $n$. The next $2c$ elements represent the string in the working tape. The last $\ell$ elements represent the state in which the machine is in. To specify, let $\bar{a} = (a_1,\ldots,a_m)$ be a tuple that represents a configuration in $M(w)$.
	\begin{itemize}
	\item $(a_1,a_2)$ represent the position of the head in the reading tape. If $a_1 = 1$, then the head is in position $a_2$ (the $a_2$-th character in the input). If $a_1 = 2$, then $a_2 = 1$ represents that the head is at the beginning of the tape (position 0), and if $a_2 = 2$, then the head is at position $n+1$.
	\item $(a_3,\ldots,a_{e+2})$ represent the position of the head in the working tape. Specifically, let $h$ be the value of $a_{e+2}a_{e+1}\cdots a_4 a_3$ in base $n$. Then the head is at position $h$ of the working tape.
	\item $(a_{e+3},\ldots,a_{2+e+2c})$ represents the string that is in the working tape. Specifically, let $h$ be the value of $a_{2+e+2c}a_{1+e+2c}\cdots a_{e+4}a_{e+3}$ in base $n$. Then the string in the working tape is $h$ in base 3, where 2 is interpreted as $\texttt{B}$.
	\item $(a_{3+e+2c},\ldots,a_{2+e+2c+\ell})$ represents the current state $q_i$. Every value in the tuple is $1$ except the element $a_{3+e+2c+i}$ which is 2.
	\end{itemize}
	Now we describe the formulas.
	\begin{align*}
	\varphi_I(\bar{x}) &= \varphi_v(\bar{x}) \wedge \exists u(\varphi_1(u) \wedge \varphi_{succ}(u,x_{3+e+2c})) \\
	\varphi_F(\bar{x}) &= \varphi_v(\bar{x}) \wedge \exists u(\varphi_1(u) \wedge \varphi_{succ}(u,x_{2+e+2c+\ell})),
	\end{align*}
	where $\varphi_v(\bar{x})$ describes that $\bar{x}$ is a valid configuration of $M(w)$ and $\varphi_1(x)$ describes that $x$ is the first element of $A$.
\end{proof}

\begin{theorem}
	$\tqso(\fo)$ captures $\shp$ over the class of ordered structures.
\end{theorem}

\begin{theorem}
	($\qfo(\lfp)$ or $\qfo(\fo)$) with algebraic least fixed point operator captures $\totp$ over the class of ordered structures.
\end{theorem}
	

\begin{theorem}
$\eqso(\fo)$ captures $\shp$ over the class of ordered structures.
\end{theorem}

\begin{proposition}
$\eqso(\fo)$ does not capture $\shp$ over the class of all structures.
\end{proposition}

%\begin{theorem}
%$\eqso(\integ,\fo)$ captures $\gp$ over the class of ordered structures.
%\end{theorem}


\begin{theorem}
$\eqso(\eso)$ captures $\spp$ over the class of ordered structures.
\end{theorem}
